\documentclass[12pt,UTF-8,openany]{ctexbook}
\usepackage{ctex}
\usepackage{titlesec}
\usepackage{xeCJK}
\usepackage{verse}
\usepackage{fontspec,xunicode,xltxtra}
\usepackage{xpinyin}
\usepackage{geometry}
\usepackage{indentfirst}
\usepackage{pifont}
\usepackage{enumitem}
\usepackage[perpage,symbol*]{footmisc}
\usepackage[table,dvipsnames]{xcolor}
% \usepackage{dblfnote}
% \DFNalwaysdouble % for this example

\geometry{a5paper,left=1.2cm,right=1.2cm,top=2.4cm,bottom=2.4cm}
\renewcommand{\footnotesize}{\fontsize{8.5pt}{10.5pt}\selectfont}
\setmainfont{Arial}
\setCJKmainfont[BoldFont=STZhongsong]{汉字之美仿宋GBK 免费}
\xeCJKDeclareCharClass{CJK}{`0 -> `9}

\xeCJKsetup{AllowBreakBetweenPuncts=true}
\DefineFNsymbols{circled}{{\ding{192}}{\ding{193}}{\ding{194}}{\ding{195}}{\ding{196}}{\ding{197}}{\ding{198}}{\ding{199}}{\ding{200}}{\ding{201}}}
\setfnsymbol{circled}

\xpinyinsetup{ratio=0.44,hsep={.6em plus .6em},vsep={1em}}
\titleformat{\chapter}{\zihao{-1}\bfseries}{ }{16pt}{}
\titleformat{\section}{\zihao{-2}\bfseries}{ }{0pt}{}
\title{\zihao{0} \bfseries 自制语文课文集萃}
\setlength{\lineskip}{24pt}
\setlength{\parskip}{6pt}
\author{}
\date{}
\begin{document}
% \maketitle
% \tableofcontents
% \newpage

\begin{center}
    \begin{Huge}
        \textbf{天文地理}
    \end{Huge}
    
    \vspace{8pt}
    《幼学琼林》\footnote{节选自《幼学琼林》“天文”“地舆”两章,有增补改动。《幼学琼林》是明朝末年程登吉编著的儿童启蒙读物。}
    \vspace{8pt}

\end{center}

\begin{normalsize}
    
    混沌初开,乾坤\footnote{〔乾坤〕天和地。乾表示天,坤表示地。}始奠。

    气之清轻者上浮为天,气之浊重者下凝为地。
    
    日月五星,谓之七曜\footnote{〔七曜〕日月和水星、金星、火星、木星、土星合称“七曜”。《春秋谷梁传注疏》:“七曜者,日月五星皆照天下,故谓之七曜。”};天地与人,谓之三才。
    
    二十八宿,是日月之旅舍\footnote{《论衡·谈天》:“二十八宿为日月舍。”}。黄道白道,载日月之行迹\footnote{黄道是从地球上来看太阳一年“走”过的路线,是由于地球绕太阳公转而产生的。白道是月球绕地球运行的轨道在天球上的投影。}。
    
    北斗为帝车之象\footnote{〔帝车之象〕《史记·天官书》:“斗为帝车,运于中央,临制四乡。”。},北辰\footnote{〔北辰〕北极星。《尔雅·释天》:“北极谓之北辰。” }是太一\footnote{〔太一〕天帝的名称。《易纬·乾凿度》所说:“太一者,北辰之神名也,居其所曰太帝。” }所居。
    
    叁星为白虎之体\footnote{叁星即西方白虎七宿之叁宿里一字并联的三颗星,西方星座中的“猎户腰带”。《史记·天官书》:“叁为白虎。”《晋书·天文志》:“叁,白兽之体。”},大火是苍龙之心\footnote{大火指心宿二,也叫商星。心宿是东方苍龙七宿之一,位于中心。}。
    
    叁商二星,其出没不相见\footnote{天球上的叁星和商星互相在正对面,其中之一升上天空时,另一个必然隐入地平线下。}。牛女两宿,惟七夕一相逢\footnote{指牛郎织女的故事。}。
    
    后羿射日,女娲补天\footnote{本为古人修正伏羲历法里一年360天与实际365.25天的偏差的做法,后讹传为神话故事。}。羲和置闰,授民以时\footnote{《尚书·尧典》:“乃命羲和,钦若昊天,历象日月星辰,敬授人时。……帝曰:‘咨!汝羲暨和。期三百有六旬有六日,以闰月定四时成岁。’”}。
    
    黄帝画野\footnote{《汉书·地理志》:“昔在黄帝,作舟车以济不通,旁行天下,方制万里,画野分州,得百里之国万区。”},始分都邑。夏禹治水,初奠山川\footnote{《尚书·吕刑》:“禹平水土,主名山川。”}。
    
    宇宙之江山不改,古今之称谓各殊。
    
    北京原属幽燕,金台是其异号;南京原为建业,金陵为其别名。
    
    浙江是武林之区\footnote{杭州古称武林,附近有武林山。浙江是钱塘江古称。隋朝设杭州,唐时改为余杭郡,北宋时为两浙路,南宋在杭州设临安府为首都,明清合为浙江省。},原为越国;江西是豫章之地,昔从九江\footnote{江西古称豫章,秦灭楚后设九江郡,包括江西和安徽、江苏北部。汉设豫章郡,隋唐设洪州,明清时改为南昌府。}。
    
    福建省属闽中\footnote{秦代设置闽中郡,包括福建,浙江宁海县及灵江、瓯江、飞云江流域。后来福建也称闽中。},湖广地名三楚\footnote{元代设置湖广等处行中书省,包括湖南湖北,明代湖广行省按春秋时楚国名号称为“楚”。秦汉时有西楚、东楚、南楚的分野,总称三楚。}。
    
    东鲁西鲁,即山东山西之分;东粤西粤,乃广东广西之域。
    
    河南在华夏之中,故曰中州;陕西即长安之地,原为秦境。
    
    四川为西蜀\footnote{四川上古有蜀国,被秦国灭亡。汉末刘备立蜀汉,一度扩张到汉中平原。明代设置四川省。},云南为古滇\footnote{云南上古有滇国,被汉朝征服。晋隋设宁州,唐时称南诏,宋时称大理。元代起设云南省。}。
    
    贵州省近蛮方\footnote{〔蛮方〕蛮:指南方未开化的部落。方:区域,部落邦国。},自古名为黔地。
    
    东岳泰山,西岳华山,南岳衡山,北岳恒山,中岳嵩山,此为天下之五岳。
    
    饶州之鄱阳\footnote{隋朝设饶州,后为鄱阳郡,今有江西上饶市。鄱阳湖古称彭蠡泽、彭泽,是江西北部的大湖。},岳州之青草\footnote{隋朝设岳州,历史上也叫巴陵、巴州,今有湖南岳阳市。青草湖是湖南古代的大湖,位于岳阳市西南,洞庭湖的南部,并与之相连。南北朝时期已连为一体。清末之后逐渐淤积枯萎,现已不存。},润州之丹阳\footnote{隋唐设润州,在今江苏西南部。丹阳湖是江南古代的大泽,因秦置丹阳县得名,也叫丹湖、南湖,位于润州西南,今南京南面,高淳、溧水、当涂一带。因泥沙淤积和围垦,逐渐消亡。},鄂州之洞庭\footnote{隋朝设鄂州,包括今湖北地区,历史上也叫江夏、武昌,治所在今湖北武汉市。这里的洞庭指唐宋时湖南北部、长江以南的大湖,唐宋时流域包括湖北一部分。},苏州之太湖\footnote{隋朝设苏州,包括今江苏、浙江地区,历史上也叫吴州,治所在今江苏苏州市。太湖是江苏南部大湖,古称震泽。},此为天下之五湖。
    
    沧海桑田,谓世事之多变。河清海晏,兆天下之升平。
    
    道不拾遗,由在上\footnote{〔在上〕指君主、掌权者、当局。}有善政。途通天堑,知中国有圣人。
    
\end{normalsize}

\clearpage

\textbf{译文}:

最初混沌开辟成宇宙,然后天和地就确定了下来。

清澈轻盈的气往上漂浮,形成了天;浑浊沉重的气往下凝聚,形成了地。

日月和五大行星合称“七曜”。天、地、人并称“三才”。

二十八星宿,是日月周游暂住的处所。黄道和白道,记录了日月运行的轨迹。

北斗七星是天帝驾的车子,北辰是天帝太一住的地方。

叁宿三星是白虎的躯体,大火星是苍龙的心脏。

叁星与商星此出彼没,永远没有机会相见。牛郎和织女隔着银河相望,每年七月初七才能相会。

后羿射日,女娲补天,(都是古代调整历法的传说)。(帝尧任命羲、和掌管天象观测,制定历法);羲、和设置闰月,让民众知道(务农的)时机。

黄帝划分大地,从此有国都和封邑的界限。夏禹治理洪水,首次确定了山川的名字。

天地间的山岭河流(虽然)不曾更改,过去和今天的称呼(却)各有不同。

北京以前属于幽州和燕国,“金台”是它的别号;南京原先叫建业,金陵是它的别名。

浙江从前称为武林地区,原来是越国的土地;江西是以前的豫章郡,更早的时候是九江郡的一部分。

福建省古时是闽中郡,湖南湖北旧名叫做三楚。

东鲁、西鲁就是山东、山西,东粤、西粤即为广东、广西。

河南位于华夏的中心位置,所以又称为中州;长安为陕西的首府,古代是秦国的国土。

四川就是西蜀,云南古时叫滇国。

贵州靠近南蛮之地,自古以来称为黔。

东岳泰山,西岳华山,南岳衡山,北岳恒山,中岳嵩山,这是天下的五座雄伟的高山。

饶州的鄱阳湖、岳州的青草湖、润州的丹阳湖、鄂州的洞庭湖、苏州的太湖,这是天下的五个大湖。

沧海桑田,用来形容世事多变。海晏河清,预示着天下太平。

路不拾遗,是因为当局实行了好的政策。能把天堑变成通途,就知道中国有圣人。

\begin{normalsize}
    
\end{normalsize}

\newpage

\textbf{注释}:

\vspace{-1em}

\begin{itemize}
    \setlength\itemsep{-0.2em}
    \item 〔乾坤初\smash{\underline{奠}}〕定,确定,规定;安定地放置。
    \item 〔\smash{\underline{始}}分都邑〕从此才,然后才,于是才。
    \item 〔古今之称谓各\underline{殊}〕不一样。
    \item 〔北京原\smash{\underline{属}}幽燕〕属于,从属。
    \item 〔\smash{\underline{昔}}从九江〕以前,从前。
    \item 〔此为天下之五\underline{岳}〕高大的山。
    \item 〔河清海\underline{晏}〕平静,平和。
    \item 〔\smash{\underline{兆}}天下之升平〕预示。
    \item 〔兆天下之\underline{升平}〕太平。
    \item 〔\smash{\underline{由}}在上之善政〕来自;因为。
    \item 〔由在上之\smash{\underline{善}}政〕好。
    \item 〔由在上之善\underline{政}〕以上正下,通过戒令管理国民的活动,政策。
    \item 〔途通天\underline{堑}〕坑壕,护城河。天堑:天生的护城河,比喻地势险峻隔断交通的大河。如:长江天堑。
\end{itemize}

\end{document}