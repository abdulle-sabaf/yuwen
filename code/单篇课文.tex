\documentclass[12pt,UTF-8,openany]{ctexbook}
\usepackage{ctex}
\usepackage{titlesec}
\usepackage{xeCJK}
\usepackage{verse}
\usepackage{fontspec,xunicode,xltxtra}
\usepackage{xpinyin}
\usepackage{geometry}
\usepackage{indentfirst}
\usepackage{pifont}
\usepackage{enumitem}
\usepackage[perpage,symbol*]{footmisc}
\usepackage[table,dvipsnames]{xcolor}
% \usepackage{dblfnote}
% \DFNalwaysdouble % for this example

\geometry{a5paper,left=1.4cm,right=1.4cm,top=2.4cm,bottom=2.4cm}
\renewcommand{\footnotesize}{\fontsize{8.5pt}{10.5pt}\selectfont}
\setmainfont{Arial}
\setCJKmainfont[BoldFont=STZhongsong]{汉字之美仿宋GBK 免费}
\xeCJKDeclareCharClass{CJK}{`0 -> `9}

\xeCJKsetup{AllowBreakBetweenPuncts=true}
\DefineFNsymbols{circled}{{\ding{192}}{\ding{193}}{\ding{194}}{\ding{195}}{\ding{196}}{\ding{197}}{\ding{198}}{\ding{199}}{\ding{200}}{\ding{201}}}
\setfnsymbol{circled}
\xpinyinsetup{ratio=0.44,hsep={.6em plus .6em},vsep={1em}}
\titleformat{\chapter}{\zihao{-1}\bfseries}{ }{16pt}{}
\titleformat{\section}{\zihao{-2}\bfseries}{ }{0pt}{}
\title{\zihao{0} \bfseries 自制语文课文集萃}
\setlength{\lineskip}{24pt}
\setlength{\parskip}{6pt}
\author{}
\date{}
\begin{document}
% \maketitle
% \tableofcontents
% \newpage

\begin{center}
    \begin{Huge}
        \textbf{洱海一枝春}
    \end{Huge}
    
    \vspace{8pt}
    曹靖华
    \vspace{8pt}

\end{center}

\begin{normalsize}
    
    大理好\footnote{〔大理〕云南地名,古代有南诏国和大理国,以大理为都城;现为大理白族自治州大理市。}。
    
    洱海,这面光洁的梳妆镜,南北长百里,东西宽十余里,就放在它前面。苍山,这扇锦屏,高达八里,宽百余里,就竖在它背后。
    
    苍山十九峰,自北而南,宛如十九位仙女,比肩并坐,相偎相依,好像在对镜理妆,凝视洱海;又好像在顾盼着苍山下、洱海边的终年盛开的繁花,默默欣赏。
    
    山巅白雪皑皑,好似一条又细又白的纱巾,披在头顶,显得分外洒脱。
    
    大理,好一幅风景画。大理,好一首抒情诗。大理,这神话之乡,处处皆神话。任你走到哪儿,谁都会津津有味地指点着告诉你:这是蛇骨塔。据说从前洱海出现了一条怪蟒,兴风作浪吞食人畜,常用尾巴堵住洱海出口,海水泛滥,淹没田舍。这一带人民可遭殃了。大理石匠段赤诚,决心为民除害。他手执宝剑,身捆钢刀,纵身入海,与蟒搏斗,被蟒吞入腹中。他在蟒肚里滚来滚去,用钢刀将蟒刺死,自己也身葬蟒腹。人民将蟒捞出,破开肚子,把段赤诚的遗体起出来,把蟒烧成灰,拌入泥中,在苍山马耳峰下,修了这座塔来纪念他。
    
    “这就是蝴蝶泉呀!据说当年泉边住着一户人家,有一个姑娘,长得可美呢!姑娘有副好心肠,她爱上了一个年轻樵夫。国王听说姑娘长得好,就把她抢到宫里。樵夫深夜把她救出来,国王派人追到泉边,二人知无可逃,就投泉而死,化为蝴蝶,双双飞去……”
    
    “瞧。那就是苍山玉局峰!‘望夫云’就出现在玉局峰上的天空呢!据说从前大理南诏王有位公主,心肠好,长相美。她爱上一个年轻的穷猎人,一块逃到玉局峰上的岩洞里。国王大怒,就请法师把猎人打死在洱海里,变为石骡。公主日夜想念,不久也就死去,化为一朵白云,出现在玉局峰上,像在探望似的。这朵云一出现,洱海上就狂风大作,白浪掀天,直到吹开海水,露出石骡,才风息云散呢。”
    
    “这是……”
    
    “瞧,那就是……”
    
    啊,只要你有情致听,这儿的故事真比《天方夜谭》\footnote{〔《天方夜谭》〕又称《一千零一夜》,是公元9世纪形成的一部著名的阿拉伯民间故事集,汇集了数百年间流传于西亚、北非多国的民间故事,以“每夜一个故事”的形式讲述。}还多呢!纵让说上一千零一夜,也未见得能说完!
    
    不论谁到这儿,都会恍如置身神话境界,禁不住从心坎里发出赞叹:大理好。
    
    可是,更好的是大理人。他们正依着党的蓝图,创造着比神话还好的现实呢!
    
    一个临别晚会上,大理白族\footnote{〔白族〕中国少数民族,主要分布在云南、贵州、湖南。}自治州文艺单位的几位同志和我们聚在一起,畅聊起来。她们都像暴雨之后,洱海猛涨,海水来不及倾泻一样,那汹涌激荡、倾吐不及的千言万语,一齐涌上心头,都争着说:
    
    “我们大理可好着呢!你喜欢我们的‘风花雪月’吗?我们这儿流行着这样的……啊!怎么说呢,也可算是咏大理的四景吧:
    
    \begin{verse}[0.5\linewidth]
    
    下关风,上关花,\\ 下关风吹上关花。\\苍山雪,洱海月,\\洱海月照苍山雪。
    
    \end{verse}
    
    来去匆匆,怎能品出‘风花雪月’的味儿呢?”
    
    于是有的同志说大理含蓄得很,不让客人到此一览无余。有的说大理是一篇好文章,越读越耐人寻味。有的说大理是杯醇酒,芬芳馥郁的味儿,一口尝不出来。最后都说:
    
    “再来吧,下次再来。可别忘记我们的风花雪月呀!下次再来,我们准备在‘花前月下’,陪你乘下关风,破洱海浪,游我们洱海的水晶宫\footnote{〔水晶宫〕中国民间传说里东海龙王的宫殿,用水晶做成。这里指洱海边古生村的一座明代建成的寺庙,供奉托塔李天王。},在我们自治州的水晶宫里表演……”
    
    我连忙笑着插道:
    
    “那时你们可再不能像刚才那样,表演《新媳妇走娘家》\footnote{〔《新媳妇走娘家》〕1958年创作的歌曲,韩岗长词、王石路曲。}了,那是水晶宫呀。”
    
    一阵笑声过后,就都又抢着说:
    
    “怎么呢,那时我们就表演《仙女遨游水晶宫》吧。不过,伴奏的将不是我们的乐队,而是下关风。风声所至,万籁齐鸣。这是我们水晶宫里的交响乐。我们将在洱海月、苍山雪的清辉交映中,翩翩起舞。咱们将在水晶宫里共餐洱海月映照的苍山雪吧,那才别有情趣呢!……”
    
    风花雪月……
    
    大理,它美,美得别致、有情趣。
    
    大理风,不,应该说是下关风吧,确实了不起。我一到这儿就领略到了。我住在楼上,窗子又正对风口,每夜狂风大作,龙吟虎啸,屋瓦齐鸣。真比黄山\footnote{〔黄山〕著名风景名胜,位于安徽省南部,被誉为“天下第一奇山”。玉屏楼、狮子林都是黄山著名景点。}玉屏楼、狮子林的风还要猛烈。大风一起,我就想这大概又是望夫云想看石骡子。风声之大,耸人听闻。原来下关地处洱海至漾濞江\footnote{〔漾濞江〕澜沧江第二大的支流。自云南西北部流向南部,汇入澜沧江。北段称为黑惠江,流经大理时称为漾濞江。}的出口,由苍山十九峰最南的斜阳峰下的山谷中,夺峡而出,直奔下游的澜沧江去。这又深又窄的峡谷,是个大风口,正对着一马平川的下关。所以,劲风阵阵,呼啸而来。其来也,如钱塘怒潮,万马奔腾,地动天摇。多少弱不禁风的人,都会闻风色变呢!
    
    大理的花,也别有风骨。狂风中,尽管它的枝叶身干,随风俯仰,而花苞却脾院一切,迎风怒放。
    
    大理繁花似锦,真是“花花世界”。大理人也别有情趣,有福气。他们爱花、养花,几乎成了风习。尤其白族同胞,几乎家家院内是繁花,户户门外有清流。他们把花当作美好生活的一部分。花好还须绿叶衬,这翡翠般的绿叶,正衬托着他们那幸福的“生活之花”呢。
    
    大理的花又多又好。尤其是茶花,如果说云南茶花甲天下,那么,大理茶花就该是盖云南了。
    
    茶花品种之多,达七十余种。什么蝶翅、大紫袍、雪狮子、大玛瑙、童子面、恨天高……啊,不是记忆力好的园艺家,谁能记得清它们的称号,辨得出它们的容颜呢!花朵之硕大,有直径达七寸者。颜色有紫、有白、有粉、有红……接枝后,一株树上可开出数千朵五彩缤纷的花朵来。漓江\footnote{〔漓江〕位于广西东北部、桂林市境内的河流。}还有上千年的万朵茶花呢。尤其是大红的茶花,老远就把人的眼睛、人的心魂都吸住了。盛开时,火焰般的满树通红。假使孙悟空到此,定以为这儿是火焰山了。至于下关风啊,也会认为这是铁扇公主\footnote{〔铁扇公主〕中国民间传说人物,见《西游记》第59至61回。}扇起来的呢!而“恨天高”这称号啊,是否也因为它想同齐天大圣比高低,待长得不能高与天齐时,才恨天高呢!总之,可不要死板地用现实主义的眼睛,来对待这神话之乡的一切。而今,党给这儿的人插上了翅膀,他们都在共产主义的伟大理想中翱翔呢!
    
    岂但茶花而已,杜鹃也出色,品种多、花朵大、颜色鲜。有红、白、粉、黄……尤其是黄杜鹃,倍加别致。
    
    岂但杜鹃而己,梅花也与众不同,有红梅、白梅、朱砂梅……更别致的怕是绿梅了。满树绿苞,比翡翠嵌镶的还逗人呢!
    
    岂但梅花而已……
    
    大理花好,不过更好的却是洱海的一枝春。它比茶花艳,比杜鹃娇,比绿梅清秀,比任何花都出类拔萃。这是近年来党的甘霖滋养的、大理人民喜爱的一枝花。它已经显露出大理人民的艺术的春天。它,大理白族自治州歌舞团,洋溢着青春的活力。据我们看过的一些演出,都别具地方色彩,民族风趣。鲁迅先生说:“有地方色彩的,倒容易成为世界的,即为别国所注意。打出世界去,即于中国之活动有利。”洱海一枝春啊,望你在人民的土壤中,在党的春风化雨中,不断成长、壮大;在自治州、全省、全国艺苑中,开出挺拔绚烂的花朵,最后,如鲁迅先生所说:“打出世界去”,替祖国艺苑更添几许春色吧!
    
    \hfill 一九六二年一月
    
\end{normalsize}


\newpage

\textbf{注释}:

\vspace{-1em}

\begin{itemize}
    \setlength\itemsep{-0.2em}
    \item 〔相偎相依〕相互紧密依靠、依赖。偎:亲密地紧靠着。
    \item 〔馥郁〕(形容香气)很浓。
    \item 〔一马平川〕能够纵马疾驰的平地。泛指宽阔而平坦的土地。
    \item 〔风习〕风俗习惯。
    \item 〔心坎〕内心深处。
    \item 〔理妆〕整理妆容、梳妆。
    \item 〔艺苑〕比喻文艺界、各种文艺作品。苑:花园。
\end{itemize}

\end{document}