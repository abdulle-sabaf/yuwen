\documentclass[12pt,UTF-8,openany]{ctexbook}
\usepackage{ctex}
\usepackage{titlesec}
\usepackage{xeCJK}
\usepackage{verse}
\usepackage{fontspec,xunicode,xltxtra}
\usepackage{xpinyin}
\usepackage{geometry}
\usepackage{indentfirst}
\usepackage{pifont}
\usepackage{enumitem}
\usepackage[perpage,symbol*]{footmisc}
\usepackage[table,dvipsnames]{xcolor}
% \usepackage{dblfnote}
% \DFNalwaysdouble % for this example

% \ctexset{
%     chapter = {
%         name = {第, 课},
%         number = \chinese{chapter}
%     }
% }

\geometry{a5paper,left=1.2cm,right=1.2cm,top=2.4cm,bottom=2.4cm}
\renewcommand{\footnotesize}{\fontsize{8.5pt}{10.5pt}\selectfont}
\setmainfont{Arial}
\setCJKmainfont[BoldFont=STZhongsong]{汉字之美仿宋GBK 免费}
\xeCJKDeclareCharClass{CJK}{`0 -> `9}

\xeCJKsetup{AllowBreakBetweenPuncts=true}
\DefineFNsymbols{circled}{{\ding{192}}{\ding{193}}{\ding{194}}{\ding{195}}{\ding{196}}{\ding{197}}{\ding{198}}{\ding{199}}{\ding{200}}{\ding{201}}}
\setfnsymbol{circled}

\xpinyinsetup{ratio=0.44,hsep={.6em plus .6em},vsep={1em}}
\renewcommand{\thechapter}{第\chinese{chapter}课}
\titleformat{\chapter}{\zihao{-1}\bfseries}{\thechapter}{16pt}{}
\titleformat{\section}{\zihao{-2}\bfseries}{ }{0pt}{}
\title{\zihao{0} \bfseries 自制语文课文集萃}
\setlength{\lineskip}{24pt}
\setlength{\parskip}{6pt}
\author{}
\date{}
\begin{document}
\maketitle
\tableofcontents
\newpage

\chapter{愤怒的葡萄(节选)}

\begin{center}
    % \begin{Huge}
    %     \textbf{愤怒的葡萄(节选)}
    % \end{Huge}
    
    \vspace{8pt}
    约翰·史坦贝克
    \vspace{8pt}

\end{center}

地主们有时也到田上来,更多的时候,来的是地主的代理人。他们坐着门窗紧闭的小汽车来,用手指头摸摸干燥的土地,有时还用钻探机钻进地里去验验土质。那些门窗紧闭的小汽车顺着田野开来的时候,佃户们聚在烈日炙烤下的院子里,不自在地望着他们。最后,地主的人把车子开进院子来,坐在车上,从摇下的车窗里跟人谈话。佃户们在汽车旁边站一会儿,就蹲在地上,找些枝条来在尘土里比画。
    
    女人们站在敞开的门里向外看,背后是孩子——脑袋尖瘦,眼睛睁得大大的,一只光脚叠在另一只光脚上,扭着脚趾。女人和孩子们瞅着家里的男人跟地主的人谈话,默不作声。
    
    地主的人有的很和气,因为他们憎恶自己不得不做的事情;有的很生气,因为他们并不愿意残忍;有的很冷酷,因为他们早就体会到人要是不冷酷,就做不了地主。他们全都被一种更大的东西控制住了。对于那些驱策着他们的算数,他们有人憎恶,有人害怕,也有人崇拜,因为那些算数可以让他们逃避思考,逃避感情。如果土地归什么银行或是什么公司所有,地主的人就说:“银行——或是公司——必须怎样——要想怎样——坚持要怎样——非怎样不可。”仿佛银行和公司是某种具有思想情感的怪物,已经把他们钳制住了似的。这些受钳制的人是不替银行和公司负任何责任的,因为他们是人,是奴隶,而银行既是机器,又是主人。为地主做事的人,给这种冷酷的、强横的主人做了奴隶,还觉得很得意。地主的人坐在汽车里,解释着:“你们知道,这土地不出庄稼。你们在这里苦干多久了,天知道的。”
    
    蹲在地上的佃户们点点头,感到惶惑,在沙尘里比画着。是呀,他们知道,老天也知道。只要不起风沙就好了。只要这沙尘能定在地上,也许就不至于这么糟糕。
    
    地主的人继续往下说,把话头往主题上转:“你们也知道,这地越来越糟了。你们知道棉花毁地,吸干了地里的血。”
    
    蹲着的人点点头——他们知道,老天也知道。要是他们可以轮种\footnote{〔轮种〕在同一块田上按顺序和时间间隔轮流种植不同类型的农作物,也叫“轮作”。轮种可以均衡利用土壤养分,改善土壤结构,减少病虫害,提高农作物产量和质量。},就能给土地补血。
    
    总之,现在来不及了。地主的人搬出那比自己更强的怪物来,把怪物的想法和做法解释给他们听。人只要能吃饱,能缴足税,就有资格保住土地。而这是办得到的。
    
    是的,这是办得到的——直到某一天收成坏了,他不得不向银行借钱。
    
    但是,你要知道,银行和公司却不能这么来,因为它们不呼吸空气,也不吃烤土豆。它们呼吸的是利润,吃的是孳息。要是得不到,就会死去。正如你呼吸不到空气,吃不到饭就会死去一样。这很可悲,但事实便是如此。
    
    蹲在地上的男人们抬起眼睛来,想理解这个问题。“让我们凑合着对付下去不行吗?明年也许是个丰年。天知道明年棉花的收成会有多么好。况且还有打不完的仗——天知道棉花的市价会涨到多么高。人家不是用棉花做炸药、做军装吗?只要老打仗,棉花的价钱就会涨上天。明年也许会这样吧。”他们以探询的眼色抬头望着。
    
    这一层我们是不能指望的。银行——这怪物必须一直盈利。它不能等。它会死的。税是不能停的。要是这怪物无法增长,它就要死了。原地踏步也不行。
    
    柔软的手指头开始轻敲着车窗的框子,粗硬的指头却紧捏着枝条,不自在地乱画。烈日炙烤下的佃户家,门口的女人叹叹气,把交叠的脚上下调换一下,扭着脚趾。一群狗走近地主的汽车去,嗅一嗅,在四个轮胎上一一撒了尿。鸡在阳光直射的沙尘里躺着,抖一抖身上的羽毛,仿佛要把尘抖起来,在沙里洗个澡。小猪圈里的猪吃着残剩的饲料,戒备地哼叫着。
    
    蹲着的男人们又低下头来。“你们叫我们怎么办呢?收成我们不能再少分了——我们都快要饿死了。孩子们老是吃不饱。我们浑身破破烂烂,穿不上衣服。如果不是左邻右舍都和我们一样,我们都不好意思去做礼拜了\footnote{〔礼拜〕基督教会组织生活的一部分,在星期天举行,因此星期天又称礼拜日。}。”
    
    最后,地主的人终于讲到了主题上。租佃制度再也行不通了。一个人开一台拖拉机能代替十二三户人家。只要付给他一些工资,就可以得到全部收成。“我们只得这么办了。我们并不喜欢这么办。但是那怪物病了。那怪物出了毛病,不这么办就不行。”
    
    “但是你们老种棉花,会把土地毁掉的。”
    
    “我们也知道。我们要趁这地还没有完蛋之前,赶快种出棉花来。然后我们就把地卖掉。东部有好多人家想买些地呢。”
    
    佃户们惊恐地抬头望着。“可是我们怎么办呢?我们靠什么吃饭呢?”
    
    “你们非离开这地方不可。拖拉机要开进这院子里来了。”
    
    蹲着的男人们愤怒地站了起来。“从前,我们的爷爷把印第安人\footnote{〔印第安人〕欧洲殖民者对北美洲原住民的统称。}打死,赶跑印第安人,占了这片土地。我们的爸爸生在这里,清除了野草,消灭了蛇。后来遇到荒年,他只好借钱。接着是我们,我们出生了,就在这土地上,在这扇门里——我们的孩子也是在这儿出生的。然后我们又必须借钱。最后,土地归了银行,可是我们还留在这里,我们种出的东西,我们多少可以分一点儿吧。”
    
    “这一切我们都知道。这并不是我们的问题,这是银行的问题。银行和人不一样。或者也可以说,有五万英亩地\footnote{〔英亩〕英制面积单位。一英亩约等于4047平方米。}的地主,他也跟人不一样了。这就是怪物。”
    
    “话倒是对的,”佃户们大声说,“可这究竟是我们的地呀。地是我们量出来的,也是我们开垦出来的。我们在这地上出生,在这地上卖命,在这地上死去。即使地不济事,究竟还是我们的。在这里生,在这里死,在这里干活——所以这块地应该算是我们的。土地的归属,应该以这些为凭据,而不是凭一张写着数字的文契。”
    
    “对不起。这不怨我们,只怨那怪物。银行跟人是不一样的。”
    
    “对,但是银行究竟也是人开的呀。”
    
    “不,那你就弄错了——大错特错了。银行跟人是完全不同的一种东西。银行所做的事情,往往是银行里的人个个都讨厌的,但银行偏要这么做。银行这种东西是在人之上的,我告诉你吧,它是个怪物。人造出了银行,可人控制不住它。”
    
    佃户们叫喊道:“为了这块地,爷爷消灭了印第安人,爸爸消灭了蛇。我们也许可以消灭银行——银行比印第安人和蛇都更可恶呢。我们为了保全我们的地,也许非起来斗争不可,像爸爸和爷爷那样干。”
    
    于是地主的人动气了。“你们非走不可。”
    
    “但是,这是我们的地呀,”佃户们叫喊道,“我们……”
    
    “不,这地是归银行这怪物管理的。你们非走不可。”
    
    “我们要像爷爷当初对付印第安人那样,拿起枪来。看你们怎么办!”
    
    “哼——首先我们有警察,其次是军队。如果你们赖在这里,你们就是犯了盗窃罪,如果你们杀了人,还赖在这里,你们就成了杀人凶手。那怪物不是人,可是它却能叫人做它要的事。”
    
    “可是如果我们离开这里,我们到什么地方去呢?我们怎么去呢?我们没有钱呀。”
    
    “对不起,”地主的人说道,“这种事,银行和五万英亩地的地主是不负责的。你们种的地并不是你们自己的。你们搬离了这地界,也许可以在秋天摘摘棉花。你们也许可以领些救济金来过活。你们为什么不往西部去,到加利福尼亚去呢?那边有工作,天气也不冷。嗐,你们无论走到什么地方,一伸手就可以摘到橙子。经常有庄稼活给你们做。你们为什么不上那儿去呢?”说完,地主的人就开动汽车,一溜烟跑掉了。
    
    佃户们又蹲在地上,用枝条拨弄着沙尘,想着心事。他们晒黑了的脸是阴沉的,太阳熬炼过的眼睛是发亮的。女人们从门口小心翼翼地移步到自己的男人身边,孩子们跟在女人们后面,小心翼翼地悄悄走着,打算跑开。年纪大些的男孩子蹲在他们的父亲身边,因为这么一来,他们就显得像大人了。过了一会儿,女人们问道:“他要怎么样?”
    
    男人们抬起头来望了一会儿,眼中显出沉痛的神情。“我们要滚蛋了。他们要派拖拉机和管理员来。像工厂一样。”
    
    “我们上哪儿去呢?”女人们问道。
    
    “我们不知道!我们不知道!”
    
    于是女人们一声不响地赶快回到屋里去,还撵着孩子们走在她们前面。她们知道,那么伤心烦恼的男人,就是对自己心爱的人,也是会发脾气的。所以她们便撇下了男人,让他们蹲在沙尘里盘算,想着心事。
    
    佃农们环视着四周——看看十年前装置的那个抽水机,那上面有一个鹅颈形的把手,喷水管的嘴上有一些铁花;看一看那块杀过上千只鸡的砧板、放在棚舍里的手犁,还有棚舍梁上挂着的那个别致的摇篮。
    
    屋子里,孩子们聚集在女人身边。“我们怎么办,妈?我们上哪儿去?”
    
    女人们说:“我们还不知道。出去玩玩吧。可是不要走近爸爸身边。如果你们到他身边去,他也许要打你们。”女人们又继续工作了,可是她们却一直望着蹲在沙尘里想着心事、大伤脑筋的男人们。
    
    几辆拖拉机从大路上开过来,开进了田野,它们是一些像虫子一般爬行的巨物,有惊人的气力。它们在地面上爬行,把履带滚下来,在地面上滚过,又把它卷上去。拖拉机停歇的时候,那上面的柴油机啪嗒啪嗒地响着;一开动,便轰隆轰隆地响,渐渐变成单调的吼声了。这些狮子鼻的怪物扬起沙尘,向沙尘里钻进去。它们一直越过原野,越过篱笆,越过家家户户门前的院子,沿着一条条的直线来回地闯过许多水沟。它们并不是在地面上跑,而是在自己的履带上跑。它们完全不把高坡、低谷、水道、篱笆和房屋放在眼里。
    
    坐在铁驾驶座上的那个人,看去并不像一个人。他戴着手套和护目镜,鼻子和嘴上套着橡皮制的防沙面具,他是那怪物的一部分,是一个坐着的机器人。汽缸的轰鸣声响彻了原野,与空气和大地合为一体,大地和空气都跟着颤动,发出低沉的声响。驾驶员控制不住它——它一直越过原野,划破十多个农庄,又一直来回转。只要拨动一下操纵杆,就可以改变拖拉机的方向,但是驾驶员的双手却不能随意拨动,因为造出拖拉机和派出拖拉机来的那个怪物仿佛控制了那双手,控制了他的脑子和筋肉,给他戴上了眼罩,套上了口罩——蒙住了他的心灵,堵住了他的嘴,掩盖了他的理智,制止了他的抗议。他看不见土地的真面目,嗅不出土地的真气息;他的两脚踏不到泥土,感觉不到大地的温暖和力量。他坐在铁驾驶座上,踏着铁踏板。他对自己扩张来的力量,既不会鼓舞,也不会遏制;既不会诅咒,也不会鼓励。因此他也无法鼓舞、鞭策、诅咒或是激励自己。他对土地既不熟悉,也没有所有权,既不信赖,也无所求。如果撒下的种子没有发芽,那也不相干;如果长出来的幼芽在大旱天枯萎了,或是在大雨里淹死了,那也与驾驶员不相干——正如与拖拉机不相关一样。
    
    驾驶员并不比银行更爱土地。他尽可以夸赞拖拉机——赞美它那机器制成的表面,它那雄伟的力量,汽缸震耳的吼声,但是这究竟不是他的拖拉机。拖拉机后边滚着亮晃晃的圆盘耙,用锋刃划开土地——这不像耕作,倒像施外科手术。一排圆盘耙把土划开,掀到右边,另一排圆盘耙又把土划开,掀到左边,圆盘耙的锋刃都被掀开的泥土擦得亮亮的。圆盘耙后面拖着的铁齿耙又把小小的泥块划开,把土均匀地铺平。耙后是长形的播种机——在车间里铸造的十二根弯曲的铁管,由齿轮推动着,按部就班地在土里插进抽出,毫无激情。驾驶员坐在铁驾驶座上,看着自己无意划出的那些直线,感到得意,看着非他所有、非他所爱的拖拉机,也感到得意,看着那股自己不能控制的力量,也感到得意。庄稼生长起来和收割的时候,没有人用手指头捏碎一撮泥土,让土屑从他的指头当中漏下去。没有人接触种子,或是渴望它成长起来。人们吃着并非他们种植的东西,大家跟面包都没什么关系了。土地在铁机器底下受苦受难,在机器底下渐渐死去。因为既没有人爱它,也没有人恨它;没有谁为它祈祷,也没有谁诅咒它。
    
    中午,拖拉机驾驶员在一家佃户近旁停下来一会儿,打开他的午餐——一个蜡纸包着的三明治:白面包夹着酸黄瓜、乳酪和午餐肉;还有一块烙着商标的馅饼,瞧着像发动机上的零件。他毫无滋味地吃着。还没有搬走的佃户们出来看他,他摘下护目镜和橡皮面具,眼睛周围留着一道白圈儿,鼻子和嘴的周围也留着一个大白圈儿,人家就趁这时候以好奇的神情望着他。拖拉机的排气管啪嗒啪嗒地继续响着,因为燃料价格低廉,与其重新烘热柴油机的管口,使它开动,还不如让它一直转着。好奇的孩子们紧紧地聚拢来,这些衣衫褴褛的小孩一面瞧着,一面吃着烤面饼。他们饥渴地看着他揭开三明治的包装纸,因饥饿而异常灵敏的鼻子嗅到了酸黄瓜、乳酪和午餐肉的气味。他们没有对驾驶员讲话,只望着他的手把食物送到嘴里去。他们没有看他咀嚼,他们的眼睛紧盯着那只拿三明治的手。过了一会儿,离不了这地方的佃户走出来,蹲在拖拉机旁边的阴影里。
    
    “原来是你,乔·戴维斯的儿子!”
    
    “不错。”驾驶员说。
    
    “那你怎么还来干这种活计,跟自己人作对?”
    
    “三块钱一天。东奔西跑找口饭吃,怎么也找不到,我受不了了。我还有老婆孩子,人得吃饭啊。三块钱一天,到手日结。”
    
    “这倒是对的。”佃户说,“可是为了你一天拿三块钱,二十户人家就要没饭吃了;为了你一天拿三块钱,上百人就得流落街头了。是不是这么回事?”
    
    驾驶员说道:“这事没法多想。我得顾自己的孩子。三块钱一天,到手日结。时代变了,先生,你还不知道吗?如今没个一万几千亩地,没一台拖拉机,就没法靠种地过活。庄稼地,我们这些人受用不起了,配不上了。你造不了汽车,又不是电话公司,光乱嚷嚷是不行的。唉,现在种庄稼也是这样,简直没办法。要不你也想想办法,到什么地方去赚一天三块钱吧,这是唯一的办法了。”
    
    佃户思量着:“这事情想来也真是奇怪。一个人如果有了一点小产业,那么这份资产就是他,是他的一部分,就像手脚一样。他有了田产,就能在田地上走,能在田地上劳作。收成不好的时候他发愁,雨下到地上的时候他就快活,那么这块田地就和他分不开。资产让他变得有本事了。即使他不是什么大人物,只要有这么一份资产,也是很有本事的。这是实话。”
    
    佃户又继续思量下去:“可要是一个人看不见自己的资产,又没时间去亲自照料,也不在上面劳作——那么,人就成了资产。他没法做自己想做的,想自己想要的。资产成了人的主宰,而且比他更强大。他自己却很渺小,也没有本事了。资产才是真本事——他成了资产的奴仆了。这也是实话。”
    
    驾驶员使劲嚼着那块烙了商标的馅饼,抛掉硬皮。“时代变了,知道吗?你转那种念头是养不活儿女的。快去挣一天三块钱,养活孩子吧。你别管旁人的儿女,只顾自己的孩子就是了。你讲那一套道理,就算讲出花儿来,也挣不到一天三块钱。要是你除了一天三块钱之外,还转着别的念头,大老板们就不会给你一天三块钱。”
    
    “为了你那三块钱,上百人要露宿街头了。我们有什么地方好去呢?”
    
    “这倒提醒我了,”驾驶员说,“你最好马上搬出去。我吃完了饭,就要穿过你门前的院子了。”
    
    “早上你把水井填掉了。”
    
    “我知道。我得照直线开才行。我吃完了饭,就得穿过你门前的院子。得照直线开。你认识我老爹乔·戴维斯,我才对你说实话。我接到了命令,每到有人家不搬出的地方——如果我闯了祸,你知道吧,就是开得太近了,把屋子撞塌一点儿——那我还可以多得两块钱奖赏。要知道,我最小的孩子还没穿过鞋呢。”
    
    “这房子可是我亲手盖的。你知不知道为盖这屋顶,我敲直了多少旧钉子。这椽子,是用钢丝捆在梁上的。这可是我的房子,我亲手盖的。你要撞倒它——我就从窗里拿枪打你。你敢开近了,我就像打兔子似的,一枪把你干掉。”
    
    “这不是我的问题。我也没法子。如果我不这么办,我就要失业。你想啊——你打死了我又怎样呢?他们只会把你绞死。可还没等到你上绞架,就会有另一个开拖拉机的家伙,把这房子撞倒。你该打死的不是我。”
    
    “这话有理。”佃户说,“是谁给你下的命令?我要把他找出来。应该杀了他才对。”
    
    “你错了。他也是听银行的命令来的。银行告诉他:‘把那些人清出去,否则我们就把你清出去。’”
    
    “那么,银行有行长,有董事会。我就拿一把来复枪\footnote{〔来复枪〕也叫线膛枪,指在枪管内壁刻有凹线(称为膛线)的枪械。膛线可以让子弹在枪管中产生旋转,使子弹出膛后飞行更稳定,提高射击精度和射程。},装好了弹药,闯进银行去。”
    
    驾驶员说道:“我听人说,银行也是听东部发来的命令。那命令上说:‘赶紧叫这块地赚钱,否则我们就要叫你关门。’”
    
    “这么说还有完没完了?我们到底该打死哪个人?不先把那个叫我饿死的人杀掉,我死也不瞑目。”
    
    “我不知道。也许你开枪打死谁都不行。也许问题根本就不在人身上。也许就像你说的,是资产在作怪。反正,我已经把我接到的命令告诉你了。”
    
    “我得想一想。”佃户说,“我们得一起把这事理清楚。要阻止这件事,总有办法的。这不像打雷地震,这是人搞出来的祸事,这是可以改正过来的,天知道的。”
    
    佃户坐在他的门口,驾驶员把机器弄得轰隆轰隆响了一阵,便开动了。拖拉机上的履带一起一落,一弯一直,铁耙梳理着土壤。播种机的铁杆插进地里。拖拉机划过门前的院子,于是,双脚踏实的地面变成了撒过种子的松土。拖拉机又从这里划过,不曾划过的空地只有十英尺\footnote{〔英尺〕英制长度单位。一英尺约等于0.305米。}宽了。于是他又开回来。钢铁的护板撞着了屋角,把墙撞倒。小屋整个一晃,便向旁边坍塌下去,像一只甲虫似的,被碾碎了。驾驶员戴着护目镜,鼻子和嘴上蒙着橡皮面具。拖拉机继续沿着直线划过去,空气和地面便随着它的轰隆声震荡着。佃户手里拿着来复枪,在它后面眼睁睁地看着。他老婆在他身边,老老实实的孩子们站在后面。大家眼巴巴地望着拖拉机开远了。


\end{document}