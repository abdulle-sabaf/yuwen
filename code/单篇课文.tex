\documentclass[12pt,UTF-8,openany]{ctexbook}
\usepackage{ctex}
\usepackage{titlesec}
\usepackage{xeCJK}
\usepackage{fontspec,xunicode,xltxtra}
\usepackage{xpinyin}
\usepackage{geometry}
\usepackage{indentfirst}
\usepackage{pifont}
\usepackage[perpage,symbol*]{footmisc}
% \usepackage{dblfnote}
% \DFNalwaysdouble % for this example

\geometry{a5paper,left=1.4cm,right=1.4cm,top=2.4cm,bottom=2.4cm}
\setmainfont{Arial}
\setCJKmainfont[BoldFont=STZhongsong]{汉字之美仿宋GBK 免费}
\xeCJKDeclareCharClass{CJK}{`0 -> `9}
\xeCJKsetup{AllowBreakBetweenPuncts=true}
\DefineFNsymbols{circled}{{\ding{192}}{\ding{193}}{\ding{194}}{\ding{195}}{\ding{196}}{\ding{197}}{\ding{198}}{\ding{199}}{\ding{200}}{\ding{201}}}
\setfnsymbol{circled}
\xpinyinsetup{ratio=0.44,hsep={.6em plus .6em},vsep={1em}}
\titleformat{\chapter}{\zihao{-1}\bfseries}{ }{16pt}{}
\titleformat{\section}{\zihao{-2}\bfseries}{ }{0pt}{}
\title{\zihao{0} \bfseries 小学语文课文集萃}
\setlength{\lineskip}{24pt}
\setlength{\parskip}{6pt}
\author{}
\date{}
\begin{document}
% \maketitle
% \tableofcontents
% \newpage

\begin{center}
    \begin{Huge}
        \textbf{太空一日}
    \end{Huge}
    
    \vspace{8pt}
    
    杨利伟\footnote{选自《天地九重》(解放军出版社2010年版),有删改。}

    \vspace{8pt}

\end{center}


% \begin{large}
    
\begin{center}
    \textbf{我以为自己要牺牲了}
\end{center}
    
    9时整,火箭尾部发出巨大的轰鸣声,数百吨高能燃料开始燃烧,八台发动机同时喷出炽热的火焰,高温高速的气体,几秒钟就把发射台下的上千吨水化为蒸汽。
    
    火箭起飞了。
    
    我全神贯注,肌肉紧绷,整个人收得像一块铁,准备执行动作。
    
    飞船缓缓升起,非常平稳,甚至比电梯还平稳。我感到压力远不像训练时想象的那么大,稍稍释然,全身绷紧的肌肉也渐渐放松下来。
    
    “逃逸塔\footnote{逃逸塔:飞船顶端的逃生装置。可在火箭升空期间出现危急状况时,让航天员迅速脱离危险区域。}分离”,“助推器分离”……
    
    火箭逐渐加速,我感到压力渐渐增强。这种负荷我们训练时承受过,变化幅度甚至比训练时还小些,所以我的身体感受还挺好,觉得没啥问题。
    
    然而,就在火箭上升到三四十公里的高度时,火箭和飞船产生了共振\footnote{共振:物体受外界振动刺激时,产生特别强烈的振动的现象。},开始急剧振动。这让我非常痛苦。
    
    人体对这种10赫兹\footnote{赫兹:每秒振动的次数。10赫兹表示每秒振动10次。}以下的振动非常敏感。它会让人的内脏产生共振。不仅如此,当时的负荷大约有六倍重力加速度\footnote{重力加速度:重力导致的加速度。六倍重力加速度相当于身体重量变为六倍,感觉如同自身五倍的重量压在全身。},两者叠加,实在太可怕了,我们从来没有进行过这种训练。
    
    意外出现了。
    
    共振时强时弱,痛苦越来越强烈,我异常清醒,只觉得五脏六腑\footnote{五脏六腑:中国传统医学将人的内脏分为五脏、六腑、奇恒之腑等,因此用五脏六腑泛指内脏。}似乎都要碎了。我几乎难以承受,觉得自己快不行了。
    
    当时,我以为飞船起飞时就是这样的。其实,起飞阶段发生的共振并非正常现象。
    
    共振持续26秒后,慢慢减轻。我从极度难受的状态解脱出来,一切不适都不见了,只感到从未有过的轻松和舒服,如释千钧重负\footnote{千钧重负:形容十分沉重的负担。钧:重量单位。一钧等于三十斤。},如同重生。我甚至觉得这个过程很耐人寻味。但在痛苦的极点,就在那短短一刹那,我真的以为自己要牺牲了。
    
    飞行回来后,我详细描述了这段难受的过程。经过分析研究,工作人员认为,飞船的共振主要来自火箭的振动。随后他们改进工艺,解决了这个问题。“神舟六号”飞行时,情况有了很大改善;后来的航天飞行中再没有出现过这种问题。聂海胜\footnote{聂海胜:中国航天员。2005年10月,他和费俊龙成功执行“神舟六号”载人航天飞行任务。}说:“我们乘坐的火箭、飞船都非常舒适,几乎感觉不到振动。”
    
    在空中度过那难以承受的26秒时,不仅我感觉特别漫长,地面的工作人员也陷入了空前的紧张中。因为通过大屏幕,飞船传回来的画面是定格的,我整个人一动不动,眼睛也不眨。大家都担心我是不是出了什么事故。
    
    后来,整流罩\footnote{整流罩:套在飞行器上的保护罩。用于减少空气阻力,免除飞行时气流、热流的影响。}打开,外面的光线透过舷窗\footnote{舷窗:飞船两侧的窗。舷:船的两侧边壁。}一下子照射进来,阳光很刺眼,我的眼睛忍不住眨了一下。
    
    就这一下,指挥大厅有人大声喊道:“快看啊,他眨眼了,利伟还活着!”所有的人都鼓掌欢呼起来。
    
    这是回到地面后,我看了升空时指挥大厅的录像才知道的。那一刻,所有的人都在流泪。看到这里的时候,我感动得说不出话来。
    
    \begin{center}
        \textbf{我看到了什么}
    \end{center}
    
    此后一切顺利。升空后10分钟左右,飞船仿佛一下子跳进了轨道。我突然有了失重的感觉。
    
    好容易等到地面指挥人员下达指令,我迫不及待地摘下束缚带,飘到舷窗边上。
    
    哈!太空和地球一下子出现在我眼前。
    
    我先望向地球。从飞船上看到的地球,只是一段弧面,不是完整的球体。因为地球的半径有六千多公里,而飞船距离地面343公里左右。我们平常在地理书上看到的地球照片,是由飞行轨道更高的同步卫星拍摄而来。
    
    地球真的太漂亮了。她散发着柔和的光芒,仿佛披着蓝色纱裙和白色飘带的仙女,款款而行\footnote{款款:徐缓、从容地。}。蓝色的弧面之外,是深远幽黑的宇宙。
    
    飞船每90分钟就绕地球一圈,一共飞行了14圈。我也看了14次日出和日落。我曾在新疆的天山上,也曾站在家乡的大海边看日出,但都无法与太空中的日出相比。一条亮白的金弧不断延伸,太阳就是镶在中间的宝珠,发出炫目的光。金弧逐渐扩散开来,把光明涂抹在广袤\footnote{广袤:广阔、辽阔。}的弧面上,一切都清晰起来。日落时,一切又追随着太阳涌去,汇成一条光弧,再彻底消失。
    
    在太空中,我可以准确判断各大洲和各个国家的方位。因为飞船有预定的飞行轨迹,显示屏上实时标示着飞船走到哪个位置,投影到地球上是哪一点。有图可依,一目了然。
    
    即使不借助仪器和地图,以我们航天课程中学到的知识,从山脉的轮廓,海岸线的走向以及河流的形状,我也基本可以判断出飞船正经过哪个洲的上空,正在经过哪个国家。
    
    经过亚洲,特别是到中国上空时,我就仔细辨别大概到哪个省了。飞船经过中国上空的时间很短,每一次飞过后,我都期待着下一次。
    
    飞船的轨迹大都是不重复的,在距离地面三百多公里的高度上俯瞰,视野广阔,祖国的各个省份我大都看到了。
    
    我曾俯瞰我们的首都北京。白天它是燕山山脉边的一片灰白色,分辨不清;夜晚则呈现一片红晕。那里有我的战友和亲人。
    
    我看到中国东部优美的海岸线、长白山脉,那里是辽宁,我的家乡;我看到甘肃、新疆,披着积雪的昆仑山脉和大片沙漠,我曾驾机飞过那里,也从那里乘火箭升空;我看到了曲折的黄河横穿陕西、山西、山东数省;我看到了西藏和青藏高原,我看到了四川、安徽、江苏、上海,蜿蜒的长江奔向大海;我看到了东南方向的台湾岛,看上去它与大陆几乎没有间隔;我看到了宽广的内蒙古一片平阔,而我将在那里降落……
    
    \begin{center}
        \textbf{神秘的敲击声}
    \end{center}
    
    作为首飞航天员,除了一些小难题,我还遇到了许多突发的、原因不明的、不在预案\footnote{预案:预备方案。预先考虑可能出现的情况,作出的计划。}中的情况。
    
    比如,当飞船刚刚入轨,进入失重状态时,百分之八九十的航天员都会产生一种“本末倒置”的错觉。这种错觉很难受,明明是朝上坐的,却感觉脑袋朝下。如果不消除这种倒悬的错觉,就会觉得自己一直在倒着飞,很难受,严重时还可能诱发空间运动病\footnote{空间运动病:人的空间平衡感失调导致的疾病。晕车、晕船、晕飞机,都属于空间运动病。},影响任务的完成。
    
    在地面时,没人提到过这种情况。即使知道,训练也无法模拟。估计在我之前遨游太空的国外航天员有类似体会,但他们从未对我说起过。
    
    在这个情况下,没别的办法,只能靠意志力克服这种错觉。我想像自己在地面训练的情景,眼睛闭着猛想,不停地想,给身体一个适应的过程。几十分钟后,我终于调整过来了。
    
    “神舟六号”和“神舟七号”升空后,航天员都产生了这种错觉,但他们已有心理准备,因为我跟他们仔细讲过。而且,飞船舱体也经过改进,内壁上下刷了不同的颜色:天花板是白色的,地板是褐色的。这样有助于航天员迅速调整感觉。
    
    我在太空还遇到另一个至今仍然原因不明的情况,那就是时不时出现的敲击声。这个声音是突然出现的。并不一直响,而是一阵一阵的。不管白天还是黑夜,毫无规律,说不准什么时候就响几声。既不是外面传进来的声音,也不是飞船里面的声音,仿佛谁在外面敲飞船的船体。很难准确描述它:不是叮叮的,也不是当当的,更像是用一把木头锤子敲铁桶,咚……咚咚……咚……
    
    鉴于飞船的运行一直很正常,我并没有向地面报告这一情况。但我自己还是很紧张,因为第一次飞行,生怕哪里出了问题。每当响声传来的时候,我就趴在舷窗那里,边听边看,试图找出响声所在,但什么也没能发现。
    
    回到地面后,人们对这个神秘的声音做过许多猜测。技术人员想弄清它到底来自哪里,就用各种办法模拟它,拿着录音让我一次又一次听,我却总是觉得不像。对航天员的最基本要求是严谨,不是当时的声音,我就不能签字,所以他们就让我反复听各种声音,断断续续听了一年多。但是直到现在,那个神秘的声音也没有在我耳边准确地再现过。
    
    在“神舟六号”和“神舟七号”飞行时,这个声音又出现了,但我告诉航天员:“出现这个声音别害怕,是正常现象。”

    \begin{center}
        \textbf{归途如此惊心动魄}
    \end{center}
    
    5时35分,北京航天指挥中心向飞船发出“返回”指令。飞船开始在343公里高的轨道上制动\footnote{制动:使运动减慢或停止。},就像刹车一样。
    
    飞船先是在轨道上进行180度调姿——返回时要让推进舱在前,这就需要“掉头”。
    
    “制动发动机关机!”5时58分,飞船的速度减到一定数值,开始脱离原来的轨道,进入无动力飞行状态。
    
    6时4分,飞船下降至距地100公里,进入逐渐稠密的大气层。
    
    这时飞船的飞行速度仍然很快,遇到空气阻力后,它急剧减速,产生了近四倍重力加速度的过载\footnote{过载:过大的加速度(比重力加速度更大的加速度)。}。我的前胸和后背都承受着很大的压力。我们平时已经训练过如何应对这种情况,因此身体应付自如,也没有紧张。
    
    让我紧张以至于惊慌的另有原因。
    
    飞船进入了“黑障”区\footnote{“黑障”:航天飞行中出现的现象。在距离地面数十公里的高空高速飞行时,飞行器和大气摩擦产生的高温,使气体分子电离,并在飞行器表面形成离子层,阻碍电磁波通过。飞行器无法用电磁波与外界联系,因此称为“黑障”。},距地大约80公里到40公里。首先是快速行进的飞船与大气摩擦,产生的高温把舷窗外面烧得一片通红;接着在映红的舷窗外,有红的白的碎片不停划过。飞船的外表面有防烧蚀层,它是耐高温的,随着温度升高,开始剥落,并在剥落的过程中会带走一部分热量。我学习过这方面的知识,看到这种情形,知道是怎么回事。
    
    但随后发生的情况让我非常紧张——右边的舷窗开始出现裂纹。窗外烧得跟炼钢炉一样,而窗上出现裂纹。那纹路就跟强化玻璃被打碎之后的小碎纹一样。这种细细的碎纹,眼看着越来越多……说不恐惧那是假话。你想啊,窗外边可是有1600至1800摄氏度!
    
    我的汗水出来了。这时舱内的温度也在升高,但并没到让我瞬间出汗的程度,主要还是因为紧张。
    
    我现在还能清楚地记起当时的情形:飞船急速下降,跟空气摩擦产生激波\footnote{激波:气流的速度超过了气体扰动传播的速度,使气流突然压缩变稠密,产生高温高热的现象。},不仅带来极高的温度,还伴随着尖利的呼啸声;飞船带着不小的过载,不停振动着,里面咯吱咯吱乱响。外面高温,不怕!有碎片划过,不怕!过载,也能承受!但是,看到舷窗玻璃开始出现裂缝,我紧张了,心想:完了,这个舷窗不行了。美国的“哥伦比亚号”航天飞机,不就是这样出事的吗?先是一块防热板出现裂缝,然后高热就使飞机解体了。这么大一个舷窗坏了,那还得了!
    
    右边的舷窗裂到一半的时候,左边的舷窗也开始出现裂纹。这反倒让我稍微放心了:哦——可能没什么问题!因为如果是故障,重复出现的概率并不高。
    
    回来之后,我才知道,飞船的舷窗外做了一层防烧涂层,是这个涂层烧裂了,而不是窗玻璃本身出现了问题。为什么两边没有同时出现裂纹呢?因为两边用了不同的材料。以前每次进行飞船发射与返回的实验,返回的飞船舱体经过高温烧灼,舷窗黑乎乎的,工作人员看不到这些裂纹。如果不是在飞船内亲眼所见,谁都不会想到有这种情况。
    
    距离地面还有40公里,飞船出了“黑障”区,速度已经降下来。一个关键的操作——抛伞,即将开始。这时舷窗已经烧得黑乎乎的,我抱着操作盒,屏息凝神,等待着配合程序:到哪里该做什么,该发什么指令,判断和操作都必须准确无误。
    
    6时14分,飞船距地10公里。飞船抛开降落伞盖,并迅速带出引导伞。
    
    这是一个剧烈的动作,能听到“砰”的一声,非常响。我在里边感觉被狠狠地一拽,瞬间过载很大,对身体的冲击也非常厉害。接下来是一连串的快速动作。引导伞出来后,紧跟着把减速伞也带出来,减速伞让飞船减速下落,16秒之后再把主伞带出来。
    
    其实最折磨人的就是这段过程了。随着一声巨响,你会感到突然一减速;引导伞一开,使劲一提,这个劲很大,会把人吓一跳;减速伞一开,又往那边一拽;主伞开时又把你拉到另一边。每次力量都相当大。飞船晃荡得很厉害,让人不知道是怎么回事。
    
    我们航天员是很重视这段过程的:伞开得好等于安全有保障,至少不会丢了性命。所以我被七七八八地拽了一通,平稳下来后心里却真踏实——数据出来了,速度控制在规定范围内。我知道,这伞肯定是开好了!
    
    离地面5公里的时候,飞船抛掉防热大底,露出缓冲发动机。同时主伞也变成双点吊挂,让飞船摆正姿态,在风中晃悠着落向地面。
    
    飞船离地面1.2米时,缓冲发动机点火。接着,飞船“嗵”的一下落地了。
    
    我感觉落地很重,飞船弹了起来。在它第二次落地时,我迅速按下了切伞开关\footnote{切伞:将飞船与降落伞分离。}。飞船停住了。此时是2003年10月16日6时23分。而这一时刻,正好是天安门当天升国旗的时刻,这是一个无法设计的巧合。
    
    那一刻四周寂静无声。舷窗黑乎乎的,看不到外面。
    
    过了几分钟,我隐约听到了叫喊声,手电的光从烧黑的舷窗上隐约照进来。他们找到飞船了!我听到外面插上钥匙的声音,舱门动弹了……
    
% \end{large}

\end{document}