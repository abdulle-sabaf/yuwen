\documentclass[12pt,UTF-8,openany]{ctexbook}
\usepackage{ctex}
\usepackage{titlesec}
\usepackage{xeCJK}
\usepackage{verse}
\usepackage{fontspec,xunicode,xltxtra}
\usepackage{xpinyin}
\usepackage{geometry}
\usepackage{indentfirst}
\usepackage{pifont}
\usepackage{enumitem}
\usepackage[perpage,symbol*]{footmisc}
\usepackage[table,dvipsnames]{xcolor}

\geometry{a5paper,left=1.4cm,right=1.4cm,top=2.3cm,bottom=2.3cm}
\renewcommand{\footnotesize}{\fontsize{8.5pt}{10.5pt}\selectfont}
\setmainfont{Arial}
\setCJKmainfont[BoldFont=STZhongsong]{汉字之美仿宋GBK 免费}
\xeCJKDeclareCharClass{CJK}{`0 -> `9}
\xeCJKsetup{AllowBreakBetweenPuncts=true}
\DefineFNsymbols{circled}{{\ding{192}}{\ding{193}}{\ding{194}}{\ding{195}}{\ding{196}}{\ding{197}}{\ding{198}}{\ding{199}}{\ding{200}}{\ding{201}}}
\setfnsymbol{circled}
\xpinyinsetup{ratio=0.5,hsep={.6em plus .6em},vsep={1em}}
\definecolor{script-1-0}{RGB}{28,120,180}
\definecolor{script-1-1}{RGB}{70,170,40}
\definecolor{script-1-2}{RGB}{119,31,180}
\definecolor{script-1-3}{RGB}{227,119,194}
\definecolor{script-1-4}{RGB}{188,189,34}
\definecolor{script-1-5}{RGB}{44,160,135}
\definecolor{script-1-6}{RGB}{214,39,40}
\definecolor{script-1-7}{RGB}{140,86,75}
\definecolor{script-1-8}{RGB}{255,127,14}
\definecolor{script-2-0}{RGB}{140,46,65}
\definecolor{script-2-1}{RGB}{31,119,180}
\definecolor{script-2-2}{RGB}{23,190,127}
\definecolor{script-3-0}{RGB}{31,119,180}
\definecolor{script-3-1}{RGB}{140,86,75}
\definecolor{script-3-2}{RGB}{160,40,127}
\definecolor{script-3-3}{RGB}{44,160,44}
\definecolor{script-3-4}{RGB}{23,190,207}
\definecolor{script-3-5}{RGB}{75,75,75}

\titleformat{\chapter}{\zihao{-1}\bfseries}{ }{16pt}{}
\titleformat{\section}{\zihao{-2}\bfseries}{ }{0pt}{}
\title{\zihao{0} \bfseries 中学语文课文集萃}
\setlength{\lineskip}{24pt}
\setlength{\parskip}{6pt}
\author{}
\date{}
\begin{document}
\maketitle
\tableofcontents
\newpage

\chapter{匆匆}

\begin{normalsize}
    
    燕子去了,有再来的时候;杨柳枯了,有再青的时候;桃花谢了,有再开的时候。但是,聪明的,你告诉我,我们的日子为什么一去不复返呢?——是有人偷了他们吧:那是谁?又藏在何处呢?是他们自己逃走了吧:现在又到了哪里呢?
    
    我不知道他们给了我们多少日子,但我的手确乎是渐渐空虚了。在默默里算着,八千多日子已经从我手中溜去;像针尖上一滴水滴在大海里,我的日子滴在时间的流里,没有声音,也没有影子。我不禁头涔涔而泪潸潸了。
    
    去的尽管去了,来的尽管来着,去来的中间,又怎样地匆匆呢?早上我起来的时候,小屋里射进两三方斜斜的太阳。太阳他有脚啊,轻轻悄悄地挪移了,我也茫茫然跟着旋转。于是——洗手的时候,日子从水盆里过去;吃饭的时候,日子从饭碗里过去;默默时,便从凝然的双眼前过去;我觉察他去得匆匆了,伸出手遮挽时,他又从遮挽的手边过去;天黑时,我躺在床上,他便伶伶俐俐地从我身上跨过,从我脚边飞去了;等我睁开眼和太阳再见,这算又溜走了一日;我掩面叹息,但是新来的日子的影儿又开始在叹息里闪过了。
    
    在逃去如飞的日子里,在千门万户的世界里的我能做什么呢?只有徘徊罢了,只有匆匆罢了。在八千多日的匆匆里,除徘徊外,又剩些什么呢?过去的日子如轻烟,被微风吹散了,如薄雾,被初阳蒸融了。我留着些什么痕迹呢?我何曾留着像游丝样的痕迹呢?我赤裸裸来到这世界,转眼间也将赤裸裸地回去吧?但不能平的,为什么偏要白白走这一遭啊?
    
    你聪明的,告诉我,我们的日子为什么一去不复返呢?
    
\end{normalsize}


\newpage

\textbf{注释}:

\vspace{-1em}

\begin{itemize}
    \setlength\itemsep{-0.2em}
    \item 〔涔涔〕流汗的样子。
    \item 〔潸潸〕流泪不止的样子。
    \item 〔伶俐〕轻快灵活。
    \item 〔遮挽〕阻拦挽留。
\end{itemize}

\chapter{济南的冬天}

\begin{normalsize}
    
    对于一个在北平\footnote{〔北平〕北京的旧称。}住惯的人,像我,冬天要是不刮风,便觉得是奇迹;济南的冬天是没有风声的。对于一个刚由伦敦\footnote{〔伦敦〕英国首都。}回来的人,像我,冬天要能看得见日光,便觉得是怪事;济南的冬天是响晴的。自然,在热带\footnote{〔热带〕地球南北回归线之间的地带。四季炎热。}的地方,日光是永远那么毒,响亮的天气,反有点叫人害怕。可是,在北中国的冬天,而能有温晴的天气,济南真得算个宝地。
    
    设若\footnote{〔设若〕如果、假若。}单单是有阳光,那也算不了出奇。请闭上眼睛想:一个老城,有山有水,全在天底下晒着阳光,暖和安适地睡着,只等春风来把它们唤醒,这是不是个理想的境界?小山整把济南围了个圈儿,只有北边缺着点口儿。这一圈小山在冬天特别可爱,好像是把济南放在一个小摇篮里,它们安静不动地低声地说:“你们放心吧,这儿准保暖和。”真的,济南的人们在冬天是面上含笑的。他们一看那些小山,心中便觉得有了着落,有了依靠。他们由天上看到山上,便不知不觉地想起:“明天也许就是春天了吧?这样的温暖,今天夜里山草也许就绿起来了吧?”就是\footnote{〔就是〕这里同“就算”。}这点幻想不能一时实现,他们也并不着急,因为有这样慈善的冬天,干啥还希望别的呢!
    
    最妙的是下点小雪呀。看吧,山上的矮松越发的青黑,树尖上顶着一髻儿白花\footnote{〔髻〕原指头顶或脑后盘成的各种形状的头发。这里比喻树顶上的积雪如发髻一般。},仿佛雪地里的灰松鼠。山尖全白了,给蓝天镶上一道银边。山坡上,有的地方雪厚点,有的地方草色还露着;这样,一道儿白,一道儿暗黄,给山们穿上一件带水纹的花衣;看着看着,这件花衣好像被风儿吹动,叫你希望看见一点更美的山的肌肤。等到快日落的时候,微黄的阳光斜射在山腰上,那点薄雪好像忽然害了羞,微微露出点粉色。就是下小雪吧,济南是受不住大雪的,那些小山太秀气!
    
    古老的济南,城里那么狭窄,城外又那么宽敞,山坡上卧着些小村庄,小村庄的房顶上卧着点雪,对,这是张小水墨画,或者是唐代的名手画的吧。
    
    那水呢,不但不结冰,倒反在绿藻上冒着点热气。水藻真绿,把终年贮蓄的绿色全拿出来了。天儿越晴,水藻越绿,就凭这些绿的精神,水也不忍得冻上;况且那长枝的垂柳还要在水里照个影儿呢!看吧,由澄清的河水慢慢往上看吧,空中,半空中,天上,自上而下全是那么清亮,那么蓝汪汪的,整个的是块空灵的蓝水晶。这块水晶里,包着红屋顶,黄草山,像地毯上的小团花的小灰色树影;这就是冬天的济南。
    
\end{normalsize}


\newpage

\textbf{注释}:

\vspace{-1em}

\begin{itemize}
    \setlength\itemsep{-0.2em}
    \item 〔响晴〕晴朗无云。
    \item 〔水墨画〕用水、墨而不用彩色颜料的国画。
    \item 〔名手〕这里指有名的画家。手:擅长做某事的人。
    \item 〔贮蓄〕储存,积聚。
    \item 〔空灵〕空静而又灵动,难以捉摸。
\end{itemize}

\chapter{紫藤萝瀑布}

\begin{normalsize}
    
    我不由得停住了脚步。
    
    从未见过开得这样盛的藤萝,只见一片辉煌的淡紫色,像一条瀑布,从空中垂下,不见其发端,也不见其终极。只是深深浅浅的紫,仿佛在流动,在欢笑,在不停地生长。紫色的大条幅上,泛着点点银光,就像迸溅的水花。仔细看时,才知道那是每一朵紫花中的最浅淡的部分,在和阳光互相挑逗。
    
    这里春红已谢,没有赏花的人群,也没有蜂围蝶阵。有的就是这一树闪光的、盛开的藤萝。花朵儿一串挨着一串,一朵接着一朵,彼此推着挤着,好不活泼热闹!
    
    “我在开花!”它们在笑。
    
    “我在开花!”它们嚷嚷。
    
    每一穗花都是上面的盛开、下面的待放 。颜色便上浅下深,好像那紫色沉淀下来了,沉淀在最嫩最小的花苞里。每一朵盛开的花就像是一个小小的张满了的帆,帆下带着尖底的舱,船舱鼓鼓的;又像一个忍俊不禁的笑容,就要绽开似的。那里装的是什么仙露琼浆?我凑上去,想摘一朵。
    
    但是我没有摘。我没有摘花的习惯。我只是伫立凝望,觉得这一条紫藤萝瀑布不只在我眼前,也在我心上缓缓流过。流着流着,它带走了这些时一直压在我心上的焦虑和悲痛,那是关于生死谜、手足情的。我沉浸在这繁密的花朵的光辉中,别的一切暂时都不存在,有的只是精神的宁静和生的喜悦。
    
    这里除了光彩,还有淡淡的芳香,香气似乎也是浅紫色的,梦幻一般轻轻地笼罩着我。忽然记起十多年前家门外也曾有过一大株紫藤萝,它依傍一株枯槐爬得很高,但花朵从来都稀落,东一穗西一串伶仃地挂在树梢,好像在试探什么。后来索性连那稀零的花串也没有了。园中别的紫藤花架也都拆掉,改种了果树。那时的说法是,花和生活腐化有什么必然关系。我曾遗憾地想:这里再也看不见藤萝花了。
    
    过了这么多年,藤萝又开花了,而且开得这样盛,这样密,紫色的瀑布遮住了粗壮的盘虬卧龙般的枝干,不断地流着,流着,流向人的心底。
    
    花和人都会遇到各种各样的不幸,但是生命的长河是无止境的。我抚摸了一下那小小的紫色的花舱,那里满装生命的酒酿,它张满了帆,在这闪光的花的河流上航行。它是万花中的一朵,也正是一朵朵花,组成了万花灿烂的流动的瀑布。
    
    在这浅紫色的光辉和浅紫色的芳香中,我不觉加快了脚步。
    
\end{normalsize}


\newpage

\textbf{注释}:

\vspace{-1em}

\begin{itemize}
    \setlength\itemsep{-0.2em}
    \item 〔伶仃〕孤独无依靠。
    \item 〔盘虬卧龙〕盘卧的虬龙。
    \item 〔忍俊不禁〕忍不住笑出来。
    \item 〔仙露琼浆〕传说中天上神仙喝的酒,形容美酒。
\end{itemize}

\chapter{太空一日}

\begin{normalsize}
    
    \begin{large}\textbf{我以为自己要牺牲了}\end{large}
    
    9时整,火箭尾部发出巨大的轰鸣声,数百吨高能燃料开始燃烧,八台发动机同时喷出炽热的火焰,高温高速的气体,几秒钟就把发射台下的上千吨水化为蒸汽。
    
    火箭起飞了。
    
    我全神贯注,肌肉紧绷,整个人收得像一块铁,准备执行动作。
    
    飞船缓缓升起,非常平稳,甚至比电梯还平稳。我感到压力远不像训练时想象的那么大,稍稍释然,全身绷紧的肌肉也渐渐放松下来。
    
    “逃逸塔\footnote{〔逃逸塔〕飞船顶端的逃生装置。可在火箭升空期间出现危急状况时,让航天员迅速脱离危险区域。}分离”,“助推器分离”……
    
    火箭逐渐加速,我感到压力渐渐增强。这种负荷我们训练时承受过,变化幅度甚至比训练时还小些,所以我的身体感受还挺好,觉得没啥问题。
    
    然而,就在火箭上升到三四十公里的高度时,火箭和飞船产生了共振\footnote{〔共振〕物体受外界振动刺激时,产生特别强烈的振动的现象。},开始急剧振动。这让我非常痛苦。
    
    人体对10赫兹\footnote{〔赫兹〕每秒振动的次数。10赫兹表示每秒振动10次。}以下的振动非常敏感。它会让人的内脏产生共振。不仅如此,当时的负荷大约有六倍重力加速度\footnote{〔重力加速度〕重力导致的加速度。六倍重力加速度相当于身体重量变为六倍,感觉如同自身五倍的重量压在全身。},两者叠加,实在太可怕了,我们从来没有进行过这种训练。
    
    意外出现了。
    
    共振时强时弱,痛苦越来越强烈,我异常清醒,只觉得五脏六腑似乎都要碎了。我几乎难以承受,觉得自己快不行了。
    
    当时,我以为飞船起飞时就是这样的。其实,起飞阶段发生的共振并非正常现象。
    
    共振持续26秒后,慢慢减轻。我从极度难受的状态解脱出来,一切不适都不见了,只感到从未有过的轻松和舒服,如释千钧重负,如同重生。我甚至觉得这个过程很耐人寻味。但在痛苦的极点,就在那短短一刹那,我真的以为自己要牺牲了。
    
    飞行回来后,我详细描述了这段难受的过程。经过分析研究,工作人员认为,飞船的共振主要来自火箭的振动。随后他们改进工艺,解决了这个问题。“神舟六号”飞行时,情况有了很大改善;后来的航天飞行中再没有出现过这种问题。聂海胜\footnote{〔聂海胜〕中国航天员。2005年10月,他和费俊龙成功执行“神舟六号”载人航天飞行任务。}说:“我们乘坐的火箭、飞船都非常舒适,几乎感觉不到振动。”
    
    在空中度过那难以承受的26秒时,不仅我感觉特别漫长,地面的工作人员也陷入了空前的紧张中。因为通过大屏幕,飞船传回来的画面是定格的,我整个人一动不动,眼睛也不眨。大家都担心我是不是出了什么事故。
    
    后来,整流罩\footnote{〔整流罩〕套在飞行器上的保护罩。用于减少空气阻力,免除飞行时气流、热流的影响。}打开,外面的光线透过舷窗一下子照射进来,阳光很刺眼,我的眼睛忍不住眨了一下。
    
    就这一下,指挥大厅有人大声喊道:“快看啊,他眨眼了,利伟还活着!”所有的人都鼓掌欢呼起来。
    
    这是回到地面后,我看了升空时指挥大厅的录像才知道的。那一刻,所有的人都在流泪。看到这里的时候,我感动得说不出话来。
    
    \begin{large}\textbf{我看到了什么}\end{large}
    
    此后一切顺利。升空后10分钟左右,飞船仿佛一下子跳进了轨道。我突然有了失重的感觉。
    
    好容易等到地面指挥人员下达指令,我迫不及待地摘下束缚带,飘到舷窗边上。
    
    哈!太空和地球一下子出现在我眼前。
    
    我先望向地球。从飞船上看到的地球,只是一段弧面,不是完整的球体。因为地球的半径有六千多公里,而飞船距离地面343公里左右。我们平常在地理书上看到的地球照片,是由飞行轨道更高的同步卫星拍摄而来。
    
    地球真的太漂亮了。她散发着柔和的光芒,仿佛披着蓝色纱裙和白色飘带的仙女,款款而行。蓝色的弧面之外,是深远幽黑的宇宙。
    
    飞船每90分钟就绕地球一圈,一共飞行了14圈。我也看了14次日出和日落。我曾在新疆的天山上,也曾站在家乡的大海边看日出,但都无法与太空中的日出相比。一条亮白的金弧不断延伸,太阳就是镶在中间的宝珠,发出炫目的光。金弧逐渐扩散开来,把光明涂抹在广袤的弧面上,一切都清晰起来。日落时,一切又追随着太阳涌去,汇成一条光弧,再彻底消失。
    
    在太空中,我可以准确判断各大洲和各个国家的方位。因为飞船有预定的飞行轨迹,显示屏上实时标示着飞船走到哪个位置,投影到地球上是哪一点。有图可依,一目了然。
    
    即使不借助仪器和地图,以我们航天课程中学到的知识,从山脉的轮廓,海岸线的走向以及河流的形状,我也基本可以判断出飞船正经过哪个洲的上空,正在经过哪个国家。
    
    经过亚洲,特别是到中国上空时,我就仔细辨别大概到哪个省了。飞船经过中国上空的时间很短,每一次飞过后,我都期待着下一次。
    
    飞船的轨迹大都是不重复的,在距离地面三百多公里的高度上俯瞰,视野广阔,祖国的各个省份我大都看到了。
    
    我曾俯瞰我们的首都北京。白天它是燕山山脉边的一片灰白色,分辨不清;夜晚则呈现一片红晕。那里有我的战友和亲人。
    
    我看到中国东部优美的海岸线、长白山脉,那里是辽宁,我的家乡;我看到甘肃、新疆,披着积雪的昆仑山脉和大片沙漠,我曾在那里驾机飞行,也从那里乘火箭升空;我看到了曲折的黄河横穿陕西、山西、山东数省;我看到了西藏和青藏高原,我看到了四川、安徽、江苏、上海,蜿蜒的长江奔向大海;我看到了东南方向的台湾岛,看上去它与大陆几乎没有间隔;我看到了宽广的内蒙古一片平阔,而我将在那里降落……
    
    \begin{large}\textbf{神秘的敲击声}\end{large}
    
    作为首飞航天员,除了一些小难题,我还遇到了许多突发的、原因不明的、不在预案中的情况。
    
    比如,当飞船刚刚入轨,进入失重状态时,百分之八九十的航天员都会产生一种“本末倒置”的错觉。这种错觉很难受,明明是朝上坐的,却感觉脑袋朝下。如果不消除这种倒悬的错觉,就会觉得自己一直在倒着飞,很难受,严重时还可能诱发空间运动病\footnote{〔空间运动病〕人的空间平衡感失调导致的疾病。晕车、晕船、晕飞机,都属于空间运动病。},影响任务的完成。
    
    在地面时,没人提到过这种情况。即使知道,训练也无法模拟。估计在我之前遨游太空的国外航天员有类似体会,但他们从未对我说起过。
    
    在这个情况下,没别的办法,只能靠意志力克服这种错觉。我想像自己在地面训练的情景,眼睛闭着猛想,不停地想,给身体一个适应的过程。几十分钟后,我终于调整过来了。
    
    “神舟六号”和“神舟七号”升空后,航天员都产生了这种错觉,但他们已有心理准备,因为我跟他们仔细讲过。而且,飞船舱体也经过改进,内壁上下刷了不同的颜色:天花板是白色的,地板是褐色的。这样有助于航天员迅速调整感觉。
    
    我在太空还遇到另一个至今仍然原因不明的情况,那就是时不时出现的敲击声。这个声音是突然出现的。并不一直响,而是一阵一阵的。不管白天还是黑夜,毫无规律,说不准什么时候就响几声。既不是外面传进来的声音,也不是飞船里面的声音,仿佛谁在外面敲飞船的船体。很难准确描述它:不是叮叮的,也不是当当的,更像是用一把木头锤子敲铁桶,咚……咚咚……咚……
    
    鉴于飞船的运行一直很正常,我并没有向地面报告这一情况。但我自己还是很紧张,因为第一次飞行,生怕哪里出了问题。每当响声传来的时候,我就趴在舷窗那里,边听边看,试图找出响声所在,但什么也没能发现。
    
    回到地面后,人们对这个神秘的声音做过许多猜测。技术人员想弄清它到底来自哪里,就用各种办法模拟它,拿着录音让我一次又一次听,我却总是觉得不像。对航天员的最基本要求是严谨,不是当时的声音,我就不能签字,所以他们就让我反复听各种声音,断断续续听了一年多。但是直到现在,那个神秘的声音也没有在我耳边准确地再现过。
    
    在“神舟六号”和“神舟七号”飞行时,这个声音又出现了,但我告诉航天员:“出现这个声音别害怕,是正常现象。”
    
    \begin{large}\textbf{归途如此惊心动魄}\end{large}
    
    5时35分,北京航天指挥中心向飞船发出“返回”指令。飞船开始在343公里高的轨道上制动,就像刹车一样。
    
    飞船先是在轨道上进行180度调姿——返回时要让推进舱在前,这就需要“掉头”。
    
    “制动发动机关机!”5时58分,飞船的速度减到一定数值,开始脱离原来的轨道,进入无动力飞行状态。
    
    6时4分,飞船下降至距地100公里,进入逐渐稠密的大气层。
    
    这时飞船的飞行速度仍然很快,遇到空气阻力后,它急剧减速,产生了近四倍重力加速度的过载\footnote{〔过载〕过大的加速度(比重力加速度更大的加速度)。}。我的前胸和后背都承受着很大的压力。我们平时已经训练过如何应对这种情况,因此身体应付自如,也没有紧张。
    
    让我紧张以至于惊慌的另有原因。
    
    飞船进入了“黑障”区\footnote{〔“黑障”〕航天飞行中出现的现象。在距离地面数十公里的高空高速飞行时,飞行器和大气摩擦产生的高温,使气体分子电离,并在飞行器表面形成离子层,阻碍电磁波通过。飞行器无法用电磁波与外界联系,因此称为“黑障”。},距地大约80公里到40公里。首先是快速行进的飞船与大气摩擦,产生的高温把舷窗外面烧得一片通红;接着在映红的舷窗外,有红的白的碎片不停划过。飞船的外表面有防烧蚀层,它是耐高温的,随着温度升高,开始剥落,并在剥落的过程中会带走一部分热量。我学习过这方面的知识,看到这种情形,知道是怎么回事。
    
    但随后发生的情况让我非常紧张——右边的舷窗开始出现裂纹。窗外烧得跟炼钢炉一样,而窗上出现裂纹。那纹路就跟强化玻璃被打碎之后的小碎纹一样。这种细细的碎纹,眼看着越来越多……说不恐惧那是假话。你想啊,窗外边可是有1600至1800摄氏度!
    
    我的汗水出来了。这时舱内的温度也在升高,但并没到让我瞬间出汗的程度,主要还是因为紧张。
    
    我现在还能清楚地记起当时的情形:飞船急速下降,跟空气摩擦产生激波\footnote{〔激波〕气流的速度超过了气体扰动传播的速度,使气流突然压缩变稠密,产生高温高热的现象。},不仅带来极高的温度,还伴随着尖利的呼啸声;飞船带着不小的过载,不停振动着,里面咯吱咯吱乱响。外面高温,不怕!有碎片划过,不怕!过载,也能承受!但是,看到舷窗玻璃开始出现裂缝,我紧张了,心想:完了,这个舷窗不行了。美国的“哥伦比亚号”航天飞机,不就是这样出事的吗?先是一块防热板出现裂缝,然后高热就使飞机解体了。这么大一个舷窗坏了,那还得了!
    
    右边的舷窗裂到一半的时候,左边的舷窗也开始出现裂纹。这反倒让我稍微放心了:哦——可能没什么问题!因为如果是故障,重复出现的概率并不高。
    
    回来之后,我才知道,飞船的舷窗外做了一层防烧涂层,是这个涂层烧裂了,而不是窗玻璃本身出现了问题。为什么两边没有同时出现裂纹呢?因为两边用了不同的材料。以前每次进行飞船发射与返回的实验,返回的飞船舱体经过高温烧灼,舷窗黑乎乎的,工作人员看不到这些裂纹。如果不是在飞船内亲眼所见,谁都不会想到有这种情况。
    
    距离地面还有40公里,飞船出了“黑障”区,速度已经降下来。一个关键的操作——抛伞,即将开始。这时舷窗已经烧得黑乎乎的,我抱着操作盒,屏息凝神,等待着配合程序:到哪里该做什么,该发什么指令,判断和操作都必须准确无误。
    
    6时14分,飞船距地10公里。飞船抛开降落伞盖,并迅速带出引导伞。
    
    这是一个剧烈的动作,能听到“砰”的一声,非常响。我在里边感觉被狠狠地一拽,瞬间过载很大,对身体的冲击也非常厉害。接下来是一连串的快速动作。引导伞出来后,紧跟着把减速伞也带出来,减速伞让飞船减速下落,16秒之后再把主伞带出来。
    
    其实最折磨人的就是这段过程了。随着一声巨响,你会感到突然一减速;引导伞一开,使劲一提,这个劲很大,会把人吓一跳;减速伞一开,又往那边一拽;主伞开时又把你拉到另一边。每次力量都相当大。飞船晃荡得很厉害,让人不知道是怎么回事。
    
    我们航天员是很重视这段过程的:伞开得好等于安全有保障,至少不会丢了性命。所以我被七七八八地拽了一通,平稳下来后心里却真是踏实——数据出来了,速度控制在规定范围内。我知道,这伞肯定是开好了!
    
    离地面5公里的时候,飞船抛掉防热大底,露出缓冲发动机。同时主伞也变成双点吊挂,让飞船摆正姿态,在风中晃悠着落向地面。
    
    飞船离地面1.2米时,缓冲发动机点火。接着,飞船“嗵”的一下落地了。
    
    我感觉落地很重,飞船弹了起来。在它第二次落地时,我迅速按下了切伞开关\footnote{〔切伞〕将飞船与降落伞分离。}。飞船停住了。此时是2003年10月16日6时23分。而这一时刻,正好是天安门当天升国旗的时刻,这是一个无法设计的巧合。
    
    那一刻四周寂静无声。舷窗黑乎乎的,看不到外面。
    
    过了几分钟,我隐约听到了叫喊声,手电的光从烧黑的舷窗上隐约照进来。他们找到飞船了!我听到外面插上钥匙的声音,舱门动弹了……
    
\end{normalsize}


\newpage

\textbf{注释}:

\vspace{-1em}

\begin{itemize}
    \setlength\itemsep{-0.2em}
    \item 〔款款〕慢慢地,从容地。
    \item 〔轮廓〕描述人和事物外形的线条。
    \item 〔广袤〕广阔。
\end{itemize}

\chapter{从百草园到三味书屋}

\begin{normalsize}
    
    我家的后面有一个很大的园,相传叫作百草园。现在是早已并屋子一起卖给朱文公的子孙了\footnote{〔朱文公〕南宋学者朱熹死后的谥号。这里指把屋子卖给一个姓朱的人。},连那最末次的相见也已经隔了七八年,其中似乎确凿只有一些野草;但那时却是我的乐园。
    
    不必说碧绿的菜畦,光滑的石井栏,高大的皂荚树\footnote{〔皂荚树〕一种乔木,果实像扁豆,长约20厘米,捣碎了泡水,可以洗衣服。},紫红的桑椹\footnote{〔桑椹〕桑树的果实,又叫桑葚。};也不必说鸣蝉在树叶里长吟,肥胖的黄蜂伏在菜花上,轻捷的叫天子\footnote{〔叫天子〕一种小鸟,又叫云雀。体长约20厘米,叫声响亮。}忽然从草间直窜向云霄里去了。单是周围的短短的泥墙根一带,就有无限趣味。油蛉\footnote{〔油蛉〕一种昆虫,俗名金钟儿,形似西瓜子,黑色,昼夜都鸣。}在这里低唱,蟋蟀们在这里弹琴。翻开断砖来,有时会遇见蜈蚣;还有斑蝥\footnote{〔斑蝥〕一种昆虫,能飞,翅上有黄黑色斑纹。这里是指类似斑蝥的“行夜虫”,俗称“放屁虫”。},倘若用手指按住它的脊梁,便会拍的一声,从后窍\footnote{〔后窍〕这里指昆虫的肛门。}喷出一阵烟雾。何首乌\footnote{〔何首乌〕一种多年生蔓草,根粗大,可入药。}藤和木莲\footnote{〔木莲〕一种蔓生的常绿灌木。}藤缠络着,木莲有莲房\footnote{〔莲房〕莲蓬。}一般的果实,何首乌有臃肿的根。有人说,何首乌根是有像人形的,吃了便可以成仙,我于是常常拔它起来,牵连不断地拔起来,也曾因此弄坏了泥墙,却从来没有见过有一块根像人样。如果不怕刺,还可以摘到覆盆子\footnote{〔覆盆子〕一种多年生草,茎长,有刺,夏天结果实。},象小珊瑚珠攒成的小球,又酸又甜,色味都比桑椹要好得远。
    
    长的草里是不去的,因为相传这园里有一条很大的赤练蛇\footnote{〔赤练蛇〕一种无毒蛇。体长一米左右,有黑红相间的斑纹。}。
    
    长妈妈\footnote{〔长妈妈〕鲁迅小时候家里的女工,下文的阿长也指她。}曾经讲给我一个故事听:先前,有一个读书人住在古庙里用功,晚间,在院子里纳凉的时候,突然听到有人在叫他。答应着,四面看时,却见一个美女的脸露在墙头上,向他一笑,隐去了。他很高兴;但竟给那走来夜谈的老和尚识破了机关\footnote{〔机关〕这里指周密而巧妙的计谋。}。说他脸上有些妖气,一定遇见“美女蛇”了;这是人首蛇身的怪物,能唤人名,倘一答应,夜间便要来吃这人的肉的。他自然吓得要死,而那老和尚却道无妨,给他一个小盒子,说只要放在枕边,便可高枕而卧。他虽然照样办,却总是睡不着,——当然睡不着的。到半夜,果然来了,沙沙沙!门外像是风雨声。他正抖作一团时,却听得豁的一声,一道金光从枕边飞出,外面便什么声音也没有了,那金光也就飞回来,敛在盒子里。后来呢?后来,老和尚说,这是飞蜈蚣,它能吸蛇的脑髓,美女蛇就被它治死了。
    
    结末的教训是:所以倘有陌生的声音叫你的名字,你万不可答应他。
    
    这故事很使我觉得做人之险,夏夜乘凉,往往有些担心,不敢去看墙上,而且极想得到一盒老和尚那样的飞蜈蚣。走到百草园的草丛旁边时,也常常这样想。但直到现在,总还没有得到,但也没有遇见过赤练蛇和美女蛇。叫我名字的陌生声音自然是常有的,然而都不是美女蛇。
    
    冬天的百草园比较的无味;雪一下,可就两样了。拍雪人和塑雪罗汉需要人们鉴赏,这是荒园,人迹罕至,所以不相宜,只好来捕鸟。薄薄的雪,是不行的;总须积雪盖了地面一两天,鸟雀们久已无处觅食的时候才好。扫开一块雪,露出地面,用一支短棒支起一面大的竹筛来,下面撒些秕谷,棒上系一条长绳,人远远地牵着,看鸟雀下来啄食,走到竹筛底下的时候,将绳子一拉,便罩住了。但所得的是麻雀居多,也有白颊的“张飞鸟”\footnote{〔张飞鸟〕鹡鸰。头部像戏台上张飞的脸谱,所以浙东也有叫张飞鸟。},性子很躁,养不过夜的。
    
    这是闰土的父亲\footnote{〔闰土〕作者在小说《故乡》中写到的儿时朋友。}所传授的方法,我却不大能用。明明见它们进去了,拉了绳,跑去一看,却什么都没有,费了半天力,捉住的不过三四只。闰土的父亲是小半天便能捕获几十只,装在叉袋\footnote{〔叉袋〕一种装粮食的布袋或者麻袋,袋口有叉角,可以打结。}里叫着撞着的。我曾经问他得失的缘由,他只静静地笑道:你太性急,来不及等它走到中间去。
    
    我不知道为什么家里的人要将我送进书塾里去了,而且还是全城中称为最严厉的书塾。也许是因为拔何首乌毁了泥墙罢,也许是因为将砖头抛到间壁的梁家去了罢,也许是因为站在石井栏上跳下来罢,……都无从知道。总而言之:我将不能常到百草园了。\texttt{Ade}\footnote{〔\texttt{Ade}〕德语,再见的意思。},我的蟋蟀们!\texttt{Ade},我的覆盆子们和木莲们!……
    
    出门向东,不上半里,走过一道石桥,便是我的先生的家了。从一扇黑油的竹门进去,第三间是书房。中间挂着一块匾额:三味书屋\footnote{〔三味书屋〕在绍兴城内作者故家附近。解放后辟为鲁迅纪念馆的一部分。};匾下面是一幅画,画着一只很肥大的梅花鹿伏在古树下。没有孔子牌位,我们便对着那匾和鹿行礼。第一次算是拜孔子,第二次算是拜先生。
    
    第二次行礼时,先生\footnote{〔先生〕作者的启蒙老师,姓寿,名怀鉴,字镜吾,是一个老秀才。}便和蔼地在一旁答礼。他是一个高而瘦的老人,须发都花白了,还戴着大眼镜。我对他很恭敬,因为我早听到,他是本城中极方正,质朴,博学的人。
    
    不知从哪里听来的,东方朔\footnote{〔东方朔〕西汉文学家,善辞赋,性格诙谐滑稽。关于他的民间传说很多。}也很渊博,他认识一种虫,名曰“怪哉”,冤气所化,用酒一浇,就消释了。我很想详细地知道这故事,但阿长是不知道的,因为她毕竟不渊博。现在得到机会了,可以问先生。
    
    “先生,‘怪哉”\footnote{〔怪哉〕据《太平广记》,汉武帝巡视时发现的人面怪虫。询问东方朔,东方朔回答说,过去秦朝拘押无辜的人太多,百姓纷纷感叹:“怪哉怪哉!”愤怒感动上天,产生这种虫。汉武帝问他有什么办法可以去掉,东方朔回答:喝酒可以消愁。用酒浇灌,可以让这虫子消失。}这虫,是怎么一回事?……”我上了生书\footnote{〔生书〕未读过的书,新课。},将要退下来的时候,赶忙问。
    
    “不知道!”他似乎很不高兴,脸上还有怒色了。
    
    我才知道做学生是不应该问这些事的,只要读书,因为他是渊博的宿儒\footnote{〔宿〕长久从事某种工作。儒:信奉孔孟之道的知识分子。},决不至于不知道,所谓不知道者,乃是不愿意说。年纪比我大的人,往往如此,我遇见过好几回了。
    
    我就只读书,正午习字,晚上对课\footnote{〔对课〕即对对联,旧时学习词句、准备作诗的一种练习。一般老师出上联,学生对下联。三言、五言,即三字一句、五字一句。字数越多越难。}。先生最初这几天对我很严厉,后来却好起来了,不过给我读的书渐渐加多,对课也渐渐地加上字去,从三言到五言,终于到七言。
    
    三味书屋后面也有一个园,虽然小,但在那里也可以爬上花坛去折腊梅花,在地上或桂花树上寻蝉蜕\footnote{〔蝉蜕〕蝉的幼虫变为蝉时脱去的外壳,可入药。}。最好的工作是捉了苍蝇喂蚂蚁,静悄悄地没有声音。然而同窗们到园里的太多,太久,可就不行了,先生在书房里便大叫起来:
    
    “人都到哪里去了?”
    
    人们便一个一个陆续走回去;一同回去,也不行的。他有一条戒尺,但是不常用,也有罚跪的规矩,但也不常用,普通总不过瞪几眼,大声道:
    
    “读书!”
    
    于是大家放开喉咙读一阵书,真是人声鼎沸。有念“仁远乎哉我欲仁斯仁至矣\footnote{见《论语·述而》,应读为“仁远乎哉?我欲仁,斯仁至矣!”}”的,有念“笑人齿缺曰狗窦大开”\footnote{见《幼学琼林·身体》,原句是:“笑人缺齿,狗窦胡为大开?”}的,有念“上九潜龙勿用”\footnote{见《周易》,原句是:“初九,潜龙勿用。”}的,有念“厥土下上上错厥贡苞茅橘柚”\footnote{见《尚书·禹贡》,原句是:“厥土惟涂泥。厥田惟下下,厥赋下上,上错。……厥包橘柚锡贡。”}的……先生自己也念书。后来,我们的声音便低下去,静下去了,只有他还大声朗读着:
    
    “铁如意,指挥倜傥,一座皆惊呢;金叵罗\footnote{〔叵罗〕古代饮酒用的一种敞口的浅杯。},颠倒淋漓噫,千杯未醉嗬……”
    
    我疑心这是极好的文章,因为读到这里,他总是微笑起来,而且将头仰起,摇着,向后面拗过去,拗过去。
    
    先生读书入神的时候,于我们是很相宜的。有几个便用纸糊的盔甲套在指甲上做戏。
    
    我是画画儿,用一种叫作“荆川纸”\footnote{〔荆川纸〕一种竹纸,薄而略透明。}的,蒙在小说的绣像\footnote{〔绣像〕明清以来,通俗小说前面往往附有书中人物的图像,称为绣像。}上一个个描下来,象习字时候的影写一样。读的书多起来,画的画也多起来;书没有读成,画的成绩却不少了,最成片断\footnote{〔片断〕片段。}的是《荡寇志》\footnote{〔《荡寇志》〕清朝俞万春著的一部反《水浒传》、诬蔑歪曲梁山起义的小说。}和《西游记》的绣像,都有一大本。后来,因为要钱用,卖给一个有钱的同窗了。他的父亲是开锡箔\footnote{〔锡箔〕这里指锡箔纸,附着一层薄锡的纸,旧时多用于祭奠死去的人。}店的;听说现在自己已经做了店主,而且快要升到绅士\footnote{〔绅士〕地方上有影响力、有威望的人,辅助官府统治、维持秩序。}的地位了。这东西早已没有了罢。
    
    \hfill 九月十八日
    
\end{normalsize}


\newpage

\textbf{注释}:

\vspace{-1em}

\begin{itemize}
    \setlength\itemsep{-0.2em}
    \item 〔菜畦〕种植蔬菜的一排排整齐的小块田地。四周围着土埂,便于管理和浇灌。
    \item 〔臃肿〕胖大。
    \item 〔秕谷〕不饱满的谷粒。
    \item 〔敛〕收拢。
    \item 〔脑髓〕脑浆。
    \item 〔无从〕没有方法、门路(做某事)。
    \item 〔同窗〕在同一窗下读书的人,指同学。
    \item 〔人迹罕至〕少有人来。迹:足迹。罕:稀少。
    \item 〔戒尺〕老师用来责罚学生的长条形木板。
    \item 〔书塾〕旧时家庭、宗族或教师自己设立的教学场所。
    \item 〔倜傥〕洒脱不受拘束的样子。
    \item 〔淋漓〕濡湿流淌的样子。也形容酣畅、痛快。
    \item 〔拗〕弯屈,弯转。
    \item 〔盔甲〕古代战士的护身装备。头戴的称为“盔”,身穿的称为“甲”。
    \item 〔影写〕把纸蒙在字帖上照着描。
    \item 〔人声鼎沸〕许多人吵闹,声音像大锅里沸腾的水。
    \item 〔渊博〕精深而广博。
\end{itemize}

\chapter{国王的新衣}

\begin{normalsize}
    
    很久以前有一位国王,他非常喜欢穿新衣服。为了穿得漂亮,他把所有的钱都花到衣服上去了。他一点也不关心他的军队,不愿意坐马车出游,甚至不爱去戏院看戏,除非是为了炫耀一下新衣服。他要求每天每个钟点都准备好新衣服。人们一提到国王,总是说:“王上\footnote{〔王上〕古代对国王的尊称。}在更衣室里。”
    
    由于国王喜爱新衣服,很多裁缝、织工、鞋匠都到王宫来,希望为国王做衣服。一天,两个年轻人来到王宫,自称是来自外国的织工。他们说,他们能织出世上最美丽的布。这种布不仅美丽无比,还有一个奇妙的作用:凡是愚蠢的人,都看不见用这布做成的衣服。
    
    “这可不是最适合我的衣服嘛!”国王心想,“我穿了这样的衣服,就可以看出我的王国里哪些人不称职;我就可以辨别出哪些人是聪明人,哪些人是傻子。没错,我要叫他们马上织出这样的布来!”他付了一大笔钱给这两个人,赐给他们带庭院的住宅,叫他们马上开始工作。
    
    外国的织工摆出两架织机来,装做在工作的样子,可是他们的织机上什么东西也没有。他们接二连三地请求国王赏赐最好的蚕丝和金线给他们。他们把这些好东西都装进自己的腰包,却假装在那两架空空的织机上忙碌地工作,一直忙到深夜。
    
    国王等了好几天。他太想知道究竟织得怎样了。不过,他立刻就想起来,愚蠢的人是看不见这布的,这让他心里不太舒服。他相信他自己是用不着害怕的。虽然如此,他还是觉得,先派一个人去看看比较妥当。
    
    “先派老宫相\footnote{〔宫相〕欧洲古代掌管宫廷政务,辅佐国王的大臣。}去看看,”国王想,“他这个人很有头脑,而且不会说谎。”
    
    因此这位正直的老宫相就到那两个外国织工家去。两个骗子正在空空的织机上忙碌地工作着。
    
    “这是怎么一回事儿?”老宫相心想。他把眼睛睁得大大的。
    
    “我什么东西也没有看见!”但是他不敢把这句话说出来。
    
    两个骗子请他走近一点,指着那两架空空的织机问他,布的纹理是不是很精细,色彩是不是很漂亮。
    
    可怜的老宫相的眼睛越睁越大,可还是看不见什么东西,因为的确没有什么东西可看。
    
    “我的老天爷!”他想,“难道我是一个愚蠢的人吗?我从来没有怀疑过我自己。我决不能让人知道这件事——不成,我决不能让人知道我看不见布料。”
    
    “哎,您一点意见也没有吗?”一个正在织布的织工说。
    
    “啊,美极了!真是美妙极了!”老宫相说。他戴着眼镜仔细地看。“多么精细的纹理!多么美的色彩!是的,我会呈报王上,说我对于这布非常满意。”
    
    “听到您的话,我们就放心了。”两个织工一起说。他们把这些丰富的色彩和纹理仔细描述了一番。这位老大臣注意地听着,以便回到国王那里去时,可以照样背得出来。事实上他也这样办了。
    
    两个骗子又要了很多的钱,更多的蚕丝和金线,他们说这是为了织布的需要。他们把这些东西全装进腰包里,连一根线也没有放到织机上去。不过他们还是继续在空空的机架上工作。
    
    过了不久,国王又派了另一位正直可信的大臣,去看布是不是很快就可以织好。这位大臣也遇到了同样的事:他看了又看,但是那两架空空的织机上什么也没有,他什么东西也看不出来。
    
    “您看这段布美不美?”两个骗子问。他们指出各种漂亮的花纹,仔细解释。事实上并没有什么花纹。
    
    “我并不愚蠢!”这位大臣想,“难道我是个蠢货吗?这也真够滑稽,但是我决不能让人看出来!”因此他就把他完全没有看见的布称赞了一番,同时对他们说,他非常喜欢这些美丽的色彩和精致的花纹。“是的,那真是太美了。”他回去对国王说。
    
    城里所有的人都在谈论这神奇的布料。
    
    国王很想亲自去看一次。他特别指定了一批随员\footnote{〔随员〕随同出行的人。}——包括已经去看过的那两位正直的大臣。两个狡猾的骗子正卖力地织布,但是一根线的影子也看不见。“各位,这不漂亮吗?”那两位正直的大臣说,“陛下\footnote{〔陛下〕古代对国王的尊称。}请看,多么精致的花纹!多么美丽的色彩!”他们指着空空的织机,因为他们以为别人一定看得见。
    
    “这是怎么一回事儿呢?”国王心想,“我什么也没有看见!这真是荒唐!难道我是一个愚蠢的人吗?难道我不配做国王吗?这太可怕了。我从没有遇过这样的事情。”
    
    “啊,它真是美极了!”国王说,“我十分满意!”
    
    于是他点头表示满意。他装作很仔细地看着织机的样子,因为他不愿意说出他什么也没有看见。全体随员也仔细地看了又看,可是他们也没有看出任何东西。不过,他们也照着国王的话说:“啊,真是美极了!”他们提出,这种新奇的布料,正适合即将举行的游行庆典\footnote{〔游行庆典〕在大街上行进的庆祝活动。}。国王穿着用它做的新衣服游行,再好不过了。
    
    “真美丽!真精致!真是好极了!”每人都随声附和着,每个人都显得开心极了。国王赐给骗子每人一个爵士的头衔,一枚可以挂在扣眼\footnote{〔扣眼〕上衣胸前用来别纽扣的孔,也可以用来佩挂饰物。}上的勋章。
    
    第二天早上就是游行庆典了。这两个骗子整夜不睡,点起十六支蜡烛。你可以看到他们在连夜赶工,要完成国王的新衣。他们装做把布料从织机上取下来。他们用两把大剪刀在空中裁了一阵子,同时又用没有穿线的针缝了一通。最后,他们齐声说:“请看!新衣服做好了!”
    
    国王带着一群贵族,亲自到来了。两个骗子每人举起一只手,好像他们拿着什么东西。“请看吧,这是裤子,这是上衣!这是披风!”他们指出每一件衣服的名称。“这衣服轻柔得像蜘蛛网一样:穿着它的人会觉得身上什么也没有似的——这也正是这衣服的妙处。”
    
    “一点也不错。”所有的贵族都赞同。可是他们什么也没有看见,因为实际上什么东西也没有。
    
    “现在请陛下脱下衣服,”两个骗子说,“我们要在这个大镜子前为陛下换上新衣。”
    
    国王把身上的衣服统统脱了。这两个骗子装作把刚才缝好的新衣服一件一件给他穿上。他们在他的腰上弄了一阵子,仿佛在系上什么东西:这就是后裙摆\footnote{〔后裙摆〕欧洲古代贵族的一种装束,是拖在身后的很长的一块布。}。国王在镜子面前转了转身子,扭了扭腰肢。
    
    “天啊,这衣服多么合身啊!这剪裁、这式样,多么好看啊!”大家都说,“多么精致的花纹!多么美妙的色彩!这真是一套前所未见、令人惊叹的衣服!”
    
    “华盖\footnote{〔华盖〕国王出行时遮阳遮雨的大伞。}已经准备好了,只等陛下穿好新衣服,就可以开始游行了!”典礼官说。
    
    “对,我已经穿好了。”国王说,“这衣服合我的身么?”于是他又在镜子面前扭动身子,因为他要叫大家看出,他在认真地欣赏他美丽的新衣服。侍从们都把手在地上东摸西摸,仿佛真的在拾起裙摆。他们抬起手来,手中托着空气——他们不敢让人瞧出,他们其实什么也没有看见。
    
    这么着,国王就在华盖下游行起来了。街上看见的人、街旁窗户里望见的人都说:“王上的新衣服真是漂亮!长长的后裙摆多么美丽!衣服多么合身!”谁也不愿意让人知道自己看不见,因为这样就会暴露自己是个愚蠢的家伙。国王所有的衣服从来没有得到过这样一致的称赞。
    
    “可是他什么衣服也没有穿呀!”突然,一个小孩子叫出声来。
    
    “天呐,你听这个天真的声音!”爸爸说。于是大家把这孩子讲的话低声地传开来。
    
    “他并没有穿什么衣服!有一个小孩子说他并没有穿什么衣服呀!”
    
    “他实在是没有穿什么衣服呀!”最终,所有的老百姓都这么说了。这话终于传到了国王耳中。国王有点儿发抖,因为他似乎觉得老百姓讲的是对的。“但是,我决不能让人看出来,否则这庆典就无法收场了。”因此他摆出一副更高傲的神气。他的侍从们跟在他后面,手中托着并不存在的裙摆。
    
\end{normalsize}



\chapter{海燕}

\begin{normalsize}
    
    在苍茫的大海上,狂风卷集着乌云。在乌云和大海之间,海燕像黑色的闪电,在高傲地飞翔。
    
    一会儿翅膀碰着波浪,一会儿箭一般地直冲向乌云,它叫喊着。就在这鸟儿勇敢的叫喊声里,乌云听出了欢乐。
    
    这叫喊声里,充满着对暴风雨的渴望!在这叫喊声里,乌云听出了愤怒的力量、热情的火焰和胜利的信心。
    
    海鸥在暴风雨来临之前呻吟着,它们在大海上飞窜,想把自己对暴风雨的恐惧,掩藏到大海深处。
    
    海鸭也在呻吟着。它们这些海鸭啊,享受不了生活的战斗的欢乐:轰隆隆的雷声就把它们吓坏了。
    
    蠢笨的企鹅,胆怯地把肥胖的身体躲藏到悬崖底下……只有那高傲的海燕,勇敢地,自由自在地,在泛起白沫的大海上飞翔!
    
    乌云越来越暗,越来越低,向海面直压下来,而波浪一边歌唱,一边冲向高空,去迎接那雷声。
    
    雷声轰响。波浪在愤怒的飞沫中呼叫,跟狂风争鸣。看吧,狂风紧紧抱起一层层巨浪,恶狠狠地把它们甩到悬崖上,把这些大块的翡翠摔成尘雾和碎末。
    
    海燕叫喊着,飞翔着,像黑色的闪电,箭一般地穿过乌云,翅膀掠起波浪的飞沫。
    
    看吧,它飞舞着,像个精灵——高傲的、黑色的暴风雨的精灵。它在大笑,它又在号叫……它笑那些乌云,它因为欢乐而号叫!
    
    这个敏感的精灵,它从雷声的震怒里,早就听出了困乏,它深信,乌云遮不住太阳——是的,遮不住的!
    
    狂风吼叫……雷声轰响……
    
    一堆堆乌云,像青色的火焰,在无底的大海上燃烧。大海抓住闪电的箭光,把它们熄灭在自己的深渊里。这些闪电的影子,活像一条条火蛇,在大海里蜿蜒游动,一晃就消失了。
    
    ——暴风雨!暴风雨就要来啦!
    
    这是勇敢的海燕,在怒吼的大海上,在闪电中间,高傲地飞翔;这是胜利的预言家在叫喊:
    
    ——让暴风雨来得更猛烈些吧!
    
\end{normalsize}


\newpage

\textbf{注释}:

\vspace{-1em}

\begin{itemize}
    \setlength\itemsep{-0.2em}
    \item 〔深渊〕非常深的水。
    \item 〔呻吟〕因痛苦而发出声音。
    \item 〔胆怯〕胆小害怕。
    \item 〔蜿蜒〕蛇屈折爬行的样子。
\end{itemize}

\chapter{春}

\begin{normalsize}
    
    盼望着,盼望着,东风来了,春天的脚步近了。
    
    一切都像刚睡醒的样子,欣欣然张开了眼。山朗润起来了,水涨起来了,太阳的脸红起来了。
    
    小草偷偷地从土里钻出来,嫩嫩的,绿绿的。园子里,田野里,瞧去,一大片一大片满是的。坐着,躺着,打两个滚,踢几脚球,赛几趟跑,捉几回迷藏。风轻悄悄的,草绵软软的。
    
    桃树、杏树、梨树,你不让我,我不让你,都开满了花赶趟儿。红的像火,粉的像霞,白的像雪。花里带着甜味儿;闭了眼,树上仿佛已经满是桃儿、杏儿、梨儿。花下成千成百的蜜蜂嗡嗡地闹着,大小的蝴蝶飞来飞去。野花遍地是:杂样儿,有名字的,没名字的,散在草丛里,像眼睛,像星星,还眨呀眨的。
    
    “吹面不寒杨柳风”,不错的,像母亲的手抚摸着你。风里带来些新翻的泥土的气息,混着青草味儿,还有各种花的香,都在微微润湿的空气里酝酿。鸟儿将窠巢安在繁花嫩叶当中,高兴起来了,呼朋引伴地卖弄清脆的喉咙,唱出宛转的曲子,与轻风流水应和着。牛背上牧童的短笛,这时候也成天在嘹亮地响。
    
    雨是最寻常的,一下就是三两天。可别恼。看,像牛毛,像花针,像细丝,密密地斜织着,人家屋顶上全笼着一层薄烟。树叶子却绿得发亮,小草也青得逼你的眼。傍晚时候,上灯了,一点点黄晕的光,烘托出一片安静而和平的夜。乡下去,小路上,石桥边,有撑起伞慢慢走着的人;还有地里工作的农夫,披着蓑,戴着笠的。他们的草屋,稀稀疏疏的,在雨里静默着。
    
    天上风筝渐渐多了,地上孩子也多了。城里乡下,家家户户,老老小小,他们也赶趟儿似的,一个个都出来了。舒活舒活筋骨,抖擞抖擞精神,各做各的一份事去。“一年之计在于春”,刚起头儿,有的是工夫,有的是希望。
    
    春天像刚落地\footnote{〔落地〕这里指婴儿出生。}的娃娃,从头到脚都是新的,他生长着。
    
    春天像小姑娘,花枝招展的,笑着,走着。
    
    春天像健壮的青年,有铁一般的胳膊和腰脚,他领着我们上前去。
    
\end{normalsize}


\newpage

\textbf{注释}:

\vspace{-1em}

\begin{itemize}
    \setlength\itemsep{-0.2em}
    \item 〔朗润〕明亮滋润。朗:明亮。润:滋润、润泽。
    \item 〔赶趟儿〕时间赶得上。这里指众多果树争先恐后地开花。
    \item 〔酝酿〕造酒的发酵过程。这里指各种气息在空气里,像发酵似的,越来越浓。
    \item 〔窠巢〕鸟兽昆虫的窝。
    \item 〔宛转〕形容声音抑扬动听。现在多写作“婉转”。
    \item 〔花针〕绣花用的细针。
    \item 〔黄晕〕昏黄,不明亮。
    \item 〔笠〕用竹篾或棕皮编制的遮阳挡雨的帽子。
    \item 〔花枝招展〕形容女子打扮得十分艳丽。这里比喻姿态优美。
    \item 〔抖擞〕振作(精神)。
\end{itemize}

\chapter{纪念白求恩}

\begin{normalsize}
    
    白求恩同志是加拿大共产党员,五十多岁了,为了帮助中国的抗日战争,受加拿大共产党和美国共产党的派遣,不远万里,来到中国。去年春上到延安,后来到五台山\footnote{〔五台山〕陕西省境内的山。1938年6月,白求恩到五台山建立前线医院。}工作,不幸以身殉职。一个外国人,毫无利己的动机,把中国人民的解放事业当作他自己的事业,这是什么精神?这是国际主义的精神,这是共产主义的精神,每一个中国共产党员都要学习这种精神。列宁主义认为:资本主义国家的无产阶级要拥护殖民地半殖民地人民的解放斗争,殖民地半殖民地的无产阶级要拥护资本主义国家的无产阶级的解放斗争,世界革命才能胜利。白求恩同志是实践了这一条列宁主义路线的。我们中国共产党员也要实践这一条路线。我们要和一切资本主义国家的无产阶级联合起来,要和日本的、英国的、美国的、德国的、意大利的以及一切资本主义国家的无产阶级联合起来,才能打倒帝国主义,解放我们的民族和人民,解放世界的民族和人民。这就是我们的国际主义,这就是我们用以反对狭隘民族主义和狭隘爱国主义的国际主义。
    
    白求恩同志毫不利己专门利人的精神,表现在他对工作的极端的负责任,对同志对人民的极端的热忱。每个共产党员都要学习他。不少的人对工作不负责任,拈轻怕重,把重担子推给人家,自己挑轻的。一事当前,先替自己打算,然后再替别人打算。出了一点力就觉得了不起,喜欢自吹,生怕人家不知道。对同志对人民不是满腔热忱,而是冷冷清清,漠不关心,麻木不仁。这种人其实不是共产党员,至少不能算一个纯粹的共产党员。从前线回来的人说到白求恩,没有一个不佩服,没有一个不为他的精神所感动。晋察冀边区的军民,凡亲身受过白求恩医生的治疗和亲眼看过白求恩医生的工作的,无不为之感动。每一个共产党员,一定要学习白求恩同志的这种真正共产主义者的精神。
    
    白求恩同志是个医生,他以医疗为职业,对技术精益求精;在整个八路军医务系统中,他的医术是很高明的。这对于一班见异思迁的人,对于一班鄙薄技术工作以为不足道、以为无出路的人,也是一个极好的教训。
    
    我和白求恩同志只见过一面。后来他给我来过许多信。可是因为忙,仅回过他一封信,还不知他收到没有。对于他的死,我是很悲痛的。现在大家纪念他,可见他的精神感人之深。我们大家要学习他毫无自私自利之心的精神。从这点出发,就可以变为大有利于人民的人。一个人能力有大小,但只要有这点精神,就是一个高尚的人,一个纯粹的人,一个有道德的人,一个脱离了低级趣味的人,一个有益于人民的人。
    
\end{normalsize}


\newpage

\textbf{注释}:

\vspace{-1em}

\begin{itemize}
    \setlength\itemsep{-0.2em}
    \item 〔殉职〕因本职工作死亡。
    \item 〔实践〕实际去做。践:踩,踏。
    \item 〔热忱〕真诚的热情。忱:真挚的情意。
    \item 〔狭隘〕狭小。狭义。气量小。
    \item 〔派遣〕正式命令、委任下级去干某事,通常到别的地方。
    \item 〔精益求精〕已经很好了,还要求更好。
    \item 〔见异思迁〕看到别的地方就想搬迁过去。看到别的目标就改变主意。
    \item 〔鄙薄〕看不起,认为没有价值。
\end{itemize}

\chapter{猫}

\begin{normalsize}
    
    我家养了好几次猫,结局总是失踪或死亡。三妹是最喜欢猫的,她常在课后回家时,逗着猫玩。有一次,从隔壁要了一只新生的猫来。花白的毛,很活泼,如带着泥土的白雪球似的,常在廊前太阳光里滚来滚去。三妹常常取了一条红带,或一根绳子,在它面前来回的拖摇着,它便扑过来抢,又扑过去抢。我坐在藤椅上看着他们,可以微笑着消耗过一二小时的光阴,那时太阳光暖暖的照着,心上感着生命的新鲜与快乐。后来这只猫不知怎地忽然消瘦了,也不肯吃东西,光泽的毛也污涩了,终日躺在厅上的椅下,不肯出来。三妹想着种种方法逗它,它都不理会。我们都很替它忧郁。三妹特地买了一个很小很小的铜铃,用红绫带穿了,挂在它颈下,但只显得不相称,它只是毫无生意的,懒惰的,郁闷的躺着。有一天中午,我从编译所\footnote{〔编译所〕1921年4月,在茅盾介绍下,郑振铎进入商务印书馆编译所工作。}回来,三妹很难过的说道:“哥哥,小猫死了!”
    
    我心里也感着一缕的酸辛,可怜这两月来相伴的小侣!当时只得安慰着三妹道:“不要紧,我再向别处要一只来给你。”
    
    隔了几天,二妹从虹口\footnote{〔虹口〕上海市辖区。在黄浦江西北岸。}舅舅家里回来,她道,舅舅那里有三四只小猫,很有趣,正要送给人家。三妹便怂恿着她去拿一只来。礼拜天,母亲回来了,却带了一只浑身黄色的小猫同来。立刻三妹一部分的注意,又被这只黄色小猫吸引去了。这只小猫较第一只更有趣、更活泼。它在园中乱跑,又会爬树,有时蝴蝶安详地飞过时,它也会扑过去捉。它似乎太活泼了,一点也不怕生人,有时由树上跃到墙上,又跑到街上,在那里晒太阳。我们都很为它提心吊胆,一天都要“小猫呢?小猫呢?”查问得好几次。每次总要寻找了一回,方才寻到。三妹常指它笑着骂道:“你这小猫呀,要被乞丐捉去后才不会乱跑呢!”我回家吃中饭,总看见它坐在铁门外边,一见我进门,便飞也似地跑进去了。饭后的娱乐,是看它在爬树。隐身在阳光隐约里的绿叶中,好像在等待着要捉捕什么似的。把它抱了下来。一放手,又极快地爬上去了。过了二三个月,它会捉鼠了。有一次,居然捉到一只很肥大的鼠,自此,夜间便不再听见讨厌的吱吱的声了。
    
    某一日清晨,我起床来,披了衣下楼,没有看见小猫,在小园里找了一遍,也不见。心里便有些亡失的预警。
    
    “三妹,小猫呢?”
    
    她慌忙地跑下楼来,答道:“我刚才也寻了一遍,没有看见。”
    
    家里的人都忙乱的在寻找,但终于不见。
    
    李嫂道;“我一早起来开门,还见它在厅上。烧饭时,才不见了它。”
    
    大家都不高兴,好像亡失了一个亲爱的同伴,连向来不大喜欢它的张婶也说;“可惜,可惜,这样好的一只小猫。”
    
    我心里还有一线希望,以为它偶然跑到远处去,也许会认得归途的。
    
    午饭时,张婶诉说道:“刚才遇到隔壁周家的丫头,她说,早上看见我家的小猫在门外,被一个过路的人捉去了。”
    
    于是这个亡失证实了。三妹很不高兴的咕噜着道:“他们看见了,为什么不出来阻止?他们明晓得它是我家的!”
    
    我也怅然的,愤恨的,在诅骂着那个不知名的夺去我们所爱的东西的人。
    
    自此,我家好久不养猫。
    
    冬天的早晨,门口蜷伏着一只很可怜的小猫。毛色是花白,但并不好看,又很瘦。它伏着不去。我们如不取来留养,至少也要为冬寒与饥饿所杀。张婶把它拾了进来,每天给它饭吃。但大家都不大喜欢它,它不活泼,也不像别的小猫之喜欢顽游,好像是具着天生的忧郁性似的,连三妹那样爱猫的,对于它也不加注意。如此的,过了几个月,它在我家仍是一只若有若无的动物。它渐渐的肥胖了,但仍不活泼。大家在廊前晒太阳闲谈着时,它也常来蜷伏在母亲或三妹的足下。三妹有时也逗着它玩,但没有对于前几只小猫那样感兴趣。有一天,它因夜里冷,钻到火炉底下去,毛被烧脱好几块,更觉得难看了。
    
    春天来了,它成了一只壮猫了,却仍不改它的忧郁性,也不去捉鼠,终日懒惰的伏着,吃得胖胖的。
    
    这时,妻买了一对黄色的芙蓉鸟来,挂在廊前,叫得很好听。妻常常叮嘱着张婶换水,加鸟粮,洗刷笼子。那只花白猫对于这一对黄鸟,似乎也特别注意,常常跳在桌上,对鸟笼凝望着。
    
    妻道:“张婶,留心猫,它会吃鸟呢。”
    
    张婶便跑来把猫捉了去。隔一会,它又跳上桌子对鸟笼凝望着了。
    
    一天,我下楼时,听见张婶在叫道:“鸟死了一只,一条腿被咬去了,笼扳上都是血。是什么东西把它咬死的?”
    
    我匆匆跑下去看,果然一只鸟是死了,羽毛松散着,好像它曾与它的敌人挣扎了许久。
    
    我很愤怒,叫道:“一定是猫,一定是猫!”于是立刻便去找它。
    
    妻听见了,也匆匆地跑下来,看了死鸟,很难过,便道:“不是这猫咬死的还有谁?它常常对鸟笼望着,我早就叫张婶要小心了。张婶!你为什么不小心?”
    
    张婶默默无言,不能有什么话来辩护。
    
    于是猫的罪状证实了。大家都去找这可厌的猫,想给它以一顿惩戒。找了半天,却没找到。我以为它真是“畏罪潜逃”了。
    
    三妹在楼上叫道:“猫在这里了。”
    
    它躺在露台板上晒太阳,态度很安详,嘴里好像还在吃着什么。我想,它一定是在吃着这可怜的鸟的腿了,一时怒气冲天,拿起楼门旁倚着的一根木棒,追过去打了一下。它很悲楚地叫了一声“咪呜!”便逃到屋瓦上了。
    
    我心里还愤愤的,以为惩戒得还没有快意。
    
    隔了几天,李嫂在楼下叫道:“猫,猫!又来吃鸟了。”同时我看见一只黑猫飞快的逃过露台,嘴里衔着一只黄鸟。我开始觉得我是错了!
    
    我心里十分的难过,真的,我的良心受伤了,我没有判断明白,便妄下断语,冤苦了一只不能说话辩诉的动物。想到它的无抵抗的逃避,益使我感到我的暴怒,我的虐待,都是针,刺我的良心的针!
    
    我很想补救我的过失,但它是不能说话的,我将怎样的对它表白我的误解呢?
    
    两个月后,我们的猫忽然死在邻家的屋脊上。我对于它的亡失,比以前的两只猫的亡失,更难过得多。
    
    我永无改正我的过失的机会了!
    
    自此,我家永不养猫。
    
\end{normalsize}


\newpage

\textbf{注释}:

\vspace{-1em}

\begin{itemize}
    \setlength\itemsep{-0.2em}
    \item 〔涩〕不顺滑。
    \item 〔怂恿〕从旁劝说,使想做。
    \item 〔提心吊胆〕不放心,心里不安。
    \item 〔乞丐〕靠要饭要钱过活的人。
    \item 〔倚〕斜靠。
    \item 〔虐待〕用狠毒残忍的手段对待人。
\end{itemize}

\chapter{阿长与山海经}

\begin{normalsize}
    
    长妈妈,已经说过,是一个一向带领着我的女工,说得阔气一点,就是我的保姆。我的母亲和许多别的人都这样称呼她,似乎略带些客气的意思。只有祖母叫她阿长。我平时叫她“阿妈”,连“长”字也不带;但到憎恶她的时候,——例如知道了谋死我那隐鼠\footnote{〔隐鼠〕鼹鼠的别称。}的却是她的时候,就叫她阿长。
    
    我们那里没有姓长的;她生得黄胖而矮,“长”也不是形容词。又不是她的名字,记得她自己说过,她的名字是叫作什么姑娘的。什么姑娘,我现在已经忘却了,总之不是长姑娘;也终于不知道她姓什么。记得她也曾告诉过我这个名称的来历:先前的先前,我家有一个女工,身材生得很高大,这就是真阿长。后来她回去了,我那什么姑娘才来补她的缺,然而大家因为叫惯了,没有再改口,于是她从此也就成为长妈妈了。
    
    虽然背地里说人长短不是好事情,但倘使要我说句真心话,我可只得说:我实在不大佩服她。最讨厌的是常喜欢切切察察,向人们低声絮说些什么事。还竖起第二个手指,在空中上下摇动,或者点着对手或自己的鼻尖。我的家里一有些小风波,不知怎的我总疑心和这“切切察察”有些关系。又不许我走动,拔一株草,翻一块石头,就说我顽皮,要告诉我的母亲去了。一到夏天,睡觉时她又伸开两脚两手,在床中间摆成一个“大”字,挤得我没有余地翻身,久睡在一角的席子上,又已经烤得那么热。推她呢,不动;叫她呢,也不闻。
    
    “长妈妈生得那么胖,一定很怕热罢?晚上的睡相,怕不见得很好罢?……”
    
    母亲听到我多回诉苦之后,曾经这样地问过她。我也知道这意思是要她多给我一些空席。她不开口。但到夜里,我热得醒来的时候,却仍然看见满床摆着一个“大”字,一条臂膊还搁在我的颈子上。我想,这实在是无法可想了。
    
    但是她懂得许多规矩;这些规矩,也大概是我所不耐烦的。一年中最高兴的时节,自然要数除夕了。辞岁\footnote{〔辞岁〕新年开始。旧年最后一夜叫做“除夕”,度过后迎来新年,称为“辞旧岁,迎新春”。}之后,从长辈得到压岁钱\footnote{〔压岁钱〕过年的习俗。长辈要给小辈压岁钱,保佑平安过年。},红纸包着,放在枕边,只要过一宵,便可以随意使用。睡在枕上,看着红包,想到明天买来的小鼓、刀枪、泥人、糖菩萨\footnote{〔糖菩萨〕一种小吃。把糖用模具做成菩萨样子。}……。然而她进来,又将一个福橘\footnote{〔福橘〕南方过年的习俗。橘音近“吉”,因此过年吃福建产的橘子,寓意“幸福吉祥”。}放在床头了。
    
    “哥儿,你牢牢记住!”她极其郑重地说。“明天是正月初一,清早一睁开眼睛,第一句话就得对我说:‘阿妈,恭喜恭喜!’记得么?你要记着,这是一年的运气的事情。不许说别的话!说过之后,还得吃一点福橘。”她又拿起那橘子来在我的眼前摇了两摇,“那么,一年到头,顺顺流流……。”
    
    梦里也记得元旦的,第二天醒得特别早,一醒,就要坐起来。她却立刻伸出臂膊,一把将我按住。我惊异地看她时,只见她惶急地看着我。
    
    她又有所要求似的,摇着我的肩。我忽而记得了——
    
    “阿妈,恭喜……。”
    
    恭喜恭喜!大家恭喜!真聪明!恭喜恭喜!”她于是十分欢喜似的,笑将起来,同时将一点冰冷的东西,塞在我的嘴里。我大吃一惊之后,也就忽而记得,这就是所谓福橘,元旦辟头\footnote{〔辟头〕开头。}的磨难,总算已经受完,可以下床玩耍去了。
    
    她教给我的道理还很多,例如说人死了,不该说死掉,必须说“老掉了”;死了人,生了孩子的屋子里,不应该走进去;饭粒落在地上,必须拣起来,最好是吃下去;晒裤子用的竹竿底下,是万不可钻过去的……。此外,现在大抵忘却了,只有元旦的古怪仪式记得最清楚。总之:都是些烦琐\footnote{〔烦琐〕繁琐。}之至,至今想起来还觉得非常麻烦的事情。
    
    然而我有一时也对她发生过空前的敬意。她常常对我讲“长毛”。她之所谓“长毛”者,不但洪秀全\footnote{〔洪秀全〕清晚期太平天国运动的发起者和领袖。}军,似乎连后来一切土匪强盗都在内,但除却革命党\footnote{〔革命党〕清末以兴中会为首、意图推翻帝制的革命团体。},因为那时还没有。她说得长毛非常可怕,他们的话就听不懂。她说先前长毛进城的时候,我家全都逃到海边去了,只留一个门房\footnote{〔门房〕大门口内侧的小房,有专人看守。也指看守门房的人。}和年老的煮饭老妈子看家。后来长毛果然进门来了,那老妈子便叫他们“大王”——据说对长毛就应该这样叫——诉说自己的饥饿。长毛笑道:“那么,这东西就给你吃了罢!”将一个圆圆的东西掷了过来,还带着一条小辫子,正是那门房的头。煮饭老妈子从此就骇破了胆,后来一提起,还是立刻面如土色,自己轻轻地拍着胸脯道:“阿呀\footnote{〔阿呀〕啊呀。},骇死我了,骇死我了……。”
    
    我那时似乎倒并不怕,因为我觉得这些事和我毫不相干的,我不是一个门房。但她大概也即觉到了,说道:“象你似的小孩子,长毛也要掳的,掳去做小长毛。还有好看的姑娘,也要掳。”
    
    “那么,你是不要紧的。”我以为她一定最安全了,既不做门房,又不是小孩子,也生得不好看,况且颈子上还有许多炙疮疤。
    
    “哪里的话?!”她严肃地说。“我们就没有用处?我们也要被掳去。城外有兵来攻的时候,长毛就叫我们脱下裤子,一排一排地站在城墙上,外面的大炮就放不出来;再要放,就炸了!”
    
    这实在是出于我意想之外的,不能不惊异。我一向只以为她满肚子是麻烦的礼节罢了,却不料她还有这样伟大的神力。从此对于她就有了特别的敬意,似乎实在深不可测;夜间的伸开手脚,占领全床,那当然是情有可原的了,倒应该我退让。
    
    这种敬意,虽然也逐渐淡薄起来,但完全消失,大概是在知道她谋害了我的隐鼠之后。那时就极严重地诘问,而且当面叫她阿长。我想我又不真做小长毛,不去攻城,也不放炮,更不怕炮炸,我惧惮她什么呢!
    
    但当我哀悼隐鼠,给它复仇的时候,一面又在渴慕着绘图的《山海经》\footnote{〔《山海经》〕古代流传的地理志,传说是大禹所作。有各种神怪奇物图解。}了。这渴慕是从一个远房的叔祖\footnote{〔叔祖〕祖父的弟弟。}惹起来的。他是一个胖胖的,和蔼的老人,爱种一点花木,如珠兰\footnote{〔珠兰〕即金粟兰。灌木,花成穗状,圆如栗米,色如黄金,香气浓郁,可做室内观赏盆栽。}、茉莉之类,还有极其少见的,据说从北边带回去的马缨花\footnote{〔马缨花〕杜鹃花品种,因花朵状似马缨而得名。产于云南、贵州、广西等地的山区。}。他的太太却正相反,什么也莫名其妙,曾将晒衣服的竹竿搁在珠兰的枝条上,枝折了,还要愤愤地咒骂道:“死尸!”这老人是个寂寞者,因为无人可谈,就很爱和孩子们往来,有时简直称我们为“小友”。在我们聚族而居的宅子里,只有他书多,而且特别。制艺和试帖诗\footnote{〔制艺和试帖诗〕清朝科举考试规定的程式化诗文。这里指当时书坊刊印的制艺和试帖诗范本。},自然也是有的;但我却只在他的书斋里,看见过陆玑的《毛诗草木鸟兽虫鱼疏》\footnote{〔《毛诗草木鸟兽虫鱼疏》〕三国时期吴国的陆玑编著,对《毛诗》中的动植物进行考注说明。《毛诗》指战国末年毛亨、毛苌辑注的《诗经》。},还有许多名目很生的书籍。我那时最爱看的是《花镜》\footnote{〔《花镜》〕即《秘传花镜》,清代陈淏子所著。主要讲藤木花草的分类,以及栽培花卉、饲养禽鸟兽畜昆虫的方法。},上面有许多图。他说给我听,曾经有过一部绘图的《山海经》,画着人面的兽,九头的蛇,三脚的鸟,生着翅膀的人,没有头而以两乳当作眼睛的怪物,……可惜现在不知道放在哪里了。
    
    很愿意看看这样的图画,但不好意思力逼他去寻找,他是很疏懒的。问别人呢,谁也不肯真实地回答我。压岁钱还有几百文,买罢,又没有好机会。有书买的大街离我家远得很,我一年中只能在正月间去玩一趟,那时候,两家书店都紧紧地关着门。
    
    玩的时候倒是没有什么的,但一坐下,我就记得绘图的《山海经》。
    
    大概是太过于念念不忘了,连阿长也来问《山海经》是怎么一回事。这是我向来没有和她说过的,我知道她并非学者,说了也无益;但既然来问,也就都对她说了。
    
    过了十多天,或者一个月罢,我还记得,是她告假回家以后的四五天,她穿着新的蓝布衫回来了,一见面,就将一包书递给我,高兴地说道:——“哥儿,有画儿的‘三哼经’,我给你买来了!”
    
    我似乎遇着了一个霹雳,全体\footnote{〔全体〕全身。}都震悚起来;赶紧去接过来,打开纸包,是四本小小的书,略略一翻,人面的兽,九头的蛇,……果然都在内。
    
    又使我发生新的敬意了,别人不肯做,或不能做的事,她却能够做成功。她确有伟大的神力。谋害隐鼠的怨恨,从此完全消灭了。
    
    这四本书,乃是我最初得到,最为心爱的宝书。
    
    书的模样,到现在还在眼前。可是从还在眼前的模样来说,却是一部刻印都十分粗拙的本子。纸张很黄;图象也很坏,甚至于几乎全用直线凑合,连动物的眼睛也都是长方形的。但那是我最为心爱的宝书,看起来,确是人面的兽;九头的蛇;一脚的牛;袋子似的帝江\footnote{〔帝江〕《山海经》记载的天山的山神。赤红如火,六足四翼,没有面目,能歌舞。};没有头而“以乳为目,以脐为口”,还要“执干戚而舞”的刑天\footnote{〔刑天〕《山海经》记载的巨人,与帝争神,被断头,葬于常羊之山。}。
    
    此后我就更其搜集绘图的书,于是有了石印\footnote{〔石印〕石版油墨印刷技术。18世纪末出现,19世纪传入中国,比称为“木刻”的雕版印刷更方便更好,广受欢迎。}的《尔雅音图》\footnote{〔《尔雅音图》〕《尔雅》是我国最早的辞典。西晋郭璞为《尔雅》注音、作图,内有大量插图。近代出版的画谱由清代姚之麟绘画。}和《毛诗品物图考》\footnote{〔《毛诗品物图考》〕18世纪日本汉学家对《毛诗》中动植物的图释著作。由冈元凤纂辑,橘国雄绘画。},又有了《点石斋丛画》\footnote{〔《点石斋丛画》〕点石斋书局印刷的画册,汇集了数百首诗及明清画家的插画。点石斋,1876年由英国商人厄内斯特·美查(Ernest Major)在上海创办。}和《诗画舫》\footnote{〔《诗画舫》〕汇集明晚期画家的唐诗插画谱,收录近五百首诗。}。《山海经》也另买了一部石印的,每卷都有图赞,绿色的画,字是红的,比那木刻的精致得多了。这一部直到前年还在,木刻的却已经记不清是什么时候失掉了。
    
    我的保姆,长妈妈即阿长,辞了这人世,大概也有了三十年了罢。我终于不知道她的姓名,她的经历;仅知道有一个过继的儿子,她大约是青年守寡的孤孀。仁厚黑暗的地母\footnote{〔地母〕中国古代信仰中的大地之神,又称后土。《山海经》中也有记载。与玉皇大帝合称皇天后土。民间认为人死后灵魂归于后土。}呵,愿在你怀里永安她的魂灵!
    
\end{normalsize}


\newpage

\textbf{注释}:

\vspace{-1em}

\begin{itemize}
    \setlength\itemsep{-0.2em}
    \item 〔憎恶〕厌恶仇恨。
    \item 〔惶急〕从旁劝说,使想做。
    \item 〔骇〕惊吓。
    \item 〔菩萨〕佛教中指修行有成的大觉悟者。也指心地慈善的人。
    \item 〔霹雳〕又急又响的雷。
    \item 〔震悚〕震惊惶恐。
    \item 〔掷〕扔,抛。
    \item 〔书斋〕书房。斋:闲居的房舍。
    \item 〔掳〕抢走,抓走。
    \item 〔孀〕称呼丧夫的寡妇。
    \item 〔图赞〕写在画面或图页上的赞美诗文。
    \item 〔保姆〕帮忙带小孩的妇女。
\end{itemize}

\chapter{谁是最可爱的人}

\begin{normalsize}
    
    在朝鲜的每一天,我都被一些东西感动着;我的思想感情的潮水,在放纵奔流着;我想把一切东西都告诉给我祖国的朋友们。但我最急于告诉你们的,是我思想感情的一段重要经历,这就是:我越来越深刻地感觉到,谁是我们最可爱的人!
    
    谁是我们最可爱的人呢?我们的战士,我感到他们是最可爱的人。
    
    也许还有人心里隐隐约约地说:你说的就是那些“兵”吗?他们看来是很平凡、很简单的哩,既看不出他们有什么高深的知识,又看不出他们有什么丰富的感情。可是,我要说,这是由于他跟我们的战士接触太少,还没有了解我们的战士:他们的品质是那样的纯洁和高尚,他们的意志是那样的坚韧和刚强,他们的气质是那样的淳朴和谦逊,他们的胸怀是那样的美丽和宽广!
    
    让我还是来说一段故事吧。
    
    还是在二次战役\footnote{〔二次战役〕1950年11月至12月,云山战役之后,中国人民志愿军和朝鲜人民军在朝鲜北部发起的一次围歼战。}的时候,有一支志愿军的部队向敌后猛插,去切断军隅里\footnote{〔军隅里〕朝鲜平安南道西北部,清川江下游南岸。}敌人的逃路。当他们赶到书堂站时,逃敌也恰恰赶到那里,眼看就要从汽车路上开过去。这支部队的先头边就匆匆占领了汽车路边一个很低的光光的小山冈,阻住敌人。一场壮烈的搏斗就开始了。敌人为了逃命,用了32架飞机、十多辆坦克发起集团冲锋,向这个连的阵地汹涌卷来,整个山顶的土都被打翻了,汽油弹的火焰把这个阵地烧红了。但是,勇士们在这烟与火的山冈上,高喊着口号,一次又一次把敌人打死在阵地前面。敌人的死尸像谷个子\footnote{〔谷个子〕收割下来、一捆一捆的谷子。}似的在山前堆满了,血也把这山冈流红了。可是敌人还是要拼死争夺,好使自己的主力不致覆灭。这场激战整整持续了八个小时。最后,勇士们的了弹打光了。蜂拥上来的敌人占领了山头,把他们压到山脚。飞机掷下的汽油弹把他们的身上烧着了火。这时候,勇士们是仍然不会后退的呀,他们把枪一摔,向敌人扑去,身上帽子上呼呼地冒着火苗,把敌人抱住,让身上的火,也把占领阵地的敌人烧死。……据这个营的营长告诉我,战后,这个连的阵地上,枪支完全摔碎了,机枪零件扔得满山都是。烈士们的遗体,保留着各种各样的姿势,。有抱住敌人腰的,有抱住敌人头的,有掐住敌人脖子把敌人摁倒在地上的,和敌人倒在一起,烧在一起。有一个战士,他手里还紧握着一个手榴弹,弹体上沾满脑浆;和他死在一起的美国鬼子,脑浆迸裂,涂了一地。另一个战士,嘴里还衔着敌人的半块耳朵。在掩埋烈士遗体的时候,由于他们两手扣着,把敌人抱得那样紧,分都分不开,以致把有些人的手指都掰断了。……这个连虽然伤亡很大,他们却打死了三百多个敌人,更重要的,他们使得我们部队的主力赶上来,聚歼了敌人。
    
    这就是朝鲜战场上一次最壮烈的战头——松骨峰战斗\footnote{〔松骨峰战斗〕现称松骨峰阻击战。参战部队为第38军112师335团一营三连,记集体特等功,授予“英雄部队”称号。},或者叫书堂站战斗。假若需要立纪念碑的话,让我把带火扑敌和用刺刀跟敌人拼死在一起的烈士们的名字记下吧。他们的名字是:王金传、邢玉堂、王文英、熊官全、王金侯、赵锡杰、隋金山、李玉安\footnote{〔李玉安〕后获救生还,1997年去世。}、丁振岱、张贵生、崔玉亮、李树国。还有一个战士,已经不可能知道他的名字了。让我们的烈士们千载万世永垂不朽吧!
    
    这个营的营长向我叙说了以上的情形,他的声调是缓慢的,他的感情是沉重的。他说在阵地上掩埋烈士的时候,他掉了眼泪。但是,他接着说:“你不要以为我是为他们伤心,不,我是为他们骄傲!我觉得我们的战士太伟大了,太可爱了,我不能不被他们感动得掉下泪来。”
    
    朋友,当你听到这段英雄事迹的时候,你的感想如何呢?你不觉得我们的战士是可爱的吗?你不以我们的祖国有着这样的英雄而自豪吗?
    
    我们的战士,对敌人这样狠,而对朝鲜人民却是那样的爱,充满国际主义的深厚热情。
    
    在汉江\footnote{〔汉江〕朝鲜半岛的河流,在汉朝设的四郡内。}北岸,我遇到一个青年战士,他今年才21岁,名叫马玉祥,是黑龙江青冈县人。他长着一副微黑透红的脸膛,高高的个儿,站在那儿,像秋天田野里一株红高粱那样淳朴可爱。不过因为他才从阵地上下来,显得稍微疲劳些,眼里的红丝还没有退净。他原来是炮兵连的。有一天夜里,他被一阵哭声惊醒了,出去一看,是一个朝鲜老妈妈坐在山冈上哭。原来她的房子被炸毁了,她在山里搭了个窝棚,窝棚又被炸毁了。回来,他马上到连部要求调到步兵连去,正好步兵连也需要人,就批准了他。我说:“在炮兵连不是一样打敌人吗?”“那,不同!”他说,“离敌人越近,越觉着打得过瘾,越觉着打得解恨!”
    
    在汉江南岸阻击敌人的日子里,有一天他从阵地上下来做饭。刚一进村,有几架敌机袭过来,打了一阵机关炮,接着就扔下了两个大燃烧弹。有几间房子着了火,火又盛,烟又大,使人不敢到跟前去。这时候,他听见烟火里有一个小孩子哇哇哭叫的声音。他马上穿过浓烟到近处一看,一个朝鲜的中年男人在院子里倒着,小孩子的哭声还在屋里。他走到屋门口,屋门口的火苗呼呼的,已经进不去人,门窗的纸已经烧着。小孩子的哭声随着那滚滚的浓烟传出来,听得真真切切。当他叙述到这里的时候,他说:“我能够不进去吗?我不能!我想,要在祖国遇见这种情形,我能够进去,那么,在朝鲜我就可以不进去吗?朝鲜人民和我们祖国的人民不是一样的吗?我就踹开门,扑了进去。呀!满屋子灰洞洞的烟,只能听见小孩哭,看不见人。我的眼也睁不开,脸烫得像刀割一般。我也不知道自己的身上着了火没有,我也不管它了,只是在地上乱摸。先摸着一个大人,拉了拉没拉动;又向大人的身后摸,才摸着小孩的腿,我就一把抓着抱起来,跳出门去。我一看小孩子,是挺好的一个小孩儿啊。他穿着小短褂儿,光着两条小腿儿,小腿儿乱蹬着,哇哇地哭。我心想:‘不管你哭不哭,不救活你家大人,谁养活你哩!’这时候,火更大了,屋子里的家具什物\footnote{〔什物〕杂物。}也烧着了。我就把他往地上一放,就又从那火门里钻了进去一拉那个大人,她哼了一声,我就使劲往外拉,见她又不动了。凑近一看,见她脸上流下来的血已经把她胸前的白衣染红了,眼睛已经闭上。我知道她不行了,才赶忙跳出门外,扑灭身上的火苗,抱起这个无父无母的孩子。……”
    
    朋友,当你听到这段事迹的时候,你的感觉又是如何呢?你不觉得我们的战士是最可爱的人吗?
    
    谁都知道,朝鲜战场是艰苦些。但战士们是怎样想的呢?有一次,我见到一个战士,在防空洞里,吃一口炒面\footnote{〔炒面〕用小麦粉等面粉炒制的干粮。},就一口雪。我问他:“你不觉得苦吗?”他把正送往嘴里的一勺雪收回来,笑了笑,说:“怎么能不觉得?我们革命军队又不是个怪物。不过我们的光荣也就在这里。”他把小勺儿干脆放下,兴奋地说,“就拿吃雪来说吧。我在这里吃雪,正是为了我们祖国的人民不吃雪。他们可以坐在挺豁亮的屋子里,泡上一壶茶,守住个小火炉子,想吃点什么就做点什么。”他又指了指狭小潮湿的防空洞说,“再比如蹲防空洞吧,多憋闷得慌哩,眼看着外面好好的太阳不能晒,光光的马路不能走。可是我在这里蹲防空洞,祖国的人民就可以不蹲防空洞啊,他们就可以在马路上不慌不忙地走啊。他们想骑车子也行,想走路也行,边遛达边说话也行。只要能使人民得到幸福,就是我们最大的幸福。所以,”他又把雪放到嘴里,像总结似的说“我在这里流点血不算什么,吃这点苦又算什么哩!”我又问:“你想不想祖国啊?”他笑起来:“谁不想哩,说不想,那是假话,可是我不愿意回去。如果回去,祖国的老百姓问,‘我们托付给你们的任务完成得怎么样啦?’我怎么答呢?我说‘朝鲜半边红,半边黑’,这算什么话呢?”我接着问:“你们经历了这么多危险,吃了这么多苦,你们对祖国对朝鲜有什么要求吗?”他想了一下,才回答我:“我们什么也不要。可是说心里话,——我这话可不一定恰当啊,我们是想要这么大的一个东西……”他笑着,用手指比个铜子儿大小,怕我不明白,“一块‘朝鲜解放纪念章’,我们愿意戴在胸脯上,回到咱们的祖国去。”
    
    朋友们,用不着多举例,你们已经可以了解我们的战士是怎样一种人,这种人有一种什么品质,他们的灵魂多么地美丽和宽广。他们是历史上、世界上第一流的战士,第一流的人!他们是世界上一切伟大人民的优秀之花!是我们值得骄傲,我们以我们的祖国有这样的英雄而骄傲,我们以生在这个英雄的国度而自豪!
    
    亲爱的朋友们,当你坐上早晨第一列电车驰向工厂的时候,当你扛上犁耙走向田野的时候,当你喝完一杯豆浆、提着书包走向学校的时候,当你坐到办公桌前开始这一天工作的时候,当你往孩子口里塞苹果的时候,当你和爱人一起散步的时候……朋友,你是否意识到你是在幸福之中呢?你也许很惊讶地说:“这是很平常的呀!”可是,从朝鲜归来的人,会知道你正生活在幸福中。请你意识到这是一种幸福吧,因为只有你意识到这一点,你才能更深刻了解我们的战士在朝鲜奋不顾身的原因。朋友!你是这么爱我们的祖国,爱我们的伟大领袖毛主席,你一定会深深地爱我们的战士——他们确实是我们最可爱的人!
    
\end{normalsize}


\newpage

\textbf{注释}:

\vspace{-1em}

\begin{itemize}
    \setlength\itemsep{-0.2em}
    \item 〔淳朴〕忠厚朴实。
    \item 〔犁耙〕农具。犁用来耕地,耙用来平整土地。
    \item 〔豁亮〕宽敞明亮。
    \item 〔遛达〕闲逛,散步。
    \item 〔奋不顾身〕奋勇而不顾自身安危。
\end{itemize}

\chapter{致杨振宁}

\begin{normalsize}
    
    \noindent 振宁:
    
    \vspace{24pt}
    
    你这次回到祖国来,老师们和同学们见到你真是感到非常高兴。我这次从外地到北京来看见你,也确实感到非常高兴。在你离京之后,我也就要回到工作岗位上去。
    
    关于你要打听的事,我已向组织上了解,寒春\footnote{〔寒春〕原名Joan Hinton,美国核物理学家。1948年到中国与男友阳早(Sid Engst)结婚并定居中国。}确实没有参加过我国任何有关制造核武器\footnote{〔核武器〕利用原子核内的结合能的武器。当时指裂变弹(原子弹)和聚变弹(氢弹)。}的事,我特地写这封信告诉你。
    
    你这次回来能见到总理\footnote{〔总理〕指周恩来。},总理这样的高龄,能在百忙中用这么长的时间和你亲切地谈话,关怀地询问你各方面的情况,使我们在座的人都受到很大的教育,希望你能经常地想起这次亲切的接见。
    
    你这次回来能看见祖国各方面的革命和建设的情况,这真是难得的机会。希望你能了解到祖国的解放是来之不易的,是无数先烈流血牺牲换来的。毛主席说:“成千成万的先烈,为着人民的利益,在我们的前头英勇地牺牲了,让我们高举起他们的旗帜,踏着他们的血迹前进吧!”你谈到人生的意义应该明确,我想人生的意义就应该遵照毛主席所说的这句话去做。我的世界观改得也很差,许多私心杂念随时冒出来,像在工作中,顺利时就沾沾自喜,不顺利时就气馁,怕负责任等等。但我愿意引用毛主席这句话,与振宁共勉。希望你在国外时能经常想到我们的祖国。
    
    这次在北京见到你,时间虽然不长,但每天晚上回来后心情总是不很平静,从小在一起,各个时期的情景,总是涌上心头。这次送你走后,心里自然有些惜别之感。和你见面几次,心里总觉得缺点什么东西似的,细想起来心里总是有“友行千里心担忧”的感觉。因此心里总是盼望着“但愿人长久,千里共同途”。
    
    夜深了,不多谈了。代我向你父母问安。祝两位老人家身体健康。祝你一路顺风。
    
    \hfill 稼先
    
    \hfill 8.13~/~71
    
    \vspace{36pt}
    
    \begin{flushright}
        
    \end{flushright}
    
    
    
\end{normalsize}


\newpage

\textbf{注释}:

\vspace{-1em}

\begin{itemize}
    \setlength\itemsep{-0.2em}
    \item 〔气馁〕失去信心和勇气,灰心丧气。
    \item 〔世界观〕对世界、社会的根本看法。
    \item 〔共勉〕相互鼓励、激励。
    \item 〔惜别〕舍不得离别。
\end{itemize}

\chapter{批评与自我批评}

\begin{normalsize}
    
    有无认真的自我批评,也是我们和其他政党互相区别的显着的标志之一。
    
    我们曾经说过,房子是应该经常打扫的,不打扫就会积满了灰尘;脸是应该经常洗的,不洗也就会灰尘满面。我们同志的思想,我们党的工作,也会沾染灰尘的,也应该打扫和洗涤。“流水不腐,户枢不蠹\footnote{〔流水不腐,户枢不蠹〕流动的水不会腐臭,经常转动的门轴不会被虫蛀。}”,是说它们在不停的运动中抵抗了微生物或其他生物的侵蚀。
    
    对于我们,经常地检讨工作,在检讨中推广民主作风,不惧怕批评和自我批评,实行“知无不言,言无不尽”“言者无罪,闻者足戒”“有则改之,无则加勉”这些中国人民的有益的格言,正是抵抗各种政治灰尘和政治微生物侵蚀我们同志的思想和我们党的肌体的唯一有效的方法。以“惩前毖后,治病救人”为宗旨的整风运动\footnote{〔整风运动〕指1941年至1945年的延安整风运动。}之所以发生了很大的效力,就是因为我们在这个运动中展开了正确的而不是歪曲的、认真的而不是敷衍的批评和自我批评。
    
    以中国最广大人民的最大利益为出发点的中国共产党人,相信自己的事业是完全合乎正义的,不惜牺牲自己个人的一切,随时准备拿出自己的生命去殉我们的事业,难道还有什么不适合人民需要的思想、观点、意见、办法,舍不得丢掉的吗?难道我们还欢迎任何政治的灰尘、政治的微生物来玷污我们的清洁的面貌和侵蚀我们的健全的肌体吗?无数革命先烈为了人民的利益牺牲了他们的生命,使我们每个活着的人想起他们就心里难过,难道我们还有什么个人利益不能牺牲,还有什么错误不能抛弃吗?
    
    我们很快就要在全国胜利了。这个胜利将冲破帝国主义的东方战线,具有伟大的国际意义。夺取这个胜利,已经是不要很久的时间和不要花费很大的气力了;巩固这个胜利,则是需要很久的时间和要花费很大的气力的事情。
    
    资产阶级怀疑我们的建设能力。帝国主义者估计我们终久会要向他们讨乞才能活下去。因为胜利,党内的骄傲情绪,以功臣自居的情绪,停顿起来不求进步的情绪,贪图享乐不愿再过艰苦生活的情绪,可能生长。因为胜利,人民感谢我们,资产阶级也会出来捧场。敌人的武力是不能征服我们的,这点已经得到证明了。资产阶级的捧场则可能征服我们队伍中的意志薄弱者。
    
    可能有这样一些共产党人,他们是不曾被拿枪的敌人征服过的,他们在这些敌人面前不愧英雄的称号;但是经不起人们用糖衣裹着的炮弹的攻击,他们在糖弹面前要打败仗。我们必须预防这种情况。
    
    夺取全国胜利,这只是万里长征走完了第一步。如果这一步也值得骄傲,那是比较渺小的,更值得骄傲的还在后头。在过了几十年之后来看中国人民民主革命的胜利,就会使人们感觉那好像只是一出长剧的一个短小的序幕。剧是必须从序幕开始的,但序幕还不是高潮。中国的革命是伟大的,但革命以后的路程更长,工作更伟大,更艰苦。
    
    这一点现在就必须向党内讲明白,务必使同志们继续地保持谦虚、谨慎、不骄、不躁的作风,务必使同志们继续地保持艰苦奋斗的作风。我们有批评和自我批评这个马克思列宁主义的武器。我们能够去掉不良作风,保持优良作风。我们能够学会我们原来不懂的东西。我们不但善于破坏一个旧世界,我们还将善于建设一个新世界。中国人民不但可以不要向帝国主义者讨乞也能活下去,而且还将活得比帝国主义国家要好些。
    
\end{normalsize}


\newpage

\textbf{注释}:

\vspace{-1em}

\begin{itemize}
    \setlength\itemsep{-0.2em}
    \item 〔惩前毖后〕纠正以前的过错,今后小心不重犯。惩:停止。毖:谨慎小心。
    \item 〔玷污〕弄脏,污损。
    \item 〔敷衍〕表面上应付。
\end{itemize}

\chapter{反对自由主义}

\begin{normalsize}
    
    我们主张积极的思想斗争,因为它是达到党内和革命团体内的团结使之利于战斗的武器。每个共产党员和革命分子,应该拿起这个武器。
    
    但是自由主义\footnote{〔自由主义〕指一种反对集体主义和进步主义的思想。应注意与其他名为“自由主义”的思想区分。}取消思想斗争,主张无原则的和平,结果是腐朽庸俗的作风发生,使党和革命团体的某些组织\footnote{〔组织〕按一定的目的、规则和形式建立起来的集体。这里指共产党组织。}和某些个人在政治上腐化起来。
    
    自由主义有各种表现。
    
    因为是熟人、同乡、同学、知心朋友、亲爱者、老同事、老部下,明知不对,也不同他们作原则上的争论,任其下去,求得和平和亲热。或者轻描淡写地说一顿,不作彻底解决,保持一团和气。结果是有害于团体,也有害于个人。这是第一种。
    
    不负责任的背后批评,不是积极地向组织建议。当面不说,背后乱说;开会不说,会后乱说。心目中没有集体生活的原则,只有自由放任。这是第二种。
    
    事不关己,高高挂起;明知不对,少说为佳;明哲保身,但求无过。这是第三种。
    
    命令不服从,个人意见第一。只要组织照顾,不要组织纪律。这是第四种。
    
    不是为了团结,为了进步,为了把事情弄好,向不正确的意见斗争和争论,而是个人攻击,闹意气,泄私愤,图报复。这是第五种。
    
    听了不正确的议论也不争辩,甚至听了反革命分子\footnote{〔反革命分子〕人民之中反对、反抗革命的人。}的话也不报告,泰然处之,行若无事。这是第六种。
    
    见群众不宣传,不鼓动,不演说,不调查,不询问,不关心其痛痒,漠然置之,忘记了自己是一个共产党员,把一个共产党员混同于一个普通的老百姓。这是第七种。
    
    见损害群众利益的行为不愤恨,不劝告,不制止,不解释,听之任之。这是第八种。
    
    办事不认真,无一定计划,无一定方向,敷衍了事,得过且过,做一天和尚撞一天钟。这是第九种。
    
    自以为对革命有功,摆老资格,大事做不来,小事又不做,工作随便,学习松懈。这是第十种。
    
    自己错了,也已经懂得,又不想改正,自己对自己采取自由主义。这是第十一种。
    
    还可以举出一些。主要的有这十一种。
    
    所有这些,都是自由主义的表现。
    
    革命的集体组织中的自由主义是十分有害的。它是一种腐蚀剂,使团结涣散,关系松懈,工作消极,意见分歧。它使革命队伍失掉严密的组织和纪律,政策不能贯彻到底,党的组织和党所领导的群众发生隔离。这是一种严重的恶劣倾向。
    
    自由主义的来源,在于小资产阶级\footnote{〔小资产阶级〕介于资产阶级和无产阶级之间的过渡阶级,如小商人、手工业者、中农等,占有一定的生产资料,或许雇佣、剥削他人,但主要依靠自身劳动,被大资产阶级剥削。}的自私自利性,以个人利益放在第一位,革命利益放在第二位,因此产生思想上、政治上、组织上的自由主义。
    
    自由主义者以抽象的教条看待马克思主义的原则。他们赞成马克思主义,但是不准备实行之,或不准备完全实行之,不准备拿马克思主义代替自己的自由主义。这些人,马克思主义是有的,自由主义也是有的:说的是马克思主义,行的是自由主义;对人是马克思主义,对己是自由主义。两样货色齐备,各有各的用处。这是一部分人的思想方法。
    
    自由主义是机会主义\footnote{〔机会主义〕也称投机主义,指只看结果、不顾原则,只考虑利用一切实现自身目标的思想。}的一种表现,是和马克思主义根本冲突的。它是消极的东西,客观上起着援助敌人的作用,因此敌人是欢迎我们内部保存自由主义的。自由主义的性质如此,革命队伍中不应该保留它的地位。
    
    我们要用马克思主义的积极精神,克服消极的自由主义。一个共产党员,应该是襟怀坦白,忠实,积极,以革命利益为第一生命,以个人利益服从革命利益;无论何时何地,坚持正确的原则,同一切不正确的思想和行为作不疲倦的斗争,用以巩固党的集体生活,巩固党和群众的联系;关心党和群众比关心个人为重,关心他人比关心自己为重。这样才算得一个共产党员。
    
    一切忠诚、坦白、积极、正直的共产党员团结起来,反对一部分人的自由主义的倾向,使他们改变到正确的方面来。这是思想战线的任务之一。
    
\end{normalsize}


\newpage

\textbf{注释}:

\vspace{-1em}

\begin{itemize}
    \setlength\itemsep{-0.2em}
    \item 〔泰然处之〕不在意,当作不重要的事。
    \item 〔漠然置之〕不经意,冷漠地处理。
    \item 〔彻底〕通透到底,完完全全,没有保留或遗留。
    \item 〔涣散〕消融散开,不再成为组织。
    \item 〔庸俗〕不出力不高尚,不显露能力,随大流。
    \item 〔敷衍了事〕表面上应付就算了,其实并不认真对待。
    \item 〔襟怀〕胸襟、胸怀。心里怀着的想法、抱负,引申为心里放得下的、能包容的,心里的志趣和抱负。
    \item 〔明哲保身〕原义是指明于事理的人善于保护自己。现在指为了保住自身利益回避原则斗争。
    \item 〔松懈〕松弛疏远,不紧密。
    \item 〔轻描淡写〕绘画中用淡色轻轻描绘。比喻说话或作文有意减轻问题的重要性。
    \item 〔意气〕偏激的主观情绪。
    \item 〔贯彻〕完全理解并彻底实践,完全实现或体现。
    \item 〔腐朽〕腐化,腐烂败坏,不再贯彻原则、制度、精神。
    \item 〔倾向〕偏向,趋势。
    \item 〔积极〕努力尝试形成正面影响的。
    \item 〔消极〕不努力,不积极,阻碍形成正面影响的。
\end{itemize}

\chapter{论雷峰塔的倒掉}

\begin{normalsize}
    
    听说,杭州西湖上的雷峰塔\footnote{〔雷峰塔〕杭州西湖净慈寺前的砖塔,北宋吴越王钱俶建造。因建在一座叫雷锋山的小山上,所以被称为雷峰塔。1924年9月25日倒塌。}倒掉了,听说而已,我没有亲见。但我却见过未倒的雷峰塔,破破烂烂的映掩于湖光山色之间,落山的太阳照着这些四近的地方,就是“雷峰夕照”,西湖十景之一。“雷峰夕照”的真景我也见过,并不见佳,我以为。
    
    然而一切西湖胜迹的名目之中,我知道得最早的却是这雷峰塔。我的祖母曾经常常对我说,白蛇娘娘就被压在这塔底下!有个叫做许仙的人救了两条蛇,一青一白,后来白蛇便化作女人来报恩,嫁给许仙了;青蛇化作丫鬟,也跟着。一个和尚,法海禅师\footnote{〔禅师〕对佛教僧侣的敬称。},得道的禅师,看见许仙脸上有妖气,——凡讨妖怪作老婆的人,脸上就有妖气的,但只有非凡的人才看得出——便将他藏在金山寺的法座后,白蛇娘娘来寻夫,于是就“水漫金山”。我的祖母讲起来还要有趣得多,大约是出于一部弹词\footnote{〔弹词〕南方地区一种把故事编成韵语,有曲有白、曲白相生、以弦乐伴唱的说唱文学。始见于明代中叶,至清代极为繁荣,是南方民间曲艺的代表之一。}叫作《义妖传》\footnote{〔《义妖传》〕清代中期弹词作品,又名《白蛇传》《雷峰塔》。}里的,但我没有看过这部书,所以也不知道“许仙”“法海”究竟是否这样写。总而言之,白蛇娘娘终于中了法海的计策,被装在一个小小的钵盂\footnote{〔钵盂〕佛教僧侣讨饭化缘用的食器,也在诵经时敲击。}里了。钵盂埋在地里,上面还造起一座镇压的塔来,这就是雷峰塔。此后似乎事情还很多,如“白状元祭塔”之类,但我现在都忘记了。
    
    那时我惟一的希望,就在这雷峰塔的倒掉。后来我长大了,到杭州,看见这破破烂烂的塔,心里就不舒服。后来我看看书,说杭州人又叫这塔作“保叔塔”,其实应该写作“保俶塔”\footnote{〔“保俶塔”〕这里作者记忆有误。保俶塔是西湖宝石山顶的另一座塔,当时仍存。作者在附记中声明更正。},是钱王的儿子造的。那么,里面当然没有白蛇娘娘了,然而我心里仍然不舒服,仍然希望他倒掉。
    
    现在,他居然倒掉了,则普天之下的人民,其欣喜为何如?
    
    这是有事实可证的。试到吴越的山间海滨,探听民意去。凡有田夫野老,蚕妇村氓\footnote{〔田夫野老,蚕妇村氓〕古代读书人对不识字的农村劳动人民的蔑称,这里用作反讽。},除了几个脑髓\footnote{〔脑髓〕这里指头脑。}里有点贵恙\footnote{〔贵恙〕称对方患的病的敬辞,这里用做反讽。恙:病。}的之外,可有谁不为白娘娘抱不平,不怪法海太多事的?
    
    和尚本应该只管自己念经。白蛇自迷许仙,许仙自娶妖怪,和别人有什么相干呢?他偏要放下经卷,横来招是搬非\footnote{〔招是搬非〕招惹是非,引起冲突。},大约是怀着嫉妒罢,——那简直是一定的。
    
    听说,后来玉皇大帝\footnote{〔玉皇大帝〕又称昊天上帝。道教神话中众神仙的君主。}也就怪法海多事,以至荼毒生灵,想要拿办\footnote{〔拿办〕捉拿法办,抓捕到官府依法办理。}他了。他逃来逃去,终于逃在蟹壳里避祸,不敢再出来,到现在还如此。我对于玉皇大帝所作的事,腹诽的非常多,独于这一件却很满意,因为“水漫金山”一案,的确应该由法海负责;他实在办得很不错的。只可惜我那时没有打听这话的出处,或者不在《义妖传》中,却是民间的传说罢。
    
    秋高稻熟时节,吴越间所多的是螃蟹,煮到通红之后,无论取哪一只,揭开背壳来,里面就有黄,有膏;倘是雌的,就有石榴子一般鲜红的子。先将这些吃完,即一定露出一个圆锥形\footnote{〔圆锥形〕底面为圆形的锥子形状。}的薄膜,再用小刀小心地沿着锥底切下,取出,翻转,使里面向外,只要不破,便变成一个罗汉模样的东西,有头脸,身子,是坐着的,我们那里的小孩子都称他“蟹和尚”,就是躲在里面避难的法海。
    
    当初,白蛇娘娘压在塔底下,法海禅师躲在蟹壳里。现在却只有这位老禅师独自静坐了,非到螃蟹断种的那一天为止出不来。莫非他造塔的时候,竟没有想到塔是终究要倒的么?
    
    活该。
    
\end{normalsize}


\newpage

\textbf{注释}:

\vspace{-1em}

\begin{itemize}
    \setlength\itemsep{-0.2em}
    \item 〔映掩〕互相遮掩映照,互相衬托。现在一般写作“掩映”。
    \item 〔湖光山色〕山和湖的风光景色,泛指风景秀美。
    \item 〔胜迹〕名胜古迹,风景优美、有文明遗迹的著名地方。胜:优美的。
    \item 〔腹诽〕把坏话放在肚子里,指内心暗自不满、说坏话,但不说出来。
    \item 〔荼毒〕毒害、使受苦。荼:一种苦菜。
    \item 〔普天之下〕全天下,全世界。
    \item 〔夕照〕夕阳的光。这里指傍晚夕阳光照下的景色。
    \item 〔罗汉〕本来是佛教用语,是阿罗汉的简称,指断绝了欲望和烦恼的僧人。由于佛教罗汉塑像的形象,也泛指矮胖、腰围粗而神态庄严的人。
\end{itemize}

\chapter{“友邦惊诧”论}

\begin{normalsize}
    
    只要略有知觉\footnote{〔知觉〕这里指了解情况。}的人就都知道:这回学生的请愿\footnote{〔学生的请愿〕指1931年12月间全国各地学生为反对“九一八”事变中蒋介石的不抵抗政策到南京请愿的事件。对于这次学生爱国行动,国民党政府于12月5日通令全国,禁止请愿;17日当各地学生联合向国民党中央党部请愿时,又命令军警逮捕和枪杀请愿学生,当场打死二十余人,打伤百余人;18日还电令各地军政当局紧急处置请愿事件。},是因为日本占据了辽吉\footnote{〔辽吉〕辽宁和吉林。},南京政府束手无策,单会去哀求国联\footnote{〔哀求国联〕“九一八”事变后,国民党政府多次向国联申诉,11月22日当日军进攻锦州时,又向国联提议划锦州为中立区,以中国军队退入关内为条件请求日军停止进攻;12月15日在日军继续进攻锦州时再度向国联申诉,请求它出面干涉,阻止日本帝国主义扩大侵华战争。},而国联却正和日本是一伙。读书呀,读书呀,不错,学生是应该读书的,但一面也要大人老爷们不至于葬送土地,这才能够安心读书。报上不是说过,东北大学\footnote{〔东北大学〕奉系军阀张作霖推动创办的一所大学,1923年在沈阳成立,“九一八”事变后流亡。}逃散,冯庸大学\footnote{〔冯庸大学〕奉系军阀冯庸创办的一所大学,1927年在沈阳成立,“九一八”事变后停办。}逃散,日本兵看见学生模样的就枪毙吗?放下书包来请愿,真是已经可怜之至。不道\footnote{〔不道〕不料,想不到。}国民党政府却在十二月十八日通电\footnote{〔通电〕用电报通知。当时电报为主要的远程快速通讯手段,简称为“电”,如“通电全国”指用电报发送到全国各地大报馆,通告全国。}各地军政当局文里,又加上他们“捣毁机关,阻断交通,殴伤中委,拦劫汽车,横击路人及公务人员,私逮刑讯,社会秩序,悉被破坏”的罪名,而且指出结果,说是“友邦人士,莫名惊诧,长此以往,国将不国”了!
    
    好个“友邦人士”!日本帝国主义的兵队强占了辽吉,炮轰机关,他们不惊诧;阻断铁路,追炸客车,捕禁官吏,枪毙人民,他们不惊诧。中国国民党治下的连年内战,空前水灾,卖儿救穷,砍头示众,秘密杀戮,电刑逼供,他们也不惊诧。在学生的请愿中有一点纷扰,他们就惊诧了!
    
    好个国民党政府的“友邦人士”!是些什么东西!即使所举的罪状是真的罢,但这些事情,是无论那一个“友邦”也都有的,他们的维持他们的“秩序”的监狱,就撕掉了他们的“文明”的面具。摆什么“惊诧”的臭脸孔呢?
    
    可是“友邦人士”一惊诧,我们的国府就怕了,“长此以往,国将不国”了,好像失了东三省,党国倒愈像一个国,失了东三省谁也不响,党国倒愈像一个国,失了东三省只有几个学生上几篇“呈文”,党国倒愈像一个国,可以博得“友邦人士”的夸奖,永远“国”下去一样。
    
    几句电文,说得明白极了:怎样的党国,怎样的“友邦”。“友邦”要我们人民身受宰割,寂然无声,略有“越轨”,便加屠戮;党国是要我们遵从这“友邦人士”的希望,否则,他就要“通电各地军政当局”,“即予紧急处置,不得于事后借口无法劝阻,敷衍塞责”了!
    
    因为“友邦人士”是知道的:日兵“无法劝阻”,学生们怎会“无法劝阻”?每月一千八百万的军费,四百万的政费,作什么用的呀,“军政当局”呀?
    
    写此文后刚一天,就见二十一日《申报》登载南京专电\footnote{〔专电〕记者专门为本报社、电台发来的电讯。}云:“考试院部员张以宽,盛传前日为学生架去重伤。兹据张自述,当时因车夫误会,为群众引至中大\footnote{〔中大〕南京中央大学。},旋出校回寓,并无受伤之事。至行政院某秘书被拉到中大,亦当时出来,更无失踪之事。”而“教育消息”栏内,又记本埠一小部分学校赴京请愿学生死伤的确数,则云:“中公\footnote{〔中公〕中公,中国公学;复旦,复旦大学;复旦附中,复旦大学附属实验中学;东亚,东亚体育专科学校;上中,上海中学;文生氏,文生氏高等英文学校。这些都是当时上海的私立学校。}死二人,伤三十人,复旦伤二人,复旦附中伤十人,东亚失踪一人(系女性),上中失踪一人,伤三人,文生氏死一人,伤五人……”可见学生并未如国府通电所说,将“社会秩序,破坏无余”,而国府则不但依然能够镇压,而且依然能够诬陷,杀戮。“友邦人士”,从此可以不必“惊诧莫名”,只请放心来瓜分就是了。
    
\end{normalsize}


\newpage

\textbf{注释}:

\vspace{-1em}

\begin{itemize}
    \setlength\itemsep{-0.2em}
    \item 〔惊诧〕惊讶诧异。
    \item 〔敷衍〕表面上应付,其实并不认真对待。
    \item 〔塞责〕(勉强)完成责任。塞:闭上,堵住。
    \item 〔兹〕现在。
    \item 〔电文〕用电报发送的文字。
    \item 〔旋〕不久后,经过很短时间。
    \item 〔本埠〕指上海。埠:码头,引申指与外国通商的城市。
\end{itemize}

\chapter{看云识天气}

\begin{normalsize}
    
    天上的云,真是姿态万千,变化无常。它们有的像羽毛,轻轻地飘在空中;有的像鱼鳞,一片片整整齐齐地排列着;有的像羊群,来来去去;有的像一床大棉被,严严实实地盖住了天空;还有的像峰峦,像河流,像雄狮,像奔马……它们有时把天空点缀得很美丽,有时又把天空笼罩得很阴森。刚才还是白云朵朵,阳光灿烂;一霎间却又是乌云密布,大雨倾盆。云就像是天气的“招牌”:天上挂什么云,就将出现什么样的天气。
    
    经验告诉我们:天空的薄云,往往是天气晴朗的象征;那些低而厚密的云层,常常是阴雨风雪的预兆。
    
    那最轻盈、站得最高的云,叫卷云。这种云很薄,阳光可以透过云层照到地面,房屋和树木的光与影依然很清晰。卷云丝丝缕缕地飘浮着,有时像白猫的尾毛,有时像细薄的白纱。如果卷云成群成行地排列在空中,好像微风吹过水面引起的鳞波,这就成了卷积云。卷云和卷积云都很高,那里水分少,它们一般不会带来雨雪。还有一种像棉花团似的白云,叫积云。它们常在两千米左右的天空,一朵朵分散着,映着灿烂的阳光,云块四周散发出金黄的光辉。积云都在上午出现,午后最多,傍晚渐渐消散。在晴天,我们还会偶见一种高积云。高积云是成群的扁球状的云块,排列很匀称,云块间露出碧蓝的天幕,远远望去,就像草原上雪白的羊群。卷云、卷积云、积云和高积云,都是很美丽的。
    
    当那连绵的雨雪将要来临的时候,卷云在聚集着,天空渐渐出现一层薄云,仿佛蒙上了白色的绸幕。这种云叫卷层云。卷层云慢慢地向前推进,天气就将转阴。接着,云层越来越低,越来越厚,隔了云看太阳或月亮,就像隔了一层毛玻璃\footnote{〔毛玻璃〕也叫磨砂玻璃,在玻璃表面做出微小的凹凸不平,通过漫散射让玻璃模糊透光。},朦胧不清。这时卷层云已经改名换姓,该叫它高层云了。出现了高层云,往往在几个钟头内便要下雨或者下雪。最后,云压得更低,变得更厚,太阳和月亮都躲藏了起来,天空被暗灰色的云块密密层层地布满了。这种云叫雨层云。雨层云一形成,连绵不断的雨雪也就降临了。
    
    夏天,雷雨到来之前,在天空先会看到积云。积云如果迅速地向上凸起,形成高大的云山,群峰争奇,耸入天顶,就变成了积雨云。积雨云越长越高,云底慢慢变黑,云峰渐渐模糊,不一会,整座云山崩塌了,乌云弥漫了天空,顷刻间,雷声隆隆,电光闪闪,马上就会哗啦哗啦地下起暴雨,有时竟会带来冰雹或者龙卷风。
    
    我们还可以根据云上的光彩现象,推测天气的情况。在太阳和月亮的周围,有时会出现一种美丽的七彩光圈,里层是红色的,外层是紫色的。这种光圈叫做“晕”。日晕和月晕常常产生在卷层云上,卷层云后面的大片高层云和雨层云,是大风雨的征兆。所以有“日晕三更雨,月晕午时风”的说法。说明出现卷层云,并且伴有晕,天气就会变坏。另有一种比晕小的彩色光环,叫做“华”。颜色的排列是里紫外红,跟晕刚好相反。日华和月华大多产生在高积云的边缘部分。华环由小变大,天气趋向晴好。华环由大变小,天气可能转为了阴雨。夏天,雨过天晴,太阳对面的云幕上,常会挂上一条彩色的圆弧,这就是虹。人们常说:“东虹轰隆西虹雨。”意思是说,虹在东方,就有雷无雨;虹在西方,将有大雨。还有一种云彩常出现在清晨或傍晚。太阳照到天空,使云层变成红色,这种云彩叫做霞。朝霞在西,表明阴雨天气在向我们进袭;晚霞在东,表示最近几天里天气晴朗。所以有“朝霞不出门,晚霞行千里”的谚语。
    
    云,能够帮助我们识别阴晴风雨,预知天气变化,这对工农业生产有着重要的意义。我们要学会看云识天气,就要虚心向有经验的人学习,留心观察云的变化,在反复的观察中掌握规律。但是,天气变化异常复杂,看云识天气毕竟有一定的限度。要准确掌握天气变化的情况,还得依靠天气预报。
    
\end{normalsize}


\newpage

\textbf{注释}:

\vspace{-1em}

\begin{itemize}
    \setlength\itemsep{-0.2em}
    \item 〔峰峦〕连绵的山峰。
    \item 〔点缀〕以少量衬托,装饰。缀:将小块布连起来,把小物连到物件边缘。
    \item 〔阴森〕阴沉、昏暗而令人害怕的。
    \item 〔预兆〕事情发生前所显示出来的迹象。
    \item 〔弥漫〕满布,到处充斥着。
\end{itemize}

\chapter{老山界}

\begin{normalsize}
    
    我们决定要爬一座三十里高\footnote{〔三十里高〕这里指上山的路程,不是海拔高度。}的瑶山\footnote{〔瑶山〕泛指瑶族生活地区的山。},地图上叫越城岭\footnote{〔越城岭〕五岭之一,位于广西湖南交界。五岭:也称南岭,在湖南与两广交界。南岭以南称为岭南。},土名叫老山界。
    
    下午才动身,沿着山沟向上走。前面不知道为什么走不动,等了好久才走了几步,又要停下来等。队伍挤得紧紧的,站累了,就在路旁坐下来,等前头喊着“走,走,走”,就站起来再走。满望可以多走一段,可是走不了几次又要停下来。天色晚了,肚子饿了,许多人烦得叫起来,骂起来。我们偷了个空儿,跑到前面去。地势渐渐更加陡起来。我们已经超过自己的纵队\footnote{〔自己的纵队〕指“红章”纵队。“红章”“红星”纵队是长征时中央机关工作人员编成的两个纵队。作者当时在“红章”纵队政治部宣传部工作。},跑到“红星”纵队的尾巴上,恰好在转弯地方发现路旁有一间房子,我们就进去歇一下。
    
    这是一家瑶民\footnote{〔瑶民〕对瑶族人的称呼。},住着母女二人;男人大概是因为听到过队伍,照着习惯,到什么地方去躲起来了。
    
    “大嫂,借你这里歇歇脚儿。”
    
    “请到里边坐。”她带着些惊惶的神情说。队伍还是极迟慢地向前行动。我们就跟瑶民攀谈起来。照我们一路上的经验,不论是谁,不论他们开始怎样怕我们,只要我们对他们说清楚了红军是什么,没有不变忧为喜,同我们十分亲热起来的。今天对瑶民,我们也要试一试。
    
    我们谈到红军,谈到苛捐杂税,谈到广西军阀禁止瑶民信仰自己的宗教,残杀瑶民,谈到她住在这里的生活情形。那女人哭起来了。
    
    她说她原来也有过地,但是汉人把他们从自己的地上赶跑了。现在住到这荒山上来,种人家的地,每年要缴特别重的租。她说:“广西的苛捐杂税对瑶民特别重,广西军阀特别欺侮瑶民。你们红军早些来就好了,我们就不会吃这样的苦了。”
    
    她问我们饿了没有。这一问正问中了我们的心事。她拿出仅有的一点米,放在房中间木头架成的一个灶上煮粥。她对我们道歉,说没有多的米,也没有大锅,要不就多煮些给部队吃。我们给她钱,她不要。好容易来了一个认识的同志,带来一袋米,够吃三天的粮食,虽然明知道前面粮食缺乏,我们还是把这整袋子米送给她。她非常欢喜地接受了。
    
    部队今天非夜里行军不可,她的房子和篱笆都是枯竹编成的,我们生怕有人拆下来当火把点,就写了几条标语,用米汤贴在外面显眼的地方,告知我们的部队不准拆篱笆当火把。我们问了瑶民,知道前面还有竹林,可以砍来作火把,就派人到前面竹林去准备。
    
    粥吃起来十分香甜,因为确是饿了。我们也拿碗盛给瑶民母女吃。打听前面的路程,知道前面有一个地方叫雷公岩,很陡,上山三十里,下山十五里,再前面才是塘坊边。我们现在还没到山脚下呢。
    
    自己的队伍来了,我们饶了些水给大家喝。一路前进,天黑了才到山脚,果然有许多竹林。
    
    满天都是星光,火把也亮起来了。从山脚向上望,只见火把排成许多“之”字形,一直连到天上,跟星光按起来,分不出是火把还是星星。达真是我生平没见过的奇观。
    
    大家都知道这座山是怎样地陡了,不由浑身紧张,前后呼喊起来,都想努一把力,好快些翻过山去。
    
    “不要掉队呀!”
    
    “不要落后做乌龟呀!”
    
    “我们顶着天啦!”
    
    大家听了,哈哈地笑起来。
    
    在“之”字拐的路上一步一步地上去。向上看,火把在头顶上一点点排到天空;向下看,简直是绝壁,火把照着人的脸,就在脚底下。
    
    走了半天,忽然前面又走不动了。传来的话说,前面又有一段路在峭壁上,马爬不上去。又等了一点多钟,传下命令来说,就在这里睡,明天一早登山。
    
    就在这里睡觉?怎么行呢?下去到竹林里睡是不可能的。但就在路上睡么?路只有一尺来宽,半夜里一个翻身不就骨碌\footnote{〔骨碌〕指翻滚。}下去了么?而且路上的石头又非常不平,睡一晚准会疼死人。
    
    但这是没有办法的,只得裹一条毯子,横着心躺下去。因为实在太疲倦,一会儿就酣然入梦了。
    
    半夜里,忽然醒来,才觉得寒气逼人,刺入肌骨,浑身打着颤。把毯子卷得更紧些把身子蜷起来,还是睡不着。天上闪烁的星星好象黑色幕上缀着的宝石,它跟我们这样地接近哪!黑的山峰象巨人一样矗立在面前。四围的山把这山谷包围得象一口井。上边和下边有几堆火没有熄;冻醒了的同志们围着火堆小声地谈着话。除此以外,就是寂静。耳朵里有不可捉摸的声响,极远的又是极近的,极洪大的又是极细切的,像春蚕在咀嚼桑叶,像野马在平原上奔驰,像山泉在呜咽,像波涛在澎湃。不知什么时候又睡着了。
    
    黎明的时候被人推醒,说是准备出发。山下有人送饭上来,不管三七二十一,抢了一碗就吃。
    
    又传下命令来,要队伍今天无论如何爬过这座山。因为山路很难走,一路上需要督促前进。我们几个人又停下来,立刻写标语,分配人到山下山上各段去喊口号,演说,帮助病员和运输员。忙了一会,再向前进。
    
    走了不多远,看见昨晚所说的峭壁上的路,也就是所谓雷公岩的,果然陡极了,几乎是垂直的石梯,只有一尺多宽;旁边就是悬崖,虽然不很深,但也够怕人的。崖下已经聚集了很多马匹,都是昨晚不能过去、要等今天全纵队过完了再过去的。有几匹曾经从崖上跌下来,脚骨都断了。
    
    很小心地过了这个石梯。上面的路虽然还是陡,但并不陡得那么厉害了。一路走,一路检查标语。我渐渐地掉了队,顺便做些鼓动工作。
    
    这很陡的山爬完了。我以为三十里的山就是那么一点;恰巧来了一个瑶民,同他谈谈,知道还差得远,还有二十多里很陡的山。
    
    昨天的晚饭,今天的早饭,都没吃饱。肚子很饿,气力不够,但是必须鼓着勇气前进。一路上,看见以前送上去的标语用完了,就一路写着标语贴。累得走不动的时候,索性在地上躺一会儿。
    
    快要到山顶,我已经落得很远了。许多运输员都走到前头去了,剩下来的是医务人员和掩护部队。医务人员真是辛苦,因为山陡,伤员病员都下了担架走,旁边需要有人搀扶着。医务人员中的女同志们英勇得很,她们还是处处在慰问和帮助伤员病员,一点也不知道疲倦。回头向来路望去,那些小山都成了“矮子”。机关枪声很密,大概是在我们昨天出发的地方,五、八军团\footnote{〔五、八军团〕指红军第五和第八军团。长征时这两个军团负责断后。}正跟敌人开火。远远地还听见敌人飞机的叹息,大概是在叹息自己的命运:为什么不到抗日的战线上去显显身手呢?
    
    到了山顶,已经是下午两点多钟。我忽然想起:将来要在这里立个纪念碑,写上某年某月某日,红军北上抗日,路过此处。我长长地吐了一口气,坐在山顶上休息一会。回头看队伍,还没有过山的只有不多的几个人了。我们完成了任务,把一个坚强的意志灌输到整个纵队每个人心中,饥饿,疲劳甚至受伤的痛苦都被这个意志克服了。难翻的老山界被我们这样笨重的队伍战胜了。
    
    下山十五里,也是很倾斜的。我们一口气儿跑下去,跑得真快。路上有几处景致很好,浓密的树林里,银子似的泉水流下山去,清得透底。在每条溪流的旁边,有很多战士们用脸盆、饭盒子、茶缸煮粥吃。我们虽然也很饿,但仍旧一气儿跑下山去,一直到宿营地。
    
    这回翻山使部队开始养成一种新的习惯:那就是用脸盆、饭盒子、茶缸煮饭吃,煮东西吃。这种习惯一直保持了很久。
    
    老山界是我们长征中所过的第一座难走的山。但是我们走过了金沙江、大渡河、雪山、草地以后,才觉得老山界的困难,比起这些地方来,还是小得很。
    
\end{normalsize}


\newpage

\textbf{注释}:

\vspace{-1em}

\begin{itemize}
    \setlength\itemsep{-0.2em}
    \item 〔满望〕十分希望。
    \item 〔攀谈〕搭话聊天。
    \item 〔苛捐杂税〕繁重的税。捐:泛指巧立名目变相收税。
    \item 〔酣然入梦〕陷入熟睡。
    \item 〔细切〕细密。
    \item 〔呜咽〕低声哭泣。
    \item 〔澎湃〕波涛猛烈产生的巨大响声。
    \item 〔咀嚼〕在嘴里反复细细咬碎
    \item 〔搀扶〕在旁挽臂扶助。
    \item 〔景致〕景色。
\end{itemize}

\chapter{土地的誓言}

\begin{normalsize}
    
    对于广大的关东\footnote{〔关东〕旧称东北三省,以位于山海关之东而得名。亦称“关外”}原野,我心里怀着炽痛的热爱。我无时无刻不听见她呼唤我的名字,我无时无刻不听见她召唤我回去。我有时把手放在我的胸膛上,我知道我的心还是跳动的,我的心还在喷涌着热血,因为我常常感到它在泛滥着一种热情。
    
    当我躺在土地上的时候,当我仰望天上的星星,手里握着一把泥土的时候,或者当我回想起儿时的往事的时候,我想起那参天碧绿的白桦林,标直漂亮的白桦树在原野上呻吟;我看见奔流似的马群,深夜嗥鸣的蒙古狗,我听见皮鞭滚落在山涧里的脆响;我想起红布似的高粱,金黄的豆粒,黑色的土地,红玉的脸庞,黑玉的眼睛,斑斓的山雕,奔驰的鹿群,带着松香气味的煤块,带着赤色的足金;我想起幽远的车铃,晴天里马儿戴着串铃在溜直的大道上跑着,狐仙姑\footnote{〔狐仙姑〕东北地区民间传说中的神仙。}深夜的谰语\footnote{〔谰语〕没有根据的话。},原野上怪诞的狂风……这时我听到故乡在召唤我,故乡有一种声音在召唤着我。她低低地呼唤着我的名字,声音是那样的急切,使我不得不回去。
    
    我总是被这种声音所缠绕,不管我走到哪里,即使我睡得很沉,或者在睡梦中突然惊醒的时候,我都会突然想到是我应该回去的时候了。我必须回去,我从来没想过离开她。这种声音是不可阻止的,是不能选择的。这种声音已经和我的心取得了永远的沟通。当我记起故乡的时候,我便能看见那大地的深层,在翻滚着一种红熟的浆液,这声音便是从那里来的。在那亘古的地层里,有着一股燃烧的洪流,像我的心喷涌着血液一样。这个我是知道的,我常常把手放在大地上,我会感到她在跳跃,和我的心的跳跃是一样的。它们从来没有停息,它们的热血一直在流,在热情的默契里它们彼此呼唤着,终有一天它们要汇合在一起。
    
    土地是我的母亲,我的每一寸皮肤,都有着土粒;我的手掌一接近土地,心就变得平静。我是土地的族系\footnote{〔族系〕家族的世系。有密切关联的同类。},我不能离开她。在故乡的土地上,我印下我无数的脚印。在那田垄里埋葬过我 的欢笑,在那稻棵上我捉过蚱蜢,在那沉重的镐头\footnote{〔镐头〕刨土用的工具。}上留着我的手印。我吃过我自己种的白菜。故乡的土壤是香的。在春天,东风吹起的时候,土壤的香气便在田野里飘扬。河流浅浅地流过,柳条像一阵烟雨似的窜出来,空气里都有一种欢喜的声音。原野到处有一种鸣叫,天空清亮透明,劳动的声音从这头响到那头。秋天,银线似的蛛丝在牛角上挂着,粮车拉粮回来,麻雀吃厌了,这里那里到处飞。稻禾的香气是强烈的,碾着新谷的场院辘辘地响着,多么美丽,多么丰饶……没有人能够忘记她。我必定为她而战斗到底。
    
    土地,原野,我的家乡,你必须被解放!你必须站立!夜夜我听见马蹄奔驰的声音,草原的儿子在黎明的天边呼唤。这时我起来,找寻天空中北方的大熊\footnote{〔大熊〕指大熊星座。},在它金色的光芒之下,乃是我的家乡。我向那边注视着,注视着,直到天边破晓。我永不能忘记,因为我答应过她,我要回到她的身边,我答应过我一定会回去。为了她,我愿付出一切。我必须看见一个更美丽的故乡出现在我的面前——或者我的坟前。而我将用我的泪水,洗去她一切的污秽和耻辱。
    
    \hfill “九一八”十周年写
    
\end{normalsize}


\newpage

\textbf{注释}:

\vspace{-1em}

\begin{itemize}
    \setlength\itemsep{-0.2em}
    \item 〔炽痛〕烧得疼痛。
    \item 〔标直〕笔直。
    \item 〔嗥鸣〕犬类叫。
    \item 〔斑斓〕色彩错杂灿烂。
    \item 〔亘古〕远古。
    \item 〔污秽〕脏东西。
\end{itemize}

\chapter{天上的街市}

\begin{normalsize}
    
    \begin{verse}[0.5\linewidth]
        远远的街灯明了 \\
        好像闪着无数的明星 \\
        天上的星星现了 \\
        好像点着无数的明灯
    \end{verse}
    
    
    \begin{verse}[0.5\linewidth]
        我想那缥缈的空中 \\
        定然有美丽繁华的街道 \\
        道旁橱窗里陈列的 \\
        定然是世间没有的珍宝
    \end{verse}
    
    
    \begin{verse}[0.5\linewidth]
        你看那浅浅的天河 \\
        定然是不甚宽广 \\
        那隔河相望的人儿 \\
        定能骑着牛来往
    \end{verse}
    
    
    \begin{verse}[0.5\linewidth]
        我想他们此刻 \\
        定然在天街闲游 \\
        不信,请看那朵流星 \\
        是他们提着灯笼在走
    \end{verse}
    
\end{normalsize}


\newpage

\textbf{注释}:

\vspace{-1em}

\begin{itemize}
    \setlength\itemsep{-0.2em}
    \item 〔缥缈〕隐隐约约,若有若无。遥不可及。
    \item 〔橱窗〕商店临街的玻璃窗,用来展览样品。
\end{itemize}

\chapter{未选择的路}

\begin{normalsize}
    
    \begin{verse}[0.5\linewidth]
        黄叶林里分出了两条路, \\
        可惜我不能同时去涉足。 \\
        在那路口我伫立无言, \\
        向其中一条极目望去, \\
        直到它消失在树丛深处。
    \end{verse}
    
    
    \begin{verse}[0.5\linewidth]
        但我选了另外一条路, \\
        路上十分幽寂,荒草萋萋, \\
        显得更加诱人、更加美丽。 \\
        其实两条路并无差别, \\
        都很少有行人的足迹。
    \end{verse}
    
    
    \begin{verse}[0.5\linewidth]
        虽然那天清晨落叶满地, \\
        虽然两条路都罕有人迹。 \\
        第一条路,改日再见吧! \\
        但我知道长路无尽头, \\
        一去之后重来谈何容易。
    \end{verse}
    
    
    \begin{verse}[0.5\linewidth]
        也许有一天,某处的我, \\
        将轻声叹息把往事回顾。 \\
        黄叶林里分出了两条路, \\
        我选了更少人走过的, \\
        而踏上了不同的旅途。
    \end{verse}
    
\end{normalsize}


\newpage

\textbf{注释}:

\vspace{-1em}

\begin{itemize}
    \setlength\itemsep{-0.2em}
    \item 〔伫立〕久立,长时间地站着。
    \item 〔极目〕尽力远望。
\end{itemize}

\chapter{驿路梨花}

\begin{normalsize}
    
    山,好大的山啊!起伏的青山一座挨一座,延伸到远方,消失在迷茫的暮色中。
    
    这是哀牢山\footnote{〔哀牢山〕山名,在云南省南部,元江和阿墨江的分水岭,云岭南延分支之一。}南段的最高处。这么陡峭的山,这么茂密的树林,走上一天,路上也难得遇见几个人。夕阳西下,我们有点着急了,今夜要是赶不到山那边的太阳寨,只有在这深山中露宿了。
    
    同行老余是在边境地区生活过多年的人。正走着,他突然指着前面叫了起来:“看,梨花!”
    
    白色梨花开满枝头,多么美丽的一片梨树林啊!
    
    老余说:“这里有梨树,前边就会有人家。”
    
    一弯新月升起了,我们借助淡淡的月光,在忽明忽暗的梨树林里走着。山间的夜风吹得人脸上凉凉的,梨花的白色花瓣轻轻飘落在我们身上。
    
    “快看,有人家了。”
    
    一座草顶、竹篾泥墙的小屋出现在梨树林边。屋里漆黑,没有灯也没有人声。这是什么人的房子呢?
    
    老余打着电筒走过去,发现门是从外扣着的。白水门板上用黑炭写着两个字:“请进!”
    
    我们推开门进去。火塘\footnote{〔火塘〕室内地上挖的小坑,四周垒上砖石,中间生火取暖。}里的灰是冷的,显然,好多天没人住过了。一张简陋的大竹床铺着厚厚的稻草。倚在墙边的大竹筒里装满了水,我尝了一口,水清凉可口。我们走累了,决定在这里过夜。
    
    老余用电筒在屋里上上下下扫射了一圈,又发现墙上写着几行粗大的字:“屋后边有干柴,梁上竹筒里有米,有盐巴,有辣子。”
    
    我们开始烧火做饭。温暖的火、喷香的米饭和滚热的洗脚水,把我们身上的疲劳、饥饿都撵走了。我们躺在软软的干草铺上,对小茅屋的主人有说不尽的感激。我问老余:“你猜这家主人是干什么的?”老余说:“可能是一位守山护林的老人。”
    
    正说着,门被推开了。一个须眉花白的瑶族老人站在门前,手里提着一杆明火枪,肩上打着一袋米。
    
    “主人”回来了。我和老余同时抓住老人的手,抢着说感谢的话;老人眼睛瞪得大大的,几次想说话插不上嘴。直到我们不作声了,老人才笑道:“我不是主人,也是过路人呢!”
    
    我们把老人请到火塘前坐下,看他也是又累又饿,赶紧给他端来了热水、热饭。老人笑了笑:“多谢,多谢,说了半天还得多谢你们。”
    
    看来他是个很有穿山走林经验的人。吃完饭,他燃起一袋旱烟笑着说:“我是给主人家送粮食来的。”
    
    “主人家是谁?”
    
    “不晓得。”
    
    “粮食交给谁呢?”
    
    “挂在屋梁上。”
    
    “老人家,你真会开玩笑。”
    
    他悠闲地吐着烟,说:“我不是开玩笑。”停了一会,又接着说:“我是红河\footnote{〔红河〕中南半岛大河,发源于云南省西部。}边上过山岩的瑶家\footnote{〔瑶家〕指瑶族人。},平常爱打猎。上个月,我追赶一群麂子\footnote{〔麂子〕一种小型鹿类动物,腿细而有力,善于跳跃。},在老林里东转西转迷失了方向,不知怎么插到这个山头来了。那时候,人走累了,干粮也吃完了,想找个寨子歇歇,偏偏这一带没有人家。我正失望的时候,突然看到了这片梨花林和这小屋,屋里有柴、有米、有水,就是没有主人。吃了用了人家的东西,不说清楚还行?我只好撕了片头巾上的红布、插了根羽毛在门上,告诉主人,有个瑶家人来打扰了,过几天再来道谢……”
    
    说到这里,他用手指了指门背后:“你们看,那东西还在呢!”
    
    一根白羽毛钉在红布上,红白相衬很好看。老人家说到这里,停了一会,又接着说下去:“我到处打听小茅屋的主人是哪个,好不容易才从一个赶马人那里知道个大概,原来对门山头上有个名叫梨花的哈尼\footnote{〔哈尼〕哈尼族,我国少数民族之一,主要居住在云南省红河哈尼族彝族自治州等几个州县。}小姑娘,她说这大山坡上,前不着村后不挨寨,她要用为人民服务的精神来帮助过路人。”
    
    我们这才明白,屋里的米、水、干柴,以及那充满了热情的“请进”二字,都是出自那哈尼小姑娘的手。多好的梨花啊!
    
    瑶族老人又说:“过路人受到照料,都很感激,也都尽力把用了的柴、米补上,好让后来人方便。我这次是专门送粮食来的。”
    
    这天夜里,我睡得十分香甜,梦中恍惚在那香气四溢的梨花林里漫步,还看见一个身穿着花衫的哈尼小姑娘在梨花丛中歌唱……
    
    第二天早上,我们没有立即上路,老人也没有离开,我们决定把小茅屋修葺一下,给屋顶加点草,把房前屋后的排水沟再挖深一些。一个哈尼小姑娘都能为群众着想,我们真应该向她学习。
    
    我们正在劳动,突然梨树丛中闪出了一群哈尼小姑娘。走在前边的约莫十四五岁,红润的脸上有两道弯弯的修长的眉毛和一对晶莹的大眼睛。我想:“她一定是梨花。”
    
    瑶族老人立即走到她们面前,深深弯下腰去,行了个大礼,吓得小姑娘们像小雀似的蹦开了,接着就哈哈大笑起来:“老爷爷,你给我们行这样大的礼,不怕折损我们吗?”老人严肃地说:“我感谢你们盖了这间小草房。”
    
    为头的那个小姑娘赶紧插手:“不要谢我们!不要谢我们!房子是解放军叔叔盖的。”
    
    接着,小姑娘向我们讲述了房子的来历。十多年前,有一队解放军路过这里,在树林里过夜,半夜淋了大雨。他们想,这里要有一间给过路人避风雨的小屋就好了,第二天早上就砍树割草盖起了房子。她姐姐恰好过这边山上来拾菌子\footnote{〔菌子〕指蘑菇。},好奇地问解放军叔叔:“你们要在这里长住?”解放军说:“不,我们是为了方便过路人。是雷锋同志教我们这样做的。”她姐姐很受感动。从那以后,常常趁砍柴、拾菌子、找草药的机会来照料这小茅屋。
    
    原来她还不是梨花。我问:“梨花呢?”
    
    “前几年出嫁到山那边了。”
    
    不用说,姐姐出嫁后,是小姑娘接过任务,常来照管这小茅屋。
    
    我望着这群充满朝气的哈尼小姑娘和那洁白的梨花,不由得想起了一句诗:“驿路梨花处处开\footnote{〔驿路梨花处处开〕出自宋代陆游《闻武均州报已复西京》。驿路:又叫驿道。古时传递公文的公路,沿途有换马或供休息的驿站。}”。
    
\end{normalsize}


\newpage

\textbf{注释}:

\vspace{-1em}

\begin{itemize}
    \setlength\itemsep{-0.2em}
    \item 〔竹篾〕劈成薄片的竹条。
    \item 〔寨子〕防守用的栅栏。有围栏的村子。
    \item 〔修葺〕修理(建筑物)。葺:修理,修建。
\end{itemize}

\chapter{假如生活欺骗了你}

\begin{normalsize}
    
    \begin{verse}[0.5\linewidth]
        假如生活欺骗了你, \\
        不要悲伤,不要心急! \\
        忧郁的日子里须要镇静: \\
        相信吧,快乐的日子将会来临。
    \end{verse}
    
    
    \begin{verse}[0.5\linewidth]
        心儿永远向往着未来; \\
        现在却常是忧郁: \\
        一切都是瞬息,一切都将会过去; \\
        而那过去了的,就会成为亲切的怀恋。
    \end{verse}
    
\end{normalsize}


\newpage

\textbf{注释}:

\vspace{-1em}

\begin{itemize}
    \setlength\itemsep{-0.2em}
    \item 〔瞬息〕极短的时间。
    \item 〔怀恋〕怀念热爱。
\end{itemize}

\chapter{繁星}

\begin{normalsize}
    
    我爱月夜,但我也爱星天。从前在家乡七、八月的夜晚在庭院里纳凉的时候,我最爱看天上密密麻麻的繁星。望着星天,我就会忘记一切,仿佛回到了母亲的怀里似的。
    
    三年前在南京我住的地方有一道后门,每晚我打开后门,便看见一个静寂的夜。下面是一片菜园,上面是星群密布的蓝天。星光在我们的肉眼里虽然微小,然而它使我们觉得光明无处不在。那时候我正在读一些关于天文学的书,也认得一些星星,好像它们就是我的朋友,它们常常在和我谈话一样。
    
    如今在海上,每晚和繁星相对,我把它们认得很熟了。我躺在舱面上,仰望天空。深蓝色的天空里悬着无数半明半昧的星。船在动,星也在动,它们是这样低,真是摇摇欲坠呢!渐渐地我的眼睛模糊了,我好像看见无数萤火虫在我的周围飞舞。海上的夜是柔和的,是静寂的,是梦幻的。我望着那许多认识的星,我仿佛看见它们在对我霎眼,我仿佛听见它们在小声说话。这时我忘记了一切。在星的怀抱中我微笑着,我沉睡着。我觉得自己是一个小孩子,现在睡在母亲的怀里了。
    
\end{normalsize}


\newpage

\textbf{注释}:

\vspace{-1em}

\begin{itemize}
    \setlength\itemsep{-0.2em}
    \item 〔纳凉〕为避热而在阴凉处歇息。
    \item 〔昧〕暗,不明。
\end{itemize}

\chapter{静寂的园子}

\begin{normalsize}
    
    没有听见房东家的狗的声音。现在园子里非常静。那棵不知名的五瓣的白色小花仍然寂寞地开着。阳光照在松枝和盆中的花树上,给那些绿叶涂上金黄色。天是晴朗的,我不用抬起眼睛就知道头上是晴空万里。
    
    忽然我听见洋铁\footnote{〔洋铁〕镀锡或镀锌的铁皮。}瓦沟上有铃子响声,抬起头,看见两只松鼠正从瓦上溜下来,这两只小生物在松枝上互相追逐取乐。它们的绒线球似的大尾巴,它们的可爱的小黑眼睛,它们颈项上的小铃子吸引了我的注意。我索性不转睛地望着窗外。但是它们跑了两三转,又从藤萝\footnote{〔藤萝〕紫藤萝。}架回到屋瓦上,一瞬间就消失了,依旧把这个静寂的园子留给我。
    
    我刚刚埋下头,又听见小鸟的叫声。我再看,桂树枝上立着一只青灰色的白头小鸟,昂起头得意地歌唱。屋顶的电灯线上,还有一对麻雀在吱吱喳喳地讲话。
    
    我不了解这样的语言。但是我在鸟声里听出了一种安闲的快乐。它们要告诉我的一定是它们的喜悦的感情。可惜我不能回答它们。我把手一挥,它们就飞走了。我的话不能使它们留住,它们留给我一个园子的静寂。不过我知道它们过一阵又会回来的。
    
    现在我觉得我是这个园子里唯一的生物了。我坐在书桌前俯下头写字,没有一点声音来打扰我。我正可以把整个心放在纸上。但是我渐渐地烦躁起来。这静寂像一只手慢慢地挨近我的咽喉。我感到呼吸不畅快了。这是不自然的静寂。这是一种灾祸的预兆,就像暴雨到来前那种沉闷静止的空气一样。
    
    我似乎在等待什么东西。我有一种不安定的感觉,我不能够静下心来。我一定是在等待什么东西。我在等待空袭警报;或者我在等待房东家的狗吠声,这就是说,预行警报已经解除,不会有空袭警报响起来,我用不着准备听见凄厉的汽笛声\footnote{〔汽笛声〕指空袭警报。}就锁门出去。近半月来晴天有警报差不多成了常例。
    
    可是我的等待并没有结果。小鸟回来后又走了;松鼠们也来过一次,但又追逐地跑上屋顶,我不知道它们消失在什么地方。从我看不见的正面楼房屋顶上送过来一阵的乌鸦叫。这些小生物不知道人间的事情,它们不会带给我什么信息。
    
    我写到上面的一段,空袭警报就响了。我的等待果然没有落空。这时我觉得空气在动了。我听见巷外大街上汽车的叫声。我又听见飞机的发动机声,这大概是民航机飞出去躲警报。有时我们的驱逐机\footnote{〔驱逐机〕歼击机。}也会在这种时候排队飞出,等着攻击敌机。我不能再写了,便拿了一本书锁上园门,匆匆地走到外面去。
    
    在城门口经过一阵可怕的拥挤后,我终于到了郊外。在那里耽搁了两个多钟头,和几个朋友在一起,还在草地上吃了他们带出去的午餐。警报解除后,我回来,打开锁,推开园门,迎面扑来的仍然是一个园子的静寂。
    
    我回到房间,回到书桌前面,打开玻璃窗,在继续执笔前还看看窗外。树上,地上,满个园子都是阳光。墙角一丛观音竹微微地在飘动它们的尖叶。一只大苍蝇带着嗡嗡声从开着的窗飞进房来,在我的头上盘旋。一两只乌鸦在我看不见的地方叫。一只黄色小蝴蝶在白色小花间飞舞。忽然一阵奇怪的声音在对面屋瓦上响起来,又是那两只松鼠从高墙沿着洋铁滴水管溜下来。它们跑到那个支持松树的木架上,又跑到架子脚边有假山的水池的石栏杆下,在那里追逐了一回,又沿着木架跑上松枝,隐在松叶后面了。松叶动起来,桂树的小枝也动了,一只绿色小鸟刚刚歇在那上面。
    
    狗的声音还是听不见。我向右侧着身子去看那条没有阳光的窄小过道。房东家的小门紧紧地闭着。这些时候那里就没有一点声音。大概这家人大清早就到城外躲警报去了,现在还不曾回来。他们回来恐怕在太阳落坡的时候。那条肥壮的黄狗一定也跟着他们“疏散”了,否则会有狗抓门的声音送进我的耳里来。
    
    我又坐在窗前写了这许多字。还是只有乌鸦和小鸟的叫声陪伴我。苍蝇的嗡嗡声早已寂灭了。现在在屋角又响起了老鼠啃东西的声音。都是响一回又静一回的,在这个受着轰炸威胁的城市里我感到了寂寞。
    
    然而像一把刀要划破万里晴空似的,嘹亮的机声突然响起来。这是我们自己的飞机。声音多么雄壮,它扫除了这个园子的静寂。我要放下笔到庭院中去看天空,看那些背负着金色阳光在蓝空里闪耀的灰色大蜻蜓。那是多么美丽的景象。
    
    \hfill 1940年10月11日在昆明
    
\end{normalsize}


\newpage

\textbf{注释}:

\vspace{-1em}

\begin{itemize}
    \setlength\itemsep{-0.2em}
    \item 〔耽搁〕停留。
\end{itemize}

\chapter{社戏}

\begin{normalsize}
    
    我们鲁镇的习惯,本来是凡有出嫁的女儿,倘自己还未当家,夏间便大抵回到母家去消夏\footnote{〔消夏〕过夏天。}。那时我的祖母虽然还康健,但母亲也已分担了些家务,所以夏期便不能多日的归省\footnote{〔归省〕回家探望(父母)。}了,只得在扫墓完毕之后,抽空去住几天,这时我便每年跟了我的母亲住在外祖母的家里。那地方叫平桥村,是一个离海边不远,极偏僻的,临河的小村庄;住户不满三十家,都种田,打鱼,只有一家很小的杂货店。但在我是乐土:因为我在这里不但得到优待,又可以免念“秩秩斯干幽幽南山\footnote{〔秩秩斯干幽幽南山〕《诗经·小雅·斯干》头两句。这里泛指难懂的古书。}”了。
    
    和我一同玩的是许多小朋友,因为有了远客,他们也都从父母那里得了减少工作的许可,伴我来游戏。在小村里,一家的客,几乎也就是公共的。我们年纪都相仿,但论起行辈\footnote{〔行辈〕辈分。}12来,却至少是叔子,有几个还是太公,因为他们合村都同姓,是本家。然而我们是朋友,即使偶尔吵闹起来,打了太公,一村的老老少少,也决没有一个会想出“犯上”这两个字来,而他们也百分之九十九不识字。
    
    我们每天的事情大概是掘蚯蚓,掘来穿在铜丝做的小钩上,伏在河沿上去钓虾。虾是水世界里的呆子,决不惮用了自己的两个钳捧着钩尖送到嘴里去的,所以不半天便可以钓到一大碗。这虾照例是归我吃的。其次便是一同去放牛,但或者因为高等动物了的缘故罢,黄牛水牛都欺生\footnote{〔欺生〕欺负新来的人。},敢于欺侮我,因此我也总不敢走近身,只好远远地跟着,站着。这时候,小朋友们便不再原谅我会读“秩秩斯干”,却全都嘲笑起来了。
    
    至于我在那里所第一盼望的,却在到赵庄去看戏。赵庄是离平桥村五里的较大的村庄;平桥村太小,自己演不起戏,每年总付给赵庄多少钱,算作合做的。当时我并不想到他们为什么年年要演戏。现在想,那或者是春赛\footnote{〔春赛〕春季酬谢神灵的祭礼。},是社戏\footnote{〔社戏〕农村一些地方春秋酬神祈福的戏,一般在庙台或野外搭台演出。在绍兴,社是一种区划名称,社戏就是每个社中每年所演的“年规戏”。}了。
    
    就在我十一二岁时候的这一年,这日期也看看等到了。不料这一年真可惜,在早上就叫不到船。平桥村只有一只早出晚归的航船\footnote{〔航船〕江浙一带定期往来城镇间的载客运货的木船。}是大船,决没有留用的道理。其余的都是小船,不合用;央人到邻村去问,也没有,早都给别人定下了。外祖母很气恼,怪家里的人不早定,絮叨起来。母亲便宽慰伊\footnote{〔伊〕她。早期白话文用法。},说我们鲁镇的戏比小村里的好得多,一年看几回,今天就算了。只有我急得要哭,母亲却竭力的嘱咐我,说万不能装模装样,怕又招外祖母生气,又不准和别人一同去,说是怕外祖母要担心。
    
    总之,是完了。到下午,我的朋友都去了,戏已经开场了,我似乎听到锣鼓的声音,而且知道他们在戏台下买豆浆喝。
    
    这一天我不钓虾,东西也少吃。母亲很为难,没有法子想。到晚饭时候,外祖母也终于觉察了,并且说我应当不高兴,他们太怠慢,是待客的礼数里从来没有的。吃饭之后,看过戏的少年们也都聚拢来了,高高兴兴的来讲戏。只有我不开口;他们都叹息而且表同情。忽然间,一个最聪明的双喜大悟似的提议了,他说,“大船?八叔的航船不是回来了么?”十几个别的少年也大悟,立刻撺掇起来,说可以坐了这航船和我一同去。我高兴了。然而外祖母又怕都是孩子,不可靠;母亲又说是若叫大人一同去,他们白天全有工作,要他熬夜,是不合情理的。在这迟疑之中,双喜可又看出底细来了,便又大声的说道,“我写包票\footnote{〔写包票〕打包票,担保。}!船又大;迅哥儿向来不乱跑;我们又都是识水性的!”
    
    诚然\footnote{〔诚然〕确实,真的。}!这十多个少年,委实没有一个不会凫水的,而且两三个还是弄潮\footnote{〔弄潮〕在潮水里搏击嬉戏。}的好手。
    
    外祖母和母亲也相信,便不再驳回,都微笑了。我们立刻一哄的出了门。
    
    我的很重的心忽而轻松了,身体也似乎舒展到说不出的大。一出门,便望见月下的平桥内泊着一只白篷的航船,大家跳下船,双喜拔\footnote{〔拔〕拨。}前篙,阿发拔后篙,年幼的都陪我坐在舱中,较大的聚在船尾。母亲送出来吩咐“要小心”的时候,我们已经点开船,在桥石上一磕,退后几尺,即又上前出了桥。于是架起两支橹,一支两人,一里一换,有说笑的,有嚷的,夹着潺潺的船头激水的声音,在左右都是碧绿的豆麦田地的河流中,飞一般径向赵庄前进了。
    
    两岸的豆麦和河底的水草所发散出来的清香,夹杂在水气中扑面的吹来;月色便朦胧在这水气里。淡黑的起伏的连山,仿佛是踊跃的铁的兽脊似的,都远远地向船尾跑去了,但我却还以为船慢。他们换了四回手,渐望见依稀的赵庄,而且似乎听到歌吹\footnote{〔歌吹〕歌声和乐器声。}了,还有几点火,料想便是戏台,但或者也许是渔火。
    
    那声音大概是横笛,宛转,悠扬,使我的心也沉静,然而又自失\footnote{〔自失〕失去自我。出神而忘记了自身存在。}起来,觉得要和他弥散在含着豆麦蕴藻\footnote{〔蕴藻〕一种水草,也叫蕰藻。}之香的夜气里。
    
    那火接近了,果然是渔火;我才记得先前望见的也不是赵庄。那是正对船头的一丛松柏林,我去年也曾经去游玩过,还看见破的石马倒在地下,一个石羊蹲在草里呢。过了那林,船便弯进了汊港,于是赵庄便真在眼前了。
    
    最惹眼的是屹立在庄外临河的空地上的一座戏台,模糊在远处的月夜中,和空间几乎分不出界限,我疑心画上见过的仙境,就在这里出现了。这时船走得更快,不多时,在台上显出人物来,红红绿绿的动,近台的河里一望乌黑的是看戏的人家的船篷。
    
    “近台没有什么空了,我们远远的看罢。”阿发说。
    
    这时船慢了,不久就到,果然近不得台旁,大家只能下了篙,比那正对戏台的神棚\footnote{〔神棚〕为演戏搭的供奉神的棚子。}还要远。其实我们这白篷的航船,本也不愿意和乌篷的船在一处,而况并没有空地呢……
    
    在停船的匆忙中,看见台上有一个黑的长胡子的背上插着四张旗,捏着长枪,和一群赤膊的人正打仗。双喜说,那就是有名的铁头老生\footnote{〔老生〕戏曲角色行当,扮演中老年男性,多为正面人物。},能连翻八十四个筋斗,他日\footnote{〔他日〕另一天。}里亲自数过的。
    
    我们便都挤在船头上看打仗,但那铁头老生却又并不翻筋斗,只有几个赤膊的人翻,翻了一阵,都进去了,接着走出一个小旦\footnote{〔小旦〕戏曲角色行当,扮演年轻女子。}来,咿咿呀呀的唱。双喜说,“晚上看客少,铁头老生也懈了,谁肯显本领给白地\footnote{〔白地〕空地。}看呢?”我相信这话对,因为其时台下已经不很有人,乡下人为了明天的工作,熬不得夜,早都睡觉去了,疏疏朗朗的站着的不过是几十个本村和邻村的闲汉。乌篷船里的那些土财主的家眷固然在,然而他们也不在乎看戏,多半是专到戏台下来吃糕饼水果和瓜子的。所以简直可以算白地。
    
    然而我的意思却也并不在乎看翻筋斗。我最愿意看的是一个人蒙了白布,两手在头上捧着一支棒似的蛇头的蛇精,其次是套了黄布衣跳老虎。但是等了许多时都不见,小旦虽然进去了,立刻又出来了一个很老的小生\footnote{〔小生〕戏曲角色行当,扮演年轻男子。}。我有些疲倦了,托桂生买豆浆去。他去了一刻,回来说:“没有。卖豆浆的聋子也回去了。日里倒有,我还喝了两碗呢。现在去舀一瓢水来给你喝罢。”
    
    我不喝水,支撑着仍然看,也说不出见了些什么,只觉得戏子\footnote{〔戏子〕旧称职业戏曲演员,有轻蔑意味。}的脸都渐渐的有些稀奇\footnote{〔稀奇〕这里指怪异。}了,那五官渐不明显,似乎融成一片的再没有什么高低。年纪小的几个多打呵欠了,大的也各管自己谈话。忽而一个红衫的小丑被绑在台柱子上,给一个花白胡子的用马鞭打起来了,大家才又振作精神的笑着看。在这一夜里,我以为这实在要算是最好的一折\footnote{〔一折〕一场戏。}。
    
    然而老旦\footnote{〔老旦〕戏曲角色行当,扮演年老女子。}终于出台了。老旦本来是我所最怕的东西,尤其是怕他坐下了唱。这时候,看见大家也都很扫兴,才知道他们的意见是和我一致的。那老旦当初还只是踱来踱去的唱,后来竟在中间的一把交椅上坐下了。我很担心;双喜他们却就破口喃喃的骂。我忍耐的等着,许多工夫,只见那老旦将手一抬,我以为就要站起来了,不料他却又慢慢的放下在原地方,仍旧唱。全船里几个人不住的吁气,其余的也打起哈欠来。双喜终于熬不住了,说道,怕他会唱到天明还不完,还是我们走的好罢。大家立刻都赞成,和开船时候一样踊跃,三四人径奔船尾,拔了篙,点退几丈,回转船头,驾起橹,骂着老旦,又向那松柏林前进了。
    
    月还没有落,仿佛看戏也并不很久似的,而一离赵庄,月光又显得格外的皎洁。回望戏台在灯火光中,却又如初来未到时候一般,又缥缈得像一座仙山楼阁,满被红霞罩着了。吹到耳边来的又是横笛,很悠扬;我疑心老旦已经进去了,但也不好意思说再回去看。
    
    不多久,松柏林早在船后了,船行也并不慢,但周围的黑暗只是浓,可知已经到了深夜。他们一面议论着戏子,或骂,或笑,一面加紧的摇船。这一次船头的激水声更其响亮了,那航船,就像一条大白鱼背着一群孩子在浪花里蹿,连夜渔\footnote{〔夜渔〕夜间捕鱼。}的几个老渔父,也停了艇子看着喝彩起来。
    
    离平桥村还有一里模样,船行却慢了,摇船的都说很疲乏,因为太用力,而且许久没有东西吃。这回想出来的是桂生,说是罗汉豆\footnote{〔罗汉豆〕蚕豆的别称。}正旺相\footnote{〔旺相〕旺盛。},柴火又现成,我们可以偷一点来煮吃。大家都赞成,立刻近岸停了船;岸上的田里,乌油油的便都是结实的罗汉豆。
    
    “呵呵,阿发,这边是你家的,这边是老六一家的,我们偷哪一边的呢?”双喜先跳下去了,在岸上说。
    
    我们也都跳上岸。阿发一面跳,一面说道,“且慢,让我来看一看罢,”他于是往来的摸了一回,直起身来说道,“偷我们的罢,我们的大得多呢。”一声答应,大家便散开在阿发家的豆田里,各摘了一大捧,抛入船舱中。双喜以为再多偷,倘给阿发的娘知道是要哭骂的,于是各人便到六一公公的田里又各偷了一大捧。
    
    我们中间几个年长的仍然慢慢的摇着船,几个到后舱去生火,年幼的和我都剥豆。不久豆熟了,便任凭航船浮在水面上,都围起来用手撮着吃。吃完豆,又开船,一面洗器具,豆荚豆壳全抛在河水里,什么痕迹也没有了。双喜所虑的是用了八公公船上的盐和柴,这老头子很细心,一定要知道,会骂的。然而大家议论之后,归结是不怕。他如果骂,我们便要他归还去年在岸边拾去的一枝枯桕树\footnote{〔桕树〕即乌桕,也叫木油树。落叶乔木。},而且当面叫他“八癞子”。
    
    “都回来了!哪里会错。我原说过写包票的!”双喜在船头上忽而大声的说。
    
    我向船头一望,前面已经是平桥。桥脚上站着一个人,却是我的母亲,双喜便是对伊说着话。我走出前舱去,船也就进了平桥了,停了船,我们纷纷都上岸。母亲颇有些生气,说是过了三更了,怎么回来得这样迟,但也就高兴了,笑着邀大家去吃炒米。
    
    大家都说已经吃了点心,又渴睡\footnote{〔渴睡〕想睡觉,困倦。},不如及早睡的好,各自回去了。
    
    第二天,我向午才起来,并没有听到什么关系八公公盐柴事件的纠葛,下午仍然去钓虾。
    
    “双喜,你们这班小鬼,昨天偷了我的豆了罢?又不肯好好的摘,踏坏了不少。”我抬头看时,是六一公公棹着小船,卖了豆回来了,船肚里还有剩下的一堆豆。
    
    “是的。我们请客。我们当初还不要你的呢。你看,你把我的虾吓跑了!”双喜说。
    
    六一公公看见我,便停了楫,笑道,“请客?——这是应该的。”于是对我说,“迅哥儿,昨天的戏可好么?”
    
    我点一点头,说道,“好。”
    
    “豆可中吃呢?”
    
    我又点一点头,说道,“很好。”
    
    不料六一公公竟非常感激起来,将大拇指一翘,得意的说道,“这真是大市镇里出来的读过书的人才识货!我的豆种是粒粒挑选过的,乡下人不识好歹,还说我的豆比不上别人的呢。我今天也要送些给我们的姑奶奶\footnote{〔姑奶奶〕娘家称已经出嫁的女儿。这里指鲁迅母亲。}尝尝去……”他于是打着楫子过去了。
    
    待到母亲叫我回去吃晚饭的时候,桌上便有一大碗煮熟了的罗汉豆,就是六一公公送给母亲和我吃的。听说他还对母亲极口夸奖我,说“小小年纪便有见识,将来一定要中状元。姑奶奶,你的福气是可以写包票的了”。但我吃了豆,却并没有昨夜的豆那么好。
    
    真的,一直到现在,我实在再没有吃到那夜似的好豆,——也不再看到那夜似的好戏了。
    
    \hfill 一九二二年十月
    
\end{normalsize}


\newpage

\textbf{注释}:

\vspace{-1em}

\begin{itemize}
    \setlength\itemsep{-0.2em}
    \item 〔礼数〕礼节。
    \item 〔撺掇〕怂恿,从旁鼓动。
    \item 〔凫水〕游水。
    \item 〔渔火〕渔船上的灯火。
    \item 〔弥散〕扩散,满布。
    \item 〔汊港〕与河流连通的小河道。
    \item 〔家眷〕妻子和儿女。
    \item 〔交椅〕有靠背,椅脚交叉能折叠的椅子。
    \item 〔喃喃〕连续不断的低语声。
    \item 〔皎洁〕明亮洁白。
    \item 〔喝彩〕博彩时呼喝作势,希望中彩,后来表示大声叫好。
    \item 〔撮〕用两三个指头捉取。
    \item 〔向午〕临近中午。
    \item 〔纠葛〕纠缠的葛蔓。比喻纠缠不清的事情。
    \item 〔篙〕行船工具。用长竹竿撑到水底使船前进。也写作“槁”。
    \item 〔橹〕一种桨。中部定在船尾,摇动上柄,下片拨水使船行进。
    \item 〔棹〕一种桨。中部定在船边,摇动上柄,下片拨水使船行进。
    \item 〔楫〕楫、棹都是桨。短的叫楫,长的叫棹。
    \item 〔状元〕科举考试第一名。
\end{itemize}

\chapter{我的老师}

\begin{normalsize}
    
    最使我难忘的,是我小学时候的女教师蔡芸芝先生。
    
    回想起来,她那时有十八九岁。嘴角右边有榆钱\footnote{〔榆钱〕榆树的果实。形状像铜钱,因此叫榆钱。}大小一块黑痣。在我的记忆里,她是一个温柔和美丽的人。
    
    她从来不打骂我们。仅仅有一次,她的教鞭好像要落下来,我用石板\footnote{〔石板〕学习用具,可以用石笔在上面写字。}一迎,教鞭轻轻地敲在石板边上,大伙笑了,她也笑了。我用儿童的狡猾的眼光察觉,她爱我们,并没有存心要打的意思。
    
    在课外的时候,她教我们跳舞,我还记得她把我扮成女孩子表演跳舞的情景。
    
    在假日里,她把我们带到她的家里和朋友的家里。在她的朋友的园子里,她还让我们观察蜜蜂,也是在那时候,我认识了蜂王,并且平生第一次吃了蜂蜜。
    
    她爱诗,并且爱用歌唱的音调教我们读诗。直到现在我还记得她读诗的音调,还能背诵她教我们的诗:
    
    \begin{verse}[0.5\linewidth]
    
    圆天盖着大海,\\
    
    黑水托着孤舟,\\
    
    远看不见山,\\
    
    那天边只有云头,\\
    
    也看不见树,\\
    
    那水上只有海鸥……\footnote{〔圆天盖着大海……〕出自周太玄的《过印度洋》。这首诗1919年发表于《少年中国》。赵元任将它谱成歌曲,流行一时。}\\
    
    \end{verse}
    
    今天想来,她对我的接近文学和爱好文学,是有着多么有益的影响!
    
    像这样的教师,我们怎么会不喜欢她,怎么会不愿意和她亲近呢?我们见了她不由地就围上去。即使她写字的时候,我们也默默地看着她,连她握铅笔的姿势都急于模仿。
    
    有一件小事,我不知道还值不值得提它,但回想起来,在那时却占据过我的心灵。我父亲那时候在军阀部队里,好几年没有回来,我跟母亲非常牵挂他,不知道他的死活。我的母亲常常站在一张褪了色的神像面前焚起香来,把两个有象征记号的字条卷着埋在香炉里,然后磕了头,抽出一个来卜问\footnote{〔卜问〕向神灵询问未来或未知的事。}吉凶。我虽不像母亲那样,也略略懂了些事。可是在孩子群中,我的那些小“反对派”们,常常在我的耳边猛喊:“哎哟哟,你爹回不来了哟,他吃了炮子儿\footnote{〔炮子儿〕枪弹或炮弹。“吃了炮子儿”指中弹身亡。}啰!”那时的我,真好像父亲死了似的那么悲伤。这时候,蔡老师援助了我,批评了我的“反对派”们,还写了一封信劝慰我,说我是“心清如水的学生”。一个老师排除孩子世界里的一件小小的纠纷,是多么平常,可是回想起来,那时候我却觉得是给了我莫大的支持!在一个孩子的眼睛里,他的老师是多么慈爱,多么公平,多么伟大的人啊!
    
    每逢放假的时候,我们就更不愿离开她。我还记得,放假前我默默地站在她的身边,看她收拾东西的情景。蔡老师!我不知道你当时是不是察觉,一个孩子站在那里,对你是多么的依恋!至于暑假,对于一个喜欢他的老师的孩子来说,又是多么漫长!记得在一个夏季的夜里,席子铺在当屋,旁边燃着蚊香,我睡熟了。不知道睡了多久,也不知道是夜里的什么时候,我忽然爬起来,迷迷糊糊地往外就走。母亲喊住我:
    
    “你要去干什么?”
    
    “找蔡老师……”我模模糊糊地回答。
    
    “不是放暑假了么?”
    
    哦,我才醒了。看看那块席子,我已经走出六七尺远。母亲把我拉回来,劝说了一会,我才睡熟了。我是多么想念我的蔡老师啊!至今回想起来,我还觉得这是我记忆中的珍宝之一。一个孩子的纯真的心,就是那些在热恋中的人们也难比啊!什么时候,我能再见一见我的蔡老师呢?
    
    可惜我没有上完初小\footnote{〔初小〕当时的学制把小学分为初小和高小。初小为前四年。},就转到县立五小上学去了,从此,我就和蔡老师分别了。
    
\end{normalsize}


\newpage

\textbf{注释}:

\vspace{-1em}

\begin{itemize}
    \setlength\itemsep{-0.2em}
    \item 〔褪色〕颜色不再鲜艳,变得暗淡。
    \item 〔平生〕有生以来。
    \item 〔依恋〕依靠留恋,舍不得离开。
    \item 〔劝慰〕劝说安慰。
    \item 〔莫大〕没有比这个更大,极大。
    \item 〔牵挂〕挂念。
    \item 〔迷迷糊糊〕神志不清。
\end{itemize}

\chapter{东方红一号发射}

\begin{normalsize}
    
    1965年初,我大学毕业,分配到七机部一院\footnote{〔七机部一院〕第七机械工业部第一研究院,现中国航天科技集团公司第一研究院,也称“中国运载火箭技术研究院”。}工作。不久,“651”工程\footnote{〔“651”工程〕研制“东方红一号”卫星的工程代称。以1965年1月周恩来总理批示赵九章建议书的时间为名。}正式启动,将要研制发射我国第一颗人造卫星。它的名字决定为“东方红一号”,运载火箭则命名为“长征一号”。我与一批同学有幸参与。
    
    我当时所在的十二所六室一组负责火箭上自动控制系统的综合设计。从方案论证、设计实验到发射成功全过程,我都参与了。
    
    1970年3月,“651”工程即将进入到酒泉基地的试验阶段。一天,我接到通知:带上图纸资料及日常生活用品,到南苑东高地\footnote{〔南苑东高地〕一院本部所在地。}开会。赶到后,我才得知是要随专列出发去基地。
    
    这是一趟保密程度极高的军运专列,相关场地及设施都采取严密的安保措施,甚至开行命令都是派专人送达,没有用电信手段,以防泄密。列车上会有一个班战士负责警卫,列车运行途中铁路沿线还有战士站岗。沿线每一段铁路局的首席军代表随车到停靠站,与下一段铁路局的军代表交接。临时停靠客站时,车站都会采取戒严措施,旅客及无关人员不准入内。
    
    东高地的一院院内专列停车场提前由警卫部队检查并派重兵警卫。上车后,我关注了一下专列的情况。机车后面只挂7节车,除了装载火箭及卫星的几节安有铁轨的厢式货车、卧铺车及餐车外,还有一节上面只有两条铁轨的平板车,这是专门用来装卸火箭这样的大型物件的。装车前货车及平板车沿铁路线被推入总装车间,在这里,火箭已被分解成3节,分别固定在装有轮子的专用支架上,先用吊车吊起一节放到平板车铁轨上,再沿与其相连的货车铁轨推入货车并固定好,全部装好再运至专列停车场。专列带上平板车便于到达目的地卸车。
    
    乘车人员很少。有警卫班、军代表、技术人员和总装车间外厂组的几位师傅。这几位师傅相当于火箭和卫星的“保姆”,负责把它们从总装车间护送到基地,沿途还要对装运设备的车厢内部情况进行检查。
    
    经过几天几夜,专列才到达基地。到达后,火箭及卫星就由基地接管了,一切检测及最后发射工作都由基地的同志完成,而设计制造方在基地的代表一定会坚守在相关的现场,但并不去操纵设备,只在仪器设备或系统出了问题时,他们才会出面协助解决。
    
    基地先将星箭运到一部\footnote{〔一部〕“长征”系列运载火箭总体设计单位。}的一个厂房。在这里,火箭处于水平状态,也未加注燃料。然后就对各仪器及系统做全面详细的检测,并对能否向发射场转运作出评估。业内把这里称为水平阵地,也叫技术阵地,对火箭测试就叫水平测试。在发射场,火箭是垂直竖立的,那里叫发射阵地,也叫垂直阵地。
    
    测试过程中,我发现火箭第一、二级的级间分离线路有点问题。在一个陀螺仪测试时,又发现其中的漂移数据\footnote{〔漂移数据〕漂移指陀螺仪测量中的累积误差。已知陀螺仪的漂移数据,可以反过来校准误差。漂移数据错误将导致测量出现误差。}不正常,经查证是生产厂家调错了。问题很快由相关单位解决了。
    
    陀螺仪是火箭上对制造环境要求极高、制造工艺极复杂、造价昂贵的仪器。它是利用其惯性对火箭飞行进行定位的,即发射前就定了飞行方向和轨迹,还能测量飞行中偏离此“定位”的程度,以便及时纠正。若调错了漂移数据,火箭飞行就有偏差,可想而知后果有多严重。
    
    4月中旬,水平阵地全部检测工作完成。4月14日下午,主要负责人及一线技术人员代表抵达北京向周总理领导的中央专委汇报。我受派参加了汇报小组。
    
    我们在人民大会堂见到了周总理。听取汇报的还有李先念副总理、国家计委副主任余秋里、总政治部主任李德生、李福泽、七机部副部长钱学森及当时中央军委办事组的几位成员。
    
    作汇报的主要是“651”工程一院负责人任新民,汇报过程中涉及到的具体问题,就由相关的技术人员作答。我作答的是控制系统计算机可靠性问题。任新民还就此问题向总理作了简要说明,总理认可了。
    
    运载火箭系统非常复杂。发射时又加注了大量燃料,其中还有剧毒物质。如果发射失败,造成的严重后果是多方面的。一、有损国家声誉;二、掉下来有的东西可能还比较完整,能够识别,造成技术泄密;三、对人民生命财产造成损失;四、若发射失败,但离卫星入轨的条件差得不多,火箭第三级及卫星会飞很远才掉下来,掉在公海大洋里没啥问题。若掉在外国,是友好国家还好办,若掉在有敌对情绪的国家,则成国际事件。
    
    这就要求精确计算,预判出各种可能,报政府做到心中有数,到时从容应对。相关人员为此,又采取了多种举措。
    
    一、合理选择航区\footnote{〔航区〕火箭飞行轨迹下方地域。}。详细了解航区情况,尽量避开城市、人口稠密区及重要地面设施。为此一院一部派专员到预定航区调查,确保一切情况在掌握中。
    
    二、在火箭上设销毁装置。一旦火箭飞行失常就启动,让火箭爆炸后在空中解体成小碎片。这样对地面危害比较小,也能避免技术泄密。我们设计了两套系统,一套是自毁系统,一套是遥控系统。无论火箭控制系统检测到问题,还是地面人员判断到问题,都能销毁火箭。设计两套系统是为保险,一套没起作用,还有一套。
    
    不过,还有一个问题,让设计人员左右为难,就是过载开关的问题。很多人认为,一旦火箭出故障,卫星不能正常入轨,此时若播放《东方红》乐曲,恐怕会造成不好的政治影响。为此,火箭上设计安装了过载开关,以确保只在发射成功时才播放乐曲。但反过来想,过载开关自身也会出故障。要是发射成功了,却没有播放乐曲,就太可惜了。总理问:“你们认为火箭、卫星到底可靠不可靠啊?”几位负责人都说,从测试检查情况来看,火箭和卫星质量是可靠的。总理听了,说:“既然你们认为可靠,那我个人认为这个开关可以不要。不过,我得先向中央报告之后,再正式通知你们。”
    
    16日深夜,汇报小组接到通知,中央政治局同意了发射方案。过载开关截断\footnote{〔截断〕指不启用。},不投入使用。
    
    17日凌晨4时,我们返回基地,钱学森也同机过来,坐镇指挥。星箭转入垂直阵地。发射前的最后准备工作开始了。期间,上级专门安排大家做事故预想,为的是把所有可能出现的问题都发现和解决在发射前。
    
    4月24日21时35分,火箭发射升空。21时48分,地面指挥所确认星箭分离成功,卫星顺利入轨。90分钟后,卫星环绕地球一圈经过喀什\footnote{〔喀什〕新疆南部城市,有当时我国最西端的地面卫星跟踪站。}上空,酒泉卫星发射中心的收音机里响起了《东方红》的歌声。发射过程完美!
    
    东方红一号卫星发射成功,说明我们能把卫星准确送上天,也就能研发出射程更远的导弹。有了“两弹一星”,我们再也不怕核大国的恐吓,就能够独立自主,专心建设我们的国家。
    
\end{normalsize}


\newpage

\textbf{注释}:

\vspace{-1em}

\begin{itemize}
    \setlength\itemsep{-0.2em}
    \item 〔左右为难〕左也不好,右也不是。形容无论怎样做都有难处。
\end{itemize}

\chapter{最后一课}

\begin{normalsize}
    
    那天早晨上学,我去得很晚,心里很怕韩麦尔先生骂我,况且他说过要问我们分词\footnote{〔分词〕欧洲语言中的语法现象。},可是我连一个字也说不上来。我想就别上学了,到野外去玩玩吧。
    
    天气那么暖和,那么晴朗!
    
    画眉在树林边宛转地唱歌;锯木厂后边草地上,普鲁士兵\footnote{〔普鲁士〕指德国。1871年,普鲁士王国击败法国后,成立德意志帝国。}正在操练。这些景象,比分词用法有趣多了;可是我还能管住自己,急忙向学校跑去。
    
    我走过镇公所的时候,看见许多人站在布告牌前边。最近两年来,我们的一切坏消息都是从那里传出来的:败仗啦,征发啦,司令部的各种命令啦,我也不停步,只在心里思量:“又出了什么事啦?”
    
    铁匠华希特带着他的徒弟也挤在那里看布告,他看见我在广场上跑过,就向我喊:“用不着那么快呀,孩子,你反正是来得及赶到学校的!”
    
    我想他在拿我开玩笑,就上气不接下气地赶到韩麦尔先生的小院子里。
    
    平常日子,学校开始上课的时候,总有一阵喧闹,就是在街上也能听到。开课桌啦,关课桌啦,大家怕吵捂着耳朵大声背书啦……还有老师拿着大铁戒尺在桌子上紧敲着,“静一点,静一点……”
    
    我本来打算趁那一阵喧闹偷偷地溜到我的座位上去;可是那一天,一切偏安安静静的,跟星期日的早晨一样。我从开着的窗子望进去,看见同学们都在自己的座位上了;韩麦尔先生呢,踱来踱去,胳膊底下夹着那怕人的铁戒尺。我只好推开门,当着大家的面走进静悄悄的教室。你们可以想像,我那时脸多么红,心多么慌!
    
    可是一点儿也没有什么。韩麦尔先生见了我,很温和地说:“快坐好,小弗郎士,我们就要开始上课,不等你了。”
    
    我一纵身跨过板凳就坐下。我的心稍微平静了一点儿,我才注意到,我们的老师今天穿上了他那件挺漂亮的绿色礼服,打着皱边的领结,戴着那顶绣边的小黑丝帽。这套衣帽,他只在督学来视察或者发奖的日子才穿戴。而且整个教室有一种不平常的严肃的气氛。最使我吃惊的是,后边几排一向空着的板凳上坐着好些镇上的人,他们也跟我们一样肃静。其中有郝叟老头儿,戴着他那顶三角帽,有从前的镇长,从前的邮递员,还有些旁的人。个个看来都很忧愁。郝叟还带着一本书边破了的初级读本,他把书翻开,摊在膝头上,书上横放着他那副大眼镜。
    
    我看见这些情形,正在诧异,韩麦尔先生已经坐上椅子,像刚才对我说话那样,又柔和又严肃地对我们说:“我的孩子们,这是我最后一次给你们上课了。柏林已经来了命令,阿尔萨斯和洛林的学校只许教德语了。新老师明天就到。今天是你们最后一堂法语课,我希望你们多多用心学习。”
    
    我听了这几句话,心里万分难过。啊,那些坏家伙,他们贴在镇公所布告牌上的,原来就是这么一回事!
    
    我的最后一堂法语课!
    
    我几乎还不会作文呢!我再也不能学法语了!难道这样就算了吗?我从前没好好学习,旷了课去找鸟窝,到萨尔河上去溜冰……想起这些,我多么懊悔!我这些课本,语法啦,历史啦,刚才我还觉得那么讨厌,带着又那么沉重,现在都好像是我的老朋友,舍不得跟它们分手了。还有韩麦尔先生也一样。他就要离开了,我再也不能看见他了!想起这些,我忘了他给我的惩罚,忘了我挨的戒尺。
    
    可怜的人!
    
    他穿上那套漂亮的礼服,原来是为了纪念这最后一课!现在我明白了,镇上那些老年人为什么来坐在教室里。这好像告诉我,他们也懊悔当初没常到学校里来。他们像是用这种方式来感谢我们老师四十年来忠诚的服务,来表示对就要失去的国土的敬意。
    
    我正想着这些的时候,忽然听见老师叫我的名字。轮到我背书了。天啊,如果我能把那条出名难学的分词用法从头到尾说出来,声音响亮,口齿清楚,又没有一点儿错误,那么任何代价我都愿意拿出来的。可是开头几个字我就弄糊涂了,我只好站在那里摇摇晃晃,心里挺难受,头也不敢抬起来。我听见韩麦尔先生对我说:
    
    “我也不责备你,小弗郎士,你自己一定够难受的了。这就是了。大家天天都这么想:‘算了吧,时间有的是,明天再学也不迟。’现在看看我们的结果吧。唉,总要把学习拖到明天,这正是阿尔萨斯人最大的不幸。现在那些家伙就有理由对我们说了:‘怎么?你们还自己说是法国人呢,你们连自己的语言都不会说,不会写!……’不过,可怜的小弗郎士,也并不是你一个人的过错,我们大家都有许多地方应该责备自己呢。
    
    “你们的爹妈对你们的学习不够关心。他们为了多赚一点儿钱,宁可叫你们丢下书本到地里,到纱厂里去干活儿。我呢,我难道就没有应该责备自己的地方吗?我不是常常让你们丢下功课替我浇花吗?我去钓鱼的时候,不是干脆就放你们一天假吗?……”
    
    接着,韩麦尔先生从这一件事谈到那一件事,谈到法国语言上来了。他说,法国语言是世界上最美的语言最明白,最精确;又说,我们必须把它记在心里,永远别忘了它,亡了国当了奴隶的人民,只要牢牢记住他们的语言,就好像拿着一把打开监狱大门的钥匙。说到这里,他就翻开书讲语法。真奇怪,今天听讲,我全都懂。他讲的似乎挺容易,挺容易。我觉得我从来没有这样细心听讲过,他也从来没有这样耐心讲解过。这可怜的人好像恨不得把自己知道的东西在他离开之前全教给我们,一下子塞进我们的脑子里去。
    
    语法课完了,我们又上习字课。那一天,韩麦尔先生发给我们新的字帖,帖上都是美丽的圆体字:“法兰西”“阿尔萨斯”“法兰西”“阿尔萨斯”。这些字帖挂在我们课桌的铁杆上,就好像许多面小国旗在教室里飘扬。个个都那么专心,教室里那么安静!只听见钢笔在纸上沙沙地响。有时候一些金甲虫飞进来,但是谁都不注意,连最小的孩子也不分心,他们正在专心画“杠子”,好像那也算是法国字。屋顶上鸽子咕咕咕咕地低声叫着,我心里想:“他们该不会强迫这些鸽子也用德国话唱歌吧!”
    
    我每次抬起头来,总看见韩麦尔先生坐在椅子里,一动也不动,瞪着眼看周围的东西,好像要把这小教室里的东西都装在眼睛里带走似的。只要想想:四十年来,他一直在这里,窗外是他的小院子,面前是他的学生;用了多年的课桌和椅子,擦光了,磨损了;院子里的胡桃树长高了;他亲手栽的紫藤,如今也绕着窗口一直爬到屋顶了。可怜的人啊,现在要他跟这一切分手,叫他怎么不伤心呢?何况又听见他的妹妹在楼上走来走去收拾行李!他们明天就要永远离开这个地方了。
    
    可是他有足够的勇气把今天的功课坚持到底。习字课完了,他又教了一堂历史。接着又教初级班拼他们的ba,be,bi,bo,bu。在教室后排座位上,郝叟老头儿已经戴上眼镜,两手捧着他那本初级读本,跟他们一起拼这些字母。他感情激动,连声音都发抖了。听到他古怪的声音,我们又想笑,又难过。啊!这最后一课,我真永远忘不了!
    
    忽然教堂的钟敲了十二下。祈祷的钟声也响了。窗外又传来普鲁士兵的号声他们已经收操了。韩麦尔先生站起来,脸色惨白,我觉得他从来没有这么高大。
    
    “我的朋友们啊,”他说,“我——我——”
    
    但是他哽住了,他说不下去了。
    
    他转身朝着黑板,拿起一支粉笔,使出全身的力量,写了两个大字:
    
    “法兰西万岁!”
    
    然后他呆在那儿,头靠着墙壁,话也不说,只向我们做了一个手势:“都结束了,你们走吧。”
    
\end{normalsize}


\newpage

\textbf{注释}:

\vspace{-1em}

\begin{itemize}
    \setlength\itemsep{-0.2em}
    \item 〔诧异〕惊讶,觉得不寻常。
\end{itemize}

\chapter{生命的意义}

\begin{normalsize}
    
    保尔沿着小镇上冷冷清清的街道踱着步子,不知不觉走到了松树林前,在岔路口停住了脚步。岔路口右面是从前的监狱,阴森森的,和松林只隔着一道挺高的尖木栅栏。监狱后面是医院的白色楼房。
    
    就在这里,瓦莉亚和她的同志们被送上了绞架\footnote{〔绞架〕执行绞刑的工具。},牺牲在这空寂的广场上。保尔在当年竖立绞架的地方默默站了很一会儿,然后他走下路边的陡坡,进了埋葬烈士的墓地。
    
    也不知是哪个有心人,用冷杉枝条把那一排坟墓装饰了起来,给这片小小的墓地围上了一圈绿色的栅栏。
    
    陡坡外高耸着挺拔的青松,谷地里满铺着如茵的嫩草。这儿是小镇的尽头,阴郁而冷清。只有松树林轻声的低语,只有复苏的大地散发出新春的气息。
    
    就在这里,故乡的同志们英勇地牺牲了。为了改变那些生来就一无所有、生来就得做奴隶的人们的命运,为了使他们的生活变得美好,他们献出了自己年轻的生命。
    
    保尔缓缓摘下军帽。悲愤,深深的悲愤充满了他的心。
    
    人最宝贵的是生命。生命,每个人只有一次。
    
    人的一生应当这样度过:每当回忆往事的时候,他不会因为虚度年华而悔恨,也不会因为碌碌无为而羞耻;在临死的时候,他能够说:“我的整个生命和全部精力,都已经献给了世界上最壮丽的事业——为人类解放而斗争。”人应当充分利用每一天,因为意外的疾病或事故随时可能终结他的生命。
    
\end{normalsize}


\newpage

\textbf{注释}:

\vspace{-1em}

\begin{itemize}
    \setlength\itemsep{-0.2em}
    \item 〔茵〕铺垫的东西,垫子、褥子、毯子的通称。
    \item 〔踱〕慢慢地走。
    \item 〔冷冷清清〕不热闹,凄凉萧条。
    \item 〔烈士〕为正义事业而牺牲的人。
    \item 〔阴郁〕树木繁盛而幽深。
    \item 〔虚度〕度过(时间)而什么也没做。
    \item 〔碌碌无为〕无能,没有做成什么事。
\end{itemize}

\chapter{花市}

\begin{normalsize}
    
    今天城里逢集,街上还很安静的时候,花市上就摆满了一片花草。紫竹、刺梅、石榴、绣球、倒挂金钟、四季海棠,真是花团锦簇,千丽百俏,半条街飘满了清淡的花香。
    
    一个小小的县城里,为什么出现了这么多卖花的人?有的人说,栽培花卉不但可以供人观赏,美化环境,而且许多花卉具有药用、食用和其他用途,可以增加社会财富;也有人说农民们见钱眼开,只要能赚钱,什么生意都想做一做;还有一种简单但是富有哲理的说法,那就是:“如今买花的人多了,卖花的人自然也就多了。”
    
    “老大爷,你买了这盆三叶梅吧,这花便宜,好活,你看它开得多么鲜艳!”
    
    花市东头,一个卖花的乡下姑娘在和一个看花的乡下老头谈生意。这个姑娘集集来卖花,经常赶集的人都认识她,但不知道她叫什么名字。姑娘不过二十一二岁,生得细眉细眼,爱笑,薄薄的嘴唇很会谈生意。
    
    那老头蹲在她的花摊前面,摇摇头,对那盆开满粉红色零星小花的三叶梅表示不感兴趣。姑娘又说:
    
    “那就买了这盆兰花吧,古人说,它是‘香祖’……”
    
    “那一盆多少钱?”老头抬起下巴朝花车儿上一指,打断她的话。
    
    那是一盆令箭荷花\footnote{〔令箭荷花〕仙人掌科植物。茎扁平状如令箭,花似睡莲,因此叫令箭荷花。令箭:军中传令用的小旗,杆头加箭头,因此叫令箭。}。在今天的花市上,这是独一份儿。葱翠的令箭似的叶状枝上,四朵花竞相开放,那花朵大,花瓣儿层层叠叠,光洁鲜亮,一层紫红,一层桃红,一层粉红,花丝弯曲嫩黄,阳光一照,整个花朵就像薄薄的彩色玻璃做的一样。
    
    姑娘说:“老大爷,那是令箭荷花。”
    
    “我要的就是令箭荷花!”
    
    “它贵。”
    
    “有价儿没有?”
    
    姑娘听他口气很大,把他仔细打量了一遍。老头瘦瘦的,大约60多岁,白布褂子,紫花裤子,敞着怀,露着黑黑的结实的胸脯,不像是养种花草的人。姑娘问:
    
    “老大爷,你是哪村的?”
    
    “严村的。”
    
    “哪村?”
    
    “严村,城北的严村。”
    
    “晓得晓得。”一个看花的小伙子打趣说,“严村,好地方啊,那里的人们身上不缺‘胡萝卜素’……”
    
    看花的人们一齐笑了,姑娘笑得弯下腰去。严村是个苦地方,多少年来,那里的人们每年分的口粮只能吃七八个月,不足部分,就用胡萝卜接济\footnote{〔接济〕物资上援助。这里有接替的意思。}。这一带人们教育自己不爱做活的姑娘时,总是这么说:“懒吧,懒吧,捉不住针,拿不起线,长大了看到哪里找个婆家。拙手笨脚没人要,就把你嫁到严村吃胡萝卜去!”这个卖花的姑娘,小时候一定也受到过大人的这种警告吧?
    
    在人们的笑声中,老头红了脸,好像受了莫大羞辱。他一横眉,冲着姑娘说:“笑!你是来做买卖的,还是来笑的!”
    
    姑娘一点也不急,反倒觉得这个老头很可爱,依然笑着说:“老大爷,如今村里怎样啊?”
    
    “不怎样!”
    
    “去年,工值多少?”
    
    老头没有回答,看看买花的人多起来了,就又指着那盆令箭荷花说:“多少钱,有价儿没有?”
    
    “十五。”姑娘止住笑说。
    
    “多少?”人们睁大眼睛。
    
    “十五。”姑娘重复道。
    
    “坑人哩!”老头站起身。
    
    “太贵了,太贵了。”人们也说。
    
    姑娘看看众人,又笑了说:“是贵。这东西不能吃,不能喝,一块钱一盆也不便宜。可是老大爷,人各一爱,自己心爱的东西,讲什么贵贱呀?想便宜买胡萝卜去,十五块钱买一大车,一冬天吃不完。——你又不买,偏偏想来挨坑,那怨谁呢?”
    
    姑娘的巧嘴儿又把人们逗笑了。老头也咧着大嘴笑了说:“不买不买,太贵太贵。”
    
    “你给多少?”姑娘赶了一句。
    
    “十块钱。”老头鼓鼓肚子。
    
    “再添两块,十二块钱叫你搬走。”姑娘最后表示慷慨。
    
    老头用手捻\footnote{〔捻〕用手指捏着搓转。}着胡子,斜着眼珠望着那盆令箭荷花,牙疼似的咂起嘴唇儿。人们说:“姑娘,自家出产的,让他两块吧!”
    
    “老头,买了吧,值!”
    
    “十块,多一分钱也不买。”老头坚定地说。
    
    “十二,少一分钱也不卖。”姑娘也不相让。
    
    “不卖,你留着自己欣赏吧!”老头白了姑娘一眼,终于走了,但他不住回头望一望那盆令箭荷花。
    
    上午十点钟,集上热闹起来,花市上也站满了人。那些买花的,看花的,和猪市、兔市、木器市上一样,大半是头上戴草帽或扎手巾的乡下人。原来乡下人除了吃饭穿衣,他们的生活中也是需要一点花香的。
    
    姑娘的生意很好,转眼工夫,就卖了许多花。她正忙着,听见人群里有人嚷道:“姑娘,拿来,买了!”抬头一看,那老头又回来了,脸上红红的,好像刚刚喝了酒。
    
    “十二。”姑娘说。
    
    “给你!”老头忍疼说,“你说得对,人各一爱。我只当耽误了八天工,只当闺女少包了半垄棉花,只当又割资本主义尾巴\footnote{〔割资本主义尾巴〕指“文革”中把农民个体户正常的农副作业生产当“资本主义尾巴”砍掉。}呢,割了我两只老母鸡!”
    
    姑娘笑了笑,把那盆令箭荷花搬到他跟前去。正要付钱,一个眉目清秀的干部打扮的年轻人挤上来说:“多少钱?多少钱?”
    
    “十二。”姑娘答。
    
    “我买我买!”年轻干部去掏钱包。
    
    “我买了,我买了!”老头胳膊一乍,急忙护住那盆花。
    
    年轻干部手里摇着黑色纸扇,上下看了老头一眼,似笑非笑地说:“老头,你晓得这是什么花?”
    
    “令箭荷花!”
    
    “原产哪里?”
    
    “原产……原产姑娘家里!”
    
    年轻干部哈哈大笑。笑罢,用扇子照老头的肩上拍了两拍,说:“墨西哥。——让给我吧,老头。”
    
    “我买的东西,为什么让给你?”
    
    “唉,你买它做什么!”
    
    “你买它做什么?”
    
    “我看。”
    
    年轻干部笑了一下,弯腰去搬那盆花。老头大手一伸,急忙捉住他的手,向后一扔,也给他笑了一下:“我也看!”
    
    人群里爆发了一片笑声。姑娘没有笑,手拿着一块小花手绢,在怀里扇着风,冷冷地注视着年轻干部的行动。年轻干部无可奈何,用扇子挡着嘴,对老头嘀咕了几句什么。老头立刻冷着脸说:“不行不行,明天也是我的生日,我也爱花!”
    
    “你这个人真难说话!这么贵,你吃它喝它?”
    
    “咦,我不吃它喝它,你那个上级吃它喝它?”
    
    人们听得明白,就又笑起来了。年轻干部不知出于一种什么心理,陡地变了脸色说:“你是哪个村的?”
    
    “严村的。”
    
    “你们村的支书是谁?”
    
    老头眨眨眼睛,向众人说:“你们看这个人怪不怪,我买一盆花,他问我们村的支书是谁做什么?”
    
    这一回,人们没有笑。乡下人自有乡下人的经验,他们望着年轻干部的脸色,猜测着他的身份、来历,纷纷说:
    
    “老头,让给他吧,与人方便自己方便。”
    
    “是啊,让给他吧,只当是学雷锋哩……”
    
    老头听人劝说,心里好像活动了一点。他望着那盆令箭荷花,用手捻着胡子,又咂起嘴唇儿。年轻干部冷冷一笑,乘势说:“就是嘛,你们乡下人,还缺花看吗?高粱花、棒子花、打破碗碗花,野花野草遍地都是。姑娘,我出十三块钱买了!”
    
    说着,把钱送到姑娘脸前。
    
    姑娘不接他的钱,手拿着小花手绢,依然那么扇着,冷冷地盯着他。他还想说什么,那老头一跳脚,从怀里掏出一把崭新的票子,扯着嗓子嚷道:
    
    “你要那么说,我出十四块钱!”
    
    “我出十五块钱!”
    
    “我出……”
    
    “你这个人真是自不量力!”姑娘好像生了很大的气,瞪了老头一眼说,“你干一天活,挣几个钱,充什么大肚汉子呢!
    
    十五不要,十四不要,十二也不要了,看在你来得早,凭着你那票子新鲜,依你,十块钱搬走吧!记住,原产墨西哥,免得叫人再拿扇子拍你!”
    
    “多少多少?”年轻干部睁大眼睛。
    
    “十块钱,我们谈好了的。”姑娘轻轻一笑,对他倒很和气。
    
    老头愣了一下,呵呵地笑了,赶快付了钱,搬起那盆令箭荷花就走。年轻干部气得脸色发白,用扇子指着姑娘的脸,一时不知说什么好:
    
    “你你……”
    
    “我叫蒋小玉,南关的,我们支书叫蒋大河,还问我们治保主任是谁吗?”
    
    人们明白姑娘的心思,一齐仰着脖子大笑起来。在笑声中,人们都去摸自己的钱包,都想买姑娘一盆花,姑娘就忙起来了。她笑微微地站在百花丛中,也像一枝花,像一枝挺秀淡雅的兰花吧。
    
\end{normalsize}


\newpage

\textbf{注释}:

\vspace{-1em}

\begin{itemize}
    \setlength\itemsep{-0.2em}
    \item 〔花卉〕花草。卉:草。
    \item 〔葱翠〕草木茂盛青翠。
    \item 〔千丽百俏〕俏丽多姿。
    \item 〔无可奈何〕不情愿又没办法。
    \item 〔陡地〕陡然,突然。
    \item 〔羞辱〕使羞耻,使耻辱。
    \item 〔花团锦簇〕形容花色繁多,美艳亮丽。
    \item 〔横眉〕愤怒地皱眉头。
    \item 〔垄〕田里农作物的行,或行与行之间的空地。
\end{itemize}

\chapter{大自然的语言}

\begin{normalsize}
    
    立春过后,大地渐渐从沉睡中苏醒过来。冰雪融化,草木萌发,各种花次第开放。再过两个月,燕子翩然归来。不久,布谷鸟也来了。于是转入炎热的夏季,这是植物孕育果实的时期。到了秋天,果实成熟,植物的叶子渐渐变黄,在秋风中簌簌地落下来。北雁南飞,活跃在田间草际的昆虫也都销声匿迹。到处呈现一片衰草连天的景象,准备迎接风雪载途的寒冬。在地球上温带和亚热带区域里,年年如是,周而复始。
    
    几千年来,劳动人民注意了草木荣枯、候鸟去来等自然现象同气候的关系,据以安排农事。杏花开了,就好像大自然在传语要赶快耕地;桃花开了,又好像在暗示要赶快种谷子。布谷鸟开始唱歌,劳动人民懂得它在唱什么:“阿公阿婆,割麦插禾\footnote{〔阿公阿婆,割麦插禾〕这是把布谷鸟的叫声想象成催促农民及时耕作的话。禾,这里指稻秧。}。”这样看来,花香鸟语,草长莺飞,都是大自然的语言。
    
    这些自然现象,我国古代劳动人民称它为物候。物候知识在我国起源很早。古代流传下来的许多农谚就包含了丰富的物候知识。到了近代,利用物候知识来研究农业生产,已经发展为一门科学,就是物候学。物候学记录植物的生长荣枯,动物的养育往来,如桃花开、燕子来等自然现象,从而了解随着时节推移的气候变化和这种变化对动植物的影响.
    
    物候观测使用的是“活的仪器”,是活生生的生物。它比气象仪器复杂得多,灵敏得多。物候观测的数据反映气温、湿度等气候条件的综合,也反映气候条件对于生物的影响。应用在农事活动里,比较简便,容易掌握。物候对于农业的重要性就在这里。下面是一个例子。
    
    北京的物候记录,1962年的山桃、杏花、苹果、榆叶梅\footnote{〔榆叶梅〕落叶灌木或小乔木,花粉红色,核果球形、红色。可供观赏。}、西府海棠、丁香、刺槐的花期比1961年迟十天左右,比1960年迟五六天。根据这些物候观测资料,可以判断北京地区1962年农业季节来得较晚。而那年春初种的花生等作物仍然是按照往年日期播种的,结果受到低温的损害。如果能注意到物候延迟,选择适宜的播种日期,这种损失就可能避免。
    
    物候现象的来临决定于哪些因素呢?
    
    首先是纬度。越往北桃花开得越迟,候鸟也来得越晚。值得指出的是物候现象南北差异的日数因季节的差别而不同。中国大陆性气候显著,冬冷夏热。冬季南北温度悬殊,夏季却相差不大。在春天,早春跟晚春也不相同。如在早春三四月间,南京桃花要比北京早开20天,但是到晚春五月初,南京刺槐开花只比北京早10天。所以在华北常感觉到春季短促,冬天结束,夏天就到了。
    
    经度的差异是影响物候的第二个因素。凡是近海的地方,比同纬度的内陆,冬天温和,春天反而寒冷。所以沿海地区的春天的来临比内陆要迟若干天。如大连纬度在北京以南约1°,但是在大连,连翘\footnote{〔连翘〕落叶灌木,春季开鲜黄色花,果实可以入药。}和榆叶梅的盛开都比北京要迟一个星期。又如济南苹果开花在四月中或谷雨节,烟台要到立夏。两地纬度相差无几,但烟台靠海,春天便来得迟了。
    
    影响物候的第三个因素是高下的差异。植物的抽青、开花等物候现象在春夏两季越往高处越迟,而到秋天乔木的落叶则越往高处越早。不过研究这个因素要考虑到特殊的情况。例如秋冬之交,天气晴朗的空中,在一定高度上气温反比低处高。这叫逆温层。由于冷空气比较重,在无风的夜晚,冷空气便向低处流。这种现象在山地秋冬两季,特别是这两季的早晨,极为显著,常会发现山脚有霜而山腰反无霜。在华南丘陵区把热带作物引种在山腰很成功,在山脚反不适宜,就是这个道理。
    
    此外,物候现象来临的迟早还有古今的差异。根据英国南部物候的一种长期记录,拿1741到1750年十年平均的春初七种乔木油青和开花日期同1921到1930年十年的平均值相比较,可以看出后者比前者早九天。就是说,春天提前九天。
    
    物候学这门科学接近生物学中的生态学和气象学中的农业气象学。物候学的研究首先是为了预报农时,选择播种日期。此外还有多方面的意义。物候资料对于安排农作物区划,确定造林和采集树木种子的日期,很有参考价值,还可以利用来引种植物到物候条件相同的地区,也可以利用来避免或减轻害虫的侵害。中国有很大面积的山区土地可以耕种,而山区的气候、土壤对农作物的适应情况,有很多地方还有待调查。为了便利山区的农业发展,开展山区物候观测是必要的。
    
    物候学是关系到农业丰产的科学,我们要进一步加强物候观测,懂得大自然的语言,争取农业更大的丰收。
    
\end{normalsize}


\newpage

\textbf{注释}:

\vspace{-1em}

\begin{itemize}
    \setlength\itemsep{-0.2em}
    \item 〔次第〕一个挨着一个地,接连地。
    \item 〔翩然〕动作轻快的样子。
    \item 〔孕育〕怀孕时在体内养育胎儿,用来比喻酝酿着新事物。
    \item 〔簌簌〕多而杂乱地接连落下的样子,纷纷。
    \item 〔销声匿迹〕躲藏起来不发声不露面。
    \item 〔风雪载途〕风雪连接不断。载途:满路,形容多。
    \item 〔农谚〕有关农业生产的谚语。谚:因有道理而在群众中流传的固定语句。
    \item 〔纬度〕地理学名词。南北为纬,纬度即衡量南北的度数。以赤道为0度,北极为90度,称为北纬。越靠北,北纬度越大。
    \item 〔经度〕地理学名词。东西为经,经度即衡量东西的度数。以英国伦敦为0度,至太平洋中央为180度,称为东经。越靠东,东经度越大。
    \item 〔抽青〕发芽。抽:拔,比喻植物生长。
\end{itemize}

\chapter{一}

\begin{normalsize}
    
    \begin{verse}[0.5\linewidth]
        天际,于褐色的群峦间, \\
        太阳,这光辉无限的花, \\
        斜倚向大地,也将睡了。 \\
        不起眼的一朵雏菊,开在田野旁, \\
        坍倒的灰墙上,四下麦草疯长。 \\
        洁白无瑕,绽放纯真晕光, \\
        这小花,独立残垣, \\
        凝视那永恒青苍。 \\
        巨大的恒星倾洒着无尽流晖。 \\
        “可我,我也有光芒!”
    \end{verse}
    
\end{normalsize}


\newpage

\textbf{注释}:

\vspace{-1em}

\begin{itemize}
    \setlength\itemsep{-0.2em}
    \item 〔峦〕小而尖的山。
    \item 〔雏菊〕欧洲常见野花,春季开花。
    \item 〔绽放〕(花)开。绽:衣服裂开。
    \item 〔坍〕竖立的建筑物(墙、堤岸)倒塌。
    \item 〔残垣〕残破的建筑。垣:矮墙。
    \item 〔恒星〕自身能发出光和热的天体。这里指太阳。
\end{itemize}

\chapter{三年以后}

\begin{normalsize}
    
    \begin{verse}[0.5\linewidth]
        推开这摇摇欲坠的窄门, \\
        我徜徉在小小的花园里。 \\
        露珠上闪动着晨光温嫩, \\
        遍撒在花瓣上晶莹欲滴。
    \end{verse}
    
    
    \begin{verse}[0.5\linewidth]
        眼前一切如旧:葡萄藤蔓 \\
        缠满凉棚,棚里数把藤椅。 \\
        泉水低语,银光清亮依然, \\
        老杨树的悲怨永不停息。
    \end{verse}
    
    
    \begin{verse}[0.5\linewidth]
        蔷薇轻颤,犹同往日;犹同 \\
        往日,百合傲然,随风摇动。 \\
        往来的云雀\footnote{〔云雀〕百灵科鸟类。大体砂棕色,头后羽毛稍长,略成羽冠状。},都与我相熟。
    \end{verse}
    
    
    \begin{verse}[0.5\linewidth]
        甚至,还有薇莉达\footnote{〔薇莉达〕女先知,传说在日耳曼部落反抗古罗马统治时,曾预言了起义初期的胜利。}的雕塑。 \\
        石膏剥落,散落走道尽处。 \\
        纤影伫立在木犀\footnote{〔木犀〕草本植物,一尺高,花白或淡黄白色。}暗香中。
    \end{verse}
    
\end{normalsize}


\newpage

\textbf{注释}:

\vspace{-1em}

\begin{itemize}
    \setlength\itemsep{-0.2em}
    \item 〔徜徉〕自由自在地行走。
    \item 〔伫立〕久久站立。
    \item 〔石膏〕硫酸盐矿物,粉末调水后可做雕塑材料。
\end{itemize}

\chapter{播种季的傍晚}

\begin{normalsize}
    
    \begin{verse}[0.5\linewidth]
        黄昏时分,我坐在大门口 \\
        用憧憬的目光,欣赏 \\
        白日剩余的时间,怎样由 \\
        最后的农活来照亮
    \end{verse}
    
    
    \begin{verse}[0.5\linewidth]
        夜色沁入田里,我凝望着 \\
        一个老人,衣衫褴褛 \\
        将未来的收成,一把把地 \\
        撒入犁沟,感动难已
    \end{verse}
    
    
    \begin{verse}[0.5\linewidth]
        他的背影高大深沉,俯临 \\
        这深耕的土地。可以 \\
        感受到,他深深相信 \\
        每分每秒流过,皆有意义
    \end{verse}
    
    
    \begin{verse}[0.5\linewidth]
        他行走在无垠的大地上 \\
        来来去去,远撒谷粒 \\
        手掌张开,合拢,又再扬起 \\
        暗中远觑,我心徜徉
    \end{verse}
    
    
    \begin{verse}[0.5\linewidth]
        和着喧嚣,夜将他的影子 \\
        铺展开来,直到天边 \\
        也将播种者神圣的身姿 \\
        放大,遍布星辰之间
    \end{verse}
    
\end{normalsize}


\newpage

\textbf{注释}:

\vspace{-1em}

\begin{itemize}
    \setlength\itemsep{-0.2em}
    \item 〔憧憬〕向往。
    \item 〔沁〕渗入,浸润。
    \item 〔褴褛〕(衣服)破烂。
    \item 〔俯临〕俯身下看。临:从上向下看。
    \item 〔无垠〕广阔无边。
    \item 〔觑〕偷看。
    \item 〔喧嚣〕声音杂乱,不清静。
    \item 〔徜徉〕陶醉于某事物之中。
\end{itemize}

\chapter{刘胡兰}

\begin{normalsize}
    
    1945年,日本投降了。在中国的日军听到了投降的消息,仍然不死心,图谋东山再起。山西的军阀阎锡山想要拥兵自重,做土皇帝\footnote{〔土皇帝〕指在地方势力范围称霸,实行专制独裁统治的军阀。}。在山西的日军就和阎锡山勾结,假投降,真合作,仍在山西横行无忌。直到1949年,山西人民仍然生活在军阀和日寇的阴影下。
    
    1946年下半年,阎锡山调集近万兵力,对晋中地区进行“扫荡”。“扫荡”后,文水县云周西村\footnote{〔文水县〕山西省中部县城。}迎来了伪村长\footnote{〔伪村长〕指军阀和日寇委任的村长,不被当地群众承认。后面“伪官”相同。}石佩怀。
    
    他积极为阎锡山军队派粮派款\footnote{〔派粮派款〕要求农民上缴粮食钱财。派:分配,指定。},抓壮丁\footnote{〔抓壮丁〕指强制要求壮年男子参加军队。丁:男子。}、抢粮食,成了当地一害。我党决定消灭石佩怀,为民除害。
    
    处决石佩怀的行动中,负责侦查放哨的是14岁的刘胡兰。
    
    刘胡兰也是云周西村人。她从小积极参加革命,已经是“妇救会”\footnote{〔“妇救会”〕全称“妇女救国联合会”,是共产党领导的抗日组织。}的秘书。石佩怀的问题,也是她向区县干部反映的。
    
    “扫荡”中,文水县委让刘胡兰跟着部分干部转移上山。刘胡兰坚决要求留下来,坚持斗争。
    
    处决石佩怀后,阎锡山军队和镇上的地主伪官惊怒交加,派出“复仇队”到云周西村搜捕抢掠,要捉拿“凶手”。
    
    1947年1月12日上午,“复仇队”包围了云周西村。由于叛徒的出卖,刘胡兰等七名同志被捕了。
    
    为了给自己的反动统治“立威”,敌人打算让刘胡兰“自白”。审问开始了。敌人问:“你就是刘胡兰?”
    
    刘胡兰说:“我就是刘胡兰。”
    
    敌人又说:“已经有人供出,你是共产党。”
    
    刘胡兰说:“我就是共产党。”
    
    敌人问:“村里还有谁是共产党?”
    
    刘胡兰说:“就是我一个。”
    
    敌人问:“你认得谁是八路军?”
    
    刘胡兰说:“谁也不认得。”
    
    敌人见她年纪小,哄她说:“今天你跟着我走,我就放你一马,还给你一块土地。”
    
    刘胡兰不理睬。
    
    敌人恼羞成怒,大叫:“小小年纪,好嘴硬!你不怕死吗?”
    
    刘胡兰说:“怕死不当共产党!”
    
    敌人要刘胡兰屈服。要她保证“今后不再给八路军办事”。刘胡兰说:“那可保不住。”
    
    敌人无计可施,决定杀死刘胡兰。刘胡兰走到铡刀前,自己躺下去,敌人就把她杀害了。
    
    在刘胡兰牺牲后,毛主席为她写了八个字:“生的伟大,死的光荣”。
    
\end{normalsize}


\newpage

\textbf{注释}:

\vspace{-1em}

\begin{itemize}
    \setlength\itemsep{-0.2em}
    \item 〔图谋〕谋求,谋划,设法做到。
    \item 〔东山再起〕比喻失败后又重新恢复地位。
    \item 〔拥兵自重〕依靠掌握的军队,强大力量,不遵从更高的权威。
    \item 〔横行无忌〕行动蛮横,胡作非为,无所顾忌。
    \item 〔反映〕把情况、意见等如实告诉上级或有关部门。
    \item 〔牺牲〕舍生取义,坚持自己的信仰、理想而死。
\end{itemize}

\chapter{挖荠菜}

\begin{normalsize}
    
    我对荠菜\footnote{〔荠菜〕草本植物,花白色,茎叶嫩时可以吃。},有着一种特殊的感情……
    
    小的时候,我是那么馋!刚抽出嫩条还没打花苞的蔷薇枝,把皮一剥,我就能吃下去;刚割下来的蜂蜜,我会连蜂房一起放进嘴巴里;更别说什么青玉米棒子、青枣、青豌豆啰。所以,只要我一出门儿,碰上财主家的胖儿子,他就总要跟在我身后,拍着手、跳着脚地叫着:“馋丫头!馋丫头!”羞得我连头也不敢回。
    
    我感到又羞恼,又冤屈!七八岁的姑娘家,谁愿意落下这么个名声?可是有什么办法呢?我饿啊!我真不记得什么时候,那种饥饿的感觉曾经离开过我,就是现在,每当我回忆起那个时候的情景,留在我记忆里最鲜明的感觉,也还是一片饥饿……
    
    吃那些没收进主人家仓房里的东西,我还一次也没有被人家抓到过。倒不是因为我的运气格外好,而是人们多半并不想认真地惩罚一个饥饿的孩子。可有一次,我在财主家的地里掰玉米棒子,被他的大管家发现了,他立刻拿着一根又粗又直的木头棒子,毫不留情地紧紧向我追来。我没命地逃着。我想我一定跑得飞快,因为风在我的耳朵旁边呼呼直响。不知是我被吓昏了,还是平时很熟悉的那些田间小路有意捉弄我,为什么面前偏偏横着一条小河?追赶我的人越来越近了。我害怕到了极点,便不顾一切地纵身跳进那条河。
    
    河水并不很深,但是足以没过我那矮小的身子。我一声不响地挣扎着,扑腾着,身子失去了平衡。冰凉的河水呛得我好难受,我几乎背过气去,而河水却依旧在我身边不停地流着,流着……在由于恐怖而变得混乱的意识里,却出奇清晰地反映出岸上那个追赶我的人的残酷笑声。
    
    我简直不知道我是怎么样才爬上对岸的。更使我丧气的是脚上的鞋子不知什么时候掉了一只。我实在没有勇气重新回头去找那只丢失了的鞋子,可我也不敢回家,我怕妈妈知道。不,我并不是怕她打我。我是怕看见她那双被贫困的生活折磨得失去了光彩的、哀愁的眼睛。那双眼睛,会因为我丢失了鞋子而更加暗淡。
    
    我独自一人游荡在田野里。太阳落山了,琥珀色的晚霞渐渐地从天边退去。远处,庙里的钟声在薄暮中响起来。羊儿咩咩地叫着,由放羊的孩子赶着回圈了;乌鸦也呱呱地叫着回巢去了。夜色越来越浓了,村落啦,树林子啦,坑洼啦,沟渠啦,好像一下子全都掉进了神秘的沉寂里。我听见妈妈在村口焦急地呼唤着我的名字,只是不敢答应。一种比饥饿更可怕的东西平生头一次潜入了我那童稚的心……
    
    说过了这些,人们也许会理解我为什么对荠菜有着那么特殊的感情。
    
    经过一个没有什么吃食可以寻觅、因而显得更加饥饿的冬天,大地春回、万物复苏的日子重新来临了!田野里长满了各种野菜:雪蒿、马齿苋、灰灰菜、野葱……最好吃的是荠菜。把它下在玉米糊糊里,再放上点盐花,真是无上的美味啊!而挖荠菜时的那种坦然的心情,更可以称得上是一种享受:提着篮子,迈着轻捷的步子,向广阔无垠的田野里奔去。嫩生生的荠菜,在微风中挥动它们绿色的手掌,招呼我,欢迎我。我再也不必担心有谁会拿着大棒子凶神恶煞似地追赶我,我甚至可以不时地抬头看看天上吱吱喳喳飞过去的小鸟,树上绽开的花儿和蓝天上白色的云朵。那时,我的心里便会不由地升起一个热切的愿望:巴不得这个世界上的一切,都像荠菜一样是属于我们每一个人的。
    
    解放以后,我进了城。偶然,在大菜场里,也可以看到人工培植的荠菜出售。长得肥肥大大的,总有半尺来长,洗得干干净净,水灵灵的。一小扎,一小扎,码得整整齐齐地摆在菜摊子上,价钱也不贵。可我,总还是怀念那长在野地里的荠菜,就像怀念那些与自己共过患难的老朋友一样。
    
    多少年来,每到春天,我总要挑个风和日丽的日子,带上孩子们到郊区的野地里去挖荠菜。我明白,孩子们之所以在我的身旁跳着,跑着,尖声地打着唿哨\footnote{〔唿哨〕把手指放在嘴里用力吹,发出尖锐的哨子一样的声音。},多半因为这对他们来说,是一种有趣的游戏——和煦的阳光,绿色的田野,就像一幅优美的风景画似的展现在他们面前,使他们的身心全都感到愉快。他们长大一些之后,陪同我去挖荠菜,似乎就变成了对我的一种迁就了,正像那些恭顺的年轻人,迁就他们那些因为上了年纪而变得有点怪癖的长辈一样。这时,我深感遗憾:他们多半不能体会我当年挖荠菜的心情!
    
    等到我把一盘用精盐、麻油、味精、白糖精心调配好的荠菜放到餐桌上去的时候(小的时候,我可是做梦也没有想到我那可爱的荠菜会享受到今天这样的“荣华富贵”),他们也还是带着那种迁就的微笑,漫不经心地用筷子挑上几根荠菜……
    
    看着他们那双懒洋洋的筷子,我的心里就像翻倒了的五味瓶,什么滋味都有。因为我知道,这种赏光似的迁就,并不只是表现在对挖荠菜这一桩事情上,它还表现在对我们这一代人的一些见解和行为上。在他们看来,我们的有些见解和行为,都像陈列在博物馆里的出土文物——离他们的现实生活太远了,不顶用了。自然,我也并不认为我们的见解和行为就完全正确。只要他们不觉得厌烦,我甚至愿意跟他们谈谈我们在探索人生方面曾经走过的弯路,以便他们少付出一些不必要的代价。我真希望我们之间不要成为隔膜很深的两代人,而是心灵相通的朋友。
    
    孩子,让我们多谈谈心吧,让妈妈多讲讲当“馋丫头”时的故事给你们听吧。想想你们妈妈当年挖荠莱的情景,你们就会珍爱荠菜,珍爱生活。你们就会懂得什么是幸福,怎样才会得到幸福。
    
\end{normalsize}


\newpage

\textbf{注释}:

\vspace{-1em}

\begin{itemize}
    \setlength\itemsep{-0.2em}
    \item 〔馋〕贪吃,想吃好吃的。
    \item 〔蔷薇〕落叶灌木,茎通常有皮刺。
    \item 〔蜂房〕蜜蜂用蜜蜡做的六角形的巢。
    \item 〔童稚〕幼小,儿童的。
    \item 〔见解〕看法,观点。
    \item 〔和煦〕(风、阳光)温暖的。
    \item 〔患难〕灾祸和苦难。患:灾祸。
    \item 〔薄暮〕傍晚,黄昏。
    \item 〔凶神恶煞〕凶恶的神怪。煞:作恶的鬼怪。
    \item 〔迁就〕委屈自己,曲意将就。
    \item 〔恭顺〕恭敬顺从(长辈)。
    \item 〔隔膜〕观念上有分歧,情感上缺少关连,没有亲密感、亲切感,仿佛隔了一层膜。
\end{itemize}

\chapter{人民英雄永垂不朽}

\begin{normalsize}
    
    从东长安街向天安门广场走去,还未进入广场,就能望见高耸的人民英雄纪念碑。
    
    它如同顶天立地的巨人,屹立在广场南部,和天安门遥遥相对。在远处就可以看到毛主席亲笔题写的“人民英雄永垂不朽”八个金色大字。
    
    花岗石石道直铺到纪念碑宽广的台阶前。沿着台阶可以登上月台。月台有两层,每层用汉白玉\footnote{〔汉白玉〕纯白的大理石,从汉代起,用于宫殿中的阶级护栏。}雕栏围起。纪念碑就在第二层月台中央的大小两层碑座上。
    
    这座纪念碑是根据1949年9月30日中国人民政治协商会议第一届全体会议的决议兴建的。当天傍晚,毛主席率领全体政协委员为纪念碑举行了庄严隆重的奠基礼。毛主席亲自执锨铲土,为纪念碑奠定基石。1958年4月22日,人民英雄纪念碑建成。
    
    这是中国自古以来最大的纪念碑。它从地面到碑顶高达三十七点九四公尺,有十层楼那么高,比纪念碑对面的天安门还高三点二四公尺。
    
    纪念碑是用一万七千块坚硬的花岗石和洁白的汉白玉砌成的。整个纪念碑的造型既保留了传统风格,又有鲜明的新时代精神。碑顶是上有卷云下有重幔的小庑殿顶,这是民族传统的建筑形式。碑身东西两侧上部,刻着以红星、松柏和旗帜组成的装饰花纹,象征着革命精神万年长存。小碑座的四周,雕刻着以牡丹花、荷花、菊花等组成的八个大花圈,这些花朵象征着品质高贵、纯洁,表示全国人民对英雄们的怀念和敬仰。
    
    碑的正面朝北。一块六十吨重、十四点七公尺高的碑心石\footnote{〔碑心石〕即碑身主体,为一整块大理石。}上,雕刻着毛主席题写的“人民英雄永垂不朽”八个镏金大字。这八个字是纪念碑的主题。碑身背面,一行行镏金字整齐地排列着,这是毛主席亲自起草、周总理亲笔写的碑文:
    
    “三年以来在人民解放战争和人民革命中牺牲的人民英雄们永垂不朽。”
    
    “三十年以来在人民解放战争和人民革命中牺牲的人民英雄们永垂不朽。”
    
    “由此上溯到一千八百四十年,从那时起,为了反对内外敌人,争取民族独立和人民自由幸福,在历次斗争中牺牲的人民英雄们永垂不朽。”
    
    十块汉白玉的大浮雕,镶嵌在大碑座的四周。这些大浮雕高二公尺,合在一起共长四十点六八公尺。每幅浮雕里有二十个左右英雄人物,每个人物都和真人一样大小,他们的面貌、性格、表情和姿态形象都不相同。
    
    从东面起,按着历史顺序,浮雕的主题分别是:东面的“虎门销烟”、“太平天国”,南面的“武昌起义”、“五四运动”、“五卅惨案”,西面的“南昌起义”、“敌后抗日”。
    
    北面,也是碑的正面,有三块大浮雕。中央是人民解放军百万雄师“胜利渡长江,解放全中国”的浮雕,这是十块浮雕中最大的一块。两旁还各有一块浮雕。左边是渡江前夕,工农妇女支援前线的场面;右边是全国人民欢迎解放军的情景。
    
    人民英雄纪念碑展现了中国革命的艰难道路。百余年来,中国人民争取自由幸福的斗争中,诞生了无数的英雄。他们带领着中国人民走向了最终的胜利。中国人民的英雄永垂不朽。
    
\end{normalsize}


\newpage

\textbf{注释}:

\vspace{-1em}

\begin{itemize}
    \setlength\itemsep{-0.2em}
    \item 〔屹立〕像山峰一样稳固地直立。
    \item 〔锨〕一种掘土工具。
    \item 〔上溯〕追求根源。溯:逆着水流的方向走。
    \item 〔镏金〕一种镀金手法,把溶解在水银里的金子涂刷在金属表面,再加热让水银蒸发,留下金子。也写作鎏金。
    \item 〔浮雕〕一种雕刻技法。雕刻形象突出周围平坦的表面,仿佛浮在表面上。
    \item 〔前线〕战争中指战斗发生的地方。
    \item 〔永垂不朽〕长久流传,永不磨灭。垂:指记录为文字而流传后世。朽:木头腐烂。
\end{itemize}

\chapter{愚公移山}

\begin{normalsize}
    
    我们开了一个很好的大会。我们做了三件事:第一,决定了党的路线,这就是放手发动群众,壮大人民\footnote{〔人民〕作为劳动者主体的大众。}力量,在我党的领导下,打败日本侵略者,解放全国人民,建立一个新民主主义的中国。第二,通过了新的党章。第三,选举了党的领导机关——中央委员会。今后的任务就是领导全党实现党的路线。我们开了一个胜利的大会,一个团结的大会。代表们对三个报告发表了很好的意见。许多同志作了自我批评,从团结的目标出发,经过自我批评,达到了团结。这次大会是团结的模范,是自我批评的模范,又是党内民主的模范。
    
    大会闭幕以后,很多同志将要回到自己的工作岗位上去,将要分赴各个战场。同志们到各地去,要宣传大会的路线,并经过全党同志向人民作广泛的解释。
    
    我们宣传大会的路线,就是要使全党和全国人民建立起一个信心,即革命一定要胜利。首先要使先锋队觉悟,下定决心,不怕牺牲,排除万难,去争取胜利。但这还不够,还必须使全国广大人民群众觉悟,甘心情愿和我们一起奋斗,去争取胜利。要使全国人民有这样的信心:中国是中国人民的,不是反动派\footnote{〔反动派〕代表阻碍生产力发展的生产关系的利益集团的政治派别。这里指代表资产阶级的政党。}的。中国古代有个寓言,叫做“愚公移山”。说的是古代有一位老人,住在华北,名叫北山愚公。他的家门南面有两座大山挡住他家的出路,一座叫做太行山,一座叫做王屋山。愚公下决心率领他的儿子们要用锄头挖去这两座大山。有个老头子名叫智叟的看了发笑,说是你们这样干未免太愚蠢了,你们父子数人要挖掉这样两座大山是完全不可能的。愚公回答说:我死了以后有我的儿子,儿子死了,又有孙子,子子孙孙是没有穷尽的。这两座山虽然很高,却是不会再增高了,挖一点就会少一点,为什么挖不平呢?愚公批驳了智叟的错误思想,毫不动摇,每天挖山不止。这件事感动了上天,他就派了两个神仙下凡,把两座山背走了。现在也有两座压在中国人民头上的大山,一座叫做帝国主义\footnote{〔帝国主义〕资本主义社会形式,依靠资本输出剥削其他国家的劳动者,进行统治。},一座叫做封建主义\footnote{〔封建主义〕地主占有土地,剥削农民的社会制度。 }。中国共产党早就下了决心,要挖掉这两座山。我们一定要坚持下去,一定要不断地工作,我们也会感动上帝的。这个上帝不是别人,就是全中国的人民大众。全国人民大众一齐起来和我们一道挖这两座山,有什么挖不平呢?
    
    昨天有两个美国人要回美国去,我对他们讲了,美国政府要破坏我们,这是不允许的。我们反对美国政府扶蒋反共的政策。但是我们第一要把美国人民和他们的政府相区别,第二要把美国政府中决定政策的人们和下面的普通工作人员相区别。我对这两个美国人说:告诉你们美国政府中决定政策的人们,我们解放区\footnote{〔解放区〕建立了人民政权的地区。}禁止你们到那里去,因为你们的政策是扶蒋反共,我们不放心。假如你们是为了打日本,要到解放区是可以去的,但要订一个条约。倘若你们偷偷摸摸到处乱跑,那是不许可的。赫尔利\footnote{〔赫尔利〕指帕特里克·赫尔利,美国外交家。1944年11月任美国驻华大使,试图斡旋国共关系。1945年8月陪同毛泽东赴重庆谈判。1945年11月辞职返美。}已经公开宣言不同中国共产党合作,既然如此,为什么还要到我们解放区去乱跑呢?
    
    美国政府的扶蒋反共政策,说明了美国反动派的猖狂。但是一切中外反动派的阻止中国人民胜利的企图,都是注定要失败的。现在的世界,民主是主流,反民主的反动只是一股逆流。目前反动的逆流企图压倒民族独立和人民民主\footnote{〔人民民主〕由占人口绝大多数、掌握关键生产资料的劳动人民统治的政治制度。}的主流,但反动的逆流终究不会变为主流。现在依然如斯大林\footnote{〔斯大林〕约瑟夫·斯大林,苏联无产阶级革命家、思想家、政治家、军事家。列宁逝世后成为苏联最高领导人。}很早就说过的一样,旧世界\footnote{〔旧世界〕当时指帝国主义的全球殖民体系和秩序。}有三个大矛盾:第一个是帝国主义国家中的无产阶级\footnote{〔无产阶级〕不拥有生产资料,依靠他人的生产资料劳动生产的集体。}和资产阶级\footnote{〔资产阶级〕拥有生产资料,依靠他人的劳动进行生产的集体。}的矛盾,第二个是帝国主义国家之间的矛盾,第三个是殖民地半殖民地国家和帝国主义宗主国之间的矛盾\footnote{〔殖民地……〕殖民地:帝国主义资本集团以直接或间接方式统治、控制、支配、垄断的地区。半殖民地:统治集团中尚有自主的民族资产阶级或封建地主阶级成分的殖民地。宗主国:支配殖民地的帝国主义国家,称为该殖民地的宗主国。}。这三种矛盾不但依然存在,而且发展得更尖锐了,更扩大了。由于这些矛盾的存在和发展,所以虽有反苏反共反民主的逆流存在,但是这种反动逆流总有一天会要被克服下去。
    
    现在中国正在开着两个大会,一个是国民党的第六次代表大会,一个是共产党的第七次代表大会。两个大会有完全不同的目的:一个要消灭共产党和中国民主势力,把中国引向黑暗;一个要打倒日本帝国主义和它的走狗中国封建势力,建设一个新民主主义的中国,把中国引向光明。这两条路线在互相斗争着。我们坚决相信,中国人民将要在中国共产党领导之下,在中国共产党第七次大会的路线的领导之下,得到完全的胜利,而国民党的反革命路线必然要失败。
    
    \hfill 一九四五年六月十一日
    
\end{normalsize}


\newpage

\textbf{注释}:

\vspace{-1em}

\begin{itemize}
    \setlength\itemsep{-0.2em}
    \item 〔模范〕值得同类效仿的人或物。
    \item 〔猖狂〕狂妄大胆,任意作为,没有顾忌。
    \item 〔侵略〕侵犯掠夺。以占领别国土地,掠夺、奴役、屠杀别国人民为目标的攻击行为。
\end{itemize}

\chapter{雄伟的人民大会堂}

\begin{normalsize}
    
    天安门广场西边,巍然耸立着一座雄伟壮丽的大厦,这就是人民大会堂。全国各族人民的代表在这里共商国是。
    
    庄严的人民大会堂,是首都最宏伟的建筑之一,建筑面积达十七万一千八百平方米,体积有一百五十九万六千九百立方米。一条黄绿相间的琉璃屋檐,把巍峨的大会堂的轮廓从蔚蓝的天空中勾画出来。那壮丽的柱廊,淡雅的色调,以及四周层次繁多的建筑立面\footnote{〔立面〕建筑与外部环境直接接触的界面,及其展现方式。},组成了一副庄严绚丽的图画。
    
    我们在建筑师的陪同下,参观了人民大会堂。在天安门广场,远远就能看见正门顶上的国徽的闪闪金光。踏上一层楼高的花岗石大台阶,迎面是十二根浅灰色的大理石门柱。门柱有二十五米高,柱身要四个人才能合抱。柱距采用我国柱廊的传统样式,明间\footnote{〔明间〕建筑中指居中四柱围成的矩形空间。两旁的柱间空间称为次间。}最宽,紧邻左右的次间较窄,再往两旁的又更窄。这样高大而有力的柱廊,是建筑师吸收了中外古今门柱造型的优点创造出来的。
    
    迈进金色大铜门,穿过宽阔的风门厅\footnote{〔风门厅〕为挡风而加设的隔厅。}和衣帽厅\footnote{〔衣帽厅〕为换衣帽而加设的隔厅。},就到了大会堂建筑的枢纽部分——中央大厅。建筑师站在这里,指着四周向我们介绍了整个建筑的布局:朝西直入万人大礼堂;往北通宴会厅;向南穿过长长的廊道,是全国人民代表大会常务委员会的办公大楼。整个建筑就是由这三部分组成的。
    
    万人大礼堂,里面宽七十六米,深六十米,中部高三十三米,体积达八万六千立方米,几乎能容下一座大楼。但是由于设计师处理得巧妙,走进大礼堂的人放眼一看,从屋顶到地面,上下浑然一体,并不感到怎样空旷。穹窿形的天花\footnote{〔天花〕屋顶梁架下、室内顶上的层面。室内顶饰的一种。}上纵横排着近五百个灯孔。灯火齐明的时候,就像满天星斗。中心的大灯是红宝石般的五角星,周围是七十条瑰丽的光芒线和四十瓣镏金的向日葵花瓣,象征着全国各族人民万众一心,紧密团结在中国共产党的周围。再往外是三环层次分明的水波形暗灯槽\footnote{〔暗灯槽〕使灯光不直射,间接照明的工具。},同周围装贴的淡青色塑料板相映,形成“水天一色”的奇观。
    
    大礼堂是椭圆形\footnote{〔椭圆形〕扁圆形,类似蛋形。}的,有两层挑台\footnote{〔挑台〕从墙壁伸出的悬空台。挑:支起横的东西。},像两弯新月,围拱着主席台,层次分明,错落有致。两层挑台连地面共三层座席,有九千六百多个席位。主席台像个小会场,能容纳三百多人。底层席位的桌柜上都装有接收同声传译\footnote{〔同声传译〕在不打断讲话者的情况下,不间断地将内容同时翻译后讲给听众的一种翻译方式。听众同时听到讲话者的内容和翻译后的内容。}的耳机,每四个席位还有一个即席发言的话筒。下层挑台最前排也装有话筒,其余席位都有扬声喇叭。屋顶和挑台下的灯光,能够把礼堂的每个角落照得通明。
    
    尽管上下三层席位高低差距很大,底层面积达三千多平方米,最远处距离主席台有六十米,但是高大的礼堂中间没有一根柱子。建筑师画了一张草图,告诉我们,大礼堂顶上藏着比北京新扩建的长安街路面还要宽的十二榀\footnote{〔榀〕屋架的单位。}钢屋架\footnote{〔屋架〕用于承托屋顶的架构。多用于内部空间较大的建筑。}。其中六榀,一端压在一个九米高的钢筋混凝土横梁上,所有这些重量又一起压在主席台台口的两根柱子上,每根柱子都能承受三千多吨的重量。这样庞大而复杂的结构,该是一项多么艰巨的工程啊!说到这里,建筑师极力推崇建筑工人的伟大智慧和创造力。是他们在短短九个月的时间内,完成了这复杂的工程,还安装了声、电、冷热风、电视转播等各种现代化设备。
    
    人民大会堂的北翼是宴会厅,面临长安街。从人民大会堂北门进去,穿过大理石柱廊、风门厅、衣帽厅,就进入宴会厅底层大厅。这是宴前休息的场所。往前走,是五组六十二级的汉白玉大台阶,迎面墙壁上镶嵌着以毛主席《沁园春·雪》为主题的巨幅国画。画的一边是一片白茫茫的江山,“山舞银蛇,原驰蜡象”;另一边,云海茫茫中旭日东升,照耀大地,显得“江山如此多娇”。从这里经过东西两侧的走马廊\footnote{〔走马廊〕建筑物外围的宽阔走廊。},就进入宴会厅。
    
    有五千个席位的宴会厅,又是另一番景象。它的面积有七千平方米,比一个足球场还大,设计的精巧也是罕见的。天花和回廊\footnote{〔回廊〕环回的走廊。}圆柱装饰精美,雍容典雅。大厅的高度只有十五米多,但天花上运用了方井\footnote{〔方井〕一种室内顶饰手法。房顶四周向中央层层上凹,如倒过来的井,也叫藻井、天井、斗八。}的手法,显得明朗宽敞。
    
    建筑师还领我们参观了设置在大厅北面东西两角的厨房。厨房直通大厅两侧的回廊,开宴的时候,服务员可以从廊道进出宴席之间。厨房里的设备都是现代化的,上部厨房与地下室冷藏间和食品加工间等,都有专用电梯和楼梯上下运输。生冷和熟食,未洗和洗净的餐具,各有专线输送。
    
    人民大会堂的南翼是人大常委会办公楼。这是一座口字形的大楼,中间有六千平方米的庭院,里面一片草坪,是理想的集体摄影场地,也是幽静的休息场所。从这庭院穿过一座拱形的洞门,就到了人民大会堂的外面。
    
    看完这座大厦,一整天已经过去了。走出人民大会堂,只见万道霞光洒在苍翠的树丛上,洒在杏黄色的墙壁上,洒在天安门的红墙黄瓦上,溅出灿烂的异彩。
    
\end{normalsize}


\newpage

\textbf{注释}:

\vspace{-1em}

\begin{itemize}
    \setlength\itemsep{-0.2em}
    \item 〔大厦〕高大的房屋。
    \item 〔巍然〕高大的样子。
    \item 〔耸立〕高高地直立,矗立。
    \item 〔共商国是〕共同商议国家大计。国是:国家的重大问题。
    \item 〔枢纽〕枢:门户的转轴。纽:器物上用来提系的部分。比喻事物的关键部分,连起各方面的中心。
    \item 〔即席〕在座位上,不用离开座位。
    \item 〔雍容〕从容大方。
    \item 〔浑然一体〕完全融合,仿佛本就同属一个身体,不可分开。
    \item 〔推崇〕推重崇敬。推:指出优点,使成为优先考虑的对象。
    \item 〔穹窿〕中间高、四周下垂的形状。
    \item 〔错落有致〕交错纷杂而有美感、有趣。
    \item 〔溅〕液体受冲击向四周飞射。
\end{itemize}

\chapter{中国石拱桥}

\begin{normalsize}
    
    石拱桥的桥洞成弧形,就像虹。古代神话里说,雨后彩虹是“人间天上的桥”\footnote{〔“人间天上的桥”〕北欧神话中,连接中土人间与天界神国的桥是彩虹做的。},通过彩虹就能上天。我国的诗人爱把拱桥比作虹,说拱桥是“卧虹”“飞虹”,把水上拱桥形容为“长虹卧波”。
    
    石拱桥在世界桥梁史上出现得比较早。这种桥不但形式优美,而且结构坚固,能几十年几百年甚至上千年雄跨在江河之上,在交通方面发挥作用。
    
    我国的石拱桥有悠久的历史。《水经注》\footnote{〔《水经注》〕后魏郦道元编著的地理志,采用游记的形式写作,记载了大量地理风貌,文笔优美。}里提到的“旅人桥”,大约建成于公元282年,可能是有记载的最早的石拱桥了。我国的石拱桥几乎到处都有。这些桥大小不一,形式多样,有许多是惊人的杰作。其中最著名的当推河北省赵县的赵州桥,还有北京丰台区的卢沟桥。
    
    赵州桥横跨在洨河\footnote{〔洨河〕河北省滏阳河上游的一条季节性支流,从石家庄流入赵县。}上,是世界著名的古代石拱桥,也是造成后一直使用的最古的石桥。这座桥修建于公元605年左右,到1962年已经1300多年了,还保持着原来的雄姿。到解放的时候,桥身有些残损了,在人民政府的领导下,经过彻底整修,这座古桥又恢复了青春。
    
    赵州桥非常雄伟,全长50.82米,两端宽9.6米,中部略窄,宽约9米。桥的设计完全合乎科学原理,施工技术更是巧妙绝伦。唐朝的张嘉贞\footnote{〔张嘉贞〕唐开元年间宰相。}说它“制造奇特,人不知其所以为”。这座桥的特点是:一、全桥只有一个大拱,长达37.4米,在当时可算是世界上最长的石拱。桥洞不是普通半圆形,而是像一张弓,因而大拱上面的道路没有陡坡,便于车马上下。二、大拱的两肩上,各有两个小拱。这是创造性的设计,不但节约了石料,减轻了桥身的重量,而且在河水暴涨的时候,还可以增加桥洞的过水量,减轻洪水对桥身的冲击。同时,拱上加拱,桥身也更美观。三、大拱由28道拱圈拼成,就像这么多同样形状的弓合拢在一起,作成了一个弧形的桥洞。每道拱圈都能独立支撑上面的重量,一道坏了,其他各道不致受到影响。四、全桥结构匀称,和四周景色配合得十分和谐;桥上的石栏石板也雕刻得古朴美观。唐朝的张鷟\footnote{〔张鷟〕唐开元年间御史,著有《朝野佥载》,记述武则天时期的社会风貌。}说,远望这座桥就像“初月出云,长虹饮涧”。赵州桥高度的技术水平和不朽的艺术价值,充分显示出了我国劳动人民的智慧和力量。桥的主要设计者李春\footnote{〔李春〕隋代工匠,主持修建了赵州桥。}就是一位杰出的工匠,在桥头的碑文里刻着他的名字。
    
    永定河上的卢沟桥,修建于公元1189到1192年间。桥长265米,由11个半圆形的石拱组成,每个石拱长度不一,自16米到21.6米。桥宽约8米,桥面平坦,几乎与河面平行。每两个石拱之间有石砌桥墩,把11个石拱联成一个整体。由于各拱相联,所以这种桥叫做联拱石桥。永定河发水时,来势很猛,以前两岸河堤常被冲毁,但是这座桥极少出事,足见它的坚固。桥面用石板铺砌,两旁有石栏石柱。每个柱头上都雕刻着不同姿态的狮子。这些石刻狮子,有的母子相抱,有的交头接耳,有的像倾听水声,有的像注视行人,千态万状,惟妙惟肖。
    
    卢沟桥地处入都要道,而且建筑优美,历来为人们所称赞。“卢沟晓月”很早就成为北京的胜景之一。
    
    卢沟桥在我国人民反抗帝国主义侵略战争的历史上,也是值得纪念的。1937年7月7日中国军队在此抗击日本帝国主义的侵略,揭开了中国人民全面抗战的序幕。
    
    为什么我国的石拱桥会有这样光辉的成就呢?首先,在于我国劳动人民的勤劳和智慧。他们制作石料的工艺极其精巧,能把石料切成整块大石碑,又能把石块雕刻成各种形象。在建筑技术上有很多创造,在起重吊装方面更有意想不到的办法。如福建漳州的江东桥,修建于700多年前,有的石梁一块就有200来吨重,究竟是怎样安装上去的,还不完全知道。其次,我国石拱桥的设计有优良传统,建成的桥,用料省,结构巧,强度高。再其次,我国富有建筑用的各种石料,便于就地取材,这也为修造石桥提供了有利条件。
    
    两千年来,我国修建了无数的石拱桥。解放后,全国大规模兴建起各种形式的公路桥和铁路桥。其中就有不少石拱桥。1961年,云南省建成了一座世界最长的独拱石桥,名叫“长虹大桥”,石拱长达112.5米。在传统的石拱桥的基础上,我们还造了大量的钢筋混凝土拱桥,其中“双曲拱桥”是我国劳动人民的新创造,是世界上所仅有的。这几年来,全国造了总长二十余万米的这种拱桥,其中最大的一孔,长达150米。我国桥梁事业的飞跃发展,表明了我国社会主义制度的无比优越。
    
\end{normalsize}


\newpage

\textbf{注释}:

\vspace{-1em}

\begin{itemize}
    \setlength\itemsep{-0.2em}
    \item 〔历来〕多年以来,一直以来,从来。
    \item 〔惟妙惟肖〕形容制作巧妙逼真,仿佛真的一样。
\end{itemize}

\chapter{漫谈无理数}

\begin{normalsize}
    
    传说在两千多年前,古希腊\footnote{〔古希腊〕公元前欧洲巴尔干半岛南部、爱琴海诸岛和小亚细亚沿岸的一些城邦的合称。}有个智者叫毕达哥拉斯。他对数的研究很深,认为世间万物都能用数来解释。这里的数,指的是1、2、3……这样的整数。他用数与数的比例解释了很多自然现象。他的弟子和崇拜者创建了以他为名的教派\footnote{〔教派〕这里指因信仰形成的团体。},提出“万物皆数”,万物按照数和数的比例构成完美和谐的秩序。
    
    有一天,教派中的一名弟子希帕索斯思考了一个问题:已知一个正方形,如何画一个面积是它两倍的正方形?这个问题并不难。把已知正方形的对角线作为边长的正方形,面积就是它的两倍。希帕索斯进一步思考:这个新正方形的边长是多少呢?
    
    如果把原来的正方形的边长记作1,希帕索斯要找的就是平方等于2的数。按毕达哥拉斯的思想,这个数必定是整数或整数之比。但是,希帕索斯证明了:这个数无法表示成任何两个整数的比!
    
    这个结论让毕达哥拉斯教派大为恐慌。他们怕希帕索斯泄露这个秘密,就把他沉到海里杀害了。
    
    远古的传说已经难以考证,但是,发现“无法表示成整数之比的数”,确实是数学史的一个里程碑。它说明了,仅仅用整数的加减乘除,无法处理现实中图形的问题。要计算关于三角形、长方形的问题,数学家就必须引入一种新的数。毕达哥拉斯教派认为数是万物的道理,能用整数之比表示的数叫做有理数,于是“平方等于2”这样不能表示为整数之比的数就被称为无理数。
    
    无理数多吗?我们把“平方等于整数的数”称为整数的平方根。如果仅仅考虑整数的平方根,我们可能会觉得无理数不算多。比如,“平方等于4的数”就是2。也就是说,不是所有整数都会生出无理数。这样看来,无理数似乎比有理数少。
    
    然而事实并非如此。首先,无理数可不仅仅包括“平方等于2”这样的数。我们还有“立方等于2的数”、“5次方等于2的数”等等。我们把“乘方\footnote{〔乘方〕数自乘的结果。}等于整数的数”称为方根。除了平方根,还有立方根、4次方根等等。比如4的立方根就是“立方等于4的数”,5的4次方根就是“4次方等于5的数”,等等。这些方根几乎都是无理数。
    
    此外,还有不是方根的无理数。圆的周长与直径之比叫做圆周率,它也是无理数。而且,圆周率的平方、立方、4次方……都不是有理数。19世纪的数学家康托\footnote{〔康托〕戈奥格·康托,19世纪德国数学家,现代集合论创立者。}严格证明了,无理数比有理数“多得多”。
    
    和有理数相比,无理数不仅多,而且杂乱无章,难以理解。方根是最容易理解的无理数。它和自身相乘,就能得到整数。又比如著名的黄金分割率\footnote{〔黄金分割〕把整体分为大小两部分,使整体与较大部分之比等于较大部分与较小部分之比。其中的比率称为黄金分割率或黄金比率。},它的平方减去自己是1。能够通过和自身及整数的加减乘除变成整数,这样的无理数称为代数数。换个角度\footnote{〔角度〕这里指看问题的出发点。}来说,代数数就是整数系数的整式方程的解。
    
    然而,无理数可不全是代数数。比如,圆周率就不是任何整数系数的整式方程的解。换句话说,把圆周率和整数一起做加减乘除,无论做多少次,都无法变成整数。相比2的方根或黄金分割率,圆周率这样的无理数更加难以用整数描述,不可捉摸。我们把这样的无理数称为超越数。
    
    超越数多吗?数学家遗憾地告诉我们,比起超越数,代数数和有理数的数量都可以“忽略不计”。无理数几乎都是超越数。
    
    直到21世纪,大多数数学家对无理数的研究,仍然停留在代数数上。对于超越数,我们只了解其中极少的一部分。
    
    借助无限的概念,我们能够理解关于一些超越数的性质。比如,借助极限\footnote{〔极限〕数学中指无限逼近的结果。}的概念,我们可以定义所有正整数的平方倒数的和\footnote{〔平方倒数〕指平方的倒数,比如2的平方倒数为四分之一。},并且证明它等于圆周率平方的六分之一。
    
    不仅如此,所有正整数的偶数次方倒数的和,都等于圆周率的偶数次方乘以某个有理数。但是,关于所有正整数的奇数次方倒数的和,我们所知甚少。
    
    1978年,法国数学家阿佩里证明了:所有正整数的立方倒数的和是无理数。它是否是超越数呢?这仍是个未解之谜。有人猜想,它等于圆周率的立方乘以某个有理数,但还没人能够证明。
    
    很多时候,我们竟然无法知道,数学研究中发现的数,是否是无理数。比如著名的欧拉常数,它是调和级数与自然对数的差的极限\footnote{〔……差的极限〕调和级数:正整数倒数的和。自然对数:自然科学中的一种基本函数。欧拉常数指前$n$个正整数的倒数的和减去$n$的自然对数的差在$n$逼近无限大时的极限。},大约等于0.577。我们至今不知道它是否是无理数。更有一种叫做“不可定义数”的,它们无法用有限的文字确定,更不用说进行研究了。
    
    直到今天,无理数中仍然埋藏着许多秘密,等着我们去发现。
    
\end{normalsize}


\newpage

\textbf{注释}:

\vspace{-1em}

\begin{itemize}
    \setlength\itemsep{-0.2em}
    \item 〔智者〕聪明的人,有智慧的人。
    \item 〔和谐〕多个声音合成好听的声音。指配合恰当。
    \item 〔秩序〕各个部分有条理有先后的状态和规则。
    \item 〔里程碑〕设置在路旁记录里数的标志。比喻历史发展过程中的重大事件。
    \item 〔杂乱无章〕无条理,无规律。
    \item 〔证明〕用确实的证据和推理说明。
    \item 〔遗憾〕这里指因为无法改变的不足而可惜。
\end{itemize}

\chapter{向沙漠进军}

\begin{normalsize}
    
    沙漠是人类最顽强的自然敌人之一。有史以来,人类就同沙漠不断地斗争。但是从古代的传说和史书的记载看来,过去人类没有能征服沙漠,若干住人的地区反而为沙漠所并吞。
    
    地中海沿岸被称为西方文明的摇篮。古代埃及、巴比伦和希腊\footnote{〔古代埃及……〕亚欧非交界地带的古代文明,现称古埃及、古巴比伦、古希腊文明。}的文明都是在这里产生和发展起来的。但是两三千年来,这个区域不断受到风沙的侵占,有些部分逐渐变成荒漠了。
    
    中国陕西榆林地区,雨量还充沛,在明末清初的时候是个天然草原区,没有多少风沙。到了清朝乾隆年间,陕西和山西北部许多人移居到榆林以北关外去开垦。当时的政府根本不关心农业生产事业,生产技术又不高,垦荒伐木,致使原来的草地露出了泥土,日晒风吹,尘沙就到处飞扬。由于长城外的风沙侵入,榆林城也受袭击,到解放以前,榆林地区关外30公里都变成沙漠了。
    
    沙漠逞强施威,所用的武器是风和沙。风沙的进攻主要有两种方式。一种可以称为“游击战”。狂风一起,沙粒随风飞扬,风愈大,沙的打击力愈强。春天四五月间禾苗刚出土,正是狂风肆虐的时候。一次大风沙袭击,可以把幼苗全部打死,甚至连根拔起。沿长城一带风沙大的地区,农民常常要补种两三次才能有点收获。一种可以称为“阵地战”,就是风推动沙丘,缓缓前进。沙丘的高度一般从几米到几十米,也有高达100米以上的。沙丘的前进并不是整体移动的。当风速达到每秒5米以上的时候,沙丘迎风面的沙粒就成批地随风移动,从沙丘的底部移到顶部,过了顶部,由于风速减弱,就在背风面的坡上落下。所以部分沙粒的移动速度虽然相当快,每天可以移动几米到几十米,可是整个沙丘波浪式地前进,移动速度并不快,每年不过5米到10米。几个沙丘常常联在一起,成为沙丘链。沙丘的移动虽然慢,可是所到之处,森林全被摧毁,田园全被埋葬,城郭变成丘墟。
    
    抵御风沙袭击的方法是培植防护林。防护林的主要作用是减小风的力量。风遇到防护林,速度就减小70\%~80\%。到距离防护林等于林木高度20倍的地方,风又恢复原来的速度。所以防护林必须是并行排列的许多林带,两列之间的距离不要超过林木高度的20倍。其次是培植草皮。有了草皮覆盖地面,即使有风,刮起的沙也不多,这就减少了沙粒的来源。
    
    抵御沙丘进攻的方法是植树种草。中国沙荒地区,有一部分沙丘已经长了草皮和灌木,不再转移阵地了。这种固定的沙丘,只要能妥善保护草皮和灌木,防止过度砍伐和任意放牧,就可以固定下来。根据近年治沙的经验,陕北榆林、内蒙古磴口、甘肃民勤地区的流动沙丘,表面干沙层的厚度一般不超过10厘米。10厘米以下,水分含量逐渐增大,到40厘米的深处,水分含量达到2\%以上,这就是湿沙层了。湿沙层的水分足够供应固定沙丘的植物的需要。所以在流动沙丘上植树种草,是可以成活的。林木和草类成长以后,沙丘就可以固定下来了。
    
    仅仅防御风沙袭击,固定沙丘阵地,还只是采取守势,自然是不够的。征服沙漠的最主要的武器是水。无论植树还是种草,土壤中必须有充足的水分。所以要取得向沙漠进军的胜利,必须有充足的水源。
    
    中国内蒙古东部和陕西、山西北部有足够的雨量。就是西北干旱地区,地面径流\footnote{〔径流〕自然降水沿重力沿地面或地下流动的水流。}和地下潜水也是很大的。有些沙荒地区,如河西走廊\footnote{〔河西走廊〕甘肃省西北部祁连山以北,合黎山、龙首山以南,乌鞘岭以西一带。自古为通往新疆的要道。}、柴达木、新疆北部准噶尔和新疆南部塔里木,都是盆地,周围的高山上有大量的积雪。这样看来,只要能充分利用这些水源,我们向沙漠进军不但有收复失地的把握,而且能在大沙漠里开辟出若干绿洲来。普通河流愈到下游,水量愈多,河流愈大。但在沙漠中,一部分水被蒸发到空中,一部分浸入到土壤岩隙中成为地下水,河流反而愈流愈小,终至于干涸不见。如地质构造\footnote{〔地质构造〕地壳岩石的构成方式、各部分形态及面貌特征。}是一个盆地,则能汇成地下海,可以作为建立绿洲的水源。据中国科学院综合考察委员会的调查,只要有水源,单新疆尚有一亿亩荒地可以开垦。
    
    沙漠是可以治理的。中国在治理沙漠方面已经取得了若干成绩。新疆生产建设兵团在天山\footnote{〔天山〕新疆西北部山脉。}南北建立国营农场,开沟挖渠,种麦种棉植树,那里原是不毛之地,现在一片葱茏,俨然成为绿洲。内蒙古沙荒区的治沙工作也获得不少成绩。
    
    我们向沙漠进军,不但保护了农田,开辟了绿洲,而且对交通线路也起了防护作用。包兰铁路从银川到兰州的一段,要经过腾格里沙漠,其间中卫县沙坡头一带,风沙特别厉害。那里沙多风大,一次大风沙就可以把铁路淹没。有关部门\footnote{〔有关部门〕指中国科学院。}在1956年成立了沙坡头治沙站,进行固沙造林。这一工作已经提前完成。包兰铁路通车以来,火车在沙漠上行驶,从来没有因为风沙的侵袭而发生事故。
    
    风是沙漠向人类进攻的武器,但是也可以为人类造福。沙漠地区地势平坦,风力很强。如新疆的星星峡、托克逊、达坂城都是著名的风口。中国科学院力学研究所在托克逊地方试制了半径两米的风力车,可以供发电、汲水、磨面之用。
    
    沙漠地区空气干燥,日光的照射特别强烈。那里日照时间又特别长,一年达到3000小时,而长江流域只有1500小时,华北地区也不过2500小时。日光可以用来发电,取暖,煮水,做饭。沙漠湖水含盐,日光使水蒸发,可以取得蒸馏水\footnote{〔蒸馏〕加热液体使变成蒸气,,再使蒸气冷却凝成液体,从而除去其中的杂质。}和盐。把日光变为热能和电能的最良好的工具是半导体\footnote{〔半导体〕导电能力介于导体和绝缘体之间的材料。},估计将来有可能在沙漠里用便宜的半导体做屋顶,人住在里边冬天不冷,夏天不热。
    
    从上面介绍的一些情况,可以清楚地认识到,只要我们正确地认识沙漠的危害,找出对付它的办法,沙漠是有可能治理的。
    
\end{normalsize}


\newpage

\textbf{注释}:

\vspace{-1em}

\begin{itemize}
    \setlength\itemsep{-0.2em}
    \item 〔肆虐〕任意干残暴的事情。
    \item 〔治理〕整理、处理好。治:消除水灾。
    \item 〔不毛之地〕长不出作物的荒地。
    \item 〔丘墟〕废墟。
    \item 〔抵御〕挡,抗,防。
    \item 〔地势〕地面高低起伏的情况。
    \item 〔开垦〕把荒地开辟成可以种植的土地。垦:翻土。
    \item 〔干涸〕河流池塘没水。
    \item 〔汲水〕从下往上取水。
    \item 〔葱茏〕植物丛聚茂盛的样子。
    \item 〔俨然〕很像。
\end{itemize}

\chapter{白杨礼赞}

\begin{normalsize}
    
    白杨树实在不是平凡的,我赞美白杨树!
    
    当汽车在望不到边际的高原上奔驰,扑入你的视野的,是黄绿错综的一条大毯子;黄的,是土,未开垦的荒地,几十万年前由伟大的自然力堆积而成的黄土高原的外壳;绿的呢,是人类劳力战胜自然的成果,是麦田,和风吹送,翻起了一轮一轮的绿波——这时你会真心佩服昔人所造的两个字“麦浪”,若不是妙手偶得,便确是经过锤炼的语言的精华。黄与绿主宰着,无边无垠,坦荡如砥,这时如果不是宛若并肩的远山的连峰提醒了你,你会忘记了汽车是在高原上行驶,这时你涌起来的感想也许是“雄壮”,也许是“伟大”,诸如此类的形容词,然而同时你的眼睛也许觉得有点倦怠,你对当前的“雄壮”或“伟大”闭了眼,而另一种味儿在你心头潜滋暗长了—— “单调”。可不是,单调,有一点儿吧?
    
    然而刹那间,要是你猛抬眼看见了前面远远地有一排,——不,或者甚至只是三五株,一二株,傲然地耸立,象哨兵似的树木的话,那你的恹恹欲睡的情绪又将如何?我那时是惊奇地叫了一声的!
    
    那就是白杨树,西北极普通的一种树,然而实在是不平凡的一种树!
    
    那是力争上游的一种树,笔直的干,笔直的枝。它的干,通常是丈把高,像加以人工似的,一丈以内,绝无旁枝;它所有的丫枝一律向上,而且紧紧靠拢,也像加以人工似的,成为一束,绝不旁逸斜出;它的宽大的叶子也是片片向上,几乎没有斜生的,更不用说倒垂了;它的皮光滑而有银色的晕圈,微微泛出淡青色。这是虽在北方风雪的压迫下却保持着倔强挺立的一种树。哪怕只有碗那样粗细,它却努力向上发展,高到丈许,二丈,参天耸立,不折不挠,对抗着西北风。
    
    这就是白杨树,西北极普通的一种树,然而决不是平凡的树!
    
    它没有婆娑的姿态,没有屈曲盘旋的虬枝,也许你要说它不美,如果美是专指“婆娑”或“旁斜逸出”之类而言,那么,白杨树算不得树中的好女子;但是它伟岸,正直,朴质,严肃,也不缺乏温和,更不用提它的坚强不屈与挺拔,它是树中的伟丈夫!当你在积雪初融的高原上走过,看见平坦的大地上傲然挺立这么一株或一排白杨树,难道你觉得树只是树?难道你就不想到它的朴质,严肃,坚强不屈,至少也象征了北方的农民?难道你竟一点也不联想到,在敌后的广大土地上,到处有坚强不屈,就象这白杨树一样傲然挺立的守卫他们家乡的哨兵?难道你又不更远一点想到这样枝枝叶叶靠紧团结,力求上进的白杨树,宛然象征了今天在华北平原纵横决荡用血写出新中国历史的那种精神和意志?
    
    白杨不是平凡的树。它在西北极普遍,不被人重视,就跟北方农民相似;它有极强的生命力,磨折不了,压迫不倒,也跟北方的农民相似。我赞美白杨树,就因为它不但象征了北方的农民,尤其象征了今天我们民族解放斗争中所不可缺的朴质,坚强,力求上进的精神。
    
    让那些看不起民众,贱视民众,顽固的倒退的人们去赞美那贵族化的楠木,去鄙视这极常见,极易生长的白杨吧,我要高声赞美白杨树!
    
\end{normalsize}


\newpage

\textbf{注释}:

\vspace{-1em}

\begin{itemize}
    \setlength\itemsep{-0.2em}
    \item 〔边际〕边界(多指地区和空间)。
    \item 〔视野〕眼睛看到的空间范围;眼界。
    \item 〔错综〕纵横交错。
    \item 〔开垦〕把荒地开辟成可以种植的土地。
    \item 〔妙手偶得〕指文学素养深的人偶然间所得到的。出自陆游《文章》:“文章本天成,妙手偶得之。”妙手:技艺高超的人 。
    \item 〔精华〕最重要、最好的部分。
    \item 〔主宰〕支配,有决定权。
    \item 〔诸如此类〕与此相似的种种事物。
    \item 〔力争上游〕努力奋斗,争取先进。
    \item 〔旁逸斜出〕意思是(树枝)从树干的旁边斜伸出来。逸:逃。
    \item 〔姿态〕姿势,样儿。还可指态度、气度。
    \item 〔屈曲〕弯曲、曲折的意思。
    \item 〔盘旋〕环绕着飞或走。
    \item 〔伟岸〕高大挺拔。
    \item 〔朴质〕质朴。
    \item 〔温和〕(性情)温柔平和。
    \item 〔坚强不屈〕坚毅刚强,不屈服。
    \item 〔宛然〕仿佛,好像。
    \item 〔纵横决荡〕纵横四方,冲杀突击。
    \item 〔磨折〕磨难,挫折。
    \item 〔贱视〕轻视。
    \item 〔秀颀〕美而高。
    \item 〔鄙视〕轻视,看不起。
    \item 〔坦荡如砥〕平坦得像磨刀石磨过的样子。
    \item 〔恹恹〕精神不好,困倦的样子。
\end{itemize}

\chapter{苏州园林}

\begin{normalsize}
    
    苏州园林据说有一百多处,我到过的不过十多处。其他地方的园林我也到过一些。倘若要我说说总的印象,我觉得苏州园林是我国各地园林的标本,各地园林或多或少都受到苏州园林的影响。因此,谁如果要鉴赏我国的园林,苏州园林就不该错过。
    
    设计者和匠师们因地制宜,自出心裁,修建成功的园林当然各个不同。可是苏州各个园林在不同之中有个共同点,似乎设计者和匠师们一致追求的是:务必使游览者无论站在哪个点上,眼前总是一幅完美的图画。为了达到这个目的,他们讲究亭台轩榭\footnote{〔亭台轩榭〕泛指园林建筑。亭:有顶无墙的小屋。台:供登高望远的平楼。轩:敞窗的小屋或长廊。榭:建在土台或水上的木屋。}的布局,讲究假山\footnote{〔假山〕园林中以造景为目的,用土、石等材料构筑的山。}池沼的配合,讲究花草树木的映衬,讲究近景远景的层次。总之,一切都要为构成完美的图画而存在,决不容许有欠美伤美的败笔。他们惟愿游览者得到“如在画图中”的美感,而他们的成绩实现了他们的愿望,游览者来到园里,没有一个不心里想着口头说着“如在画图中”的。
    
    我国的建筑,从古代的宫殿到近代的一般住房,绝大部分是对称的,左边怎么样,右边也怎么样。苏州园林可绝不讲究对称,好像故意避免似的。东边有了一个亭子或者一道回廊,西边决不会来一个同样的亭子或者一道同样的回廊。这是为什么?我想,用图画来比方,对称的建筑是图案画,不是美术画,而园林是美术画,美术画要求自然之趣,是不讲究对称的。
    
    苏州园林里都有假山和池沼。假山的堆叠,可以说是一项艺术而不仅是技术。或者是重峦叠嶂,或者是几座小山配合着竹子花木,全在乎设计者和匠师们生平多阅历,胸中有丘壑,才能使游览者攀登的时候忘却苏州城市,只觉得身在山间。至于池沼,大多引用活水\footnote{〔活水〕有源头长流不断的水。}。有些园林池沼宽敞,就把池沼作为全园的中心,其他景物配合着布置。水面假如成河道模样,往往安排桥梁。假如安排两座以上的桥梁,那就一座一个样,决不雷同。池沼或河道的边沿很少砌齐整的石岸,总是高低屈曲任其自然。还在那儿布置几块玲珑的石头,或者种些花草:这也是为了取得从各个角度看都成一幅画的效果。池沼里养着金鱼或各色鲤鱼,夏秋季节荷花或睡莲开放,游览者看“鱼戏莲叶间”,又是入画的一景。
    
    苏州园林栽种和修剪树木也着眼在画意。高树与低树俯仰生姿。落叶树与常绿树相间,花时不同的多种花树相间,这就一年四季不感到寂寞。没有修剪得像宝塔那样的松柏,没有阅兵式似的道旁树:因为依据中国画的审美观点看,这是不足取的。有几个园里有古老的藤萝,盘曲嶙峋的枝干就是一幅好画。开花的时候满眼的珠光宝气,使游览者感到无限的繁华和欢悦,可是没法说出来。
    
    游览苏州园林必然会注意到花墙和廊子。有墙壁隔着,有廊子界着,层次多了,景致就见得深了\footnote{〔深〕景色深,指景观有层次,移步换景,不会一览无余。}。可是墙壁上有砖砌的各式镂空图案,廊子大多是两边无所依傍的,实际是隔而不隔,界而未界,因而更增加了景致的深度。有几个园林还在适当的位置装上一面大镜子,层次就更多了,几乎可以说把整个园林翻了一番。
    
    游览者必然也不会忽略另外一点,就是苏州园林在每一个角落都注意图画美。阶砌旁边栽几丛书带草\footnote{〔书带草〕多年生草本植物,多见于沟旁及山坡草丛,庭园绿化植物。}。墙上蔓延着爬山虎或者蔷薇木香\footnote{〔蔷薇木〕也叫红木,常用来做家具的名贵木材。}。如果开窗正对着白色墙壁,太单调了,给补上几竿竹子或几棵芭蕉。诸如此类,无非要游览者即使就极小范围的局部看,也能得到美的享受。
    
    苏州园林里的门和窗,图案设计和雕镂琢磨功夫都是工艺美术的上品。大致说来,那些门和窗尽量工细\footnote{〔工细〕指做工细致不马虎,注重细节。}而决不庸俗,即使简朴而别具匠心。四扇,八扇,十二扇,综合起来看,谁都要赞叹这是高度的图案美。摄影家挺喜欢这些门和窗,他们斟酌着光和影,摄成称心满意的照片。
    
    苏州园林与北京的园林不同,极少使用彩绘。梁和柱子以及门窗栏杆大多漆广漆\footnote{〔广漆〕生漆或熟漆中加入桐油制成,棕黑色,又叫金漆,是明清木家具常用涂料。},那是不刺眼的颜色。墙壁白色。有些室内墙壁下半截铺水磨方砖,淡灰色和白色对衬。屋瓦和檐漏一律淡灰色。这些颜色与草木的绿色配合,引起人们安静闲适的感觉。花开时节,更显得各种花明艳照眼。
    
    可以说的当然不止以上这些,这里不再多写了。
    
\end{normalsize}


\newpage

\textbf{注释}:

\vspace{-1em}

\begin{itemize}
    \setlength\itemsep{-0.2em}
    \item 〔因地制宜〕根据不同地方的具体条件,制定相应的妥善措施
    \item 〔自出心裁〕出于自己心中的设计,指构思独到。
    \item 〔嶙峋〕形容山石高直而细,突兀而立。引申为人消瘦或刚直有骨气。
    \item 〔斟酌〕计算着倒酒。引申为反复考虑以后决定取舍。
    \item 〔阅历〕亲身见闻、经历。
    \item 〔檐漏〕屋檐和墙之间的空档。空档中的椽,梁,枋等部分都外露,梁枋上住往做出丰富多彩的装饰,非常漂亮,是讲究的做法。
    \item 〔俯仰生姿〕指高低错落,如人俯仰的姿态。
    \item 〔重峦叠嶂〕山峰连绵不绝。
    \item 〔珠光宝气〕珍珠美玉的光辉。
\end{itemize}

\chapter{万紫千红的花}

\begin{normalsize}
    
    又到了赏花的季节。看着争相开放的花朵,你有没有想过,花为什么有这么多美丽鲜艳的色彩呢?
    
    花怎么会有各种美丽鲜艳的色彩呢?这是由于花瓣的细胞液\footnote{〔细胞液〕细胞内的液体。}中存在着色素\footnote{〔色素〕生物体内使生物呈现各种颜色的物质。}。有一些花的颜色是红的、蓝的或紫的。这些花里含的色素叫花青素。花青素平常是紫色,遇到酸就变红,遇到碱就变蓝。你可以拿一朵喇叭花\footnote{〔喇叭花〕园林花卉,旋花科缠绕草本植物,也叫牵牛花。}来做实验,把红色的喇叭花泡在肥皂水里,它很快就变成蓝色,因为肥皂是碱性的。再把这朵蓝色的花泡到醋里,它又重新变成红色,因为醋是酸性的。
    
    还有一些花的颜色是黄的、橙黄的、橙红的。它们的花瓣里含的色素叫胡萝卜素。胡萝卜素最初是在胡萝卜里发现的,有六十多种。柑橘、南瓜的颜色,也来自胡萝卜素。含有胡萝卜素的花也是五颜六色的。
    
    白色的花含有什么色素呢?白色的花什么色素也没有。它看来是白色的,那是因为花瓣里充满了小气泡的缘故。你拿一朵白花来,用手捏一捏花瓣,把里面的小气泡挤掉,它就成为无色透明的了。
    
    各种花含有的色素和酸、碱的浓度不一样。随着养料、水分、温度等条件经常在变化,花的颜色就有深有浅,有浓有淡,有的还会变色。
    
    会变色的花很多。例如红喇叭花,它初开的时候是红色,败落的时候就变成紫色了。杏花含苞待放的时候是红色,开放后逐渐变淡,最后几乎变成白色了。最有趣的要数“弄色木芙蓉\footnote{〔弄色木芙蓉〕我国传统观赏花卉,锦葵科落叶灌木,花色多变,也叫文官花。}”。它的花初开是白色,第二天变成浅红色,后来又变成深红色,到花落的时候又变成紫色了。这些变化看起来很玄妙,其实都是花内色素随着温度和酸碱浓度变化所玩的把戏。
    
    我国有种樱草\footnote{〔樱草〕我国传统观赏花卉,报春花科草本植物,花色多样。},在普通温度下,花是红色。在30摄氏度的暗室里,就变成白色了。八仙花\footnote{〔八仙花〕也叫绣球花,我国传统观赏花卉,绣球科落叶灌木,花密集成簇,花色多样。}在有些土壤中开蓝色的花,在另一些土壤中开粉红色的花。还有一些花,受精\footnote{〔受精〕花的雌蕊中的卵细胞接受来自雄蕊的精细胞的过程。}以后也会变色。比如海桐花\footnote{〔海桐〕我国传统植物,海桐科常绿灌木,常用于防风绿化。},起初是黄色,受精后就变成白色了。红锦带花\footnote{〔锦带花〕观赏花卉,忍冬科落叶灌木,花红色。}受精后,也会变成白色。
    
    有人统计了4197种花的颜色。做了如下的分类:
    
    \begin{center}\begin{tabular}{| c | c | c | c | c | c | c | c | c | c |}
    
    \hline
    
    颜色 & 白 & 黄 & 红 & 蓝 & 紫 & 绿 & 橙 & 茶 & 黑 \\
    
    \hline
    
    数量 & 1193 & 951 & 923 & 594 & 307 & 153 & 50 & 18 & 8 \\
    
    \hline
    
    \end{tabular}\end{center}
    
    从这个统计可以看出,白色、黄色和红色的花最多。这三种颜色的花有个好处,配着绿叶非常鲜艳,容易惹昆虫注意。
    
    昆虫对花的颜色也是有选择的。比如蜜蜂就不大喜欢黄色,而喜欢红色和蓝色。更有趣的是有些花还选择昆虫。例如金鱼草\footnote{〔金鱼草〕车前科草本植物,常见庭园花卉。},他的花平时闭合着,等到它所喜爱的一种小蜂飞来的时候,花就立即开放了。别的小昆虫来“叩门”,它理也不理。还有待宵草\footnote{〔待宵草〕观赏花卉,柳叶菜科草本植物,花黄色。},它的花到夜间才能张开笑脸。这时候,有一种白天躲在阴暗地方的小蛾,就飞来帮它传送花粉。夜间开的花,大多是白色或黄色的,否则在黑暗中就不容易被昆虫发现。
    
    美丽的花朵对人有很大的吸引力。意大利的诗人但丁\footnote{〔但丁〕但丁·阿利杰里,13世纪意大利诗人,文艺复兴的先驱。主要作品为长诗《神曲》三部曲。}在他的《神曲》中写道:
    
    \begin{quotation}
    
    我向前走,但我一看到花,脚步就慢下来了。
    
    \end{quotation}
    
    世界上恐怕没有人不喜爱花。人们用万紫千红的花来点缀生活环境,用它的形象来装饰衣服和用具,把它作为美丽、纯洁和幸福的象征。
    
\end{normalsize}


\newpage

\textbf{注释}:

\vspace{-1em}

\begin{itemize}
    \setlength\itemsep{-0.2em}
    \item 〔浓度〕度量浓淡的量。
    \item 〔败落〕花开过后衰落。
    \item 〔玄妙〕事物的道理深奥难明。
    \item 〔点缀〕以少量衬托,装饰。缀:将小块布连起来,把小物连到物件边缘。
\end{itemize}

\chapter{在烈日和暴雨下}

\begin{normalsize}
    
    六月十五\footnote{〔六月十五〕这里指农历六月十五日。}那天,天热得发了狂。太阳刚一出来,地上已经像下了火。一些似云非云似雾非雾的灰气低低地浮在空中,使人觉得憋气。一点风也没有。祥子在院子里看了看那灰红的天,喝了瓢凉水就走出去。
    
    街上的柳树像病了似的,叶子挂着层灰土在枝上打着卷;枝条一动也懒得动,无精打采地低垂着。马路上一个水点也没有,干巴巴地发着白光。便道上尘土飞起多高,跟天上的灰气联接起来,结成一片毒恶的灰沙阵,烫着行人的脸。处处干燥,处处烫手,处处憋闷,整个老城像烧透了的砖窑,使人喘不过气来。狗趴在地上吐出红舌头,骡马的鼻孔张得特别大,小贩们不敢吆喝,柏油路晒化了,甚至于铺户门前的铜牌好像也要晒化。街上非常寂静,只有铜铁铺里发出使人焦躁的一些单调的丁丁当当。拉车的人们,只要今天还不至于挨饿,就懒得去张罗买卖:有的把车放在有些阴凉的地方,支起车棚,坐在车上打盹;有的钻进小茶馆去喝茶;有的根本没有拉出车来,只到街上看看有没有出车的可能。那些拉着买卖的,即使是最漂亮的小伙子,也居然甘于丢脸,不敢再跑,只低着头慢慢地走。每一口井都成了他们的救星,不管刚拉了几步,见井就奔过去,赶不上新汲的水,就跟驴马同在水槽里灌一大气。还有的,因为中了暑,或是发痧\footnote{〔痧〕病名。由饮食不干净引起,患者又吐又泻,四肢发凉,严重时会失去知觉。由于常在热天发病,一般认为是中暑。},走着走着,一头栽到地上,永不起来。
    
    祥子有些胆怯了。拉着空车走了几步,他觉出从脸到脚都被热气围着,连手背上都流了汗。可是见了座儿\footnote{〔座儿〕指想要坐车的人。}他还想拉,以为跑起来也许倒能有点风。他拉上了个买卖,把车拉起来,他才晓得天气的厉害已经到了不允许任何人工作的程度。一跑,就喘不上气来,而且嘴唇发焦,明明心里不渴,也见水就想喝。不跑呢,那毒花花的太阳把手和脊背都要晒裂。好歹拉到了地方,他的裤褂全裹在了身上。拿起芭蕉扇扇扇,没用,风是热。他已经不知喝了几气凉水,可是又跑到茶馆去。
    
    两壶热茶喝下去,他心里安静了些。茶从嘴里进去,汗马上从身上出来,好像身子已经是空膛的,不会再储藏一点水分。他不敢再动了。
    
    坐下了好久,他心里腻烦\footnote{〔腻烦〕这里指感觉心烦。}了。既不敢出去,又没事可作,他觉得天气仿佛成心跟他过不去。想出去,可是腿真懒得动,身上非常软,好像洗澡没洗痛快那样,汗虽然出了不少,心里还是不舒畅。又坐了会儿,他再也坐不住了,反正坐着也是出汗,不如爽性出去试试。
    
    一出来,才晓得自己错了。天上的那层灰气已经散开,不很憋闷了,可是阳光也更厉害了:没人敢抬头看太阳在哪里,只觉得到处都闪眼,空中,屋顶上,墙壁上,地上,都白亮亮的,白里透着点红,从上至下整个地像一面极大的火镜,每一条光都像火镜的焦点,晒得东西要发火。在这个白光里,每一个颜色都刺目,每一个声响都难听,每一种气味都搀合着地上蒸发出来的腥臭。街上仿佛没了人,道路好像忽然加宽了许多,空旷而没有一点凉气,白花花的令人害怕。祥子不知怎么是好了,低着头,拉着车,慢腾腾地往前走,没有主意,没有目的,昏昏沉沉的,身上挂着一层粘汗,发着馊臭的味儿。走了会儿,脚心跟鞋袜粘在一块,好像踩着块湿泥,非常难过,本来不想再喝水,可是见了井不由得又过去灌了一气,不为解渴,似乎专为享受井水那点凉气,从口腔到胃里,忽然凉了一下,身上的毛孔猛地一收缩,打个冷战,非常舒服。喝完,他连连地打嗝,水要往上漾\footnote{〔漾〕这里是胃里的东西向上涌的意思。}。
    
    走一会儿,坐一会儿,他始终懒得张罗买卖。一直到了正午,他还觉不出饿来。想去照例地吃点什么,可是看见食物就要恶心。胃里差不多装满了各样的水,有时候里面会轻轻地响,像骡马喝完水那样,肚子里光光光地响动。
    
    正在午后一点的时候,他又拉上个买卖。这是一天里最热的时候,又赶上这一夏里最热的一天。刚走了几步,他觉到一点凉风,就像在极热的屋里从门缝进来一点凉气似的。他不敢相信自己;看看路旁的柳枝,的确微微地动了两下。街上突然加多了人,铺子里的人争着往外跑,都攥着把蒲扇遮着头,四下里找。“有了凉风!有了凉风!凉风下来了!”大家都嚷着,几乎要跳起来。路旁的柳树忽然变成了天使似的,传达着上天的消息。“柳条儿动了!老天爷,多赏点凉风吧!”
    
    还是热,心里可镇定多了。凉风,即使是一点点,也给了人们许多希望。几阵凉风过去,阳光不那么强了,一阵亮,一阵稍暗,仿佛有片飞沙在上面浮动似的。风忽然大起来,那半天没动的柳条像猛地得到什么可喜的事,飘洒地摇摆,枝条都像长出一截儿来。一阵风过去,天暗起来,灰尘全飞到半空。尘土落下一些,北面的天边出现了墨似的乌云。祥子身上没了汗,向北边看了一眼,把车停住,上了雨布,他晓得夏天的雨是说来就来,不容工夫的。
    
    刚上好了雨布,又是一阵风,墨云滚似地遮黑了半边天。地上的热气跟凉风搀合起来,夹杂着腥臊的干土,似凉又热;南边的半个天响晴白日,北边的半个天乌云如墨,仿佛有什么大难来临,一切都惊慌失措。车夫急着上雨布,铺户忙着收幌子\footnote{〔幌子〕商店门外表明所卖商品的标志。},小贩们慌手忙脚地收拾摊子,行路的加紧往前奔。又一阵风。风过去,街上的幌子,小摊,行人,仿佛都被风卷走了,全不见了,只剩下柳枝随着风狂舞。
    
    云还没铺满天,地上已经很黑,极亮极热的晴午忽然变成了黑夜似的。风带着雨星,像在地上寻找什么似的,东一头西一头地乱撞。北边远处一个红闪,像把黑云掀开一块,露出一大片血似的。风小了,可是利飕有劲,使人颤抖。一阵这样的风过去,一切都不知怎么好似的,连柳树都惊疑不定地等着点什么。又一个闪,正在头上,白亮亮的雨点紧跟着落下来,极硬的,砸起许多尘土,土里微带着雨气。几个大雨点砸在祥子的背上,他哆嗦了两下。雨点停了,黑云铺满了天。又一阵风,比以前的更厉害,柳枝横着飞,尘土往四下里走,雨道往下落;风,土,雨,混在一处,联成一片,横着竖着都灰茫茫冷飕飕,一切的东西都裹在里面,辨不清哪是树,哪是地,哪是云,四面八方全乱,全响,全迷糊。风过去了,只剩下直的雨道,扯天扯底地垂落,看不清一条条的,只是那么一片,一阵,地上射起无数的箭头,房屋上落下万千条瀑布。几分钟,天地已经分不开,空中的水往下倒,地上的水到处流,成了灰暗昏黄的,有时又白亮亮的,一个水世界。
    
    祥子的衣服早已湿透,全身没有一点干松的地方;隔着草帽,他的头发已经全湿。地上的水过了脚面,湿裤子裹住他的腿,上面的雨直砸着他的头和背,横扫着他的脸。他不能抬头,不能睁眼,不能呼吸,不能迈步。他像要立定在水里,不知道哪是路,不晓得前后左右都有什么,只觉得透骨凉的水往身上各处浇。他什么也不知道了,只茫茫地觉得心有点热气,耳边有一片雨声。他要把车放下,但是不知放在哪里好。想跑,水裹住他的腿。他就那么半死半活地,低着头一步一步地往前拽。坐车的仿佛死在了车上,一声不出地任凭车夫在水里挣命。
    
    雨小了些,祥子微微直了直脊背,吐出一口气:“先生,避避再走吧!”
    
    “快走!你把我扔在这儿算怎么回事?”坐车的跺着脚喊。
    
    祥子真想硬把车放下,去找个地方避一避。可是,看看浑身上下都流水,他知道一站住就会哆嗦成一团。他咬上了牙,蹚着水,不管高低深浅地跑起来。刚跑出不远,天黑了一阵,紧跟着一亮,雨又迷住他的眼。拉到了,坐车的连一个铜板也没多给。祥子没说什么,他已经顾不过命来。
    
    雨住一会儿,又下一阵儿。比以前小了许多。祥子一气跑回了家。抱着火,烤了一阵,他哆嗦得像风雨中的树叶。虎妞给他冲了碗姜糖水,他傻子似的抱着碗一气喝完。喝完,他钻了被窝,什么也不知道了,似睡非睡,耳中刷刷的一片雨声。
    
\end{normalsize}


\newpage

\textbf{注释}:

\vspace{-1em}

\begin{itemize}
    \setlength\itemsep{-0.2em}
    \item 〔张罗买卖〕指想办法找坐车的客人。
    \item 〔打盹〕打瞌睡。
    \item 〔汲〕从下往上打水。
    \item 〔蹚〕从浅水里走过去。
\end{itemize}

\chapter{背影}

\begin{normalsize}
    
    我与父亲不相见已二年余了,我最不能忘记的是他的背影。
    
    那年冬天,祖母死了,父亲的差使\footnote{〔差使〕旧时指官场中临时委任的职务,后来也泛指职务或官职。}也交卸\footnote{〔交卸〕交付货物时卸货。引申指离职时交接职位事务,比喻离职。此处实指失业。}了,正是祸不单行的日子。我从北京到徐州,打算跟着父亲奔丧回家。到徐州见着父亲,看见满院狼藉的东西,又想起祖母,不禁簌簌地流下眼泪。父亲说:“事已如此,不必难过,好在天无绝人之路!”
    
    回家变卖典质,父亲还了亏空;又借钱办了丧事。这些日子,家中光景很是惨澹\footnote{〔惨澹〕惨淡,光线暗淡。引申指悲惨凄凉。},一半为了丧事,一半为了父亲赋闲\footnote{〔赋闲〕这里指失业在家。晋代潘岳辞官家居,作《闲居赋》,后因称罢官闲居,事业无事为赋闲。}。丧事完毕,父亲要到南京谋事,我也要回北京念书,我们便同行。
    
    到南京时,有朋友约去游逛,勾留\footnote{〔勾留〕短时间停留。}了一日;第二日上午便须渡江到浦口\footnote{〔浦口〕地名,即今南京市浦口区,在南京西北部的长江北岸,是早年津浦地铁的终点。},下午上车北去。父亲因为事忙,本已说定不送我,叫旅馆里一个熟识的茶房\footnote{〔茶房〕旧时称在旅馆、车站等从事供应茶水等杂物的人。}陪我同去。他再三嘱咐茶房,甚是仔细。但他终于不放心,怕茶房不妥帖;颇踌躇了一会。其实我那年已二十岁,北京已来往过两三次,是没有什么要紧的了。他踌躇了一会,终于决定还是自己\footnote{〔己〕停止。这里是控制的意思。}送我去。我再三劝他不必去;他只说:“不要紧,他们去不好!”
    
    我们过了江,进了车站。我买票,他忙着照看行李。行李太多,得向脚夫\footnote{〔脚夫〕旧时对搬运工人的称呼。}行些小费才可过去。他便又忙着和他们讲价钱。我那时真是聪明过分,总觉他说话不大漂亮,非自己插嘴不可,但他终于讲定了价钱;就送我上车。他给我拣定了靠车门的一张椅子;我将他给我做的紫毛大衣铺好座位。他嘱我路上小心,夜里要警醒\footnote{〔警醒〕睡眠时容易醒来。这里是不要睡得太沉的意思。}些,不要受凉。又嘱托茶房好好照应我。我心里暗笑他的迂;他们只认得钱,托他们只是白托!而且我这样大年纪的人,难道还不能料理自己么?我现在想想,我那时真是太聪明了。
    
    我说道:“爸爸,你走吧。”他望车外看了看,说:“我买几个橘子去。你就在此地,不要走动。”我看那边月台的栅栏外有几个卖东西的等着顾客。走到那边月台,须穿过铁道,须跳下去又爬上去。父亲是一个胖子,走过去自然要费事些。我本来要去的,他不肯,只好让他去。我看见他戴着黑布小帽,穿着黑布大马褂,深青布棉袍,蹒跚地走到铁道边,慢慢探身下去,尚不大难。可是他穿过铁道,要爬上那边月台,就不容易了。他用两手攀着上面,两脚再向上缩;他肥胖的身子向左微倾,显出努力的样子。这时我看见他的背影,我的泪很快地流下来了。我赶紧拭干了泪。怕他看见,也怕别人看见。我再向外看时,他已抱了朱红的橘子往回走了。过铁道时,他先将橘子散放在地上,自己慢慢爬下,再抱起橘子走。到这边时,我赶紧去搀他。他和我走到车上,将橘子一股脑儿放在我的皮大衣上。于是扑扑衣上的泥土,心里很轻松似的。过一会儿说:“我走了,到那边来信!”我望着他走出去。他走了几步,回过头看见我,说:“进去吧,里边没人。”等他的背影混入来来往往的人里,再找不着了,我便进来坐下,我的眼泪又来了。
    
    近几年来,父亲和我都是东奔西走,家中光景是一日不如一日。他少年出外谋生,独力支持,做了许多大事。哪知老境却如此颓唐!他触目伤怀,自然情不能自已。情郁于中,自然要发之于外;家庭琐屑便往往触他之怒。他待我渐渐不同往日。但最近两年不见,他终于忘却我的不好,只是惦记着我,惦记着他的儿子。我北来后,他写了一信给我,信中说道:“我身体平安,惟膀子疼痛厉害,举箸提笔,诸多不便,大约大去\footnote{〔大去〕委婉语,指死亡。}之期不远矣。”我读到此处,在晶莹的泪光中,又看见那肥胖的、青布棉袍黑布马褂的背影。唉!我不知何时再能与他相见!
    
\end{normalsize}


\newpage

\textbf{注释}:

\vspace{-1em}

\begin{itemize}
    \setlength\itemsep{-0.2em}
    \item 〔狼藉〕乱七八糟的样子。
    \item 〔簌簌〕纷纷落下的样子。
    \item 〔典质〕把财务等典当,抵押出去。典:典当。质:抵押。
    \item 〔踌躇〕停下来思考往哪里走的样子。引申指犹豫。
    \item 〔迂〕言行守旧,不合时宜。
    \item 〔月台〕站台。
    \item 〔拭〕擦。
    \item 〔颓唐〕衰颓败落。
    \item 〔郁〕(忧愁,气愤等)积聚。
    \item 〔琐屑〕细小而琐碎的事。
    \item 〔膀子〕胳膊。
    \item 〔箸〕筷子。
\end{itemize}

\chapter{故乡}

\begin{normalsize}
    
    我冒了严寒,回到相隔二千余里,别了二十余年的故乡去。
    
    时候既然\footnote{〔既然〕已经。这里和后面的“又”连用,表递进关系。}是深冬;渐近故乡时,天气又阴晦了,冷风吹进船舱中,呜呜的响,从蓬隙向外一望,苍黄的天底下,远近横着几个萧索的荒村,没有一些活气。我的心禁不住悲凉起来了。阿!这不是我二十年来时时记得的故乡?
    
    我所记得的故乡全不如此。我的故乡好得多了。但要我记起他的美丽,说出他的佳处来,却又没有影像,没有言辞了。仿佛也就如此。于是我自己解释说:故乡本也如此,——虽然没有进步,也未必有如我所感的悲凉,这只是我自己心情的改变罢了,因为我这次回乡,本没有什么好心绪。
    
    我这次是专为了别他而来的。我们多年聚族而居的老屋,已经公同卖给别姓了,交屋的期限,只在本年,所以必须赶在正月初一以前,永别了熟识的老屋,而且远离了熟识的故乡,搬家到我在谋食的异地去。
    
    第二日清早晨我到了我家的门口了。瓦楞\footnote{〔瓦楞〕瓦屋屋顶一行一行的瓦铺成的凹凸相间的行列。}上许多枯草的断茎当风抖着,正在说明这老屋难免易主的原因。几房的本家大约已经搬走了,所以很寂静。我到了自家的房外,我的母亲早已迎着出来了,接着便飞出了八岁的侄儿宏儿。
    
    我的母亲很高兴,但也藏着许多凄凉的神情,教我坐下,歇息,喝茶,且不谈搬家的事。宏儿没有见过我,远远的对面站着只是看。
    
    但我们终于谈到搬家的事。我说外间的寓所已经租定了,又买了几件家具,此外须将家里所有的木器卖去,再去增添。母亲也说好,而且行李也略已齐集,木器不便搬运的,也小半卖去了,只是收不起钱来。
    
    “你休息一两天,去拜望亲戚本家一回,我们便可以走了。”母亲说。
    
    “是的。”
    
    “还有闰土,他每到我家来时,总问起你,很想见你一回面。我已经将你到家的大约日期通知他,他也许就要来了。”
    
    这时候,我的脑里忽然闪出一幅神异的图画来:深蓝的天空中挂着一轮金黄的圆月,下面是海边的沙地,都种着一望无际的碧绿的西瓜,其间有一个十一二岁的少年,项带银圈,手捏一柄钢叉,向一匹猹\footnote{〔猹〕作者在1929年5月4日致舒新城的信中说:“‘猹’字是我据乡下人所说的声音,生造出来的,读如‘查’。……现在想起来,也许是獾罢。”}尽力的刺去,那猹却将身一扭,反从他的胯下逃走了。
    
    这少年便是闰土。我认识他时,也不过十多岁,离现在将有三十年了;那时我的父亲还在世,家景也好,我正是一个少爷。那一年,我家是一件大祭祀的值年\footnote{〔大祭祀的值年〕封建社会中的大家族,每年都有祭祀祖先的活动,费用从族中“祭产”收入支取,由各房按年轮流主持,轮到的称为“值年”。}。这祭祀,说是三十多年才能轮到一回,所以很郑重;正月里供祖像,供品很多,祭器很讲究,拜的人也很多,祭器也很要防偷去。我家只有一个忙月(我们这里给人做工的分三种:整年给一定人家做工的叫长工;按日给人做工的叫短工;自己也种地,只在过年过节以及收租时候来给一定人家做工的称忙月),忙不过来,他便对父亲说,可以叫他的儿子闰土来管祭器的。
    
    我的父亲允许了;我也很高兴,因为我早听到闰土这名字,而且知道他和我仿佛年纪,闰月生的,五行缺土\footnote{〔五行缺土〕旧社会所谓算“八字”的迷信说法。即用天干(甲乙丙丁戊己庚辛壬癸)和地支(子丑寅卯辰巳午未申酉戌亥)相配,来记一个人出生的年、月、日、时,各得两字,合为“八字”;又认为它们在五行(金、木、水、火、土)中各有所属,如甲乙寅卯属木,丙丁巳午属火等等,如八个字能包括五者,就是五行俱全。“五行缺土”,就是这八个字中没有属土的字,需用土或土作偏旁的字取名等办法来弥补。},所以他的父亲叫他闰土。他是能装弶捉小鸟雀的。
    
    我于是日日盼望新年,新年到,闰土也就到了。好容易到了年末,有一日,母亲告诉我,闰土来了,我便飞跑的去看。他正在厨房里,紫色的圆脸,头戴一顶小毡帽,颈上套一个明晃晃的银项圈,这可见他的父亲十分爱他,怕他死去,所以在神佛面前许下愿心,用圈子将他套住了。他见人很怕羞,只是不怕我,没有旁人的时候,便和我说话,于是不到半日,我们便熟识了。
    
    我们那时候不知道谈些什么,只记得闰土很高兴,说是上城之后,见了许多没有见过的东西。
    
    第二日,我便要他捕鸟。他说:
    
    “这不能。须大雪下了才好。我们沙地上,下了雪,我扫出一块空地来,用短棒支起一个大竹匾,撒下秕谷,看鸟雀来吃时,我远远地将缚在棒上的绳子只一拉,那鸟雀就罩在竹匾下了。什么都有:稻鸡,角鸡,鹁鸪,蓝背……”
    
    我于是又很盼望下雪。
    
    闰土又对我说:
    
    “现在太冷,你夏天到我们这里来。我们日里到海边捡贝壳去,红的绿的都有,鬼见怕也有,观音手也有\footnote{〔鬼见怕……〕鬼见怕和观音手,都是小贝壳的名称。旧时浙江沿海的人把这种小贝壳用线串在一起,戴在孩子的手腕或脚踝上,认为可以“避邪”。这类名称多是根据“避邪”的意思取的。}。晚上我和爹管西瓜去,你也去。”
    
    “管贼么?”
    
    “不是。走路的人口渴了摘一个瓜吃,我们这里是不算偷的。要管的是獾猪,刺猬,猹。月亮底下,你听,啦啦的响了,猹在咬瓜了。你便捏了胡叉,轻轻地走去……”
    
    我那时并不知道这所谓猹的是怎么一件东西——便是现在也没有知道——只是无端的觉得状如小狗而很凶猛。
    
    “他不咬人么?”
    
    “有胡叉呢。走到了,看见猹了,你便刺。这畜生很伶俐,倒向你奔来,反从胯下窜了。他的皮毛是油一般的滑……”
    
    我素不知道天下有这许多新鲜事:海边有如许五色的贝壳;西瓜有这样危险的经历,我先前单知道他在水果店里出卖罢了。
    
    “我们沙地里,潮汛要来的时候,就有许多跳鱼儿只是跳,都有青蛙似的两个脚……”
    
    阿!闰土的心里有无穷无尽的希奇的事,都是我往常的朋友所不知道的。他们不知道一些事,闰土在海边时,他们都和我一样只看见院子里高墙上的四角的天空。
    
    可惜正月过去了,闰土须回家里去,我急得大哭,他也躲到厨房里,哭着不肯出门,但终于被他父亲带走了。他后来还托他的父亲带给我一包贝壳和几支很好看的鸟毛,我也曾送他一两次东西,但从此没有再见面。
    
    现在我的母亲提起了他,我这儿时的记忆,忽而全都闪电似的苏生\footnote{〔苏生〕苏醒,复活。这里指重新记起来。}过来,似乎看到了我的美丽的故乡了。我应声说:
    
    “这好极!他,——怎样?……”
    
    “他?……他景况也很不如意……”母亲说着,便向房外看,“这些人又来了。说是买木器,顺手也就随便拿走的,我得去看看。”
    
    母亲站起身,出去了。门外有几个女人的声音。我便招宏儿走近面前,和他闲话:问他可会写字,可愿意出门。
    
    “我们坐火车去么?”
    
    “我们坐火车去。”
    
    “船呢?”
    
    “先坐船,……”
    
    “哈!这模样了!胡子这么长了!”一种尖利的怪声突然大叫起来。
    
    我吃了一吓,赶忙抬起头,却见一个凸颧骨,薄嘴唇,五十岁上下的女人站在我面前,两手搭在髀间,没有系裙,张着两脚,正像一个画图仪器里细脚伶仃\footnote{〔细脚伶仃〕身板细长单薄。}的圆规。
    
    我愕然了。
    
    “不认识了么?我还抱过你咧!”
    
    我愈加愕然了。幸而我的母亲也就进来,从旁说:
    
    “他多年出门,统忘却了。你该记得罢,”便向着我说,“这是斜对门的杨二嫂,……开豆腐店的。”
    
    哦,我记得了。我孩子时候,在斜对门的豆腐店里确乎终日坐着一个杨二嫂,人都叫伊“豆腐西施\footnote{〔西施〕春秋时越国的美女,后来用以泛称一般美女。}”。但是擦着白粉,颧骨没有这么高,嘴唇也没有这么薄,而且终日坐着,我也从没有见过这圆规式的姿势。那时人说:因为伊,这豆腐店的买卖非常好。但这大约因为年龄的关系,我却并未蒙着一毫感化,所以竟完全忘却了。然而圆规很不平,显出鄙夷的神色,仿佛嗤笑法国人不知道拿破仑\footnote{〔拿破仑〕即拿破仑·波拿巴,18世纪末19世纪初法国资产阶级革命时期的军事家、政治家。建立法兰西第一帝国。},美国人不知道华盛顿\footnote{〔华盛顿〕即乔治·华盛顿,18世纪末美国政治家。曾领导独立战争,胜利后任美国第一任总统。}似的,冷笑说:
    
    “忘了?这真是贵人眼高……”
    
    “那有这事……我……”我惶恐着,站起来说。
    
    “那么,我对你说。迅哥儿,你阔了,搬动又笨重,你还要什么这些破烂木器,让我拿去罢。我们小户人家,用得着。”
    
    “我并没有阔哩。我须卖了这些,再去……”
    
    “阿呀呀,你放了道台\footnote{〔道台〕清朝官职道员的俗称,分总管一个区域行政职务的道员和专掌某一特定职务的道员。前者是省以下、府州以上的行政长官;后者掌管一省特定事务,如督粮道、兵备道等。辛亥革命后,北洋军阀政府也曾沿用此制,改称道尹。}了,还说不阔?你现在有三房姨太太;出门便是八抬的大轿,还说不阔?吓,什么都瞒不过我。”
    
    我知道无话可说了,便闭了口,默默的站着。
    
    “阿呀阿呀,真是愈有钱,便愈是一毫不肯放松\footnote{〔放松〕松手,放手。},愈是一毫不肯放松,便愈有钱……”圆规一面愤愤的回转身,一面絮絮的说,慢慢向外走,顺便将我母亲的一副手套塞在裤腰里,出去了。
    
    此后又有近处的本家和亲戚来访问我。我一面应酬,偷空便收拾些行李,这样的过了三四天。
    
    一日是天气很冷的午后,我吃过午饭,坐着喝茶,觉得外面有人进来了,便回头去看。我看时,不由的非常出惊,慌忙站起身,迎着走去。
    
    这来的便是闰土。虽然我一见便知道是闰土,但又不是我这记忆上的闰土了。他身材增加了一倍;先前的紫色的圆脸,已经变作灰黄,而且加上了很深的皱纹;眼睛也像他父亲一样,周围都肿得通红,这我知道,在海边种地的人,终日吹着海风,大抵是这样的。他头上是一顶破毡帽,身上只一件极薄的棉衣,浑身瑟索着;手里提着一个纸包和一支长烟管,那手也不是我所记得的红活圆实的手,却又粗又笨而且开裂,像是松树皮了。
    
    我这时很兴奋,但不知道怎么说才好,只是说:
    
    “阿!闰土哥,——你来了?……”
    
    我接着便有许多话,想要连珠一般涌出:角鸡,跳鱼儿,贝壳,猹,……但又总觉得被什么挡着似的,单在脑里面回旋,吐不出口外去。
    
    他站住了,脸上现出欢喜和凄凉的神情;动着嘴唇,却没有作声。他的态度终于恭敬起来了,分明的叫道:
    
    “老爷!……”
    
    我似乎打了一个寒噤\footnote{〔寒噤〕因寒冷或受惊,身体不自主的颤动,寒战。};我就知道,我们之间已经隔了一层可悲的厚障壁了。我也说不出话。
    
    他回过头去说,“水生,给老爷磕头。”便拖出躲在背后的孩子来,这正是一个廿年前的闰土,只是黄瘦些,颈子上没有银圈罢了。“这是第五个孩子,没有见过世面,躲躲闪闪……”
    
    母亲和宏儿下楼来了,他们大约也听到了声音。
    
    “老太太。信是早收到了。我实在喜欢的不得了,知道老爷回来……”闰土说。
    
    “阿,你怎的这样客气起来。你们先前不是哥弟称呼么?还是照旧:迅哥儿。”母亲高兴的说。
    
    “阿呀,老太太真是……这成什么规矩。那时是孩子,不懂事……”闰土说着,又叫水生上来打拱,那孩子却害羞,紧紧的只贴在他背后。
    
    “他就是水生?第五个?都是生人,怕生也难怪的;还是宏儿和他去走走。”母亲说。
    
    宏儿听得这话,便来招水生,水生却松松爽爽同他一路出去了。母亲叫闰土坐,他迟疑了一回,终于就了坐,将长烟管靠在桌旁,递过纸包来,说:
    
    “冬天没有什么东西了。这一点干青豆倒是自家晒在那里的,请老爷……”
    
    我问问他的景况。他只是摇头。
    
    “非常难。第六个孩子也会帮忙了,却总是吃不够……又不太平……什么地方都要钱,没有规定\footnote{〔规定〕定规,定下来不变的规矩。}……收成又坏。种出东西来,挑去卖,总要捐几回钱\footnote{〔捐钱〕这里指清末民初以“捐款”为名义的变相收税盘剥。},折了本\footnote{〔折了本〕亏本。折:亏损。};不去卖,又只能烂掉……”
    
    他只是摇头;脸上虽然刻着许多皱纹,却全然不动,仿佛石像一般。他大约只是觉得苦,却又形容不出,沉默了片时,便拿起烟管来默默的吸烟了。
    
    母亲问他,知道他的家里事务忙,明天便得回去;又没有吃过午饭,便叫他自己到厨下炒饭吃去。
    
    他出去了;母亲和我都叹息他的景况:多子,饥荒,苛税,兵,匪,官,绅,都苦得他像一个木偶人了。母亲对我说,凡是不必搬走的东西,尽可以送他,可以听他自己去拣择。
    
    下午,他拣好了几件东西:两条长桌,四个椅子,一副香炉和烛台,一杆抬秤\footnote{〔抬秤〕一种能秤上百斤东西的大杆秤。}。他又要所有的草灰(我们这里煮饭是烧稻草的,那灰,可以做沙地的肥料),待我们启程的时候,他用船来载去。
    
    夜间,我们又谈些闲天,都是无关紧要的话;第二天早晨,他就领了水生回去了。
    
    又过了九日,是我们启程的日期。闰土早晨便到了,水生没有同来,却只带着一个五岁的女儿管船只。我们终日很忙碌,再没有谈天的工夫。来客也不少,有送行的,有拿东西的,有送行兼拿东西的。待到傍晚我们上船的时候,这老屋里的所有破旧大小粗细东西,已经一扫而空了。
    
    我们的船向前走,两岸的青山在黄昏中,都装成了深黛颜色,连着退向船后梢去。
    
    宏儿和我靠着船窗,同看外面模糊的风景,他忽然问道:
    
    “大伯!我们什么时候回来?”
    
    “回来?你怎么还没有走就想回来了。”
    
    “可是,水生约我到他家玩去咧……”他睁着大的黑眼睛,痴痴的想。
    
    我和母亲也都有些惘然,于是又提起闰土来。母亲说,那豆腐西施的杨二嫂,自从我家收拾行李以来,本是每日必到的,前天伊在灰堆里,掏出十多个碗碟来,议论之后,便定说是闰土埋着的,他可以在运灰的时候,一齐搬回家里去;杨二嫂发见了这件事,自己很以为功,便拿了那狗气杀(这是我们这里养鸡的器具,木盘上面有着栅栏,内盛食料,鸡可以伸进颈子去啄,狗却不能,只能看着气死),飞也似的跑了,亏伊装着这么高低的小脚\footnote{〔装着……小脚〕小脚:封建陋习,用布将女性双脚紧紧缠裹,使之畸形变小,以之为美。这样高低:这么高。},竟跑得这样快。
    
    老屋离我愈远了;故乡的山水也都渐渐远离了我,但我却并不感到怎样的留恋。我只觉得我四面有看不见的高墙,将我隔成孤身,使我非常气闷;那西瓜地上的银项圈的小英雄的影像,我本来十分清楚,现在却忽地模糊了,又使我非常的悲哀。
    
    母亲和宏儿都睡着了。
    
    我躺着,听船底潺潺的水声,知道我在走我的路。我想:我竟与闰土隔绝到这地步了,但我们的后辈还是一气,宏儿不是正在想念水生么。我希望他们不再像我,又大家隔膜起来……然而我又不愿意他们因为要一气,都如我的辛苦辗转而生活,也不愿意他们都如闰土的辛苦麻木而生活,也不愿意都如别人的辛苦恣睢而生活。他们应该有新的生活,为我们所未经生活过的。
    
    我想到希望,忽然害怕起来了。闰土要香炉和烛台的时候,我还暗自以为他总是崇拜偶像,什么时候都不忘却。现在我所谓希望,不也是我自己手制的偶像么?只是他的愿望切近,我的愿望茫远罢了。
    
    我在朦胧中,眼前展开一片海边碧绿的沙地来,上面深蓝的天空中挂着一轮金黄的圆月。我想:希望本是无所谓有,无所谓无的。这正如地上的路;其实地上本没有路,走的人多了,也便成了路。
    
    \hfill 一九二一年一月
    
\end{normalsize}


\newpage

\textbf{注释}:

\vspace{-1em}

\begin{itemize}
    \setlength\itemsep{-0.2em}
    \item 〔潺潺〕流水声。
    \item 〔瑟索〕不由自主地哆嗦
    \item 〔惘然〕心里迷茫的样子。
    \item 〔连珠〕串联在一起的珠子,形容一个接一个不间断。
    \item 〔应酬〕应答酬谢,指与人交际来往。
    \item 〔恣睢〕放纵张狂。
\end{itemize}

\chapter{想和做}

\begin{normalsize}
    
    有些人只会空想,不会做事。他们凭空想了许多念头,滔滔不绝地说了许多空话,可是从来没认真做过一件事。
    
    也有些人只顾做事,不动脑筋。他们一天忙到晚,做他们一向做惯的或者别人要他们做的事。他们做事的方法只是根据自己的习惯,或者别人的命令,或一般人的通例。自己一向这样做,别人要他们这样做,一般人都这样做,他们就“依葫芦画瓢”,照样做去。到底为什么要做这件事,为什么要这样做,有没有更好的办法,他们从来不想一想。
    
    我们瞧不起前一种人,说他们是“空想家”,可是往往赞美后一种人,说他们能够“埋头苦干”。能够苦干固然是好的,但是只顾埋着头,不肯动动脑筋来想想自己做的事,其实并不值得赞美。
    
    这种埋头做事不动脑筋的人简直是——说得不客气一点——跟牛马一样。拉磨的牛成年累月地在鞭子下绕着石磨转,永远不会想一想为什么要做这件事,为什么要这样做,有没有更好的办法。能够这样想的只有人。人在劳动中不断地动脑筋,想办法,才清清楚楚地知道自己做这件事为什么目的,有什么意义,有什么缺点,才渐渐想出节省劳力,提高效率的方法。人类能够这样劳动,能够一面做,一面想,所以文化能够不断地进步。要不,今天的人类就只能像几万年以前的人类一样,过着最原始最简单的生活了。
    
    一事不做,凭空设想,那是“空想”。不动脑筋,埋头苦干,那是“死做”。无论什么事情,工作也好,学习也好,“空想”和“死做”都不会得到进步。想和做是分不开的,一定要联结起来。
    
    想和做怎样才能够联结起来呢?我们常常听说“从实际出发”这句话,这就是想和做联结起来的一条路。想的时候要从实际出发,就不能“空想”,必须去接近实际。怎样才能够接近实际?当然要观察。光靠观察还不够,还得有行动。举个例子来说,人怎样学会游泳的呢?光靠观察各种物体在水中浮沉的现象,光靠观察鱼类和水禽类\footnote{〔水禽类〕生活在水上或近水的禽类,如鸭、鹅、灰雁等。}的动作,那是不够的;一定要自己跳下水去试验,一次,两次,十次,几十次地试验,才学会了游泳。如果只站在水边,先是一阵子呆看,再发一阵子空想,即使能够想出一大堆“道理”来,自己还是不会游泳,对于别的游泳的人也没有好处。这样空想出来的“道理”其实并不算什么道理。真正的道理是在行动中取得的经验,再根据经验想出来的。而且想出来的道理到底对不对,还得拿行动来证明:行得通的就是对的,行不通的就是错的。
    
    一面做,一面想。做,要靠想来指导;想,要靠做来证明。想和做是紧密地联结在一起的。
    
    在学校里,有些同学很“用功”,可是不会用思想。他们学习语文,就硬读\footnote{〔硬〕不讲方法,不顾一切,强行。}课文。因为只读不想,同一个语言文字上的道理,在这一课里老师讲明白了,出现在别一课里,他们又不理解了。他们学习数学,就硬记公式。因为只记不想,用这个公式算出了一道题,碰到同类的第二道题就又不会算了。从旧经验里得到的道理,不能应用在新事物上,这就是不会用思想的缘故。另外也有些同学,他们能想出些省力的有效的方法,拿来记住动植物的分类,弄清历史的年代。我们固然不赞成为了应付考试想出一些投机取巧的办法;但是我们承认,在学习各种功课和训练记忆力上,是可以有一些比较省力的有效的方法的。这些方法也得从学习的经验中取得。假如只是埋头苦读,不动脑筋想一想,那就得不到。除了学习功课以外,做种种课外活动,也要把想和做联结起来。例如开会,演说,办壁报,组织班会和学术团体,这些实际的行动,如果光凭一腔热情,埋头苦干,不根据已有的成绩和经验,想想怎样才能把这些事情做得更好,更有效果,那么,结果常常会劳而无功。
    
    无论什么人,不管他怎样忙,应该抽点功夫\footnote{〔功夫〕做事所费的精力和时间。}来想一想。想什么?想他自己做过的事,想自己做事得到的经验。这样,他脑子里所有的就不是空想,他的行动也就可以不断地得到进步。
    
\end{normalsize}


\newpage

\textbf{注释}:

\vspace{-1em}

\begin{itemize}
    \setlength\itemsep{-0.2em}
    \item 〔空话〕没有意义的话;对现实无用的话;并未实现的诺言。
    \item 〔滔滔不绝〕像流水一样不断绝。滔滔:形容水势浩大,奔流不息。
    \item 〔一腔热情〕多写作一腔热血,表示满怀着热情。腔:动物身体中空的部分,此处指胸腔(肺),比喻想要喊出来。
    \item 〔凭空〕没有凭据,没有依靠。
    \item 〔投机取巧〕利用心机和伪装,指用不正当的手段。
\end{itemize}

\chapter{畏惧错误就是毁灭进步}

\begin{normalsize}
    
    “畏惧错误就是毁灭进步。”——怀特黑德\footnote{〔怀特黑德〕阿尔弗雷德·怀特黑德,20世纪英国哲学家,提出历程哲学。原句为“畏惧错误毁灭进步,对真理的爱守护进步。”}这句名言,蕴含着丰富的哲理,它给人们——特别是热心改革、勇于创新的人们以深刻的启示。
    
    人们从事各项活动,总是希望获得成功,避免失败。可是实际上往往事与愿违。这原因,从认识论\footnote{〔认识论〕对什么是知识和获取知识的讨论。个人对此的观点和信念。}讲,是由于客观事物的本质并不是显露在外,而是潜藏在内的;不仅如此,它有时还会以颠倒的形式——“假象”出现,就像达尔文有一次半认真地说的那样:“大自然是一有机会就要说谎的。”加上人们认识能力、水平、经验的限制,就难免发生错误了。
    
    用辩证\footnote{〔辩证〕通过对对立矛盾的研究,建立对事物真理的认知。}的观点来看,错误并不可怕,叹息、感伤、畏惧是大可不必的。错误同真理,失败同成功,像睡梦同清醒、黑夜同黎明一样紧密相联。一个人从错误的“梦”中醒来,就会以新的智慧和力量奔向真理。经历着失败的黑夜,正预示了成功的黎明即将来临。黑格尔说过:错误本身乃是“达到真理的一个必然的环节”,“由于这种错误,真理才会出现”。人们的知识、能力以至发明创造,并不单单是在总结成功经验,也是在汲取失败教训的基础上产生出来的。戴维谈到自己获得成功,就说过\footnote{〔戴维……〕汉弗里·戴维,18至19世纪英国化学家,发现了钾、钠、钙、锶等多种化学元素。这句话通常被认为是戴维说的,也符合他的学术研究生涯的特点,但事实上并没有相关记录。}:“我的那些最主要的发现是受到失败的启示而作出的。”
    
    一个人若要有发现,有创造,就不应当畏惧错误。倘若你想把一切错误都关在门外,那你也必将永远被关在真理的门外;倘若你想避免任何失败,那你也必定永远得不到成功。有人说:“若不让错误出生,便不会有真理降世。”这是有道理的。奠定电磁学实验基础的法拉第\footnote{〔法拉第〕迈克尔·法拉第,19世纪物理学家,发现电磁感应现象。},正是由于不怕一而再再而三的失败,经过近十年的艰辛努力,才终于使磁铁产生了电流,开拓了电磁学的新领域。欧立希\footnote{〔欧立希〕保罗·欧立希,19世纪德国生物学家、医学家,化学疗法的先驱。}也以惊人的毅力,在失败了数百次之后,才成功地制成了药品坤凡纳明。无怪德国物理学家普朗克\footnote{〔普朗克〕马克斯·普朗克,19至20世纪德国物理学家,量子力学创始人。}在获取诺贝尔奖金\footnote{〔诺贝尔奖〕1901年起根据瑞典化学家、发明家阿尔弗雷德·诺贝尔的遗嘱设立的奖项。分为物理学、化学、生理学或医学、经济学、文学五个奖,由挪威评议颁发。普朗克于1918年获得诺贝尔物理学奖。}时深有感受地说:“回顾……最后通向发现(量子论)的漫长曲折的道路,我对歌德\footnote{〔歌德〕约翰·歌德,18世纪德国戏剧家、诗人、文艺理论家、政治人物。以《浮士德》等作品闻名。}的话记忆犹新。他说,人们若要有所追求就不能不犯错误。”当然,这绝不是说不要努力去防止和减少错误,或者说可以对错误持满不在乎的态度,而是说不要因为惧怕错误而畏首畏尾,缩手缩脚。这也怕那也怕,是成就不了事业的。
    
    目前,我国正在深入进行体制改革。改革,是破旧创新,没有现成的道路可走,没有定型\footnote{〔定型〕已经有固定形制,可以直接拿来用的。}的模式可依,要靠自己的努力去实践,探索,开拓,创造。在这个过程中是难免出现缺陷和错误的。我们既不应当因为出了点错误便偃旗息鼓,悲观泄气,更不应当因为有了错误就否定改革。你毫不动弹,当然是再保险不过的了,不过,正像鲁迅指出的:“这毫不动弹,却也就是一个大错。”对待错误和失败的正确态度应该是:分析原因,总结教训,找到正确的道路。正是从这个意义上,我们也可以这样说:“畏惧错误就是毁灭改革。”
    
\end{normalsize}


\newpage

\textbf{注释}:

\vspace{-1em}

\begin{itemize}
    \setlength\itemsep{-0.2em}
    \item 〔事与愿违〕事情的发展与愿望相违背。
    \item 〔偃旗息鼓〕放倒旗子,停止击鼓。表示停止战斗,比喻停止做事或声势减弱。偃:仰卧,指平放。
    \item 〔蕴含〕包含。
    \item 〔哲理〕深刻的道理。
\end{itemize}

\chapter{理想的阶梯}

\begin{normalsize}
    
    青年最爱谈理想,青年最苦恼的是理想和现实常常有矛盾。
    
    有的青年虽有理想,但刻苦勤奋不足;有的也很想为理想努力,但不能抓紧一点一滴的时间;有的自以为条件差,岗位平凡,无用武之地,不能充分发挥主观能动作用。结果,常常在碌碌无为的苦闷中慨叹\footnote{〔慨叹〕感慨叹息}蹉跎。
    
    奋斗,是实现理想的阶梯。离开奋斗,理想就只能是幻想而已。有理想的青年,都应从眼前的现实起步,以非常艰苦的奋斗,作为通往理想境界的阶梯。
    
    理想的阶梯,属于刻苦勤奋的人。马克思为实现解放全人类的崇高理想奋斗一生。他积极投身于火热的工人运动,研读无数种著作,学会了欧洲好几个国家的语言。他不断在图书馆钻研,数十年如一日,座位下的地面竟然磨掉一层。化学家诺贝尔\footnote{〔诺贝尔〕阿尔弗雷德·诺贝尔,瑞典化学家、发明家、工程师、企业家。以他的遗嘱成立的诺贝尔奖是科学界、文学界的著名奖项。}为减轻工地上挖土工人的繁重劳动,决心发明炸药。废寝忘食,四年里做了几百次试验。最后一次试验时,他聚精会神地盯着燃延的导火线\footnote{〔导火线〕连着爆炸物的线。点燃一端后,将火沿线传导到爆炸物而引爆。比喻直接引发冲突的事物。}。一声巨响,在旁的人们惊叫:“诺贝尔完了!”诺贝尔却从浓烟中跳出来,面孔乌黑,身上还带着血,兴奋地狂呼:“成功了!”那些杰出的人物正是被一种崇高的目标所鼓舞,才产生了惊人的毅力与忘我的精神。是理想的浪涛激励着他们去刻苦奋斗。今天,我们为实现四化\footnote{〔四化〕四个现代化,指工业、农业、国防、科技现代化。1950年代我党提出的国家战略目标。}而奋斗,这是中华民族空前的事业,其任务之艰,难度之大,更需要亿万人民,特别是青年,百折不回地艰苦奋斗。有志于为这一崇高理想而奋斗的青年要敢于面对现实,不怕一切艰难困苦,不怨天尤人,以凌云的壮志,用刻苦勤奋的汗水浇开灿烂的理想之花。
    
    理想的阶梯,属于珍惜时间的人。富兰克林\footnote{〔富兰克林〕本杰明·富兰克林,18世纪政治家、外交家、科学家、发明家,美利坚合众国创建者之一,共济会成员。}有句名言:“你热爱生命吗?那么别浪费时间,因为时间是组成生命的材料。”许多科学家、文艺家都是同时间赛跑的能手。爱迪生\footnote{〔爱迪生〕托马斯·爱迪生,19至20世纪发明家、企业家。}一生有一千多项发明。这无数次试验的时间从哪里来?就是从常常连续工作两天三天的极度紧张中挤出来的。鲁迅以“时间就是生命”的格言律己,从事无产阶级文学艺术事业三十年,视时间如生命,笔耕不辍。巴尔扎克\footnote{〔巴尔扎克〕奥诺雷·德·巴尔扎克,19世纪现实主义作家,著有《人间喜剧》共91部小说,被誉为当时法国社会的“百科全书”。}用如痴如狂的拼劲,每天奋笔疾书十六七个小时,即使累得手臂疼痛,双眼流泪,也不肯浪费一刻时间。他一生留下为人民深深喜爱的巨著《人间喜剧》,共九十多部小说。这些血汗的结晶不正是时间与生命的光辉记录吗?
    
    时间的流逝是无情的,可怕的。人生七十古来稀\footnote{〔人生七十古来稀〕七十岁的老人自古少有。出自唐代杜甫《曲江二首·其二》。},三分之一要睡去,再除去幼年玩耍的时间,学习与工作大约只有三十几年,一万多天。虚度一日就等于耗费生命的万分之一。朱自清在散文《匆匆》中说:“洗手的时候,日子从水盆里过去;吃饭的时候,日子从饭碗里过去;默默时,便从凝然的双眼前过去。我觉察他去的匆匆了,伸出手遮挽时,他又从遮挽着的手边过去,……”可是,有人甚至从未想过遮挽一下时光呢。对时间的态度,可以检验一个人的世界观。没有理想的人,不懂人生的意义,自然不爱惜时间。真正有理想的人,必定珍惜一分一秒,因为每一瞬间的奋斗都关系着目标的实现。
    
    理想的阶梯,属于迎难而上的人。奋斗的必要,正是由于困难的存在。在通往四化的征途上,坎坷、曲折、荆棘、浪涛是不会少的。幻想一切都顺顺利利,就等于在四化面前止步。有的青年埋怨条件差。这也许是事实。但今天的处境,总不致像伽利略\footnote{〔伽利略〕伽利略·伽利莱,16至17世纪意大利物理学家、天文学家、数学家,欧洲科学革命重要人物,现代物理学奠基者之一。}、布鲁诺\footnote{〔布鲁诺〕乔达诺·布鲁诺,16世纪哲学家、数学家。支持日心说和宇宙无限观,被教会裁判为异端并烧死。}那样冒着受宗教极刑的危险,总不致像高尔基\footnote{〔高尔基〕马克西姆·高尔基,19至20世纪社会主义、现实主义作家、诗人,政治活动家。主要著作为《童年》《在人间》《我的大学》等。}那样在老板的皮鞭下学写作吧。艰苦的环境更能激发有理想的人奋发向上。高尔基从小饱尝人间的辛酸,旧社会血泪的鞭笞铸成了他伟大的心灵。他坚持在敌人的明枪暗箭下写作,在饥饿与死亡的威胁中战斗,为了共产主义事业,不在任何艰难困苦中屈服、畏缩,永远像海燕一样在雷鸣电闪中展翅翱翔。相比之下,我们的困难又算什么呢?有的青年埋怨自己的岗位平凡。这也可能是事实。但革命事业需要三百六十行,绝大多数人都要在平凡岗位上工作。无志之人,将使生命比岗位更平凡;有志之人,将在平凡岗位上成功。华罗庚\footnote{〔华罗庚〕20世纪中国数学家,中国现代数学奠基人。}年轻时在一个中学干杂活,夜间在如豆的昏黄油灯下演算,打下牢固的根基,后来才成为著名的数学家。开普勒\footnote{〔开普勒〕约翰尼斯·开普勒,16至17世纪德国天文学家、数学家。总结了描述天体运动的三大定律。}长期操劳杂役,业余苦钻,发现了行星运动三大定律。道尔顿\footnote{〔道尔顿〕约翰·道尔顿,18至19世纪英国化学家、物理学家,近代原子理论的提出者。}是中学教员,爱因斯坦\footnote{〔爱因斯坦〕阿尔伯特·爱因斯坦,20世纪德国物理学家,创立了相对论和量子力学,现代最伟大的科学家之一。}是小职员,那些发明纺织机、蒸汽机、飞机、火车的,他们的职业、岗位不也都很平凡吗?可见问题不在于岗位,而在于有没有真正的崇高理想和为这理想而奋斗不息的顽强精神。一个有理想有抱负的青年,决不应让困难攫住自己的心灵,而要在奋斗中舒展自己的双臂。当为崇高理想而奋斗一生的双臂收拢时,抱住的必将是令人欣慰的硕果。
    
    奋斗,是改变现实的杠杆,是亿万人民共攀四化高峰的坚实阶梯。只有以不懈的韧劲,一级级攀登,才能一步步接近那光辉灿烂的理想高峰。让我们在四化的伟大征途上谱写出自己的奋斗之歌吧。
    
\end{normalsize}


\newpage

\textbf{注释}:

\vspace{-1em}

\begin{itemize}
    \setlength\itemsep{-0.2em}
    \item 〔蹉跎〕任由时间过去而无作为,浪费时间。
    \item 〔废寝忘食〕顾不得睡觉,忘记吃饭。形容非常专注努力。
    \item 〔荆棘〕泛指山野丛生的多刺灌木。比喻阻挡前进的障碍、艰难局面或纷扰。
    \item 〔笔耕不辍〕坚持勤奋写作。笔耕:用笔耕耘,比喻以写作谋生或勤奋写作。辍:中止。
    \item 〔攫住〕快速有力地抓住。
    \item 〔硕果〕巨大的成果。
\end{itemize}

\chapter{记一辆纺车}

\begin{normalsize}
    
    我曾经使用过一辆纺车\footnote{〔纺车〕把原料纺成纱的设备。},离开延安的那年,把它跟一些书籍一起留在蓝家坪了。后来常常想起它。想起它,就像想起旅途的旅伴,战场的战友,心里充满了深深的怀念。
    
    那是一辆普通的纺车。说它普通,一来它的车架,轮叶\footnote{〔轮叶〕纺车部件,用来转动纱线的轮子。},锭子\footnote{〔锭子〕纺车部件,用来把纤维捻成纱并把纱绕在筒管上成一定形状。},跟一般农村用的手摇纺车没有什么两样;二来它是延安上千上万辆纺车中的一辆。的确,那个时候在延安的人,无论是机关的干部,学校的教员和学员,也无论是部队的指挥员和战斗员,在工作,学习或者练兵的间隙里,谁没有使用过纺车呢?纺车跟战斗用的枪,耕田用的犁,学习用的书和笔一样,成为大家亲密的伙伴。
    
    在延安,纺车是作为战斗的武器使用的。那是在抗日战争最艰苦的时候,国民党反动派发动反共高潮\footnote{〔反共高潮〕1939年春至1943年夏国民党多次对日媾和,与日本共同反共的行动。},配合日寇重重封锁陕甘宁边区\footnote{〔陕甘宁边区〕陕西省北部、甘肃省和宁夏省的东部合称,我党的革命抗日根据地。},想困死我们。我们边区军民热烈响应毛泽东同志的伟大号召:“自己动手,丰衣足食”,结果彻底粉粹了敌人围困的阴谋。在延安的人,在所有抗日根据地的人,不但吃得饱,而且穿得暖,坚持了抗战,争取到了抗战的最后胜利。开荒,种庄稼,种蔬菜,是保证足食的战线;纺羊毛,纺棉花,是保证丰衣的战线。
    
    大家用纺的毛线织毛衣,织呢子\footnote{〔呢子〕一种较厚较密的毛织品,多用来做制服、大衣等。};用纺的棉纱合线\footnote{〔合线〕把细纱线拧合成一定粗细的线。},织布。同志们穿的衣服鞋袜,有的就是自己纺线或者跟同志换工劳动做成的。开垦南泥湾\footnote{〔南泥湾〕延安城东南45公里的荒地。1940年,王震率领三五九旅开垦南泥湾,展开大生产运动。}的部队甚至能够在打仗练兵和进行政治、文化学习而外,纺毛线给指战员发军装呢。同志们亲手纺线织布做的衣服,穿着格外舒适,也格外爱惜。那个时候,人们对一身灰布制服,一件本色的粗毛线衣,或者自己打的一副手套,一双草鞋,都很有感情。衣服旧了,也“敝帚自珍”,不舍得丢弃。总是脏了洗洗,破了补补,穿一水又穿一水,穿一年又穿一年。衣服只要整齐干净,越朴素穿着越随心。西装革履,华丽的服饰,只有在演剧的时候作演员的服装,平时不要说穿,就是看看也觉得碍眼。美的概念里是更健康的内容,那就是整洁,朴素,自然。
    
    纺线,劳动量并不太小,纺久了会胳膊疼腰酸;不过在刻苦学习和紧张工作的间隙里纺线,除了经济上对敌斗争的意义而外,也是一种很有兴趣的生活。在纺线的时候,眼看着匀净的毛线或者棉纱从拇指和食指中间的毛卷里或者棉条里抽出来,又细又长,连绵不断,简直会有一种艺术创作的快感。摇动的车轮,旋转的锭子,争着发出嗡嗡嘤嘤的声音,像演奏弦乐,像轻轻地唱歌。那有节奏的乐音和歌声是和谐的,优美的。
    
    纺线也需要技术。车摇慢了,线抽快了,线会断头;车摇快了,线抽慢了,毛卷、棉条会拧成绳,线全打成结。摇车,抽线,配合恰当,成为熟悉的技巧,可不简单,需要用很大的耐心和毅力下一番功夫。初学纺线,往往不知道劲往哪儿使。一会儿毛卷拧成绳了,一会儿棉纱打成结了,纺手急得满头大汗。性子躁一些的人甚至为断头接不好生纺车的气,摔摔打打,恨不得把纺车砸碎。可是那关纺车什么事呢?尽管人急得站起来,坐下去,一点也没有用,纺车总是安安稳稳地待在那里,像露出头角的蜗牛,像着陆停驶的飞机,一声不响,仿佛只是在等待,等待。一直等到使用纺车的人心平气和了,左右手动作协调,用力适当,快慢均匀了,左手拇指和食指间的毛线或者棉纱就会像魔术家帽子里的彩绸一样无穷无尽地抽出来。那仿佛不是用羊毛、棉花纺线,而是从毛卷里或者棉条里往外抽线。线是现成的,早就藏在毛卷里或者棉条里的。熟练的纺手,趁着一豆灯光或者朦胧的月光,也能摇车,抽线,上线,一切做得从容自如。线上在锭子上,线穗子\footnote{〔线穗子〕纺纱时绕在锭子上成团的纱线。}就跟着一层层加大,直到沉甸甸的,像成熟了的肥桃。从锭子上取下穗子,也像从果树上摘下果实,劳动后收获的愉快,那是任何物质享受都不能比拟的。这个时候,就连起初想砸碎纺车的人也对纺车发生了感情。那种感情,是凯旋的骑士对战马的感情,是“仰手接飞猱,俯身散马蹄”\footnote{〔“仰手接飞猱,俯身散马蹄”〕出自曹植的《白马篇》。形容技艺高超。猱:一种猴子。马蹄:箭靶。}的射手对良弓的感情。
    
    纺线有几种姿势:可以坐着蒲团纺,可以坐着矮凳纺,也可以把纺车垫得高高的站着纺。站着纺线,步子有进有退,手臂尽量伸直,像“白鹤晾翅”\footnote{〔“白鹤晾翅”〕太极拳的招式,两手上举,如鹤展开双翅。},一抽线能拉得很长很长。这样气势最开阔,肢体最舒展;兴致高的时候,很难说那是生产,是舞蹈,还是体育锻炼。
    
    为了提高生产率,大家也进行技术改革,运用物理学上轮轴\footnote{〔轮轴〕上有轮子旋转的枢轴。}和摩擦传动的道理,在轮子和锭子中间安装加速轮,加快锭子旋转的速度,把手工生产的工具变成半机械化。大多数纺车是在纺羊毛、纺棉花的劳动实践中培养出来的木工做的;安装加速轮也是在劳动实践中大家摸索出来的创造发明。从劳动实践中还不断总结出一些新的经验。譬如,纺羊毛跟纺棉布常有不同的要求:羊毛要松一些,干一些;棉花要紧一些,潮一些。因此弹过的羊毛要卷成卷,棉花要搓条,烘晒毛卷和阴润棉条都有一定的火候分寸。这些技术经验,不靠实践是一辈子也不知道里边的奥妙的。
    
    为了交流经验,互相提高,纺线也开展竞赛。三五十辆或者百几十辆纺车搬在一起,在同一个时间里比纺线的数量和质量。成绩好的有奖励,譬如,奖一辆纺车,奖手巾、肥皂、笔记本之类。那是很光荣的。更光荣的是被称为纺毛突击手、纺纱突击手。竞赛,有的时候在礼堂,有的时候在窑洞\footnote{〔窑洞〕}前边,更有的时候在山根河边的坪坝\footnote{〔坪坝〕山间平整的场地。}上。在坪坝上竞赛的那种场面最壮阔,“沙场秋点兵”\footnote{〔“沙场秋点兵”〕出自宋代辛弃疾的《破阵子·为陈同甫赋壮词以寄之》。沙场:战场。点兵:检阅军队。}或者能有那种气派?不,阵容相近,热闹不够。那是盛大的节日里赛会的场面。只要想想:天地是厂房,深谷是车间,幕天席地,群山环拱,怕世界上还没有哪个地方哪种轻工业生产有那样的规模哩。你看,整齐的纺车行列,精神饱满的纺手队伍,一声号令,百车齐鸣,别的不说,只那嗡嗡的响声就有点像飞机场上机群起飞,扬子江\footnote{〔扬子江〕长江的旧称。}边船只拔锚。那哪儿是竞赛,那是万马奔腾,在共同完成一项战斗任务。因此竞赛结束,无论是纺得多的还是纺得比较少的,得奖的还是没有得奖的,大家都感到胜利的快乐。
    
    就这样,用劳动的双手,自力更生。纺线,不只在经济上保证了革命根据地的人大家有衣穿,使大家学会了一套生产劳动的本领,而且在思想上还教育了大家认识劳动“本身成了生活的第一需要”的意义;自觉地克服了那种“认为劳动只是一种负担,凡是劳动都应当付给一定报酬的习惯”。劳动为集体,同时也为自己。在劳动的过程里,很少人为了个人的什么去锱铢计较;倒是为集体做了些什么有意义的事情,才感到是真正的幸福。
    
    就因为这些,我常常想起那辆纺车。想起它像想起老朋友,心里充满了深深的怀念。围绕着这种怀念,也想起延安的种种生活。在党中央和毛泽东同志的周围工作,学习,劳动,同志的友谊,革命大家庭的温暖,把大家团结得像一个人。真是既团结,紧张,又严肃,活泼。那个时候,物质生活曾经是艰苦的、困难的吧,但是,比起无限丰富的精神生活来,那算得了什么!凭着崇高的理想、豪迈的气概、乐观的志趣,克服困难不也是一种享受吗?
    
    跟困难作斗争,其乐无穷。
    
\end{normalsize}


\newpage

\textbf{注释}:

\vspace{-1em}

\begin{itemize}
    \setlength\itemsep{-0.2em}
    \item 〔凯旋〕打仗胜利归来。凯:打胜仗回来奏的乐曲。旋:回归。
    \item 〔幕天席地〕把天做幕,把地当席,指露天。
    \item 〔自力更生〕用自己的力量办事、解决问题。
    \item 〔报酬〕
    \item 〔锱铢〕
\end{itemize}

\chapter{生物的入侵者}

\begin{normalsize}
    
    当你在路边草地或自家庭院里发现一两只从未见过的甲虫时,你肯定不会感到惊讶。但在生物学家和生态学家们看来,这或许不是件寻常小事。专家们把这种原本生活在异国他乡、通过非自然途径来到新的环境中的“移民”称为“生物入侵者”——它们不仅会破坏某个地区原有的生态系统\footnote{〔生态系统〕自然界的一定范围内,生物和环境构成的统一整体。},而且还可能给人类社会造成难以估量的经济损失。
    
    在人类文明的早期,陆路和航海技术尚不发达,自然界中的生态平衡\footnote{〔生态平衡〕生态系统中各个成分稳定持续,相互关系大体不变的状态。}并没有受到太大的破坏。在自然条件下,一颗蒲公英的种子可能随风飘荡几千米后才会落地,如果各种条件适合,它会在那里生根、发芽、成长;山涧中的鱼虾可能随着水流游到大江大河中安家落户……凡此种种,都是在没有人为干预的条件下缓慢进行的,时间和空间跨度都非常有限,因此不会造成生态系统的严重失衡\footnote{〔失衡〕失去平衡。这里指生态系统的成分比例发生难以复原的变化。}。
    
    如果一个物种\footnote{〔物种〕生物种类的基本单位,能持续繁衍的同类生物的集合。}在新的生存环境中不受同类的食物竞争以及天敌伤害等诸多因素制约,它很可能会无节制地繁殖。1988年,几只原本生活在欧洲大陆的斑贝\footnote{〔斑贝〕一种类似河蚌的贝类。}被一艘货物带到北美大陆。当时,这些混杂在仓底货物中的“偷渡者”并没有引起当地人的注意,它们被随便丢弃在五大湖附近的水域中。然而令人始料不及的是,这里竟成了斑贝的“天堂”。由于没有天敌的制约,斑贝的数量急剧增加,五大湖内的疏水管道几乎全被它们“占领”了。到目前为止,人们为了清理和更换管道已耗资数十亿美元。来自亚洲的天牛和南美的红蚂蚁是另外两种困扰美国人的“入侵者”,前者疯狂破坏芝加哥和纽约的树木,后者则专门叮咬人畜,传播疾病。
    
    “生物入侵者”在给人类造成难以估量的经济损失的同时,也对被入侵地的其他物种以及物种多样性\footnote{〔物种多样性〕一定区域或生态系统中物种的丰富程度和分布均匀程度。}构成极大威胁。二战期间,棕树蛇\footnote{〔棕树蛇〕一种栖息在树上的蛇。}随一艘军用货船落户关岛,这种栖息在树上的爬行动物专门捕食鸟类,偷袭鸟巢,吞食鸟蛋。从二战至今,关岛本地的11种鸟类中已有9种被棕树蛇赶尽杀绝,仅存的两种鸟类的数量也在与日俱减,随时有绝种\footnote{〔绝种〕某个物种在某个生态系统中彻底消失。}的危险。一些生物学家在乘坐由关岛飞往夏威夷岛的飞机上曾先后6次看到棕树蛇的身影。他们警告说,夏威夷岛上没有任何可以扼制棕树蛇繁殖的天敌,一旦棕树蛇在夏威夷安家落户,该岛的鸟类将在劫难逃。许多生物学家和生态学家将“生物入侵者”的增多归罪于日益繁荣的国际贸易,事实上许多“生物入侵者”正是搭乘跨国贸易的“便车”达到“偷渡”的目的的。以目前全球新鲜水果和蔬菜贸易为例,许多昆虫和昆虫的卵附着在这些货物上,其中包括危害极大的害虫,如地中海果蝇等。尽管各国海关\footnote{〔海关〕审核出入国境的人员、货物的机关。}动植物检疫中心\footnote{〔检疫中心〕检验出入境人员、动植物的卫生安全和货物卫生质量的机构。}对这些害虫严加防范,但由于进出口货物数量极大,很难保证没有漏网之“虫”。此外,跨国宠物贸易也为“生物入侵者”提供了方便。近年来,由于引进五彩斑斓的观赏鱼而给某些地区带来霍乱病源的消息时常见诸报端。一些产自他乡的宠物,如蛇、蜥蜴、山猫等,往往会因主人的疏忽或被遗弃而逃出,为害一方。
    
    一些生物学家指出,一旦某种“生物入侵者”在新的环境中站稳脚跟并大规模繁殖,其数量将很难控制。即使在科学技术高度发达的今天,面对那些适应能力和繁殖能力极强的动植物,人们仍将束手无策。
    
    我国地域辽阔,横跨多个气候带,世界各地的外来物种,都能在国内找到合适的栖息地。加之我国还是一个地形复杂、生态系统多种多样的大国,这使得我国很容易遭受外来物种的入侵。例如,20世纪70年代入侵我国的小麦蚊虫,在30年间已遍及全国主要的小麦产区;1982年在江苏南京首次发现的松材线虫,已扩展到江苏、安徽、广东、山东和江西等省。据了解,除极少数位于青藏高原的保护区外,我国各地几乎都有外来入侵物种的踪影。
    
    生物学和生态学界的一些学者主张人类不应该过多地干预生物物种的迁移过程,因为失衡是暂时的,一个物种在新的环境中必然遵循物竞天择\footnote{〔物竞天择〕各种生物相互竞争,最终存留下来的是最能适应自然环境的。}的法则。“生物入侵者”并不都能够生存下来,能够生存下来的就是强者。即使生态系统中的强者也同样受到各种因素的制约,不可能为所欲为,因此,自然界终究会产生新的平衡。然而更多的学者持反对意见。他们认为自然调节的过程是漫长的,如果听任“生物入侵者”自由发展,许多本土物种将难逃绝种厄运,自然界的物种多样性将受到严重破坏。另外,“生物入侵者”给人类社会造成的经济损失是惊人的。据估计,我国2001年由“生物入侵者”造成的经济损失高达1200亿元,接近全国农业生产总值的十分之一。面对这样的天文数字,人们岂能无动于衷?
    
    随着我国深入参与世界贸易,越来越多的人认识到这一问题的严重性。2003年开始,我国对境内的“生物入侵者”展开排查,陆续发布四批《外来入侵物种名单》。2020年公布的《中华人民共和国生物安全法》以及2022年8月开始施行的《外来入侵物种管理办法》,标志着我国初步搭建了应对外来物种入侵的治理体系,建设了配套的制度,打响了抵抗“生物入侵者”的战争。
    
    目前,我国已经发现超过660个外来入侵物种。与“生物入侵者”的战争注定是长久的,保护我们国家乃至全球的生态系统,是每个人的责任。
    
\end{normalsize}


\newpage

\textbf{注释}:

\vspace{-1em}

\begin{itemize}
    \setlength\itemsep{-0.2em}
    \item 〔干预〕干涉,介入。
    \item 〔跨度〕桥梁、屋顶等跨越空间的大小。引申指事物、概念覆盖的范围。
    \item 〔听任〕任由,不干涉。
    \item 〔制约〕限制约束。
    \item 〔节制〕限制。也指限制的程度。
    \item 〔厄运〕坏运气,不幸。
    \item 〔扼制〕制止,控制并阻碍发展。扼:掐住。
    \item 〔无动于衷〕一点也不动心、不关心,置之不理。
    \item 〔在劫难逃〕迷信认为命中注定要遭受的灾难无法逃脱。引申为无法避免某种灾难。劫:命中注定的灾难。
\end{itemize}

\chapter{怀疑与学问}

\begin{normalsize}
    
    \begin{quotation}
    
    学者先要会疑。\\
    
    \hfill ——程颐
    
    \end{quotation}
    
    \begin{quotation}
    
    在可疑而不疑者,不曾学;学则须疑。\\
    
    \hfill ——张载
    
    \end{quotation}
    
    学问的基础是事实和证据。事实和证据的来源有两种:一种是自己亲眼看见的,一种是听别人传说的。譬如在国难危急的时候,各地一定有许多口头的消息,说得如何凶险,那便是别人的传说,不一定可靠。 要知道实际的情形,只有靠自己亲自视察。做学问也是这样,最要紧最可靠的材料是自己亲见的事实证据;但这种证据有时候不能亲自见到,便只能靠别人的传说了。
    
    我们对于传说的话,不论信不信,都应当经过一番思考,不应当随随便便就信了。我们信它,因为它“是”;不信它,因为它“非”。 这一番事前的思索,不肯随便轻信,便是怀疑的精神,做一切学问的基本条件。我们听说古代有三皇五帝\footnote{〔三皇五帝〕出自《周礼·春官·外史》:“外史……掌三皇五帝之书。”},便要问:这是谁说的话?最先见于何书? 书是何时人著的?著者何以知道? 我们又听说“腐草为萤”\footnote{〔腐草为萤〕出自《礼记·月令》。原句为“季夏三月,腐草为萤”。},便要问:死了的植物如何会变飞动的甲虫?有什么科学根据? 我们若能这样追问,一切虚妄的学说便不攻自破了。
    
    我们对于不论哪一本书,哪一种学问,都要先经过怀疑,因怀疑而思索,因思索而辨别是非。经过怀疑、思索、辨别三个步骤以后,那本书才是我的书,那种学问才是我的学问。否则便是盲从,便是迷信。孟子所谓“尽信书不如无书”\footnote{〔尽信书不如无书〕出自《孟子·尽心下》。原句为“尽信《书》,则不如无《书》”,对《尚书》表示怀疑。},也就是教我们有一点怀疑的精神,不要随便盲从或迷信。
    
    怀疑不仅是消极方面辨伪去妄的必须步骤,也是积极方面建设新学说、启迪新发明的基本条件。对于别人的话,都不打折扣地承认,那是思想上的懒惰。 这样的脑筋永远是被动的,永远不能做学问。只有常常怀疑、常常发问的脑筋才有问题,有问题才想求解答。在不断的发问和求解中,一切学问才会发展起来。许多大学问家、大哲学家都是从怀疑中锻炼出来的。清代的一位大学问家——戴震 ,幼时读朱子的《大学·章句》,便问《大学》是何时的书,朱子是何时的人。塾师告诉他《大学》是周代的书,朱子是宋代的大儒,他便问宋代的人如何能知道一千多年前的著者的意思\footnote{〔戴震……〕出自段玉裁《戴东原先生年谱》。戴震,18世纪学者。}。法国的大哲学家笛卡儿也说:“我怀疑,所以我存在。”\footnote{〔……我怀疑,所以我存在〕勒内·笛卡尔,17世纪法国哲学家、数学家,西方现代哲学奠基人之一,唯心主义、理性主义开拓者。为了找出绝对可靠的知识作为思考的起点,笛卡尔提出“我怀疑,所以我存在”,他解释为“我怀疑我的存在,这证明我在思考,因此证明在思考的我必然存在。”}他的哲学就建设在对于万事万物的怀疑和明辨上。一切学问家,不但对于流俗传说,就是对于过去学者的学说也常常要抱怀疑的态度,常常和书中的学说辩论,常常评判书中的学说,常常修正书中的学说。要这样才能有更新更善的学说产生。古往今来,科学上新的发明,哲学上新的理论,美术上新的作风,都是这样来的。 如果后来的学者都墨守前人的旧说,那就没有新问题,没有新发明,一切学术就停滞了,人类的文化也就不会进步了。
    
\end{normalsize}



\chapter{发问的精神}

\begin{normalsize}
    
    我们日常所见所闻所接触的事物里,有很多的道理。大家因为时常见到听到接触到,都觉得那些事物平淡无奇,不足介意。其实这是一种损失。
    
    事物里的道理,不比课本的文字,教师的讲解,看了听了就懂。这种道理犹如封锁在密库石室里的珍奇,我们要用一把钥匙去开启。
    
    这把钥匙就是发问的精神。
    
    发问是思想的初步,研究的动机。一切知识的获得,大都从发问而来;新发明、新创造也常常由发问开端。能发问,勤发问的人,头脑自然会日益丰富,眼光自然会日益敏锐。别人不肯动脑筋的地方,他偏会想出惊人的见解;别人以为平常的事物,他偏会看出不平常的道理。这样的人,古今中外都有的是。
    
    苹果落地,是多么平常的事情,牛顿\footnote{〔牛顿〕艾萨克·牛顿,17世纪英国物理学家、数学家、哲学家。发明了微积分,发现万有引力定律,创立了经典力学三大定律。}看见了,却要问出个所以然来,结果创立了“万有引力”说,支配了人类思想。
    
    壶水滚沸,谁不常常见到?只有瓦特\footnote{〔瓦特〕詹姆斯·瓦特,18世纪英国发明家、机械工程师,发明了可实用的蒸汽机,奠定了工业革命的基础。}把它当作问题研究,因而发明了蒸汽机,使人类至今蒙受其利。
    
    生、老、病、死,都是极普遍的人生现象,释迦牟尼\footnote{〔释迦牟尼〕乔达摩·悉达多,通称释迦牟尼,公元前5世纪南亚思想家,佛教创始人。}偏偏要寻根究底,求个解答。他因此抛弃尊位和家庭,独自去潜修静想,终于创立了佛教。
    
    我们中国的孔子\footnote{〔孔子〕春秋末期哲学家、思想家、政治活动家,儒家思想创始人和代表人物。言行被其弟子门人编纂为《论语》。},也是个好问的人。他到太庙\footnote{〔太庙〕中国古代帝王的宗庙,祭祀帝王的祖先。}里去,看见每样事物都要问。他知道老子\footnote{〔老子〕春秋时期哲学家、思想家,道家思想创始人,留有著作《道德经》。}熟悉典籍,就去向老子问礼。他能够成为万世景仰的圣人,难道真是天生的?
    
    够了,够了,不必多举了。举出这几个人物,无非要证明发问精神的可贵。我们虽然不一定人人能成为大科学家大思想家,但是我们不能不求知识,不能不明道理。要求知识,明道理,处处都会碰到问题。不能勤于发问,是多么可惜啊!
    
    有人也许会说,知识和道理,可以从书上读到,老师也会给我们讲解,只要努力学习,牢牢记住就成了,何必多问?说这话,大概自以为是。可是仔细想想,就会看出漏洞。
    
    第一,书本的记载,老师的讲解,大都是前人思想和研究的结果。可是世界是天天在变动,在进步的。变动和进步又不断地给我们带来许多新道理,新知识,新问题,往往不是前人留下的知识和道理所能包含的,有时甚至彼此冲突。假使墨守旧说,不能发问,那怎样能研究,文明又怎样能进步呢?
    
    其次,知识和道理,是各个人研究出来的。各个人或少数人的见识有限,不免要弄出错误来。而各种知识和道理,经过传播,往往会有歪曲和失实的地方。我们若是一味吸取,不去审问,岂不会把错的、伪的当作了对的、真的,使那些对的、真的反而永不可见?再说,即使我们所吸取的都是真的、对的,经过反复的审问,真的自会益见其真\footnote{〔益〕越,更加。},对的自会益见其对。这样一来,我们可以得到切实和透彻的了解,获得正确的定论。
    
    更进一步说,求知识,明道理,不光是懂得了、记住了就算完事,更要紧的,是把知识和道理贯穿到生活和习惯里去。必须这样,知识和道理才能让我们终身受用,才不会失去学习的价值。怎样贯穿到生活和习惯里去呢?第一步就要把书本上看到的,教师那里听到的,同实际生活里的事物参照比较。在参照比较中,发问是最重要的。发现的问题越多,对于事物一定看得越清楚;同时对于自己的所学也一定更有把握,知道怎样把它安排到生活里去。
    
    你忍心让你的智慧之门永闭吗?你愿意永远盲目地让别人带着你走吗?你愿意永远只做两脚书柜\footnote{〔两脚书柜〕比喻空有知识却不理解、不懂得运用的人。}吗?假如你的答复是否定的,那么,你万不可忘记带那把钥匙,你得能发问,勤发问。
    
\end{normalsize}


\newpage

\textbf{注释}:

\vspace{-1em}

\begin{itemize}
    \setlength\itemsep{-0.2em}
    \item 〔珍奇〕珍贵奇特的物品。
    \item 〔动机〕做某事的理由。
    \item 〔日益〕越来越,每天都比前一天更加……。
    \item 〔万世景仰〕被长久地敬佩仰慕。景:慕。
    \item 〔寻根究底〕寻求和追究事物的原由。
    \item 〔典籍〕记载法令、制度等的重要文献。典:标准、规则。籍:书册。
    \item 〔墨守〕战国时墨子善于守城。后指固执保守,不会变通。
    \item 〔盲目〕没有明确目标,对事物认识不清,做事没有主见没有计划,仿佛突然眼盲了。
    \item 〔贯穿〕从头到尾穿过。贯:古代串钱的绳索。
\end{itemize}

\chapter{七根火柴}

\begin{normalsize}
    
    天亮的时候,雨停了。
    
    草地的气候就是奇怪,明明是月朗星稀的好天气,忽然一阵冷风吹来,浓云像从平地上出来似的,霎时把天遮得严严的,接着,暴雨夹杂着栗子般大的冰雹,不分点地倾泻下来\footnote{〔不分点〕不间断。}。
    
    卢进勇从树丛里探出头来,四下里望了望。整个草地都沉浸在一片迷蒙的雨雾里,看不见人影,听不到人声。
    
    被暴雨冲洗过的荒草,像用梳子梳理过似的,躺倒在烂泥里,连路也给遮没了。天,还是阴沉沉的,偶尔还有几颗冰雹洒落下来,打在那浑浊的绿色水面上,溅起一朵朵浪花。他苦恼地叹了口气。因为小腿伤口发炎,他掉队了。两天来,他日夜赶路,原想在今天赶上大队的,却又碰上了这倒霉的暴雨,耽误了半个晚上。
    
    他咒骂着这鬼天气,从树丛里钻出来,长长地伸了个懒腰。一阵凉风吹得他连打了几个寒颤。他这才发现衣服完全湿透了。
    
    “要是有堆火烤,该多好啊!”他使劲绞着衣服,望着那顺着裤脚流下的水滴想道。他也知道这是妄想——不但是现在,就在他掉队的前一天,他们连里已经因为没有引火的东西而只好吃生干粮了。他下意识地把手插进裤袋里,意外地,手指触到了一点黏黏的东西。他心里一喜,连忙蹲下身,把裤袋翻过来。果然,在裤袋底部粘着一小撮青稞面粉;面粉被雨水一泡,成了稀糊了。他小心地把这些稀糊刮下来,居然有鸡蛋那么大的一团。他吝惜地捏着这块面团,心里不由得暗自庆幸:“幸亏昨天早晨没有发现它们。已经一昼夜没有吃东西了,这会儿看见了可吃的东西,更觉饿得难以忍受。为了不致一口吞下去,他把面团捏成了长条。正要把它送到嘴边,突然听见一声低低的叫声:“同志——”
    
    这声音那么微弱、低沉,就像从地底下发出来的。他略微愣了一下,便一瘸一拐地向着那声音走去。卢进勇蹒跚地跨过两道水沟,来到一棵小树底下,才看清楚那个打招呼的人。他倚着树杈半躺在那里,身子底下是一汪浑浊的污水,看来已经有很长时间没有挪动了。他的脸色更是怕人,被雨打湿了的头发粘贴在前额上,雨水沿着头发、脸颊滴滴地流着。眼眶深深地塌陷下去,眼睛努力地闭着,只有腭下\footnote{〔腭下〕指下巴后方。}的喉结在一上一下地抖动,干裂的嘴唇一张一翕地发出低低的声音:“同志——同志——”
    
    听见卢进勇的脚步声,那个同志吃力地张开眼睛,挣扎了一下,似乎想坐起来,但动不了。
    
    卢进勇看着这情景,眼睛里像揉进了什么\footnote{〔揉进了什么〕指仿佛揉进了沙子而流泪。},一阵酸涩。在掉队的两天里,他这已经是第三次看见战友倒下来了。“一定是饿坏了!”他想,连忙抢上一步,搂住那个同志的肩膀,把那点青稞面递到那同志的嘴边说:“同志,快吃点吧!”
    
    那同志抬起失神的眼睛,呆滞地望了卢进勇一眼,吃力地举起手推开他的胳膊,嘴唇翕动了好几下,齿缝里挤出了几个字:“不,没……没用了。”
    
    卢进勇一时不知怎么好。他望着那张被寒风冷雨冻得乌青的脸,和那脸上挂着的雨滴,痛苦地想:“要是有一堆火,有一杯热水,也许他能活下去!”他抬起头,望望那雾蒙蒙的远处,随即拉住那同志的手腕说:“走,我扶你走吧。”那同志闭着眼睛摇了摇头,没有回答,看来是在积攒着浑身的力量。好大一会儿,他忽然睁开了眼,右手指着自己的左腋窝,急急地说:“这……这里!”
    
    卢进勇惶惑地把手插进那湿漉漉的衣服。他觉得那同志的胸口和衣服一样冰冷了,在左腋窝里,他摸出了一个硬硬的纸包,递到那个同志的手里。
    
    那同志一只手抖抖索索地打开了纸包,那是一个党证,揭开党证,里面并排摆着一小堆火柴,干燥的火柴。
    
    红红的火柴头聚集在一起,正压在那朱红的印章的中心,像一簇火焰在跳。
    
    “同志,你看着……”那同志向卢进勇招招手,等他凑近广使伸开一个僵直的手指,小心翼翼地一根根拨弄着火柴.口里小声数着:“一,二,三,四……”’一共只有七根火柴,他却数了很长时间。数完了,又向卢进勇望了一眼,意思好像说:“看明白了?”
    
    “是,看明白了!”卢进勇高兴地点点头,心想:这下子可好办了!他仿佛看见了一个通红的火堆,他正抱着这个同志偎依在火旁……
    
    就在这一瞬间,他发现那个同志的脸色好像舒展开来,眼睛里那死灰般的颜色忽然不见了,发射出一种喜悦的光。那同志合拢了夹着火柴的党证,双手捧起,像擎着一只贮满水的碗一样,小心地放到卢进勇的手里,紧紧地把它连手握在一起,两眼直直地盯着卢进勇的脸。
    
    “记住,这,这是,大家的!”他蓦地抽回手去,深深地吸了一口气,用尽所有的力气举起手来,直指着正北方向:“好,好同志……你……你把它带给……’,话就在这里停住了。卢进勇觉得自己的臂弯猛然沉了下去!他的眼睛模糊了。远处的树、近处的草、那湿漉漉的衣服、那双紧闭的眼睛……一切都像整个草地一样,雾蒙蒙的;只有那只手是清晰的,它高高地攀着,像一只路标,笔直地指向长征部队前进的方向……
    
    这以后的路,卢进勇走得特别快。天黑的时候,他追上了后卫部队。
    
    在无边的暗夜里,一簇簇的黄火烧起来了。在风雨中、在烂泥里跃滚了几天的战士们,围着这熊熊的野火谈笑着,湿透的衣服上冒起一层雾气,洋瓷碗\footnote{〔洋瓷〕搪瓷,涂烧在金属底坯表面上的无机玻璃瓷釉。}里的野菜“前南”地响着……
    
    卢进勇悄悄走到后卫连指导员的身边。映着那闪闪跳动的火光,他用颤抖的手指打开了那个党证,把剩下的六根火柴一根根递到指导员的手里,何时,以一种异样的声调在数着:“一,二,三,四……”
    
\end{normalsize}


\newpage

\textbf{注释}:

\vspace{-1em}

\begin{itemize}
    \setlength\itemsep{-0.2em}
    \item 〔蹒跚〕走动迟缓、摇晃不稳的样子。
    \item 〔惶惑〕慌乱,惊慌害怕而迷茫。
    \item 〔霎时〕立刻,极短时间内。
    \item 〔抖抖索索〕不停地颤抖。
    \item 〔一张一翕〕一张一合。
    \item 〔翕动〕一张一合地动。
    \item 〔一瘸一拐〕行走时无法保持身体平衡,双腿动作不自然。
    \item 〔脸颊〕脸的两侧。
    \item 〔蓦地〕突然地,没有预兆地。
\end{itemize}

\chapter{最后一次的讲演}

\begin{normalsize}
    
    这几天,大家晓得,在昆明出现了历史上最卑劣,最无耻的事情!李先生\footnote{〔李先生〕指李公朴。李公朴:中国社会教育家,中国民主同盟的发起人之一。}究竟犯了甚么罪?竟遭此毒手,他只不过用笔写写文章,用嘴说说话,而他所写的,所说的,都无非是一个没有失掉良心的中国人的话!大家都有一只笔有一张嘴,有什么理由拿出来讲啊!有事实拿出来说啊!为什么要打要杀,而且又不敢光明正大的来打来杀,而偷偷摸摸的来暗杀!这成什么话?
    
    今天,这里有没有特务!你站出来,是好汉的站出来!你出来讲!凭什么要杀死李先生?杀死了人,又不敢承认,还要诬蔑人,说什么“桃色案件”\footnote{〔“桃色案件”〕云南省警备代理总司令霍揆彰暗杀李公朴后,指示散播谣言。},说什么共产党杀共产党,无耻啊!无耻啊!这是某集团的无耻,恰是李先生的光荣!李先生在昆明被暗杀,是李先生留给昆明的光荣!也是昆明人的光荣!
    
    去年“一二·一”\footnote{〔“一二·一”〕1945年10月12月1日,国民党特务镇压云南大学反战运动,杀死师生四人,称为“一二·一”惨案。}昆明青年学生为了反对内战,遭受屠杀,那算是年青的一代,献出了他们的血,献出了他们最宝贵的生命!现在李先生为了争取民主和平,而遭受了反动派的暗杀,我们骄傲一点说,这算是像我这样大年纪的一代,我们的老战友,献出了最宝贵的生命。这两桩事发生在昆明,这算是昆明无限的光荣!
    
    反动派暗杀李先生的消息传出后,大家听了都摇头。我心里想,这些无耻的东西,不知他们是怎么想法?他们的心理是什么状态?他们的心是怎么长的?其实很简单,他们这样疯狂的来制造恐怖,正是他们自己在慌啊!在害怕啊!所以他们制造恐怖,其实是他们自己在恐怖啊!特务们,你们想想,你们还有几天,你们完了,快完了!你们以为打伤几个,杀死几个,就可以了事,就可以把人民吓倒了吗?其实广大的人民是打不尽的,杀不完的,要是这样可以的话,世界上早没有人了。你们杀死了一个李公朴,会有千百万个李公朴站起来!你们将失去千百万的人民!你们看着我们人少,没有力量。告诉你们,我们的力量大得很!多得很!看今天来的这些人,都是我们的人,都是我们的力量!此外还有广大的市民!我们有这个信心:人民的力量是要胜利的,真理是永远存在的。历史上没有一个反人民的势力不被人民毁灭的!希特勒\footnote{〔希特勒〕阿道夫·希特勒,纳粹德国元首,法西斯主义者,发动了第二次世界大战。},莫索里尼\footnote{〔莫索里尼〕贝尼托·墨索里尼,二战时期意大利的独裁者,法西斯主义者。}不都在人民之前倒下去了吗?翻开历史看看,你还站得住几天!你完了,快完了!我们的光明就要出现了。我们看,光明就在我们的眼前,而现在正是黎明之前那个最黑暗的时候。我们有力量打破这个黑暗,争到光明!我们的光明,就是反动派的末日!
    
    李先生的血,不会白流的。李先生赔上了这条性命,我们要换来一个代价。“一二·一”四烈士倒下了,年青的战士们的血,换来了政治协商会议\footnote{〔政治协商会议〕指按照国共《双十协定》,于1946年1月在重庆召开的会议。}的召开,现在李先生倒下了,他的血要换取政协会议的重开!我们有这个信心!
    
    “一二·一”是昆明的光荣,是云南人民的光荣。云南有光荣的历史,远的如护国\footnote{〔护国〕指护国战争,1915年南方各省反对袁世凯复辟帝制的运动。},这不用说了,近的如“一二·一”,都是属于云南人民的,我们要发扬云南光荣的历史!
    
    反动派挑拨离间,卑鄙无耻,你们看见联大\footnote{〔联大〕指国立西南联合大学,由抗战时期迁移到云南的多所大学联合而成。1946年5月联大宣布结束,师生开始回迁。}走了,学生放暑假了,便以为我们没有力量了吗?特务们!你们错了!你们看看今天到会的一千多青年,又握起手来了,我们昆明的青年决不会让你们这样横干下去的!
    
    正义是杀不完的,因为真理永远存在!
    
    历史赋予昆明的任务是争取民主和平,我们昆明的青年必须完成这任务!
    
    我们不怕死,我们有牺牲的精神,我们随时像李先生一样,前脚跨出大门,后脚就不准备再跨进大门!
    
\end{normalsize}


\newpage

\textbf{注释}:

\vspace{-1em}

\begin{itemize}
    \setlength\itemsep{-0.2em}
    \item 〔挑拨离间〕搬弄是非,制造矛盾,煽动仇恨,破坏团结。
    \item 〔诬蔑〕捏造事实来损害别人的名誉。
    \item 〔赋予〕给予,交给,寄托。
\end{itemize}

\chapter{藤野先生}

\begin{normalsize}
    
    东京\footnote{〔东京〕日本首都。}也无非是这样。上野\footnote{〔上野〕这里指东京台东区的上野公园。}的樱花烂熳\footnote{〔烂熳〕烂漫。}的时节,望去确也象绯红的轻云,但花下也缺不了成群结队的“清国留学生”的速成班\footnote{〔“清国留学生”的速成班〕指清末到日本留学,先在东京弘文书院速成班学习日语的中国学生。},头顶上盘着大辫子,顶得学生制帽的顶上高高耸起,形成一座富士山\footnote{〔富士山〕日本第一高峰,位于本州岛中南部,为活火山,山体圆锥形,高3600米。在日本文化中有重要地位。}。也有解散辫子,盘得平的,除下帽来,油光可鉴\footnote{〔油光可鉴〕指油亮得像镜子一样可以照人。},宛如小姑娘的发髻一般,还要将脖子扭几扭。实在标致极了。
    
    中国留学生会馆\footnote{〔会馆〕古代同乡、同业的人在京城等大都市、大商埠设立的机构,为同乡、同业提供聚会场所和住宿。这里指设在东京供中国留学生活动、居住的场所。}的门房里有几本书买,有时还值得去一转;倘在上午,里面的几间洋房里倒也还可以坐坐的。但到傍晚,有一间的地板便常不免要咚咚咚地响得震天,兼以满房烟尘斗乱\footnote{〔斗乱〕飞舞杂乱。斗:抖。};问问精通时事的人,答道,“那是在学跳舞。”
    
    到别的地方去看看,如何呢?
    
    我就往仙台\footnote{〔仙台〕日本城市名,在本州岛东北,距离东京约400公里。}的医学专门学校去。从东京出发,不久便到一处驿站,写道:日暮里。不知怎地,我到现在还记得这名目。其次却只记得水户了,这是明的遗民朱舜水\footnote{〔朱舜水〕朱之瑜,明清之际的学者和教育家,流亡日本,在水户讲学。}先生客死的地方。仙台是一个市镇,并不大;冬天冷得厉害;还没有中国的学生。
    
    大概是物以希为贵罢。北京的白菜运往浙江,便用红头绳系住菜根,倒挂在水果店头,尊为“胶菜”;福建野生着的芦荟,一到北京就请进温室,且美其名曰“龙舌兰”。我到仙台也颇受了这样的优待,不但学校不收学费,几个职员还为我的食宿操心。我先是住在监狱旁边一个客店里的,初冬已经颇冷,蚊子却还多,后来用被盖了全身,用衣服包了头脸,只留两个鼻孔出气。在这呼吸不息的地方,蚊子竟无从插嘴,居然睡安稳了。饭食也不坏。但一位先生却以为这客店也包办囚人的饭食,我住在那里不相宜,几次三番,几次三番地说。我虽然觉得客店兼办囚人的饭食和我不相干,然而好意难却,也只得别寻相宜的住处了。于是搬到别一家,离监狱也很远,可惜每天总要喝难以下咽的芋梗汤。
    
    从此就看见许多陌生的先生,听到许多新鲜的讲义\footnote{〔照相〕照片、相片。}。解剖学\footnote{〔解剖学〕研究生命体的结构和组织的学科。}是两个教授分任的。最初是骨学。其时进来的是一个黑瘦的先生,八字须,戴着眼镜,挟着一迭\footnote{〔一迭〕一叠。}大大小小的书。一将书放在讲台上,便用了缓慢而很有顿挫的声调,向学生介绍自己道:
    
    “我就是叫作藤野严九郎的……。”
    
    后面有几个人笑起来了。他接着便讲述解剖学在日本发达\footnote{〔发达〕发展。}的历史,那些大大小小的书,便是从最初到现今关于这一门学问的著作。起初有几本是线装的;还有翻刻中国译本的,他们的翻译和研究新的医学,并不比中国早。
    
    那坐在后面发笑的是上学年不及格的留级学生,在校已经一年,掌故\footnote{〔掌故〕历史上的制度、文化沿革以及人物事迹等。这里指学校的旧事。}颇为熟悉的了。他们便给新生讲演每个教授的历史。这藤野先生,据说是穿衣服太模糊\footnote{〔模糊〕这里指穿着打扮随意,不讲究。}了,有时竟会忘记带领结;冬天是一件旧外套,寒颤颤的,有一回上火车去,致使管车的疑心他是扒手,叫车里的客人大家小心些。
    
    他们的话大概是真的,我就亲见他有一次上讲堂没有带领结。
    
    过了一星期,大约是星期六,他使助手来叫我了。到得研究室,见他坐在人骨和许多单独的头骨中间,——他其时正在研究着头骨,后来有一篇论文在本校的杂志上发表出来。
    
    “我的讲义,你能抄下来么?”他问。
    
    “可以抄一点。”
    
    “拿来我看!”
    
    我交出所抄的讲义去,他收下了,第二三天便还我,并且说,此后每一星期要送给他看一回。我拿下来打开看时,很吃了一惊,同时也感到一种不安和感激。原来我的讲义已经从头到末,都用红笔添改过了,不但增加了许多脱漏的地方,连文法的错误,也都一一订正。这样一直继续到教完了他所担任的功课\footnote{〔功课〕指课程。}:骨学、血管学、神经学。
    
    可惜我那时太不用功,有时也很任性。还记得有一回藤野先生将我叫到他的研究室里去,翻出我那讲义上的一个图来,是下臂的血管,指着,向我和蔼的说道:
    
    “你看,你将这条血管移了一点位置了。——自然,这样一移,的确比较的好看些,然而解剖图不是美术,实物是那么样的,我们没法改换它。现在我给你改好了,以后你要全照着黑板上那样的画。”
    
    但是我还不服气,口头答应着,心里却想道:
    
    “图还是我画的不错;至于实在的情形,我心里自然记得的。”
    
    学年试验完毕之后,我便到东京玩了一夏天,秋初再回学校,成绩早已发表了,同学一百余人之中,我在中间,不过是没有落第。这回藤野先生所担任的功课,是解剖实习和局部解剖学。
    
    解剖实习了大概一星期,他又叫我去了,很高兴地,仍用了极有抑扬的声调对我说道:
    
    “我因为听说中国人是很敬重鬼的,所以很担心,怕你不肯解剖尸体。现在总算放心了,没有这回事。”
    
    但他也偶有使我很为难的时候。他听说中国的女人是裹脚的,但不知道详细,所以要问我怎么裹法,足骨变成怎样的畸形,还叹息道,“总要看一看才知道。究竟是怎么一回事呢?”
    
    有一天,本级的学生会干事到我寓里来了,要借我的讲义看。我检出来交给他们,却只翻检了一通,并没有带走。但他们一走,邮差就送到一封很厚的信,拆开看时,第一句是:
    
    “你改悔罢!”
    
    这是《新约》\footnote{〔《新约》〕基督教的圣经。讲述耶稣的故事,和犹太教经典区别,后者称为《旧约》。}上的句子罢,但经托尔斯泰\footnote{〔托尔斯泰〕列夫·托尔斯泰,19世纪俄罗斯作家。主要著作有《战争与和平》《安娜·卡列尼娜》《复活》等。}新近引用过的。其时正值日俄战争\footnote{〔日俄战争〕1904至1905年日本与俄罗斯为争夺在我国东北和朝鲜的殖民利益而进行的战争,主要在我国境内进行。},托老先生便写了一封给俄国和日本的皇帝的信\footnote{〔托老先生……信〕托尔斯泰写给日本和俄罗斯君主的信。刊登在1904年6月27日英国《泰晤士报》上。两个月后翻译成日语登载在日本《平民新闻》上。},开首便是这一句。日本报纸上很斥责他的不逊,爱国青年也愤然,然而暗地里却早受了他的影响了。其次的话,大略是说上年解剖学试验的题目,是藤野先生讲义上做了记号,我预先知道的,所以能有这样的成绩。末尾是匿名。 我这才回忆到前几天的一件事。因为要开同级会,干事便在黑板上写广告,末一句是“请全数到会勿漏为要”,而且在“漏”字旁边加了一个圈。我当时虽然觉到圈得可笑,但是毫不介意,这回才悟出那字也在讥刺我了,犹言我得了教员漏泄出来的题目。
    
    我便将这事告知了藤野先生;有几个和我熟识的同学也很不平,一同去诘责干事托辞检查的无礼,并且要求他们将检查的结果,发表出来。终于这流言消灭了,干事却又竭力运动,要收回那一封匿名信去。结末是我便将这托尔斯泰式的信退还了他们。
    
    中国是弱国,所以中国人当然是低能儿,分数在六十分以上,便不是自己的能力了:也无怪他们疑惑。但我接着便有参观枪毙中国人的命运了。第二年添教霉菌学\footnote{〔霉菌学〕研究真菌的学科。霉菌:真菌。},细菌的形状是全用电影\footnote{〔电影〕这里指幻灯片。}来显示的,一段落已完而还没有到下课的时候,便影\footnote{〔影〕放映。}几片时事的片子,自然都是日本战胜俄国的情形。但偏有中国人夹在里边:给俄国人做侦探,被日本军捕获,要枪毙了,围着看的也是一群中国人;在讲堂里的还有一个我。
    
    “万岁!”他们都拍掌欢呼起来。
    
    这种欢呼,是每看一片都有的,但在我,这一声却特别听得刺耳。此后回到中国来,我看见那些闲看枪毙犯人的人们,他们也何尝不酒醉似的喝彩,——呜呼,无法可想!但在那时那地,我的意见\footnote{〔意见〕想法。}却变化了。
    
    到第二学年的终结\footnote{〔终结〕结束。},我便去寻藤野先生,告诉他我将不学医学,并且离开这仙台。他的脸色仿佛有些悲哀,似乎想说话,但竟没有说。
    
    “我想去学生物学\footnote{〔生物学〕研究生命现象和生命活动规律的学科。},先生教给我的学问,也还有用的。”其实我并没有决意要学生物学,因为看得他有些凄然,便说了一个慰安\footnote{〔慰安〕安慰。}他的谎话。
    
    “为医学而教的解剖学之类,怕于生物学也没有什么大帮助。”他叹息说。
    
    将走的前几天,他叫我到他家里去,交给我一张照相\footnote{〔照相〕照片、相片。},后面写着两个字道:“惜别”,还说希望将我的也送他。但我这时适值没有照相了;他便叮嘱我将来照了寄给他,并且时时通信告诉他此后的状况。
    
    我离开仙台之后,就多年没有照过相,又因为状况也无聊,说起来无非使他失望,便连信也怕敢写了。经过的年月一多,话更无从说起,所以虽然有时想写信,却又难以下笔,这样的一直到现在,竟没有寄过一封信和一张照片。从他那一面看起来,是一去之后,杳无消息了。
    
    但不知怎地,我总还时时记起他,在我所认为我师的之中,他是最使我感激,给我鼓励的一个。有时我常常想:他的对于我的热心的希望,不倦的教诲,小而言之,是为中国,就是希望中国有新的医学;大而言之,是为学术,就是希望新的医学传到中国去。他的性格,在我的眼里和心里是伟大的,虽然他的姓名并不为许多人所知道。
    
    他所改正的讲义,我曾经订成三厚本,收藏着的,将作为永久的纪念。不幸七年前迁居的时候\footnote{〔七年前迁居〕指1919年12月鲁迅从绍兴搬家到北京。},中途毁坏了一口书箱,失去半箱书,恰巧这讲义也遗失在内了\footnote{〔这讲义也遗失〕《解剖学笔记》后来1951年从鲁迅家藏书中找到。现存于鲁迅纪念馆。}。责成运送局去找寻,寂无回信。只有他的照相至今还挂在我北京寓居的东墙上,书桌对面。每当夜间疲倦,正想偷懒时,仰面在灯光中瞥见他黑瘦的面貌,似乎正要说出抑扬顿挫的话来,便使我忽又良心发现,而且增加勇气了,于是点上一枝烟,再继续写些为“正人君子”之流所深恶痛疾的文字。
    
    \hfill 十月十二日
    
\end{normalsize}


\newpage

\textbf{注释}:

\vspace{-1em}

\begin{itemize}
    \setlength\itemsep{-0.2em}
    \item 〔不逊〕不恭敬,没有礼貌。
    \item 〔讥刺〕暗中挖苦,说坏话。讥:旁敲侧击地批评。刺:用尖锐的话指出别人的坏处。
    \item 〔诘责〕质问并责备。诘:追问。
    \item 〔匿名〕隐藏名字,不表露身份。
    \item 〔侦探〕受托探听、调查的人。
    \item 〔适值〕正好处于……的时候。
    \item 〔托辞〕借口。找借口。
    \item 〔畸形〕生物某部分在发育中形成的不正常的形状。
    \item 〔杳无消息〕毫无消息。杳:幽深,高远。
    \item 〔深恶痛疾〕极端厌恶仇恨。痛:深切地、彻底地。疾:恨。
\end{itemize}

\chapter{孔乙己}

\begin{normalsize}
    
    鲁镇的酒店\footnote{〔酒店〕提供酒水、小食的小店,也叫酒吧、酒馆。}的格局,是和别处不同的:都是当街一个曲尺形\footnote{〔曲尺形〕曲尺的形状。曲尺:由张成直角的两个直尺组成的尺子,用于画直角,也叫角尺、拐尺。}的大柜台,柜里面预备着热水,可以随时温酒。做工的人,傍午傍晚散了工,每每花四文\footnote{〔文〕铜钱的单位。一枚铜钱为一文。}铜钱,买一碗酒,——这是二十多年前的事,现在每碗要涨到十文,——靠柜外站着,热热的喝了休息;倘肯多花一文,便可以买一碟盐煮笋,或者茴香豆,做下酒物了,如果出到十几文,那就能买一样荤菜,但这些顾客,多是短衣帮,大抵没有这样阔绰。只有穿长衫的,才踱进店面隔壁的房子里,要酒要菜,慢慢地坐喝。
    
    我从十二岁起,便在镇口的咸亨酒店里当伙计\footnote{〔伙计〕餐厅酒馆里服侍顾客的雇员,也叫小二。},掌柜说,我样子太傻,怕侍候不了长衫主顾,就在外面做点事罢。外面的短衣主顾,虽然容易说话,但唠唠叨叨缠夹不清的也很不少。他们往往要亲眼看着黄酒从坛子里舀出,看过壶子底里有水没有,又亲看将壶子放在热水里,然后放心:在这严重\footnote{〔严重〕严格。}监督下,羼水\footnote{〔羼水〕往酒里加水,也写作掺水。}也很为难。所以过了几天,掌柜又说我干不了这事。幸亏荐头\footnote{〔荐头〕介绍佣工为职业的人,用工中介。}的情面大,辞退不得,便改为专管温酒的一种无聊职务了。
    
    我从此便整天的站在柜台里,专管我的职务。虽然没有什么失职,但总觉得有些单调,有些无聊。掌柜是一副凶脸孔,主顾也没有好声气,教人活泼不得;只有孔乙己到店,才可以笑几声,所以至今还记得。
    
    孔乙己是站着喝酒而穿长衫的唯一的人。他身材很高大;青白脸色,皱纹间时常夹些伤痕;一部乱蓬蓬的花白的胡子。穿的虽然是长衫,可是又脏又破,似乎十多年没有补,也没有洗。他对人说话,总是满口之乎者也,叫人半懂不懂的。因为他姓孔,别人便从描红\footnote{〔描红〕习字手段。用墨笔在红字上描着抄写。}纸上的“上大人孔乙己”\footnote{〔“上大人孔乙己”〕一种古代儿童习字用的启蒙短文的开头六个字。}这半懂不懂的话里,替他取下一个绰号,叫作孔乙己。孔乙己一到店,所有喝酒的人便都看着他笑,有的叫道,“孔乙己,你脸上又添上新伤疤了!”他不回答,对柜里说,“温两碗酒,要一碟茴香豆。”便排出九文大钱\footnote{〔大钱〕清朝咸丰年间铸造的劣质铜铁货币。}。他们又故意的高声嚷道,“你一定又偷了人家的东西了!”孔乙己睁大眼睛说,“你怎么这样凭空污人清白……”“什么清白?我前天亲眼见你偷了何家的书,吊着打。”孔乙己便涨红了脸,额上的青筋条条绽出\footnote{〔绽出〕指激动时血管鼓起像裂纹。},争辩道,“窃书不能算偷……窃书!……读书人的事,能算偷么?”接连便是难懂的话,什么“君子固穷\footnote{〔君子固穷〕出自《论语·卫灵公》。意思是君子不因为贫穷而改变品行操守。}”,什么“者乎”之类,引得众人都哄笑起来:店内外充满了快活的空气。
    
    听人家背地里谈论,孔乙己原来也读过书,但终于没有进学\footnote{〔进学〕清代指通过科举院试,进入官学的人,称为生员或秀才。},又不会营生;于是愈过愈穷,弄到将要讨饭了。幸而写得一笔好字,便替人家抄抄书,换一碗饭吃。可惜他又有一样坏脾气,便是好喝懒做。坐不到几天,便连人和书籍纸张笔砚,一齐失踪。如是几次,叫他抄书的人也没有了。孔乙己没有法,便免不了偶然做些偷窃的事。但他在我们店里,品行却比别人都好,就是从不拖欠;虽然间或没有现钱,暂时记在粉板\footnote{〔粉板〕旧时店铺里用来记事的一种白漆木板。}上,但不出一月,定然还清,从粉板上拭去了孔乙己的名字。
    
    孔乙己喝过半碗酒,涨红的脸色渐渐复了原,旁人便又问道,“孔乙己,你当真认识字么?”孔乙己看着问他的人,显出不屑置辩的神气。他们便接着说道,“你怎的连半个秀才\footnote{〔秀才〕清代科举中,通过院试称为秀才,也叫生员。秀才可免除自身赋税徭役,出入公堂,不被随意惩处用刑。}也捞不到呢?”孔乙己立刻显出颓唐不安模样,脸上笼上了一层灰色,嘴里说些话;这回可是全是之乎者也之类,一些不懂了。在这时候,众人也都哄笑起来:店内外充满了快活的空气。
    
    在这些时候,我可以附和着笑,掌柜是决不责备的。而且掌柜见了孔乙己,也每每这样问他,引人发笑。孔乙己自己知道不能和他们谈天,便只好向孩子说话。有一回对我说道,“你读过书么?”我略略点一点头。他说,“读过书,……我便考你一考。茴香豆的茴字,怎样写的?”我想,讨饭一样的人,也配考我么?便回过脸去,不再理会。孔乙己等了许久,很恳切的说道,“不能写罢?……我教给你,记着!这些字应该记着。将来做掌柜的时候,写账要用。”我暗想我和掌柜的等级还很远呢,而且我们掌柜也从不将茴香豆上账;又好笑,又不耐烦,懒懒的答他道,“谁要你教,不是草头底下一个来回的回字么?”孔乙己显出极高兴的样子,将两个指头的长指甲敲着柜台,点头说,“对呀对呀!……回字有四样写法,你知道么?\footnote{〔回字有四样写法〕“回”字较常见的写法有三种:“回”“囘”“囬”,其他写法都极罕见。}”我愈不耐烦了,努着嘴\footnote{〔努嘴〕翘起嘴唇。}走远。孔乙己刚用指甲蘸了酒,想在柜上写字,见我毫不热心,便又叹一口气,显出极惋惜的样子。
    
    有几回,邻居孩子听得笑声,也赶热闹,围住了孔乙己。他便给他们一人一颗。孩子吃完豆,仍然不散,眼睛都望着碟子。孔乙己着了慌,伸开五指将碟子罩住,弯腰下去说道,“不多了,我已经不多了。”直起身又看一看豆,自己摇头说,“不多不多!多乎哉?不多也。\footnote{〔多乎哉?不多也。〕出自《论语·子罕》。}”于是这一群孩子都在笑声里走散了。
    
    孔乙己是这样的使人快活,可是没有他,别人也便这么过。
    
    有一天,大约是中秋前的两三天,掌柜正在慢慢的结账,取下粉板,忽然说,“孔乙己长久没有来了。还欠十九个钱呢!”我才也觉得他的确长久没有来了。一个喝酒的人说道,“他怎么会来?……他打折了腿了。”掌柜说,“哦!”“他总仍旧是偷。这一回,是自己发昏,竟偷到丁举人\footnote{〔举人〕清代科举中,秀才通过乡试,称为举人(指被荐举的人)。举人有做官的资格,可以免除百人的税赋徭役,可通过出租免税名额获得稳定收入。}家里去了。他家的东西,偷得的吗?”“后来怎么样?”“怎么样?先写服辩\footnote{〔服辩〕认罪书。这里是说孔乙己为了私了而认罪。},后来是打,打了大半夜,再打折了腿。”“后来呢?”“后来打折了腿了。”“打折了怎样呢?”“怎样?……谁晓得?许是死了。”掌柜也不再问,仍然慢慢的算他的账。
    
    中秋过后,秋风是一天凉比一天,看看将近初冬;我整天的靠着火,也须穿上棉袄了。一天的下半天,没有一个顾客,我正合了眼坐着。忽然间听得一个声音,“温一碗酒。”这声音虽然极低,却很耳熟。看时又全没有人。站起来向外一望,那孔乙己便在柜台下对了门槛坐着。他脸上黑而且瘦,已经不成样子;穿一件破夹袄\footnote{〔夹袄〕双层的上衣。},盘着两腿,下面垫一个蒲包,用草绳在肩上挂住;见了我,又说道,“温一碗酒。”掌柜也伸出头去,一面说,“孔乙己么?你还欠十九个钱呢!”孔乙己很颓唐的仰面答道,“这……下回还清罢。这一回是现钱,酒要好。”掌柜仍然同平常一样,笑着对他说,“孔乙己,你又偷了东西了!”但他这回却不十分分辩,单说了一句“不要取笑!”“取笑?要是不偷,怎么会打断腿?”孔乙己低声说道,“跌断,跌,跌……”他的眼色,很像恳求掌柜,不要再提。此时已经聚集了几个人,便和掌柜都笑了。我温了酒,端出去,放在门槛上。他从破衣袋里摸出四文大钱,放在我手里,见他满手是泥,原来他便用这手走来的。不一会,他喝完酒,便又在旁人的说笑声中,坐着用这手慢慢走去了。
    
    自此以后,又长久没有看见孔乙己。到了年关\footnote{〔年关〕农历年底。古代过年时生活困难,所以将过年比作过关。},掌柜取下粉板说,“孔乙己还欠十九个钱呢!”到第二年的端午,又说“孔乙己还欠十九个钱呢!”到中秋可是没有说,再到年关也没有看见他。
    
    我到现在终于没有见——大约孔乙己的确死了。
    
\end{normalsize}


\newpage

\textbf{注释}:

\vspace{-1em}

\begin{itemize}
    \setlength\itemsep{-0.2em}
    \item 〔当街〕就在街边。
    \item 〔荤菜〕荤:指辛辣的香菜,引申指肉食。
    \item 〔缠夹不清〕纠缠,把各种是非杂七杂八搅在一起。
    \item 〔绰号〕外号。
    \item 〔阔绰〕有钱,能花钱。
    \item 〔间或〕偶尔,有时候。
    \item 〔颓唐〕情绪低落,精神苦闷。
    \item 〔附和〕顺着别人的话,表示赞同或理解。
    \item 〔置辩〕值得辩论、辩解。
\end{itemize}

\chapter{鲁提辖拳打镇关西}

\begin{normalsize}
    
    三人上到潘家酒楼上,拣个齐楚阁儿\footnote{〔齐楚阁儿〕干净整洁的包房。}里坐下。鲁提辖\footnote{〔提辖〕宋代一类官职的称呼。多用于称呼负责武备、治安、采买等实际事务的主管。}坐了主位,李忠对席,史进下首坐了。酒保唱了喏,认得是鲁提辖,便道:“提辖官人\footnote{〔官人〕对有官职的人的敬称。},打多少酒?”鲁达道:“先打四角酒来\footnote{〔角〕古代盛酒器的泛称,一角大约0.6至0.8升。}。”一面铺下菜蔬、果品按酒,又问道:“官人,吃甚下饭?”鲁达道:“问甚么?但有,只顾卖来,一发算钱还你。这厮只顾来聒噪。”酒保下去,随即烫酒上来,但是\footnote{〔但是〕但凡是,只要是。}下口肉食,只顾将来\footnote{〔将来〕拿来,多指上菜。}\footnote{〔将来〕拿过来。将:拿,用。},摆一桌子。
    
    三个酒至数杯,正说些闲话,较量些枪法,说得入港\footnote{〔入港〕指谈得投入,谈得高兴。},只听得隔壁阁子里有人哽哽咽咽啼哭。鲁达焦躁,便把碟儿、盏儿,都丢在楼板上。酒保听得,慌忙上来看时,见鲁提辖气愤愤地。酒保抄手\footnote{〔抄手〕双手放到胸前,交互插在衣袖中。也指两臂交叉放在胸前。}道:“官人要甚东西,分付买来。”鲁达道:“洒家\footnote{〔洒家〕我。鲁达自称用语。}要甚么?你也须认的洒家,却恁地教甚么人在间壁\footnote{〔间壁〕隔壁。}吱吱的哭,搅俺弟兄们吃酒。洒家须不曾少了你酒钱!”酒保道:“官人息怒,小人怎敢教人啼哭,打搅官人吃酒。这个哭的,是绰酒座儿唱的父子两人\footnote{〔绰酒座〕驻唱,指在酒肆饭店桌边巡走,应客人要求唱歌,赚取小费收入。}。不知官人们在此吃酒,一时间自苦了啼哭。”鲁提辖道:“可是作怪!你与我唤的他来。”
    
    酒保去叫,不多时,只见两个到来:前面一个十八九岁的妇人,背后一个五六十岁的老儿,手里拿串拍板,都来到面前。那妇人,虽无十分的容貌,也有些动人的颜色,但拭着眼泪,向前来深深的道了三个万福\footnote{〔万福〕古代女子行的敬礼。双手轻轻抱拳在胸前右下侧上下移动,同时做鞠躬的姿势。}。那老儿也都相见了。鲁达问道:“你两个是那里人家\footnote{〔那里〕哪里。古白话中“哪”“那”不分。}?为甚啼哭?”那妇人便道:“官人不知,容奴\footnote{〔奴〕古代女性谦称自己。}告禀:奴家是东京\footnote{〔东京〕指北宋开封府治所汴梁,北宋的首都,又称汴京。}人氏。因同父母来这渭州,投奔亲眷,不想搬移南京去了。母亲在客店里染病身故,子父二人,流落在此生受。此间有个财主,叫做镇关西郑大官人,因见奴家,便使强媒硬保,要奴作妾。谁想写了三千贯文书\footnote{〔贯〕宋代货币单位,一贯是一千文。},虚钱实契\footnote{〔虚钱实契〕指签的契约中说已经付了钱,但其实没有付钱。},要了奴家身体。未及三个月,他家大娘子好生利害\footnote{〔利害〕厉害。},将奴赶打出来,不容完聚。着落\footnote{〔着落〕要求落实某事。}店主人家追要原典身钱三千贯\footnote{〔典〕典当,将物品寄放换钱,以后有钱了再拿钱换回来。这里“典身”是“卖身”的委婉说法。}。父亲懦弱,和他争执不得,他又有钱有势。当初不曾得他一文,如今那讨钱来还他?没计奈何,父亲自小教得奴家些小曲儿,来这里酒楼上赶座子。每日但得些钱来,将大半还他;留些少子父们盘缠\footnote{〔盘缠〕古代出远门时盘起缠在腰上的袋子,装金银等重要物件。比喻出远门路上的花费。}。这两日酒客稀少,违了他钱限,怕他来讨时,受他羞耻。子父们想起这苦楚来,无处告诉,因此啼哭。不想误触犯了官人,望乞恕罪,高抬贵手。”
    
    鲁提辖又问道:“你姓甚么?在那个客店里歇?那个镇关西郑大官人在那里住?”老儿答道:“老汉姓金,排行第二;孩儿小字翠莲;郑大官人便是此间状元桥下卖肉的郑屠,绰号镇关西。老汉父子两个,只在前面东门里鲁家客店安下。”鲁达听了道:“呸!俺只道哪个郑大官人,却原来是杀猪的郑屠。这个腌臜\footnote{〔腌臜〕肮脏,下贱。}泼才,投托着俺小种经略相公\footnote{〔小种经略相公〕指种师中,名将种世衡之孙,种师道的弟弟。种师道被称为“老种”,种师中被称为“小种”。经略,指经略使,北宋末期皇帝选派忠心能干的大臣到边疆掌管一路军政防务,但不管财赋漕运等民事。}门下做个肉铺户,却原来这等欺负人!”回头看着李忠、史进道:“你两个且在这里,等洒家去打死了那厮便来。”史进、李忠抱住劝道:“哥哥息怒,明日却理会。”两个三回五次劝得他住。
    
    鲁达又道:“老儿,你来,洒家与你些盘缠,明日便回东京去如何?”父子两个告道:“若是能够回乡去时,便是重生父母,再长爷娘。只是店主人家如何肯放?郑大官人须着落他要钱。”鲁提辖道:“这个不妨事,俺自有道理。”便去身边摸出五两来银子,放在桌上,看着史进道:“洒家今日不曾多带得些出来,你有银子,借些与俺,洒家明日便送还你。”史进道:“直甚么\footnote{〔直〕值。},要哥哥还。”去包裹里取出一锭十两银子,放在桌上。鲁达看着李忠道:“你也借些出来与洒家。”李忠去身边摸出二两来银子。鲁提辖看了见少,便道:“也是个不爽利的人。”鲁达只把十五两银子与了金老,分付道:“你父子两个将去做盘缠,一面收拾行李,俺明日清早来,发付你两个起身,看那个店主人敢留你!”金老并女儿拜谢去了。
    
    鲁达把这二两银子丢还了李忠。三人再吃了两角酒,下楼来叫道:“主人家,酒钱洒家明日送来还你。”主人家连声应道:“提辖只顾自去,但吃不妨,只怕提辖不来赊。”三个人出了潘家酒肆,到街上分手,史进、李忠各自投客店去了。只说鲁提辖回到经略府前下处\footnote{〔下处〕下脚处,临时的住所。},到房里,晚饭也不吃,气愤愤的睡了。主人家又不敢问他。
    
    再说金老得了这一十五两银子,回到店中,安顿了女儿。先去城外远处觅下一辆车儿,回来收拾了行李,还了房宿钱,算清了柴米钱,只等来日天明。当夜无事。
    
    次早五更起来,子父两个先打火做饭,吃罢,收拾了,天色微明,只见鲁提辖大踏步走入店里来,高声叫道:“店小二,那里是金老歇处?”小二哥道:“金公,提辖在此寻你。”金老开了房门,便道:“提辖官人,里面请坐。”鲁达道:“坐甚么?你去便去,等甚么?”金老引了女儿,挑了担儿,作谢提辖,便待出门,店小二拦住道:“金公,那里去?”鲁达问道:“他少你房钱?”小二道:“小人房钱,昨夜都算还了。须欠郑大官人典身钱,着落在小人身上看管他哩!”鲁提辖道:“郑屠的钱,洒家自还他。你放这老儿还乡去。”那店小二那里肯放。鲁达大怒,揸开五指,去那小二脸上只一掌,打的那店小二口中吐血;再复一拳,打下当门两个牙齿。小二扒将起来,一道烟走向店里去躲了。店主人那里敢出来拦他?金老父子两个,忙忙离了店中,出城自去寻昨日觅下的车儿去了。
    
    且说鲁达寻思:恐怕店小二赶去拦截他,且向店里掇条凳子,坐了两个时辰。约莫\footnote{〔约莫〕估摸,估计。}金公去的远了,方才起身,径到状元桥来。
    
    且说郑屠开着两间门面,两副肉案,悬挂着三五片猪肉。郑屠正在门前柜身内坐定,看那十来个刀手卖肉。鲁达走到面前,叫声:“郑屠!”郑屠看时,见是鲁提辖,慌忙出柜身来唱喏道:“提辖恕罪。”便叫副手:“掇条凳子来,提辖请坐。”鲁达坐下道:“奉着经略相公钧旨\footnote{〔钧旨〕对上司命令的敬称。},要十斤精肉,切做臊子,不要见半点肥的在上头。”郑屠道:“使得,你们快选好的,切十斤去。”鲁提辖道:“不要那等腌臜厮们动手,你自与我切。”郑屠道:“说得是。小人自切便了。”自去肉案上,拣下十斤精肉,细细切做臊子。那店小二把手帕包了头,正来郑屠家报说金老之事,却见鲁提辖坐在肉案门边,不敢拢来,只得远远的立住,在房檐下望。
    
    这郑屠整整的自切了半个时辰,用荷叶包了道:“提辖,教人送去。”鲁达道:“送甚么?且住!再要十斤,都是肥的,不要见些精的在上面,也要切做臊子。”郑屠道:“却才精的,怕府里要裹馄饨,肥的臊子何用?”鲁达睁着眼道:“相公钧旨,分付洒家,谁敢问他?”郑屠道:“是合用的东西,小人切便了。”又选了十斤实膘的肥肉,也细细的切做臊子,把荷叶来包了。整弄了一早晨,却得饭罢时候。那店小二那里敢过来,连那正要买肉的主顾,也不敢拢来。
    
    郑屠道:“着人与提辖拿了,送将府里去。”鲁达道:“再要十斤寸金软骨,也要细细地剁做臊子,不要见些肉在上面。”郑屠笑道:“却不是特地来消遣我!”鲁达听罢,跳起身来,拿着那两包臊子在手里,睁眼看着郑屠道:“洒家特地要消遣你!”把两包臊子,劈面打将去,却似下了一阵的肉雨。
    
    郑屠大怒,两条忿气从脚底下直冲到顶门心头。那一把无明业火\footnote{〔无明业火〕佛教用语,这里指怒火。}焰腾腾的按纳不住,从肉案上抢了一把剔骨尖刀,托地跳将下来。鲁提辖早拔步在当街上。众邻舍并十来个火家\footnote{〔火家〕伙计。},那个敢向前来劝?两边过路的人都立住了脚,和那店小二也惊的呆了。
    
    郑屠右手拿刀,左手便来要揪鲁达,被这鲁提辖就势按住左手,赶将入去,望小腹上只一脚,腾地踢倒在当街上,鲁达再入一步,踏住胸脯,提着那醋钵儿大小拳头\footnote{〔醋钵儿〕盛醋的小钵。钵:类似碗盆的一种平底器皿。},看着这郑屠道:“洒家始投老种经略相公,做到关西五路廉访使,也不枉了叫做镇关西。你是个卖肉的操刀屠户,狗一般的人,也叫做镇关西!你如何强骗了金翠莲?”扑的只一拳,正打在鼻子上,打得鲜血迸流,鼻子歪在半边,却便似开了个油酱铺,咸的、酸的、辣的,一发都滚出来。郑屠挣不起来\footnote{〔挣〕挣扎。},那把尖刀,也丢在一边,口里只叫:“打得好!”鲁达骂道:“直娘贼,还敢应口!”提起拳头来,就眼眶际眉梢只一拳,打得眼棱缝裂,乌珠迸出,也似开了个彩帛铺的,红的、黑的、绛的,都绽将出来。两边看的人,惧怕鲁提辖,谁敢向前来劝。郑屠当不过,讨饶。鲁达喝道:“咄!你是个破落户,若是和俺硬到底,洒家倒饶了你;你如何对俺讨饶,洒家偏不饶你。”又只一拳,太阳上正着,却似做了一个全堂水陆的道场\footnote{〔全堂水陆的道场〕道场,佛教为死人做法事的仪式。全堂水陆,指超度水中陆上一切亡灵。这里指做法事时各种乐器的声音。},磬儿、钹儿、铙儿一齐响。鲁达看时,只见郑屠挺在地下,口里只有出的气,没了入的气,动弹不得。鲁提辖假意道:“你这厮诈死,洒家再打。”只见面皮渐渐的变了。鲁达寻思道:“俺只指望痛打这厮一顿,不想三拳真个打死了他。洒家须吃官司,又没人送饭,不如及早撒开。”拔步便走,回头指着郑屠尸道:“你诈死,洒家和你慢慢理会。”一头骂,一头大踏步去了。街坊邻舍,并郑屠的火家,谁敢向前来拦他?
    
    鲁提辖回到下处,急急卷了些衣服、盘缠、细软、银两,但是旧衣粗重,都弃了。提了一条齐眉短棒,奔出南门,一道烟走了。
    
\end{normalsize}


\newpage

\textbf{注释}:

\vspace{-1em}

\begin{itemize}
    \setlength\itemsep{-0.2em}
    \item 〔聒噪〕烦人的吵闹。
    \item 〔禀〕下对上的报告。
    \item 〔掇〕拾取,用双手拿。
\end{itemize}

\chapter{回忆我的母亲}

\begin{normalsize}
    
    得到母亲去世的消息,我很悲痛。我爱我母亲,特别是她勤劳一生,很多事情是值得我永远回忆的。
    
    我家是佃农。祖籍广东韶关,客籍人\footnote{〔客籍人〕即客家人,汉族的一支民系。主要由五胡乱华至唐宋元时期从中原南迁的汉人组成。主要分布于广东广西北部、江西南部、福建西部以及东南亚地区。},在“湖广填四川”\footnote{〔“湖广填四川”〕清朝初期一次大规模移民。湖广,最初指元代设立的湖广行省。当时主要包括湖南、广西。清军灭明时在四川屠杀过多,导致西南地区人口锐减,因此清朝初期鼓动湖南、湖北、广西、广东、江西等地往西移民,持续近百年。}时迁移四川仪陇县马鞍场。世代为地主耕种,家境是贫苦的,和我们来往的朋友也都是老老实实的贫苦农民。
    
    母亲一共生了十三个儿女。因为家境贫穷,无法全部养活,只留下了八个,以后再生下的被迫溺死了。这在母亲心里是多么惨痛悲哀和无可奈何的事情啊!母亲把八个孩子一手养大成人。可是她的时间大半被家务和耕种占去了,没法多照顾孩子,只好让孩子们在地里爬着。
    
    母亲是个好劳动的人。从我能记忆时起,总是天不亮就起床。全家二十多口人,妇女们轮班煮饭,轮到就煮一年。母亲把饭煮了,还要种田、种菜、喂猪、养蚕、纺棉花。因为她身体高大结实,还能挑水挑粪。
    
    母亲这样地整日劳碌着。我到四五岁时就很自然地在旁边帮她的忙,到八九岁时就不但能挑能背,还会种地了。记得那时我从私塾\footnote{〔私塾〕私人出资设立的学校、课堂。}回家,常见母亲在灶上汗流满面地烧饭,我就悄悄把书一放,挑水或放牛去了。有的季节里,我上午读书,下午种地;一到农忙,便整日在地里跟着母亲劳动。这个时期母亲教给我许多生产知识。
    
    佃户\footnote{〔佃户〕向地主或官府租种土地的农民。}家庭的生活自然是艰苦的,可是由于母亲的聪明能干,也勉强过得下去。我们用桐子\footnote{〔桐子〕桐树的果实种子。桐树产于我国西南部地区,形似梧桐、种子可榨油,因此也叫“油桐”。}榨油来点灯,吃的是豌豆饭、菜饭、红薯饭、杂粮饭,把菜籽榨出的油放在饭里做调料。这类地主富人家看也不看的饭食,母亲却能做得使一家人吃起来有滋味。赶上丰年,才能缝上一些新衣服,衣服也是自己生产出来的。母亲亲手纺出线,请人织成布,染了颜色,我们叫它“家织布”,有铜钱那样厚。一套衣服老大穿过了,老二老三接着穿还穿不烂。
    
    勤劳的家庭是有规律有组织的。我的祖父是一个中国标本式的农民,到八九十岁还非耕田不可,不耕田就会害病,直到临死前不久还在地里劳动。祖母是家庭的组织者,一切生产事务由她管理分派,每年除夕就分派好一年的工作。每天天还没亮,母亲就第一个起身,接着听见祖父起来的声音,接着大家都离开床铺,喂猪的喂猪,砍柴的砍柴,挑水的挑水。母亲在家庭里极能任劳任怨。她性格和蔼,没有打骂过我们,也没有同任何人吵过架。因此,虽然在这样的大家庭里,长幼、伯叔、妯娌\footnote{〔妯娌〕兄弟的妻子的合称。}相处都很和睦。母亲同情贫苦的人——这是朴素的阶级意识,虽然自己不富裕,还周济和照顾比自己更穷的亲戚。她自己是很节省的。父亲有时吸点旱烟,喝点酒;母亲管束着我们,不允许我们染上一点。母亲那种勤劳俭朴的习惯,母亲那种宽厚仁慈的态度,至今还在我心中留有深刻的印象。
    
    但是灾难不因为中国农民的和平就不降临到他们身上。庚子年\footnote{〔庚子年〕即公元1900年。}前后,四川连年旱灾,很多的农民饥饿、破产,不得不成群结队地去“吃大户”。我亲眼见到,六七百穿得破破烂烂的农民和他们的妻子儿女被所谓官兵一阵凶杀毒打,血溅四五十里,哭声动天。在这样的年月里,我家也遭受更多的困难,仅仅吃些小菜叶、高粱,通年\footnote{〔通年〕整年,一年到头。}没吃过白米。特别是乙未\footnote{〔乙未〕即公元1895年。}那一年,地主欺压佃户,要在租种的地上加租子,因为办不到,就趁大年除夕,威胁着我家要退佃,逼着我们搬家。在悲惨的情况下,我们一家人哭泣着连夜分散。从此我家被迫分两处住下。人手少了,又遇天灾,庄稼没收成,这是我家最悲惨的一次遭遇。母亲没有灰心,她对穷苦农民的同情和对为富不仁者的反感却更强烈了。母亲沉痛的三言两语的诉说以及我亲眼见到的许多不平事实,启发了我幼年时期反抗压迫追求光明的思想,使我决心寻找新的生活。
    
    我不久就离开母亲,因为我读书了。我是一个佃农家庭的子弟,本来是没有钱读书的。那时乡间豪绅\footnote{〔豪绅〕地方上仗势欺人的士绅。豪:势大强横。}地主的欺压,衙门差役的横蛮,逼得母亲和父亲决心节衣缩食培养出一个读书人来“支撑门户”。我念过私塾,光绪三十一年\footnote{〔光绪三十一年〕即公元1905年。}考了科举,以后又到更远的顺庆和成都去读书。这个时候的学费都是东挪西借来的,总共用了二百多块钱,直到我后来当护国军\footnote{〔护国军〕指1915年蔡锷在护国运动中组织讨伐袁世凯的军队。}旅长时才还清。
    
    光绪三十四年\footnote{〔光绪三十四年〕即公元1908年。}我从成都回来,在仪陇县办高等小学\footnote{〔高等小学〕20世纪上半叶的教育制度,分为初等小学(四年)和高等小学(三年)。新中国成立后小学采用一贯制,不再区分。},一年回家两三次去看母亲。那时新旧思想冲突得很厉害。我们抱了科学民主的思想,想在家乡做点事情,守旧的豪绅们便出来反对我们。我决心瞒着母亲离开家乡,远走云南,参加新军和同盟会\footnote{〔新军和同盟会〕新军:甲午战争后清末朝廷的“新政”之一,采用西方军队的制度训练,招收知识分子当兵。同盟会:清末由兴中会、华兴会等团体集合组成的革命组织。1905年在日本东京成立,尝试推翻清朝统治,把新军作为笼络活动的对象。}。我到云南后,从家信中知道,我母亲对我这一举动不但不反对,还给我许多慰勉。
    
    从宣统元年\footnote{〔宣统元年〕即公元1909年。}到现在,我再没有回过一次家,只在民国八年\footnote{〔民国八年〕即公元1919年。}我曾经把父亲和母亲接出来。但是他俩劳动惯了,离开土地就不舒服,所以还是回了家。父亲就在回家途中死了。母亲回家继续劳动,一直到最后。
    
    中国革命继续向前发展,我的思想也继续向前发展。当我发现了中国革命的正确道路时,我便加入了中国共产党。大革命\footnote{〔大革命〕指1924年至1927年国民政府推翻帝国主义扶持的北洋军阀统治的革命运动。主要运动为1926年开始的北伐战争。}失败了,我和家庭完全隔绝了。母亲就靠那三十亩地独立支持一家人的生活。抗战以后,我才能和家里通信。母亲知道我所做的事业,她期望着中国民族解放的成功。她知道我们党的困难,依然在家里过着勤苦的农妇生活。七年中间,我曾寄回几百元钱和几张自己的照片给母亲。母亲年老了,但她永远想念着我,如同我永远想念着她一样。去年收到侄儿的来信说:“祖母今年已有八十五岁,精神不如昨年之健康,饮食起居亦不如前,甚望见你一面,聊叙别后情景。”但我献身于民族抗战事业,竟未能报答母亲的希望。
    
    母亲最大的特点是一生不曾脱离过劳动。母亲生我前一分钟还在灶上煮饭。虽到老年,仍然热爱生产。去年另一封外甥的家信中说:“外祖母大人因年老关系,今年不比往年健康,但仍不辍劳作,尤喜纺棉。”
    
    我应该感谢母亲,她教给我与困难作斗争的经验。我在家庭中已经饱尝艰苦,这使我在三十多年的军事生活和革命生活中再没感到过困难,没被困难吓倒。母亲又给我一个强健的身体,一个勤劳的习惯,使我从来没感到过劳累。
    
    我应该感谢母亲,她教给我生产的知识和革命的意志,鼓励我以后走上革命的道路。在这条路上,我一天比一天更加认识:只有这种知识,这种意志,才是世界上最可宝贵的财产。
    
    母亲现在离我而去了,我将永不能再见她一面了,这个哀痛是无法补救的。母亲是一个平凡的人,她只是中国千百万劳动人民中的一员,但是,正是这千百万人创造了和创造着中国的历史。我用什么方法来报答母亲的深恩呢?我将继续尽忠于我们的民族和人民,尽忠于我们的民族和人民的希望——中国共产党,使和母亲同样生活着的人能够过快乐的生活。这是我能做到的,一定能做到的。
    
    愿母亲在地下安息!
    
\end{normalsize}


\newpage

\textbf{注释}:

\vspace{-1em}

\begin{itemize}
    \setlength\itemsep{-0.2em}
    \item 〔溺死〕在水中无法呼吸而死。
    \item 〔劳碌〕做的事情多而辛苦。碌:忙。
    \item 〔和睦〕和平友爱相处,不争斗。
    \item 〔朴素〕没有加工修饰。这里指直接产生的、没有形成成熟理论的思想。朴:木头没有经过细加工。素:布没有染色。
    \item 〔安息〕安静休息。对死者表示哀悼的用语。
    \item 〔节衣缩食〕省吃省穿。
    \item 〔为富不仁〕作为富有的人,不讲仁义。
    \item 〔衙门〕古代官吏办公的地方。比喻官府。
    \item 〔差役〕在官府中办事的底层人员。差:派遣去办事,引申指办的事和办事的人。役:统治者强制看守边疆,引申指强制劳动和被强制劳动的人。
    \item 〔慰勉〕安慰勉励。
\end{itemize}

\chapter{人类的语言}

\begin{normalsize}
    
    语言,也就是说话,好像是极其稀松平常的事儿。可是仔细想想,实在是一件了不起的大事。正是因为说话跟吃饭、走路一样的平常,人们才不去想它究竟是怎么回事儿。其实这三件事儿都是极不平常的,都是使人类不同于别的高等动物的特征。别的动物都吃生的,只有人类会烧熟了吃。别的动物,除了天上飞的和水里游的,走路都是让身体跟地面平行,有几条腿使几条腿,只有人类直起身子来用两条腿走路,把上肢\footnote{〔上肢〕指哺乳类动物靠近头部的一对肢。}解放出来干别的、更重要的活儿。同样,别的动物的嘴只会吃东西,人类的嘴除了吃东西还会说话。
    
    记得在小学里读书的时候,班上有一位“能文”的大师兄,在一篇作文的开头写下这么两句:“鹦鹉能言,不离于禽;猩猩能言,不离于兽。”我们看了都非常佩服。后来知道这两句话是有来历的,只是字句有些出入。又过了若干年,才知道这两句话都有问题。鹦鹉能学人说话,可只是作为现成的公式\footnote{〔公式〕通用的方式方法。}来说,不会加以变化(所以我们管人云亦云的说话叫“鹦鹉学舌”)。只有人们的说话是从具体情况(包括外界情况和本人意图)出发,情况一变,话也跟着变。至于猩猩,根据西方学者拿黑猩猩做试验的结果,它们能学会极其有限的一点符号语言,可是学不会把它变成有声语言。人类语言之所以能够“随机应变”,在于一方面能够把语音分析成若干音素\footnote{〔音素〕语音的最小单位,音素变化会导致语义变化。}(当然是不自觉地),又把这些音素组合成音节\footnote{〔音节〕语音的最小结构单位。汉语中,一个音节基本对应一个字。},再把音节连缀起来,——音素数目有限,各种语言一般都只有几十个音素,可是组成音节就可以成百上千,再组成双音节、三音节,就能有几十万、几百万。另一方面,人们又能分析外界事物及其变化,形成无数的“意念”,——配以语音,然后综合运用,表达各种复杂的意思。一句话,人类语言的特点就在于能用变化无穷的语音,表达变化无穷的意义。这是任何其他动物办不到的。
    
    人类语言采用声音作为手段,而不采用手势或图画,也不是偶然的。人类的视觉最发达,可是语言诉之于听觉\footnote{〔诉之于〕采用、依靠(某事物、某种方法),也写作“诉诸”。诉:控告,求助。}。这是因为一切倚赖视觉的手段,要发挥作用,离不开光线,夜里不成,黑暗的地方或者有障碍物的地方也不成,声音则白天黑夜都可以发挥作用,也不容易受阻碍。手势之类,距离大了看不清,声音的有效距离大得多。打手势或者画画儿要用手,手就不能同时做别的事,说话用嘴,可以一边儿说话,一边儿劳动。论快慢,打手势赶不上说话,画画儿更赶不上。声音唯一不如形象的地方在于缺乏稳定性和持久性,但在原始社会的交际情况下,这方面的要求是次要的,是可以用图形来补充的。总之,正是由于采用了嘴里的声音作为手段,人类语言才得到前程万里的发展。
    
\end{normalsize}


\newpage

\textbf{注释}:

\vspace{-1em}

\begin{itemize}
    \setlength\itemsep{-0.2em}
    \item 〔稀松平常〕平常普通,不难办到。稀松:(土地)松散,容易耕作。
    \item 〔人云亦云〕别人说什么,也跟着说什么。没有主见。云:说。
    \item 〔符号〕传达特定意义的印记、标识。
    \item 〔倚赖〕依赖,依靠。
\end{itemize}

\chapter{食物从何处来}

\begin{normalsize}
    
    一切生物都离不开食物。如何获得食物?这有两种不同的途径和方法。
    
    一种叫自养,绿色植物都属于这一类。他们自己把无机物\footnote{〔无机物〕传统上把生物机体的部分称为有机物,此外的物质称为无机物。现代科学已经发现,生物体内也有传统所说的无机物。现在一般把有碳骨架的分子结构称为有机物,把不含碳的分子结构称为无机物。一些含碳的物质也依照传统称为无机物。}制造成有机的食物,满足生长的需要。
    
    绿色开花的植物有庞大得惊人的根系。每条根的尖端都有很多根毛,每一个根毛就是一个最基层的原料采集站。大量吸收土壤中的水分和无机盐\footnote{〔无机盐〕一类无机化合物,由酸根离子和金属离子组成,是很多矿石的主要成分,所以也叫矿物质。代表是食盐。}等原料,经过运输干线——茎,源源不断送入叶子里。叶子就是一个食品工厂。叶子上面有着许多气孔。在阳光下,这些气孔一面排出氧气和蒸腾水分,一面还吸入大量的二氧化碳\footnote{〔二氧化碳〕一种碳、氧的化合物,常态为气体,是空气的成分。}。有时,一个气孔在一秒钟内能吸进两万五千亿个二氧化碳分子。
    
    二氧化碳和水在合成车间——叶绿体里,发生奇妙的变化。叶绿体是叶绿素\footnote{〔叶绿素〕一种有机物,光合作用的关键。}和蛋白质\footnote{〔蛋白质〕人类食物主要养分之一。瘦肉、蛋白主要为蛋白质。}等组成的小颗粒,一个叶肉细胞里,一般含20至100个。叶子的绿色就是它们的颜色。叶绿素吸收了太阳的光能\footnote{〔光能〕光中的能量。},就把二氧化碳和水合成为含有高能\footnote{〔高能〕很多能量。}的有机物质(主要是碳水化合物),同时放出废气——氧,由气孔排出。这就是赫赫有名的光合作用。看来很简单,实际上是一个非常复杂的过程。
    
    植物合成了这些食物,大部分都用来组成躯体和贮藏在种子或块根、块茎\footnote{〔块根〕块根、块茎,即块状的根茎。}中,小部分经呼吸作用\footnote{〔呼吸作用〕生物体细胞把有机物和氧气转化为水和二氧化碳,并释放能量,支持生命活动的化学过程。}又被分解成水和二氧化碳,同时,放出能量,供给生命活动之用。
    
    另一种叫异养。所有的动物和大部分微生物\footnote{〔微生物〕肉眼看不见的微小生物的总称。}都是这一类。它们自己不能制造食物,靠植物来生活。
    
    例如,野兔靠吃野草来生活。狼以野兔为食物。狼一旦碰到了老虎,也就成了牺牲品。老虎死后,又成了细菌的乐园;不用多久,尸体就分解得精光,变成了二氧化碳、水和无机盐,回到大自然中,又成了植物制造食物的原料。
    
    所以兔、狼、虎、细菌,归根结底都是靠植物来生活。
    
    人,每天除了要吃进一定量的水和盐以外,还要吃淀粉\footnote{〔淀粉〕人类食物的主要养分之一。米面等谷物类食物主要为淀粉。}、蛋白质、脂肪\footnote{〔脂肪〕人类食物主要养分之一。肥肉、油脂主要为脂肪。}。我们皮肤上不会长出叶绿素,当然是属于异养型。吃荤也好,吃素也好,反正都是靠植物而生活。不过人是靠劳动获得食物的,能够用各种方式改造植物,使它更好地为人服务。很久以前,人们就懂得了农业,办起了绿色工厂,让庄稼来把二氧化碳和水变成食物。人们把其中营养最丰富的部分如种子、果实、块根、块茎等拿来做粮食;剩下的秸秆\footnote{〔秸秆〕农作物的茎杆。}、糠麸\footnote{〔糠麸〕谷物的外壳,加工时的副产品。 糠:从稻、麦等谷上脱落的皮。也就是谷的外壳。 麸:小麦磨面过箩后剩下的皮。}也是有机物,就再拿来办加工厂:养猪,养牛,养鸡。那些不好吃的东西经过猪、牛、鸡的消化吸收和转化,就变成了猪肉、牛奶、鸡蛋等高级食物。
    
    所以,世界上除了极个别的细菌能不依赖阳光而靠化学能\footnote{〔化学能〕化学反应中吸收或释放的能量。}来合成食物以外,其他一切生物都靠绿色植物的光合作用来获得食物。全世界的植物,一年中能制造出好几千亿吨有机物,这真是一个无比巨大的合成工厂。
    
\end{normalsize}


\newpage

\textbf{注释}:

\vspace{-1em}

\begin{itemize}
    \setlength\itemsep{-0.2em}
    \item 〔颗粒〕谷物、果实的种子。引申指很小的球体,很小的东西。
    \item 〔根系〕植物的所有根须。系:顶部连合的垂下的散丝,引申为从一点散出的线条。
    \item 〔蒸腾〕让水变成水蒸气升腾到空气中。腾:奔跑,跳跃,上升到空中。
    \item 〔赫赫有名〕形容非常有名。
    \item 〔归根结底〕归结到根基上。
\end{itemize}

\chapter{中国人民寻求救国真理的道路}

\begin{normalsize}
    
    我们党走过二十八年了,大家知道,不是和平地走过的,而是在困难的环境中走过的,我们要和国内外党内外的敌人作战。谢谢马克思、恩格斯、列宁\footnote{〔列宁〕弗拉基米尔·列宁,俄罗斯无产阶级革命家、政治家、理论家、思想家。领导俄罗斯十月革命取得成功,建立苏联。}和斯大林\footnote{〔斯大林〕约瑟夫·斯大林,苏联无产阶级革命家、思想家、政治家、军事家。列宁逝世后成为苏联最高领导人。},他们给了我们以武器。这武器不是机关枪,而是马克思列宁主义。
    
    列宁在1920年在《共产主义运动中的“左派”幼稚病》一书中,描写过俄国人寻找革命理论的经过\footnote{〔列宁在……〕指第二章中:“在将近半个世纪里,大约从上一世纪四十年代至九十年代,俄国进步的思想界在空前野蛮和反动的沙皇制度的压迫之下,曾如饥似渴地寻求正确的革命理论,专心致志地、密切地注视着欧美在这方面的每一种‘最新成就’。俄国在半个世纪里,经受了闻所未闻的痛苦和牺牲,表现了空前未有的革命英雄气概,以难以置信的毅力和舍身忘我的精神去探索、学习和实验,经受了失望,进行了验证,参照了欧洲的经验,真是饱经苦难找到了马克思主义这个唯一正确的革命理论。”}。俄国人曾经在几十个年头内,经历艰难困苦,方才找到了马克思主义。中国有许多事情和十月革命\footnote{〔十月革命〕1917年俄罗斯爆发的社会主义革命,建立了人类历史上首个由共产主义政党领导的无产阶级统治的社会主义国家:苏联。}以前的俄国相同,或者近似。封建主义的压迫,这是相同的。经济和文化落后,这是近似的。两个国家都落后,中国则更落后。先进的人们,为了使国家复兴,不惜艰苦奋斗,寻找革命真理,这是相同的。
    
    自从1840年鸦片战争\footnote{〔鸦片战争〕1840年英国为了倾销鸦片而侵略中国的战争,以中国战败,签订不平等的《南京条约》,割地赔款告终,是中国近代史的屈辱开端。}失败那时起,先进的中国人,经过千辛万苦,向西方国家寻找真理。洪秀全\footnote{〔洪秀全〕清晚期太平天国运动的发起者和领袖。}、康有为\footnote{〔康有为〕清末政治家、思想家。提倡君主立宪。1898年策动戊戌变法失败后逃亡海外,成为保皇派,反对辛亥革命。}、严复\footnote{〔严复〕清末翻译家、思想家。主张引进西学,改革教育,开启民智,提倡民主科学。}和孙中山\footnote{〔孙中山〕清末革命家、政治家、思想家。多次发起推翻清王朝的革命,最终取得成功,建立中华民国。},代表了在中国共产党出世以前向西方寻找真理的一派人物。那时,求进步的中国人,只要是西方的新道理,什么书也看。向日本、英国、美国、法国、德国派遣留学生之多,达到了惊人的程度。国内废科举,兴学校,好像雨后春笋,努力学习西方。我自己在青年时期,学的也是这些东西。这些是西方资产阶级民主主义的文化,即所谓新学,包括那时的社会学说和自然科学,和中国封建主义的文化即所谓旧学是对立的。学了这些新学的人们,在很长的时期内产生了一种信心,认为这些很可以救中国,除了旧学派,新学派自己表示怀疑的很少。要救国,只有维新,要维新,只有学外国。那时的外国只有西方资本主义国家是进步的,它们成功地建设了资产阶级的现代国家。日本人向西方学习有成效,中国人也想向日本人学。在那时的中国人看来,俄国是落后的,很少人想学俄国。这就是十九世纪四十年代至二十世纪初期中国人学习外国的情形。
    
    帝国主义\footnote{〔帝国主义〕资本主义社会形式,依靠资本输出剥削其他国家的劳动者,进行统治。}的侵略打破了中国人学西方的迷梦。很奇怪,为什么先生老是侵略学生呢?中国人向西方学得很不少,但是行不通,理想总是不能实现。多次奋斗,包括辛亥革命\footnote{〔辛亥革命〕1911年中国爆发的推翻清王朝统治的武装革命。}那样全国规模的运动,都失败了。国家的情况一天一天坏,环境迫使人们活不下去。怀疑产生了,增长了,发展了。第一次世界大战震动了全世界。俄国人举行了十月革命,创立了世界上第一个社会主义国家。过去蕴藏在地下为外国人所看不见的伟大的俄国无产阶级和劳动人民的革命精力,在列宁、斯大林领导之下,像火山一样突然爆发出来了,中国人和全人类对俄国人都另眼相看了。这时,也只是在这时,中国人从思想到生活,才出现了一个崭新的时期。中国人找到了马克思列宁主义这个放之四海而皆准的普遍真理,中国的面目就起了变化了。
    
    中国人找到马克思主义,是经过俄国人介绍的。在十月革命以前,中国人不但不知道列宁、斯大林,也不知道马克思、恩格斯。十月革命一声炮响,给我们送来了马克思列宁主义。十月革命帮助了全世界的也帮助了中国的先进分子,用无产阶级的宇宙观作为观察国家命运的工具,重新考虑自己的问题。走俄国人的路——这就是结论。1919年,中国发生了五四运动。1921年,中国共产党成立。孙中山在绝望里,遇到了十月革命和中国共产党。孙中山欢迎十月革命,欢迎俄国人对中国人的帮助,欢迎中国共产党同他合作。孙中山死了,蒋介石\footnote{〔蒋介石〕蒋中正,世称蒋介石,清末政治家,南京国民政府主席,中国国民党总裁,实行独裁统治,依靠美帝国主义,残酷镇压中国共产党与民主进步人士,国共内战失败后退守台湾。}起来。在二十二年的长时间内,蒋介石把中国拖到了绝境。在这个时期中,以苏联为主力军的反法西斯\footnote{〔法西斯〕20世纪诞生的一种政治思想,标志为一捆棍棒中间插着一把斧头,象征团结一致服从一个意志、一个权力。法西斯主义主张独裁集权下的集体主义、民族主义,用军事武力维护民族资产阶级利益。}的第二次世界大战,打倒了三个帝国主义大国,两个帝国主义大国在战争中被削弱了,世界上只剩下一个帝国主义大国即美国没有受损失。而美国的国内危机是很深重的。它要奴役全世界,它用武器帮助蒋介石杀戮了几百万中国人。中国人民在中国共产党领导之下,在驱逐日本帝国主义之后,进行了三年的人民解放战争,取得了基本的胜利。
    
    就是这样,西方资产阶级的文明,资产阶级的民主主义,资产阶级共和国的方案,在中国人民的心目中,一齐破了产。资产阶级的民主主义让位给工人阶级领导的人民民主主义,资产阶级共和国让位给人民共和国。这样就造成了一种可能性:经过人民共和国到达社会主义和共产主义,到达阶级的消灭和世界的大同。康有为写了《大同书》,他没有也不可能找到一条到达大同的路。资产阶级的共和国,外国有过的,中国不能有,因为中国是受帝国主义压迫的国家。唯一的路是经过工人阶级领导的人民共和国。
    
\end{normalsize}


\newpage

\textbf{注释}:

\vspace{-1em}

\begin{itemize}
    \setlength\itemsep{-0.2em}
    \item 〔放之四海而皆准〕无论放在什么地方都不会错。
    \item 〔雨后春笋〕比喻新生事物大量快速涌现。
    \item 〔阶级〕因社会地位和对生产资料关系不同形成的利益集团。
    \item 〔杀戮〕大量杀害。
    \item 〔宇宙观〕即世界观,对世界、社会的根本看法。
    \item 〔大同〕出自《礼记》,是儒家学说对理想社会的描述,指人人友爱互助,家家安居乐业,没有差异,没有战争的社会。
    \item 〔破产〕企业依法宣布无力偿还债务,由司法部门接收其财产抵债。比喻失败、破灭。
\end{itemize}

\chapter{《农村调查》序言}

\begin{normalsize}
    
    现在党的农村政策,不是十年内战时期\footnote{〔十年内战时期〕指1927年8月南昌起义至1937年9月国共合作为止的十年。}那样的土地革命政策,而是抗日民族统一战线的政策。全党应该执1940年7月7日和12月25日的中央指示,应该执行即将到来的七次大会的指示。所以印这个材料,是为了帮助同志们找一个研究问题的方法。现在我们很多同志,还保存着一种粗枝大叶、不求甚解的作风,甚至全然不了解下情,却在那里担负指导工作,这是异常危险的现象。对于中国各个社会阶级的实际情况,没有真正具体的了解,真正好的领导是不会有的。
    
    要了解情况,唯一的方法是向社会作调查,调查社会各阶级的生动\footnote{〔生动〕这里指实际的鲜活的(情况)。}情况。对于担负指导工作的人来说,有计划地抓住几个城市、几个乡村,用马克思主义的基本观点,即阶级分析的方法,作几次周密的调查,乃是了解情况的最基本的方法。只有这样,才能使我们具有对中国社会问题的最基础的知识。
    
    要做这件事,第一是眼睛向下,不要只是昂首望天。没有眼睛向下的兴趣和决心,是一辈子也不会真正懂得中国的事情的。
    
    第二是开调查会。东张西望,道听途说,决然得不到什么完全的知识。我用开调查会的方法得来的材料,湖南的几个,井冈山的几个,都失掉了。这里印的,主要的是一个《兴国调查》\footnote{〔《兴国调查》〕1930年10月毛泽东对江西省赣州市兴国县做的社会调查。},一个《长冈乡调查》\footnote{〔《长冈乡调查》〕1933年11月毛泽东对兴国县长冈乡做的社会调查。}和一个《才溪乡调查》\footnote{〔《才溪乡调查》〕1933年11月毛泽东对福建省龙岩市上杭县才溪乡做的社会调查。}。开调查会,是最简单易行又最忠实可靠的方法,我用这个方法得了很大的益处,这是比较什么大学还要高明的学校。到会的人,应是真正有经验的中级和下级的干部,或老百姓。我在湖南五县调查\footnote{〔湖南五县调查〕1927年1月毛泽东对湖南湘潭、湘乡、衡山、醴陵、长沙五县做的社会调查。}和井冈山两县调查\footnote{〔井冈山两县调查〕1927年底、1928年初毛泽东对江西、湖南省交界的永新、宁冈二县做的社会调查。},找的是各县中级负责干部;寻乌调查\footnote{〔寻乌调查〕1930年5月毛泽东对闽粤赣三省交界的寻乌县做的社会调查。}找的是一部分中级干部,一部分下级干部,一个穷秀才,一个破产了的商会会长,一个在知县衙门管钱粮的已经失了业的小官吏。他们都给了我很多闻所未闻的知识。使我第一次懂得中国监狱全部腐败情形的,是在湖南衡山县作调查时该县的一个小狱吏。兴国调查和长冈、才溪两乡调查,找的是乡级工作同志和普通农民。这些干部、农民、秀才、狱吏、商人和钱粮师爷,就是我的可敬爱的先生,我给他们当学生是必须恭谨勤劳和采取同志态度的,否则他们就不理我,知而不言,言而不尽。开调查会每次人不必多,三五个七八个人即够。必须给予时间,必须有调查纲目,还必须自己口问手写,并同到会人展开讨论。因此,没有满腔的热忱,没有眼睛向下的决心,没有求知的渴望,没有放下臭架子、甘当小学生的精神,是一定不能做,也一定做不好的。必须明白:群众是真正的英雄,而我们自己则往往是幼稚可笑的,不了解这一点,就不能得到起码的知识。
    
    我再度申明:出版这个参考材料的主要目的,在于指出一个如何了解下层情况的方法,而不是要同志们去记那些具体材料及其结论。一般地说,中国幼稚的资产阶级还没有来得及也永远不可能替我们预备关于社会情况的较完备的甚至起码的材料,如同欧美日本的资产阶级那样,所以我们自己非做搜集材料的工作不可。特殊地说,实际工作者须随时去了解变化着的情况,这是任何国家的共产党也不能依靠别人预备的。所以,一切实际工作者必须向下作调查。对于只懂得理论不懂得实际情况的人,这种调查工作尤有必要,否则他们就不能将理论和实际相联系。“没有调查就没有发言权”,这句话,虽然曾经被人讥为“狭隘经验论”的,我却至今不悔\footnote{〔“没有调查就没有发言权”……〕“没有调查就没有发言权”出自1930年5月毛泽东写的《反对本本主义》,是毛泽东在井冈山时期就持有的观点,被王明等教条主义者认为是“狭隘经验主义”。};不但不悔,我仍然坚持没有调查是不可能有发言权的。有许多人,“下车伊始\footnote{〔下车伊始〕新官刚到任所下车。这里讽刺带有教条主义的新上任干部。}”,就哇喇哇喇地发议论,提意见,这也批评,那也指责,其实这种人十个有十个要失败。因为这种议论或批评,没有经过周密调查,不过是无知妄说。我们党吃所谓“钦差大臣”的亏,是不可胜数的。而这种“钦差大臣”则是满天飞,几乎到处都有。斯大林的话说得对:“理论若不和革命实践联系起来,就会变成无对象的理论。”当然又是他的话对:“实践若不以革命理论为指南,就会变成盲目的实践。”除了盲目的、无前途的、无远见的实际家,是不能叫做“狭隘经验论”的。
    
    我现在还痛感有周密研究中国事情和国际事情的必要,这是和我自己对于中国事情和国际事情依然还只是一知半解这种事实相关联的,并非说我是什么都懂得了,只是人家不懂得。和全党同志共同一起向群众学习,继续当一个小学生,这就是我的志愿。
    
\end{normalsize}


\newpage

\textbf{注释}:

\vspace{-1em}

\begin{itemize}
    \setlength\itemsep{-0.2em}
    \item 〔粗枝大叶〕粗鲁,不细心,容易疏漏。
    \item 〔师爷〕旧时州、县官雇请的幕僚。
    \item 〔阶级〕因社会地位和对生产资料关系不同形成的利益集团。
    \item 〔热忱〕真诚的热情。忱:真挚的情意。
    \item 〔高明〕见解独到不凡,技艺高超。
    \item 〔道听途说〕路上听来的消息,指没有根据的传闻。
    \item 〔闻所未闻〕从来没听说过。
    \item 〔不可胜数〕数量极多,数不过来。
    \item 〔痛感〕深深地感觉到、体会到。
    \item 〔周密〕周到严密。
\end{itemize}

\chapter{竞选州长}

\begin{normalsize}
    
    几个月以前,我以独立党人\footnote{〔独立党人〕作者虚构的一个政党。19世纪中叶起,美国主要政党为两大党:民主党和共和党。}的身份,被提名为纽约州\footnote{〔纽约州〕美国州名,位于美国东部沿海地区。}的州长候选人,与斯图尔特·林登·伍德福德先生和约翰·霍夫曼先生一同竞选\footnote{〔几个月以前……一同竞选〕这里指历史上1868年的纽约州州长竞选。提到的名字都是当时真实的候选人。}。相比这两位先生,我觉得我有那么一点优势,那就是——声望还好。从报纸上很容易看出,他们早已声名不保。近几年来,他们显然对各式各样可耻的罪行都习以为常了。不过,就在我因自己的优势心生欢喜,暗自得意的时候,我的心底却有一股阴暗的浊流在暗涌。那就是——我不得不听到自己的名字动辄被人拿来与那两位相提并论,到处传播了。我心里越来越烦乱,就写信给我的祖母,报告这桩事情。她的信回得又快又干脆。她说:
    
    \begin{quotation}
    
    你生平从来没有干过一桩可惭愧的事——从来没有。你看看报纸吧——你得明白,伍德福德和霍夫曼这两位先生是什么人。然后想一想你是否情愿把自己降到他们的水平,和他们公开竞选。
    
    \end{quotation}
    
    我也正是这么想的呀!那天晚上我片刻也没有睡着。可是事已至此,我究竟无法撒手了。
    
    我已经完全卷入了漩涡,不得不继续这场斗争。早餐时,我漫不经心地打开一份报纸,忽然发现以下这么一段。老实说,我从没有那么吃惊过。
    
    \begin{quotation}
    
    伪证犯罪——马克·吐温先生既为州长候选人,不知可否屈尊稍作说明,他1863年时为何在交趾支那\footnote{〔交趾支那〕中南半岛的历史地名,位于今日越南南部,是法国殖民地。}的瓦卡瓦克\footnote{〔瓦卡瓦克〕仿照北美原住民文化虚构的地名。},被三十四个证人证明犯下伪证罪?该次伪证事涉当地一块大蕉园。该园本为当地一位寡妇所有,谁知吐温先生竟作伪证,意欲将其夺走。其园甚为贫瘠,更为孤儿寡母惟一依靠,吐温先生竟然忍心强夺。若不趁此参选良机厘清事实,吐温先生当如何自处,又当如何面对广大选民呢?
    
    \end{quotation}
    
    我简直被惊呆了,这样冷血无良的诬蔑!我一辈子连见也没有见过交趾支那!瓦卡瓦克我连听也没有听说过!什么大蕉园的,我不知道它和牛肉圆有什么区别!我都不知道该做什么了。我被这新闻弄得焦躁不安,不知所措。那一天便浑浑噩噩地过去了,我根本没有采取任何应对措施。第二天早上,同一份报纸上就登着这么光秃秃的一句:
    
    \begin{quotation}
    
    耐人寻味——各位想必已经注意到了,吐温先生对交趾支那的伪证案保持缄默,似有隐衷。
    
    \end{quotation}
    
    (附注——从此以后,在竞选期间,这家报纸只要提到我,就称呼我为“无耻的伪证犯吐温”。)
    
    其次是《新闻报》,上面登着这么一段:
    
    \begin{quotation}
    
    敬请说明——新公布的州长竞选人可否向选民略微透露他在蒙大拿\footnote{〔蒙大拿〕美国州名,位于美国西北部。}的往事,以释众疑?吐温先生在蒙大拿时,同住工友不时遗失贵重物品,而后又总能从他身上或他的“行囊”(报纸卷成的包裹)里寻回,以至于各位工友不得不善加劝告,赐他膏冠\footnote{〔膏冠〕当时的一种私刑。把犯人全身抹上柏油,沾上羽毛,让他骑在锐边的木杆上(模仿骑马的印第安人),抬着游街示众。},送他上路,让他莫要回头。个中缘由,是否能说明之?
    
    \end{quotation}
    
    世间还能有比这更居心险恶的事吗?我这一辈子从没去过蒙大拿。
    
    (从此以后,这个报纸就照例把我叫做“蒙大拿的小偷吐温”。)
    
    于是我渐渐对报纸有了戒心,一拿起来就觉得提心吊胆——很像一个人困了想揭开床毯,却又怕底下会有一条蛇似的。有一天,我又看到这么一段:
    
    \begin{quotation}
    
    撒谎被抓——经五角地\footnote{〔五角地〕五角地和下文中的海街都是纽约市曼哈顿区的地点。五角地是龙蛇混杂、贫民密集、犯罪率高发的街区。海街是临港口码头的大街,以进出口、工商业活动为主,人口流动频繁。}的迈克尔·欧弗兰纳根律师和海街的基特·伯恩斯先生、约翰·艾伦先生三人宣誓作证,查实马克·吐温先生污蔑我党德高望重的领袖约翰·霍夫曼已故的祖父,说他因犯强盗罪被处绞刑。这谎言毫无根据,纯属臆造,委实卑鄙下流。此等玷污死者清誉的无耻手段,竟被人用以博得政治上的成功,实在令正人君子寒心。想到这等恶毒言论为死者亲友所带来哀痛之巨大,我们感同身受,几欲挺身呼唤,号召各位义愤填膺之士,采取断然之行动,对诽谤者施行非法的报复。当然,此非君子所为!且让他遭受自己良心的谴责而苦痛吧。(但倘有义愤难平者,理智为热血所蔽,竟下手株除之,想必到了公堂之上,陪审团员也定会通融谅解,不致定罪。)
    
    \end{quotation}
    
    末尾那句匠心独运的话确实起了作用。当天夜里就有一群“仗义之士”敲破我家前门,吓得我从床上跳起来,由后门逃出去。那些人满腔义愤,来势汹汹,一进门就捣毁了家具和窗户,走的时候把能带走的财物都拿去了。但是我可以手按《圣经》发誓\footnote{〔手按《圣经》发誓〕《圣经》指基督教的《旧约》和《新约》。当时美国人以一手按《圣经》、一手举起向耶和华发誓作为最严肃的宣誓方法。},我从来没有诽谤过霍夫曼先生的祖父。不仅如此,看到这篇文章前,我从没听说过,更不必说提起过他。
    
    (顺便说一声,从那以后,那个报纸就始终把我称为“侮尸犯吐温”。)
    
    紧接着引起我注意的下一条新闻是这样说的:
    
    \begin{quotation}
    
    醉心之选——马克·吐温先生原定于昨晚在独立党人的集会上作一次中伤他人的演说,但是他未能按时出席!据他的医生所说,他被失控的马群撞倒了,腿脚两处受伤,此刻“僵卧在床,痛苦难当”。诸如此般胡诌甚多,恕不一一呈现。独立党人极力迎合此等无耻托词,妄图掩盖他们拥戴之人缺席的真正原因。昨晚上有人分明看见一个烂醉如泥的身影,跌跌撞撞,晃进吐温先生住的旅馆。独立党人这次又要如何证明,这位无可救药的酗酒之徒,并非马克·吐温本人?任你独立党人滑如泥鳅,这次避无可避了!奉劝独立党人莫要继续逃避。人民的呼声如响雷。我们只要一个答案:“那个人究竟是谁?”
    
    \end{quotation}
    
    不可思议,实在不可思议,我用了好些时间,才敢确认,他们真的把我的名字和这个可耻的罪名放到一起了。我已经整整三年没有碰过酒了,无论是麦酒、啤酒、葡萄酒或是任何一种酒。
    
    (从下一期开始,这份杂志就信心十足地给我安上了“酒疯子吐温先生”的诨名,直到一切结束。而我对此竟没有太多感觉。看来时间确实是一剂良药。我对这套把戏也麻木了。)
    
    到了这个地步,匿名信已经成了家常便饭。常见的内容是这样的:
    
    \begin{quotation}
    
    在你家门口讨饭时被你一脚踢开的那位老太现在还好吗?包打听启\footnote{〔包打听启〕“包打听”,和下文中的“常伸手”都是当时常用假名。“启”为写信人名字后的格式。}
    
    \end{quotation}
    
    还有这样的:
    
    \begin{quotation}
    
    你干的那些破事,别人不知道,我可是一清二楚。你最好把我打点舒服了,不然全给你登到报上去。常伸手启
    
    \end{quotation}
    
    情况大致便是如此。如果需要的话,我还可以举出更多例子,直到各位读者腻烦为止。
    
    不久,共和党的主要报纸又给我“判了罪”——大规模的贿选行为;而民主党的权威报纸则将一桩大肆渲染的讹诈案硬栽到我头上。
    
    (就是这样,我又获得了两个称号:“肮脏的作弊者吐温”和“可恶的诈骗犯吐温”。)
    
    这时候舆论已经沸腾了,要求我“回应”那一切可怕的控诉的呼声越来越高,以致我们党的领袖和报社主编都说,我要是再不说点什么,我的政治生涯就快完了。好像是嫌这些控诉逼迫得还不够紧似的,就在第二天,有一家报纸上又登出了下面这么一段:
    
    \begin{quotation}
    
    瞧这个人——独立党的候选人仍旧选择沉默,根本不敢回应。对他的指控莫不证据充足,他一直以来的沉默更是给这些罪状反复背书——至今连一句狡辩也没有。他再也无力翻供了。独立党人,睁开眼睛看看你们的候选人吧!好好看看这个声名狼藉的伪证犯!蒙大拿的小偷!侮尸犯!酒精中毒的化身!肮脏的作弊者!可恶的诈骗犯!睁开眼睛吧,好好看清他的真面目,然后再作决定:如此败类,犯下滔天恶行,被人历数其罪却不敢分辩,你们应不应该把至关重要的一票投给他!
    
    \end{quotation}
    
    要想逃避是不可能的。所以,深感羞辱之余,我也着手“回应”那一大堆无稽的指控,那些恶毒卑鄙的谣言。可我终究没完成这个工作,因为就在第二天早上,又有报纸登出新的恐怖故事,再度恶意中伤,严厉控诉我烧毁了一间疯人院,烧死了里头的所有病人,仅仅因为它碍了我家的风景。这可使我陷入了恐慌的境地。紧接着又来了一个指控,说我为了谋夺我叔父的财产,把他毒死了,并紧急要求挖坟验尸。这把我吓得几乎要发疯。这一切还不够,他们又给我加了一个罪名,说我在孤儿院当院长的时候,雇的厨子全是我的老掉牙的亲戚,做的饭菜根本不能入口。我开始动摇了——真的动摇了。最后,党争的仇恨加在我身上的无耻迫害终于很自然地迎来了高潮:一个公开集会上,九个刚学走路的小孩子,各种肤色样貌都有,被教唆着闯到讲台上来,抱住我的腿,叫我爸爸!
    
    我放弃了。我偃旗息鼓,甘拜下风。我够不上纽约州州长竞选所需要的条件,于是我发表了退出竞选的声明,只把我最后一丝怨气放进声明的署名里:
    
    \begin{quotation}
    
    “你忠实的朋友——从前是个正派人,现在成了伪证犯、小偷、侮尸犯、酒疯子、作弊者和诈骗犯的马克·吐温。”
    
    \end{quotation}
    
\end{normalsize}


\newpage

\textbf{注释}:

\vspace{-1em}

\begin{itemize}
    \setlength\itemsep{-0.2em}
    \item 〔动辄〕动不动就。
    \item 〔缄默〕沉默,不说话。缄:捆东西的绳索;封住、闭上。
    \item 〔隐衷〕不愿意对人说起的苦衷。
    \item 〔漩涡〕水流遇到阻碍所激成的螺旋形回流。
    \item 〔漫不经心〕不集中注意力,不专心地。
    \item 〔义愤填膺〕被正义感激发的愤怒充满心胸。膺:胸。
    \item 〔厘清〕完全整理清楚。
    \item 〔戒心〕警惕戒备的心思。
    \item 〔玷污〕污损了本来美好的事物。玷:白玉上的斑点,比喻人的过失。
    \item 〔屈尊〕降低身份(做某事)。
    \item 〔匠心独运〕运用了独特巧妙的心思。
    \item 〔中伤〕恶意造谣损害他人名誉。
    \item 〔清誉〕清白的名誉,美好的名声。
    \item 〔托词〕借口,找借口。
    \item 〔胡诌〕随口胡说。
    \item 〔家常便饭〕平时家里常见的饭食。比喻经常发生的事,已经不稀奇了。
    \item 〔打点〕送人钱财以求关照。
    \item 〔偃旗息鼓〕放倒旗子,停止击鼓。表示停止战斗,比喻停止做事或声势减弱。偃:仰卧,指平放。
    \item 〔甘拜下风〕甘心认输,自愿居于下位。
    \item 〔无稽〕没有根据,无从查证。稽:考核,核查。
    \item 〔背书〕在票据背面签名或盖章。比喻支持某人某事或某言论,为其信誉负责、担保真实。
    \item 〔大肆渲染〕渲染:把水、墨淋在纸上再擦匀的画法,引申为夸大地形容。比喻为了达到目的肆意夸大事实,加以宣传。
    \item 〔高潮〕涨潮时达到的最高水位。比喻事情发展到最紧张激烈之处。
    \item 〔声名狼藉〕形容名声极坏。狼藉:狼群把草地弄乱以掩饰踪迹,形容杂乱不堪。
    \item 〔历数〕列举。
\end{itemize}

\chapter{我的叔叔于勒}

\begin{normalsize}
    
    一个穷老头儿,胡子花白,向我们乞求施舍。我的同伴约瑟夫·达夫朗什竟给了他五法郎\footnote{〔法郎〕法国货币单位,相当于“元”。一法郎等于一百分钱。}。我很吃惊。他对我说:
    
    这个可怜人使我想起一段往事。这段往事我一直忘不了,下面我就来讲给您听。
    
    我家原籍勒阿弗尔\footnote{〔勒阿弗尔〕法国西北部诺曼底大区滨海塞纳省的海港城市。},并不富裕,也就是勉强度日罢了。我的父亲在外做事,很晚才从办公室回来,挣的钱不多。我有两个姐姐。
    
    拮据的生活让我母亲非常痛苦,总找些刻薄话,含蓄又恶毒的责备,发泄在她丈夫身上。这可怜人总会用一个叫我看了难过的手势回应:张开手抹一抹额头,好像要抹去根本不存在的汗珠;从不答话。
    
    我能体会他那种无力的痛苦。那时家里样样都要节省;有人请吃饭是从来不敢答应的,以免回请;日用品也总是买减价货,压底货。姐姐们自己做衣服,买十五分钱一米的花边还常常要杀半天价\footnote{〔杀价〕讲价,砍价。买卖时商议更优惠的价格。}。我们日常吃的是菜汤和蘸各种酱汁的牛肉。据说这又卫生又富于营养,不过我还是希望换换口味。
    
    我要是弄丢了纽扣或撕破了裤子,就得挨一顿狠骂。
    
    不过,每逢星期日,我们都要衣冠整齐地到防波堤上去散步。父亲穿着礼服,戴着礼帽,套着手套,让我的母亲挽着胳膊;母亲打扮得好像节日悬万国旗的海船\footnote{〔万国旗〕国际船舶上连成一排挂着的旗帜,由40个国家的国旗组成,用于传递信号。这里形容衣服颜色花哨。}。姐姐们总是最先打扮整齐,只等着出发的命令;可是到了最后一刻,总会在一家之主的礼服上发现一块忘记擦掉的污迹,于是赶快用旧布蘸了汽油来把它擦掉。
    
    父亲头顶着大礼帽,只穿着衬衫,等着这道手续做完;母亲则手忙脚乱,架上近视眼镜,又脱下手套免得弄脏。
    
    于是全家隆重上路。姐姐们挽着胳膊走在最前面。她们已经到了出嫁的年龄,所以得叫城里人好好看看。我依在母亲左边,父亲在她右侧。我现在还记得我可怜的双亲在星期日散步时候那僵硬的表情、庄严的举止、浮夸的神气。他们挺直了腰,伸直了腿,迈着沉着的步伐向前走着,就仿佛他们的态度和表现关系着一桩极端重要的大事。
    
    每个星期日,只要一看见从不知哪个遥远地方回来的大海船开进港口,我的父亲总要说他那句从不变更的话:“唉!如果于勒就在这条船上,那会多么叫人惊喜呀!”
    
    我父亲的弟弟,于勒叔叔,是全家惟一的希望,而在这以前,曾经是全家的祸害。我从小就听家里人谈论这位叔叔,我对他已是那样熟悉,大概一见面就能立刻认出他来。他动身到美洲去以前的生活,连细枝末节我都完全知道,虽然家里人谈起他这一段生活总是压低了声音。
    
    据说他当初行为很不端正,花起钱来大手大脚,这在穷人的家庭里是罪大恶极了。在有钱人的家里,一个人吃喝玩乐无非算是糊涂荒唐。大家笑嘻嘻地称呼他一声花花公子。在生活困难的家庭里,一个人要是逼得父母动老本儿,那他就是一个坏蛋,一个流氓,一个无赖了。
    
    虽然事情是一样的事情,这样区别开来还是对的,因为行为的好坏,只有结果能够决定。
    
    总之,于勒叔叔把自己应得的那部分遗产吃得一干二净之后,还大大减少了我父亲指望的那一部分。
    
    按照当时的惯例,他被送上一条从勒阿弗尔开往纽约的商船,到美洲去了。
    
    一到了那里,我这位于勒叔叔就做上了不知什么买卖,不久就写信来说他赚了点钱,并且希望能够赔偿我父亲的损失。这封信在我的家庭里引起了极大的震动。于勒,大家都认为分文不值的于勒,一下子成了正直的人,有良心的人,达夫朗什家的好男儿,跟所有达夫朗什家的子弟一样正直可亲了。
    
    有一位船长又告诉我们,说他已租了一间大店铺,做着一桩很大的买卖。
    
    两年后又接到第二封信,信上说:
    
    “亲爱的菲利普,我给你写这封信,免得你担心我的健康。我身体很好,买卖也好。明天我要动身到南美去。这一去大概很久,也许好几年没法给你写信。如果接下来几年没收到我的信,你也不必担心。我发了财就会回勒阿弗尔的。我希望为期不远,那时我们就可以一起快活地过日子了……”
    
    这封信成了我们家里的福音书\footnote{〔福音书〕耶稣门徒传教的言行记录成的书。福音:好消息,指耶稣宣称“信者即可得救上天堂”。这里比喻带来极大希望的东西。}。一有机会就要拿出来念,见人就拿出来给他看。
    
    果然,十年之内,于勒叔叔再没有来过信,可父亲的希望却与日俱增。母亲也常常这样说:
    
    “只要我们的好于勒回来,一切境况就不同了。幸好我们家还有个有办法的人!”
    
    于是每个星期日,看见大轮船向天空喷着蜿蜒的黑烟,从天边驶过来的时候,我父亲总是重复他那句永不变更的话:
    
    “唉!如果于勒就在这条船上,那会多么叫人惊喜呀!”
    
    仿佛下一刻就能看见他,挥着手帕喊道:“喂!菲利普!”
    
    叔叔回国这桩事十拿九稳,大家以此拟定了上千种计划,甚至于计划到用叔叔的钱在安古维尔\footnote{〔安古维尔〕勒阿弗尔东北的海边小镇,距离约70公里。}附近购置一栋别墅。我怀疑父亲是不是已经找地产中介商谈过了。
    
    我大姐那时二十八岁,二姐二十六岁。她们还没有结婚。全家都为这事发愁。
    
    后来终于有人上门了。他是一个公务员,看上了二姐。他没什么钱,但是诚实可靠。我总认为这个年轻人下决心求婚,不再迟疑,完全是某天晚上我们给他看了于勒叔叔的信的缘故。
    
    我们家赶忙答应了他的请求,并且决定婚礼之后全家到泽西岛\footnote{〔泽西岛〕法国诺曼底大区芒什省西部岛屿。}旅游。
    
    泽西岛是穷人旅游的理想去处。路并不远;乘小轮船渡过海,便到了外国的土地上,因为这个小岛是属于英国的。因此,一个法国人只要航行两个钟头,就有幸一睹异国风光,研究一下这个一面国旗就能覆盖的邻国小岛上糟糕卑劣的风俗人情——就如那些说话简洁的人总结的那样。
    
    泽西岛的旅行成了我们朝思暮想、时刻盼望着的一件事了。
    
    我们终于动身了。我现在想起来还像是昨天刚发生的事:轮船靠着格朗维尔码头\footnote{〔格朗维尔码头〕法国诺曼底大区芒什省阿夫朗什市西部的码头,正对着泽西岛。}生火待发;我的父亲慌慌张张地监视着我们的三个包袱搬上船;我的母亲不放心地挽着我那未嫁姐姐的胳膊。自从二姐出嫁后,我的大姐就像鸡窝里落下的小鸡一样,失魂落魄的;我们后边是那对新婚夫妇,他们总落在后面,使我常常要回过头去看看。
    
    汽笛响了。我们上了船,轮船离开防波堤,驶入平静得好似一块绿色大理石桌面的大海。我们看着海岸向后退去,感到快活而骄傲——不常旅行的人都这样。
    
    父亲高高挺着藏在礼服里面的肚子。这件礼服,家里人在当天早上仔细地擦掉了所有的污迹。此刻他四周散布着出门的日子里必有的汽油味;我一闻到这股气味,就知道星期日到了。
    
    父亲忽然发现了什么。两位先生在请两位优雅入时的太太吃牡蛎\footnote{〔牡蛎〕一种食用贝类,生长在海岸礁石上。当时牡蛎是比较富有的人的食物。}。一个衣衫褴褛的老水手拿小刀撬开牡蛎,递给两位先生,再由他们传给两位太太。她们的吃法也很雅致,一方精致的手帕托着蛎壳,嘴稍稍前伸,免得弄脏了裙袍;然后微微一动嘴,就把汁水喝了进去,蛎壳就扔在海里。
    
    在行驶的海船上吃牡蛎!这件高雅的事无疑打动了我父亲的心。他认为这是雅致高贵的好派头儿,于是走到母亲和两位姐姐身边问道:
    
    “你们要不要吃点牡蛎?”
    
    母亲有点迟疑不决,她怕花钱;但是两位姐姐马上表示赞成。于是母亲很不情愿地说:
    
    “我怕伤胃,你买给孩子们吃就好了。可别太多,吃多了要生病的。”
    
    然后转过身对着我,她又说:
    
    “至于约瑟夫,他用不着吃了,别把男孩子惯坏了。”
    
    我只好留在母亲身边,心里觉得这种待遇很不公平。我一直望着父亲,看见他郑重地带着两个女儿和女婿向那个衣衫褴褛的老水手走去。
    
    先前的那两位太太已经走开,父亲就教姐姐怎样吃才不至于让汁水洒出来,他甚至要吃一个做做样子给她们看。他刚一试着模仿那两位太太,就立刻把牡蛎的汁水全溅在他的礼服上。我听见母亲嘟囔:
    
    “何苦来!老老实实待一会儿多好!”
    
    突然间,父亲不安起来。他向旁边走了几步,盯着挤在卖牡蛎的身边的女儿女婿看了一会儿;然后猛地朝我们走来。他的脸色十分苍白,眼神也和往常不一样了。他低声对母亲说:
    
    “真奇怪!这个卖牡蛎的怎么这样像于勒!”
    
    母亲有点莫名其妙,就问:
    
    “哪个于勒?”
    
    父亲说:
    
    “就……就是我的弟弟呀……如果我不知道他现在是在美洲,有很好的地位,我真会以为就是他哩。”
    
    母亲也怕起来了,她结结巴巴地说:
    
    “你疯了!既然你知道不是他,为什么这样胡说八道?”
    
    可是我的父亲还是放不下心,他说:
    
    “克拉丽丝,你去看看吧!最好还是你去亲眼看看,把事情弄清楚。”
    
    她站起身来去找她两个女儿。我也端详了一下那个人。他又老又脏,满脸都是皱纹,眼睛始终不离开他手里干的活儿。
    
    母亲回来了。能看出她在哆嗦。她很快地说:
    
    “我看就是他。去跟船长打听一下吧。可要多加小心,别叫这小子又回来缠上咱们!”
    
    父亲赶紧去了,但我跟了上去。我心里感到异常激动。
    
    船长是个大高个儿,瘦瘦的,蓄着长长的颊须,他正在驾驶台上散步,那不可一世的神气,仿佛他指挥的是一艘开往印度的大邮轮。
    
    我的父亲客客气气地和他搭上了话,一面恭维,一面打听与他职业上有关的事情,例如:泽西是否重要?有何出产?人口多少?风俗习惯如何?土地性质如何?等等。
    
    不知道内情的人,还以为他们谈论的至少是美利坚合众国呢。
    
    后来终于谈到我们搭乘的这条船,名叫“速运号”的,接着又谈到船员。最后,我的父亲才有点局促不安地问道:
    
    “您船上有一个卖牡蛎的,看上去倒很有趣。您知道点儿这个人的底细吗?”
    
    船长已经有些不耐烦了,他冷冷地回答:
    
    “他是个法国老流浪汉,去年我在美洲碰到他,就把他带回国。据说他在勒阿弗尔还有亲戚,不过他不愿回去找他们,因为他欠着他们钱。他叫于勒……姓达勒芒什,还是达勒房什来着,反正都差不多吧。听说他在那边一度阔绰过,可是您看他今天,落魄到了什么地步。”
    
    父亲脸色煞白,两眼呆直,哑着嗓子说:
    
    “啊!啊!好……很好……想必也是如此,毫不令人意外……谢谢您,船长。”
    
    他说完就走了,船长困惑不解地望着他走远了。
    
    他回到我母亲身旁,神色是那么张皇,母亲赶紧对他说:
    
    “你先坐下吧!别叫他们看出来。”
    
    他跌坐在长凳上,结结巴巴地说道:
    
    “是他,真是他!”
    
    然后他就问:
    
    “咱们怎么办呢?……”
    
    母亲马上回答:
    
    “得把孩子们领开。约瑟夫既然都知道了,就让他去把他们找回来。千万要留心,别叫咱们女婿起疑心。”
    
    父亲好像吓傻了,低声嘟哝着:
    
    “真是飞来横祸\footnote{〔飞来横祸〕指突然到来,来不及防备的灾祸。}!”
    
    母亲突然暴怒起来,说:
    
    “我早就知道这个贼不会有出息,早晚会再来缠上我们!倒好像一个达夫朗什家的人还能让人抱什么希望似的!”
    
    父亲用手抹了抹额头,正如平常受到妻子责备时那样。
    
    母亲接着又说:
    
    “把钱交给约瑟夫,叫他赶快去把牡蛎钱付清。已经够倒霉的了,要是再被这个讨饭的认出来,在这船上可就有热闹看了。咱们到船那头去,注意别叫那人挨近我们!”
    
    她站了起来,他们在给了我五法郎以后,就走了。
    
    我的两个姐姐等着父亲不来,正在纳闷。我说妈妈有点晕船,随即问那个卖牡蛎的:
    
    “应该付您多少钱,先生?”
    
    我真想喊他:“我的叔叔。”
    
    他回答:
    
    “两个半法郎。”
    
    我把五法郎的银币给了他,他把找头递回给我。
    
    我看了看他的手,那是一只满是皱纹的水手的手;我又看了看他的脸,那是一张饱经沧桑的脸,满面愁容,疲惫不堪。我心里默念道:
    
    “这是我的叔叔,父亲的弟弟,我的亲叔叔。”
    
    我给他半个法郎的小费,他赶紧谢我:
    
    “愿主保佑您,年轻的好先生!”
    
    那腔调是穷人接到施舍时的腔调。我心想,他在那边一定要过饭。
    
    两个姐姐看我这么慷慨,觉得奇怪,仔细地端详着我。
    
    等我把两法郎交给我父亲,母亲诧异起来,问:
    
    “吃了三个法郎?……这不可能。”
    
    我用坚定的口气宣布:
    
    “我给了半个法郎的小费。”
    
    我的母亲吓了一跳,瞪着眼睛盯着我说:
    
    “你简直是疯了!拿半个法郎给这个人,给这个无赖!……”
    
    她没有再往下说,因为我的父亲望望女婿对她使了个眼色。
    
    之后再没人说话。
    
    远远地,一片紫色的阴影从海里钻出来。那就是泽西岛了。
    
    我们回来的时候改乘“圣玛洛号”,以免再遇见他。母亲满腹心事,愁得不得了。
    
    此后我再也没见过我父亲的弟弟!
    
    至今我见了流浪汉,有时还会给五法郎,就是这个缘故。
    
\end{normalsize}


\newpage

\textbf{注释}:

\vspace{-1em}

\begin{itemize}
    \setlength\itemsep{-0.2em}
    \item 〔拮据〕穷困窘迫,没有闲钱。
    \item 〔嘟囔〕不断地、含混地自言自语。多表示不满。
    \item 〔慷慨〕大方,不吝啬。
    \item 〔细枝末节〕比喻细小而无关紧要的事情或问题。
    \item 〔与日俱增〕一天天地增长。
    \item 〔端详〕仔细地看。
    \item 〔潦倒〕惊惶,慌张。
    \item 〔沧桑〕惊惶,慌张。
    \item 〔张皇〕惊惶,慌张。
\end{itemize}

\chapter{葫芦僧判断葫芦案}

\begin{normalsize}
    
    如今且说贾雨村授了应天府\footnote{〔应天府〕明朝行政单位,约等于现在的南京市。这里指应天府知府这个官职。},一到任就有件人命官司详至案下\footnote{〔详至案下〕呈报到公案前。详:下级官吏向上呈报的一种公文,这里作动词,指“呈报”。案:公案,官吏审判用的台子。},却是两家争买一婢,各不相让,以致殴伤人命。彼时雨村即拘原告来审,那原告道:“被打死的乃是小人的主人。因那日买了个丫头,不想系\footnote{〔系〕是。}拐子拐来卖的:这拐子先已得了我家的银子,我家小主人原说第三日方是好日,再接入门;这拐子又悄悄的卖与了薛家,被我们知道了,去找拿卖主,夺取丫头。无奈薛家原系金陵一霸,倚财仗势,众豪奴将我小主人竟打死了。凶身\footnote{〔凶身〕凶手。}主仆已皆逃走,无有踪迹,只剩了几个局外的人。小人告了一年的状,竟无人作主;求太老爷拘拿凶犯,以扶善良,存殁\footnote{〔存殁〕生死。这里指活着的人和死去的人。}感激大恩不尽!”
    
    雨村听了大怒道:“那有这等事!打死人竟白白的走了拿不来的!”便发签\footnote{〔发签〕签发命令。签:古代官府交给差役办理事务的凭证,一般为木制长条,插在公案上的签筒里。}差公人\footnote{〔公人〕官衙的差役,办事人员。}立刻将凶犯家属拿来拷问。只见案旁站着一个门子\footnote{〔门子〕泛指官署中的差役、仆役。},使眼色不叫他发签。雨村心下狐疑\footnote{〔狐疑〕疑虑,怀疑。},只得停了手。退堂至密室,令从人退去,只留这门子一人伏侍\footnote{〔伏侍〕服侍。};门子忙上前请安,笑问:“老爷一向加官进禄,八九年来,就忘了我了?”雨村道:“我看你十分眼熟,但一时总想不起来。”门子笑道:“老爷怎么把出身之地竟忘了!老爷不记得当年葫芦庙里的事么?”
    
    雨村大惊,方想起往事。原来这门子本是葫芦庙里一个小沙弥\footnote{〔沙弥〕佛教用语,指受了一定戒律的年轻和尚。这里泛指佛家寺庙中打杂的年轻人。},因被火\footnote{〔被火〕遭受火灾。}之后,无处安身,想这件生意倒还轻省,耐不得寺院凄凉,遂趁年纪轻,蓄了发,充当门子。雨村那里想得是他?便忙携手笑道:“原来还是故人。”因赏他坐了说话。这门子不敢坐,雨村笑道:“你也算贫贱之交了;此系私室,但坐不妨。”门子才斜签着坐下\footnote{〔斜签着坐下〕侧身坐下。}。
    
    雨村道:“方才何故不令发签?”门子道:“老爷荣任到此,难道就没抄一张本省的‘护官符’来不成?”雨村忙问:“何为‘护官符’?”门子道:“如今凡作地方官的都有一个私单,上面写的是本省最有权势极富贵的大乡绅名姓,各省皆然;倘若不知,一时触犯了这样的人家,不但官爵,只怕连性命也难保呢!——所以叫做‘护官符’。方才所说的这薛家,老爷如何惹得他!他这件官司并无难断之处,从前的官府,都因碍着情分脸面,所以如此。”一面说,一面从顺袋\footnote{〔顺袋〕腰间挂的小袋子,本叫“慎袋”,用来放小件贵重物品,防止丢失。}中取出一张抄的“护官符”来,递与雨村,看时,上面皆是本地大族名宦之家的俗谚口碑\footnote{〔口碑〕口头流传的重要事情。古代立碑记录重大事情,口碑指大众口头流传形成共识,就像碑刻一样。},云:
    
    \begin{verse}[0.5\linewidth]
    
    贾不假,白玉为堂金作马。\\阿房宫\footnote{〔阿房宫〕秦始皇统一六国后建造的皇宫,传说占地极大,富丽堂皇,最后被项羽烧毁。},三百里,住不下金陵一个史。\\东海缺少白玉床,龙王来请金陵王。\\丰年好大“雪”,珍珠如土金如铁。
    
    \end{verse}
    
    雨村尚未看完,忽闻传点\footnote{〔传点〕敲点传递信号。点,一种云头形金属板,又叫“云板”,古代衙门或权贵家宅大门边设点,打点就是向内院通报信息。},报:“王老爷来拜。”雨村忙具\footnote{〔具〕备,穿上(衣服),穿戴上(衣冠)。}衣冠接迎。有顿饭工夫方回来,问这门子,门子道:“四家皆连络有亲,一损俱损,一荣俱荣,今告打死人之薛,就是‘丰年大雪’之‘薛’,——不单靠这三家,他的世交亲友在都在外的本也不少,老爷如今拿谁去?”雨村听说,便笑问门子道:“这样说来,却怎么了结此案?——你大约也深知这凶犯躲的方向了?”
    
    门子笑道:“不瞒老爷说,不但这凶犯躲的方向,并这拐的人我也知道,死鬼买主也深知道,待我细说与老爷听:这个被打死的是一个小乡宦\footnote{〔乡宦〕回乡居住的官僚。}之子,名唤冯渊,父母俱亡,又无兄弟,守着些薄产度日,年纪十八九岁。这也是前生冤孽\footnote{〔冤孽〕佛教指因造恶业导致的报应。}:可巧遇见这丫头,他便一眼看上了,立意买来作妾,设誓不再娶第二个了,所以郑重其事,必得三日后方进门。谁知这拐子又偷卖与薛家,——他意欲卷了两家的银子逃去,谁知又走不脱,两家拿住,打了个半死,都不肯收银,各要领人。那薛公子便喝令下人动手,将冯公子打了个稀烂,抬回去三日竟死了。这薛公子原择下日子要上京的,既打了人,夺了丫头,他便没事人一般,只管带了家眷走他的路,并非为此而逃;这人命些些小事,自有他弟兄奴仆在此料理。——这且别说,老爷可知这被卖的丫头是谁?”雨村道:“我如何晓得?”门子冷笑道:“这人还是老爷的大恩人呢!他就是葫芦庙旁住的甄老爷\footnote{〔甄老爷〕指甄士隐。甄士隐住在葫芦庙旁,女儿是甄英莲。贾雨村上京赶考的费用就是甄士隐资助的。}的女儿,小名英莲的。”雨村骇然道:“原来是他!听见他自五岁被人拐去,怎么如今才卖呢?”
    
    门子道:“这种拐子单拐幼女,养至十二三岁,带至他乡转卖。当日这英莲,我们天天哄他玩耍,极相熟的,所以隔了七八年,虽模样儿出脱\footnote{〔出脱〕出落,指女孩子发育长成少女。}的齐整\footnote{〔齐整〕姿容周正。},然大段\footnote{〔大段〕大的方面,大部分。}未改,所以认得,——且他眉心中原有米粒大的一点胭脂记,从胎里带来的。偏这拐子又租了我的房子居住,那日拐子不在家,我也曾问他,他说是打怕了的,万不敢说,只说拐子是他的亲爹,因无钱还债才卖的。再四哄他,他又哭了,只说:‘我原不记得小时的事!’这无可疑了。那日冯公子相见了,兑了银子,因拐子醉了,英莲自叹说:‘我今日罪孽可满了!’后又听见三日后才过门,他又转有忧愁之态。我又不忍,等拐子出去,又叫内人\footnote{〔内人〕男子称自己的妻子。}去解劝他:‘这冯公子必待好日期来接,可知必不以丫鬟相看。今竟破价买你,后事不言可知。只耐得三两日,何必忧闷?’他听如此说,方略解些;自谓从此得所。——谁料天下竟有不如意事,第二日,他偏又卖与了薛家!若卖与第二家还好,这薛公子的混名,人称他‘呆霸王’,最是天下第一个弄性尚气\footnote{〔弄性尚气〕凭感情做事,爱耍脾气。}的人,而且使钱如土,只打了个落花流水,生拖死拽,把个英莲拖去,如今也不知死活。这冯公子空喜一场,一念未遂\footnote{〔一念未遂〕一个想法都没有实现。},反花了钱,送了命,岂不可叹!”
    
    雨村听了也叹道:“这也是他们的孽障遭遇,亦非偶然,不然这冯渊如何偏只看上了这英莲?这英莲受了拐子这几年折磨,才得了个路头,若果聚合了,倒是件美事;偏又生出这段事来!——且不要议论他人,只目今这官司如何剖断\footnote{〔剖断〕辨明是非,作出判决。}才好?”门子笑道:“老爷当年何其明决,今日何反成个没主意的人了!小的听见老爷补升\footnote{〔补升〕实际官职出现空缺,称为实缺。这时举人可以补上实缺。如果实缺品位高于举人原来的官职,就称为补升。贾雨村之前是四品官,应天府知府是从三品,因此称为补升。}此任,系贾府王府之力;此薛蟠即贾府之亲:老爷何不顺水行舟,做个人情,将此案了结,日后也好去见贾王二公。”雨村道:“你说的何尝不是。但事关人命,蒙皇上隆恩起复\footnote{〔起复〕因故离职或被革职的官员被重新起用。}委用,正竭力图报之时,岂可因私枉法,是实不忍为的。”门子听了冷笑道:“老爷说的自是正理,但如今世上是行不去的!岂不闻古人说的‘大丈夫相时而动’,又说‘趋吉避凶者为君子’,依老爷这话,不但不能报效朝廷,亦且自身不保:还要三思为妥。”
    
    雨村低了头,半日说道:“依你怎么着?”门子道:“小人已想了个很好的主意在此:老爷明日坐堂\footnote{〔坐堂〕(官员)在官署审理案件。},只管虚张声势,动文书\footnote{〔动文书〕发公文。},发签拿人,——凶犯自然是拿不来的,原告固是不依,只用将薛家族人及奴仆人等拿几个来拷问,小的在暗中调停,令他们报个‘暴病身亡’,合族中及地方上共递一张保呈\footnote{〔保呈〕类似保证书的一种呈文。},老爷只说善能扶鸾\footnote{〔扶鸾〕又叫“扶乩”,一种迷信的占卜术,常用来诈骗。}请仙,堂上设了乩坛,令军民人等只管来看,老爷便说:‘乩仙批了,死者冯渊与薛蟠原系夙孽\footnote{〔夙孽〕前世的旧孽。夙:早,旧有的。},今狭路相遇,原因了结。今薛蟠已得了无名之病,被冯渊的魂魄追索而死。其祸皆由拐子而起,除将拐子按法处治外,余不累及……’等语。小人暗中嘱咐拐子,令其实招;众人见乩仙批语与拐子相符,自然不疑了。薛家有的是钱,老爷断一千也可,五百也可,与冯家作烧埋\footnote{〔烧埋〕烧纸钱和埋葬死者。}之费;那冯家也无其要紧的人,不过为的是钱,有了银子,也就无话了。——老爷细想,此计如何?”雨村笑道:“不妥,不妥。等我再斟酌斟酌,压服得口声\footnote{〔口声〕众人的舆论。}才好。”二人计议已定。
    
    至次日坐堂,勾取一干有名人犯\footnote{〔勾取一干有名人犯〕将一批有名字的犯人提到公堂上。一干:一批。},雨村详加审问,果见冯家人口稀少,不过赖此欲得些烧埋之银;薛家仗势倚情\footnote{〔仗势倚情〕倚仗着势力大,情面广。},偏不相让,故致颠倒未决\footnote{〔颠倒未决〕颠来倒去,反反复复,仍然没有定论。}。雨村便徇情\footnote{〔徇情〕为了私人的情面而不讲原则。}枉法,胡乱判断了此案,冯家得了许多烧埋银子,也就无甚话说了。雨村便疾忙修书二封与贾政并京营节度使\footnote{〔京营节度使〕小说中的虚构官职。一般认为是京城亲兵营的最高长官。}王子腾,不过说“令甥之事已完,不必过虑”之言寄去。此事皆由葫芦庙内沙弥新门子所为,雨村又恐他对人说出当日贫贱时事来,因此心中大不乐意;后来到底寻了他一个不是,远远的充发\footnote{〔充发〕充军发配,流放到边疆。}了才罢。
    
\end{normalsize}


\newpage

\textbf{注释}:

\vspace{-1em}

\begin{itemize}
    \setlength\itemsep{-0.2em}
    \item 〔婢〕被役使的女子。古代家族养的女性家仆,也叫丫鬟。
    \item 〔落花流水〕形容晚春花朵败落的场面。比喻场面混乱,一片狼藉,常用于描述被打败后的场景。
    \item 〔孽障〕佛教指妨碍修行的罪恶,也写作“业障”。引申指不肖后代,用来骂晚辈的话。
    \item 〔狐疑〕疑虑,怀疑。
\end{itemize}

\chapter{卧看牵牛织女星}

\begin{normalsize}
    
    \begin{verse}[0.5\textwidth]
    
    银烛秋光冷画屏,轻罗小扇扑流萤。\\天阶夜色凉如水,卧看牵牛织女星。\\——杜牧《秋夕》
    
    \end{verse}
    
    秋天晚上,我们所看到的最亮的星是织女星\footnote{〔织女星〕位于二十八宿的牛宿,是北半球夜空中第三亮的星,仅次于天狼星和大角星。}。初秋时节,她在晚上九点钟左右越过天顶;秋越深,她越过天顶的时间也越早。织女星的东方,白蒙蒙地像云一样的一长片,断断续续从北到南横过天空,这就是银河,也叫做天河。正像我们把北方的七颗星连成个“斗勺”一样,西洋人\footnote{〔西洋人〕指西方古希腊、古罗马和基督教文明。}把织女星和附近的几颗星连在一起,成为一架七弦琴\footnote{〔七弦琴〕古希腊的一种拨弦乐器。}的样子,把它叫做天琴座\footnote{〔天琴座〕现代88星座之一,位于北半球。},说这就是古希腊音乐家俄耳甫斯的七弦琴,用这七弦琴,他弹奏出神妙的曲调,使森林里的野兽都陶醉了。有一幅名画,画的就是这个故事。俄耳甫斯坐在森林里的大石上,弹奏他的七弦琴,几只狮子俯首贴耳地伏在他面前。看了那些狮子凝神静听的表情,我们似乎听到了画上弹奏的七弦琴的曲调。我们如果看过这张画,知道了这个故事,再看那织女星,一定更觉得耐人寻味了。
    
    我们中国关于织女星也有同样美丽的传说,说织女星是天帝的孙女,因此也叫做“天孙”。以这颗星的光辉和美丽柔和来说,确也当得起这个高贵的名号。天帝把他的孙女嫁给了牛郎——一个牧牛童子。他们两个结婚之后你欢我爱的,把应做的工作都抛弃了,一个不再牧牛,一个也不再纺织。这惹得天帝动了怒,命令他们一个住在天河的东岸,一个住在天河的西岸,每年农历七月七日才得以在天河中流相会一次。这故事充满了农人们的幻想。他们把自己的生活反映到天空里,以为在神的世界里,不论哪一个都应该勤劳地工作,要是谁怠惰了,谁就该受惩罚,连天帝的孙女也不能例外。这平等的劳动世界显露出农人们对于人的世界的期望。可是历代的诗人们却大多把这一点忽略了,他们常把这个故事写在诗篇里,来歌咏离愁别苦,着重的只在那一年一度的会面。这当然因为诗人们很少是农人出身的,他们不能体会农人们的思想。
    
    前面抄录的诗也就是这样的一首。这首诗并不是以我们在乘凉时候躺在院子里看星星的那种悠闲舒适的情致来写的。作者所描摹的是宫女\footnote{〔宫女〕皇宫中的婢女。}的心境。她望着光彩明亮的织女星,想到织女每年还能够和牛郎相会一次,而自己却被禁闭在深宫里,永远度着孤寂的时光;她在这“凉如水”的夜色里,所感到的不是凉爽而是凄凉;她“扑流萤”并不是因为萤火有趣,而是因为夜长无聊——都是宫中怨女的心境。
    
    那么牛郎在哪里呢?我们且把那白茫茫的银河当作一条真的河流,我们的眼光渐渐地向东南移,渡过这河流最宽阔的渡口,就遇到排成一条直线的三颗星。中间的一颗很亮,两旁的光芒较弱,看去与中间的一颗距离恰好相等。用直线把这三颗星连起来,正像一条两臂相等的杠杆,因此,阿拉伯人\footnote{〔阿拉伯人〕一般指生活在阿拉伯半岛,说阿拉伯语,信仰伊斯兰教的人。}把这三颗星叫做天平星,我们也把它们叫做挑担星。这中间一颗最大最亮的就是牛郎星,也叫做牵牛星。秋季的夜空中,确实只有这一颗星能够和织女星相配,它的光辉稍稍带点儿黄,不及织女星亮,可是其余的星全比不上它。若是我们一连几晚,每晚上都观察牵牛星和织女星,就可以知道它们的相对位置是不变的,正如故事里所说的一样,一个在天河的东岸,一个在天河的西岸。可是我们也不要太相信这个故事,在农历七月初七的晚上,为了要看两星相会,白白地熬个通夜。天文学家告诉我们,这两颗星永远没有相遇的机会。它们和太阳一样,都是恒星。织女星的光辉是太阳的50倍,牵牛星的光辉是太阳的10倍有余,只因距离我们太远了,所以看去只不过是两颗比较亮的星。天文学家又告诉我们,织女星距离地球26光年,牵牛星距离我们较近,但也有16光年。什么叫做光年呢?光年是天文学上表示距离的单位,表示光一年所走的路程的长短。光的速度是每秒钟30万公里,一天能走259亿2000万公里,这长度的365倍,就是一光年。这种用时间来表示距离的方法,在日常生活中其实也时常用到的。比如从成都到重庆,有450公里,我们步行的话,每天走60公里,因此说从成都到重庆有七天半的路程。太阳光从太阳射到地球上才8分18秒,而织女星的光射到地球上要26年。这样一比较,就会使我们惊异得叫起来:“真了不得!有这样远!”可是这个话给天文学家听见了,一定要笑我们少见多怪。他们会告诉我们,距离太阳几万光年的星不计其数,而10光年以内的星却只有十多个。天文学家还告诉我们一件奇怪的事,说恒星并不是真的不动,织女星以每秒钟14公里的速度移近太阳,牵牛星也在移动,每秒钟接近太阳33公里,这速度和地球的公转速度——每秒钟29公里——差不多。可是天空是如此广阔,因而我们观察不出来。依牵牛星的速度来说,也得9000年才比现在离太阳近一光年。那么在16个9000年之后,它不是要和太阳相撞了吗?这我们用不着担心,一则因为它并不是正对着太阳在移动,二则移动的方向渐渐在改变,说不定在多少年代后,它又离我们渐渐地远了。
    
    那白蒙蒙的银河是什么呢?天文学家告诉我们,这是无数密集的小星,在天空绕成一周。所谓小星,只不过我们看起来小,实际上有很多都比太阳还大。这些星离我们更远。天文学家把银河所围绕成的空间叫做银河系。和地球是太阳系中的一员一样,太阳和别的恒星都不过是银河系中的小星。这银河系像一个扁平的车轮,直径约八万多光年,而且和车轮一样旋转着,因此各个星都以不同的速度绕着中心在移动。太阳并不在这个大车轮的中心,与中心黑洞\footnote{〔黑洞〕某些恒星塌缩后形成太强的引力场,连光都无法逃逸,变成无法直接观测到的天体,因此叫黑洞。天文学家认为银河系中心是一个巨大的黑洞。}的距离大约2.8万光年,所以它和邻近的恒星都以每秒钟约220公里的速度绕着银河系的中心在转动。以这样的速度,也得2.5亿年才转一周。那么银河系之外是什么呢?天文学家说,银河系外面还有很多类似银河系的星系,天文学家把它们叫做河外星系。我们所看到的扁平的漩涡状的星云是河外星系。因此天文学家猜想,在别的星系上看银河系,也是这样一团扁平的漩涡状的星云。
    
    看了这些天文上的时间和空间的数字\footnote{〔天文上的……数字〕一般称为天文数字,也用来指极大的数字。},常会使人想起人生在世,真有“寄蜉蝣于天地,渺沧海之一粟”\footnote{〔“寄蜉蝣于天地,渺沧海之一粟”〕出自苏轼的《赤壁赋》,感叹人相对宇宙的短暂和渺小。蜉蝣:一种非常小的虫子,只能活一天。}的感觉。可是从另一方面想,那悠久的时间和广阔的空间,都不能逃出自然法则的支配,我们人研究各种自然科学,能够发现并利用自然法则,这就是人的高明处。凭这点高明处,我们就不必叹息生命的短暂和渺小了。
    
\end{normalsize}


\newpage

\textbf{注释}:

\vspace{-1em}

\begin{itemize}
    \setlength\itemsep{-0.2em}
    \item 〔俯首贴耳〕形容顺从听话。
    \item 〔凝神〕集中注意力,集中精神。
    \item 〔少见多怪〕见的少懂的少,遇事容易感到奇怪。多用以嘲讽别人没见识,孤陋寡闻。
    \item 〔怠惰〕懒惰,松懈不再认真。
\end{itemize}

\chapter{变色龙}

\begin{normalsize}
    
    警官奥楚米叶洛夫穿着新的军大衣,手里拿着个小包,穿过市集的广场。他身后跟着个红头发的警察,端着一个箩筐,筐里满满的盛着没收来的醋栗。广场上阒无人迹。酒吧和店铺心怀不满地对世界敞开大门,像一张张饥饿的嘴巴。附近连一个乞丐都没有。
    
    “你竟敢咬人,该死的东西!”奥楚米叶洛夫忽然听见一声大喊。“伙计们,别放走它!如今咬人可不行!抓住它!哎哟……哎哟!”
    
    狗的尖叫声响起来。奥楚米叶洛夫往那边一看,瞧见商人彼楚京的木柴场里窜出来一条狗,用三条腿跑路,不住地回头看。它身后追出来一个男人,穿着浆硬\footnote{〔浆硬〕浆洗后发硬的。}的花布衬衫和敞开怀的坎肩。他追上那条狗,身子往前一探,扑倒在地,抓住狗的后腿。紧跟着又传来狗叫声和人喊声:“别放走它!”带着睡意的脸纷纷从店铺里探出来。不久,木柴场门口就聚上了一群人,像是从地底下钻出来的一样。
    
    “好像出乱子了,长官!”警察说。
    
    奥楚米叶洛夫微微侧身,迈步朝人群走去。他看见那个敞开坎肩的人站在木柴场门前,举起右手,伸出一根血淋淋的手指头给人群看,半醉的脸上露出这样的神情:“我要揭你的皮,坏蛋!”那根手指头就像一面胜利的旗帜。奥楚米叶洛夫认得这个人。他是首饰匠赫留金。闹出这场乱子的罪魁祸首是一条白毛小猎狗,尖尖的脸,背上有一块黄斑,这时候坐在人群中央的地上,前腿劈开,浑身发抖。它那含泪的眼睛里流露出苦恼和恐惧。
    
    “这儿出了什么事?”奥楚米叶洛夫挤到人群中去,问道:
    
    “你在这儿干什么?你干吗竖起手指头?……是谁在嚷?”
    
    “我本来走我的路,长官,没招谁没惹谁……”赫留金凑着空拳头咳嗽,开口说。“我正跟米特利·米特利奇谈木柴的事,忽然间,这个坏东西无缘无故把我的手指头咬一口……请您原谅我,我是个干活的人……我干的是细致活。这得赔我一笔钱才成,因为我也许一个星期都不能动这根手指头了……法律上,长官,也没有这么一条,说是人受了畜生的害就该忍着……要是人人都遭狗咬,那还不如别在这个世界上活着的好……”
    
    “嗯……好……”奥楚米叶洛夫严厉地说,咳嗽着,动了动眉毛。“好……这是谁家的狗?这种事我不能放过不管。我要拿点颜色出来,叫那些放狗闯祸的人看看!现在也该管管不愿意遵守法令的老爷们了!等到罚了款,他,这个混蛋,才会明白把狗和别的畜生放出来有什么下场!我要给他点厉害瞧瞧……叶尔迪陵,”警官对警察说,“你去调查清楚这是谁家的狗,打个报告上来!这条狗得打死才成。不许拖延!这多半是条疯狗……我问你们:这是谁家的狗?”
    
    “这条狗像是席加洛夫将军家的!”人群里有个人说。
    
    “席加洛夫将军家的?嗯!……你,叶尔迪陵,把我身上的大衣脱下来……天好热!大概快要下雨了……只是有一件事我不懂:它怎么会咬你的?”奥楚米叶洛夫对赫留金说。
    
    “难道它够得到你的手指头?它身子矮小,可是你,长得这么高大!你这个手指头多半是让小钉子扎破了,后来却异想天开,要人家赔你钱了。你这种人啊……谁都知道是个什么路数!我可懂你们这些恶鬼了!”
    
    “他,长官,他把他的雪茄\footnote{〔雪茄〕用烟叶卷的烟,比普通香烟粗大}戳到它脸上去,拿它开心。它呢,不肯做傻瓜,就咬了他一口……他就是闲的慌,长官!”
    
    “你胡说,独眼龙!你眼睛看不见,为什么胡说?长官是明白人,看得出来谁胡说,谁凭良心说话……我要胡说,就让调解法官\footnote{〔调解法官〕沙俄时代的治安法官,只审理小案子。}审判我好了。他的法律上写得明白……如今大家都平等了……不瞒您说……我弟弟就在当宪兵\footnote{〔宪兵〕维持警察、军队内部纪律的警察。}………”
    
    “少说废话!”
    
    “不,这条狗不是将军家的……”警察皱着眉头说,“将军家里没有这样的狗。他家里的狗大多是大猎狗……”
    
    “你拿得准吗?”
    
    “拿得准,长官。”
    
    “我早就知道。将军家里的狗都名贵,都是良种,这条狗呢,鬼知道是什么东西!毛色不好,模样也不中看……完全是下贱货……他老人家会养这样的狗?!你的脑筋上哪儿去了?要是这样的狗在彼得堡或者莫斯科让人碰上,你们知道会怎样?那儿才不管什么法律不法律,一转眼的工夫就叫它断了气!你,赫留金,受了苦,这件事不能放过不管……得教训他们一下!是时候了……”
    
    “不过也可能是将军家的狗……”警察把他的想法说出来,“它脸上又没写着……前几天我在他家院子里就见过这样一条狗。”
    
    “没错儿,是将军家的!”人群里有人说。
    
    “嗯……你,叶尔迪陵老弟,给我穿上大衣吧……好象起风了……怪冷的……你带着这条狗到将军家里去一趟,在那儿问一下……你就说这条狗是我找着,派你送去的……你说以后不要把它放到街上来。也许它是名贵的狗,要是每个猪猡都拿雪茄烟戳到它脸上去,要不了多久就能把它作践死。狗是娇嫩的动物嘛……你,蠢货,把手放下来!用不着把你那根蠢手指头摆出来!这都怪你自己不好!”
    
    “将军家的厨师来了,我们来问问他吧……喂,普罗霍尔!你过来,亲爱的!你看看这条狗……是你们家的吗?”
    
    “瞎猜!我们那儿从来也没有过这样的狗!”
    
    “那就用不着费工夫去问了,”奥楚米叶洛夫说,“这是条野狗!用不着多说了。既然他说是野狗,那就是野狗。弄死它算了。”
    
    “这条狗不是我们家的,”普罗霍尔继续说,“可这是将军哥哥的狗,他前几天到我们这儿来了。我们的将军不喜欢这种狗。他老人家的哥哥却喜欢……”
    
    “莫非他老人家的哥哥来了?符拉季米尔·伊凡内奇来了?”奥楚米叶洛夫问,他整个脸上洋溢着动情的笑容。“可了不得,主啊!\footnote{〔主啊〕基督教中把神称为“主”。“主啊”为感叹语,类似“天啊”。}我还不知道呢!他要来住一阵吧?”
    
    “住一阵……”
    
    “可了不得,主啊!……他是惦记弟弟了……可我还不知道呢!那么这是他老人家的狗?太荣幸了……你把它带去吧……这条小狗怪不错的……挺伶俐……它把这家伙的手指头咬了一口!哈哈哈哈!……咦,你干吗发抖?呜呜……呜呜……它生气了,小坏包……好个小家伙……”
    
    普罗霍尔把狗叫过来,带着它离开了木柴场。那群人就对着赫留金哈哈大笑。
    
    “我早晚要收拾你!”奥楚米叶洛夫对他威胁了一句,然后把身上的大衣裹一裹紧,穿过市集的广场,径自走了。
    
\end{normalsize}


\newpage

\textbf{注释}:

\vspace{-1em}

\begin{itemize}
    \setlength\itemsep{-0.2em}
    \item 〔阒无人迹〕寂静无人。
    \item 〔坎肩〕不带袖子的上衣,多为棉制或毛皮制,增强肩部御寒,多无领。
    \item 〔路数〕底细,来路。
    \item 〔猪猡〕骂人的话。
    \item 〔作践〕摧残,虐待。
    \item 〔伶俐〕机灵,灵活。
\end{itemize}

\chapter{威尼斯商人}

\begin{normalsize}
    
    第四幕~第一场
    
    威尼斯\footnote{〔威尼斯〕今意大利北部港口城市。当时是独立的城邦,称为威尼斯共和国,公爵为贵族推举出的统治者。},法庭。
    
    \noindent $\triangleright$~公爵、安东尼奥、巴萨尼奥、葛拉骞诺、萨拉里诺、众绅士及其余人等齐上。
    
    \begin{description}[itemsep=1ex,leftmargin=4.5em,labelwidth=4em]
    
    \item[{\color{script-1-0} 公爵}]安东尼奥可在?
    
    \item[{\color{script-1-1} 安东尼奥}]在,大人。
    
    \item[{\color{script-1-0} 公爵}]我为你难过;你的对手心如铁石、没有人性,是个不懂得怜悯、没有一丝慈悲心的恶棍。
    
    \item[{\color{script-1-1} 安东尼奥}]听说大人曾经尽力劝他莫为已甚,可是他固持己见,不肯略作让步。既然用合法的手段摆脱不了他,我只有用默忍迎受他的愤怒,耐心等待他残暴的处置。
    
    \item[{\color{script-1-0} 公爵}]来人,传那犹太人\footnote{〔犹太人〕中东民族之一,因叛乱被古罗马帝国驱逐,在欧洲流亡,善经商营利。}到庭。
    
    \item[{\color{script-1-2} 萨拉里诺}]他在门口等着;他来了,大人。
    
    \end{description}
    
    \noindent $\triangleright$~沙义洛上。
    
    \begin{description}[itemsep=1ex,leftmargin=4.5em,labelwidth=4em]
    
    \item[{\color{script-1-0} 公爵}]大家让开些,让他站在我们面前。沙义洛,人人都认为——我也同意——你不过故意装出这一副凶恶的姿态,到了最后关头,就会显出你的仁慈恻隐来,比你现在表现出来的残酷更加出人意料。现在你虽然坚持着照约处罚,一定要从这个不幸的商人身上割下一磅\footnote{〔磅〕英制重量单位,一磅约为453克。莎士比亚创作时为了方便观众,使用了英国的重量单位。当时意大利威尼斯并不用这个单位。}肉来;到了那时候,你不但愿意放弃这一种处罚,而且因为受到良心上的感动,说不定还会豁免他一部分的欠款。你看他最近接连遭逢的巨大损失,无论多富有的商人也要倾家荡产。即使铁石心肠,从不知同情为何物的野蛮人,也不能不怜悯他境遇之凄惨。犹太人,我们都在等候你一个温柔的答案。
    
    \item[{\color{script-1-3} 沙义洛}]我的意思已经向大人告禀过了;我也已经在我们的圣安息日\footnote{〔圣安息日〕即星期日,犹太教认为该日是神的休息日,不得工作。}起誓,一定要照约执行处罚;要是大人不准许我的请求,那就是蔑视您定下的自由宪章,您的城邦的信誉将蒙受巨大损失。您要是问我为什么不愿接受三千块钱,宁愿拿一块腐烂的臭肉,那我可没有什么理由可以回答您,我只能说我欢喜这样,这样回答够不够?要是我的屋子里有了耗子,我高兴出一万块钱叫人把它们赶掉,谁管得了我?这算不算是回答了您?有的人不爱看张开嘴的猪,有的人瞧见一头猫就要发脾气,还有人听见哀乐\footnote{〔哀乐〕葬礼上演奏的音乐。}的风笛声,就忍不住要小便;因为一个人的感情完全受着喜恶的支配,做不了自己的主。现在我就这样回答您:为什么有人受不住一头张开嘴的猪,有人受不住一头有益无害的猫,还有人受不住咿咿唔唔的风笛声?这些都是毫无理由的,只是因为天生的癖性,使他们一受到刺激,就禁不住现出丑相来。所以我没法举什么理由,也不愿举什么理由。我对安东尼奥抱着久积的仇恨和深刻的反感,所以才会进行这场于我自己并无好处的诉讼。现在您不是已经得到我的回答了吗?
    
    \item[{\color{script-1-4} 巴萨尼奥}]你这冷酷无情的家伙,这样的回答可不能作为你的残忍的辩解。
    
    \item[{\color{script-1-3} 沙义洛}]我的回答本来不是为了讨你的欢喜。
    
    \item[{\color{script-1-4} 巴萨尼奥}]难道人们对自己不喜欢的东西,都一定要置之死地吗?
    
    \item[{\color{script-1-3} 沙义洛}]哪一个人会恨他所不愿意杀死的东西?
    
    \item[{\color{script-1-4} 巴萨尼奥}]初次的冒犯,不应该就引为仇恨。
    
    \item[{\color{script-1-3} 沙义洛}]什么!你愿意给毒蛇咬两次吗?
    
    \item[{\color{script-1-1} 安东尼奥}]请你想一想,你现在跟这个犹太人讲理,就像站在海滩上,叫那大海的怒涛减低它奔腾的威力,责问豺狼为什么害母羊失去它的羔羊,或是叫那山上的松柏,在受到天风吹拂的时候,不要摇头摆脑,发出飒飒的声音。要是你能够叫这个犹太人的心变软——世上还有什么东西比它更硬呢?——那么还有什么难事不可以做到?所以您不用再跟他商量什么条件,也不用替我想什么办法,让我爽爽快快受到判决,满足这犹太人的心愿吧。
    
    \item[{\color{script-1-4} 巴萨尼奥}]借了你三千块钱,现在拿六千块钱还你好不好?
    
    \item[{\color{script-1-3} 沙义洛}]即使这六千块钱中间的每一块钱都可以分做六份,每一份都可以变成一块钱,我也不要它们;我只要照约处罚。
    
    \item[{\color{script-1-0} 公爵}]你这样毫无慈悲心,将来怎能指望别人对你慈悲呢?
    
    \item[{\color{script-1-3} 沙义洛}]我又不干错事,怕什么刑罚?你们买了许多奴隶,把他们当作驴狗骡马一样看待,叫他们做种种卑贱的工作,因为他们是你们出钱买来的。我可不可以对你们说,让他们自由,叫他们跟你们的子女结婚?为什么他们要在重担之下流着血汗?让他们的床铺得跟你们的床同样柔软,让他们的舌头也尝尝你们吃的东西吧。你们会回答说:“这些奴隶是属于我们的东西。”所以我也可以回答你们:我向他要求的这一磅肉,是我出了很大代价买来的;它是属于我的,我一定要把它拿到手里。您要是拒绝了我,那么让你们的法律去见鬼吧!威尼斯城的法律就是一纸空文。我现在等候着判决,请快些回答我,我可不可以拿到这一磅肉?
    
    \item[{\color{script-1-0} 公爵}]我已经差人去请贝拉里奥,一位有学问的博士\footnote{〔博士〕指“法学博士”,学历头衔。取得执照后可以做律师。},来替我们审判这件案子;要是他今天不来,我有权宣布延期判决。
    
    \item[{\color{script-1-2} 萨拉里诺}]大人,外面有一个使者刚从帕度亚\footnote{〔帕度亚〕今帕多瓦,意大利北部城市,在威尼斯以西约40公里。}来,带着这位博士的书信,等候大人的召唤。
    
    \item[{\color{script-1-0} 公爵}]把信拿来给我;叫那使者进来。
    
    \item[{\color{script-1-4} 巴萨尼奥}]高兴起来吧,安东尼奥!老兄,不要灰心!这犹太人可以把我的肉、我的血、我的骨头、我的一切都拿去,可是我决不让你为了我的缘故流一滴血。
    
    \item[{\color{script-1-1} 安东尼奥}]我是羊群里一头不中用的病羊,死是我的应分;最软的果子最先落到地上,让我也就这样结束了我的一生吧。巴萨尼奥,我只要你活下去,将来替我写一篇墓志铭\footnote{〔墓志铭〕刻在墓碑上的文字。},那你就是做了再好不过的事。
    
    \end{description}
    
    \noindent $\triangleright$~尼莉莎易容\footnote{〔易容〕修饰变化容貌,让人认不出来。}扮律师书记\footnote{〔书记〕法庭职位,负责传唤当事人及证人、制作笔录、起草文件,传达意见等工作。律师书记负责协助律师整理资料、与庭上人员沟通等等。}上。
    
    \begin{description}[itemsep=1ex,leftmargin=4.5em,labelwidth=4em]
    
    \item[{\color{script-1-0} 公爵}]你是从帕度亚的贝拉里奥那里来的吗?
    
    \item[{\color{script-1-5} 尼莉莎}]是,大人。贝拉里奥叫我向公爵大人致意。(呈上一信)
    
    \item[{\color{script-1-4} 巴萨尼奥}]你这样使劲儿磨着刀干吗?
    
    \item[{\color{script-1-3} 沙义洛}]从那破产的家伙身上割下那磅肉来。
    
    \item[{\color{script-1-6} 葛拉骞诺}]狠心的犹太人,你不是在鞋口上磨刀,你这把刀是放在你的心口上磨;无论哪种兵刃,就连刽子手的钢刀,都赶不上你这刻毒的心肠一半的锋利。难道什么恳求都不能打动你吗?
    
    \item[{\color{script-1-3} 沙义洛}]不能,无论你说得多么婉转动听,都没有用。
    
    \item[{\color{script-1-6} 葛拉骞诺}]万恶不赦的狗,看你死后不下地狱\footnote{〔地狱〕犹太教、基督教的概念,认为不悔改的罪人将进入的地方。}!让你这种东西活在世上,真是公道不生眼睛。你简直使我的信仰动摇,相信起毕达哥拉斯\footnote{〔毕达哥拉斯〕古希腊思想家,提出灵魂转生的说法。}所说,畜生的灵魂可以转生人体的理论了;你的前生一定是头豺狼,吃了人给人捉住吊死,凶恶的灵魂从绞架上逃了出来,钻进了你母亲的胎里,因为你的性情正像豺狼一样,残暴贪婪,嗜血无情。
    
    \item[{\color{script-1-3} 沙义洛}]除非你能够把我这张契约上的印章骂掉,否则像你这样拉开了喉咙直嚷嚷,不过白白伤了肺,何苦来呢?年轻人,我劝你爱护你的心智,免得将来崩坏。我支持法律,只请求法律的裁判。
    
    \item[{\color{script-1-0} 公爵}]贝拉里奥在这封信上介绍一位年轻有学问的博士出席我们的法庭。他在什么地方?
    
    \item[{\color{script-1-5} 尼莉莎}]他就在这儿附近等着您的答复,不知道大人准不准许他进来?
    
    \item[{\color{script-1-0} 公爵}]非常欢迎。来,你们去三四个人,恭恭敬敬领他到这儿来。现在让我们把贝拉里奥的来信当庭宣读。
    
    \item[{\color{script-1-7} 书记}](读)“尊翰\footnote{〔尊翰〕尊函,书信里对对方的来信的尊称。}到时,鄙人\footnote{〔鄙人〕对自己的谦称。}抱疾方剧。适有青年博士名鲍尔萨泽者\footnote{〔鲍尔萨泽〕鲍西娅为女性,无法做律师,只好假扮作男性。“鲍尔萨泽”是音相近的男性名。}自罗马\footnote{〔罗马〕今意大利首都,在意大利中部。当时罗马是基督教廷所在地,威尼斯作为城邦独立于基督教廷统辖的教皇国。}来,因与之详究犹太人诉安东尼奥一案,遍稽典故,折衷是非,恳请为鄙人庖代\footnote{〔庖代〕指越俎代庖。出自《庄子·逍遥游》。主祭的人越过礼器去做厨子的活,比喻做不在职权范围内的事。这里指让没有执照的鲍尔萨泽代理律师。},以应君上之召。凡鄙人对此案所具意见,鲍生皆已深悉;其人学问才识,鄙人平生之仅见也,足当此任,倘蒙延纳,必不辱命,惟盼勿以其年少而轻慢之。敬祈钧裁\footnote{〔敬祈钧裁〕请您做出裁决(指让鲍尔萨泽做代理律师)。}。”
    
    \item[{\color{script-1-0} 公爵}]你们已经听到了博学的贝拉里奥的来信。这儿来的大概就是那位博士了。
    
    \end{description}
    
    \noindent $\triangleright$~鲍西娅易容扮律师上。
    
    \begin{description}[itemsep=1ex,leftmargin=4.5em,labelwidth=4em]
    
    \item[{\color{script-1-0} 公爵}]把您的手给我。您可是从我们的老贝拉里奥那里来的?
    
    \item[{\color{script-1-8} 鲍西娅}]正是,大人。
    
    \item[{\color{script-1-0} 公爵}]欢迎欢迎;请上坐。本案两方争执何在,您是否明白?
    
    \item[{\color{script-1-8} 鲍西娅}]这件案子的详细情形,我已经完全清楚了。此处哪一个是那商人,哪一个是犹太人?
    
    \item[{\color{script-1-0} 公爵}]安东尼奥,沙义洛,你们两人都上来。
    
    \item[{\color{script-1-8} 鲍西娅}]你的名字就叫沙义洛吗?
    
    \item[{\color{script-1-3} 沙义洛}]沙义洛是我的名字。
    
    \item[{\color{script-1-8} 鲍西娅}]你这场官司打得倒也奇怪,可是按照威尼斯的法律,你的控诉是成立的。(向安东尼奥)你的生死现在操在他的手里,是也不是?
    
    \item[{\color{script-1-1} 安东尼奥}]他是这样说的。
    
    \item[{\color{script-1-8} 鲍西娅}]你承认这借约吗?
    
    \item[{\color{script-1-1} 安东尼奥}]我承认。
    
    \item[{\color{script-1-8} 鲍西娅}]那么犹太人,你应该慈悲一点。
    
    \item[{\color{script-1-3} 沙义洛}]为什么我应该慈悲一点?把您的理由告诉我。
    
    \item[{\color{script-1-8} 鲍西娅}]慈悲不是出于勉强,它像甘霖一样降下尘世;它不但赐福于受施的人,也同样赐福于施与的人;它有超乎一切的无上伟力,比王冠更能显出君王的高贵:王冠不过象征着俗世的威权,让凡人对君主的威严凛然生畏;而慈悲的力量更在威权之上,它深藏在君王的内心,是一种属于神明的仁德。执法的人倘能把慈悲调剂着公道,他在人间的权力就和神明的伟力没有差别。所以,犹太人,虽然你所要求的是公道,可是请你想一想,要是真的按照公道执行起赏罚来,谁也没有死后得救的希望;我们既然祈求主的慈悲,就应该按照经文的指点,自己做一些慈悲的事。我说这一番话,是希望你能够作几分让步;可是如果你坚持原来的要求,那么威尼斯的法庭是公正无私的,只好把那商人宣判定罪了。
    
    \item[{\color{script-1-3} 沙义洛}]我自己做的事,我自己当!我只要求法律允许我照约执行处罚。
    
    \item[{\color{script-1-8} 鲍西娅}]他是不是无力偿还这笔借款?
    
    \item[{\color{script-1-4} 巴萨尼奥}]不,我愿意替他当庭还清;照原数加倍也可以;要是这样他还不满足,那么我愿意签署契约,还他十倍的数目,拿我的手、我的头、我的心做抵押;要是这样还不能使他满足,那就是存心害人,不顾天理。请堂上\footnote{〔堂上〕庭上,对法庭的尊称。}运用权力,稍稍变通法律,犯一次小小的错误,成就一件大功德,莫让这残忍的恶魔逞他杀人的兽欲。
    
    \item[{\color{script-1-8} 鲍西娅}]那可不行。在威尼斯,谁也无权变更既成的法律;要是开了这个恶例,以后谁都有例可援,什么坏事都可以干了。这是不行的。
    
    \item[{\color{script-1-3} 沙义洛}]一个但尼尔\footnote{〔但尼尔〕犹太人宗教中有名的法官。}来做法官了!真是但尼尔再世!聪明的青年法官啊,我真佩服你!
    
    \item[{\color{script-1-8} 鲍西娅}]请你让我瞧一瞧那借约。
    
    \item[{\color{script-1-3} 沙义洛}]在这儿,尊敬的博士;请看吧。
    
    \item[{\color{script-1-8} 鲍西娅}]沙义洛,他们愿出三倍的钱还你呢。
    
    \item[{\color{script-1-3} 沙义洛}]不行,不行,我已经对神明发过誓啦,难道我可以让我的灵魂背上毁誓的罪名吗?不,把整个儿的威尼斯给我,我都不能答应。
    
    \item[{\color{script-1-8} 鲍西娅}]好,那么就应该照约处罚;根据法律,这犹太人有权要求从这商人的胸口割下一磅肉来。还是慈悲一点,把三倍原数的钱拿去,让我撕了这张约吧。
    
    \item[{\color{script-1-3} 沙义洛}]等他照约受罚以后,再撕不迟。您瞧上去像是一个很好的法官;您精通法律,讲的话也很有道理,不愧是法律界的中流砥柱。所以现在我以法律的名义,请您立刻进行宣判。凭着我的灵魂起誓,谁也不能用他的口舌改变我的决心。我现在只等着执行约定。
    
    \item[{\color{script-1-1} 安东尼奥}]我也诚心请求堂上从速宣判。
    
    \item[{\color{script-1-8} 鲍西娅}]好,那么就是这样:你必须准备让他的刀子刺进你的胸膛。
    
    \item[{\color{script-1-3} 沙义洛}]啊,尊贵的法官!好一位优秀的青年!
    
    \item[{\color{script-1-8} 鲍西娅}]因为这约上所订定的惩罚,与法律条文的涵义并无抵触。
    
    \item[{\color{script-1-3} 沙义洛}]很对很对!啊,聪明正直的法官!想不到你瞧上去这样年轻,见识却这么老练!
    
    \item[{\color{script-1-8} 鲍西娅}]所以你应该把你的胸膛袒露出来。
    
    \item[{\color{script-1-3} 沙义洛}]没错,“他的胸部”,约上是这么说的——不是吗,尊敬的法官?——“靠近心口的所在”,约上写得明明白白。
    
    \item[{\color{script-1-8} 鲍西娅}]不错,称肉的天平有没有预备好?
    
    \item[{\color{script-1-3} 沙义洛}]我已经带来了。
    
    \item[{\color{script-1-8} 鲍西娅}]沙义洛,去请一位外科医生来替他堵住伤口,免得他流血而死。费用由你负担。
    
    \item[{\color{script-1-3} 沙义洛}]约上有这样的规定吗?
    
    \item[{\color{script-1-8} 鲍西娅}]约上并没有这样的规定;可是那又有什么相干呢?肯做一件好事总是好的。
    
    \item[{\color{script-1-3} 沙义洛}]我找不到;约上没有这一条。
    
    \item[{\color{script-1-8} 鲍西娅}]商人,你还有什么话说吗?
    
    \item[{\color{script-1-1} 安东尼奥}]我没有多少话要说;我已经准备好了。把你的手给我,巴萨尼奥,再会吧!不要为我的结局悲伤自责,因为命运对我已经特别照顾了:她往往让一个不幸的人在家产散尽后苟活下去,用他凹陷的眼睛和满是皱纹的额角经受贫困的暮年;这一种漫漫无期的酷刑,她已经把我豁免了。替我向您的夫人致意,告诉她安东尼奥的结局;对她讲述我怎样爱你,又怎样从容就死;你讲完这故事,再请她判断一句,巴萨尼奥是不是曾经有过一个真心爱他的朋友。不要因为你将要失去一个朋友而懊恨,替你还债的人死而无怨;只要那犹太人的刀刺得深一点,我就可以在一刹那的时间把那笔债完全还清。
    
    \item[{\color{script-1-4} 巴萨尼奥}]安东尼奥,我爱我的妻子,她就是我的生命。可是我的生命、我的妻子以及整个的世界,在我的眼中都不比你的生命更为重要;我愿意丧失一切,把它们献给这恶魔做牺牲,来救出你的生命。
    
    \item[{\color{script-1-8} 鲍西娅}]尊夫人要是在此听见您说这话,恐怕不见得会感谢您吧。
    
    \item[{\color{script-1-6} 葛拉骞诺}]我有一个妻子,我可以发誓我是爱她的;可是我希望她马上归天,好去求主改变这恶狗一样的犹太人的心。
    
    \item[{\color{script-1-5} 尼莉莎}]幸亏您在她背后说这样的话,否则您家里定要鸡犬不宁。
    
    \item[{\color{script-1-3} 沙义洛}]这些便是相信基督教的丈夫!我有一个女儿,我宁愿她嫁给强盗的子孙,不愿她嫁给一个基督徒。别再浪费时间了,请快些儿宣判吧。
    
    \item[{\color{script-1-8} 鲍西娅}]那商人身上的一磅肉是你的,法庭判给你,法律许可你。
    
    \item[{\color{script-1-3} 沙义洛}]公平正直的法官!
    
    \item[{\color{script-1-8} 鲍西娅}]你必须从他的胸前割下这磅肉来;法律许可你,法庭判给你。
    
    \item[{\color{script-1-3} 沙义洛}]博学多才的法官!判得好!来,预备!
    
    \item[{\color{script-1-8} 鲍西娅}]且慢,还有别的话哩。这约上并没有允许你取他的一滴血,只是写明着“一磅肉”。所以你可以照约拿一磅肉去,可是在割肉的时候,要是流下一滴基督徒的血,你的土地财产,按照威尼斯的法律,就要全部充公。
    
    \item[{\color{script-1-6} 葛拉骞诺}]啊,公平正直的法官!听着,犹太人!啊,博学多才的法官!
    
    \item[{\color{script-1-3} 沙义洛}]法律上是这样说吗?
    
    \item[{\color{script-1-8} 鲍西娅}]你自己可以去查查明白。既然你要求公道,我就给你公道,而且比你所要求的更地道。
    
    \item[{\color{script-1-6} 葛拉骞诺}]啊,博学多才的法官!听着,犹太人!好一个博学多才的法官!
    
    \item[{\color{script-1-3} 沙义洛}]那么我愿意接受还款。照约上的数目三倍还我,放了那基督徒。
    
    \item[{\color{script-1-4} 巴萨尼奥}]钱在这儿。
    
    \item[{\color{script-1-8} 鲍西娅}]别忙!这犹太人必须得到绝对的公道。别忙!他除了照约处罚以外,不能接受其他的赔偿。
    
    \item[{\color{script-1-6} 葛拉骞诺}]啊,犹太人!一个公平正直的法官,博学多才的法官!
    
    \item[{\color{script-1-8} 鲍西娅}]所以你准备着动手割肉吧。不准流一滴血,也不准割得比一磅多或比一磅少;要是你割下来的肉,比一磅略微轻一点或是重一点,即使相差只有一丝一毫,哪怕只差一根毫毛,就要你偿命,你的财产全部充公。
    
    \item[{\color{script-1-6} 葛拉骞诺}]再世的但尼尔,但尼尔!犹太人!现在你可掉在我的手里了,你这异教徒\footnote{〔异教徒〕信仰别的宗教的人。}!
    
    \item[{\color{script-1-8} 鲍西娅}]那犹太人,为什么还不动手?
    
    \item[{\color{script-1-3} 沙义洛}]把我的本钱还我,放我去吧。
    
    \item[{\color{script-1-4} 巴萨尼奥}]钱我已经备好在此,你拿去吧。
    
    \item[{\color{script-1-8} 鲍西娅}]他当庭拒绝过了。我们现在只能给他公道,让他履行原约。
    
    \item[{\color{script-1-6} 葛拉骞诺}]好一个但尼尔,一个再世的但尼尔!谢谢你,犹太人,你教会我说这句话。
    
    \item[{\color{script-1-3} 沙义洛}]难道我单单拿回我的本钱都不成吗?
    
    \item[{\color{script-1-8} 鲍西娅}]犹太人,除了冒着你自己生命的危险割下那一磅肉以外,你不能拿一个钱。
    
    \item[{\color{script-1-3} 沙义洛}]好,那么魔鬼保佑他去享用吧!我不打这场官司了。
    
    \item[{\color{script-1-8} 鲍西娅}]等一等,犹太人,法律上还有一点牵涉你。威尼斯的法律规定:凡异邦人企图用直接或间接的手段,谋害任何公民\footnote{〔公民〕城邦的法定身份。有资产的男性,向城邦纳税,获得公民身份,可以在城邦中自由生活并参与公共政治。},查明确有实据者,其财产的半数应当归受害的一方所有,其余半数没入公库;犯罪者的性命悉听公爵处置,他人不得干涉。你现在正正犯了这一条规矩。因为事情发展至此,已经足以证明,你确有运用直接间接手段,危害被告生命的企图,所以你已经遭逢着我刚才所说的那种危险了。还不赶快跪下,乞求公爵开恩?
    
    \item[{\color{script-1-6} 葛拉骞诺}]求公爵开恩,让你自己去寻死吧;可是你的财产现在充了公,一根绳子也买不起啦,所以还是要让公家破费把你吊死。
    
    \item[{\color{script-1-0} 公爵}]且让你好好看看我们的慈悲,你虽然没有向我开口,我自动饶恕了你的死罪。你的财产一半划归安东尼奥,还有一半没入公库;要是你能够诚心悔过,也许还可以减免,只处你一笔较轻的罚款。
    
    \item[{\color{script-1-8} 鲍西娅}]这是说没入公库的部分,不是说划归安东尼奥的部分。
    
    \item[{\color{script-1-3} 沙义洛}]不,把我的生命连着财产一起拿了去吧,我不要你们的宽恕。拿掉了梁柱,就是拆了房子;你们夺去了我养家活命的根本,就是活活要了我的命。
    
    \item[{\color{script-1-8} 鲍西娅}]安东尼奥,你能不能够给他一点慈悲?
    
    \item[{\color{script-1-6} 葛拉骞诺}]白送他一根上吊的绳子吧;看在主的份上,不要给他别的东西!
    
    \item[{\color{script-1-1} 安东尼奥}]要是大人愿意从宽发落,不没收他的财产充公的一半,我就十分满足了;只要他能够让我接管他另一半的财产,等他死了以后,把它交给最近和他女儿私奔的那位绅士。还有两个附带条件:第一,他接受了这样的恩典,必须立刻改信基督教;第二,他必须当庭写下一张文契,声明他死了以后,他的全部财产传给他的女婿罗兰佐和他的女儿\footnote{〔罗兰佐……〕沙义洛的女儿杰西卡与威尼斯青年罗兰佐相爱。但沙义洛因为罗兰佐是基督徒而反对他们。罗兰佐带着杰西卡私奔,这使得沙义洛对安东尼奥仇恨更深。}。
    
    \item[{\color{script-1-0} 公爵}]他必须履行这两个条件,否则我就撤销刚才宣布的赦令。
    
    \item[{\color{script-1-8} 鲍西娅}]犹太人,你满意吗?你有什么话说?
    
    \item[{\color{script-1-3} 沙义洛}]我满意。
    
    \item[{\color{script-1-8} 鲍西娅}]书记,写下一张授赠产业的文契。
    
    \item[{\color{script-1-3} 沙义洛}]请你们允许我退庭,我身子不大舒服。文契写好了送到我家里,我在上面签名就是了。
    
    \item[{\color{script-1-0} 公爵}]去吧,可是临时变卦是不成的。
    
    \item[{\color{script-1-6} 葛拉骞诺}]你受洗礼\footnote{〔受洗礼〕基督教的入教仪式。}的时候会有两个教父\footnote{〔教父〕基督教新教徒入教的担保人,负责指导教育其担保的新教徒。};要是我做了法官,我一定给你请十二个教父\footnote{〔十二个教父〕“十二个”指陪审团的最低人数,这里用于讽刺,指想要凑齐十二个人审判沙义洛。},不是领你去受洗,是送你上绞架。
    
    \end{description}
    
    \noindent $\triangleright$~沙义洛下。
    
    
\end{normalsize}


\newpage

\textbf{注释}:

\vspace{-1em}

\begin{itemize}
    \setlength\itemsep{-0.2em}
    \item 〔中流砥柱〕逆大流毫不动摇力挽狂澜的人。中流:河中央的水流。砥:阻滞、磨。砥柱:指砥柱山,在黄河三门峡处河中央。
    \item 〔万恶不赦〕指罪恶极大,不可饶恕。
    \item 〔莫为已甚〕不要做得太过分。已甚:过分。
    \item 〔恻隐〕为他人的不幸而悲痛。
    \item 〔癖性〕难以改变的偏好。
    \item 〔私奔〕相爱的情侣逃离家族自行生活。
    \item 〔逞〕施展,显露,实现意愿。
    \item 〔甘霖〕农民对久旱后下雨、及时雨的称呼。
    \item 〔凛然〕畏惧的样子。
    \item 〔履行〕按约定或职责去做。
    \item 〔一纸空文〕无人遵守的规定、命令。
    \item 〔鸡犬不宁〕比喻家中发生骚乱变故,让全家不得安宁。
    \item 〔从速〕尽快,越快越好。从:依据,遵照。速:快。
    \item 〔从宽〕越宽松越好。宽:指处置的时候宽容、宽松、通融,和“严”相对。
    \item 〔发落〕处置、惩罚。
    \item 〔赦令〕减免罪行、义务的命令。
    \item 〔恩典〕原指君主的恩赐和礼遇。现泛指恩惠。
    \item 〔调剂〕适当地调和。
    \item 〔援例〕引用(以前的)例子。援:引用。
    \item 〔文契〕用文字写下的契约。
    \item 〔变卦〕(擅自)改变主意。
\end{itemize}

\chapter{假使我们不去打仗}

\begin{normalsize}
    
    \begin{verse}[0.5\linewidth]
        假使我们不去打仗, \\
        敌人用刺刀 \\
        杀死了我们, \\
        还要用手指着我们骨头说: \\
        “看, \\
        这是奴隶!”
    \end{verse}
    
\end{normalsize}



\chapter{消息两则}

\begin{normalsize}
    
    \section{我三十万大军胜利南渡长江}
    
    【新华社长江前线22日2时电】英勇的人民解放军二十一日已有大约三十万人渡过长江。渡江战斗于二十日午夜开始,地点在芜湖、安庆之间\footnote{〔安庆……〕安庆:安徽省西南部城市,在长江下游北岸。芜湖:安徽东南部城市,长江流过市区,主城区在长江南岸。}。国民党反动派经营了三个半月的长江防线,遇着人民解放军好似摧枯拉朽,军无斗志,纷纷溃退。长江风平浪静,我军万船齐发,直取对岸。不到二十四小时,三十万人民解放军即已突破敌阵,占领南岸广大地区,现正向繁昌、铜陵、青阳、荻港、鲁港\footnote{〔繁昌……〕繁昌铜陵、青阳、荻港、鲁港:安徽长江南岸县市,在安庆、芜湖东南方向。}诸城进击中,人民解放军正以自己的英雄式的战斗,坚决地执行毛主席朱总司令的命令。
    
    \section{人民解放军百万大军横渡长江}
    
    【新华社长江前线22日22时电】人民解放军百万大军,从一千余华里\footnote{〔华里〕即里,长度单位。一里等于500米。}的战线上,冲破敌阵,横渡长江。西起九江(不含),东至江阴,均是人民解放军的渡江区域。二十日夜起,长江北岸人民解放军中路军首先突破安庆、芜湖线,渡至繁昌、铜陵、青阳、荻港、鲁港地区,二十四小时内即已渡过三十万人。二十一日下午五时起,我西路军开始渡江,地点在九江、安庆段。至发电时止,该路三十五万人民解放军已渡过三分之二,余部二十三日可渡完。这一路现已占领贵池、殷家汇、东流、至德、彭泽之线的广大南岸阵地\footnote{〔贵池……〕贵池、殷家汇、东流、至德、彭泽之线:安徽县市,在九江到安庆之间。},正向南扩展中。和中路军所遇敌情一样,我西路军当面之敌亦纷纷溃退,毫无斗志,我军所遇之抵抗,甚为微弱。此种情况,一方面由于人民解放军英勇善战,锐不可当;另一方面,这和国民党反动派拒绝签定和平协定,有很大关系。国民党的广大官兵一致希望和平,不想再打了,听见南京拒绝和平,都很泄气。战犯汤恩伯\footnote{〔汤恩伯〕民国将领,被视为蒋介石嫡系。抗日战争中军纪败坏,贪污腐败,残害人民;解放战争中负隅顽抗,卖师求荣。1949年逃至台湾,1954年病故。}二十一日到芜湖督战,不起丝毫作用。汤恩伯认为南京江阴段防线是很巩固的,弱点只存在于南京九江一线。不料正是汤恩伯到芜湖的那一天,东面防线又被我军突破了。我东路三十五万大军与西路同日同时发起渡江作战。所有预定计划,都已实现。至发电时止,我东路各军已大部渡过南岸,余部二十三日可以渡完。此处敌军抵抗较为顽强,然在二十一日下午至二十二日下午的整天激战中,我已歼灭及击溃一切抵抗之敌,占领扬中、镇江、江阴诸县的广大地区,并控制江阴要塞\footnote{〔扬中……要塞〕扬中、镇江、江阴:江苏中南部县市,在长江南岸。江阴要塞为长江下游最窄处,为兵家必争之地。},封锁长江。我军前锋,业已切断镇江无锡段铁路线。
    
\end{normalsize}


\newpage

\textbf{注释}:

\vspace{-1em}

\begin{itemize}
    \setlength\itemsep{-0.2em}
    \item 〔摧枯拉朽〕摧折枯枝朽木。形容强大的力量轻易摧毁腐朽势力的场景。
    \item 〔溃退〕失败后没有秩序地退逃。溃:大水冲开堤坝。
    \item 〔巩固〕牢固。
    \item 〔战犯〕发动非正义战争或在战争中犯下严重罪行的人。
    \item 〔督战〕监督作战。
    \item 〔歼灭〕通过战斗消灭。
    \item 〔锐不可当〕锐:锋利。当:抵挡。形容攻势强硬,无法阻挡。
    \item 〔业已〕已经。
\end{itemize}

\chapter{应有真正的格物致知精神}

\begin{normalsize}
    
    我父亲是受中国传统教育长大的,我受的教育一部分是传统教育,一部分是西方教育。多年来,我在学校里接触到不少中国学生,因此,我想借这个机会向大家谈谈:学习自然科学的中国学生,应该怎样了解自然科学。
    
    在中国传统教育里,最重要的书是“四书”\footnote{〔四书〕指儒家的四部经典:《论语》《中庸》《孟子》《大学》。}。“四书”之一的《大学》\footnote{〔《大学》〕一篇论述儒家修身齐家治国平天下思想的散文,原是《小戴礼记》第42篇,相传为春秋战国时期曾子所作。宋代以后成为儒家科举必读经典。}里这样说:一个人教育的出发点是“格物”和“致知”。就是说,从探察物体而得到知识。用这两个词语描写现代学术发展是再恰当也没有的了。现代学术的基础就是实地的探察,就是我们现在所谓的实验。
    
    但是传统的中国教育并不重视真正的格物和致知。这可能是因为传统教育的目的并不是寻求新知识,而是适应一个固定的社会制度。《大学》本身就说,格物致知的目的,是使人能达到诚意、正心、修身、齐家、治国的田地,从而追求儒家的最高理想——平天下。因为这样,格物致知的真正意义便被埋没了。
    
    大家都知道明朝的大理论家王阳明\footnote{〔王阳明〕王守仁,号阳明,明代儒家思想家,文学家,教育家。宋明理学中“心学”的集大成者。},他的思想可以代表传统儒家对实验的态度。有一天,王阳明要依照《大学》的指示,先从“格物”做起。他决定要“格”院子里的竹子。于是他搬了一条凳子坐在院子里,面对着竹子硬想了七天,结果因为头痛而宣告失败。这位先生显然是把探察外界误当成了探讨自己。
    
    王阳明的观点,在当时的社会环境里是可以理解的。因为儒家传统的看法认为天下有不变的真理,而真理是“圣人”从内心领悟的。圣人知道真理以后,就传给一般人。所以经书上的道理是可“推之于四海,传之于万世”\footnote{〔“推之于四海,传之于万世”〕源自朱熹《近思录》第八卷:“推之以及四海,则万世幸甚。”这里指把真理普及到每一个地方,流传到万世以后。}的。经验告诉我们,这种观点是不能适用于现在的世界的。
    
    我是研究科学的人,所以先让我谈谈实验精神在科学上的重要性。
    
    科学发展的历史告诉我们,新的知识只能通过实地实验而得到,不是由自我检讨或哲理的清谈就可求到的。
    
    实验的过程不是消极的观察,而是积极的、有计划的探测。比如,我们要知道竹子的性质,就要特地栽种竹树,以研究它生长的过程,要把叶子切下来拿到显微镜\footnote{〔显微镜〕将肉眼不可见的微小物体的影像放大观察的仪器。}下去观察,绝不是袖手旁观就可以得到知识的。
    
    实验不是毫无选择的测量,它需要有小心具体的计划。特别重要的,是要有一个适当的目标,以作为整个探索过程的向导。至于这目标怎样选定,就要靠实验者的判断力和灵感。一个成功的实验需要的是眼光、勇气和毅力。
    
    由此我们可以了解,为什么基本知识上的突破是不常有的事情。我们也可以了解,为什么历史上学术的进展只依靠极少数人的关键发现。
    
    今天,王阳明的思想仍然支配着一些中国读书人的头脑。因为文化的背景,中国学生大部偏向于理论而轻视实验,偏向于抽象的思维而不愿动手。中国学生往往念书成绩很好,考试都能得近100分,但是面临着需要主意的研究工作时,就常常不知所措了。
    
    在这方面,我有个人的经验为证。我是受传统教育长大的。到美国大学念物理的时候,起先以为只要很“用功”,什么都遵照老师的指导,就可以一帆风顺了,但是事实并不是这样。一开始做研究便马上发现不能光靠教师,需要自己做主张、出主意。当时因为事先没有准备,不知吃了多少苦。最使我惶恐的,是当时的唯一办法——以埋头读书应付一切,对于实际的需要毫无帮助。
    
    我觉得真正的格物致知精神,不但在研究学术中不可缺少,在应付今天的世界环境时也是不可少的。今天的普通教育里,我们需要培养实验的精神。就是说,不管研究科学,还是人文学科,或者在个人行动上,我们都要保留一个怀疑求真的态度,要靠实践来发现事物的真相。现在世界和社会的环境变化得很快。世界上不同文化的交流也越来越密切。我们不能盲目地接受过去认为的真理,也不能等待“学术权威”的指示。我们要自己有判断力。在环境激变的今天,我们应该重新体会到几千年前经书里说的格物致知真正的意义。这意义有两个方面:第一,寻求真理的唯一途径是对事物客观的探索;第二,探索的过程不是消极的袖手旁观,而是有想象力的有计划的探索。希望我们这一代对于格物和致知有新的认识和思考,使得实验精神真正地变成中国文化的一部分。
    
\end{normalsize}


\newpage

\textbf{注释}:

\vspace{-1em}

\begin{itemize}
    \setlength\itemsep{-0.2em}
    \item 〔袖手旁观〕把手放在袖子里,在一旁看。表示不实际参与。
    \item 〔遵照〕按照,依照。
    \item 〔抽象〕将事物的表象抽去,只关心剩余特性的思考过程。不具体的,不直观的。
    \item 〔一帆风顺〕一切顺利。
    \item 〔惶恐〕惊恐,害怕,不知所措。
\end{itemize}

\chapter{中国人失掉自信力了吗}

\begin{normalsize}
    
    从公开的文字上看起来:两年以前,我们总自夸着“地大物博”,是事实;不久就不再自夸了,只希望着国联\footnote{〔国联〕国际联盟,第一次世界大战后于1920年成立的国际政府间组织。“九一八”事变后,国民政府多次向国联申诉,要求制止日本侵略中国。但国联采取袒护日本的立场,不采取任何制裁措施。},也是事实;现在是既不夸自己,也不信国联,改为一味求神拜佛\footnote{〔求神拜佛〕指当时国民政府高层和社会名流以“解救国难”为名,多次在大城市举行法会祈祷。},怀古伤今了——却也是事实。
    
    于是有人慨叹曰:中国人失掉自信力了\footnote{〔中国人失掉自信力了〕当时舆论界有这类论调。}。
    
    如果单据这一点现象而论,自信其实是早就失掉了的。先前信“地”,信“物”,后来信“国联”,都没有相信过“自己”。假使这也算一种“信”,那也只能说中国人曾经有过“他信力”,自从对国联失望之后,便把这他信力都失掉了。
    
    失掉了他信力,就会疑,一个转身,也许能够只相信了自己,倒是一条新生路,但不幸的是逐渐玄虚起来了。信“地”和“物”,还是切实的东西,国联就渺茫,不过这还可以令人不久就省悟到依赖它的不可靠。一到求神拜佛,可就玄虚之至了,有益或是有害,一时就找不出分明的结果来,它可以令人更长久的麻醉着自己。
    
    中国人现在是在发展着“自欺力”。
    
    “自欺”也并非新东西,只不过日见其明显,笼罩了一切罢了。然而,在这笼罩之下,我们有并不失掉自信力的中国人在。
    
    我们从古以来,就有埋头苦干的人,有拼命硬干的人,有为民请命的人,有舍身求法的人,……虽是等于为帝王将相作家谱的所谓“正史”\footnote{〔“正史”〕指清代指定从《史记》到《明史》的二十四部纪传体史书为正史。},也往往掩不住他们的光耀,这就是中国的脊梁。
    
    这一类的人们,也何尝少呢?他们有确信,不自欺;他们在前仆后继的战斗,不过一面总在被摧残,被抹杀,消灭于黑暗中,不能为大家所知道罢了。说中国人失掉了自信力,用以指一部分人则可,倘若加于全体,那简直是诬蔑。
    
    要论中国人,必须不被搽在表面的自欺欺人的脂粉所诓骗,却看看他的筋骨和脊梁。自信力的有无,状元宰相的文章是不足为据的,要自己去看地底下。
    
    \hfill 九月二十五日
    
\end{normalsize}


\newpage

\textbf{注释}:

\vspace{-1em}

\begin{itemize}
    \setlength\itemsep{-0.2em}
    \item 〔怀古伤今〕怀念古代,对现今的情况感到伤心。
    \item 〔自欺欺人〕用自己都无法相信的话和手法来欺骗别人。
    \item 〔玄虚〕神秘难测,使人捉摸不透。
    \item 〔诓骗〕欺骗。
    \item 〔摧残〕造成严重损害,残害。
    \item 〔慨叹〕感慨叹息。
\end{itemize}

\chapter{夜}

\begin{normalsize}
    
    \begin{verse}[0.5\linewidth]
        河水悄悄流入梦乡, \\
        幽暗的松林失去喧响。 \\
        夜莺的歌声沉寂了, \\
        长脚秧鸡不再叫嚷。 \\
        夜来了,四下一片寂静, \\
        只有溪水轻轻歌唱。 \\
        明月撒下它的光辉, \\
        给周围一切披上银装。 \\
        大河银星万点, \\
        小溪银波微漾。 \\
        原野上浸水的青草, \\
        也闪着银色的光芒。 \\
        夜来了,四下一片寂静, \\
        大自然沉浸在梦乡。 \\
        明月撒下它的光辉, \\
        给周围一切披上银装。
    \end{verse}
    
\end{normalsize}



\chapter{拿来主义}

\begin{normalsize}
    
    中国一向是所谓“闭关主义”,自己不去,别人也不许来。自从给枪炮打破了大门之后,又碰了一串钉子,到现在,成了什么都是“送去主义”了。别的且不说罢,单是学艺上的东西,近来就先送一批古董到巴黎去展览,但终“不知后事如何”;还有几位“大师”们捧着几张古画和新画,在欧洲各国一路的挂过去,叫作“发扬国光”\footnote{〔“发扬国光”〕指1932年至1934年美术家徐悲鸿、刘海粟曾分别去欧洲一些国家举办中国美术展览或个人美术作品展览。}。听说不远还要送梅兰芳博士到苏联去,以催进“象征主义”\footnote{〔梅兰芳……〕1934年梅兰芳应苏联对外文化协会的邀请出国演出。鲁迅对外国把“男扮女装”作为文化猎奇的对象十分厌恶,并反对梅兰芳的迎合。},此后是顺便到欧洲传道。我在这里不想讨论梅博士演艺和象征主义的关系,总之,活人替代了古董,我敢说,也可以算得显出一点进步了。
    
    但我们没有人根据了“礼尚往来”的仪节,说道:拿来!
    
    当然,能够只是送出去,也不算坏事情,一者见得丰富,二者见得大度。尼采\footnote{〔尼采〕19世纪德国哲学家,唯意志论和“超人”哲学的鼓吹者。}就自诩过他是太阳,光热无穷,只是给与,不想取得。然而尼采究竟不是太阳,他发了疯。中国也不是,虽然有人说,掘起地下的煤来,就足够全世界几百年之用,但是,几百年之后呢?几百年之后,我们当然是化为魂灵,或上天堂,或落了地狱,但我们的子孙是在的,所以还应该给他们留下一点礼品。要不然,则当佳节大典之际,他们拿不出东西来,只好磕头贺喜,讨一点残羹冷炙做奖赏。这种奖赏,不要误解为“抛来”的东西,这是“抛给”的,说得冠冕些,可以称之为“送来”,我在这里不想举出实例\footnote{〔……不想举出实例〕1933年国民政府和美国签订五千万美元的“棉麦借款”,购买美国的小麦、面粉和棉花。实质是强买强卖。本文可能指类似的事情。}。
    
    我在这里也并不想对于“送去”再说什么,否则太不“摩登”了。我只想鼓吹我们再吝啬一点,“送去”之外,还得“拿来”,是为“拿来主义”。
    
    但我们被“送来”的东西吓怕了。先有英国的鸦片,德国的废枪炮,后有法国的香粉,美国的电影,日本的印着“完全国货”的各种小东西。于是连清醒的青年们,也对于洋货发生了恐怖。其实,这正是因为那是“送来”的,而不是“拿来”的缘故。
    
    所以我们要运用脑髓,放出眼光,自己来拿!
    
    譬如罢,我们之中的一个穷青年,因为祖上的阴功\footnote{〔阴功〕荫蔽和功德。指祖先的功绩,让后代也享福。}(姑且让我这么说说罢),得了一所大宅子,且不问他是骗来的,抢来的,或合法继承的,或是做了女婿换来的。那么,怎么办呢?我想,首先是不管三七二十一,“拿来”!但是,如果反对这宅子的旧主人,怕给他的东西染污了,徘徊不敢走进门,是孱头\footnote{〔孱头〕骂懦弱的人的话。};勃然大怒,放一把火烧光,算是保存自己的清白,则是昏蛋。不过因为原是羡慕这宅子的旧主人的,而这回接受一切,欣欣然的蹩\footnote{〔蹩〕躲躲闪闪地走动。}进卧室,大吸剩下的鸦片,那当然更是废物。“拿来主义”者是全不这样的。
    
    他占有,挑选。看见鱼翅,并不就抛在路上以显其“平民化”,只要有养料,也和朋友们像萝卜白菜一样的吃掉,只不用它来宴大宾;看见鸦片,也不当众摔在茅厕里,以见其彻底革命,只送到药房里去,以供治病之用,却不弄“出售存膏,售完即止”的玄虚。只有烟枪和烟灯,虽然形式和印度,波斯,阿拉伯的烟具都不同,确可以算是一种国粹,倘使背着周游世界,一定会有人看,但我想,除了送一点进博物馆之外,其余的是大可以毁掉的了。还有一群姨太太,也大以请她们各自走散为是,要不然,“拿来主义”怕未免有些危机。
    
    总之,我们要拿来。我们要或使用,或存放,或毁灭。那么,主人是新主人,宅子也就会成为新宅子。然而首先要这人沉着,勇猛,有辨别,不自私。没有拿来的,人不能自成为新人,没有拿来的,文艺不能自成为新文艺。
    
    \hfill 六月四日
    
\end{normalsize}


\newpage

\textbf{注释}:

\vspace{-1em}

\begin{itemize}
    \setlength\itemsep{-0.2em}
    \item 〔残羹冷炙〕吃剩的饭菜。比喻别人施舍的东西。
    \item 〔鸦片〕通称大烟,一种刺激性的成瘾毒品。
\end{itemize}

\chapter{桥}

\begin{normalsize}
    
    \begin{verse}[0.5\linewidth]
        当土地与土地被水分割, \\
        当道路与道路被水截断, \\
        智慧的人类伫立在水边: \\
        于是产生了桥。
    \end{verse}
    
    
    \begin{verse}[0.5\linewidth]
        苦于跋涉的人类, \\
        你应该感谢桥啊。
    \end{verse}
    
    
    \begin{verse}[0.5\linewidth]
        桥是土地与土地的连系; \\
        桥是河流与道路的爱情。 \\
        桥是异乡游子睡梦里的寄托; \\
        桥是送别的人断不了的回忆。
    \end{verse}
    
\end{normalsize}



\chapter{树}

\begin{normalsize}
    
    \begin{verse}[0.5\linewidth]
        一棵树,一棵树 \\
        彼此孤离地兀立着 \\
        风与阳光 \\
        告诉着它们的距离
    \end{verse}
    
    
    \begin{verse}[0.5\linewidth]
        但是在泥土的覆盖下 \\
        它们的根生长着 \\
        在看不见的深处 \\
        它们把根须纠缠在一起
    \end{verse}
    
\end{normalsize}



\chapter{我爱这土地}

\begin{normalsize}
    
    \begin{verse}[0.5\linewidth]
        假如我是一只鸟, \\
        我也应该用嘶哑的喉咙歌唱: \\
        这被暴风雨所打击的土地, \\
        这永远汹涌着我们的悲愤的河流, \\
        这无止息地吹刮着的激怒的风, \\
        和那来自林间的无比温柔的黎明…… \\
        ——然后我死了, \\
        连羽毛也腐烂在土地里。
    \end{verse}
    
    
    \begin{verse}[0.5\linewidth]
        为什么我的眼里常含泪水? \\
        因为我对这土地爱得深沉……
    \end{verse}
    
\end{normalsize}



\chapter{绿}

\begin{normalsize}
    
    我第二次到仙岩\footnote{〔仙岩〕山名,位于浙江省温州与瑞安两市之间。}的时候,我惊诧于梅雨潭的绿了。
    
    梅雨潭是一个瀑布潭。仙岩有三个瀑布,梅雨瀑最低。走到山边,便听见哗哗哗哗的声音;抬起头,镶在两条湿湿的黑边儿里的,一带白而发亮的水便呈现于眼前了。我们先到梅雨亭。梅雨亭正对着那条瀑布;坐在亭边,不必仰头,便可见它的全体了。亭下深深的便是梅雨潭。这个亭踞在突出的一角的岩石上,上下都空空儿的;仿佛一只苍鹰展着翼翅浮在天宇中一般。三面都是山,像半个环儿拥着;人如在井底了。这是一个秋季的薄阴的天气。微微的云在我们顶上流着;岩面与草丛都从润湿中透出几分油油的绿意。而瀑布也似乎分外的响了。那瀑布从上面冲下,仿佛已被扯成大小的几绺;不复是一幅整齐而平滑的布。岩上有许多棱角;瀑流经过时,作急剧的撞击,便飞花碎玉般乱溅着了。那溅着的水花,晶莹而多芒;远望去,像一朵朵小小的白梅,微雨似的纷纷落着。据说,这就是梅雨潭之所以得名了。但我觉得像杨花,格外确切些。轻风起来时,点点随风飘散,那更是杨花了。这时偶然有几点送入我们温暖的怀里,便倏地钻了进去,再也寻它不着。
    
    梅雨潭闪闪的绿色招引着我们;我们开始追捉她那离合的神光了。揪着草,攀着乱石,小心探身下去,又鞠躬过了一个石穹门,便到了汪汪一碧的潭边了。瀑布在襟袖之间;但我的心中已没有瀑布了。我的心随潭水的绿而摇荡。那醉人的绿呀,仿佛一张极大极大的荷叶铺着,满是奇异的绿呀。我想张开两臂抱住她;但这是怎样一个妄想呀。站在水边,望到那面,居然觉着有些远呢!这平铺着,厚积着的绿,着实可爱。她松松的皱缬着,像少妇拖着的裙幅,她轻轻的摆弄着;像跳动的初恋的处女的心,她滑滑的明亮着,像涂了“明油”一般,有鸡蛋清那样软,那样嫩,她又不杂些儿尘滓,宛然一块温润的碧玉,只清清的一色,但你却看不透她!我曾见过北京什刹海拂地的绿杨,脱不了鹅黄的底子,似乎太淡了。我又曾见过杭州虎跑寺旁高峻而深密的“绿壁”,丛叠着无穷的碧草与绿叶的,那又似乎太浓了。其余呢,西湖\footnote{〔西湖〕杭州市西的浅水湖。}的波太明了,秦淮河\footnote{〔秦淮河〕在江苏省西南部,经南京市区西入长江。}的水又太暗了。可爱的,我将什么来比拟你呢?我怎么比拟得出呢?大约潭是很深的、故能蕴蓄着这样奇异的绿;仿佛蔚蓝的天融了一块在里面似的,这才这般的鲜润呀。那醉人的绿呀!我若能裁你以为带,我将赠给那轻盈的舞女;她必能临风飘举了。我若能挹你以为眼,我将赠给那善歌的盲姝;她必明眸善睐了。我舍不得你;我怎舍得你呢?我用手拍着你,抚摩着你,如同一个十二三岁的小姑娘。我又掬你入口,便是吻着她了。我送你一个名字,我从此叫你“女儿绿”,好么?
    
    我第二次到仙岩的时候,我不禁惊诧于梅雨潭的绿了。
    
\end{normalsize}


\newpage

\textbf{注释}:

\vspace{-1em}

\begin{itemize}
    \setlength\itemsep{-0.2em}
    \item 〔踞〕蹲。
    \item 〔神光〕仿佛有灵性的光彩。
    \item 〔皱缬〕有花纹的绸缎皱起。
    \item 〔绺〕量词,指一束理顺了的丝,线、须、发等。
    \item 〔倏地〕极快地、疾速地。
    \item 〔尘滓〕细小的尘灰渣滓。
    \item 〔挹〕舀,把液体盛出来。
    \item 〔姝〕美丽的女子。
    \item 〔明眸善睐〕明亮的眼珠善于左右顾盼,形容女性眼睛美丽。
    \item 〔掬〕用两手捧。
\end{itemize}

\chapter{快乐王子}

\begin{normalsize}
    
    快乐王子的雕像高高耸立在城市上空一根高大的石柱上面。他浑身镶满了薄薄的纯金叶片,明亮的蓝宝石做成他的双眼,剑柄上,一颗硕大的红色宝石熠熠生辉。
    
    世人对他爱慕不已。“他像风向标\footnote{〔风向标〕测定风的来向的仪器,经常做成装饰。}一样漂亮,”一位想表现自己有艺术品味的市议员说了一句,接着又担心人们认为他不务实际,便补充道:“只是不如风向标那么实用。”
    
    “你为什么不能像快乐王子一样呢?”母亲对哭喊着要月亮的孩子说,“快乐王子做梦也不会靠哭闹要东西。”
    
    “世上还有如此快乐的人,真让我高兴,”潦倒的人盯着这座非凡的雕像,喃喃自语。
    
    “他看上去就像一位天使,”孤儿院的孩子们说着,从大教堂走出来,身上披着鲜红的斗篷,胸前挂着白净的围嘴儿。
    
    “你们是怎么知道的?”数学教师问道,“你们又没见过天使的模样。”
    
    “啊!可我们见过,是在梦里见到的。”孩子们答道。数学教师皱皱眉头并绷起了面孔,因为他不赞成孩子们做梦。
    
    有天夜里,一只小燕子从城市上空飞过。他的朋友们早在六个星期前就飞往埃及去了,可他却留在了后面,因为他太留恋那美丽无比的芦苇了。
    
    他飞了整整一天,夜晚时才来到这座城市。“我去哪儿过夜呢?”他说,“我希望城里有过夜的好地方。”
    
    这时,他看见了高大圆柱上的雕像。
    
    “我就在那儿过夜,”他高声说,“这是个好地方,充满了新鲜空气。”于是,他就在快乐王子两脚之间落了窝。
    
    “我有黄金做的卧室,”他朝四周看看后轻声自语,准备入睡。但就在他把头埋进羽翅的时候,一颗大大的水珠落在他的身上。“真是不可思议!”他叫了起来,“天上没有一丝云彩,星星清晰又明亮,却偏偏下起了雨。北欧的天气真是可怕。”
    
    紧接着又落下来一滴。
    
    “一座雕像连雨都遮挡不住,还有什么用处?”他说,“我得去找一个好烟囱做窝。”
    
    他决定飞离此处。
    
    可是还没等他张开羽翼,第三滴水又掉了下来,他抬头望去,看见了——啊!他看见了什么呢?
    
    快乐王子的双眼充满了泪水,泪珠顺着他金黄的脸颊淌了下来。王子的脸在月光下美丽无比,小燕子顿生怜悯之心。
    
    “你是谁?”他问对方。
    
    “我是快乐王子。”
    
    “那么你为什么哭呢?”燕子又问,“你把我的身上都打湿了。”
    
    “以前,我还活着的时候,”雕像开口说道,“我并不知道眼泪是什么东西,因为那时我住在王宫里,那是个哀愁无法进去的地方。白天人们伴着我在花园里玩,晚上我在大厅里领头跳舞。沿着花园有一堵高高的围墙,可我从没想过围墙那边是什么。我身边的一切太美好了。我的仆人都叫我快乐王子。的确,如果欢愉就是快乐的话,那我真是快乐无比。我就这么活着,也这么死去。他们把我高高地立在这儿,我能看见自己城市中所有的丑恶和困苦。尽管我现在的心是铅做的,可我还是忍不住要哭泣。”
    
    “啊!难道他不是铁石心肠的金像?”燕子对自己说。他很讲礼貌,不愿大声议论别人的私事。
    
    “远处,”雕像用低缓而悦耳的声音继续说,“远处的一条小街上住着一户穷人。一扇窗户开着,透过窗户我能看见一个女人坐在桌旁。她瘦削的脸上满是倦意,一双粗糙发红的手上到处是针眼,因为她是一个裁缝。她正在给缎子衣服绣上西番莲花,这是皇后最喜爱的宫女准备在下一次宫廷舞会上穿的。房间角落里的一张床上躺着她生病的孩子。孩子在发烧,嚷着要吃桔子。他的妈妈除了给他喂几口河水,什么也没有,因此孩子老是哭个不停。燕子,燕子,小燕子,你愿意把我剑柄上的红宝石取下来送给她吗?我的双脚被固定在这基座上,不能动弹。”
    
    “伙伴们在埃及等我,”燕子说,“他们正在尼罗河上飞来飞去,同一朵朵大莲花说着话儿,不久就要到伟大法老\footnote{〔法老〕古埃及王国的统治者。}的墓穴里去过夜。法老睡在彩色的棺材里,他的身体裹在黄色的亚麻布里,还填满了防腐的香料。他的脖子上系着一圈浅绿色翡翠项链,他的双手像是枯萎的树叶。”
    
    “燕子,燕子,小燕子,”王子又说,“你不肯陪我过一夜,做我的信使吗?那个孩子太饿了,他的母亲伤心极了。”
    
    “我觉得自己不喜欢小孩,”燕子回答说,“去年夏天,我到过一条河边,有两个顽皮的孩子,是磨坊主的儿子,他们老是扔石头打我。当然,他们永远也别想打中我,我们飞得多快呀。可不管怎么说,这是不礼貌的行为。”
    
    可是快乐王子满脸愁容,叫小燕子的心里很不好受。“这儿太冷了,”他说,“不过我愿意陪你过上一夜,并做你的信使。”
    
    “谢谢你,小燕子,”王子说。
    
    于是燕子从王子的宝剑上取下那颗硕大的红宝石,用嘴衔着,越过城里一座连一座的屋顶,朝远方飞去。
    
    他飞过大教堂的塔顶,看见了上面白色大理石雕刻的天使像。他飞过王宫,听见了跳舞的歌曲声。一位美丽的姑娘同她的心上人走上了天台。“多么奇妙的星星啊,”他对她说,“多么美妙的爱情啊!”
    
    “我希望我的衣服能按时做好,赶得上舞会,”她回答说,“我已经要求绣上西番莲花,只是裁缝太慢了。”
    
    他飞过河流,看见高挂在船桅上的无数灯笼。他飞过犹太区,看见犹太商人们讨价还价,还把钱币放在铜制的天平上称重量。最后他来到了那个穷人的屋舍,朝里面望去。发烧的孩子在床上辗转反侧,母亲已经睡熟了,因为她太疲倦了。他跳进屋里,将硕大的红宝石放在那女人顶针旁的桌子上。随后他又轻轻地绕着床飞了一圈,用羽翅扇着孩子的前额。“我觉得好凉爽,”孩子说,“我一定是好起来了。”说完就进入了甜蜜的梦乡。
    
    然后,燕子回到快乐王子的身边,告诉他自己做过的一切。“你说怪不怪,”他接着说,“虽然天气很冷,可我现在觉得好暖和。”
    
    “那是因为你做了一件好事,”王子说。于是小燕子开始思考王子的话,不过没多久便睡着了。对燕子来说,一思考问题就会想困觉。
    
    黎明时分他飞下河去洗了个澡。“真是不可思议的现象,”一位鸟禽学教授从桥上走过时开口说道,“冬天竟会有燕子!”于是他就此事给当地的报社写了一封长信。每个人都引用他信中的话,尽管信中的很多术语是人们理解不了的。
    
    “今晚我要到埃及去,”燕子说,一想到远方,他就精神百倍。他走访了城里所有的公共纪念物,还在教堂的顶端上坐了好一阵子。每到一处,麻雀们就吱吱喳喳地相互说,“多么难得的贵客啊!”所以他玩得很开心。
    
    月亮升起的时候他飞回到快乐王子的身边。“你在埃及有什么事要办吗?”他高声问道,“我就要动身了。”
    
    “燕子,燕子,小燕子,”王子说,“你愿意陪我再过一夜吗?”
    
    “伙伴们在埃及等我呀,”燕子回答说,“明天我的朋友们要飞往瀑布,那儿的河马在莎草丛中过夜。古埃及的门农神\footnote{〔门农神〕指位于埃及底比斯(今卢克索)的门农巨像。}安坐在巨大的花岗岩宝座上,他整夜守望着星星,每当星星闪烁的时候,他就发出欢快的叫声,随后便沉默不语。中午时,黄色的狮群下山来到河边饮水,他们的眼睛像绿色的宝石,咆哮起来比瀑布的怒吼还要响亮。”“燕子,燕子,小燕子,”王子说,“远处在城市的那一头,我看见一个住在阁楼中的年轻男子。他在一张铺满纸张的书桌上埋头用功,旁边的玻璃杯中放着一束干枯的紫罗兰。他有一头棕色的卷发,嘴唇红得像石榴,他还有一双睡意朦胧的大眼睛。他正尽力为剧院经理写一个剧本,但是他已经冻得写不下去了。壁炉里没有柴火,饥饿又让他头昏眼花。”
    
    “我愿意陪你再过一夜,”燕子说,他的确有颗善良的心。“我是不是再送他一块红宝石?”
    
    “唉!我现在没有红宝石了。”王子说,“所剩的只有我的双眼。它们是稀有的蓝宝石,是一千多年前从印度出产的。你取出一颗给他送去。他会将它卖给珠宝商,好买回食物和木柴,完成他写的剧本。”
    
    “亲爱的王子,”燕子说,“我不能这样做,”说完就哭了起来。
    
    “燕子,燕子,小燕子,”王子说,“就照我说的话去做吧。”
    
    因此燕子取下了王子的一只眼睛,朝学生住的阁楼飞去了。屋顶上有一个洞,燕子很容易就进去了。燕子穿过洞来到屋里。年轻人双手捂着脸,没有听见燕子翅膀的扇动声,等他抬起头时,正看见那颗美丽的蓝宝石放在干枯的紫罗兰上面。
    
    “我开始受人欣赏了,”他叫道,“这准是某个极其钦佩我的人送来的。现在我可以完成我的剧本了。”他脸上露出了幸福的笑容。
    
    第二天燕子飞到下面的海港,他坐在一艘大船的桅杆上,望着水手们用绳索把大箱子拖出船舱。随着他们“嘿哟!嘿哟!”的声声号子\footnote{〔号子〕劳动时劳动的人一起唱的歌。一人领唱,众人应和,以统一步调减轻疲劳。},一个个大箱子给拖了上来。“我要去埃及了!”燕子嘀咕道,但是没有人理会他。等月亮升起后,他又飞回到快乐王子的身边。
    
    “我是来向你道别的,”他叫着说。
    
    “燕子,燕子,小燕子,”王子说,“你不愿再陪我过一夜吗?”
    
    “冬天到了,”燕子回答说,“寒冷的雪就要来了。而在埃及,太阳挂在葱绿的棕榈树上,暖和极了,还有躺在泥塘中的鳄鱼懒洋洋地环顾着四周。我的朋友们正在巴尔贝克古城\footnote{〔巴尔贝克古城〕位于黎巴嫩,有大量古罗马神殿。}的神庙里建筑巢穴,那些粉红和银白色的鸽子们一边望着他们干活,一边相互倾诉着情话。亲爱的王子,我不得不离你而去了,只是我永远也不会忘记你的,明年春天我要给你带回两颗美丽的宝石,弥补你因送给别人而失掉的。红宝石会比一朵红玫瑰还红,蓝宝石也比大海更蓝。”
    
    “在下面的广场上,”快乐王子说,“坐着一个卖火柴的小女孩。她的火柴都掉在阴沟里了,都不能用了。如果她不带钱回家,她的父亲会打她的,她正在哭着呢。她既没穿鞋,也没有穿袜子,头上什么也没戴。请把我的另一只眼睛取下来,给她送去,这样她父亲就不会揍她了。”
    
    “我愿意陪你再过一夜,”燕子说,“但我不能取下你的眼睛,否则你就变成个瞎子了。”
    
    “燕子,燕子,小燕子,”王子说,“就照我说的话去做吧。”
    
    于是他又取下了王子的另一只眼珠,带着它朝下飞去。他一下子落在小女孩的面前,把宝石悄悄地放在她的手掌心上。“一块多么美丽的玻璃呀!”小女孩高声叫着,她笑着朝家里跑去。
    
    这时,燕子回到王子身旁。“你现在瞎了,”燕子说,“我要永远陪着你。”
    
    “不,小燕子,”可怜的王子说,“你得到埃及去。”
    
    “我要一直陪着你,”燕子说着就睡在了王子的脚下。
    
    第二天他整日坐在王子的肩头上,给他讲自己在异国他乡的所见所闻和种种经历。他还给王子讲那些红色的朱鹭,它们排成长长的一行站在尼罗河的岸边,用它们的尖嘴去捕捉金鱼;还讲到司芬克斯\footnote{〔司芬克斯〕古埃及神话中的怪兽,狮身人面。埃及吉萨有司芬克斯石像。},它的岁数跟世界一样长久,住在沙漠中,通晓世间的一切;他讲那些商人,牵着骆驼缓缓而行,手中摸着玛瑙做的念珠;他讲到月亮山\footnote{〔月亮山〕传说中尼罗河的源头。}的国王,他皮肤黑得像乌木,崇拜一块巨大的水晶;他讲到那条睡在棕榈树上的绿色大蟒蛇,要20个僧侣用蜜糖做的糕点来喂它;他又讲到那些小矮人,他们乘坐扁平的大树叶在湖泊中往来横渡,还老与蝴蝶发生战争。”
    
    “亲爱的小燕子,”王子说,“你为我讲了好多稀奇的事情,可更稀奇的,还是世人所受的苦难。没有什么比苦难更不可思议的了。小燕子,你就到我城市的上空去飞一圈吧,告诉我你在上面都看见了些什么。”,
    
    于是燕子飞过了城市上空,看见富人们在自己漂亮的大楼里寻欢作乐,而乞丐们却坐在大门口忍饥挨饿。他飞进阴暗的小巷,看见饥饿的孩子们露出苍白的小脸没精打采地望着昏暗的街道,两个孩子在一座桥的桥洞里面相互搂抱着,想使彼此温暖一些。“我们好饿呀!”他俩说。“你们不准躺在这儿,”看守高声喊道,两个孩子又蹒跚着朝雨中走去。
    
    随后他飞了回来,把所见的一切告诉给了王子。
    
    “我身上贴满了上好的金片,”王子说,“你把它们一片片取下来,给穷人们送去。活着的人都相信黄金会使他们幸福的。”
    
    燕子将足赤\footnote{〔足赤〕指金子因为纯而显赤红色。}的金叶子一片一片地啄了下来,直到快乐王子变得灰暗无光。他又把这些纯金叶片一一送给了穷人,孩子们的脸上泛起了红晕,他们在大街上欢欣无比地玩着游戏。
    
    “我们现在有面包了!”孩子们喊叫着。
    
    随后下起了雪,白雪过后又迎来了严寒。街道看上去白花花的,像是银子做成的,又明亮又耀眼;长长的冰柱如同水晶做的宝剑垂悬在屋檐下。人人都穿上了皮衣,小孩子们也戴上了皮帽子去户外溜冰。
    
    可怜的小燕子觉得越来越冷了,但是他却不愿离开王子,他太爱这位王子了。他只好趁面包师不注意的时候,从面包店门口弄点面包屑充饥,并扑扇着翅膀为自己取暖。
    
    然而最后他也知道自己快要死去了。他剩下的力气只够再飞到王子的肩上一回。“再见了,亲爱的王子!”他喃喃地说,“你愿意让我亲吻你的手吗?”
    
    “我真高兴你终于要飞往埃及去了,小燕子,”王子说,“你在这儿呆得太长了。不过你得亲我的嘴唇,因为我爱你。”
    
    “我要去的地方不是埃及,”燕子说,“我要去死亡之家。死亡是安眠的兄弟,不是吗?”
    
    他吻了吻快乐王子的嘴唇,然后就跌落在王子的脚下,死去了。
    
    雕像内传出一声奇特的爆响,好像什么东西破碎了。
    
    第二天一早,市长在市议员陪同下,散步来到下面的广场。他们走过圆柱的时候,市长抬头看了一眼雕像。“我的天啊!快乐王子怎么如此难看!”他说。
    
    “真是难看极了!”市议员们异口同声地叫道,他们总跟市长同一个腔调。说完大家纷纷走上前去细看个明白。
    
    “他剑柄上的红宝石已经掉了,蓝宝石眼珠也不见了,他也不再是黄金的了,”市长说,“实际上,他比一个要饭的乞丐强不了多少!”
    
    “的确比要饭的强不了多少,”市议员们附和着说。
    
    “他的脚下还躺着一只死鸟!”市长继续说,“我们真应该发布一个声明,禁止鸟类死在这个地方。”于是市书记员把这个建议记录了下来。
    
    后来他们就把快乐王子的雕像给推倒了。“既然他已不再美丽,也就不再有用了,”大学的美术教授说。
    
    他们把雕像放在炉里熔化了。市长还召集了一次专家会议,决定如何处理这些金属。“当然,我们必须再铸一个雕像。”他说,“那应该就是我的雕像!”
    
    “我的雕像!”每一位市议员都争着说,他们还吵了起来。我最后听到人们说起他们时,他们还没结束争吵。
    
    “多么稀奇古怪的事!”铸像厂的工头说,“这颗破裂的铅心在炉子里熔化不了。我们只好把它扔掉。”他们便把它扔到了垃圾堆里,死去的那只燕子也躺在那儿。
    
    “把城市里最珍贵的两件东西给我拿来,”天主\footnote{〔天主〕基督教对所信仰的神的称呼。}对他的一位天使说。于是天使就把铅心和死鸟带了回来。
    
    “你的选择对极了,”天主说,“在我的花园里,小鸟可以放声歌唱,而快乐王子将不再忧伤。”
    
\end{normalsize}


\newpage

\textbf{注释}:

\vspace{-1em}

\begin{itemize}
    \setlength\itemsep{-0.2em}
    \item 〔硕大〕
    \item 〔熠熠生辉〕
    \item 〔不务实际〕
    \item 〔潦倒〕
    \item 〔嘀咕〕
    \item 〔弥补〕
    \item 〔辗转反侧〕
    \item 〔蹒跚〕
\end{itemize}

\chapter{多收了三五斗}

\begin{normalsize}
    
    万盛米行的河埠头\footnote{〔埠头〕停船的码头。},横七竖八停泊着乡村里出来的敞口船。船里装载的是新米,把船身压得很低。齐船舷的菜叶和垃圾给白腻的泡沫包围着,一漾一漾地,填没了这船和那船之间的空隙。河埠上去是仅容两三个人并排走的街道。万盛米行就在街道的那一边。朝晨的太阳光从破了的明瓦\footnote{〔明瓦〕用蚌壳磨成半透明的薄片当瓦。}天棚斜射下来,光柱子落在柜台外面晃动者的几顶旧毡帽上。
    
    那些戴旧毡帽的大清早摇船出来,到了埠头,气也不透一口,便来到柜台前面占卜\footnote{〔占卜〕这里指探问。}他们的命运。“糙米五块,谷三块,”米行里的先生有气没力地回答他们。
    
    “什么!”旧毡帽朋友几乎不相信自己的耳朵。美满的希望突然一沉,一会儿大家都呆了。
    
    “在六月里,你们不是卖十三块么?”
    
    “十五块也卖过,不要说十三块。”
    
    “哪里有跌得这样利害的!”
    
    “现在是什么时候,你们不知道么?各处的米像潮水一般涌来,过几天还要跌呢!”
    
    刚才出力摇船犹如赛龙船似的一股劲儿,现在在每个人的身体里松懈下来了。今年天照应,雨水调匀,小虫子也不来作梗\footnote{〔作梗〕制造麻烦,捣乱。},一亩田多收这么三五斗,谁都以为该得透一透气了。
    
    哪里知道临到最后的占卜,却得到比往年更坏的课兆\footnote{〔课兆〕这里指苗头。}!
    
    “还是不要粜\footnote{〔粜〕卖出粮食。}的好,我们摇回去放在家里吧!”从简单的心里喷出了这样的愤激的话。
    
    “嗤,”先生冷笑着,“你们不粜,人家就饿死了么?各处地方多的是洋米,洋面,头几批还没吃完,外洋大轮船又有几批运来了。”
    
    洋米,洋面,外洋大轮船,那是遥远的事情,仿佛可以不管。而不粜那已经送到河埠头来的米,却只能作为一句愤激的话说说罢了。怎么能够不粜呢?田主方面的租是要缴的,为了雇帮工,买肥料,吃饱肚皮,借下的债是要还的。
    
    “我们摇到范墓去粜吧,”在范墓,或许有比较好的命运等候着他们,有人这么想。
    
    但是,先生又来了一个“嗤”,捻着稀微的短须说道:“不要说范墓,就是摇到城里去也一样。我们同行公议,这两天的价钱是糙米五块,谷三块。”
    
    “到范墓去粜没有好处,”同伴间也提出了驳议。“这里到范墓要过两个局子\footnote{〔局子〕指厘金局,设在交通要道,对过境商品收厘金。厘金,也叫厘捐,是对商品按价值定比收的税。厘金是晚清民国税收的重要来源。},知道他们捐我们多少钱!就说依他们捐,哪里来的现洋钱?”
    
    “先生,能不能抬高一点?”差不多是哀求的声气。
    
    “抬高一点,说说倒是很容易的一句话。我们这米行是拿本钱来开的,你们要知道,抬高一点,就是说替你们白当差,这样的傻事谁肯干?”
    
    “这个价钱实在太低了,我们做梦也没想到。去年的粜价是七块半,今年的米价又卖到十三块,不,你先生说的,十五块也卖过;我们想,今年总该比七块半多一点吧。哪里知道只有五块!”
    
    “先生,就是去年的老价钱,七块半吧。”
    
    “先生,种田人可怜,你们行行好心,少赚一点吧。”
    
    另一位先生听\footnote{〔听〕指装煤油的洋铁桶。}得厌烦,把嘴里的香烟屁股扔到街心,睁大了眼睛说:“你们嫌价钱低,不要粜好了。是你们自己来的,并没有请你们来。只管多啰嗦做什么!我们有的是洋钱,不买你们的,有别人的好买。你们看,船埠头又有两只船停在那里了。”
    
    三四顶旧毡帽从石级下升上来,旧毡帽下面是表现着希望的酱赤的脸。他们随即加入先到的一群。斜伸下来的光柱子落在他们的破布袄的肩背上。
    
    “听听看,今年什么价钱。”
    
    “比去年都不如,只有五块钱!”伴着一副懊丧到无可奈何的神色。
    
    “什么!”希望犹如肥皂泡,一会儿又迸裂了三四个。
    
    希望的肥皂泡虽然迸裂了,载在敞口船里的米可总得粜出;而且命里注定,只有卖给这一家万盛米行。米行里有的是洋钱,而破布袄的空口袋里正需要洋钱。
    
    在米质好和坏的辩论之中,在斛子\footnote{〔斛子〕量器,方形,口小底大。一斛等于五斗。}浅和满的争持之下,结果船埠头的敞口船真个敞口朝天了;船身浮起了好些,填没了这船那船之间的空隙的菜叶和垃圾就看不见了。旧毡帽朋友把自己种出来的米送进了万盛米行的廒间\footnote{〔廒间〕粮仓。},换到手的是或多或少的一叠钞票。”
    
    “先生,给现洋钱,袁世凯\footnote{〔袁世凯〕指民国初年发行的银元,上面印有袁世凯的头像。},不行么?”白白的米换不到白白的现洋钱,好像又被他们打了个折扣,怪不舒服。
    
    “乡下曲辫子\footnote{〔曲辫子〕嘲笑劳动人民的话。清朝的辫子需要经常打理。这里指劳动人民因为没时间梳理辫子而蓬松、弯曲。}!”夹着一枝水笔的手按在算盘珠上,鄙夷不屑的眼光从眼镜上边射出来,“一块钱钞票就作一块钱用,谁好少作你们一个铜板。我们这里没有现洋钱,只有钞票。”
    
    “那末,换中国银行\footnote{〔中国银行〕1912年创办的银行。因为年份久,在农民中的信誉比国民政府1928年创办的中央银行好。}的吧。”从花纹上辨认,知道手里的钞票不是中国银行的。
    
    “吓!”声音很严厉,左手的食指强硬地指着,“这是中央银行的,你们不要,可是要想吃官司?”
    
    不要这钞票就得吃官司,这个道理弄不明白。但是谁也不想弄明白,大家看了看钞票上的人像,又彼此交换了将信将疑的一眼,便把钞票塞进破布祆的空口袋或者缠着裤腰的空褡裢\footnote{〔褡裢〕一种长口袋,可以装东西,也可以缠在腰间。}。”
    
    一批人咕噜着离开了万盛米行,另一批人又从船埠头跨上来。同样地,在柜台前迸裂了希望的肥皂泡,赶走了入秋以来望着沉重的稻穗所感到的快乐。同样地,把万分舍不得的白白的米送进万盛的廒间,换到了并非白白的现洋钱的钞票。
    
    街道上见得热闹起来了。
    
    旧毡帽朋友今天上镇来,原来有很多的计划的。洋肥皂用完了,须得买十块八块回去。洋火也要带几匣。洋油向挑着担子到村里去的小贩买,十个铜板只有这么一小瓢,太吃亏了;如果几家人家合买一听分来用,就便宜得多。陈列在橱窗里的花花绿绿的洋布听说只要八分半一尺,女人早已眼红了好久,今天粜米就嚷着要一同出来,自己几尺,阿大几尺,阿二几尺,都有了预算。有些女人的预算里还有一面蛋圆的洋镜,一方雪白的毛巾,或者一顶结得很好看的绒线的小囝\footnote{〔小囝〕(浙江方言)小孩子。}帽。难得今年天照应,一亩田多收这么三五斗,让一向捏得紧紧的手稍微放松一点,谁说不应该?缴租,还债,解会钱,大概能够对付过去吧;对付过去之外,大概还有多馀吧。在这样的心境之下,有些人甚至想买一个热水瓶。这东西实在怪,不用生火、热水冲下去,等会儿倒出来照旧是烫的;比起稻柴做成的茶壶窠\footnote{〔茶壶窠〕用棉被包着铁壶保暖的保温瓶,像鸟巢。窠:鸟兽昆虫的巢穴。}来,真是一个在天上,一个在地下。
    
    他们咕噜着离开万盛米行的时候,犹如走出一个一向于己不利的赌场——这回又输了!输多少呢?他们不知道。总之,袋里的一叠钞粟没有半张或者一角是自己的了。还要添补上不知在哪里的多少张钞票给人家,人家才会满意,这要等人家说了才知道。
    
    输是输定了,马上开船回去未必就会好多少,镇上走一转,买点东西回去,也不过在输账上加上一笔,况且有些东西实在等着要用。于是街道上见得热闹起来了。
    
    他们三个一群,五个一簇,拖着短短的身影,在狭窄的街道上走。嘴里还是咕噜着,复算刚才得到的代价,咒骂那黑良心的米行。女人臂弯里钩着篮子,或者一只手牵着小孩,眼光只是向两旁的店家直溜。小孩给赛璐珞\footnote{〔赛璐珞〕一种塑料,用来制造玩具、文具等。}的洋囝囝,老虎,狗,以及红红绿绿的洋铁铜鼓,洋铁喇叭勾引住了,赖在那里不肯走开。
    
    “小弟弟,好玩呢,洋铜鼓,洋喇叭,买一个去,”故意作一种引诱的声调。接着是——冬,冬,冬——叭,叭,叭。
    
    当,当,当——“洋瓷面盆刮刮叫,四角一只真公道,乡亲,带一只去吧。”
    
    “喂,乡亲,这里有各色花洋布,特别大减价,八分五一尺,足尺加三,要不要剪些回去?”
    
    万源祥大利老福兴几家的店伙特别卖力,不惜工本叫着“乡亲”,同时拉拉扯扯地牵住“乡亲”的布袄,他们知道惟有今天,“乡亲”的口袋是充实的,这是不容放过的好机会。
    
    在节约预算的踌躇之后,“乡亲”把刚到手的钞票一张两张地交到店伙手里。洋火,洋肥皂之类必需用,不能不买,只好少买一点。整听的洋油价钱太“咬手”,不买吧,还是十个铜板一小瓢向小贩零沽。衣料呢,预备剪两件的就剪了一件,预备娘儿俩一同剪的就单剪了儿子的。蛋圆的洋镜拿到了手里又放进了橱窗。绒线的帽子套在小孩头上试戴,刚刚合式,给爷老子\footnote{〔爷老子〕父亲。}一句“不要买吧”,便又脱了下来。想买热水瓶的简直不敢问一声价。说不定要一块半吧。如果不管三七二十一买回去,别的不说,几个白头发的老太公老太婆就要一阵阵地骂:“这样的年时,你们贪安逸,花了一块半买这些东西来用,永世不得翻身是应该的!你们看,我们这么一把年纪,谁用过这些东西来!”这啰嗦也就够受了。有几个女人拗不过\footnote{〔拗不过〕无法改变(别人的)意见。}孩子的欲望,便给他们买了最便宜的小洋囝囝。小洋囝囝的腿臂可以转动,要他坐就坐,要他站就站,要他举手就举手;这不但使拿不到手的别的孩子眼睛里几乎冒火,就是大人看了也觉得怪有兴趣。
    
    “乡亲”还沾了一点酒,向熟肉店里买了一点肉,回到停泊在万盛米行船埠头的自家的船上,又从般梢头拿出盛着咸菜和豆腐汤之类的碗碟来,便坐在船头开始喝酒。女人在船梢头煮饭。一会儿,这条船也冒烟,那条船也冒烟,个个人淌着眼泪。小孩在敞口朝天的空舱里跌交打滚,又捞起浮在河面的脏东西来玩,惟有他们有说不出的快乐。
    
    酒到了肚里,话就多起来。相识的,不相识的,落在同一的命运里,又在同一的河面上喝酒,你端起酒碗来说几句,我放下筷子来接几声,中听的,喊声“对”,不中听,骂一顿:大家觉得正需要这样的发泄。
    
    “五块钱一担\footnote{〔一担〕十斗。},真是碰见了鬼!”
    
    “去年是水灾,收成不好,亏本。今年算是好年时,收成好,还是亏本!”
    
    “今年亏本比去年都厉害;去年还粜七块半呢。”
    
    “又得把自己吃的米粜出去了。唉,种田人吃不到自己种出来的米!”
    
    “为什么要粜出去呢,你这死鬼!我一定要留在家里,给老婆吃,给儿子吃。我不缴租,宁可跑去吃官司,让他们关起来!”
    
    “也只好不缴租呀。缴租立刻借新债。借了四分钱五分钱\footnote{〔四分钱五分钱〕指每月百分之四、五利息的高利贷。}的债去缴租,贪图些什么,难道贪图明年背着重重的债!” 
    
    “田真个种不得了!”
    
    “退了租逃荒去吧。我看逃荒的倒是满写意\footnote{〔写意〕(江浙方言)舒心,开心。}的。”
    
    “逃荒去,债也赖了,会钱也不用解了,好打算,我们一块儿去!”
    
    “谁出来当头脑?他们逃荒的有几个头脑,男男女女,老老小小,都听头脑的话。”
    
    “我看,到上海去做工也不坏。我们村里的小王,不是么?在上海什么厂里做工,听说一个月工钱有十五块。十五块,照今天的价钱,就是三担米呢!”
    
    “你翻什么隔年旧历本!上海东洋人打仗\footnote{〔上海东洋人打仗〕指1932年淞沪战争。},好多的厂关了门,小王在那里做叫化子了,你还不知道?”
    
    路路断绝。一时大家沉默了。酱赤的脸受着太阳光又加上酒力,个个难看不过,好像就会有殷红的血从皮肤里迸出来似的。
    
    “我们年年种田,到底替谁种的?”一个人呷了一口酒,幽幽地\footnote{〔幽幽地〕低沉、轻柔地。}提出疑问。
    
    就有另一个人指着万盛的半新不旧的金字招牌说:“近在眼前,就是替他们种的。
    
    我们吃辛吃苦,赔重利钱借债,种了出来,他们嘴唇皮一动,说‘五块钱一担!’就把我们的油水一古脑儿吞了去!”
    
    “要是让我们自己定价钱,那就好了。凭良心说,八块钱一担,我也不想多要。”
    
    “你这囚犯,在那里做什么梦!你不听见么?他们米行是拿本钱来开的,不肯替我们白当差。”
    
    “那么,我们的田也是拿本钱来种的,为什么要替他们白当差!为什么要替田主白当差!”
    
    “我刚才在廒间里这么想:现在让你们沾便宜,米放在这里;往后没得吃,就来吃你们的!”故意把声音压得很低,网着红丝的眼睛向岸上斜溜。
    
    “真个没得吃的时候,什么地方有米,拿点来吃是不犯王法的!”理直气壮的声口。
    
    “今年春天,丰桥地方不是闹过抢米么?”
    
    “保卫团\footnote{〔保卫团〕地方豪绅组织的反动武装。}开了枪,打死两个人。”
    
    “今天在这里的,说不定也会吃枪,谁知道!”
    
    散乱的谈话当然没有什么议决案。酒喝干了,饭吃过了,大家开船回自己的乡村。
    
    船埠头便冷清清地荡漾着暗绿色的脏水。
    
    第二天又有一批敞口船来到这里停泊。镇上便表演着同样的故事。这种故事也正在各处市镇上表演着,真是平常而又平常的。
    
\end{normalsize}


\newpage

\textbf{注释}:

\vspace{-1em}

\begin{itemize}
    \setlength\itemsep{-0.2em}
    \item 〔鄙夷〕轻视,看不起。
    \item 〔沽〕买(酒等液体)。
    \item 〔呷〕小口地喝。
    \item 〔不屑〕认为不值得。也形容轻视。
    \item 〔踌躇〕犹豫不决,思前想后。
    \item 〔懊丧〕懊恼沮丧。
    \item 〔缴〕交出,交付。
\end{itemize}

\chapter{关于自由落体的讨论}

\begin{normalsize}
    
    \begin{description}[itemsep=1ex,leftmargin=4.5em,labelwidth=4em]
    
    \item[{\color{script-2-0} 心朴理休}]据我所知,亚里士多德反对古人所谓“先有真空再有运动”的说法。他认为运动排斥真空。他的依据有二。他首先讨论同一介质\footnote{〔介质〕一种物质存在于另一种物质之中,后者就是前者的介质。}中,重量不同的物体自由运动\footnote{〔自由运动〕即自由落体。古人认为这是不施加任何外力时物体的自由运动方式。}的情况;再讨论同一物体在不同介质中自由运动的情况。第一种情况中,他认为物体在同一介质中,自由运动的速度与其重量成比例。也就是说,若一物重量十倍于另一物,则其自由运动的速度也十倍于它。第二种情况中,他认为同一物体在不同介质中自由运动的速度,与介质的疏密成反比例。举例来说,若水十倍密于空气,则物体在空气中自由运动的速度十倍于水中。自第二点出发,他论证道:既然真空无限疏于任何介质,物体在真空中自由运动的速度,也应该是它在任何介质中运动速度的无限倍。因此,物体穿过真空不用任何时间。然而,不用任何时间就穿过一段距离是不可能的。因此,运动的存在就否定了真空。
    
    \item[{\color{script-2-1} 萨为亚第}]如您所见,这段论述有所针对。运动排斥真空,并非绝对否定真空的存在。即便如此,为了更好地理解亚里士多德这段论证是否有说服力,我想对您谈谈古代先贤对此可能的回应。我们会发现,他的两个前提,都应当被否定。对于第一个假设,我十分怀疑,亚里士多德是否实地做过验证。如果让两块石头——其中一块是另一块的十倍重——同时从一百尺的高度落下,它们的速度是否会相差如此之大,以至于较重的石块落地时,较轻的石块只下落了十尺。
    
    \item[{\color{script-2-0} 心朴理休}]从亚里士多德的原话看来,我们可以认为他做过这样的实验。
    
    \item[{\color{script-2-2} 撒格列托}]但是,心朴理休,我也做过实验。我可以向您保证,如果将一颗一两百磅甚至更重的炮弹从两百尺的高度落下,它绝不会比一颗火枪子弹快上一尺。
    
    \item[{\color{script-2-1} 萨为亚第}]即便尚未做进一步检验,我们也可以用简明的方法清晰地证明,在亚里士多德所述的条件下,重物自由下落时不会比轻物更快。至少对同一种物质如此。心朴理休,请告诉我,您是否承认,自由落体的速度取决于自然,而没有其他的干扰?
    
    \item[{\color{script-2-0} 心朴理休}]毫无疑问,同一物体在单纯的介质中自由运动,其速度取决于自然,不会变化,除非受到额外的阻力和推力\footnote{〔同一物体……〕这里是说用同一物体重复做实验,结果不变;而不是指同一次实验中,物体下落速度不变。}。
    
    \item[{\color{script-2-1} 萨为亚第}]如果两个轻重不等的物体,其取决于自然的速度不相等同,我们将两者绑定,那么,速度更快的会被速度更慢的拖累,速度更慢的会被速度更快的促进。您是否同意这个结论?
    
    \item[{\color{script-2-0} 心朴理休}]您无疑是正确的。
    
    \item[{\color{script-2-1} 萨为亚第}]但是,如果这是正确的,两个物体连合之后,其整体自由下落的速度将小于较重物体的速度。然而,连合后的整体,必然比其任一部分更重。因此,连合后自由下落的速度,也应当比其任一部分要快。也就是说,应当比较重物体更快。所以,要么我之前说的不正确,要么较重的物体下落更慢。又或者说,我从您给的假设:“越重的物体自由运动速度越快”出发,证明了“较重的物体自由运动速度更慢”。
    
    \item[{\color{script-2-0} 心朴理休}]您让我身处汪洋大海之中了。既然如此,如何才是正确的呢?
    
    \item[{\color{script-2-1} 萨为亚第}]如果我们假设轻重不等的物体自由下落,其取决于自然的速度也相等,就不会有前后的矛盾了。
    
    \item[{\color{script-2-0} 心朴理休}]您的论证让我佩服,可我仍不敢相信,一块鸟粪下落的速度与一颗炮弹相同。
    
    \item[{\color{script-2-1} 萨为亚第}]我相信您不会像某些人一样,把讨论的重点从主要的地方移开,而抓着一些细微的地方不放。被丝毫的差错吸引,忽视了船缆一般的真相。亚里士多德说,一百磅的铁球从一百尺高处下落,一磅的铁球从一尺高处下落,两个铁球同时到达。我说:不对;一百磅的铁球和一磅的铁球从一百尺高处一齐下落,两个铁球同时到达。您在做实验的时候发现,重球落地时,轻球还相差几个手指的宽度。您不会指责我犯了几个手指宽度的错误,却对亚里士多德的九十九尺保持沉默。当然,金子打成箔片,石头磨成齑粉,也许会漂浮在空中。但去掉这样显然的影响,从亚里士多德的主张出发,您需要证明,二十磅的石头,下落的速度比两磅的石头快十倍。而我说这是错误的。如果他们从五十尺或一百尺高处落下,他们将同时到达地面。
    
    \item[{\color{script-2-0} 心朴理休}]即便这个前提是错的,亚里士多德仍然有另一个前提。而我想另一个前提才是更重要的。
    
    \item[{\color{script-2-1} 萨为亚第}]但您看,另一个前提也同样是错误的。亚里士多德认为,如果水比空气密实十倍,那么同一物体在空气中自由下落的速度是水中的十倍。然而很多物体在空气中自由下落,却浮在水上,根本不下落。这比例的关系如何成立呢?
    
    \item[{\color{script-2-0} 心朴理休}]我想,亚里士多德仅仅讨论那些在水中和空气中同样下落的物体。
    
    \item[{\color{script-2-1} 萨为亚第}]就算如此,让我们仅仅讨论这样的物体。请告诉我您认为水与空气应该有怎样的疏密关系,请按您的意愿定下一个数字。
    
    \item[{\color{script-2-0} 心朴理休}]这样的比值必然存在。我们姑且认为是十倍吧。同一个物体在空气与水中自由下落,在前者中的速度应当是后者中的十倍。
    
    \item[{\color{script-2-1} 萨为亚第}]想象一个木球。您提到木头在水中不下落,但木头在空气中下落。请你为它指定一个下落的速度。
    
    \item[{\color{script-2-0} 心朴理休}]我们姑且认为它以二十的速度运动。
    
    \item[{\color{script-2-1} 萨为亚第}]那么,按照亚里士多德的观点,它在水中应当以二的速度下落。
    
    \item[{\color{script-2-0} 心朴理休}]我想亚里士多德的观点并不针对这种情况。
    
    \item[{\color{script-2-1} 萨为亚第}]但是,我想你会同意我的观点,我们可以找到另一种材料(不是木头)的球,它确实以二的速度在水中下落。
    
    \item[{\color{script-2-0} 心朴理休}]毫无疑问,我们可以找到这样的材料,它必定比木球重得多。
    
    \item[{\color{script-2-1} 萨为亚第}]正是如此!现在,让我们想象这个材料制作的球在空气中下落。按照亚里士多德的观点,它下落的速度应当也是二十。如此,它就以和木球相等的速度在空气中下落。但正如您所说,它必定比木球重得多。现在,亚里士多德如何将这个结论与他的另一个观点相协调,即不同重量的物体以不同的速度在同一介质中自由下落?但是,如果不深入研究这个问题,这些明显的共性是如何被您忽视的呢?您没有注意过吗?两个物体落入水中,一个物体的速度是另一个的一百倍,它们在空气中下落的速度却几乎相等,相差不会超过百分之一?例如,鸡蛋大小的大理石块,在水中下沉的速度比鸡蛋快一百倍。而在空气中,同样从二十尺高处下落,大理石比鸡蛋慢不到四指宽。一些物体在水中下沉十尺需要三小时,可在空中只需要一两个脉搏;如果这个物体是一个铅球,它穿过十尺深的水所需的时间,不到穿过十尺深空气所需时间的两倍。我敢肯定,心朴理休,您在这里找不到任何反对的理由。
    
    \end{description}
    
    \noindent $\triangleright$~心朴理休陷入了沉默。
    
    
\end{normalsize}


\newpage

\textbf{注释}:

\vspace{-1em}

\begin{itemize}
    \setlength\itemsep{-0.2em}
    \item 〔箔片〕通常是纯金属的极薄的薄片。
    \item 〔齑粉〕细粉,粉末。
    \item 〔姑且〕用于暂时的目的,满足暂时的需要。表示一种让步。
    \item 〔额外〕超出定额或范围。其他的。
\end{itemize}

\chapter{断章}

\begin{normalsize}
    
    \begin{verse}[0.5\linewidth]
        你站在桥上看风景, \\
        看风景人在楼上看你。 \\
        明月装饰了你的窗子, \\
        你装饰了别人的梦。
    \end{verse}
    
\end{normalsize}



\chapter{乡愁}

\begin{normalsize}
    
    \begin{verse}[0.5\linewidth]
        小时候 \\
        乡愁是一枚小小的邮票 \\
        我在这头 \\
        母亲在那头
    \end{verse}
    
    
    \begin{verse}[0.5\linewidth]
        长大后 \\
        乡愁是一张窄窄的船票 \\
        我在这头 \\
        新娘在那头
    \end{verse}
    
    
    \begin{verse}[0.5\linewidth]
        后来啊 \\
        乡愁是一方矮矮的坟墓 \\
        我在外头 \\
        母亲在里头
    \end{verse}
    
    
    \begin{verse}[0.5\linewidth]
        而现在 \\
        乡愁是一湾浅浅的海峡 \\
        我在这头 \\
        大陆在那头
    \end{verse}
    
\end{normalsize}



\chapter{错误}

\begin{normalsize}
    
    \begin{verse}[0.5\linewidth]
        我打江南走过 \\
        那等在季节里的容颜如莲花的开落 \\
        东风不来,三月的柳絮不飞 \\
        你底心如小小寂寞的城 \\
        恰若青石的街道向晚 \\
        音不响,三月的春帷不揭 \\
        你底心是小小的窗扉紧掩 \\
        我达达的马蹄是美丽的错误 \\
        我不是归人,是个过客……
    \end{verse}
    
\end{normalsize}



\chapter{我用残损的手掌}

\begin{normalsize}
    
    \begin{verse}[0.5\linewidth]
        我用残损的手掌 \\
        摸索这广大的土地: \\
        这一角已变成灰烬, \\
        那一角只是血和泥; \\
        这一片湖该是我的家乡, \\
        (春天,堤上繁花如锦障, \\
        嫩柳枝折断有奇异的芬芳) \\
        我触到荇藻和水的微凉; \\
        这长白山的雪峰冷到彻骨, \\
        这黄河的水夹泥沙在指间滑出; \\
        江南的水田,你当年新生的禾草 \\
        是那么细,那么软……现在只有蓬蒿; \\
        岭南的荔枝花寂寞地憔悴,尽那边, \\
        我蘸着南海没有渔船的苦水…… \\
        无形的手掌掠过无限的江山, \\
        手指沾了血和灰,手掌粘了阴暗, \\
        只有那辽远的一角依然完整, \\
        温暖,明朗,坚固而蓬勃生春。 \\
        在那上面,我用残损的手掌轻抚, \\
        像恋人的柔发,婴孩手中乳。 \\
        我把全部的力量运在手掌,贴在上面, \\
        寄予爱和一切希望, \\
        因为只有那里是太阳,是春, \\
        将驱逐阴暗,带来苏生, \\
        因为只有那里我们不像牲口一样活, \\
        蝼蚁一样死……那里,永恒的中国!
    \end{verse}
    
\end{normalsize}



\chapter{季节就这样逝去}

\begin{normalsize}
    
    \begin{verse}[0.5\linewidth]
        季节就这样逝去了 \\
        影子变长了,蓝天深了 \\
        清新的风,拂过山冈 \\
        鸟儿打颤,草儿也觉凉
    \end{verse}
    
    
    \begin{verse}[0.5\linewidth]
        八月和九月仍在纠缠 \\
        浪涛打破了海的安详 \\
        每个白昼都变得更短 \\
        每次朝霞都少一丝光
    \end{verse}
    
    
    \begin{verse}[0.5\linewidth]
        苍蝇,仿佛入了罗网 \\
        静静地停在天花板上 \\
        似一片洁白的雪花 \\
        一点一滴,夏天在融化
    \end{verse}
    
\end{normalsize}



\chapter{故都的秋}

\begin{normalsize}
    
    秋天,无论在什么地方的秋天,总是好的;可是啊,北国的秋,却特别地来得清,来得静,来得悲凉。我的不远千里,要从杭州赶上青岛,更要从青岛赶上北平来的理由,也不过想饱尝一尝这“秋”,这故都的秋味。
    
    江南,秋当然也是有的,但草木凋得慢,空气来得润,天的颜色显得淡,并且又时常多雨而少风;一个人夹在苏州上海杭州,或厦门香港广州的市民中间,混混沌沌地过去,只能感到一点点清凉,秋的味,秋的色,秋的意境与姿态,总看不饱,尝不透,赏玩不到十足。秋并不是名花,也并不是美酒,那一种半开、半醉的状态,在领略秋的过程上,是不合适的。
    
    不逢北国之秋,已将近十余年了。在南方每年到了秋天,总要想起陶然亭\footnote{〔陶然亭〕位于北京城南,亭名出自白居易诗句“更待菊黄家酿熟,共君一醉一陶然”。}的芦花,钓鱼台\footnote{〔钓鱼台〕在北京阜成门外三里河,玉渊潭公园北面。}的柳影,西山\footnote{〔西山〕北京西郊群山的总称,是京郊名胜。}的虫唱,玉泉\footnote{〔玉泉〕指玉泉山,是西山东麓支脉。}的夜月,潭柘寺\footnote{〔潭柘寺〕在北京西山,相传“寺址本在青龙潭上,有古柘千章,寺以此得名”。}的钟声。在北平即使不出门去吧,就是在皇城人海之中,租人家一椽\footnote{〔一椽〕一间屋。椽,放在房檩上架着木板或瓦的木条。}破屋来住着,早晨起来,泡一碗浓茶,向院子一坐,你也能看得到很高很高的碧绿的天色,听得到青天下驯鸽的飞声。从槐树叶底,朝东细数着一丝一丝漏下来的日光,或在破壁腰中,静对着像喇叭似的牵牛花(朝荣)的蓝朵,自然而然地也能够感觉到十分的秋意。说到了牵牛花,我以为以蓝色或白色者为佳,紫黑色次之,淡红色最下。最好,还要在牵牛花底,教长着几根疏疏落落的尖细且长的秋草,使作陪衬。
    
    北国的槐树,也是一种能使人联想起秋来的点缀。像花而又不是花的那一种落蕊,早晨起来,会铺得满地。脚踏上去,声音也没有,气味也没有,只能感出一点点极微细极柔软的触觉。扫街的在树影下一阵扫后,灰土上留下来的一条条扫帚的丝纹,看起来既觉得细腻,又觉得清闲,潜意识下并且还觉得有点儿落寞,古人所说的梧桐一叶而天下知秋\footnote{〔梧桐一叶而天下知秋〕《淮南子·说山训》〕“以小明大,见叶落而知岁之将暮。”《太平御览》卷二十四引用“一叶落而知天下秋”。}的遥想,大约也就在这些深沉的地方。
    
    秋蝉的衰弱的残声,更是北国的特产,因为北平处处全长着树,屋子又低,所以无论在什么地方,都听得见它们的啼唱。在南方是非要上郊外或山上去才听得到的。这秋蝉的嘶叫,在北方可和蟋蟀耗子一样,简直像是家家户户都养在家里的家虫。
    
    还有秋雨哩,北方的秋雨,也似乎比南方的下得奇,下得有味,下得更像样。
    
    在灰沉沉的天底下,忽而来一阵凉风,便息列索落地下起雨来了。一层雨过,云渐渐地卷向了西去,天又晴了,太阳又露出脸来了,穿着很厚的青布单衣或夹袄的都市闲人,咬着烟管,在雨后的斜桥影里,上桥头树底下去一立,遇见熟人,便会用了缓慢悠闲的声调,微叹着互答着地说:
    
    “唉,天可真凉了——”(这了字念得很高,拖得很长。)
    
    “可不是吗?一层秋雨一层凉了!”
    
    北方人念阵字,总老像是层字,平平仄仄起来\footnote{〔平平仄仄起来〕意即推敲起字的韵律来。},这念错的歧韵,倒来得正好。
    
    北方的果树,到秋天,也是一种奇景。第一是枣子树,屋角,墙头,茅房边上,灶房门口,它都会一株株地长大起来。像橄榄又像鸽蛋似的这枣子颗儿,在小椭圆形的细叶中间,显出淡绿微黄的颜色的时候,正是秋的全盛时期,等枣树叶落,枣子红完,西北风就要起来了,北方便是沙尘灰土的世界,只有这枣子、柿子、葡萄,成熟到八九分的七八月之交,是北国的清秋的佳日,是一年之中最好也没有的“golden days”\footnote{〔golden days〕英语“黄金般的日子”,指一年中最好的时光。}。
    
    有些批评家说,中国的文人学士,尤其是诗人,都带着很浓厚的颓废的色彩,所以中国的诗文里,赞颂秋的文字的特别的多。但外国的诗人,又何尝不然?我虽则外国诗文念的不多,也不想开出帐来,做一篇秋的诗歌散文钞\footnote{〔钞〕同“抄”。},但你若去一翻英德法意等诗人的集子,或各国的诗文的“anthology”\footnote{〔anthology〕英语“选集”,指编辑自行选择内容组成的合集。}来,总能够看到许多关于秋的歌颂和悲啼。各著名的大诗人的长篇田园诗或四季诗里,也总以关于秋的部分,写得最出色而最有味。足见有感觉的动物,有情趣的人类,对于秋,总是一样地特别能引起深沉,幽远、严厉、萧索的感触来的。不单是诗人,就是被关闭在牢狱里的囚犯,到了秋天,我想也一定能感到一种不能自已的深情,秋之于人,何尝有国别,更何尝有人种阶级的区别呢?不过在中国,文字里有一个“秋士”\footnote{〔秋士〕古时指到了暮年仍不得志的知识分子。}的成语,读本里又有着很普遍的欧阳子的《秋声》\footnote{〔欧阳子的《秋声》〕指欧阳修的《秋声赋》。}与苏东坡的《赤壁赋》等,就觉得中国的文人,与秋的关系特别深了,可是这秋的深味,尤其是中国的秋的深味,非要在北方,才感受得到底。
    
    南国之秋,当然也是有它的特异的地方的,比如廿四桥的明月,钱塘江的秋潮,普陀山的凉雾,荔枝湾\footnote{〔荔枝湾〕位于广州城西。}的残荷等等,可是色彩不浓,回味不永。比起北国的秋来,正像是黄酒之与白干,稀饭之与馍馍,鲈鱼之与大蟹,黄犬之与骆驼。
    
    秋天,这北国的秋天,若留得住的话,我愿把寿命的三分之二折去,换得一个三分之一的零头。
    
    \hfill 一九三四年八月在北平
    
\end{normalsize}


\newpage

\textbf{注释}:

\vspace{-1em}

\begin{itemize}
    \setlength\itemsep{-0.2em}
    \item 〔颓废〕意志消沉,精神萎靡,思想消极。
    \item 〔落寞〕冷落,寂寞。
\end{itemize}

\chapter{荷塘月色}

\begin{normalsize}
    
    这几天心里颇不宁静。今晚在院子里坐着乘凉,忽然想起日日走过的荷塘,在这满月的光里,总该另有一番样子吧。月亮渐渐地升高了,墙外马路上孩子们的欢笑,已经听不见了;妻在屋里拍着闰儿,迷迷糊糊地哼着眠歌。我悄悄地披了大衫,带上门出去。
    
    沿着荷塘,是一条曲折的小煤屑路。这是一条幽僻的路;白天也少人走,夜晚更加寂寞。荷塘四面,长着许多树,蓊蓊郁郁的。路的一旁,是些杨柳,和一些不知道名字的树。没有月光的晚上,这路上阴森森的,有些怕人。今晚却很好,虽然月光也还是淡淡的。
    
    路上只我一个人,背着手踱着。这一片天地好像是我的;我也像超出了平常的自己,到了另一世界里。我爱热闹,也爱冷静;爱群居,也爱独处。像今晚上,一个人在这苍茫的月下,什么都可以想,什么都可以不想,便觉是个自由的人。白天里一定要做的事,一定要说的话,现在都可不理。这是独处的妙处,我且受用这无边的荷香月色好了。
    
    曲曲折折的荷塘上面,弥望的是田田的叶子。叶子出水很高,像亭亭的舞女的裙。层层的叶子中间,零星地点缀着些白花,有袅娜地开着的,有羞涩地打着朵儿的;正如一粒粒的明珠,又如碧天里的星星,又如刚出浴的美人。微风过处,送来缕缕清香,仿佛远处高楼上渺茫的歌声似的。这时候叶子与花也有一丝的颤动,像闪电般,霎时传过荷塘的那边去了。叶子本是肩并肩密密地挨着,这便宛然有了一道凝碧的波痕。叶子底下是脉脉的流水,遮住了,不能见一些颜色;而叶子却更见风致了。
    
    月光如流水一般,静静地泻在这一片叶子和花上。薄薄的青雾浮起在荷塘里。叶子和花仿佛在牛乳中洗过一样;又像笼着轻纱的梦。虽然是满月,天上却有一层淡淡的云,所以不能朗照;但我以为这恰是到了好处——酣眠固不可少,小睡也别有风味的。月光是隔了树照过来的,高处丛生的灌木,落下参差的斑驳的黑影,峭楞楞如鬼一般;弯弯的杨柳的稀疏的倩影,却又像是画在荷叶上。塘中的月色并不均匀;但光与影有着和谐的旋律,如梵婀玲\footnote{〔梵婀玲〕小提琴。}上奏着的名曲。
    
    荷塘的四面,远远近近,高高低低都是树,而杨柳最多。这些树将一片荷塘重重围住;只在小路一旁,漏着几段空隙,像是特为月光留下的。树色一例是阴阴的,乍看像一团烟雾;但杨柳的丰姿,便在烟雾里也辨得出。树梢上隐隐约约的是一带远山,只有些大意罢了。树缝里也漏着一两点路灯光,没精打采的,是渴睡人的眼。这时候最热闹的,要数树上的蝉声与水里的蛙声;但热闹是它们的,我什么也没有。
    
    忽然想起采莲的事情来了。采莲是江南的旧俗,似乎很早就有,而六朝时为盛;从诗歌里可以约略知道。采莲的是少年的女子,她们是荡着小船,唱着艳歌去的。采莲人不用说很多,还有看采莲的人。那是一个热闹的季节,也是一个风流的季节。梁元帝《采莲赋》\footnote{〔《采莲赋》〕南朝梁元帝萧绎写的小赋,描写采莲少女欢笑嬉戏的场景。}里说得好:
    
    \begin{quotation}
    
    于是妖童媛女,荡舟心许;鹢首徐回,兼传羽杯;櫂将移而藻挂,船欲动而萍开。尔其纤腰束素,迁延顾步;夏始春余,叶嫩花初,恐沾裳而浅笑,畏倾船而敛裾。
    
    \end{quotation}
    
    可见当时嬉游的光景了。这真是有趣的事,可惜我们现在早已无福消受了。
    
    于是又记起《西洲曲》\footnote{〔《西洲曲》〕南朝乐府民歌,描写一位少女从初春到深秋,从现实到梦境,对爱人的苦苦思念。}里的句子:
    
    \begin{verse}[0.5\linewidth]
    
    采莲南塘秋,莲花过人头。\\低头弄莲子,莲子清如水。
    
    \end{verse}
    
    今晚若有采莲人,这儿的莲花也算得“过人头”了;只不见一些流水的影子,是不行的。这令我到底惦着江南了。——这样想着,猛一抬头,不觉已是自己的门前;轻轻地推门进去,什么声息也没有,妻已睡熟好久了。
    
    \hfill 1927年7月北京清华园
    
\end{normalsize}


\newpage

\textbf{注释}:

\vspace{-1em}

\begin{itemize}
    \setlength\itemsep{-0.2em}
    \item 〔弥望〕满眼。
    \item 〔蓊蓊郁郁〕草木蓬勃茂盛的样子
    \item 〔袅娜〕形容草或枝条细长柔软。
    \item 〔霎时〕极短的时间。
    \item 〔凝碧〕浓绿。唐代皇家公园中有凝碧池。
    \item 〔酣眠〕沉睡,熟睡。
    \item 〔脉脉〕水无声流动的样子。
    \item 〔风致〕风味,美好的样子。
    \item 〔妖童媛女〕艳丽的少女。妖:妩媚,艳丽。媛:美貌。
    \item 〔峭楞楞〕寂然无声地直立着。
    \item 〔鹢首〕船头。鹢:一种水鸟,似鹭而更大。古时船首常画有鹢,所以用鹢指代船头。
    \item 〔櫂〕棹,船桨。
    \item 〔迁延〕徘徊。
    \item 〔惦〕记挂,挂念。
\end{itemize}

\chapter{反对党八股}

\begin{normalsize}
    
    党八股在我们党内已经有了一个长久的历史;特别是在土地革命时期\footnote{〔土地革命时期〕指1927年8月到1937年7月的十年时期。},有时竟闹得很严重。
    
    从历史来看,党八股是对于五四运动的一个反动。
    
    五四运动时期,一班新人物反对文言文,提倡白话文,反对旧教条,提倡科学和民主,这些都是很对的。在那时,这个运动是生动活泼的,前进的,革命的。那时的统治阶级都拿孔夫子\footnote{〔孔夫子〕指孔子。民间对孔子的称呼。}的道理教学生,把孔夫子的一套当作宗教教条一样强迫人民信奉,做文章的人都用文言文。总之,那时统治阶级及其帮闲者们\footnote{〔帮闲〕有钱有势的人雇养来为自己说话的文人。}的文章和教育,不论它的内容和形式,都是八股式的,教条式的。这就是老八股、老教条。揭穿这种老八股、老教条的丑态给人民看,号召人民起来反对老八股、老教条,这就是五四运动时期的一个极大的功绩。五四运动还有和这相联系的反对帝国主义的大功绩;这个反对老八股、老教条的斗争,也是它的大功绩之一。但到后来就产生了洋八股、洋教条。我们党内的一些违反了马克思主义的人则发展这种洋八股、洋教条,成为主观主义、宗派主义和党八股的东西。这些就都是新八股、新教条。这种新八股、新教条,在我们许多同志的头脑中弄得根深蒂固,使我们今天要进行改造工作还要费很大的气力。这样看来,“五四”时期的生动活泼的、前进的、革命的、反对封建主义的老八股、老教条的运动,后来被一些人发展到了它的反对方面,产生了新八股、新教条。它们不是生动活泼的东西,而是死硬的东西了;不是前进的东西,而是后退的东西了;不是革命的东西,而是阻碍革命的东西了。这就是说,洋八股或党八股,是五四运动本来性质的反动。但五四运动本身也是有缺点的。那时的许多领导人物,还没有马克思主义的批判精神,他们使用的方法,一般地还是资产阶级的方法,即形式主义的方法。他们反对旧八股、旧教条,主张科学和民主,是很对的。但是他们对于现状,对于历史,对于外国事物,没有历史唯物主义的批判精神,所谓坏就是绝对的坏,一切皆坏;所谓好就是绝对的好,一切皆好。这种形式主义地看问题的方法,就影响了后来这个运动的发展。五四运动的发展,分成了两个潮流。一部分人继承了五四运动的科学和民主的精神,并在马克思主义的基础上加以改造,这就是共产党人和若干党外马克思主义者所做的工作。另一部分人则走到资产阶级的道路上去,是形式主义向右的发展。但在共产党内也不是一致的,其中也有一部分人发生偏向,马克思主义没有拿得稳,犯了形式主义的错误,这就是主观主义、宗派主义和党八股,这是形式主义向“左”的发展。这样看来,党八股这种东西,一方面是五四运动的积极因素的反动,一方面也是五四运动的消极因素的继承、继续或发展,并不是偶然的东西。我们懂得这一点是有好处的。如果“五四”时期反对老八股和老教条主义是革命的和必需的,那末,今天我们用马克思主义来批判新八股和新教条主义也是革命的和必需的。如果“五四”时期不反对老八股和老教条主义,中国人民的思想就不能从老八股和老教条主义的束缚下面获得解放,中国就不会有自由独立的希望。这个工作,五四运动时期还不过是一个开端,要使全国人民完全脱离老八股和老教条主义的统治,还须费很大的气力,还是今后革命改造路上的一个大工程。如果我们今天不反对新八股和新教条主义,则中国人民的思想又将受另一个形式主义的束缚。至于我们党内一部分(当然只是一部分)同志所中的党八股的毒,所犯的教条主义的错误,如果不除去,那末,生动活泼的革命精神就不能启发,拿不正确态度对待马克思主义的恶习就不能肃清,真正的马克思主义就不能得到广泛的传播和发展;而对于老八股和老教条在全国人民中间的影响,以及洋八股和洋教条在全国许多人中间的影响,也就不能进行有力的斗争,也就达不到加以摧毁廓清的目的。
    
    主观主义、宗派主义和党八股,这三种东西,都是反马克思主义的,都不是无产阶级所需要的,而是剥削阶级所需要的。这些东西在我们党内,是小资产阶级\footnote{〔小资产阶级〕介于资产阶级和无产阶级之间的过渡阶级,如小商人、手工业者、中农等,占有一定的生产资料,或许雇佣、剥削他人,但主要依靠自身劳动,被大资产阶级剥削。}思想的反映。中国是一个小资产阶级成分极其广大的国家,我们党是处在这个广大阶级的包围中,我们又有很大数量的党员是出身于这个阶级的,他们都不免或长或短地拖着一条小资产阶级的尾巴进党来。小资产阶级革命分子的狂热性和片面性,如果不加以节制,不加以改造,就很容易产生主观主义、宗派主义,它的一种表现形式就是洋八股,或党八股。
    
    要做对于这些东西的肃清工作和打扫工作,是不容易的。做起来必须得当,就是说,要好好地说理。如果说理说得好,说得恰当,那是会有效力的。说理的首先一个方法,就是重重地给患病者一个刺激,向他们大喝一声,说:“你有病呀!”使患者为之一惊,出一身汗,然后好好地叫他们治疗。
    
    现在来分析一下党八股的坏处在什么地方。我们也仿照八股文章\footnote{〔八股文章〕清朝科举应试的一种文体,在字数、体式、语气、题旨、思想内容上均有严格限定。起到束缚思想、维护封建统治的作用。}的笔法来一个“八股”,以毒攻毒,就叫做八大罪状吧。
    
    党八股的第一条罪状是:空话连篇,言之无物。我们有些同志欢喜写长文章,但是没有什么内容,真是“懒婆娘的裹脚,又长又臭”。为什么一定要写得那么长,又那么空空洞洞的呢?只有一种解释,就是下决心不要群众看。因为长而且空,群众见了就摇头,哪里还肯看下去呢?只好去欺负幼稚的人,在他们中间散布坏影响,造成坏习惯。去年六月二十二日,苏联进行那么大的反侵略战争,斯大林在七月三日发表了一篇演说,还只有我们《解放日报》\footnote{〔《解放日报》〕中共中央的机关报,1941年5月16日在延安创刊,1947年3月27日终刊。}一篇社论那样长。要是我们的老爷写起来,那就不得了,起码得有几万字。现在是在战争的时期,我们应该研究一下文章怎样写得短些,写得精粹些。延安虽然还没有战争,但军队天天在前方打仗,后方也唤工作忙,文章太长了,有谁来看呢?有些同志在前方也喜欢写长报告。他们辛辛苦苦地写了,送来了,其目的是要我们看的。可是怎么敢看呢?长而空不好,短而空就好吗?也不好。我们应当禁绝一切空话。但是主要的和首先的任务,是把那些又长又臭的懒婆娘的裹脚,赶快扔到垃圾桶里去。或者有人要说:《资本论》不是很长的吗?那又怎么办?这是好办的,看下去就是了。俗话说:“到什么山上唱什么歌。”又说:“看菜吃饭,量体裁衣。”我们无论做什么事都要看情形办理,文章和演说也是这样。我们反对的是空话连篇言之无物的八股调,不是说任何东西都以短为好。战争时期固然需要短文章,但尤其需要有内容的文章。最不应该、最要反对的是言之无物的文章。演说也是一样,空话连篇言之无物的演说,是必须停止的。
    
    党八股的第二条罪状是:装腔作势,借以吓人。有些党八股,不只是空话连篇,而且装样子故意吓人,这里面包含着很坏的毒素。空话连篇,言之无物,还可以说是幼稚;装腔作势,借以吓人,则不但是幼稚,简直是无赖了。鲁迅曾经批评过这种人,他说:“辱骂和恐吓决不是战斗。”科学的东西,随便什么时候都是不怕人家批评的,因为科学是真理,决不怕人家驳。主观主义和宗派主义的东西,表现在党八股式的文章和演说里面,却生怕人家驳,非常胆怯,于是就靠装样子吓人;以为这一吓,人家就会闭口,自己就可以“得胜回朝”了。这种装腔作势的东西,不能反映真理,而是妨害真理的。凡真理都不装样子吓人,它只是老老实实地说下去和做下去。从前许多同志的文章和演说里面,常常有两个名词:一个叫做“残酷斗争”,一个叫做“无情打击”。这种手段,用了对付敌人或敌对思想是完全必要的,用了对付自己的同志则是错误的。党内也常常有敌人和敌对思想混进来,如《苏联共产党(布)历史简要读本》\footnote{〔《苏联共产党(布)历史简要读本》〕1938年苏联出版的历史著作,斯大林主编,主要论述全联盟共产党历史,对苏联国内意识形态和国际共产主义运动影响深远。1941年延安整风运动期间,毛泽东表示应当用作研究马克思列宁主义的中心教材。}结束语第四条所说的那样。对于这种人,毫无疑义地是应该采用残酷斗争或无情打击的手段的,因为那些坏人正在利用这种手段对付党,我们如果还对他们宽容,那就会正中坏人的奸计。但是不能用同一手段对付偶然犯错误的同志;对于这类同志,就须使用批评和自我批评的方法,这就是《苏联共产党(布)历史简要读本》结束语第五条所说的方法。从前我们那些同志之所以向这些同志也大讲其“残酷斗争”和“无情打击”,一方面是没有分析对象,一方面就是为着装腔作势,借以吓人。无论对什么人,装腔作势借以吓人的方法,都是要不得的。因为这种吓人战术,对敌人是毫无用处,对同志只有损害。这种吓人战术,是剥削阶级以及流氓无产者\footnote{〔流氓无产者〕阶级社会中被统治阶级离弃、驱逐出劳动生产的无产者。他们一无所有,无法恢复自食其力的生活,往往以偷盗、讹诈、赌博、拐卖、行骗、卖淫等不正当手段和非正当职业谋生。}所惯用的手段,无产阶级不需要这类手段。无产阶级的最尖锐最有效的武器只有一个,那就是严肃的战斗的科学态度。共产党不靠吓人吃饭,而是靠马克思列宁主义的真理吃饭,靠实事求是吃饭,靠科学吃饭。至于以装腔作势来达到名誉和地位的目的,那更是卑劣的念头,不待说的了。总之,任何机关做决定,发指示,任何同志写文章,做演说,一概要靠马克思列宁主义的真理。只有靠了这个才能争取革命胜利,其它都是无益的。
    
    党八股的第三条罪状是:无的放矢,不看对象。早几年,在延安城墙上,曾经看见过这样一个标语:“工人农民联合起来争取抗日胜利。”这个标语的意思并不坏,可是那工人的工字第二笔不是写的一直,而是转了两个弯子。人字呢?在右边加了三撇。这位同志是古代文人学士的学生是无疑的了,可是他却要写在抗日时期延安这地方的墙壁上,就有些莫名其妙了。大概他的意思也是发誓不要老百姓看,否则就很难得到解释。共产党员如果真想做宣传,就要看对象,就要想一想自己的文章、演说、谈话、写字是给什么人看、给什么人听的,否则就等于下决心不要人看,不要人听。许多人常常以为自己写的讲的人家都看得很懂,听得很懂,其实完全不是那么一回事,因为他写的和讲的是党八股,人家哪里会懂呢?“对牛弹琴”这句话,含有讥笑对象的意思。如果我们除去这个意思,放进尊重对象的意思去,那就只剩下讥笑弹琴者这个意思了。为什么不看对象乱弹一顿呢?何况这是党八股,简直是老鸦声调,却偏要向人民群众哇哇地叫。射箭要看靶子,弹琴要看听众,写文章做演说倒可以不看读者不看听众吗?我们和无论什么人做朋友,如果不懂得彼此的心,不知道彼此心里面想些什么东西,能够做成知心朋友吗?做宣传工作的人,对于自己的宣传对象没有调查,没有研究,没有分析,乱讲一顿,是万万不行的。
    
    党八股的第四条罪状是:语言无味,像个瘪三\footnote{〔瘪三〕上海对无业流民的称呼,也用于贬低称呼他人。}。上海人叫小瘪三的那批角色,也很像我们的党八股,干瘪得很,样子十分难看。如果一篇文章,一个演说,颠来倒去,总是那几个名词,一套“学生腔”,没有一点生动活泼的语言,这岂不是语言无味,面目可憎,像个瘪三吗?一个人七岁入小学,十几岁入中学,二十多岁在大学毕业,没有和人民群众接触过,语言不丰富,单纯得很,那是难怪的。但我们是革命党,是为群众办事的,如果也不学群众的语言,那就办不好。现在我们有许多做宣传工作的同志,也不学语言。他们的宣传,乏味得很;他们的文章,就没有多少人欢喜看;他们的演说,也没有多少人欢喜听。为什么语言要学,并且要用很大的气力去学呢?因为语言这东西,不是随便可以学好的,非下苦功不可。第一,要向人民群众学习语言。人民的语汇是很丰富的,生动活泼的,表现实际生活的。我们很多人没有学好语言,所以我们在写文章做演说时没有几句生动活泼切实有力的话,只有死板板的几条筋,像瘪三一样,瘦得难看,不像一个健康的人。第二,要从外国语言中吸收我们所需要的成分。我们不是硬搬或滥用外国语言,是要吸收外国语言中的好东西,于我们适用的东西。因为中国原有语汇不够用,现在我们的语汇中就有很多是从外国吸收来的。例如今天开的干部大会,这“干部”两个字,就是从外国学来的。我们还要多多吸收外国的新鲜东西,不但要吸收他们的进步道理,而且要吸收他们的新鲜用语。第三,我们还要学习古人语言中有生命的东西。由于我们没有努力学习语言,古人语言中的许多还有生气的东西我们就没有充分地合理地利用。当然我们坚决反对去用已经死了的语汇和典故,这是确定了的,但是好的仍然有用的东西还是应该继承。现在中党八股毒太深的人,对于民间的、外国的、古人的语言中有用的东西,不肯下苦功去学,因此,群众就不欢迎他们枯燥无味的宣传,我们也不需要这样蹩脚的不中用的宣传家。什么是宣传家?不但教员是宣传家,新闻记者是宣传家,文艺作者是宣传家,我们的一切工作干部也都是宣传家。比如军事指挥员,他们并不对外发宣言,但是他们要和士兵讲话,要和人民接洽,这不是宣传是什么?一个人只要他对别人讲话,他就是在做宣传工作。只要他不是哑巴,他就总有几句话要讲的。所以我们的同志都非学习语言不可。
    
    党八股的第五条罪状是:甲乙丙丁,开中药铺。你们去看一看中药铺,那里的药柜子上有许多抽屉格子,每个格子上面贴着药名,当归、熟地、大黄、芒硝,应有尽有。这个方法,也被我们的同志学到了。写文章,做演说,着书,写报告,第一是大壹贰叁肆,第二是小一二三四,第三是甲乙丙丁,第四是子丑寅卯,还有大ABCD,小abcd,还有阿拉伯数字,多得很!幸亏古人和外国人替我们造好了这许多符号,使我们开起中药铺来毫不费力。一篇文章充满了这些符号,不提出问题,不分析问题,不解决问题,不表示赞成什么,反对什么,说来说去还是一个中药铺,没有什么真切的内容。我不是说甲乙丙丁等字不能用,而是说那种对待问题的方法不对。现在许多同志津津有味于这个开中药铺的方法,实在是一种最低级、最幼稚、最庸俗的方法。这种方法就是形式主义的方法,是按照事物的外部标志来分类,不是按照事物的内部联系来分类的。单单按照事物的外部标志,使用一大堆互相没有内部联系的概念,排列成一篇文章、一篇演说或一个报告,这种办法,他自己是在做概念的游戏,也会引导人家都做这类游戏,使人不用脑筋想问题,不去思考事物的本质,而满足于甲乙丙丁的现象罗列。什么叫问题?问题就是事物的矛盾。哪里有没有解决的矛盾,哪里就有问题。既有问题,你总得赞成一方面,反对另一方面,你就得把问题提出来。提出问题,首先就要对于问题即矛盾的两个基本方面加以大略的调查和研究,才能懂得矛盾的性质是什么,这就是发现问题的过程。大略的调查和研究可以发现问题,提出问题,但是还不能解决问题。要解决问题,还须作系统的周密的调查工作和研究工作,这就是分析的过程。提出问题也要用分析,不然,对着模糊杂乱的一大堆事物的现象,你就不能知道问题即矛盾的所在。这里所讲的分析过程,是指系统的周密的分析过程。常常问题是提出了,但还不能解决,就是因为还没有暴露事物的内部联系,就是因为还没有经过这种系统的周密的分析过程,因而问题的面貌还不明晰,还不能做综合工作,也就不能好好地解决问题。一篇文章或一篇演说,如果是重要的带指导性质的,总得要提出一个什么问题,接着加以分析,然后综合起来,指明问题的性质,给以解决的办法,这样,就不是形式主义的方法所能济事。因为这种幼稚的、低级的、庸俗的、不用脑筋的形式主义的方法,在我们党内很流行,所以必须揭破它,才能使大家学会应用马克思主义的方法去观察问题、提出问题、分析问题和解决问题,我们所办的事才能办好,我们的革命事业才能胜利。
    
    党八股的第六条罪状是:不负责任,到处害人。上面所说的那些,一方面是由于幼稚而来,另一方面也是由于责任心不足而来的。拿洗脸作比方,我们每天都要洗脸,许多人并且不止洗一次,洗完之后还要拿镜子照一照,要调查研究一番,(大笑)生怕有什么不妥当的地方。你们看,这是何等地有责任心呀!我们写文章,做演说,只要像洗脸这样负责,就差不多了。拿不出来的东西就不要拿出来。须知这是要去影响别人的思想和行动的啊!一个人偶然一天两天不洗脸,固然也不好,洗后脸上还留着一个两个黑点,固然也不雅观,但倒并没有什么大危险。写文章做演说就不同了,这是专为影响人的,我们的同志反而随随便便,这就叫做轻重倒置。许多人写文章,做演说,可以不要预先研究,不要预先准备;文章写好之后,也不多看几遍,像洗脸之后再照照镜子一样,就马马虎虎地发表出去。其结果,往往是“下笔千言,离题万里”,仿佛像个才子,实则到处害人。这种责任心薄弱的坏习惯,必须改正才好。
    
    第七条罪状是:流毒全党,妨害革命。第八条罪状是:传播出去,祸国殃民。这两条意义自明,无须多说。这就是说,党八股如不改革,如果听其发展下去,其结果之严重,可以闹到很坏的地步。党八股里面藏的是主观主义、宗派主义的毒物,这个毒物传播出去,是要害党害国的。
    
    上面这八条,就是我们申讨党八股的檄文。
    
    党八股这个形式,不但不便于表现革命精神,而且非常容易使革命精神窒息。要使革命精神获得发展,必须抛弃党八股,采取生动活泼新鲜有力的马克思列宁主义的文风。这种文风,早已存在,但尚未充实,尚未得到普遍的发展。我们破坏了洋八股和党八股之后,新的文风就可以获得充实,获得普遍的发展,党的革命事业,也就可以向前推进了。
    
\end{normalsize}


\newpage

\textbf{注释}:

\vspace{-1em}

\begin{itemize}
    \setlength\itemsep{-0.2em}
    \item 〔反动〕反对某种运动,与之对抗。
    \item 〔无的放矢〕没有目标就射出箭矢。
    \item 〔檄文〕古代写在木简上的官方文书,用于晓谕、征召、声讨。很多时候特指声讨的文告。
    \item 〔蹩脚〕不老练;质量低劣;技艺差。
    \item 〔窒息〕呼吸困难、无法呼吸。形容因受阻而断绝。
\end{itemize}

\chapter{祝福}

\begin{normalsize}
    
    旧历的年底毕竟最像年底,村镇上不必说,就在天空中也显出将到新年的气象来。灰白色的沉重的晚云中间时时发出闪光,接着一声钝响,是送灶\footnote{〔送灶〕传说农历12月24日灶神升天述职,在23日或24日祭拜灶神,称为送灶。}的爆竹;近处燃放的可就更强烈了,震耳的大音还没有息,空气里已经散满了幽微的火药香。我是正在这一夜回到我的故乡鲁镇的。虽说故乡,然而已没有家,所以只得暂寓在鲁四老爷的宅子里。他是我的本家,比我长一辈,应该称之曰“四叔”,是一个讲理学\footnote{〔理学〕也叫道学,宋代周敦颐、程颢、程颐、朱熹阐释儒家学说形成的封建唯心主义思想体系。}的老监生\footnote{〔监生〕国子监的生员,指明清时期在国子监读书的学生。清乾隆以后,国子监只存空名,地主豪绅可以凭“祖荫”或捐钱取得监生资格,参加科举乡试。}。他比先前并没有什么大改变,单是老了些,但也还未留胡子,一见面是寒暄,寒暄之后说我“胖了”,说我“胖了”之后即大骂起新党\footnote{〔新党〕清末对主张或倾向维新的人的称呼。辛亥革命前后,也用来称呼革命党人或拥护革命的人。}。但我知道,这并非借题在骂我:因为他所骂的还是康有为\footnote{〔康有为〕清末政治家、思想家。提倡君主立宪。1898年策动戊戌变法失败后逃亡海外,成为保皇派,反对辛亥革命。}。但是,谈话是总不投机的了,于是不多久,我便一个人剩在书房里。
    
    第二天我起得很迟,午饭之后,出去看了几个本家和朋友;第三天也照样。他们也都没有什么大改变,单是老了些;家中却一律忙,都在准备着“祝福”。这是鲁镇年终的大典,致敬尽礼,迎接福神,拜求来年一年中的好运气的。杀鸡,宰鹅,买猪肉,用心细细的洗,女人的臂膊都在水里浸得通红,有的还带着绞丝银镯\footnote{〔绞丝银镯〕用多股银丝绞缠做成的镯子。}子。煮熟之后,横七竖八的插些筷子在这类东西上,可就称为“福礼”了,五更天陈列起来,并且点上香烛,恭请福神们来享用,拜的却只限于男人,拜完自然仍然是放爆竹。年年如此,家家如此,——只要买得起福礼和爆竹之类的——今年自然也如此。天色愈阴暗了,下午竟下起雪来,雪花大的有梅花那么大,满天飞舞,夹着烟霭和忙碌的气色,将鲁镇乱成一团糟。我回到四叔的书房里时,瓦楞上已经雪白,房里也映得较光明,极分明的显出壁上挂着的朱拓\footnote{〔朱拓〕用银朱等红颜料从碑刻上拓印下来的文字、图形。}的大“寿”字,陈抟老祖\footnote{〔陈抟老祖〕五代末宋初的道士,传说养生修炼成了神仙。}写的,一边的对联已经脱落,松松的卷了放在长桌上,一边的还在,道是“事理通达心气和平\footnote{〔事理通达心气和平〕出自朱熹《论语集注》,是理学家宣扬的自我修养的标准。}”。我又无聊赖的到窗下的案头去一翻,只见一堆似乎未必完全的《康熙字典》\footnote{〔《康熙字典》〕清康熙年间官方组织编纂的大型字典。},一部《近思录集注》\footnote{〔《近思录集注》〕清初茅星来、江永为《近思录》作的集注。《近思录》是宋代朱熹、吕祖谦选编的理学家文章语录,是理学的入门书。}和一部《四书衬》\footnote{〔《四书衬》〕清代骆培解说《四书》的一本书。}。无论如何、我明天决计要走了。
    
    况且,一直到昨天遇见祥林嫂的事,也就使我不能安住。那是下午,我到镇的东头访过一个朋友,走出来,就在河边遇见她;而且见她瞪着的眼睛的视线,就知道明明是向我走来的。我这回在鲁镇所见的人们中,改变之大,可以说无过于她的了:五年前的花白的头发,即今已经全白,会不像四十上下的人;脸上瘦削丕堪,黄中带黑,而且消尽了先前悲哀的神色,仿佛是木刻似的;只有那眼珠间或一轮\footnote{〔间或一轮〕偶尔转动一下。间或:时不时、偶尔。},还可以表示她是一个活物。她一手提着竹篮。内中一个破碗,空的;一手技着一支比她更长的竹竿,下端开了裂:她分明已经纯乎是一个乞丐了。
    
    我就站住,豫备\footnote{〔豫备〕预备。“豫”通“预”。}她来讨钱。
    
    “你回来了?”她先这样问。
    
    “是的。”
    
    “这正好。你是识字的,又是出门人,见识得多。我正要问你一件事——”她那没有精采的眼睛忽然发光了。
    
    我万料不到她却说出这样的话来,诧异的站着。
    
    “就是——”她走近两步,放低了声音,极秘密似的切切的说,“一个人死了之后,究竟有没有魂灵的?”
    
    我很悚然,一见她的眼钉着我的\footnote{〔她的眼钉着我的〕钉:现在一般用“盯”。},背上也就遭了芒刺一般,比在学校里遇到不及豫防的临时考,教师又偏是站在身旁的时候,惶急得多了。对于魂灵的有无,我自己是向来毫不介意的;但在此刻,怎样回答她好呢?我在极短期的踌躇中,想,这里的人照例相信鬼,“然而她,却疑惑了,——或者不如说希望:希望其有,又希望其无……,人何必增添末路的人的苦恼,一为她起见,不如说有罢。
    
    “也许有罢,——我想。”我于是吞吞吐吐的说。
    
    “那么,也就有地狱了?”
    
    “啊!地狱?”我很吃惊,只得支吾着,“地狱?——论理,就该也有。——然而也未必,……谁来管这等事……。”
    
    “那么,死掉的一家的人,都能见面的?”
    
    “唉唉,见面不见面呢?……”这时我已知道自己也还是完全一个愚人,什么踌躇,什么计画\footnote{〔计画〕计划。},都挡不住三句问,我即刻胆怯起来了,便想全翻过先前的话来,“那是,……实在,我说不清……。其实,究竟有没有魂灵,我也说不清。”
    
    我乘她不再紧接的问,迈开步便走,勿勿的逃回四叔的家中,心里很觉得不安逸。自己想,我这答话怕于她有些危险。她大约因为在别人的祝福时候,感到自身的寂寞了,然而会不会含有别的什么意思的呢?——或者是有了什么豫感了?倘有别的意思,又因此发生别的事,则我的答话委实该负若干的责任……。但随后也就自笑,觉得偶尔的事,本没有什么深意义,而我偏要细细推敲,正无怪教育家要说是生着神经病;而况明明说过“说不清”,已经推翻了答话的全局,即使发生什么事,于我也毫无关系了。
    
    “说不清”是一句极有用的话。不更事的勇敢的少年,往往敢于给人解决疑问,选定医生,万一结果不佳,大抵反成了怨府\footnote{〔怨府〕怨恨集中的地方,指受埋怨的对象。},然而一用这说不清来作结束,便事事逍遥自在了。我在这时,更感到这一句话的必要,即使和讨饭的女人说话,也是万不可省的。
    
    但是我总觉得不安,过了一夜,也仍然时时记忆起来,仿佛怀着什么不祥的豫感,在阴沉的雪天里,在无聊的书房里,这不安愈加强烈了。不如走罢,明天进城去。福兴楼的清燉鱼翅,一元一大盘,价廉物美,现在不知增价了否?往日同游的朋友,虽然已经云散,然而鱼翅是不可不吃的,即使只有我一个……。无论如何,我明天决计要走了。
    
    我因为常见些但愿不如所料,以为未毕竟如所料的事,却每每恰如所料的起来,所以很恐怕这事也一律。果然,特别的情形开始了。傍晚,我竟听到有些人聚在内室里谈话,仿佛议论什么事似的,但不一会,说话声也就止了,只有四叔且走而且高声的说:
    
    “不早不迟,偏偏要在这时候——这就可见是一个谬种\footnote{〔谬种〕坏蛋,骂人的话。}!”
    
    我先是诧异,接着是很不安,似乎这话于我有关系。试望门外,谁也没有。好容易待到晚饭前他们的短工来冲茶,我才得了打听消息的机会。
    
    “刚才,四老爷和谁生气呢?”我问。
    
    “还不是和样林嫂?”那短工简捷的说。
    
    “祥林嫂?怎么了?”我又赶紧的问。
    
    “老了。”
    
    “死了?”我的心突然紧缩,几乎跳起来,脸上大约也变了色,但他始终没有抬头,所以全不觉。我也就镇定了自己,接着问:
    
    “什么时候死的?”
    
    “什么时候?——昨天夜里,或者就是今天罢。——我说不清。”
    
    “怎么死的?”
    
    “怎么死的?——还不是穷死的?”他淡然的回答,仍然没有抬头向我看,出去了。
    
    然而我的惊惶却不过暂时的事,随着就觉得要来的事,已经过去,并不必仰仗我自己的“说不清”和他之所谓“穷死的”的宽慰,心地已经渐渐轻松;不过偶然之间,还似乎有些负疚。晚饭摆出来了,四叔俨然的陪着。我也还想打听些关于祥林嫂的消息,但知道他虽然读过“鬼神者二气之良能也\footnote{〔鬼神者二气之良能也〕鬼神是阴阳二气自然形成的。出自宋代张载《张子全书·正蒙》,也见于《近思录》。}”,而忌讳仍然极多,当临近祝福时候,是万不可提起死亡疾病之类的话的,倘不得已,就该用一种替代的隐语,可惜我又不知道,因此屡次想问,而终于中止了。我从他俨然的脸色上,又忽而疑他正以为我不早不迟,偏要在这时候来打搅他,也是一个谬种,便立刻告诉他明天要离开鲁镇,进城去,趁早放宽了他的心。他也不很留。这佯闷闷的吃完了一餐饭。
    
    冬季日短,又是雪天,夜色早已笼罩了全市镇。人们都在灯下匆忙,但窗外很寂静。雪花落在积得厚厚的雪褥上面,听去似乎瑟瑟有声,使人更加感得沉寂。我独坐在发出黄光的莱油灯下,想,这百无聊赖的祥林嫂,被人们弃在尘芥堆\footnote{〔尘芥堆〕垃圾堆。尘芥:尘土、小草,泛指不重要的东西。}中的,看得厌倦了的陈旧的玩物,先前还将形骸露在尘芥里,从活得有趣的人们看来,恐怕要怪讶她何以还要存在,现在总算被无常\footnote{〔无常〕佛教的概念,指世间事物都处在变化中,没有常住性。引申为死亡。}打扫得于干净净了。魂灵的有无,我不知道;然而在现世,则无聊生者不生,即使厌见者不见,为人为己,也还都不错\footnote{〔然而在现世……〕这句话的意思是说,现在世上无所依靠而活不下去的穷苦人都死了,结果倒让不想见到这些人的人眼前清静了,对人对己都是好事。这是“我”激愤下的反话。}。我静听着窗外似乎瑟瑟作响的雪花声,一面想,反而渐渐的舒畅起来。
    
    然而先前所见所闻的她的半生事迹的断片,至此也联成一片\footnote{〔联成一片〕连成一片。“联”通“连”。}了。
    
    她不是鲁镇人。有一年的冬初,四叔家里要换女工,做中人\footnote{〔中人〕中介、中间人。}的卫老婆子带她进来了,头上扎着白头绳,乌裙,蓝夹袄,月白背心,年纪大约二十六七,脸色青黄,但两颊却还是红的。卫老婆子叫她祥林嫂,说是自己母家的邻舍,死了当家人,所以出来做工了。四叔皱了皱眉,四婶已经知道了他的意思,是在讨厌她是一个寡妇。但是她模样还周正,手脚都壮大,又只是顺着限,不开一句口,很像一个安分耐劳的人,便不管四叔的皱眉,将她留下了。试工期内,她整天的做,似乎闲着就无聊,又有力,简直抵得过一个男子,所以第三天就定局\footnote{〔定局〕把事情定下来。这里指确定雇佣祥林嫂。},每月工钱五百文。
    
    大家都叫她祥林嫂;没问她姓什么,但中人是卫家山人,既说是邻居,那大概也就姓卫了。她不很爱说话,别人问了才回答,答的也不多。直到十几天之后,这才陆续的知道她家里还有严厉的婆婆,一个小叔子,十多岁,能打柴了;她是春天没了丈夫的;他本来也打柴为生,比她小十岁:大家所知道的就只是这一点。
    
    日子很快的过去了,她的做工却毫没有懈,食物不论,力气是不惜的。人们都说鲁四老爷家里雇着了女工,实在比勤快的男人还勤快。到年底,扫尘,洗地,杀鸡,宰鹅,彻夜的煮福礼,全是一人担当,竟没有添短工。然而她反满足,口角边渐渐的有了笑影,脸上也白胖了。
    
    新年才过,她从河边淘米回来时,忽而失了色,说刚才远远地看见几个男人在对岸徘徊,很像夫家的堂伯,恐怕是正在寻她而来的。四婶很惊疑,打听底细,她又不说。四叔一知道,就皱一皱眉,道:
    
    “这不好。恐怕她是逃出来的。”
    
    她诚然\footnote{〔诚然〕确实,真的。}是逃出来的,不多久,这推想就证实了。
    
    此后大约十几天,大家正已渐渐忘却了先前的事,卫老婆子忽而带了一个三十多岁的女人进来了,说那是祥林嫂的婆婆。那女人虽是山里人模样,然而应酬很从容,说话也能干,寒暄之后,就赔罪,说她特来叫她的儿媳回家去,因为开春事务忙,而家中只有老的和小的,人手不够了。
    
    “既是她的婆婆要她回去,那有什么话可说呢。”四叔说。
    
    于是算清了工钱,一共一千七百五十文,她全存在主人家,一文也还没有用,便都交给她的婆婆。那女人又取了衣服,道过谢,出去了。其时已经是正午。
    
    “阿呀,米呢?祥林嫂不是去淘米的么?……”好一会,四婶这才惊叫起来。她大约有些饿,记得午饭了。
    
    于是大家分头寻淘箩\footnote{〔淘箩〕淘米用的竹篾。}。她先到厨下,次到堂前,后到卧房,全不见淘箩的影子。四叔踱出门外,也不见,一直到河边,才见平平正正的放在岸上,旁边还有一株菜。
    
    看见的人报告说,河里面上午就泊了一只白篷船,篷是全盖起来的,不知道什么人在里面,但事前也没有人去理会他。待到祥林嫂出来淘米,刚刚要跪下去,那船里便突然跳出两个男人来,像是山里人,一个抱住她,一个帮着,拖进船去了。样林嫂还哭喊了几声,此后便再没有什么声息,大约给用什么堵住了罢。接着就走上两个女人来,一个不认识,一个就是卫婆子。窥探舱里,不很分明,她像是捆了躺在船板上。
    
    “可恶!然而……。”四叔说。
    
    这一天是四婶自己煮中饭;他们的儿子阿牛烧火。
    
    午饭之后,卫老婆子又来了。
    
    “可恶!”四叔说。
    
    “你是什么意思?亏你还会再来见我们。”四婶洗着碗,一见面就愤愤的说,“你自己荐她来,又合伙劫她去,闹得沸反盈天的,大家看了成个什么样子?你拿我们家里开玩笑么?”
    
    “阿呀阿呀,我真上当。我这回,就是为此特地来说说清楚的。她来求我荐地方,我那里料得到是瞒着她的婆婆的呢。对不起,四老爷,四太太。总是我老发昏不小心,对不起主顾。幸而府上是向来宽洪大量,不肯和小人计较的。这回我一定荐一个好的来折罪……。”
    
    “然而……。”四叔说。
    
    于是祥林嫂事件便告终结,不久也就忘却了。
    
    只有四嫂,因为后来雇用的女工,大抵非懒即馋,或者馋而且懒,左右不如意,所以也还提起祥林嫂。每当这些时候,她往往自言自语的说,“她现在不知道怎么佯了?”意思是希望她再来。但到第二年的新正\footnote{〔新正〕农历新年正月。},她也就绝了望。
    
    新正将尽,卫老婆子来拜年了,已经喝得醉醺醺的,自说因为回了一趟卫家山的娘家,住下几天,所以来得迟了。她们问答之间,自然就谈到祥林嫂。
    
    “她么?”卫若婆子高兴的说,“现在是交了好运了。她婆婆来抓她回去的时候,是早已许给了贺家坳的贺老六的,所以回家之后不几天,也就装在花轿里抬去了。”
    
    “阿呀,这样的婆婆!……”四婶惊奇的说。
    
    “阿呀,我的太太!你真是大户人家的太太的话。我们山里人,小户人家,这算得什么?她有小叔子,也得娶老婆。不嫁了她,那有这一注钱\footnote{〔一注钱〕一笔钱。}来做聘礼?他的婆婆倒是精明强干的女人呵,很有打算,所以就将地嫁到里山\footnote{〔里山〕深山里面。}去。倘许给本村人,财礼就不多;惟独肯嫁进深山野坳里去的女人少,所以她就到手了八十千\footnote{〔八十千〕八十贯钱。一千钱为一贯。}。现在第二个儿子的媳妇也娶进了,财礼花了五十,除去办喜事的费用,还剩十多千。吓\footnote{〔吓〕这里做语气词,同“嗬”。},你看,这多么好打算?……”
    
    “祥林嫂竟肯依?……”
    
    “这有什么依不依。——闹是谁也总要闹一闹的,只要用绳子一捆,塞在花轿里,抬到男家,捺上花冠,拜堂,关上房门,就完事了。可是祥林嫂真出格,听说那时实在闹得利害\footnote{〔利害〕厉害。},大家还都说大约因为在念书人家做过事,所以与众不同呢。太太,我们见得多了:回头人\footnote{〔回头人〕指再嫁的寡妇。}出嫁,哭喊的也有,说要寻死觅活的也有,抬到男家闹得拜不成天地的也有,连花烛都砸了的也有。样林嫂可是异乎寻常,他们说她一路只是嚎,骂,抬到贺家坳,喉咙已经全哑了。拉出轿来,两个男人和她的小叔子使劲的捺住她也还拜不成天地。他们一不小心,一松手,阿呀,阿弥陀佛,她就一头撞在香案角上,头上碰了一个大窟窿,鲜血直流,用了两把香灰,包上两块红布还止不住血呢。直到七手八脚的将她和男人反关在新房里,还是骂,阿呀呀,这真是……。”她摇一摇头,顺下眼睛,不说了。
    
    “后来怎么样呢?”四婶还问。
    
    “听说第二天也没有起来。”她抬起眼来说。
    
    “后来呢?”
    
    “后来?——起来了。她到年底就生了一个孩子,男的,新年就两岁了。我在娘家这几天,就有人到贺家坳去,回来说看见他们娘儿俩,母亲也胖,儿子也胖;上头又没有婆婆,男人所有的是力气,会做活;房子是自家的。——唉唉,她真是交了好运了。”
    
    从此之后,四婶也就不再提起祥林嫂。
    
    但有一年的秋季,大约是得到祥林嫂好运的消息之后的又过了两个新年,她竟又站在四叔家的堂前了。桌上放着一个荸荠式的圆篮,檐下一个小铺盖。她仍然头上扎着白头绳,乌裙,蓝夹祆,月白背心,脸色青黄,只是两颊上已经消失了血色,顺着眼\footnote{〔顺着眼〕垂着眼,形容顺从的样子。},眼角上带些泪痕,眼光也没有先前那样精神了。而且仍然是卫老婆子领着,显出慈悲模样,絮絮的对四婶说:
    
    “……这实在是叫作‘天有不测风云’,她的男人是坚实人,谁知道年纪青青,就会断送在伤寒\footnote{〔伤寒〕泛指外感风寒引起的发烧热病。}上?本来已经好了的,吃了一碗冷饭,复发了。幸亏有儿子;她又能做,打柴摘茶养蚕都来得,本来还可以守着,谁知道那孩子又会给狼衔去的呢?春天快完了,村上倒反来了狼,谁料到?现在她只剩了一个光身了。大伯来收屋,又赶她。她真是走投无路了,只好来求老主人。好在她现在已经再没有什么牵挂,太太家里又凄巧要换人,所以我就领她来。——我想,熟门熟路,比生手实在好得多……。”
    
    “我真傻,真的,”祥林嫂抬起她没有神采的眼睛来,接着说。“我单知道下雪的时候野兽在山坳里没有食吃,会到村里来;我不知道春天也会有。我一清早起来就开了门,拿小篮盛了一篮豆,叫我们的阿毛坐在门槛上剥豆去。他是很听话的,我的话句句听;他出去了。我就在屋后劈柴,淘米,米下了锅,要蒸豆。我叫阿毛,没有应,出去口看,只见豆撒得一地,没有我们的阿毛了。他是不到别家去玩的;各处去一问,果然没有。我急了,央人出去寻。直到下半天,寻来寻去寻到山坳里,看见刺柴上挂着一只他的小鞋。大家都说,糟了,怕是遭了狼了。再进去;他果然躺在草窠里,肚里的五脏已经都给吃空了,手上还紧紧的捏着那只小篮呢。……”她接着但是\footnote{〔但是〕只是。}呜咽,说不出成句的话来。
    
    四婶起刻还踌躇,待到听完她自己的话,眼圈就有些红了。她想了一想,便教拿圆篮和铺盖到下房去。卫老婆子仿佛卸了一肩重相似的嘘一口气,祥林嫂比初来时候神气舒畅些,不待指引,自己驯熟的安放了铺盖。她从此又在鲁镇做女工了。
    
    大家仍然叫她祥林嫂。
    
    然而这一回,她的境遇却改变得非常大。上工之后的两三天,主人们就觉得她手脚已没有先前一样灵活,记性也坏得多,死尸似的脸上又整日没有笑影,四婶的口气上,已颇有些不满了。当她初到的时候,四叔虽然照例皱过眉,但鉴于向来雇用女工之难,也就并不大反对,只是暗暗地告诫四姑说,这种人虽然似乎很可怜,但是败坏风俗的,用她帮忙还可以,祭祀时候可用不着她沾手,一切饭莱,只好自已做,否则,不干不净,祖宗是不吃的。
    
    四叔家里最重大的事件是祭祀,祥林嫂先前最忙的时候也就是祭祀,这回她却清闲了。桌子放在堂中央,系上桌帏\footnote{〔桌帏〕也叫“桌围”,婚丧事或祭祀的桌案前面垂下的布或绸缎。},她还记得照旧的去分配酒杯和筷子。
    
    “祥林嫂,你放着罢!我来摆。”四婶慌忙的说。
    
    她讪讪的缩了手,又去取烛台。
    
    “祥林嫂,你放着罢!我来拿。”四婶又慌忙的说。
    
    她转了几个圆圈,终于没有事情做,只得疑惑的走开。她在这一天可做的事是不过坐在灶下烧火。
    
    镇上的人们也仍然叫她祥林嫂,但音调和先前很不同;也还和她讲话,但笑容却冷冷的了。她全不理会那些事,只是直着眼睛,和大家讲她自己日夜不忘的故事:
    
    “我真傻,真的,”她说,“我单知道雪天是野兽在深山里没有食吃,会到村里来;我不知道春天也会有。我一大早起来就开了门,拿小篮盛了一篮豆,叫我们的阿毛坐在门槛上剥豆去。他是很听话的孩子,我的话句句听;他就出去了。我就在屋后劈柴,淘米,米下了锅,打算蒸豆。我叫,‘阿毛!’没有应。出去一看,只见豆撒得满地,没有我们的阿毛了。各处去一向,都没有。我急了,央人去寻去。直到下半天,几个人寻到山坳里,看见刺柴上挂着一只他的小鞋。大家都说,完了,怕是遭了狼了;再进去;果然,他躺在草窠里,肚里的五脏已经都给吃空了,可怜他手里还紧紧的捏着那只小篮呢。……”她于是淌下眼泪来,声音也呜咽了。
    
    这故事倒颇有效,男人听到这里,往往敛起笑容,没趣的走了开去;女人们却不独宽恕了她似的,脸上立刻改换了鄙薄的神气,还要陪出许多眼泪来。有些老女人没有在街头听到她的话,便特意寻来,要听她这一段悲惨的故事。直到她说到呜咽,她们也就一齐流下那停在眼角上的眼泪,叹息一番,满足的去了,一面还纷纷的评论着。
    
    她就只是反复的向人说她悲惨的故事,常常引住了三五个人来听她。但不久,大家也都听得纯熟了,便是最慈悲的念佛的老太太们,眼里也再不见有一点泪的痕迹。后来全镇的人们几乎都能背诵她的话,一听到就烦厌得头痛。
    
    “我真傻,真的,”她开首说。
    
    “是的,你是单知道雪天野兽在深山里没有食吃,才会到村里来的。”他们立即打断她的话,走开去了。
    
    她张着口怔怔的站着,直着眼睛看他们,接着也就走了,似乎自己也觉得没趣。但她还妄想,希图从别的事,如小篮,豆,别人的孩子上,引出她的阿毛的故事来。倘一看见两三岁的小孩子,她就说:
    
    “唉唉,我们的阿毛如果还在,也就有这么大了……”
    
    孩子看见她的眼光就吃惊,牵着母亲的衣襟催她走。于是又只剩下她一个,终于没趣的也走了,后来大家又都知道了她的脾气,只要有孩子在眼前,便似笑非笑的先问她,道:
    
    “祥林嫂,你们的阿毛如果还在,不是也就有这么大了么?”
    
    她未必知道她的悲哀经大家咀嚼赏鉴了许多天,早已成为渣滓,只值得烦厌和唾弃;但从人们的笑影上,也仿佛觉得这又冷又尖,自己再没有开口的必要了。她单是一瞥他们,并不回答一句话。
    
    鲁镇永远是过新年,腊月二十以后就火起来了。四叔家里这回须雇男短工,还是忙不过来,另叫柳妈做帮手,杀鸡,宰鹅;然而柳妈是善女人\footnote{〔善女人〕信奉佛教但没有出家的女子。善:佛教徒对供养者和信众的尊称。},吃素,不杀生的,只肯洗器皿。祥林嫂除烧火之外,没有别的事,却闲着了,坐着只看柳妈洗器皿。微雪点点的下来了。
    
    “唉唉,我真傻,”祥林嫂看了天空,叹息着,独语似的说。
    
    “祥林嫂,你又来了。”柳妈不耐烦的看着她的脸,说。“我问你:你额角上的伤痕,不就是那时撞坏的么?”
    
    “唔唔。”她含胡的回答。
    
    “我问你:你那时怎么后来竟依了呢?”
    
    “我么?……”,
    
    “你呀。我想:这总是你自己愿意了,不然……。”
    
    “阿阿,你不知道他力气多么大呀。”
    
    “我不信。我不信你这么大的力气,真会拗他不过。你后来一定是自己肯了,倒推说他力气大。”
    
    “阿阿,你……你倒自己试试着。”她笑了。
    
    柳妈的打皱的脸也笑起来,使她蹙缩得像一个核桃,干枯的小眼睛一看祥林嫂的额角,又钉住她的眼。祥林嫂似很局促了,立刻敛了笑容,旋转眼光,自去看雪花。
    
    “祥林嫂,你实在不合算。”柳妈诡秘的说。“再一强\footnote{〔再一强〕继续反抗下去。强:犟。},或者索性撞一个死,就好了。现在呢,你和你的第二个男人过活不到两年,倒落了一件大罪名。你想,你将来到阴司\footnote{〔阴司〕指阴间的官府。迷信中认为死后审判的地方。}去,那两个死鬼的男人还要争,你给了谁好呢?阎罗大王\footnote{〔阎罗大王〕阎罗:佛教用语,掌管阿鼻地狱的平等王的梵语音译“阎魔罗阇”的简称。迷信中指阴间地狱审判死者的王。生前行善者可以转世成人,作恶者堕入阿鼻地狱受苦。}只好把你锯开来,分给他们。我想,这真是……”
    
    她脸上就显出恐怖的神色来\footnote{〔恐怖〕恐惧。},这是在山村里所未曾知道的。
    
    “我想,你不如及早抵当\footnote{〔抵当〕抵挡,应对。}。你到土地庙里去捐一条门槛,当作你的替身,给千人踏,万人跨,赎了这一世的罪名,免得死了去受苦。”
    
    她当时并不回答什么话,但大约非常苦闷了,第二天早上起来的时候,两眼上便都围着大黑圈。早饭之后,她便到镇的西头的土地庙里去求捐门槛,庙祝\footnote{〔庙祝〕寺庙里管香火祭祀的人。}起初执意不允许,直到她急得流泪,才勉强答应了。价目是大钱十二千。她久已不和人们交口,因为阿毛的故事是早被大家厌弃了的;但自从和柳妈谈了天,似乎又即传扬开去,许多人都发生了新趣味,又来逗她说话了。至于题目,那自然是换了一个新样,专在她额上的伤疤。
    
    “祥林嫂,我问你:你那时怎么竟肯了?”一个说。
    
    “唉,可惜,白撞了这一下。”一个看着她的疤,应和道。
    
    她大约从他们的笑容和声调上,也知道是在嘲笑她,所以总是瞪着眼睛,不说一句话,后来连头也不回了。她整日紧闭了嘴唇,头上带着大家以为耻辱的记号的那伤痕,默默的跑街,扫地,洗莱,淘米。快够一年,她才从四婶手里支取了历来积存的工钱,换算了十二元鹰洋\footnote{〔鹰洋〕指墨西哥银元。鸦片战争后大量流入我国,一度流通。},请假到镇的西头去。但不到一顿饭时候,她便回来,神气很舒畅,眼光也分外有神,高兴似的对四婶说,自己已经在土地庙捐了门槛了。
    
    冬至的祭祖时节,她做得更出力,看四婶装好祭品,和阿牛将桌子抬到堂屋中央,她便坦然的去拿酒杯和筷子。
    
    “你放着罢,祥林嫂!”四婶慌忙大声说。
    
    她像是受了炮烙\footnote{〔炮烙〕古代传说中的刑罚,让犯人在烧红的铜柱上走。这里指仿佛被烫到。}似的缩手,脸色同时变作灰黑,也不再去取烛台,只是失神的站着。直到四叔上香的时候,教她走开,她才走开。这一回她的变化非常大,第二天,不但眼睛窈陷\footnote{〔窈陷〕凹陷。}下去,连精神也更不济了。而且很胆怯,不独怕暗夜,怕黑影,即使看见人,虽是自己的主人,也总惴惴的,有如在白天出穴游行的小鼠,否则呆坐着,直是一个木偶人。不半年,头发也花白起来了,记性尤其坏,甚而至于常常忘却了去淘米。
    
    “祥林嫂怎么这样了?倒不如那时不留她。”四婶有时当面就这样说,似乎是警告她。
    
    然而她总如此,全不见有伶俐起来的希望。他们于是想打发她走了,教她回到卫老婆子那里去。但当我还在鲁镇的时候,不过单是这样说;看现在的情状,可见后来终于实行了。然而她是从四叔家出去就成了乞丐的呢,还是先到卫老婆子家然后再成乞丐的呢?那我可不知道。
    
    我给那些因为在近旁而极响的爆竹声惊醒,看见豆一般大的黄色的灯火光,接着又听得毕毕剥剥的鞭炮,是四叔家正在“祝福”了;知道已是五更将近时候。我在蒙胧\footnote{〔蒙胧〕朦胧。}中,又隐约听到远处的爆竹声联绵不断,似乎合成一天音响的浓云,夹着团团飞舞的雪花,拥抱了全市镇。我在这繁响的拥抱中,也懒散而且舒适,从白天以至初夜\footnote{〔初夜〕上半夜。}的疑虑,全给祝福的空气一扫而空了,只觉得天地圣众歆享了牲醴和香烟\footnote{〔天地圣众……〕祭祀的鬼神享受了祭品和香火。天地圣众:受祭祀的各路鬼神。歆享:鬼神享受祭品。牲醴:指祭祀用的牺牲和甜酒。香烟:香火。},都醉醺醺的在空中蹒跚,豫备给鲁镇的人们以无限的幸福。
    
\end{normalsize}


\newpage

\textbf{注释}:

\vspace{-1em}

\begin{itemize}
    \setlength\itemsep{-0.2em}
    \item 〔不更事〕经历世事不多,缺乏社会经验,不懂人情世故。
    \item 〔如所料〕和猜想的一致。如:依照、依从。料:猜想、估计。
    \item 〔一律〕同样,没有例外。
    \item 〔淡然〕不在意,不挂心。
    \item 〔俨然〕庄重的样子。
    \item 〔隐语〕不明确说出意思的用语。
    \item 〔月白〕极淡的蓝色,接近白色。
    \item 〔百无聊赖〕思想感情无所寄托,感到无聊。聊:依赖,寄托。
    \item 〔周正〕端正。
    \item 〔折罪〕抵罪,赎罪。
    \item 〔境遇〕环境和遭遇,受到的对待。
    \item 〔出格〕超出寻常的规矩。
    \item 〔沸反盈天〕像沸腾翻滚的水满天遍地。形容人声喧闹嘈杂,叫人难以忍受。
    \item 〔希图〕希望、企图。
    \item 〔捺〕用手往下按,抑制。
    \item 〔讪讪〕不好意思的样子。
    \item 〔蹙缩〕皱缩。
    \item 〔伶俐〕干脆爽快,干活麻利。
    \item 〔惴惴〕发愁又害怕的样子。
\end{itemize}

\chapter{雷雨}

\begin{normalsize}
    
    \noindent $\triangleright$~午饭后,天气很阴沉,郁热潮湿的空气,低压着屋内的人,使人烦躁。
    
    \noindent $\triangleright$~鲁侍萍正要出屋,周朴园点着一枝吕宋烟\footnote{〔吕宋烟〕菲律宾吕宋岛生产的雪茄烟,在二十世纪初的中国大城市富人中很有销路。},看见桌上的雨衣。
    
    \begin{description}[itemsep=1ex,leftmargin=3.5em,labelwidth=3em]
    
    \item[{\color{script-3-0} 周朴园}](向鲁侍萍)这是太太找出来的雨衣吗?
    
    \item[{\color{script-3-1} 鲁侍萍}](看着他)大概是的。
    
    \item[{\color{script-3-0} 周朴园}](拿起看看)不对,不对,这都是新的。我要我的旧雨衣,你回头跟太太说。
    
    \item[{\color{script-3-1} 鲁侍萍}]嗯。
    
    \item[{\color{script-3-0} 周朴园}](看她不走)你不知道这间房子底下人不准随便进来么?
    
    \item[{\color{script-3-1} 鲁侍萍}](看着他)不知道,老爷。
    
    \item[{\color{script-3-0} 周朴园}]你是新来的下人?
    
    \item[{\color{script-3-1} 鲁侍萍}]不是的,我找我的女儿来的。
    
    \item[{\color{script-3-0} 周朴园}]你的女儿?
    
    \item[{\color{script-3-1} 鲁侍萍}]四凤是我的女儿。
    
    \item[{\color{script-3-0} 周朴园}]那你走错屋子了。
    
    \item[{\color{script-3-1} 鲁侍萍}]哦。——老爷没有事了?
    
    \item[{\color{script-3-0} 周朴园}](指窗)窗户谁叫打开的?
    
    \item[{\color{script-3-1} 鲁侍萍}]哦。(很自然地走到窗户,关上窗户,慢慢地走向中门。)
    
    \item[{\color{script-3-0} 周朴园}](看她关好窗门,忽然觉得她很奇怪)你站一站,(鲁侍萍停)你——你贵姓?
    
    \item[{\color{script-3-1} 鲁侍萍}]我姓鲁。
    
    \item[{\color{script-3-0} 周朴园}]姓鲁。你的口音不像北方人。
    
    \item[{\color{script-3-1} 鲁侍萍}]对了,我不是,我是江苏的。
    
    \item[{\color{script-3-0} 周朴园}]你好像有点无锡口音。
    
    \item[{\color{script-3-1} 鲁侍萍}]我自小就在无锡长大的。
    
    \item[{\color{script-3-0} 周朴园}](沉思)无锡?嗯,无锡(忽而)你在无锡是什么时候?
    
    \item[{\color{script-3-1} 鲁侍萍}]光绪二十年,离现在有三十多年了。
    
    \item[{\color{script-3-0} 周朴园}]哦,三十年前你在无锡?
    
    \item[{\color{script-3-1} 鲁侍萍}]是的,三十多年前呢,那时候我记得我们还没有用洋火呢。
    
    \item[{\color{script-3-0} 周朴园}](沉思)三十多年前,是的,很远啦,我想想,我大概是二十多岁的时候。那时候我还在无锡呢。
    
    \item[{\color{script-3-1} 鲁侍萍}]老爷是那个地方的人?
    
    \item[{\color{script-3-0} 周朴园}]嗯,(沉吟)无锡是个好地方。
    
    \item[{\color{script-3-1} 鲁侍萍}]哦,好地方。
    
    \item[{\color{script-3-0} 周朴园}]你三十年前在无锡么?
    
    \item[{\color{script-3-1} 鲁侍萍}]是,老爷。
    
    \item[{\color{script-3-0} 周朴园}]三十年前,在无锡有一件很出名的事情——
    
    \item[{\color{script-3-1} 鲁侍萍}]哦。
    
    \item[{\color{script-3-0} 周朴园}]你知道么?
    
    \item[{\color{script-3-1} 鲁侍萍}]也许记得,不知道老爷说的是哪一件?
    
    \item[{\color{script-3-0} 周朴园}]哦,很远的,提起来大家都忘了。
    
    \item[{\color{script-3-1} 鲁侍萍}]说不定,也许记得的。
    
    \item[{\color{script-3-0} 周朴园}]我问过许多那个时候到过无锡的人,我想打听打听。可是呢个时候在无锡的人,到现在不是老了就是死了,活着的多半是不知道的,或者忘了。
    
    \item[{\color{script-3-1} 鲁侍萍}]如若老爷想打听的话,无论什么事,无锡那边我还有认识的人,虽然许久不通音信,托他们打听点事情总还可以的。
    
    \item[{\color{script-3-0} 周朴园}]我派人到无锡打听过。——不过也许凑巧你会知道。三十年前在无锡有一家姓梅的。
    
    \item[{\color{script-3-1} 鲁侍萍}]姓梅的?
    
    \item[{\color{script-3-0} 周朴园}]梅家的一个年轻小姐,很贤慧,也很规矩,有一天夜里,忽然地投水死了,后来,后来,——你知道么?
    
    \item[{\color{script-3-1} 鲁侍萍}]不敢说。
    
    \item[{\color{script-3-0} 周朴园}]哦。
    
    \item[{\color{script-3-1} 鲁侍萍}]我倒认识一个年轻的姑娘姓梅的。
    
    \item[{\color{script-3-0} 周朴园}]哦?你说说看。
    
    \item[{\color{script-3-1} 鲁侍萍}]可是她不是小姐,她也不贤慧,并且听说是不大规矩的。
    
    \item[{\color{script-3-0} 周朴园}]也许,也许你弄错了,不过你不妨说说看。
    
    \item[{\color{script-3-1} 鲁侍萍}]这个梅姑娘倒是有一天晚上跳的河,可是不是一个,她手里抱着一个刚生下三天的男孩。听人说她生前是不规矩的。
    
    \item[{\color{script-3-0} 周朴园}](苦痛)哦!
    
    \item[{\color{script-3-1} 鲁侍萍}]这是个下等人,不很守本分的。听说她跟那时周公馆的少爷有点不清白,生了两个儿子。生了第二个,才过三天,忽然周少爷不要了她,大孩子就放在周公馆,刚生的孩子抱在怀里,在年三十夜里投河死的。
    
    \item[{\color{script-3-0} 周朴园}](汗涔涔地)哦。
    
    \item[{\color{script-3-1} 鲁侍萍}]她不是小姐,她是无锡周公馆梅妈的女儿,她叫侍萍。
    
    \item[{\color{script-3-0} 周朴园}](抬起头来)你姓什么?
    
    \item[{\color{script-3-1} 鲁侍萍}]我姓鲁,老爷。
    
    \item[{\color{script-3-0} 周朴园}](喘出一口气,沉思地)侍萍,侍萍,对了。这个女孩子的尸首,说是有一个穷人见着埋了。你可以打听得她的坟在哪儿么?
    
    \item[{\color{script-3-1} 鲁侍萍}]老爷问这些闲事干什么?
    
    \item[{\color{script-3-0} 周朴园}]这个人跟我们有点亲戚。
    
    \item[{\color{script-3-1} 鲁侍萍}]亲戚?
    
    \item[{\color{script-3-0} 周朴园}]嗯,——我们想把她的坟墓修一修。
    
    \item[{\color{script-3-1} 鲁侍萍}]哦——那用不着了。
    
    \item[{\color{script-3-0} 周朴园}]怎么?
    
    \item[{\color{script-3-1} 鲁侍萍}]这个人现在还活着。
    
    \item[{\color{script-3-0} 周朴园}](惊愕)什么?
    
    \item[{\color{script-3-1} 鲁侍萍}]她没有死。
    
    \item[{\color{script-3-0} 周朴园}]她还在?不会吧?我看见她河边上的衣服,里面有她的绝命书。
    
    \item[{\color{script-3-1} 鲁侍萍}]不过她被一个慈善的人救活了。
    
    \item[{\color{script-3-0} 周朴园}]哦,救活啦?
    
    \item[{\color{script-3-1} 鲁侍萍}]以后无锡的人是没见着她,以为她那夜晚死了。
    
    \item[{\color{script-3-0} 周朴园}]那么,她呢?
    
    \item[{\color{script-3-1} 鲁侍萍}]一个人在外乡活着。
    
    \item[{\color{script-3-0} 周朴园}]那个小孩呢?
    
    \item[{\color{script-3-1} 鲁侍萍}]也活着。
    
    \item[{\color{script-3-0} 周朴园}](忽然立起)你是谁?
    
    \item[{\color{script-3-1} 鲁侍萍}]我是这儿四凤的妈,老爷。
    
    \item[{\color{script-3-0} 周朴园}]哦。
    
    \item[{\color{script-3-1} 鲁侍萍}]她现在老了,嫁给一个下等人,又生了个女孩,境况很不好。
    
    \item[{\color{script-3-0} 周朴园}]你知道她现在在哪儿?
    
    \item[{\color{script-3-1} 鲁侍萍}]我前几天还见着她!
    
    \item[{\color{script-3-0} 周朴园}]什么?她就在这儿?此地?
    
    \item[{\color{script-3-1} 鲁侍萍}]嗯,就在此地。
    
    \item[{\color{script-3-0} 周朴园}]哦!
    
    \item[{\color{script-3-1} 鲁侍萍}]老爷,你想见一见她么?
    
    \item[{\color{script-3-0} 周朴园}]不,不,谢谢你。
    
    \item[{\color{script-3-1} 鲁侍萍}]她的命很苦。离开了周家,周家少爷就娶了一位有钱有门第的小姐。她一个单身人,无亲无故,带着一个孩子在外乡什么事都做,讨饭,缝衣服,当老妈,在学校里伺候人。
    
    \item[{\color{script-3-0} 周朴园}]她为什么不再找到周家?
    
    \item[{\color{script-3-1} 鲁侍萍}]大概她是不愿意吧?为着她自己的孩子,她嫁过两次。
    
    \item[{\color{script-3-0} 周朴园}]以后她又嫁过两次?
    
    \item[{\color{script-3-1} 鲁侍萍}]嗯,都是很下等的人。她遇人都很不如意,老爷想帮一帮她么?
    
    \item[{\color{script-3-0} 周朴园}]好,你先下去。让我想一想。
    
    \item[{\color{script-3-1} 鲁侍萍}]老爷,没有事了?(望着朴园,眼泪要涌出)老爷,您那雨衣,我怎么说?
    
    \item[{\color{script-3-0} 周朴园}]你去告诉四凤,叫她把我樟木箱子里那件旧雨衣拿出来,顺便把那箱子里的几件旧衬衣也捡出来。
    
    \item[{\color{script-3-1} 鲁侍萍}]旧衬衣?
    
    \item[{\color{script-3-0} 周朴园}]你告诉她在我那顶老的箱子里,纺绸的衬衣,没有领子的。
    
    \item[{\color{script-3-1} 鲁侍萍}]老爷那种纺绸衬衣不是一共有五件?您要哪一件?
    
    \item[{\color{script-3-0} 周朴园}]要哪一件?
    
    \item[{\color{script-3-1} 鲁侍萍}]不是有一件,在右袖襟上有个烧破的窟窿,后来用丝线绣成一朵梅花补上的?还有一件,——
    
    \item[{\color{script-3-0} 周朴园}](惊愕)梅花?
    
    \item[{\color{script-3-1} 鲁侍萍}]还有一件绸衬衣,左袖襟也绣着一朵梅花,旁边还绣着一个萍字。还有一件,——
    
    \item[{\color{script-3-0} 周朴园}](徐徐立起)哦,你,你,你是——
    
    \item[{\color{script-3-1} 鲁侍萍}]我是从前伺候过老爷的下人。
    
    \item[{\color{script-3-0} 周朴园}]哦,侍萍!(低声)怎么,是你?
    
    \item[{\color{script-3-1} 鲁侍萍}]你自然想不到,侍萍的相貌有一天也会老得连你都不认识了。
    
    \item[{\color{script-3-0} 周朴园}]你——侍萍?(不觉地望望柜上的相片,又望鲁侍萍。)
    
    \item[{\color{script-3-1} 鲁侍萍}]朴园,你找侍萍么?侍萍在这儿。
    
    \item[{\color{script-3-0} 周朴园}](忽然严厉地)你来干什么?
    
    \item[{\color{script-3-1} 鲁侍萍}]不是我要来的。
    
    \item[{\color{script-3-0} 周朴园}]谁指使你来的?
    
    \item[{\color{script-3-1} 鲁侍萍}](悲愤)命!不公平的命指使我来的。
    
    \item[{\color{script-3-0} 周朴园}](冷冷地)三十年的工夫你还是找到这儿来了。
    
    \item[{\color{script-3-1} 鲁侍萍}](愤怨)我没有找你,我没有找你,我以为你早死了。我今天没想到到这儿来,这是天要我在这儿又碰见你。
    
    \item[{\color{script-3-0} 周朴园}]你可以冷静点。现在你我都是有子女的人,如果你觉得心里有委屈,这么大年纪,我们先可以不必哭哭啼啼的。
    
    \item[{\color{script-3-1} 鲁侍萍}]哭?哼,我的眼泪早哭干了,我没有委屈,我有的是恨,是悔,是三十年一天一天我自己受的苦。你大概已经忘了你做的事了!三十年前,过年三十的晚上我生下你的第二个儿子才三天,你为了要赶紧娶那位有钱有门第的小姐,你们逼着我冒着大雪出去,要我离开你们周家的门。
    
    \item[{\color{script-3-0} 周朴园}]从前的恩怨, 过了几十年,又何必再提呢?
    
    \item[{\color{script-3-1} 鲁侍萍}]那是因为周大少爷一帆风顺,现在也是社会上的好人物。可是自从我被你们家赶出来以后,我没有死成,我把我的母亲可给气死了,我亲生的两个孩子你们家里逼着我留在你们家里。
    
    \item[{\color{script-3-0} 周朴园}]你的第二个孩子你不是已经抱走了么?
    
    \item[{\color{script-3-1} 鲁侍萍}]那是你们老太太看着孩子快死了,才叫我抱走的。(自语)哦,天哪,我觉得我像在做梦。
    
    \item[{\color{script-3-0} 周朴园}]我看过去的事不必再提起来吧。
    
    \item[{\color{script-3-1} 鲁侍萍}]我要提,我要提,我闷了三十年了!你结了婚,就搬了家,我以为这一辈子也见不着你了;谁知道我自己的孩子个个命定要跑到周家来,又做我从前在你们家做过的事。
    
    \item[{\color{script-3-0} 周朴园}]怪不得四凤这样像你。
    
    \item[{\color{script-3-1} 鲁侍萍}]我伺候你,我的孩子再伺候你生的少爷们。这是我的报应,我的报应。
    
    \item[{\color{script-3-0} 周朴园}]你静一静。把脑子放清醒点。你不要以为我的心是死了,你以为一个人做了一件于心不忍的是就会忘了么?你看这些家俱都是你从前顶喜欢的动向,多少年我总是留着,为着纪念你。
    
    \item[{\color{script-3-1} 鲁侍萍}](低头)哦。
    
    \item[{\color{script-3-0} 周朴园}]你的生日——四月十八——每年我总记得。一切都照着你是正式嫁过周家的人看,甚至于你因为生萍儿,受了病,总要关窗户,这些习惯我都保留着,为的是不忘你,祢补我的罪过。
    
    \item[{\color{script-3-1} 鲁侍萍}](叹一口气)现在我们都是上了年纪的人,这些傻话请你不必说了。
    
    \item[{\color{script-3-0} 周朴园}]那更好了。那么我见可以明明白白地谈一谈。
    
    \item[{\color{script-3-1} 鲁侍萍}]不过我觉得没有什么可谈的。
    
    \item[{\color{script-3-0} 周朴园}]话很多。我看你的性情好像没有大改,——鲁贵像是个很不老实的人。
    
    \item[{\color{script-3-1} 鲁侍萍}]你不明白。他永远不会知道的。
    
    \item[{\color{script-3-0} 周朴园}]那双方面都好。再有,我要问你的,你自己带走的儿子在哪儿?
    
    \item[{\color{script-3-1} 鲁侍萍}]他在你的矿上做工。
    
    \item[{\color{script-3-0} 周朴园}]我问,他现在在哪儿?
    
    \item[{\color{script-3-1} 鲁侍萍}]就在门房等着见你呢。
    
    \item[{\color{script-3-0} 周朴园}]什么?鲁大海?他!我的儿子?
    
    \item[{\color{script-3-1} 鲁侍萍}]他的脚趾头因为你的不小心,现在还是少一个的。
    
    \item[{\color{script-3-0} 周朴园}](冷笑)这么说,我自己的骨肉在矿上鼓励罢工,反对我!
    
    \item[{\color{script-3-1} 鲁侍萍}]他跟你现在完完全全是两样的人。
    
    \item[{\color{script-3-0} 周朴园}](沉静)他还是我的儿子。
    
    \item[{\color{script-3-1} 鲁侍萍}]你不要以为他还会认你做父亲。
    
    \item[{\color{script-3-0} 周朴园}](忽然)好!痛痛快快地!你现在要多少钱吧?
    
    \item[{\color{script-3-1} 鲁侍萍}]什么?
    
    \item[{\color{script-3-0} 周朴园}]留着你养老。
    
    \item[{\color{script-3-1} 鲁侍萍}](苦笑)哼,你还以为我是故意来敲诈你,才来的么?
    
    \item[{\color{script-3-0} 周朴园}]也好,我们暂且不提这一层。那么,我先说我的意思。你听着,鲁贵我现在要辞退的,四凤也要回家。不过——
    
    \item[{\color{script-3-1} 鲁侍萍}]你不要怕,你以为我会用这种关系来敲诈你么?你放心,我不会的。大后天我就会带四凤回到我原来的地方。这是一场梦,这地方我绝对不会再住下去。
    
    \item[{\color{script-3-0} 周朴园}]好得很,那么一切路费,用费,都归我担负。
    
    \item[{\color{script-3-1} 鲁侍萍}]什么?
    
    \item[{\color{script-3-0} 周朴园}]这于我的心也安一点。
    
    \item[{\color{script-3-1} 鲁侍萍}]你?(笑)三十年我一个人都过了,现在我反而要你的钱?
    
    \item[{\color{script-3-0} 周朴园}]好,好,好,那么你现在要什么?
    
    \item[{\color{script-3-1} 鲁侍萍}](停一停)我,我要点东西。
    
    \item[{\color{script-3-0} 周朴园}]什么?说吧?
    
    \item[{\color{script-3-1} 鲁侍萍}](泪满眼)我——我只要见见我的萍儿。
    
    \item[{\color{script-3-0} 周朴园}]你想见他?
    
    \item[{\color{script-3-1} 鲁侍萍}]嗯,他在哪儿?
    
    \item[{\color{script-3-0} 周朴园}]他现在在楼上陪着他的母亲看病。我叫他,他就可以下来见你。不过是——
    
    \item[{\color{script-3-1} 鲁侍萍}]不过是什么?
    
    \item[{\color{script-3-0} 周朴园}]他很大了。
    
    \item[{\color{script-3-1} 鲁侍萍}](追忆)他大概是二十八了吧?我记得他比大海只大一岁。
    
    \item[{\color{script-3-0} 周朴园}]并且他以为他母亲早就死了的。
    
    \item[{\color{script-3-1} 鲁侍萍}]哦,你以为我会哭哭啼啼地叫他认母亲么?我不会那么傻的。我难道不知道这样的母亲只给自己的儿子丢人么?我明白他的地位,他的教育,不容他承认这样的母亲。这些年我也学乖了,我只想看看他,他究竟是我生的孩子。你不要怕,我就是告诉他,白白地增加他的烦恼,他自己也不愿意认我的。
    
    \item[{\color{script-3-0} 周朴园}]那么,我们就这样解决了。我叫他下来,你看一看他,以后鲁家的人永远不许再到周家来。
    
    \item[{\color{script-3-1} 鲁侍萍}]好,希望这一生不至于再见你。
    
    \item[{\color{script-3-0} 周朴园}](由衣内取出皮夹的支票\footnote{〔支票〕从银行支出钱款的票证。支票由银行为客户制作。客户在支票上写明金额、收款人、日期等信息,并签名,收款人即可凭支票到该银行支取相应钱款。}签好)很好,这一张五千块钱的支票,你可以先拿去用。算是拟补我一点罪过。
    
    \item[{\color{script-3-1} 鲁侍萍}](接过支票)谢谢你。(慢慢撕碎支票)
    
    \item[{\color{script-3-0} 周朴园}]侍萍。
    
    \item[{\color{script-3-1} 鲁侍萍}]我这些年的苦不是你那钱就算得清的。
    
    \item[{\color{script-3-0} 周朴园}]可是你——
    
    \end{description}
    
    \noindent $\triangleright$~外面争吵声。鲁大海的声音:“放开我,我要进去。”三四个男仆声:“不成 ,不成,老爷睡觉呢。”门外有男仆等与大海的挣扎声。
    
    \begin{description}[itemsep=1ex,leftmargin=3.5em,labelwidth=3em]
    
    \item[{\color{script-3-0} 周朴园}](走至中门)来人!(仆人由中门进)谁在吵?
    
    \item[{\color{script-3-5} 仆人}]就是那个工人鲁大海!他不讲理,非见老爷不可。
    
    \item[{\color{script-3-0} 周朴园}]哦。(沉吟)那你叫他进来吧。等一等,叫人到楼上请大少爷下楼,我有话问他。
    
    \item[{\color{script-3-5} 仆人}]是,老爷。
    
    \end{description}
    
    \noindent $\triangleright$~仆人由中门下。
    
    \begin{description}[itemsep=1ex,leftmargin=3.5em,labelwidth=3em]
    
    \item[{\color{script-3-0} 周朴园}](向鲁侍萍)侍萍,你不要太固执。这一点钱你不收下,将来你会后悔的。
    
    \item[{\color{script-3-1} 鲁侍萍}](望着他,一句话也不说。)
    
    \end{description}
    
    \noindent $\triangleright$~仆人领着大海进,大海站在左边,三四仆人立一旁。
    
    \begin{description}[itemsep=1ex,leftmargin=3.5em,labelwidth=3em]
    
    \item[{\color{script-3-2} 鲁大海}](见鲁侍萍)妈,您还在这儿?
    
    \item[{\color{script-3-0} 周朴园}](打量鲁大海)你叫什么名字?
    
    \item[{\color{script-3-2} 鲁大海}](大笑)董事长,您不要向我摆架子,您难道不知道我是谁么?
    
    \item[{\color{script-3-0} 周朴园}]你?我只知道你是罢工闹得最凶的工人代表。
    
    \item[{\color{script-3-2} 鲁大海}]对了,一点儿也不错,所以才来拜望拜望您。
    
    \item[{\color{script-3-0} 周朴园}]你有什么事吧?
    
    \item[{\color{script-3-2} 鲁大海}]董事长当然知道我是为什么来的。
    
    \item[{\color{script-3-0} 周朴园}](摇头)我不知道。
    
    \item[{\color{script-3-2} 鲁大海}]我们老远从矿上来,今天我又在您府上大门房里从早上六点钟一直等到现在,我就是要问问董事长,对于我们工人的条件,究竟是允许不允许?
    
    \item[{\color{script-3-0} 周朴园}]哦,那么——那么,那三个代表呢?
    
    \item[{\color{script-3-2} 鲁大海}]我跟你说吧,他们现在正在联络旁的工会呢。
    
    \item[{\color{script-3-0} 周朴园}]哦,——他们没告诉旁的事情么?
    
    \item[{\color{script-3-2} 鲁大海}]告诉不告诉于你没有关系。——我问你,你的意思,忽而软,忽而硬,究竟是怎么回事?
    
    \end{description}
    
    \noindent $\triangleright$~周萍由饭厅上,见有人,即想退回。
    
    \begin{description}[itemsep=1ex,leftmargin=3.5em,labelwidth=3em]
    
    \item[{\color{script-3-0} 周朴园}](看周萍)不要走,萍儿!(看向鲁侍萍,鲁侍萍知周萍为其子,眼泪汪汪地望着他。)
    
    \item[{\color{script-3-3} 周萍}]是,爸爸。
    
    \item[{\color{script-3-0} 周朴园}](指身侧)萍儿,你站在这儿。(向鲁大海)你这么只凭意气是不能交涉事情的。
    
    \item[{\color{script-3-2} 鲁大海}]哼,你们的手段,我都明白。你们这样拖延时候不就是想去花钱收买少数不要脸的败类,暂时把我们骗在这儿。
    
    \item[{\color{script-3-0} 周朴园}]你的见地也不是没有道理。
    
    \item[{\color{script-3-2} 鲁大海}]可是你完全错了。我们这次罢工是有团结的,有组织的。我们代表这次来并不是来求你们。你听清楚,不求你们。你们允许就允许;不允许,我们一直罢工到底,我们知道你们不到两个月整个地就要关门的。
    
    \item[{\color{script-3-0} 周朴园}]你以为你们那些代表们,那些领袖们都可靠吗?
    
    \item[{\color{script-3-2} 鲁大海}]至少比你们只认识洋钱的结合要可靠得多。
    
    \item[{\color{script-3-0} 周朴园}]那么我给你一件东西看。
    
    \end{description}
    
    \noindent $\triangleright$~周朴园在桌上找电报,仆人递给他;此时周冲偷偷由左书房进,在旁偷听。
    
    \begin{description}[itemsep=1ex,leftmargin=3.5em,labelwidth=3em]
    
    \item[{\color{script-3-0} 周朴园}](给大海电报)这是昨天从矿上来的电报。
    
    \item[{\color{script-3-2} 鲁大海}](拿过去看)什么?他们又上工了。(放下电报)不会,不会。
    
    \item[{\color{script-3-0} 周朴园}]矿上的工人已经在昨天早上复工,你当代表的反而不知道么?
    
    \item[{\color{script-3-2} 鲁大海}](惊,怒)怎么矿上警察开枪打死三十个工人就白打了么?(又看电报,忽然笑起来)哼,这是假的。你们自己假作的电报来离间我们的。(笑)哼,你们这种卑鄙无赖的行为!
    
    \item[{\color{script-3-3} 周萍}](忍不住)你是谁?敢在这儿胡说?
    
    \item[{\color{script-3-0} 周朴园}]萍儿!没有你的话。(低声向大海)你就这样相信你那同来的代表么?
    
    \item[{\color{script-3-2} 鲁大海}]你不用多说,我明白你这些话的用意。
    
    \item[{\color{script-3-0} 周朴园}]好,那我把那复工的合同给你瞧瞧。
    
    \item[{\color{script-3-2} 鲁大海}](笑)你不要骗小孩子,复工的合同没有我们代表的签字是不生效力的。
    
    \item[{\color{script-3-0} 周朴园}]哦,(向仆人)合同!(仆人由桌上拿合同递他)你看,这是他们三个人签字的合同。
    
    \item[{\color{script-3-2} 鲁大海}](看合同)什么?(慢慢地,低声)他们三个人签了字。他们怎么会不告诉我就签了字呢?他们就这样把我不理啦?
    
    \item[{\color{script-3-0} 周朴园}]对了,傻小子,没有经验只会胡喊是不成的。
    
    \item[{\color{script-3-2} 鲁大海}]那三个代表呢?
    
    \item[{\color{script-3-0} 周朴园}]昨天晚车就回去了。
    
    \item[{\color{script-3-2} 鲁大海}](如梦初醒)他们三个就骗了我了,这三个没有骨头的东西,他们就把矿上的工人们卖了。哼,你们这些不要脸的董事长,你们的钱这次又灵了。
    
    \item[{\color{script-3-3} 周萍}](怒)你混帐!
    
    \item[{\color{script-3-0} 周朴园}]不许多说话。(回头向鲁大海)鲁大海,你现在没有资格跟我说话——矿上已经把你开除了。
    
    \item[{\color{script-3-2} 鲁大海}]开除了?
    
    \item[{\color{script-3-4} 周冲}]爸爸,这是不公平的。
    
    \item[{\color{script-3-0} 周朴园}](向周冲)你少多嘴,出去!(周冲由中门走下)
    
    \item[{\color{script-3-2} 鲁大海}]哦,好,好,(切齿)你的手段我早就领教过,只要你能弄钱,你什么都做得出来。你叫警察杀了矿上许多工人,你还——
    
    \item[{\color{script-3-0} 周朴园}]你胡说!
    
    \item[{\color{script-3-1} 鲁侍萍}](至鲁大海前)别说了,走吧。
    
    \item[{\color{script-3-2} 鲁大海}]哼,你的来历我都知道,你从前在哈尔滨包修江桥,故意在叫江堤出险——
    
    \item[{\color{script-3-0} 周朴园}](低声)下去!
    
    \end{description}
    
    \noindent $\triangleright$~仆人等拉他,说“走!走!”
    
    \begin{description}[itemsep=1ex,leftmargin=3.5em,labelwidth=3em]
    
    \item[{\color{script-3-2} 鲁大海}](对仆人)你们这些混帐东西,放开我。我要说,你故意淹死了二千二百个小工,每一个小工的性命你扣三百块钱!姓周的,你发的是绝子绝孙的昧心财!你现在还——
    
    \item[{\color{script-3-3} 周萍}](忍不住气,走到大海面前,重重地大他两个嘴巴。)你这种混帐东西!(大海立刻要还手,倒是被周宅的仆人们拉住。)打他。
    
    \item[{\color{script-3-2} 鲁大海}](向周萍高声)你,你(正要骂,众仆人一起打鲁大海。鲁大海头流血。鲁侍萍哭喊着护鲁大海。 )
    
    \item[{\color{script-3-0} 周朴园}](厉声)不要打人!(仆人们停止打鲁大海,仍拉着鲁大海的手。)
    
    \item[{\color{script-3-2} 鲁大海}]放开我,你们这一群强盗!
    
    \item[{\color{script-3-3} 周萍}](向仆人)把他拉下去。
    
    \item[{\color{script-3-1} 鲁侍萍}](大哭起来)哦,这真是一群强盗!(走至周萍前,抽咽)你是萍——凭——凭什么打我的儿子?
    
    \item[{\color{script-3-3} 周萍}]你是谁?
    
    \item[{\color{script-3-1} 鲁侍萍}]我是你的——你打的这个人的妈。
    
    \item[{\color{script-3-2} 鲁大海}]妈,别理这东西,您小心吃了他们的亏。
    
    \item[{\color{script-3-1} 鲁侍萍}](呆呆地看着周萍的脸,忽而又大哭起来)大海,走吧,我们走吧。(抱着鲁大海受伤的头哭。)
    
    \end{description}
    
    \noindent $\triangleright$~鲁大海被仆人们拥下,鲁侍萍随下。
    
    
\end{normalsize}


\newpage

\textbf{注释}:

\vspace{-1em}

\begin{itemize}
    \setlength\itemsep{-0.2em}
    \item 〔汗涔涔〕汗水不断往下流的样子。
    \item 〔纺绸〕一种平纹丝织品,质地薄而细软。
    \item 〔昧心〕违背本意,违背良心。昧:隐藏,违背。
    \item 〔如梦初醒〕像刚从梦中醒来。形容突然醒悟的样子。
\end{itemize}

\chapter{林教头风雪山神庙}

\begin{normalsize}
    
    话说当日林冲正闲走间,忽然背后人叫。回头看时,却认得是酒生儿李小二。当初在东京时,多得林冲看顾;后来不合偷了店主人家财,被捉住了,要送官司同罪,却得林冲主张陪话,救了他免送官司。又也陪了些钱财,方得脱免。京中安不得身,又亏林冲赍发\footnote{〔赍发〕赠与(财物)。}他盘缠,于路投奔人。不想今日却在这里撞见。林冲道:“小二哥,你如何也在这里?”李小二便拜道:“自从得恩人救济,赍发小人,一地里投奔人不着。迤逦不想来到沧州,托一个酒店里姓王,留小人在店中做过卖。因见小人勤谨,安排的好菜蔬,调和的好汁水,来吃的人都喝采,以此买卖顺当。主人家有个女儿,就招了小人做女婿。如今丈人丈母都死了,只剩得小人夫妻两个,权在营前开了个茶酒店。因讨钱过来,遇见恩人。恩人不知为何事在这里?”林冲指着脸上道:“我因恶了高太尉,生事陷害,受了一场官司,刺配到这里。如今叫我管天王堂,未知久后如何。不想今日到此遇见。”李小二就请林冲到家里面坐定,叫妻子出来拜了恩人。两口儿欢喜道:我夫妻二人正没个亲眷,今日得恩人到来,便是从天降下。“林冲道:我是罪囚,恐怕玷辱你夫妻两个。”李小二道:“谁不知恩人大名,休恁地说。但有衣服,便拿来家里浆洗缝补。”当时管待林冲酒食,至晚送回天王堂。次日,又来相请。因此林冲得李小二家来往,不时间送汤送水,来营里与林冲吃。林冲因见他两口儿恭勤孝顺,常把些银两与他做本钱。
    
    且把闲话休题,只说正话。迅速光阴,却早冬来。林冲的绵衣裙袄,都是李小二浑身整治缝补。忽一日,李小二正在门前安排菜蔬下饭,只见一个人闪将进来,酒店里坐下,随后又一人入来。看时,前面那个人是军官打扮,后面这个走卒模样。跟着也来坐下。李小二入来问道:“要吃酒?”只见那个人将出一两银子与小二道:“且收放柜上,取三四瓶好酒来。客到时,果品酒馔只顾将来,不必要问。”李小二道:“官人请甚客?”那人道:“烦你与我去营里请管营、差拨两个来说话。问时,你只说有个官人请说话,商议些事务。专等,专等。”李小二应承了,来到牢城里,先请了差拨,同到管营家里,请了管营,都到酒店里。只见那个官人和管营、差拨两个讲了礼。管营道:“素不相识,动问官人高姓大名。”那人道:“有书在此,少刻便知。且取酒来。”李小二连忙开了酒,一面铺下菜蔬果品酒馔。那人叫讨副劝盘来,把了盏,相让坐了。小二独自一个撺梭\footnote{〔撺梭〕穿梭,往来频繁。}也似扶侍不暇。那跟来的人,讨了汤桶,自行荡酒。约计吃过十数杯,再讨了按酒,铺放桌上。只见那人说道:“我自有伴当荡酒。不叫,你休来。我等自要说话。”李小二应了,自来门首叫老婆道:“大姐,这两个人来的不尴尬\footnote{〔不尴尬〕好不尴尬,指尴尬。尴尬:鬼鬼祟祟,不正派。}。”老婆道:“怎么的不尴尬?”小二道:“这两个人语言声音是东京人。初时又不认得管营。向后我将按酒入去,只听得差拨口里呐出一句‘高太尉’三个字来。这人莫不与林教头身上有些干碍?我自在门前理会。你且去阁子背后,听说什么。”老婆道:“你去营中寻林教头来认他一认。”李小二道:“你不省得。林教头是个性急的人。摸不着便要杀人放火。倘或叫的他来看了,正是前日说的什么陆虞候,他肯便罢?做出事来,须连累了我和你。你只去听一听再理会。老婆道:“说的是。”便入去听了一个时辰,出来说道:“他那三四个交头接耳说话,正不听得说什么。只见那一个军官模样的人,去伴当怀里,取出一帕子物事,逃与管营和差拨。帕子里面的莫不是金银。只听差拨口里说道:‘都在我身上,好歹要结果了他性命。’……”正说之间,阁子里叫:“将汤来。”李小二急去里面换汤时,看见管营手里拿着一封书。小二换了汤,添些下饭。又吃了半个时辰,算还了酒钱。管营、差拨先去了。次后,那两个低着头也去了。
    
    转背没多时,只见林冲走将入店里来,说道:“小二哥,连日好买卖。”李小二慌忙道:“恩人请坐。小人却待正要寻恩人,有些要紧话说。”林冲问道:“什么要紧的事?”小二哥请林冲到里面坐下,说道:“却才有个东京来的尴尬人,在我这里请管营、差拨吃了半日酒。差拨口里呐出‘高太尉’三个字来。小人心下疑惑。又着浑家听了一个时辰,他却交头接耳说话,都不听得。临了只见差拨口里应道:‘都在我两个身上,好歹要结果了他。’那两个把一包金银,都与管营、差拨。又吃一回酒,各自散了。不知什么样人。小人心疑,只怕恩人身上有些妨碍。”林冲道:“那人生得什么模样?”李小二道:“五短身材,白净面皮,没什髭须。约有三十余岁。那跟的也不长大,紫棠色面皮。”林冲听了,大惊道:“这三十岁的正是陆虞候。那泼贱贼也敢来这里害我!休要撞着我,只教他骨肉为泥!”李小二道:“只要提防他便了。岂不闻古人言:‘吃饭防噎,走路防跌。’”
    
    林冲大怒,离了李小二家,先去街上买把解腕尖刀\footnote{〔解腕尖刀〕一种短匕首,通常用来割肉分骨。},带在身上。前街后巷,一地里去寻。李小二夫妻两个,捏着两把汗。当晚无事。次日,天明起来,早洗漱罢,带了刀又去沧州城里城外,小街夹巷,团团寻了一日。牢城营里都没动静。林冲又来对李小二道:“今日又无事。”小二道:“恩人,只愿如此。只是自放仔细便”了。林冲自回天王堂,过了一夜。街上寻了三五日,不见消耗,林冲也自心下慢了。
    
    到第六日,只见管营叫唤林冲到点视厅上,说道:“你来这里许多时,柴大官人面皮不曾抬举的你。此间东门外十五里,有座大军草场,每月但是纳草纳料的,有些常例\footnote{〔常例〕指“常例钱”,官吏向人勒索的一种名目。}钱取觅。原是一个老军看管。如今,我抬举你去替那老军来守天王堂。你在那里寻几贯盘缠。你可和差拨便去那里交割。”林冲应道:“小人便去。”当时离了营中,径到李小二家,对他夫妻两个说道:“今日管营拨我去大军草场管事,却如何?”李小二道:“这个差使又好似天王堂。那里收草料时,有些常例钱钞。往常不使钱时,不能够得这差使。”林冲道:“却不害我,倒与我好差使,正不知何意?”李小二道:“恩人休要疑心。只要没事便好了。只是小人家离得远了,过几时那工夫来望恩人。”就时家里安排几杯酒,请林冲吃了。
    
    话不絮烦\footnote{〔话不絮烦〕说话不要半天讲不清,絮絮叨叨的,不着重点,令人厌烦。},两个相别了。林冲自来天王堂取了包裹,带了尖刀,拿了条花枪,与差拨一同辞了管营。两个取路投草料场来。正是严冬天气,彤云密布,朔风\footnote{〔朔风〕北风。}渐起,却早纷纷扬扬卷下一天大雪来。林冲和差拨两个,在路上又没买酒吃处,早来到草料场外。看时,一周遭有些黄土墙,两扇大门,推开看里面时,七八间草房做着仓廒\footnote{〔仓廒〕储藏粮食的仓库。},四下里都是马草堆,中间两座草厅。到那厅里,只见那老军在里面向火。差拨说道:“管营差这个林冲来替你回天王堂看守。你可即使交割。”老军拿了钥匙,引着林冲,分付道:“仓廒内自有官司封记。这几堆草,一堆堆都有数目。”老军都点见了堆数,又引林冲到草厅上。老军收拾行李,临了说道:“火盆锅子碗碟,都借与你。”林冲道:“天王堂内,我也有在那里。你要便拿了去。”老军指壁上挂一个大葫芦说道:“你若买酒吃时,只出草场,投东大路去三二里,便有市井。”老军自和差拨回营里来。
    
    只说林冲就床上放了包裹被卧,就坐下生些焰火起来。屋边有一堆柴炭,拿几块来,生在地炉里。仰面看那草屋时,四下里崩坏了,又被朔风吹撼,摇振得动。林冲道:“这屋如何过得一冬?待雪晴了,去城中唤个泥水匠来修理。”向了一回火,觉得身上寒冷。寻思:“却才老军所说,五里路外有那市井,何不去沽些酒来吃?”便去包里取些碎银子,把花枪挑了酒葫芦,将火炭盖了,取毡笠子戴上,拿了钥匙,出来把草厅门拽上。出到大门首,把两扇草场门反拽上锁了。带了钥匙,信步投东。雪地里踏着碎琼乱玉,迤逦背着北风而行。那雪正下得紧。
    
    行不上半里多路,看见一所古庙。林冲顶礼道:“神明庇佑,改日来烧钱纸。”又行了一回,望见一簇人家。林冲住脚看时,见篱笆中挑着一个草帚儿在露天里。林冲迳到店里。主人道:“客人那里来?”林冲道:“你认得这个葫芦么?”主人看了道:“这葫芦是草料场老军的。”林冲道:“如何便认的?”店主道:“既是草料场看守大哥,且请少坐。天气寒冷,且酌三杯,权当接风。”店家切一盘熟牛肉,荡一壶热酒,请林冲吃。又自买了些牛肉,又吃了数杯,就又买了一葫芦酒,包了那两块牛肉,留下碎银子,把花枪挑了酒葫芦,怀内揣了牛肉,叫声“相扰”,便出篱笆门,依旧迎着朔风回来。看那雪到晚越下的紧了。
    
    再说林冲踏着那瑞雪,迎着北风,飞也似奔到草场门口,开了锁入内看时,只叫得苦。原来天理昭然,佑护善人义士。因这场大雪,救了林冲的性命。那两间草厅,已被雪压倒了。林冲寻思:“怎地好?”放下花枪、葫芦在雪里,恐怕火盆内有火炭延烧起来。搬开破壁子,探半身入去摸时,火盆内火种,都被雪水浸灭了。林冲把手床上摸时,只拽的一条絮被。林冲钻将出来,见天色黑了。寻思:“又没打火处,怎生安排?”想起:“离了这半里路上,有个古庙,可以安身。我且去那里宿一夜。等到天明,却做理会。”把被卷了,花枪挑着酒葫芦,依旧把门拽上锁了,望那庙里来。入的庙门,再把门掩上,傍边止有一块大石头,掇将过来靠了门。入的里面看时,殿上做着一尊金甲山神。两边一个判官\footnote{〔判官〕唐宋时辅助地方长官处理公事的人员,道教借指阎王手下掌管生死簿的官。},一个小鬼。侧边推着一堆纸。团团看来,又没邻舍,又无庙主。林冲把枪和酒葫芦放在纸堆上,将那条絮被放开,先取下毡笠子,把身上雪都抖了,把上盖白布衫脱将下来。早有五分湿了。和毡笠放在供桌上。把被扯来盖了半截下身。却把葫芦冷酒提来便吃。就将怀中牛肉下酒。
    
    正吃时,只听得外面必必剥剥地爆响。林冲跳起身来,就壁缝里看时,只见草料场里火,刮刮杂杂烧着。当时林冲便拿枪,却待开门来救火,只听得前面有人说将话来。林冲就伏在庙听时,是三个人脚步响,且奔庙里来。用手推门,却被林冲靠住了,推也推不开。三人在庙檐下立地看火。数内一个道:“这条计好么?”一个应道:“端的亏管营、差拨两位用心。回到京师,禀过太尉,都保你二位做大官。这番张教头没的推故。”那人道:“林冲今番直吃我们对付了。高衙内这病必然好了。”又一个道:“张教头那厮,三回五次托人情去说‘你的女婿殁了’,张教头越不肯应承,因此衙内病患看看重了。太尉特使俺两个央浼\footnote{〔央浼〕恳求,拜托。}二位干这件事。不想而今完备了。”又一个道:“小人直爬入墙里去,四下草堆上点了十来个火把,待走那里去?”那一个道:“这早晚烧个八分过了。”又听一个道:“便逃得性命时,烧了大军草料场,也得个死罪。”又一个道:“我们回城里去罢。”一个道:“再看一看,拾得他一两块骨头回京府里见太尉和衙内时,也道我们也能会干事。”
    
    林冲听那三个人时,一个是差拨,一个是陆虞候,一个是富安。林冲心道:天可怜见林冲!若不是倒了草厅,我准定被这厮们烧死了!轻轻把石头掇开,挺着花枪,一手拽开庙门,大喝一声:“泼贼那里去!”三个人急要走时,惊得呆了,正走不动。林冲举手,胳察的一枪,先戳倒差拨。陆虞候叫声“饶命!”吓的慌了手脚,走不动。那富安走不到十来步,被林冲赶上,后心只一枪,又戳倒了。翻身回来,陆虞候却才行的三四步。林冲喝声道:“奸贼!你待那里去?”劈胸只一提,丢翻在雪地上,把枪搠在地里,用脚踏住胸脯,身边取出那口刀来,便去陆谦脸上阁着,喝道:“泼贼!我自来又和你无什么冤仇,你如何这等害我!正是‘杀人可恕,情理难容’!”陆虞候告道:“不干小人事,太尉差遣,不敢不来。”林冲骂道:“奸贼,我与你自幼相交,今日倒来害我,怎不干你事!且吃我一刀。”把陆谦上身衣服扯开,把尖刀向心窝里只一剜,七窍迸出血来。将心肝提在手里。回头看时,差拨正爬将起来要走。林冲按住喝道:“你这厮原来也恁的歹,且吃我一刀。”又早把头割下来,挑在枪上。回来把富安、陆谦头都割下来。把尖刀插了,将三个人头发结做一处,提入庙里来,都摆在山神面前供桌上,再穿了白布衫,系了胳膊,把毡笠子带上,将葫芦里冷酒都吃尽了。被与葫芦都丢了不要。提了枪,便出庙门投东去……
    
\end{normalsize}


\newpage

\textbf{注释}:

\vspace{-1em}

\begin{itemize}
    \setlength\itemsep{-0.2em}
    \item 〔迳〕径,直接。
    \item 〔拽〕拉,用手牵引。
    \item 〔搠〕戳,刺,插。
    \item 〔庇佑〕庇护保佑。
    \item 〔殁〕死。
    \item 〔迤逦〕曲折辗转。
\end{itemize}

\chapter{套中人}

\begin{normalsize}
    
    在米罗诺西茨村边,在村长普罗科菲的堆房里,误了归时的猎人们正安顿下来过夜。他们只有二人:兽医伊凡·伊凡内奇和中学教员布尔金。伊凡·伊凡内奇有个相当古怪的复姓:奇木沙-喜马拉雅斯基,这个姓跟他很不相称\footnote{〔这个姓跟他很不相称〕因旧俄用复姓者多为名人,望族,而伊凡·伊凡内奇只是个普通的兽医。},所以省城里的人通常只叫他的名字和父称。他住在城郊的养马场,现在出来打猎是想呼吸点新鲜空气。中学教员布尔金每年夏天都在n姓伯爵家里做客,所以在这一带早已不算外人了。
    
    暂时没有睡觉。伊凡·伊凡内奇,一个又高又瘦的老头,留着长长的胡子,坐在门外月光下吸着烟斗,布尔金躺在里面的干草上,在黑暗中看不见他。
    
    他们天南海北地闲聊着。顺便提起村长的老婆玛芙拉,说这女人身体结实,人也不蠢,就是一辈子没有走出自己的村子,从来没有见过城市,没有见过铁路,最近十年间更是成天守着炉灶,只有到夜里才出来走动走动。
    
    “这有什么奇怪的!”布尔金说,“有些人生性孤僻,他们像寄居蟹或蜗牛那样,总想缩进自己的壳里,这种人世上还不少哩。也许这是一种返祖现象,即返回太古时代,那时候人的祖先还不成其为群居的动物,而是独自居住在自己的洞穴里;也许这仅仅是人的性格的一种变异——谁知道呢。我不是搞自然科学的,这类问题不关我的事。我只是想说,像玛芙拉这类人,并不是罕见的现象。哦,不必去远处找,两个月前,我们城里死了一个人,他姓别利科夫,希腊语教员,我的同事。您一定听说过他。他与众不同的是:他只要出门,哪怕天气很好,也总要穿上套鞋,带着雨伞,而且一定穿上暖和的棉大衣。他的伞装在套子里,怀表装在灰色的鹿皮套子里,有时他掏出小折刀削铅笔,那把刀也装在一个小套子里。就是他的脸似乎也装在套千里,因为他总是把脸藏在竖起的衣领里。他戴墨镜,穿绒衣,耳朵里塞着棉花,每当他坐上出租马车,一定吩咐车夫支起车篷。总而言之,这个人永远有一种难以克制的愿望——把自己包在壳里,给自己做一个所谓的套子,使他可以与世隔绝,不受外界的影响。现实生活令他懊丧、害怕,弄得他终日惶惶不安。也许是为自己的胆怯、为自己对现实的厌恶辩护吧,他总是赞扬过去,赞扬不曾有过的东西。就连他所教的古代语言,实际上也相当于他的套鞋和雨伞,他可以躲在里面逃避现实。
    
    “‘啊,古希腊语是多么响亮动听,多么美妙!’他说时露出甜美愉快的表情。仿佛为了证实自己的话,他眯细眼睛,竖起一个手指头,念道:‘安特罗波斯!’\footnote{〔安特罗波斯〕希腊文“人”。}
    
    “别利科夫把自己的思想也竭力藏进套子里。对他来说,只有那些刊登各种禁令的官方文告和报纸文章才是明白无误的。既然规定晚九点后中学生不得外出,或者报上有篇文章提出禁止性爱,那么他认为这很清楚,很明确,既然禁止了,那就够了。至于文告里批准、允许干什么事,他总觉得其中带有可疑的成分,带有某种言犹未尽,令人不安的因素。每当城里批准成立戏剧小组,或者阅览室,或者茶馆时,他总是摇着头小声说:
    
    “‘这个嘛,当然也对,这都很好,但愿不要惹出什么事端!’
    
    “任何违犯、偏离、背弃所谓规章的行为,虽说跟他毫不相干,也总让他忧心忡忡。比如说有个同事做祷告时迟到了,或者听说中学生调皮捣乱了,或者有人看到女学监很晚还和军官在一起,他就会非常激动,总是说:但愿不要惹出什么事端。在教务会议上,他那种顾虑重重、疑神疑鬼的作风和一套纯粹套子式的论调,把我们压得透不过气来。他说什么某某男子中学、女子中学的年轻人行为不轨,教室里乱哄哄的——唉,千万别传到当局那里,哎呀,千万不要惹出什么事端!又说,如果把二年级的彼得罗夫、四年级的叶戈罗夫开除出校,那么情况就会好转。后来怎么样呢?他不住地唉声叹气,老是发牢骚,苍白的小脸上架一副墨镜——您知道,那张小尖脸跟黄鼠狼的一样——他就这样逼迫我们,我们只好让步,把彼得罗夫和叶戈罗夫的操行分数压下去,关他们的禁闭,最后把他们开除了事。他有一个古怪的习惯——到同事家串门。他到一个教员家里,坐下后一言不发,像是在监视什么。就这样不声不响坐上个把钟头就走了。他把这叫做‘和同事保持良好关系’。显然,他上同事家闷坐并不轻松,可他照样挨家挨户串门,只因为他认为这是尽到同事应尽的义务。我们这些教员都怕他。连校长也怕他三分。您想想看,我们这些教员都是些有头脑、极正派的人,受过屠格涅夫和谢德林的良好教育,可是我们的学校却让这个任何时候都穿着套鞋、带着雨伞的小人把持了整整十五年!何止一所中学呢?全城都捏在他的掌心里!我们的太太小姐们到星期六不敢安排家庭演出,害怕让他知道;神职人员在他面前不好意思吃荤和打牌。在别利科夫这类人的影响下,最近十到十五年间,我们全城的人都变得谨小慎微,事事都怕。怕大声说话,怕写信,怕交朋友,怕读书,怕周济穷人,怕教人识字……”
    
    伊凡·伊凡内奇想说点什么,嗽了嗽喉咙,但他先抽起烟斗来,看了看月亮,然后才一字一顿地说:
    
    “是的,我们都是有头脑的正派人,我们读屠格涅夫和谢德林的作品,以及巴克莱\footnote{〔巴克莱〕英国历史学家。}等人的著作,可是我们又常常屈服于某种压力,一再忍让……问题就在这儿。”
    
    “别利科夫跟我住在同一幢房里,”布尔金接着说,“同一层楼,门对门,我们经常见面,所以了解他的家庭生活。在家里也是那一套:睡衣,睡帽,护窗板,门闩,无数清规戒律,还有那句口头掸:‘哎呀,千万不要惹出什么事端!’斋期吃素不利健康,可是又不能吃荤,因为怕人说别利科夫不守斋戒。于是他就吃牛油煎鲈鱼——这当然不是素食,可也不是斋期禁止的食品。他不用女仆,害怕别人背后说他的坏话。他雇了个厨子阿法纳西,老头子六十岁上下,成天醉醺醺的,还有点痴呆。他当过勤务兵,好歹能弄几个菜。这个阿法纳西经常站在房门口,交叉抱着胳膊,老是叹一口长气,嘟哝那么一句话:
    
    “‘如今他们这种人多得很呢!’
    
    “别利科夫的卧室小得像口箱子,床上挂着帐子。睡觉的时候,他总用被子蒙着头。房间里又热又闷,风敲打着关着的门,炉子里像有人呜呜地哭,厨房里传来声声叹息,不祥的叹息……
    
    “他躺在被子里恐怖之极。他生怕会出什么事情,生怕阿法纳西会宰了他,生怕窃贼溜进家来,这之后就通宵做着噩梦。到早晨我们一道去学校的时候,他无精打采,脸色苍白。看得出来,他要进去的这所学生很多的学校令他全身心感到恐慌和厌恶,而他这个生性孤僻的人觉得与我同行也很别扭。
    
    “‘我们班上总是闹哄哄的,’他说,似乎想解释一下为什么他心情沉重,‘真不像话!’
    
    “可是这个希腊语教员,这个套中人,您能想象吗,差一点还结婚了呢!”
    
    伊凡·伊凡内奇很快回头瞧瞧堆房,说:
    
    “您开玩笑!”
    
    “没惜,他差一点结婚了,尽管这是多么令人奇怪。我们学校新调来了一位史地课教员,叫米哈伊尔·萨维奇·柯瓦连科,小俄罗斯人\footnote{〔小俄罗斯〕乌克兰的旧称。}。他不是一个人来的,还带着姐姐瓦莲卡。他年轻,高个子,肤色黝黑,一双大手,看模样就知道他说话声音低沉,果真没错,他的声音像从木桶里发出来的:卜,卜,卜……他姐姐年纪已经不轻,三十岁上下,个子高挑,身材匀称,黑黑的眉毛,红红的脸蛋——一句话,不是姑娘,而是果冻,她那样活跃,吵吵嚷嚷,不停地哼着小俄罗斯的抒情歌曲,高声大笑,动不动就发出一连串响亮的笑声:哈,哈,哈!我们初次正经结识科瓦连科姐弟,我记得是在校长的命名日宴会上。一群教员神态严肃,闷闷不乐,把参加校长命名日宴会也当作例行公事。就在这么一群教员中间,我们忽地看到,一位新的阿芙洛狄忒\footnote{〔阿芙洛狄忒〕希腊神话中爱与美的女神,即罗马神话中的维纳斯。传说她在大海的泡沫中诞生。}从大海的泡沫中诞生了:她双手叉腰走来走去,又笑又唱,翩翩起舞……她动情地唱起一首《风飘飘》,随后又唱一支抒情歌曲,接着再唱一曲,我们大家都让她迷住了——所有的人,甚至包括别利科夫。他在她身旁坐下,甜蜜地微笑着,说:
    
    “‘小俄罗斯语柔和,动听,使人联想到古希腊语。’
    
    “这番奉承使她感到得意,于是她用令人信服的语气动情地告诉他,说他们在加佳奇县有一处田庄,现在妈妈还住在那里。那里有那么好的梨,那么好的甜瓜,那么好的‘卡巴克\footnote{〔卡巴克〕俄语中意为“酒馆”,乌克兰语中意为“南瓜”。}’!小俄罗斯人把南爪叫‘卡巴克’,把酒馆叫‘申克’。他们做的西红柿加紫甜菜浓汤‘可美味啦,可美味啦,简直好吃得——要命!’
    
    “我们听着,听着,忽然大家不约而同冒出一个念头:
    
    “‘把他们撮合成一对,那才好哩’,校长太太悄悄对我说。
    
    “我们大家不知怎么都记起来,我们的别利科夫还没有结婚。我们这时都感到奇怪,对他的终身大事我们竟一直没有注意,完全给忽略了。他对女人一般持什么态度?他准备怎么解决这个重大问题?以前我们对此完全不感兴趣,也许我们甚至不能设想,这个任何时候都穿着套鞋、挂着帐子的人还能爱上什么人。
    
    “‘他早过了四十,她也三十多了……’校长太太说出自己的想法,‘我觉得她是愿意嫁给他的。’
    
    “在我们省,人们出于无聊,什么事干不出来呢?干了无数不必要的蠢事!这是因为,必要的事却没人去做。哦,就拿这件事来说吧,既然我们很难设想别利科夫会结婚,我们又为什么突然之间头脑发热要给他做媒呢?校长太太,督学太太,以及全体教员太太全都兴致勃勃,甚至连模样都变好看了,仿佛一下子找到了生活的目标。校长太太订了一个剧院包厢,我们一看——她的包厢里坐着瓦莲卡,拿着这么小的一把扇子,眉开眼笑,喜气洋洋。身旁坐着别利科夫,瘦小,佝偻,倒像是让人用钳子夹到这里来的。我有时在家里请朋友聚会,太太们便要我一定邀上别利科夫和瓦莲卡。总而言之,机器开动起来了。原来瓦莲卡本人也不反对出嫁。她跟弟弟生活在一起不大愉快,大家只知道,他们成天争吵不休,还互相对骂。我来跟您说一段插曲:柯瓦连科在街上走着,一个壮实的大高个子,穿着绣花衬衫,一给头发从制帽里耷拉到额头上。他一手抱着一包书,一手拿一根多疖的粗手杖。她姐姐跟在后面,也拿着书。
    
    “‘你啊,米哈伊里克\footnote{〔米哈伊里克〕和后面的“明契克”一样,都是米哈伊尔的小名。},这本书就没有读过!’她大声嚷道,‘我对你说,我可以起誓,你根本没有读过这本书!’
    
    “‘可我要告诉你,我读过!’柯瓦连科也大声嚷道,还用手杖敲得人行道咚咚响。
    
    “‘哎呀,我的天哪,明契克!你干吗发脾气,要知道我们的谈话带原则性。”
    
    “‘可我要告诉你:我读过这本书!’他嚷得更响了。
    
    “在家里,即使有外人在场,他们也照样争吵不休。这种生活多半让她厌倦了,她一心想有个自己的窝,再说也该考虑到年龄了。现在已经不是挑挑拣拣的时候,嫁谁都可以,哪怕希腊语教员也凑合。可也是,我们这儿的大多数小姐只要能嫁出去就行,嫁给谁是无所谓的。不管怎么说,瓦莲卡开始对我们的别利科夫表露出明显的好感。
    
    “那么,别利科夫呢,他也去柯瓦连科家,就像上我们家一样。他到他家,坐下来就一言不发。他默默坐着,瓦莲卡就为他唱《风飘飘》,或者用那双乌黑的眼睛若有所思地望着他,或者突然发出一串朗朗大笑:
    
    “‘哈哈哈!’
    
    “在恋爱问题上,特别是在婚姻问题上,撮合起着很大的作用。于是全体同事和太太们都去劝说别利科夫,说他应当结婚了,说他的生活中没有别的欠缺,只差结婚了。我们大家向他表示祝贺,一本正经地重复着那些老生常谈,比如说婚姻是终身大事等等,又说瓦莲卡相貌不错,招人喜欢,是五品文官\footnote{〔五品文官〕沙皇时代俄罗斯将官阶分为十四品,一品最高。}的女儿,又有田庄,最主要的,她是头一个待他这么温存又真心诚意的女人。结果说得他晕头转向,他认定自己当真该结婚了。”
    
    “这下该有人夺走他的套鞋和雨伞了,”伊凡·伊凡内奇说。
    
    “您要知道,这是不可能的。虽然他把瓦莲卡的相片放在自己桌子上,还老来找我谈论瓦莲卡,谈论家庭生活,也说婚姻是人生大事,虽然他也常去柯瓦连科家,但他的生活方式却丝毫没有改变。甚至相反,结婚的决定使他像得了一场大病:他消瘦了,脸色煞白,似乎更深地藏进自己的套子里去了。
    
    “‘瓦尔瓦拉·萨维什娜\footnote{〔瓦尔瓦拉〕瓦莲卡的正式名字。}我是中意的,’他说道,勉强地淡淡一笑,‘我也知道,每个人都该结婚的,但是……这一切,您知道吗,来得有点突然……需要考虑考虑。’
    
    “‘这有什么好考虑的?’我对他说,‘您结婚就是了。”
    
    “‘不,结婚是一件大事,首先应当掂量一下将要承担的义务和责任……免得日后惹出什么麻烦。这件事弄得我不得安宁,现在天天夜里都睡不着觉。老实说吧,我心里害怕:他们姐弟俩的思想方法有点古怪,他们的言谈,您知道吗,也有点古怪。她的性格太活泼。真要结了婚,恐怕日后会遇上什么麻烦。’
    
    “就这样他一直没有求婚,老是拖着,这使校长太太和我们那里所有太太们大为恼火。他反反复复掂量着面临的义务和责任,与此同时几乎每天都跟瓦莲卡一道散步,也许他认为处在他的地位必须这样做。他还常来我家谈论家庭生活,若不是后来出了一件荒唐的事,很可能他最终会去求婚的,那样的话,一门不必要的、愚蠢的婚姻就完成了在我们这里,由于无聊,由于无事可做,这样的婚姻可以说成千上万。这里须要说明一下,瓦莲卡的弟弟柯瓦连科,从认识别利科夫的第一天起就痛恨他,不能容忍他。
    
    “‘我不明白’他耸耸肩膀对我们说,‘不明白你们怎么能容忍这个爱告密的家伙,这个卑鄙的小人。哎呀,先生们,你们怎么能在这儿生活!你们这里的空气污浊,能把人活活憋死。难道你们是教育家、师长?不,你们是一群官吏,你们这里不是科学的殿堂,而是城市警察局,有一股酸臭味,跟警察亭子里一样。不,诸位同事,我再跟你们待上一阵,不久就回到自己的田庄去。我宁愿在那里捉捉虾,教小俄罗斯的孩子们读书认字。我一定要走,你们跟你们的犹太就留在这里吧,叫他见鬼去!’
    
    “有时他哈哈大笑,笑得流出眼泪来,笑声时而低沉,时而尖细。他双手一摊,问我:
    
    “‘他干什么来我家坐着?他要什么?坐在那里东张西望的!’
    
    “他甚至给别利科夫起了个绰号叫‘毒蜘蛛’。自然,我们当着他的面从来不提他的姐姐要嫁给‘毒蜘蛛’的事。有一天,校长太太暗示他,说如果把他的姐姐嫁给像别利科夫这样一个稳重的、受人尊敬的人倒是不错的。他皱起眉头,埋怨道:
    
    “‘这不关我的事。她哪怕嫁一条毒蛇也由她去,我可不爱管别人的闲事。’
    
    “现在您听我说下去。有个好恶作剧的人画了一幅漫画:别利科夫穿着套鞋,卷起裤腿,打着雨伞在走路,身边的瓦莲卡挽着他的胳臂,下面的题词是:‘堕入情网的安特罗波斯’。那副神态,您知道吗,简直惟妙惟肖。这位画家想必画了不止一夜,因为全体男中女中的教员、中等师范学校的教员和全体文官居然人手一张。别利科夫也收到一份。漫画使他的心情极其沉重。
    
    “我们一道走出家门——这一天刚好是五月一日,星期天,我们全体师生约好在校门口集合,然后一道步行去城外树林里郊游。我们一道走出家门,他的脸色铁青,比乌云还要阴沉。
    
    “‘天底下竟有这样坏、这样恶毒的人!’他说时嘴唇在发抖。
    
    “我甚至可怜起他来了。我们走着,突然,您能想象吗,柯瓦连科骑着自行车赶上来了,后面跟着瓦莲卡,也骑着自行车。她满脸通红,很累的样子,但兴高采烈,快活得很。
    
    “‘我们先走啦!’她大声嚷道,‘天气多好啊,多好啊,简直好得要命!’
    
    “他们走远了,不见了。我的别利科夫脸色由青变白,像是吓呆了。他站住,望着我……
    
    “‘请问,这是怎么回事?’他问,‘还是我的眼睛看错了?中学教员和女人都能骑自行车,这成何体统?’
    
    “‘这有什么不成体统的?’我说,‘愿意骑就由他们骑好了。’
    
    “‘那怎么行呢?’他喊起来,对我的平静感到吃惊,‘您这是什么话?!’
    
    “他像受到致命的一击,不愿再往前走,转身独自回家去了。
    
    “第二天,他老是神经质地搓着手,不住地打颤,看脸色他像是病了。没上完课就走了,这在他还是平生第一次。也没有吃午饭。傍晚,他穿上暖和的衣服,尽管这时已经是夏天了,步履蹒跚地朝柯瓦连科家走去。瓦莲卡不在家,他只碰到了她的弟弟。
    
    “‘请坐吧,’柯瓦连科皱起眉头,冷冷地说。他午睡后刚醒,睡眼惺忪,心情极坏。
    
    “别利科夫默默坐了十来分钟才开口说:
    
    “‘我到府上来,是想解解胸中的烦闷。现在我的心情非常非常沉重。有人恶意诽谤,把我和另一位你我都亲近的女士画成一幅可笑的漫画。我认为有责任向您保证,这事与我毫不相干……我并没有给人任何口实,可以招致这种嘲笑,恰恰相反,我的言行举止表明我是一个极其正派的人。’
    
    “柯瓦连科坐在那里生闷气,一言不发。别利科夫等了片刻,然后忧心忡忡地小声说:
    
    “‘我对您还有一言相告。我已任教多年,您只是刚开始工作,因此,作为一个年长的同事,我认为有责任向您提出忠告。您骑自行车,可是这种玩闹对身为青年的师表来说,是有伤大雅的!’
    
    “‘那为什么?’柯瓦连科粗声粗气地问。
    
    “‘这难道还须要解释吗,米哈伊尔·萨维奇,难道这还不明白吗?如果教员骑自行车,那么学生们该做什么呢?恐怕他们只好用头走路了!既然这事未经正式批准,那就不能做。昨天我吓了一大跳!我一看到您的姐姐,我的眼前就发黑。一个女人或姑娘骑自行车——这太可怕了!’
    
    “‘您本人到底有什么事?’
    
    “‘我只有一件事——对您提出忠告,米哈伊尔·萨维奇。您还年轻,前程远大,所以您的举止行为要非常非常小心谨慎,可是您太随便了,哎呀,太随便了!您经常穿着绣花衬衫出门,上街时老拿着什么书,现在还骑自行车。您和您姐姐骑自行车的事会传到校长那里,再传到督学那里……那会有什么好结果?’
    
    “‘我和我姐姐骑自行车的事,跟谁都没有关系!”柯瓦连科说时涨红了脸,‘谁来干涉我个人的和家庭的私事,我就叫他——滚蛋!’
    
    “别利科夫脸色煞白,站起身来。
    
    “‘既然您用这种口气跟我讲话,那我就无话可说了,’他说,‘我请您注意,往后在我的面前千万别这样谈论上司。对当局您应当尊敬才是。’
    
    “‘怎么,难道我刚才说了当局的坏话了吗?’柯瓦连科责问,愤恨地瞧着他,‘劳驾了,请别来打扰我。我是一个正直的人,跟您这样的先生根本就不想交谈。我不喜欢告密分子。’
    
    “别利科夫神经紧张地忙乱起来,很快穿上衣服,一脸惊骇的神色。他这是平生第一回听见这么粗鲁的话。
    
    “‘您尽可以随便说去,’他说着从前室走到楼梯口,‘只是我得警告您:我们刚才的谈话也许有人听见了,为了避免别人歪曲谈话的内容,惹出什么事端,我必须把这次谈话内容的要点向校长报告。我有责任这样做。’
    
    “‘告密吗?走吧,告密去吧!’
    
    “柯瓦连科从后面一把揪住他的领子,只一推,别利科夫就滚下楼去,套鞋碰着楼梯啪啪地响。楼梯又高又陡,他滚到楼下却平安无事,他站起来,摸摸鼻子,看眼镜摔破了没有?正当他从楼梯上滚下来的时候,瓦莲卡和两位太太刚好走进来;她们站在下面看着——对别利科夫来说这比什么都可怕。看来,他宁可摔断脖子,摔断两条腿,也不愿成为别人的笑柄:这下全城的人都知道了,还会传到校长和督学那里——哎呀,千万别惹出麻烦来!——有人会画一幅新的漫画,这事闹到后来校方会勒令他退职……
    
    “他爬起来后,瓦莲卡才认出他来。她瞧着他那可笑的脸,皱巴巴的大衣和套鞋,不明白是怎么回事,还以为他是自己不小心摔下来的。她忍不住放声大笑起来,笑声响彻全楼:
    
    “‘哈哈哈!’
    
    “这一连串清脆响亮的‘哈哈哈’断送了一切:断送了别利科夫的婚事和他的尘世生活。他已经听不见瓦莲卡说的话,也看不见眼前的一切。他回到家里,首先收走桌上瓦莲卡的相片,然后在床上躺下,从此再也没有起来。
    
    “三天后,阿法纳西来找我,问要不要去请医生,因为他家老爷‘出事’了。我去看望别利科夫。他躺在帐子里,蒙着被子,一声不响。问他什么,除了‘是’‘不是’外,什么话也没有。他躺在床上,阿法纳西在一旁转来转去。他脸色阴沉,紧皱眉头,不住地唉声叹气。他浑身酒气,那气味跟小酒馆里的一样。
    
    “一个月后别利科夫去世了。我们大家,也就是男中、女中和师范专科学校的人,都去为他送葬。当时,他躺在棺木里,面容温和,愉快,甚至有几分喜色,仿佛很高兴他终于被装进套子,从此再也不必出来了。是的,他实现了他的理想!连老天爷也表示对他的敬意,下葬的那一天,天色阴沉,下着细雨,我们大家都穿着套鞋,打着雨伞。瓦莲卡也来参加了他的葬礼,当棺木下了墓穴时,她大声哭了一阵。我发现,小俄罗斯女人不是哭就是笑,介于二者之间的情绪是没有的。
    
    “老实说,埋葬别利科夫这样的人,是一件令人高兴的事。从墓地回来的路上,我们都是一副端庄持重、愁眉不展的面容,谁也不愿意流露出这份喜悦的心情——它很像我们在很久很久以前还在童年时代体验过的一种感情:等大人们出了家门,我们就在花园里跑来跑去,玩上一两个钟头,享受一番充分自由的欢乐。啊,自由呀自由!哪怕有它的半点迹象,哪怕有它的一丝希望,它也会给我们的心灵插上翅膀。难道不是这样吗?
    
    “我们从墓地回来,感到心情愉快。可是,不到一个星期,生活又回到了原来的样子,依旧那样严酷,令人厌倦,毫无理性。这是一种虽没有明令禁止、但也没有充分开戒的生活。情况不见好转。的确,我们埋葬了别利科夫,可是还有多少这类套中人留在世上,而且将来还会有多少套中人啊!”
    
    “问题就在这儿,”伊凡·伊凡内奇说着,点起了烟斗。
    
    “将来还会有多少套中人啊!”布尔金重复道。
    
    中学教员走出板棚。这人身材不高,很胖,秃顶,留着几乎齐腰的大胡子。两条狗也跟了出来。
    
    “好月色,好月色!”他说着,抬头望着天空。
    
    已是午夜。向右边望去,可以看到整个村子,一条长街伸向远处,足有四五俄里。万物都进入寂静而深沉的梦乡。没有一丝动静,没有,一丝声息,甚至叫人难以置信,大自然竟能这般沉寂。在这月色溶溶的深夜里,望着那宽阔的街道、街道两侧的农舍、草垛和睡去的杨柳,内心会感到分外平静。摆脱了一切辛劳、忧虑和不幸,隐藏在朦胧夜色的庇护下,村子在安然歇息,显得那么温柔、凄清、美丽。似乎天上的繁星都亲切地、深情地望着它,似乎在这片土地上邪恶已不复存在,一切都十分美好。向左边望去,村子尽头处便是田野。田野一望无际,一直延伸到远方的地平线。沐浴在月光中的这片广袤土地,同样没有动静,没有声音。
    
    “问题就在这儿,”伊凡·伊凡内奇重复道,“我们住在空气污浊、拥挤不堪的城市里,写些没用的公文,玩‘文特’牌戏——难道这不是套子?至于我们在游手好闲的懒汉、图谋私利的讼棍和愚蠢无聊的女人们中间消磨了我们的一生,说着并听着各种各样的废话——难道这不是套子?哦,如果您愿意的话,我现在就给您讲一个很有教益的故事。”
    
    “不用了,该睡觉了,”布尔金说,“明天再讲吧。”
    
    两人回到板棚里,在干草上躺下。他们盖上被子,正要朦胧入睡,忽然听到轻轻的脚步声:吧嗒,吧嗒……有人在堆房附近走动:走了一会儿,站住了,不多久又吧嗒吧嗒走起来……狗唔唔地叫起来。
    
    “这是玛芙拉在走动,”布尔金说。
    
    脚步声听不见了。
    
    “看别人作假,听别人说谎,”伊凡·伊凡内奇翻了一个身说,“如若你容忍这种虚伪,别人就管你叫傻瓜。你只好忍气吞声,任人侮辱,不敢公开声称你站在正直自由的人们一边,你只好说谎,陪笑,凡此种种只是为了混口饭吃,有个温暖的小窝,捞个分文不值的一官半职!不,再也不能这样生活下去了!”
    
    “哦,您这是另一个话题了,伊凡·伊凡内奇,”教员说,“我们睡觉吧。”
    
    十分钟后,布尔金已经睡着了。伊凡·伊凡内奇却还在不断地翻身叹气。后来他索性爬起来,走到外面,在门口坐下,点起了烟斗。
    
\end{normalsize}


\newpage

\textbf{注释}:

\vspace{-1em}

\begin{itemize}
    \setlength\itemsep{-0.2em}
    \item 〔睡眼惺忪〕
    \item 〔撮合〕
    \item 〔煞白〕(脸色)惨白,没有血色。
    \item 〔惟妙惟肖〕
    \item 〔广袤〕
    \item 〔溶溶〕
\end{itemize}

\chapter{大卫·科波菲尔}

\begin{normalsize}
    
    现在的我,涉世已久,几乎不会为了什么事吃惊了。即便如此,每当我想起当年,我那么小就被轻易地遗弃了\footnote{〔那么小……〕小说中,主角在不到十岁时被养父莫德斯通送到他开的货行做苦工。},也多少感到诧异。这么一个才能优异,观察力强,聪慧热情,机灵而敏感的孩子,早早就身心受创,却没有一个人出来为我说句话,也是不可思议。就这样,在我十岁那年,我成了莫德斯通–古林比公司的一个小苦力\footnote{〔苦力〕廉价体力劳动者,多形容码头工人。}。
    
    莫德斯通–古林比公司的货行坐落在河边,在墨修士区\footnote{〔墨修士区〕伦敦中心偏东的地名,在泰晤士河北岸边。13世纪多明我会在该区开设修道院,故得名。多明我会的修士穿黑衣,也叫黑衣修士。}。整个地段如今已经改建,大为改观。当时,货行在一条窄巷里,在弯弯曲曲通往河边的下坡道的尽头。那里有几级供人上落船的台阶。房子相当破旧,但有自己的码头。涨潮时与水相接,退潮时则裸露一片烂泥,老鼠横行。房子里那些被尘污烟熏了怕有上百年的镶板房间,腐朽的地板和楼梯,地下室里东奔西窜、吱吱尖叫的灰色大老鼠,遍地的污秽、腐臭,对我来说,仿佛就在眼前,而不是多年前的往事。种种景象,仍然那样清晰地刻在我的脑海里,和当初那个倒霉的日子里,我被昆宁先生握着颤抖的手,第一次从中经过时看到的完全一样。
    
    莫德斯通–古林比公司的生意牵涉甚广,与各色人等都有往来。不过其中重要的一笔生意,是给邮轮提供葡萄酒和烈酒。这些邮轮开的什么航线,我现在记不太清了,但我想有去东印度群岛\footnote{〔东印度群岛〕现在的印度尼西亚和马来群岛。当时为英国、荷兰殖民地。}和西印度群岛\footnote{〔西印度群岛〕现在中美洲的一系列岛屿。当时为英国、法国、荷兰殖民地。}的。这种艰难的航线的结果是大量的空酒瓶。得有人把它们对着光好好检查,把破损的剔掉,把完好的洗刷干净。空瓶子收拾完了,还有在灌满酒的瓶子上贴标签、塞木塞、封盖,把完工的酒瓶装箱的活儿。这许多活都得雇人,大人小孩都有。我就是雇来干活的孩子中的一个,这些活我都做过。
    
    我所在的一组,连我在内,一共有三四个工人。工作地点在货行的一角。昆宁先生只要站在账房里凳脚的横木上,就能透过桌案上的窗子看到我。在我有幸独自谋生的第一个早上,童工中年纪最大的男孩被指派来教我怎么干活。他叫路米,身上系一条破围裙,头戴一顶破帽子。他告诉我,他爸爸是个船夫,但曾经在伦敦市长就职日,戴着黑天鹅绒帽子走在仪仗队里。他还告诉我,我们主要的伙伴是另一个男孩,他的名字竟然叫白薯。后来我才发现,这并不是他的教名\footnote{〔教名〕指法律上正式的名字。},而是货行里的人给他取的诨名,因为他脸色煞白,像煮熟的马铃薯一样。白薯的爸爸是个水手,还在某个大剧院里做消防员。他家还有人在剧院里的哑剧里扮演小鬼——我想是他妹妹。
    
    我竟沦落到跟这么一班人为伍!想想我从此就要和这些家伙朝夕为伴,再想想我幸福童年里的伙伴——更不必说斯第弗斯,特拉德尔等同学了\footnote{〔斯第弗斯,特拉德尔〕主角之前在寄宿学校上学时的同学。},心中隐藏的苦痛,真是无法用语言表述。长大后成为一个博学、卓越的人,这样的希望,就在胸中破灭了。当时那种绝望,处境带来的羞耻感,觉得自己学过的、思考过的、喜爱过的、给自己带来过美好遐想和上进心的一切就要一天天、一点点地离自己而去,永远回不来了,种种心情,都深深刻在我的回忆中,绝非笔墨能述说。那天上午,路米离开以后,洗酒瓶的时候,我抽泣着,泪水直掉,混在洗瓶子的水里。胸口仿佛裂了一个口子,随时要迸碎一样。
    
    账房里的钟指向了十二点半,大家都准备去吃午饭了。这时,昆宁先生敲了敲账房的窗子,打手势叫我进去。我进了帐房,发现房间里还有个肥壮的中年人,穿着褐色外套、黑色马裤和黑色皮鞋。脑袋光秃秃的,又大又亮,像个鸡蛋;一张阔脸正对着我。他的衣装破旧,却戴了条醒目的假领\footnote{〔假领〕一种衣饰,穿在外衣里面假装衬衣,领子部分和衬衣领子一样,可以用来打领带。}。他手里拿着一根很气派的手杖,手杖上系了一对褪了色的大穗子。外套上还挂了个单片眼镜——我后来发现这只是个装饰,因为他几乎从不用来看东西,即便用了也看不清。
    
    “这,”昆宁先生指着我说,“就是他。”
    
    “这,”那位陌生人说,他给我印象很深的是那种屈就下交的语调,还有一种难以形容的故作文雅的神态,“就是科波菲尔少爷了。你好吗,先生?”
    
    我说我很好,也希望他很好。其实,老天爷知道,当时我心里非常局促难堪;但我不愿当场诉苦,所以我说我很好,并希望他也很好。
    
    “谢天谢地,”那陌生人说,“我很好。我收到莫德斯通先生的一封信,他在信中说,希望我把我家后头的一个空置的房间——简言之,出租——简言之,”那陌生人笑了笑,迸发出勇气说,“当作卧室——租给我此刻有幸结识的年轻创业人——”那陌生人挥挥手,把下巴搁在那硬假领上。
    
    “这是米考伯先生,”昆宁先生对我说。
    
    “嗯哼!”陌生人说,“这是我的姓氏。”
    
    “米考伯先生,”昆宁先生说,“和莫德斯通先生相识。他给我们拉生意,只要他拉到了客户,我们就付他佣金,他收到了莫德斯通先生的信,请他替你安排住处。他愿意接收你做他的房客。”
    
    “我的地址是,”米考伯先生说,“都会路,温泽巷。我——简言之,”米考伯先生又一度迸发出勇气说,但还是用那种故作文雅的神态——“我就住在那里。”
    
    我向他鞠了一躬。
    
    “依我管见,”米考伯先生说,“你在这大都市中游历尚浅,要穿过这现代巴比伦\footnote{〔现代巴比伦〕指伦敦。巴比伦:基督教《圣经》中提到的异教的大城市,以道路建筑金碧辉煌、错综复杂闻名。常用来影射纸醉金迷、腐朽堕落的大都市。}的迷宫,找到都会路——简言之,”米考伯先生又一次迸发出勇气说,“你可能会迷失方向——我很高兴今晚来这里,教你熟悉最近的路线。”
    
    我真心真意地谢了他。因为他竟愿意费心,真是太热诚了。
    
    “几点,”米考伯先生说,“我可以——”
    
    “八点左右。”昆宁先生说。
    
    “八点左右,”米考伯先生说,“再见,昆宁先生。我就不叨扰了。”
    
    于是,他戴上帽子,夹着手杖,哼着曲子,腰杆儿笔挺地离开了帐房。
    
    就这样,我正式受雇于昆宁先生,在莫德斯通–古林比公司的货行里做工。薪水嘛,我想是一星期六先令\footnote{〔先令〕英国货币单位。一英镑等于二十先令,一先令等于十二便士。}吧。我记不清是六先令还是七先令了。现在想来,大概是六先令;先是六先令,后来是七先令。他立刻付了我一星期的(我相信是他从自己口袋里掏的)工钱。我从中拿出六便士给白薯,请他晚上帮我把行李箱子拿到温泽巷去——箱子虽说不重,也不是我那时的力气能扛得起的。我又花六便士吃了午饭。午饭是一块肉饼,还有附近水龙头的生水。午饭时间规定是一小时。我吃完饭,在附近的街上闲逛了一通,打发了剩余的时间。
    
    晚上,到了约定的时间,米考伯先生又来了。我洗了手和脸,以示对他的文雅派头的敬意,然后我们一起回家(我想,现在得这么称呼了)。一路上,米考伯先生把街名、拐角处的房屋式样等,都教我记住,以免明天早上我找不着回货行的路。
    
    到了他在温泽巷的住宅后(我看出,这住宅也和他一样寒酸,也和他一样尽可能装体面),他把我介绍给米考伯太太。米考伯太太是个瘦削憔悴的女人,一点也不年轻了。她正坐在小客厅里(楼下空空如也,没有任何家具,窗帘是拉上的,好挡住邻居的目光)给婴儿喂奶。婴儿是一对双胞胎。顺便说一下,我和这家人相处时,从没见过这对双胞胎同时离开米考伯太太的怀抱——总有一个在吃奶。
    
    还有两个孩子——米考伯少爷,大约四岁;米考伯小姐,大约三岁。还有一个皮肤很黑的年轻女仆,她有哼鼻子的习惯。不过半个小时,她就告诉我她是个“苦儿”\footnote{〔“苦儿”〕孤儿的误称,这里暗示女仆没受过什么教育。},从附近的圣路加济贫院里来这儿的。我的卧室在后面的顶楼上,没有窗。小小的房间墙上全贴着一种花纹的墙纸。在我童稚的想象力中,这花纹是蓝色的松饼\footnote{〔松饼〕一种以发酵的面糊在烤盘或平底锅上烹饪制成的薄扁状饼,口感松软。}。屋里只有很少的几件家具。
    
    “结婚前,”米考伯太太喘着气说道。她带了双胞胎和另两个孩子上楼带我看住处,这时她坐了下来,“我和爸爸妈妈住的时候,从没想过,会有不得不招收房客的一天。可没办法,我先生\footnote{〔我先生〕资产阶级女性对外人提起自己丈夫时的称呼。}遇到困难了,我自己的感受只好让步了。”
    
    我说:“是的,夫人。”
    
    “眼下我先生的困难,几乎要把我们压垮了,”米考伯太太说,“我不知道他能不能度过这难关。我和爸爸妈妈住的时候,不知困难为何物,即便用了这个词,也不是今天我感受到的意思。凡事只有经历了才会懂——正如爸爸常说的那样。”
    
    我不能肯定,我究竟是从她那里知道米考伯先生做过海军军官,还是出于自己的想象。我只知道,至今我还是无来由地认为他当过海军。他现在为各种商户跑腿招揽顾客,不过恐怕不怎么赚钱,也许根本赚不到钱。
    
    “如果我先生的债主不肯给他宽限,”米考伯太太说,“他们就得自食其果了。这件事越快了结越好。石头榨不出血,就是榨干了我们,也榨不出钱还帐,更别说付诉讼费了。”
    
    是因为我过早自食其力,米考伯太太弄不清楚我的年龄吗?还是说她愁思难抑,实在找不到人倾诉了?要不是我,她或许就得向双胞胎诉苦了吧。总之,之后我们相处的日子里,她就一直把我当作倾诉对象了。
    
    可怜的米考伯太太!她说她曾尽过最大的努力,我毫不怀疑她尝试过。临街门上中间几乎被一块大铜牌子遮住了,那牌子上刻着:“米考伯夫人青年妇女宿舍”,可我从没见到任何青年妇女在这里住宿过,没见过任何青年妇女来过或提出过要来,也没见过这里做过任何接待青年妇女的最低标准的准备。我见到或听到的来客全是债主。这些家伙会在任何时候出现,有的人凶的不得了。有一个一脸脏兮兮的人,我猜他是个鞋匠,总是早上七点钟就钻到走廊里,朝楼上的米考伯先生嚷嚷说:“下来!你还没出门呢,你知道的。还我们钱,好不好?别藏着,你知道,那太卑鄙了,我要是你就不会这么卑鄙。还我们钱,好不好?你要还我们钱,听见了吗?下来!”这番辱骂得不到回应,他就气得骂出“骗子”、“强盗”这些字眼来。但这样仍得不到回应,他就走到街对面,冲着三楼的窗子(他知道米考伯先生住哪一层)叫骂。这时,米考伯先生可真是羞愧难当,悲愤难抑,甚至(有一次他太太尖叫起来,我才发现)要用刮胡子刀抹自己的脖子。可是再过半个小时不到,他就会不惜力气地擦亮皮鞋,哼着曲子出门,摆着比平时更神气的架势出门去。米考伯太太也同样能屈能伸。我曾亲眼看到她下午三点钟时被税单逼得急昏过去,可是四点钟时,她就吃起炸羊排,喝起热麦酒来(这些吃食可是当掉两个银茶匙后买的)。有一次,我偶然提前在六点钟回家,见她昏倒在火炉前(还带着双胞胎中的一个),头发披散在脸上,原来法庭刚刚采取了强制手段。可就在那天晚上,她一面在厨房的灶前烤牛肉,一面给我讲她爸爸妈妈的故事,还讲起他们过去交往的朋友\footnote{〔他们过去交往的朋友〕指米考伯太太的父母以前交往的上流社会的朋友。},我再没见过她那样兴高采烈了。
    
    我就在这所房子里,和这一家人一起,度过工余的闲暇,我给自己的早餐是一便士的面包和一便士的牛奶。我把另一小片面包和另一小块干酪\footnote{〔干酪〕干奶酪,一种发酵的牛奶制品,欧洲常见食物。}收在某个碗橱里固定的一层,留着我晚上回家做晚餐。我很清楚,在我六或七先令的工资里,这是笔很大的开支了。我整天都在货行,那笔钱得支撑我过完整整一星期。从星期一早晨直到星期六的夜晚,没有任何人给我任何忠告、意见、鼓励、安慰、帮助或支持,我记得一清二楚。
    
    米考伯先生的困难更加重了我的精神痛苦。在这种孤苦伶仃的情形下,我和米考伯一家建立了很深的感情,心里时时惦着米考伯太太的各种筹款计划,心头时时压着米考伯先生的债务。星期六的夜里是我的好时光——一方面因为我口袋里有了六或七个先令,回家的路上望着街边的店铺,盘算着这笔钱可以买什么,这可是了不起的事;另一方面因为我能提早回家——米考伯太太会把最伤心的秘密向我倾诉;星期天早上,我把头天晚上买回的茶或咖啡在刮脸用的小罐里调好,开始坐下吃早餐时,也是如此。有一次,星期六的夜间谈话开始时,米考伯先生就泣不成声,而谈话将近结尾时,他却又在唱“杰克爱的是他可爱的南”了\footnote{〔杰克爱的是他可爱的南〕18世纪英国作曲家查理斯·迪布丁所作歌曲《可爱的南》的第一句。}。我曾看到他流着泪回家吃晚饭,嘴里叨念说除了进监狱,没有别的路了;可到了上床睡觉前,他又开始盘算如果“有朝一日,时来运转”(这是他很引以自得的句子),给房子装上凸肚窗\footnote{〔凸肚窗〕街边高层居屋凸出外墙,类似阳台的外凸悬窗。富人改装居屋的常见做法。}要多少钱。米考伯太太跟他完全一样。
    
    我们各自的境遇在我们之间形成了一种奇特的友好平等关系,虽然我们的年龄悬殊得可笑。我从不肯接受他们的出钱宴请,和他们吃喝,因为我知道他们和屠户及面包商关系紧张,他们自己通常也没什么太多的吃食。直到某天夜里发生的一件事,让米考伯太太把我视作她的心腹之交。情况是这样的:
    
    “科波菲尔先生,”米考伯太太找到我说,“我不把你当外人,所以不怕对你说:米考伯先生的困难已经到了危急关头了。”
    
    听到这话,我好生难过,看着米考伯太太红红的眼睛,我无比同情。
    
    “除了一块荷兰干酪的皮——这可不是小孩子能吃的”米考伯太太说,“——食品间\footnote{〔食品间〕食品储藏间,寒冷地区住宅里专门放食物的房间。}里真是什么也没有了。抱歉,我和爸爸妈妈住在一起时,总习惯了说食品间,不知不觉就用上了。我的意思是:家里什么吃的也没有了。”
    
    “天哪!”我叫起来。
    
    那时我口袋里那星期的工钱还有两或三先令——这么想来,我们的谈话大概发生在某个星期三的晚上——我连忙掏出来,恳请米考伯太太收下,权当向我借的。米考伯太太一边吻我,一边叫我把钱放回口袋,并说她连想也不能这么想。
    
    “不能这样,我亲爱的科波菲尔先生,”她说,“我压根就没往这上面想!不过,你显得比你的实际年龄要老成,如果你愿意,你可以在另一件事上帮我,我一定满怀谢意接受这种帮助。”
    
    我请米考伯太太说出来。
    
    “我已经把日常餐具脱手了,”米考伯太太说,“六把茶匙,两把盐匙,一把糖夹,我分了几次偷偷拿出去,亲自当掉了。这可是爸爸妈妈留下来的呀——每想到这我就痛心,但这双胞胎是个大包袱呀。我们还有几件小物件可以脱手。米考伯先生的感情决不允许他亲自来处置这些东西,克莉吉特呢,”——那个从济贫院来的女孩——“又是个下流坯子\footnote{〔下流坯子〕资产阶级基于出身论的说法,认为贫苦出身必然导致道德败坏。},得了信任,怕就要放肆起来了。科波菲尔先生,是不是可以请你——”
    
    我明白了米考伯太太的意思,便求她只管差使我。当天夜里,处置就开始了,从那些较轻便的财物起头。每天早晨去货行之前,我几乎总要为这样的交易出门一次。
    
    在一个米考伯称做图书馆的小柜上,有几本书,是首先脱手的。我把这些书一本接一本拿到都会路的一家书摊上——那条路上,离我们家近的一块,当年几乎全是卖书和卖鸟的——不管多少钱就都卖了。摊主住在书摊后的小房子里,他每天晚上都酩酊大醉,早晨就被他妻子痛骂一顿。不止一次,我一早上到那儿时,他就在一张翻直的床前接见我,额头的伤痕或淤肿的眼睛,证明他头天晚上又喝多了;他把发颤的手伸进乱扔在地板上的衣服,在一只只口袋里摸出钱来给我,他的妻子则抱着一个小毛头,趿着一双便鞋,骂他个没完没了。有时,他把钱弄丢了,就让我下次再来,可他老婆总有点钱,我们一起下楼时,就偷偷把钱结了。
    
    在当铺里,我也混出了名头。掌柜的很留心我,常常考我拉丁文的变位\footnote{〔拉丁文的变位〕当时英国只有上流社会的儿童能接受拉丁文教育。这里典当行的掌柜用来判断主角是否是冒充有钱人家小孩的小偷。},生怕我的东西来路不正。每次我把什么东西顺利当掉了,米考伯太太就会办一个小庆祝会,大致是顿晚餐。那来之不易的美味,我至今都记得很清楚。
    
    终于,米考伯先生的困难到了危急关头。一天清早,他被捕了,进了国王法庭监狱\footnote{〔国王法庭监狱〕伦敦监狱,19世纪以关押底层破产债务人出名,位于泰晤士河南岸的南垛区,离墨修士区约3公里。}。他走出家门时对我说,他的末日降临了——我真的以为他的心都碎了,我的心也碎了。可后来我听说,就在那天上午,还有人看见他兴高采烈地在玩九柱戏\footnote{〔九柱戏〕投球撞倒九根瓶柱的游戏,现代保龄球的前身。}呢。
    
    他入狱后的第一个星期天,我打算去看望他,和他一起吃午饭。由于从来没去过,也不熟地方,我还得问路。得先到一个地方,到了地方就能看到另一个地方,到了那里又能看到附近的一个院子,过了那院子一直走下去,直到看到监狱的看守。我如此这般一一做来,竟没有迷路(当年的我是个多么可怜的小家伙!)。当我终于看到看守的时候,我的心跳得那样快,那看守的模样在我模糊的泪眼里直晃荡。
    
    米考伯先生在大门里等我。我们走上去到他的房间里,大哭了一场。我记得,他郑重地要求我以他的下场为鉴,要我牢记:年收入二十镑\footnote{〔镑〕指英镑,英国货币单位。一英镑等于二十先令。}的人,如果每年花去十九镑十九先令又六便士,他会过得快活无比;但如果花了二十镑一先令,那他就惨了。他向我借了一先令,买了杯黑啤酒\footnote{〔黑啤酒〕18世纪产生于英国伦敦的一种啤酒,广受搬运工欢迎,也叫搬运工啤酒。},还给我写了收据,说让太太还我,然后就收起满是涕泪的小手帕,兴致又高了起来。
    
    我们坐在一个小火炉前。生锈的炉栅上,一边放了一块砖,以免烧煤太多。我们在那里坐着,等到米考伯先生的狱友从面包房回来。他带来一盘羊里脊肉。这就是我们的午餐了。然后,我又被派到最顶头的房间去见“霍普金斯船长”,带去米考伯先生的问候,就说我是米考伯先生的小朋友,向他借一把刀叉。
    
    霍普金斯船长借了我刀叉,并要我向米考伯先生问好。他的房间里有个脏兮兮的女人,还有两个病恹恹的女孩蓬着头发,大概是他的女儿。我想幸好我借的是霍普金斯船长的刀叉而不是他的梳子。船长本人邋遢得无以复加。他长着一脸大胡子,穿着件很旧的褐色外套,外套下没有内衣。我看到他的床折起放在角落里,看到搁板上放着锅碗瓢盆。我断定那两个蓬头发的女孩是霍普金斯船长的女儿,可那脏兮兮的女人并不是霍普金斯船长的妻子。我怯怯地在他门口呆了不过两分钟,却带回这么多见识,牢靠得就像我握在手里的刀叉一样。
    
    那顿午饭有种吉普赛\footnote{〔吉普赛〕欧洲的底层流浪民族。}的风情,很是惬意。饭后,我去还了刀叉,然后就回家,将探访的情况向米考伯太太汇报,以给她安慰。一看到我回来,她就昏了过去;然后我们细细谈起里面的情况,她热了一小罐鸡蛋甜酒来压惊。
    
    我不知道那些家具是怎么卖出去的,又是谁经手的——肯定不是经我的手。不过,家具确实都被卖掉了,用一辆货车拖走的,只剩下床和几把椅子,还有一张厨房用的桌子。带着这点东西,我们像野营一样住在温泽巷那所空房子的两间客厅里。米考伯太太,孩子们,那孤儿,还有我,都日夜住在那两间房间里。我不知道到底住了多久;不过我觉得很久很久。最终,米考伯太太决定搬进监狱去住,因为米考伯先生在那里搞到了一个单间。于是,我把钥匙还给房东,他很乐意地收回钥匙。几张床全送到国王法庭监狱去了,只剩下我的床。我把它送到我另外租的一个小房间里。新的住处离监狱不远,这很合我意。因为我和米考伯一家患难与共,已经难舍难分了。那孤女也在附近找到一个房租低廉的住处。我的卧室是一个斜屋顶下的后阁楼,面朝一个木料场,风景大好。住在这里,想到米考伯先生家的困境,我觉得这小屋算是天堂了。
    
    那段日子里,我一直在莫德斯通–古林比公司的货行,与同样卑贱的工友为伍做苦力,心里仍和刚来的时候一样,怀着屈辱。每天都有很多少年进出货行,午饭后在街头游荡,我从不和他们之中任何人交谈。现在想来,幸亏如此。还是这样,把悲伤藏在心里,独自难过,一切只靠自己。我能觉察到的变化只有两点:第一,我的衣着更寒碜了;第二,对米考伯夫妇的忧虑,也在逐渐减轻;因为他们的一些亲戚和朋友,也出面来帮助他们渡过难关了。他们在狱中的生活,反比入狱前还过得更惬意。靠着某种安排,我得以常常和他们一起吃早饭。监狱的门早上什么时候开,什么时候能让我进去,我也记不清了。我记得我常常六点起床,监狱的门还没开,我就到老伦敦桥\footnote{〔老伦敦桥〕泰晤士河上的桥。在墨修士桥东边,从国王法庭监狱门口沿大道往东北约700米即达。由于年久失修,1831年主角11岁时,新伦敦桥在旁边建成,次年老桥被拆除。}走走。我常坐在桥上的石龛里,看过往行人,或趴在栏杆上,看朝阳在照在河面上,泛起万点波光,又把伦敦大火纪念碑\footnote{〔伦敦大火纪念碑〕为纪念1666年伦敦大火而建的石碑柱。在伦敦桥北侧100米,高62米,碑顶为火焰形状的镀金雕塑。}顶上的黄金火焰点亮。晚上,我总回监狱去,和米考伯先生在空地上散步,或和米考伯太太玩牌,听她回忆她爸爸妈妈的事情。莫德斯通先生是否知道我在什么地方,我也说不准。我从来不告诉莫德斯通–古林比公司的那些人。
    
    米考伯先生虽然捱过了危急的关头,却又卷入了某个“契据”的麻烦里。关于“契据”的事,当时的我听了不少。现在想来,那应该是他先前写给债权人的某种文书,不过当时我怎么也闹不明白,还把那玩意儿和德国人说的那种“和魔鬼的契约”搞混了\footnote{〔魔鬼的契约〕指浮士德与魔鬼的契约。浮士德是16世纪德国民间传说的神秘人物,传说与魔鬼立下契约,把灵魂卖给了魔鬼,以换取财富和知识。}。后来,不知怎么地,那个“契据”似乎失效了。不管怎么说,不再碍事了。米考伯太太告诉我,“她娘家人”已决定:米考伯先生应当根据破产债务人法的要求,获得释放。她预计,再过六个星期,就能获得自由。
    
    “到那时候,”米考伯先生说,当时他也在一旁,“我就再也不欠债了。谢天谢地呀!如果——简言之,有朝一日,时来运转,我一定要洗心革面,重新做人。”
    
    大约这个时期,米考伯先生还起草了一篇呈文\footnote{〔呈文〕向政府官员递交的文书,陈述意见。}给下议院\footnote{〔下议院〕英国国会分上下两议院。下议院议员由选区的选民直选产生。上议院议员为基督教会指定的神职人员和君王指定的贵族。},恳请修改因债务坐牢的法律。监狱里有个俱乐部\footnote{〔俱乐部〕进行社交、文娱活动的团体和场所。}。由于米考伯先生是上流人物,他成了其中了不起的权威人士。米考伯先生把这呈文的意见在俱乐部里宣布后,得到了热烈赞同。于是,米考伯先生——他是个地地道道的好好先生,哪怕对利益无关的事也热心积极——就着手写起呈文来。起草后又用一张大纸誊好,铺在一张桌子上,让有意者——俱乐部成员或监狱里的任何人——来他房里签名。
    
    我听说了这盛典,自然渴望见识一下。即便我已经和狱中大多数人见过面了,我还是向货行请了一个小时的假,来仔细观察。我缩在角落里,看俱乐部的要员们挤进小单间里。米考伯先生被簇拥在那呈文前,而我的老朋友,洗沐一新的霍普金斯船长就站在呈文旁边,把它读给人听。房门大开,狱友们一个接一个地进来,签了名就出去。霍普金斯对每一个进来的人都说:“你读了呈文吗?”——“没有。”——“你想听人读呈文吗?”哪怕那人略有半点想听的表示,霍普金斯就会响亮地把呈文逐字读给他听。如果有两万个人想听他读,这位船长一定会把它读上两万遍。我还记得,每当读到“出席国会的人民代表们”、“故请愿人敬向贵院请求”、“仁慈陛下的不幸小民”这类话时,他总要摇头晃脑,好像这些话在他嘴里变成了什么珍馐美味一样。米考伯先生也听着,怀着作者的些许虚荣心,目光空洞地停留在对面墙上的大铁钉上。
    
    每天,我都来往于南垛区和墨修士区之间。午饭后,就在偏僻的小路上转悠。街上的石头想必都被我那双孩子的脚踩平了。当年伴着霍普金斯洪亮的朗诵声一个个受我检阅的那些人中,多少人已不在人世了!回首前尘,恍如隔世。那模糊的回忆里,笼罩着在层层迷雾中的人和事,有多少是想象,多少是真实?可是我毫不怀疑,当我重返旧地,眼前一定会浮现出一个天真浪漫的少年,在我面前走着,凭着那些坎坷,那些独特的经历、悲惨的环境,创造出自己幻想的世界。
    
\end{normalsize}


\newpage

\textbf{注释}:

\vspace{-1em}

\begin{itemize}
    \setlength\itemsep{-0.2em}
    \item 〔涉世〕接触社会,经历世事。
    \item 〔遐想〕超越现实作无拘无束的想象。
    \item 〔费心〕操心,耗费心思。
    \item 〔宽限〕放宽限期。
    \item 〔管见〕管中窥物所见,比喻所见浅小。多用作形容自己意见的谦词。
    \item 〔悬殊〕差别、差距很大。
    \item 〔闲暇〕空闲的时间。
    \item 〔病恹恹〕精神萎靡不振,仿佛生病的样子。
    \item 〔憔悴〕瘦弱无力脸色难看的样子。
    \item 〔空空如也〕空荡荡,什么也没有。
    \item 〔心腹之交〕可靠的、可以相信的人。心腹:指身体重要部位。可以将心腹交给其照看,比喻可以分享秘密、托付重任的人。
    \item 〔孤苦伶仃〕孤独而困苦,没有依靠。
    \item 〔酩酊大醉〕形容醉得很厉害。
    \item 〔洗心革面〕清洗旧的思想,去掉旧的面貌,指彻底悔改。
    \item 〔捱〕承受,拖延。
    \item 〔邋遢〕脏乱污秽,不整洁。
    \item 〔寒碜〕形象差,不体面。
    \item 〔誊〕照原稿抄写清楚。
    \item 〔恍如隔世〕恍惚间如同相隔了一辈子,形容人与事或者景物的前后差别非常大,引起感触。
    \item 〔珍馐〕珍奇名贵的食物。
    \item 〔趿〕把鞋子后帮踩在后脚跟下。
    \item 〔惬意〕舒畅愉快,称心如意。
    \item 〔坎坷〕道路高低不平。比喻生活困难、事情不顺利,不得志。
\end{itemize}

\chapter{复活}

\begin{normalsize}
    
    玛丝洛娃转过身,抬起头,挺起胸,带着聂赫留朵夫熟悉的温顺表情,走到铁栅栏跟前,从两个女犯中间挤过来,惊讶而疑惑地盯着聂赫留朵夫,却没有认出他来。
    
    不过,她从衣装上看出他是个有钱人,就凑出一个笑容。
    
    “您找我吗?”她问道。笑脸盈盈,稍稍眯起眼,凑近铁栅栏。
    
    “我想见见……”聂赫留朵夫不知道该用“您”还是“你”,但随即决定用“您”。他用平常的声量说:
    
    “我想见见您……我……”
    
    “不!我跟你说,你就是胡扯!”他旁边那个衣衫褴褛的男人大叫道,“你到底吃了没?”
    
    “……快死了,我跟你说!你还想怎么着?”对面有一个人尖声喊道。房间里实在太吵了\footnote{〔房间里实在太吵了〕探监的人和囚犯在两个分开的房间里,中间隔着铁丝网和两米宽的隔道。因此两边的人都要大声喊才能听见对方说的话。}。玛丝洛娃听不清聂赫留朵夫在说些什么,但他说话时脸上的表情使她突然想起了他。她不敢相信自己的眼睛。她的笑容消失了,眉头痛苦地皱起来。
    
    “您说什么?我听不见!”她叫起来,眯细眼睛,眉头皱得更紧了。
    
    “我来是……”
    
    “对,我在做我该做的事,我在认罪。”聂赫留朵夫想。他一想到这里,眼泪就夺眶而出,喉咙也哽住了。他用手指抓住铁栅栏,说不下去,竭力控制住感情,免得哭出声来。
    
    “我说,你怎么总出现在不该出现的地方?”这边有人喝道。
    
    “老天爷在上,我什么也不知道!”另一头有个女犯大声说。
    
    玛丝洛娃看到聂赫留朵夫激动的神气,认出他来了。
    
    “您好像是……不,我不认识您。”玛丝洛娃眼睛不看他,叫道。她那涨红的脸突然变得阴沉了。
    
    “我来是要请求您饶恕。”聂赫留朵夫大声说,但声调没有起伏,像背诵熟读的课文一样。
    
    他大声说出这句话,感到害臊,往四下里张望了一下。但他立刻想到,要是他觉得羞耻,那倒是好事,因为他确是可耻的。于是他高声说下去:
    
    “请您饶恕我,我害惨了您……”他又叫道。
    
    她一动不动地站着,眯起的眼睛盯住他不放。
    
    他再也说不下去,就离开铁栅栏,竭力忍住翻腾着的泪水,不让自己哭出声来。
    
    把聂赫留朵夫领到女监来的副典狱长,显然对他发生了兴趣,这时走了过来。他看见聂赫留朵夫不在铁栅栏旁边,就问他为什么不跟他要探望的女犯谈话。聂赫留朵夫擤了擤鼻涕,提起精神,竭力让自己平静下来,回答说:
    
    “隔着铁栅栏没法说话,什么也听不见。”
    
    副典狱长沉思了一下。
    
    “嗯,好吧,让她过来一下也行。”
    
    “马丽雅·卡尔洛夫娜!”他转身对女看守说,“把玛丝洛娃带到外边来。”
    
    过了一分钟,玛丝洛娃从侧门出来了。她步履轻盈,走到聂赫留朵夫跟前,站住,皱着眉头看了他一眼。乌黑的鬈发和两天前一样,一圈圈飘在额头上;苍白而微肿的脸有点病态,但很诱人,而且十分冷静;她那双乌黑发亮的眼睛在浮肿的眼皮下显得特别有神。
    
    “你们可以在这里谈。”副典狱长说完耸了耸肩,带着有些诧异的神情离开了。
    
    聂赫留朵夫走到靠墙的长凳旁边。
    
    玛丝洛娃困惑地瞧了瞧副典狱长,然后惊讶地耸耸肩膀,跟着聂赫留朵夫走到长凳那儿,理了理裙子,在他旁边坐下。
    
    “我知道要您饶恕我很困难,”聂赫留朵夫开口说,但又停住,觉得喉咙哽住了,“不过,尽管过去的事已经无法改变,现在我一定尽我所能。您说说……”
    
    “您怎么找到我的?”她不理他的话,径自问。眼睛眯起来,又像在瞧他,又像在瞧别处。
    
    “老天啊,帮帮我吧!教教我该怎么办!”聂赫留朵夫望着她改变了的容貌,暗自说。
    
    “前天您受审的时候,我是陪审员。”他说,“您没有认出我来?”
    
    “没有,没认出来。我没那个工夫认人。我根本没想去看。”玛丝洛娃说。
    
    “我们有过一个孩子,对吧?”聂赫留朵夫问,感到脸红了。
    
    “谢天谢地,他当时就死了。”她简短而恶毒地答道,眼睛移开,不去看他。
    
    “什么意思?为什么?”
    
    “我当时也病了,差一点死掉。”玛丝洛娃用同样平静的语调说,没有抬起眼睛来。聂赫留朵夫没有料到这种平静,他无法理解。
    
    “姑妈她们怎么会放您走的?”
    
    “谁会把生了孩子的女佣人留在家里呢?她们一发现这事,就把我赶出来了。说这些干什么呀!我早就忘光了。这事早就过去了。”
    
    “不,没有过去,在我心里,这事还没有过去。哪怕到今天,我也要补偿我的罪过。”
    
    “没有什么要补偿的。过去的事都过去了,都结束了。”玛丝洛娃说。接着,她忽然瞟了他一眼,微微一笑,完全出乎他的意料。那是一种勾引人的、卖弄可怜的笑容,令人不快。
    
    玛丝洛娃怎么也没想到会看见他,特别是在此时此地,因此最初的一刹那,他的到来震惊了她。这使她回想起她从不愿回想的往事。最初的一刹那,她模模糊糊地想起那个充满感情和理想的新奇天地。那个迷人的青年给她打开了一扇门。她是那么爱他,他也那么爱她。然后她想到了他那让人难以理解的残酷,想到了由那醉人的幸福带来的、紧随着降临的一连串屈辱和苦难。她感到痛苦,她无法理解。于是她就依从人的天性,把这些事从头脑里摒去,竭力用堕落生活的种种迷雾把它遮住。此刻她也是这么做的。最初的一刹那,她把坐在她面前的这个人同她一度爱过的那个青年联系起来,但紧接着,她就觉得太痛苦,必须要把这两个形象分开。眼前这个衣冠楚楚、脸色红润、胡须上洒了香水的老爷,对她来说,已不是她曾经爱过的那个聂赫留朵夫,而是另一种人了。那种人玩弄像她这样的女人,只为了满足自己的欲求;而像她这样的女人也必须尽量从他们身上多弄些好处。于是,她换上了勾引男人的笑容。她默默地考虑着该怎样利用他,多弄些好处。
    
    “一切都结束了。”她说。“如今我被判决,要去西伯利亚\footnote{〔西伯利亚〕亚洲北部地区,环境严酷,是俄罗斯流放犯人的地方。}服苦役了。”
    
    她说出这句悲痛的话,嘴唇都颤抖起来。
    
    “我知道;我很清楚,您是无罪的。”聂赫留朵夫说。
    
    “我当然是无罪的。我又不是小偷,又不是强盗。”,她说到这儿顿了顿,想着怎么说才能从他那儿多拿点好处,“这儿大家都说,一切都要看律师,”她继续说道,“大家都说应该上诉,可上诉要花很多钱……”
    
    “是的,一定要上诉。”聂赫留朵夫说,“我已经找过律师了。”
    
    “别舍不得花钱,得请一个好律师。”她说。
    
    “我一定尽力。”
    
    接着是一阵沉默。
    
    她又微微一笑,还是那种勾引人的、卖弄可怜的笑。
    
    “我想请您……给些钱,要是您答应的话。不多……只要十个卢布\footnote{〔卢布〕俄罗斯货币单位。}就行。”她突然说。
    
    “当然,当然。”聂赫留朵夫窘态毕露地说,伸手去掏皮夹子。
    
    她瞅了一眼正在屋里踱步的副典狱长。
    
    “别当着他的面给,等他走开了再给,不然会被他拿走的。”
    
    等副典狱长一转过身去,聂赫留朵夫就掏出皮夹子,但他还没来得及把十卢布钞票递给她,副典狱长又转过身来,脸对着他们。他只得把钞票团在手心里。
    
    “这个女人已经死了。”他心想。端详着这张原来亲切可爱、如今饱经沧桑的浮肿的脸。那双乌黑发亮的眼睛眯缝着,闪着恶毒的光,紧盯着副典狱长和聂赫留朵夫那只紧捏着钞票的手。突然间,他的内心动摇了。
    
    昨晚诱惑过聂赫留朵夫的魔鬼,此刻又在他心里说话,又竭力阻止他思考该怎样行动,却让他去考虑他的行动会有什么后果,怎样才能对他有利。
    
    “这个女人已经无可救药了,”魔鬼说,“你只会把石头吊在自己脖子上,活活淹死,再也不能做什么对别人有益的事了。给她一些钱,把你身边所有的钱全给她,同她分手,从此一刀两断,岂不更好?”这个声音在他心里回荡。
    
    不过,他同时又感到,他的心灵里此刻正要发生一种极其重大的变化,他的精神世界这会儿仿佛搁在不稳定的天平上,只要稍稍加一点力气,就会向这边或者那边倾斜。他花了一点力气,向昨天感受到的,存于心里的神明呼救,果然,神明立刻回应了他。他决定此刻把所有的话全向她说出来。
    
    “卡秋莎!我来是要请求你的饶恕,可是你没有回答我,你是不是饶恕我,或者,什么时候能饶恕我?”他说,忽然对玛丝洛娃改称“你”了。
    
    她没有听他说话,却一会儿瞧瞧他那只手,一会儿瞧瞧副典狱长。等副典狱长一转身,她连忙把手伸过去,抓住钞票,把它塞进腰带里。
    
    “您的话真怪。”她笑了笑,笑里带着鄙夷。他也马上察觉到这鄙夷。
    
    聂赫留朵夫觉得她身上有一样东西,同他水火不相容,使她永远保持现在这种样子,并且不让他闯进她的内心世界。
    
    不过,说也奇怪,这不但没有使他疏远她,反而产生一种特别的新的力量,促使他去接近她。聂赫留朵夫觉得他应该在精神上唤醒她,这虽然极其困难,但正因为困难就格外吸引他。他现在对她的这种感情,是以前所不曾有过的,对任何人都不曾有过,其中不带丝毫私心。他对她毫无所求,只希望她不要像现在这样,希望她能觉醒,能恢复她的本性。
    
    “卡秋莎,你为什么说这样的话?你要明白,我是了解你的,我记得当时你在巴波伏的样子……”
    
    “何必提那些旧事?”她冷冷地说。
    
    “我提起这些事是为了改正错误,弥补我的罪过,卡秋莎。”聂赫留朵夫开了头,本来还想说要同她结婚。但他接触到她的目光;那目光如此拒否、如此粗暴、如此可怕,令他不敢继续说下去。
    
    这时候,探监的人渐渐出去了。副典狱长走到聂赫留朵夫跟前,说探望的时间结束了。玛丝洛娃站起来,顺从地等待看守把她带回牢房。
    
    “再见!我还有许多话要对你说,可是,你看,现在没时间了。”聂赫留朵夫说着伸出一只手。“我会再来的。”
    
    “我想您说的已经够了。”
    
    她轻轻碰了碰他的手。
    
    “不,我会再来看你的。我们找个可以畅谈的地方,我还有话要对你说——非常重要的话——”聂赫留朵夫说。
    
    “好呀,那您就来吧。”她说着,微微一笑。是那种熟练的、招徕的、暗示的笑,那种取悦男人的媚笑。
    
    “你对我来说比妹妹还重要!”聂赫留朵夫说。
    
    “您的话真怪!”她又重复了一遍,然后摇摇头,向铁栅栏那边走去。
    
\end{normalsize}


\newpage

\textbf{注释}:

\vspace{-1em}

\begin{itemize}
    \setlength\itemsep{-0.2em}
    \item 〔鬈发〕卷曲的头发。
    \item 〔擤〕捏住鼻子,呼气排出鼻涕。
    \item 〔毕露〕完全暴露。
    \item 〔鄙夷〕鄙视,轻视。
    \item 〔招徕〕招揽(顾客)。
\end{itemize}

\chapter{老人与海}

\begin{normalsize}
    
    航行很顺利。老头儿把手浸在盐水里,努力保持头脑清醒。天上,积云堆得很高,上面还有相当多的卷云。老头儿知道,风要刮上一整夜了。老头儿一直盯着鱼看,好确定它是真的。一小时后,第一条鲨鱼袭来了。
    
    鲨鱼的出现不是偶然的。一大片暗红的血云沉入深海,四处扩散,自然把它从深处吸引来了。它上来得那么快,全然不顾一切,竟然冲出了蓝色的水面,现身阳光之下。接着它又掉回海里,嗅着血腥气,沿着鱼和船的轨迹追来。
    
    有时它也会跟丢了腥味,但总会重新嗅到。就嗅到那么一丁点儿,它就使劲飞快跟上。它是条很大的灰鲭鲨,生就一副好体格,能游得跟海里最快的鱼一般快。全身都透着一种美感,除了上下颚。它的背部和剑鱼的一般蓝,肚子是银色的,鱼皮光滑而漂亮。高耸的脊鳍刀子一般,稳稳地划开水面,就和剑鱼一般,除了全速游动时紧闭着的大嘴。紧闭的厚唇里面,是八排内倾的利齿。它的牙齿和大多数鲨鱼的不同,不是一般的金字塔形排列。它们像爪子,像蜷曲的手指,几乎跟老头儿的手指一般长,两边都有刀片般锋利的刃口。这种鱼生来就拿海里所有的鱼当食物。它们游得那么快,长得那么壮健,武器齐备,所向无敌。它嗅到了这新鲜的血腥气,此刻正加快了速度,蓝色的脊鳍划破了水面。
    
    老头儿看着它游来,就知道这是条毫无畏惧、恣睢蛮横的鲨鱼。他准备好了鱼叉,系紧了绳子,看着鲨鱼游近来。绳子被他割了一截去绑鱼,此时显短了。
    
    老头儿此刻很清醒,心中坚定,但并不抱多少希望。好事难留啊。他抽空瞟了一眼那条大鱼。它太美好,也许是一场梦,他想。也许我阻止不了它咬掉我的鱼,可我能弄死它。尖牙利嘴的家伙\footnote{〔尖牙利嘴的家伙〕当地对灰鲭鲨的说法。},你死期到了。
    
    鲨鱼追上了船梢。它袭击那鱼的时候,老头儿看见它张开了嘴,看见它那双奇异的眼睛。它咬住鱼尾巴上面一点儿的地方,牙齿咬得嘎吱嘎吱地响。鲨鱼的头露出水面,背部也露了出来,老头儿听见大鱼的皮肉被撕裂的声音。就在此时,他的鱼叉猛地朝下一戳,扎进鲨鱼的脑袋,正扎在它两眼正中间,两眼连线和鼻子到脊鳍连线的交点。鲨鱼并不会在身上标出这两条线。只有那尖而沉重的蓝色脑袋,两只大眼睛,和那嘎吱作响、吞噬一切的突出的两颚。但那交点是鲨鱼脑子所在,而老头儿击中了。他使出全身力气,用糊着鲜血的双手,把一根好鱼叉扎了进去。这一扎并不抱希望,但充满决心和恶意。
    
    鲨鱼翻了个身,老头儿看得出,它眼睛里已经没有生气了;随后它又翻了个身,自行缠上了两道绳子。老头儿知道,这鲨鱼快死了。但它还是不肯认输,它肚皮朝上,大半的身体露在水面上,两颚嘎吱作响,尾巴扑打着,像螺旋桨一样拍击水面,把水拍得泛出白色。绳子绷得太紧,抖动了一下,啪地断了。鲨鱼在水面上静静地躺了片刻,老头儿紧盯着它。然后它慢慢沉下去了。
    
    “它吃了约莫四十磅\footnote{〔磅〕英制重量单位,一磅约为453克。}。”老头儿大声说。它把我的鱼叉也带走了,还有许多绳子,他想。而且现在我这条鱼又淌血了,别的鲨鱼也会来的。
    
    他不忍心再朝这死鱼看上一眼,因为它已经被咬得残缺不全了。鱼遭到袭击的时候,他感觉就像自己遭了袭击一样。不过,我把吃我鱼的鲨鱼干掉了,他想。它是我见过最大的灰鲭鲨。老天作证,我可算见过大鱼了。
    
    好事难留啊。但愿这是一场梦,我根本没有钓到这条大鱼,而是在家里,在床上,一个人,躺在旧报纸上。
    
    “不过,人不是为失败而生的。”他说,“人可以被摧毁,但不能被打败。”不过我现在后悔杀这条鱼了,他想。接下来还有大麻烦,而我连鱼叉也没有。这条鲨下手无情,既强壮又聪明。但我比它更有智慧。真是如此吗,他想,也许我仅仅是武器比它强。
    
    “别想啦,老家伙,”他大声说道,“就这么驶下去,见招拆招吧。”
    
    可不早作打算不行啊,他想,现在我只剩下头脑了。头脑,还有棒球。伟大的迪马吉奥\footnote{〔迪马吉奥〕乔·迪马吉奥,二十世纪上半叶美国职业棒球联盟的明星球员,是一个渔夫的儿子,效力于洋基队,被称为“洋基快艇”。},不知道他会不会喜欢我击中它的脑子那一下?这不是什么了不起的事儿,他想,谁都做得到。难道我手上的伤比得过迪马吉奥的骨刺吗\footnote{〔……骨刺〕指迪马吉奥的脚后跟等地方有骨刺增生,职业生涯后期饱受伤病困扰。}?我可不知道。我的脚后跟从没出过毛病,除了有一次游水时,踩着了一条海鳐鱼,被它扎了一下,整个小腿都麻了,痛得受不了。
    
    “想点开心的事儿吧,老家伙,”他说,“每过一分钟,你就离家近一步。丢了四十磅鱼肉,你航行起来更轻快了。”他很清楚他驶进海流的中部以后会发生什么事。可是眼下他一点办法也没有。
    
    “不,有办法。”他大声说道,“我可以把刀子绑在桨把上。”
    
    于是他把舵把夹在腋下,一只脚踩住帆脚索,就这么把刀子绑了上去。
    
    “行了,”他说,“我不过是个老头儿,可现在我又有武器了。”
    
    风渐渐变强,船走得更快了。他只盯着鱼的前半看,恢复了一点儿希望。
    
    不抱希望才蠢哪,他想。再说,我猜我是犯了罪过了。别想罪过了,他想。麻烦已经够多了,还想什么罪过。何况我根本不懂这个。
    
    我根本不懂这个。我虔诚吗?我也不知道。也许杀死这条鱼确是罪过,我看像。尽管我是为了谋生,为了给人提供吃食才这么干的。不过话得说回来,什么事不是罪过呢?别想罪过了吧,现在想它也迟了。不是有靠研究这事吃饭的人吗?让他们去考虑吧。你天生是个渔夫,正如那鱼天生就是一条鱼一样。圣彼得\footnote{〔彼得〕基督教中耶稣的门徒之一。}是个渔夫,跟伟大的迪马吉奥的父亲一样。
    
    但是他就喜欢研究跟自己相关的事。而且因为没有书报可看,又没有收音机,他就想得很多。他一直在想罪过的事。你不光是为了谋生、为了卖鱼才杀它的,他想,你杀死它是为了你内心的骄傲,因为你是个渔夫。它活着的时候你爱它,它死了你还是爱它。如果你爱它,杀死它就不是罪过。还是说,罪过更大了?
    
    “你想得太多了,老家伙。”他大声说道。但是你很乐意杀死那条灰鲭鲨,他想。它跟你一样,吃活鱼为生。它可不要残渣碎屑,也不像有些鲨鱼那样,只为了饱腹而活。它是美丽而高贵的,无所畏惧。“我杀死它是为了自卫,”老头儿大声说道,“我给了它一个痛快。”
    
    再说,他想,万物相生也相克相杀,就看你怎么看待罢了。捕鱼养活了我,同样也在要我的命。我就指着那孩子活了,他想,好吧,别再自己骗自己了。
    
    他把身子探出船舷,从鱼身上被鲨鱼咬过的地方撕下一块肉嚼起来。肉质很好,味道鲜美,爽滑多汁,像牛羊的肉,不过不是红色的。一点筋也没有。他知道这肉在市场上能卖最高的价钱。可他无法不让这味儿在水下散开去。老头儿知道,大劫难就要来了。
    
    风还在吹,稍微转向东北方,这说明短时间里风不会停。老头儿朝前方望去,前面一艘船的影子也没有,也没有船帆,没有船上冒的烟,什么都没有。只有从他船头下跃起的飞鱼,向两边逃去,还有一摊摊黄色的马尾藻。他连一只鸟也看不见。
    
    两个小时后,他在船梢歇下,不时从大鱼身上撕下一点肉来,咀嚼着,努力休息,保持精力。就在这时,他看到了鲨鱼。一条首先露面,然后是另一条。
    
    “欸——”他发出声来。这个音是没法写出来的,只是人体里发出的声音,就像钉子穿过人的手,钉进木头时,不由自主地发出的声音。
    
    “鲨!”他大声说道。他看见一个鳍,然后另一个在它后面冒出来。他根据这褐色的三角鳍和甩来甩去的尾巴,认出它们是铲鼻鲨。它们嗅到了血腥味,很兴奋。它们想必饿昏了头,激动得乱窜。一会儿迷失了踪迹,一会儿又嗅到了;但它们一直在逼近。
    
    老头儿系紧帆脚索,卡住了舵把。然后他拿起上面绑着刀子的桨。他尽量轻地把它举起来,因为他那双手痛得不听使唤了。然后他把手张开,再握紧,让双手松弛下来,再把手握紧——这样更能承受疼痛,不会抽搐——同时注视着靠近的鲨鱼。现在他能看见它们那又宽又扁的铲子形的头,和白色尖端的宽阔的胸鳍。它们是可恶的鲨鱼,气味难闻,既杀活鱼,也吃腐烂的死鱼,饥饿的时候,它们甚至会咬船上的桨舵。这些鲨鱼会趁海龟在水面上睡觉的时候咬掉它们的脚。饿得不行的时候,还会在水里袭击没有血腥味的人。
    
    “欸——”老头儿说,“鲨!来吧,鲨!”
    
    它们来了,但来的方式和那条灰鲭鲨不同。一条鲨鱼转了个身,钻到小船底下不见了。它用嘴拉扯着死鱼,老头儿觉得小船在晃动。另一条用它一条缝似的黄眼睛注视着老头儿,然后飞快地游来,半圆形的上下颚大大张开,朝鱼身上被咬过的地方咬去。老头儿看准了它褐色顶背上的那两条连线,一桨朝交点扎去,拔出来,再扎进那猫一样的黄眼。鲨鱼放开了咬住的鱼,身子朝下溜去,临死时还把咬下的肉吞了下去。
    
    另一条鲨鱼还在啃鱼,弄得小船摇晃不稳。老头儿就松开了帆脚索,让小船横过来,把鲨鱼从船底下暴露出来。他一看见鲨鱼,就从船舷上探出身子,一桨朝它戳去。他只戳在肉上,但鲨鱼的皮紧绷着,刀子几乎戳不进去。这一戳不仅震痛了他那双手,也震痛了他的肩膀。但是鲨鱼迅速地浮上来,露出了脑袋,老头儿趁它的鼻子伸出水面挨上那条鱼的时候,对准它扁平的脑袋正中扎去。老头儿拔出刀刃,朝同一地方又扎了那鲨鱼一下。它依旧紧锁着上下颚,咬住了鱼不放,老头儿一刀戳进它的左眼。鲨鱼还是吊在那里。
    
    “还不够吗?”老头儿说着,把刀刃戳进它的脊骨和脑子之间。这里扎起来很容易,他感到它的软骨折断了。老头儿把桨倒过来,把刀刃插进鲨鱼两颚之间,想把它的嘴撬开。他把刀刃一错,鲨鱼松了嘴溜开了。他说:“走吧,鲨,回深海去吧。找你的同伴去吧,还是你妈妈?”
    
    老头儿擦了擦刀刃,把桨放下。然后他摸到了帆脚索,张起帆来,使小船顺着原来的航线走。
    
    “它们吃了不少,得有四分之一了,而且都是上好的肉。”他大声说道。“但愿这是一场梦,我压根儿没有钓到它。我真后悔了。鱼啊,我搞砸啦。”他说不下去了。他现在不敢再看那鱼了。它流尽了血,被海水冲刷着,看上去像镜子背面镀的银色,身上的条纹依旧看得出来。“我就不该出海这么远的。鱼啊,”他说,“这对你对我都是祸事。鱼啊,我知错了。”
    
    得了,他对自己说,去看看绑刀子的绳子吧,看有没有断。然后把你的手弄好,因为还有鲨鱼要来。
    
    “要是有块磨刀石就好了,”老头儿检查了绑在桨上的刀子后说。“我应该带一块磨刀石来。”你应该带的东西多着哪,他想,但是你没有带。老家伙啊,眼下可不是想还有什么东西没带的时候,想想用手头的东西能做什么事儿吧。
    
    “好了,别讲大道理了,”他大声喊道。“我听够啦。”他把舵把夹在腋下,双手浸在水里,小船朝前驶去。“天知道最后那条鲨鱼咬掉了多少鱼肉,”他说,“这船现在轻多了。”他不愿去想那鱼残缺不全的肚子。他知道鲨鱼每次猛撞上去,总要撕去一点肉;他还知道,现在鱼流出来的血腥味,宽得就像海里有一条公路一样。
    
    它是条大鱼,一个人整个冬天也吃不完,他想。别想这个啦。还是休息休息,把你的手弄弄好,保护这剩下的鱼肉吧。水里的血腥味这么浓,我手上的血腥气就算不上什么了。再说,手上出的血也不多。割伤的地方都不算严重。出点血也许能让我的左手不再抽筋。
    
    现在还有什么好想?他想,什么也没有。我必须什么也不想,等待下一条鲨鱼到来。但愿这真是一场梦,他想。不过谁说得准呢?也许事情本可以不这么糟。
    
    接着来的是条铲鼻鲨。看它的来势,就像一头猪奔向饲料槽,只是这猪的嘴大得能塞下人头。老头儿让它咬住了鱼,然后把绑在桨上的刀子扎进它的脑子。但是鲨鱼朝后猛地一扭,打了个滚,刀刃啪地一声断了。
    
    老头儿坐定下来掌舵。他没去看那条大鲨鱼慢慢沉进水底。它的身影渐渐变小,最后只有一丁点儿大了。这情景平常总叫老头儿看得入迷,可这会他看也不看一眼。
    
    “我现在还有那根鱼钩,”他说,“不过它没什么用处。我还有两把桨,舵把和那根短棍。”
    
    它们如今可把我打败了,他想,我太老了,不能用棍子打死鲨鱼了。但只要我还有桨、短棍和舵把,我就要试试。他又把双手浸在水里泡着。下午渐渐过去,快到傍晚了。除了海和天,他什么也看不见。风又大了,很快,他开始盼望看到陆地。
    
    “你累坏了,老家伙,”他说,“你精疲力尽了。”
    
    之后再没有鲨鱼来袭,直到快日落的时候。
    
    老头儿看见两片褐色的鳍沿着大鱼留下的血腥道路游来。它们竟没有迷失一点方向,并肩朝着小船笔直地游来。
    
    他刹住了舵把,系紧帆脚索,伸手到船梢下去拿棍子。这棍子原本是个桨把,是从一支断桨上锯下来的,大约两英尺半长\footnote{〔英尺〕英制长度单位。一英尺约等于0.305米。}。上面有个把手,所以他只能用一只手握着。于是他就用右手牢牢握着棍子,同时望着游近来的鲨鱼。两条鲨鱼。
    
    我必须让第一条鲨鱼好好咬住了,再打它的鼻尖,要么直接朝它头顶正中打,他想。
    
    两条鲨鱼一起紧逼过来。他一看到离他较近的那条张开嘴,直咬进那鱼的银色胁腹,就高高举起棍子,重重地打下去,“砰”的一声打在鲨鱼宽阔的头顶上。棍子落下去,他觉得仿佛打在了坚韧的橡胶上。但他也感觉到坚硬的骨头,他就趁鲨鱼从那鱼身上朝下溜的当儿,再重重地朝它鼻尖上打了一下。
    
    另一条鲨鱼刚才窜来后就走了,这时又张大了嘴扑上来。它直撞在鱼身上,闭上两颚,老头儿看见一块块白色的鱼肉从它嘴角漏出来。他抡起棍子打去,只打中了头部。鲨鱼朝他看看,把咬在嘴里的肉一口撕下了。老头儿趁它溜开去把肉咽下时,又抡起棍子朝它打下去,只打中了一块厚实地方,像打在橡胶上。
    
    “来吧,鲨,”老头儿说,“再来一次。”
    
    鲨鱼冲上前来,老头儿趁它合上两颚时给了它一下。他结结实实地打中了它,是把棍子举得尽量高才打下去的。这一回他感到打中了脑后的骨头,于是朝同一部位又是一下,鲨鱼呆滞地撕下嘴里咬着的鱼肉,从鱼身边溜下去了。
    
    老头儿守望着,等它再来,可两条鲨鱼都不露面。接着他看见其中的一条在海面上绕着圈儿游着。他没看见另外一条的鳍。
    
    我没法指望打死它们了,他想。我年轻力壮时能行。不过我已经把它俩都打成重伤,哪一条都不好过。要是我能用双手抡起一根棒球棒,我准能把第一条打死。即使现在也能行,他想。
    
    他不愿朝那条鱼看。他知道它的半个身子已经被咬烂了。在他跟鲨鱼搏斗的功夫,太阳已经落下去了。
    
    “马上就天黑了,”他说,“到时候我就能看见哈瓦那\footnote{〔哈瓦那〕古巴首都,北边是港口,西北是墨西哥湾,东北是大西洋。}的灯光。如果我往东走得太远了,那就是新开发的海滩的灯光。”
    
    我现在离陆地不会太远,他想,但愿没惹人担心。当然啦,会担心的也就只有那孩子。可我相信他,他对我有信心。好多老渔夫也会担心吧。还有不少人,他想。我住在一个好镇子里啊。
    
    他没法再跟这鱼说下去了,因为它给糟蹋得太厉害了。接着他想起了一件事。
    
    “半条鱼,”他说,“你原来是完整的。我很抱歉,我出海太远了,我把咱俩都毁了。不过今天我们联手,杀死了不少鲨鱼,还打伤了好多条。你杀死过多少鲨鱼啊,好鱼?你头上那只长嘴,可不是白长的啊。”
    
    他喜欢想象这条鱼,想象它如果还自由地游着,会怎样去对付一条鲨鱼。我应该砍下它这长嘴,拿来跟那些鲨鱼斗,他想。但是没有斧头,后来又弄丢了那把刀子。
    
    但是,如果把它砍下来,就能把它绑在桨把上,该是多好的武器啊。这样,我们就能一起跟它们斗啦。要是它们夜里来,你该怎么办?你又有什么办法?
    
    “跟它们斗!”他说,“我要跟它们斗到死。”
    
    然而,眼下的黑暗里,看不见天际的反光,也看不见灯火,只有风和那稳定地拉曳着的帆。他觉得,自己说不定已经死了。他合上双手,掌心相触。这双手没有死,张开手再握紧,就能感觉到痛楚。这就是活着的证明。他把背靠在船梢上,再次知道自己没有死——这次是他的肩膀告诉他的。
    
    我许过愿,如果逮住了这条鱼,要念多少遍祷文,他想。不过我现在太累了,没法念。我还是把麻袋拿来披在肩上。
    
    他躺在船梢掌着舵,注视着天空,等着天际的反光出现。我还有半条鱼,他想。也许我运气好,能把前半条带回去。我多少总该有点运气吧。不,他说。你出海太远了,把好运给冲掉啦。
    
    “别傻了,”他大声说道,“保持清醒,掌好舵。你还需要不少运气呢。”
    
    “要是有什么地方卖好运,我倒想买一些。”他说。我能拿什么来买呢?他问自己。能用一支弄丢了的鱼叉、一把折断的刀子和两只受了伤的手吗?
    
    “也许能。”他说,“你曾想拿在海上的八十四天来买它\footnote{〔海上的八十四天〕指捕到这条大鱼前连续八十四天没捕到鱼。}。老天也几乎把它卖给了你。”
    
    我不能胡思乱想,他想。运气这玩意儿,出现的方式千奇百怪,谁认得出啊?可是不管什么样的好运,我都要一点儿,要多少钱就给多少。但愿我能看到灯火的反光,他想。我的愿望太多了。但眼下的愿望就只有这个了。他竭力坐得舒服些,好掌舵,那疼痛再次教他知道自己还活着。
    
    大约夜里十点的时候,他看见了城市的灯火映在天际的反光。起初只能依稀看出,就像月亮升起前天上的微光。然后渐渐清晰了,就在此刻正被越来越大的风刮得波涛汹涌的海洋的另一边。他驶进了这反光的圈子,他想,要不了多久就能驶到湾流的边缘了。
    
    总算过去了,他想。它们也许还会再来袭击我。不过,一个人在黑夜里,没有武器,怎么对付它们呢?现在他身子僵硬、疼痛,在夜晚的寒气里,他的伤口和身上所有用力过度的地方都在发痛。我希望不必再斗了,他想。我真希望不必再斗了。
    
    但是到了午夜,他又搏斗了,而这一回他明白搏斗也是徒劳。它们是成群袭来的,朝那鱼直扑,他只看见它们的鳍在水面上划出的一道道线,还有鱼背的磷光。他朝它们的头打去,听到它们上下颚咬住的声音,在船底咬鱼让船摇晃的声音。他看不清目标,只能感觉到,听到,就不顾死活地挥棍打去。他感到什么东西攫住了棍子。棍子丢了。
    
    他把舵把从舵上猛地扭下,用它又打又砍,双手攥住了一次次朝下戳去。可是它们此刻都在船头边,一条接一条地窜来,成群地一起来,咬下一块块鱼肉,当它们转身再来时,鱼肉在水面下发亮。
    
    最后,有条鲨鱼朝鱼头冲来,他知道这下子可完了。他把舵把朝鲨鱼的脑袋抡去,打在它咬住厚实的鱼头的两颚上,那儿的肉咬不下来。他抡了一下,一下,又一下。他听见舵把“啪”的断了,就把断下的把手向鲨鱼扎去。他感到把手扎了进去,知道它很尖利,就再把它扎进去。鲨鱼松了嘴,一翻身就走了。这是前来的鲨鱼中的最后一条。它们再没有什么可吃的了。
    
    老头儿已经累得喘不过气了。他觉得嘴里有股怪味儿。这味儿带着甜甜的铜腥气,他一时害怕起来。但是这味儿并不太浓。
    
    他朝海里啐了一口:“吃了它吧,鲨。做个梦吧,梦见你杀了一个人。”
    
    他明白他终于给打败了,没法补救了,就回到船梢,发现舵把那锯齿形的断头还可以安在舵的狭槽里,让他用来掌舵。他把麻袋在肩头围好,让小船顺着航线驶去。航行得很轻松,他什么念头都没有,什么感觉也没有。他此刻超脱了这一切,只想着尽力把小船驶回他家乡的港口。夜里有些鲨鱼来咬死鱼的残骸,就像人从饭桌上捡面包屑吃一样。老头儿不去理睬它们,除了掌舵,他什么都不理睬。他只留意到船舷边没有什么沉重的东西,小船这时驶起来多么轻松,多么美妙。
    
    船还是好好的,他想。它是完好的,没受一点儿损伤,除了那个舵把。那是容易更换的。
    
    他感觉到已经在湾流中行驶,看得见海滨住宅区的灯光了。他知道此刻到了什么地方,回家是不在话下了。不管怎么样,风总是我们的朋友,他想。然后他加上一句:有时候是。还有大海,海里有我们的朋友,也有我们的敌人。还有床,他想。床是朋友。只有床,他想。床可是了不起的东西。吃了败仗,上床是很舒服的,他想。我从来不知道床竟然这么舒服。那么,是什么把你打败的,他想。“我没有败。”他大声说道,“只怪我出海太远了。”
    
    等他驶进小港,露台饭店的灯光早就熄灭了。他知道人们都上床睡觉了。海风一步步加强,此刻刮得很猛了。然而港湾里静悄悄的,他直驶到岩石下一小片卵石滩前。没人来帮他的忙,他只好尽自己的力量把船划得紧靠岸边。然后他跨出船来,把它系在一块岩石上。
    
    他拔下桅杆,把帆卷起,系住。然后他打起桅杆往岸上爬。这时他才明白自己疲乏到什么程度。他停了一会儿,回头一望,在街灯的反光中,看见那鱼的大尾巴直竖在小船的船梢后边。他看得清楚,它裸露的脊骨像一条白线,那黑黢黢的脑袋带着突出的长嘴;头尾之间,什么也没有。
    
    他再往上爬,到了顶上,摔倒在地,躺了一会儿,桅杆还是横在肩上。他设法爬起身来,可是太困难了,就扛着桅杆坐在那儿,望着大路。一只猫从路对面走过,去干它自己的事。老头儿盯着它,直到它消失在路口,然后只望着大路。
    
    最终,他放下桅杆,站起身来。他举起桅杆,扛在肩上,顺着大路走去。他不得不坐下歇了五次,才走到他的窝棚。
    
    进了窝棚,他把桅杆靠在墙上。他摸黑找到一个水瓶,喝了一口水。然后他在床上躺下了。他拉起毯子,盖住肩头,裹住背和腿,他脸朝下躺在报纸上,两臂伸得笔直,手掌向上。
    
\end{normalsize}



\chapter{白求恩传}

\begin{normalsize}
    
    他已经努力工作了不少年,取得了成就和名望。他自己主持一个科室。他不缺钱,可以追求自己想要做的事,设立自己的目标。他的脾气仍旧暴躁,可现在没有领导来和他发生冲突了。他可以自由地实行他的主张,平等地和所有人讨论问题。
    
    他常在医学刊物上发表文章,有时候讨论的不再是医学问题,而是更有争议的问题。他是美洲大陆胸外科医生集会上的风流人物、顶尖的胸外科专家。他在阿奇博尔德大夫门下工作仅仅四年,就被选为美国胸外科学会理事。随着名望增长,他被聘为圣阿加莎·德蒙\footnote{〔圣阿加莎·德蒙〕加拿大魁北克省的市镇,在蒙特利尔西北郊区,环境优美,空气质量高,当时是北美富人别墅区和肺病患者的顶级疗养院所在。}的疗养院及联邦政府和多个省卫生部的卫生顾问。他关注的重心逐渐从外科治疗的技术问题转向对肺结核的理论认识。
    
    从前他自己病危时,他不满于对根治疗法的顾虑,现在,他也对局部处理肺结核的办法表示不满。他认为,必须放弃肺结核只是一种肺部疾病的观念。它实际上是一种全身的疾病。结核杆菌对肺部的攻击源自环境对于人体整体的影响。“任何医治肺结核的办法,”他常说,“如果不把人看作一个整体,看作环境的种种压力之下的产物,就非失败不可。”
    
    1932年年中,他首次发表了《吁请作肺结核早期压缩》的文章。他同意将来或许有直接消灭结核杆菌的疗法,但他认为,在那样的疗法到来之前,就可以消灭肺结核。他继续主张应该尽早压缩病肺。“肺结核,”他说,“本身就有一种痊愈的倾向。如果和几乎没有痊愈倾向的心脏、肾脏、肝脏的各种慢性疾病相比,这种倾向使肺病在人类所患的长期疾病中成为独特的一类……作为一个民族,我们是能够消灭肺结核的,只要我们认为花足够的钱来做这件工作是值得的。我们需要一整套的措施,包括改进医学教育和公共教育,强制定期体检和爱克斯光\footnote{〔爱克斯光〕也称为X光、伦琴射线。指波长在0.01纳米和10纳米之间的电磁辐射。可以穿透生物体等物质,可用于探测人体内部情况。}检查,及早诊断、及早卧床静养、及早治疗,隔离并保护年轻人。”
    
    他愈发批判医学界的保守和惰性——这种对进步的抗拒不仅时时刻刻让无数人丧失生命,而且使国家不得不花大钱在本可以避免的重症治疗上。那些反对用“积极疗法”来治疗早期肺结核和显著期肺结核的人,不过是重复着本世纪初年“反对用手术割治阑尾炎”的意见。他多次呼吁采取预防措施,及早诊断、及早治疗,并充分利用外科方法,但反响甚微。
    
    在圣心医院,他的疗法效果令人鼓舞。许多病例,十年甚至五年前可能会认为是无法可治的,现在要么能用外科技术治愈,要么取得显著的疗效。但是,当他查看入院病人数和病愈率的数据时,他能隐约察觉出一种矛盾,这使他感到懊恼。“这事不对头,”他常常说,“病愈率和入院人数在同步增长。我们的外科疗法越进步,我们收的结核病人就越多。治疗这种疾病的科技水平达到高峰的时候,这个病的发病率也达到了最高峰。”魁北克省的肺结核病人占人口的比例,比加拿大其它省都要高。病人的数目超过各医院和疗养院所能容纳的人数。生活水平最低的省,肺结核病人比率最高。在城市贫民区和破产的内陆农村里,成千上万的人慢慢地死于肺结核,甚至还不知道自己得了这种病。
    
    为什么会这样?一方面,他继续实践肺结核的早期干预治疗,于此同时,他开始寻找这个令他不安的问题的答案。他发现了另一种正在全世界蔓延的疾病——一种比结核杆菌更加致命、比中世纪的霍乱传染起来更加迅速的疾病。
    
    作为医生,他一向知道肺病的根源是贫困。但是现在,不知什么缘故,贫困似乎四处蔓延,源源不绝地制造肺结核病人。他和其他医生每治好一个病人,就又有十个新的病人出现。他提出了问题,而问题的答案把他引上了一条新奇的、使他不安的道路。
    
    报纸上充斥着危机、萧条、破产、失业、救济、争论这样的字眼。1929年10月,纽约股票市场有过一点波动,然后就一发不可收拾\footnote{〔纽约股票市场……〕指1929年华尔街股灾,于10月24日爆发,引起28日后市场暴跌。这次股灾是1930年代的全球经济大萧条的先导。}。当然喽,没有必要过分惊慌——每个人都这么安慰别人——但是许多银行、工厂、矿山,却像纸牌搭的房子一样轻易垮掉,破产了。而那些高唱“慎重的乐观”的人士一个接一个地从华尔街高层办公室的窗口跳下去。
    
    白求恩当然早就听说了,各国的经济出了些小毛病,大家因此议论纷纷。但是经济学家们说,那只是周期性的波动,不久市场自己就会调整过来的。然而,五年已经过去了,什么也没有解决。这位专心医治肺结核的医生烦恼了起来。他一切美妙的理论虽然在手术室里成功实用,却在手术室外遭受着破坏。
    
    整整一百万人——加拿大总人口的十分之一——依靠政府的救济过活。白求恩查阅了关于失业者生活水平的报导。他发现失业家庭依靠政府津贴过活,成人每星期一元二角、小孩每星期八角\footnote{〔失业者的家庭……〕加拿大货币单位为元、角、分。一元等于十角,一角等于十分。当时加拿大贫困家庭每月必要支出普遍在四十元以上,平均水平为五十到八十元。}。这种情况使他大为愤慨。这简直太荒谬了,非改变不可!但是当他说起这些事的时候,同事们看着他,仿佛他是个疯子。难道他不知道,到处都是这样吗?他发现确实也是如此。总统和部长们高谈着“近在眼前的繁荣”,而失业、破产和恐惧深入到美洲大陆的每一个角落。世界上,到处都是混乱、都在崩溃:西班牙正处在普里姆·德·里维拉\footnote{〔普里姆·德·里维拉〕西班牙独裁者,1920至1930年统治西班牙。}的军事独裁统治下;德国,希特勒和国族社会主义\footnote{〔国族社会主义〕也称纳粹主义,主张法西斯主义、种族优越论,主张国家是中央集权的民族共同体,主张牺牲其他种族、民族的利益使德国繁荣。}夺取了政权;中国的蒋介石忙着用屠杀异己的方法来减少人口;日本,一个军国主义的集团做着统治全亚洲的迷梦。
    
    在白求恩看来,全世界仿佛患了一种集体的疯病。黑夜好像就是白天,而白天却永远不来。“勒紧你们的裤腰带!”大腹便便的内阁发言人发出这样的忠告。而当全世界的失业人数达到四千万的时候,他们就停止统计人数了。他们说这完全是一个生产过剩的问题,但老百姓两手空空,度日如年。
    
    他注意到全世界都有一种令人不安的矛盾。成千上万的人没有衣服穿,而美国却把地里的棉花翻耕入土;几千万人挨着饿,加拿大却把小麦烧掉;街边有人讨五分钱买一杯咖啡喝,而巴西却把咖啡往大海里倒;在蒙特利尔\footnote{〔蒙特利尔〕当时是加拿大最繁荣的新兴大都会,位于魁北克省。}的工人区,孩子们因为软骨病\footnote{〔软骨病〕因为缺乏营养和代谢问题导致骨质生长迟缓的疾病。}而成了罗圈腿,而美国南方的橙子却一卡车一卡车地被毁掉。加拿大医学会的主席也发出了警告:必须采取紧急措施给大多数无力付款的公民供药,并补助坚持免费治疗的医生,否则加拿大医界和加拿大人民将遭受严重的灾难。
    
    他已经是一个成功的外科医生,社交界的红人,黄金单身汉,各个晚会上的贵宾。趋炎附势的人、玩世不恭的人以及社会名流都围着他转,可一旦真正了解他,会发现他的思想完全不可捉摸。
    
    对白求恩的同事来说,人生的支柱是一个舒适的家、还清的贷款、安稳全面的保险以及不断增长的存款。在他们看来,他浪费得惊人。他挣很多钱,但花得干干净净。只要他中意的东西他就买,不管价钱多少。他尤其照顾蒙特利尔那些艰苦奋斗的青年艺术家,他们的作品如果中他的意,他一见就买。对于那些因为他花钱太随便而觉得不安的人们,他有一个现成的回答:“钱嘛,无非是交换的媒介而已。”
    
    碰到爱吹嘘医生这个职业的同行时,他喜欢说,外科医生和修水管的没什么区别,除了还不如修水管的熟练。如果和一群使他厌烦的人在一起,他会毫不客气、不找托辞就离开。
    
    他的公寓在河狸堂坡道\footnote{〔河狸堂坡道〕蒙特利尔市富人区的一条斜坡道。}上,布置得相当奢华——算是他早年在伦敦的放纵日子的回响。家具是他自己设计的。画儿包括他自己的和许多加拿大青年美术家的作品。每个房间里都堆满了书。艺术品、地毯、窗帘都是严格挑选过的:从色彩、图案到质地。但正如他的翻领毛衣是对他的客人的晚礼服的讽刺一样,他独有的一种幽默的特质也使他家的华美装潢变得轻松一些。他的各种文凭——医生们常常精心地挂出来作为自我宣传用的东西——都挂在浴室的墙上。
    
    只要是他喜欢的东西,他没有嫌贵的。然而朋友们都知道,所有他喜欢的东西,他没有不乐于和人共享的。他可观的藏书每一本里都有一张设计得很简单的贴头,上面写着:“这本书属于诺尔曼·白求恩和他的朋友们。”他对书如此,对家也一样。在贫困中努力的无名艺术家,在当时加拿大艺术界的惨况下只能梦想筹足钱去纽约或伦敦的演员,没有经济保障的聪明年青人——都在欢迎之列,大家也都看得出,他的殷勤好客有着真诚的情意。有时他们来的时候他正在卧室里换衣服。于是他们看他的书,听他的唱片,然后他出来,愉快地说:“我现在出去看一个病人。冰箱里有吃的,浴室里有酒。我不在的时候你们好好玩吧。”
    
    在二十世纪那黯淡的第三十五个年头,春天像往常一样来到了蒙特利尔。医院里一切顺利;白求恩的方法和技术取得了进展——但是新的疑虑开始在他心里活动起来。
    
    对他的助手和忠实朋友乔治·德塞大夫,他吐露了一些使他感到不安的事。他常常抱怨说,肺结核是可以从社会上清除的,但实际上病人的数量却在增加。在手术台上,外科医生治疗的只是肺结核在个人身上产生的严重后果,而不是它普遍存在的原因。
    
    而且谁知道有多少万人患着肺结核而从不去医院——因为他们没钱,或是没知识,不知道自己得的是什么病?有时他觉得他还不如把手术刀扔掉,站在街头大声警告过路的人。那样或许比单单做外科手术更有用。
    
    他很少和弗朗西丝\footnote{〔弗朗西丝〕弗朗西丝·坎贝尔,白求恩的妻子。}见面;见着的时候,就在她面前大吵大骂,而她安静地坐在那儿听着,明亮的眼睛里充满了耐心和同情。“医学!”他往往嚷道,“你不知道有多大一部分走进了死胡同。将来有一天,教科书上要把我们说成前医学时代的人。”很久以前,在底特律,他就抱怨说太多的医生只关心“表面”的医药。他脑子里充斥的种种念头,在弗朗西丝听起来都很奇怪,可是令人神往。
    
    “医生,”他说,“应该献身于维护人民健康的事业。有多少医生是这么行医的?是不是总是他们的错呢?不是,是整个社会错了。每个城市每条街都有自来水、下水道、公共厕所、电气以及其他设备,我们认为这是理所当然的事。可是我们会把医药包括在内吗?并不会。为什么不呢?因为人没有享受健康生活的权利。我们要花钱买这个特权,像在街角的杂货铺里买一罐糖果一样。你一定得出大价钱才行。我们大家挂牌行医,心满意足,得意洋洋——像裁缝铺一样。我们矫正一条胳臂,一条腿,就像裁缝补一件旧上衣一样。说老实话,我们并不是在行医——我们是在做生意。我告诉你我们需要什么:我们需要一个新的医学概念,一个新的保护全民健康的概念,一个新的医生职责的概念。”
    
    有一天他突然去看弗朗西丝。他好像兴奋极了。“这工作我干不下去了。”他说。
    
    “说真的,白,”弗朗西丝说,“你又在闹什么?”
    
    他不高兴了一会儿,接着就把她的问题撇开了。他以前想过,如果把自己训练成一个医术高超的医生,就可以在消灭肺结核方面尽一些力量。但这办法行不通。一定得把医药直接送到人民那儿去。可是怎么做呢?“我们到人民中间去!我们到人民中间去!取消挂牌行医;改变整个医疗制度。你看看窗外——整街整道的房屋。那才是医生必须去的地方。深入每座城市,每个农庄,每间屋子。挨家挨户,我们把医药直接送到每个人那儿去。我们不能呆在诊所里,等病人送上门来,诊断一次收一次钱;我们必须在他得病之前就去找他,教他怎么保持健康。如果他已经病了,我们就迅速止住他的病。挨家挨户,大街小巷,大城小镇……”
    
    “但是你说的‘我们’,”弗朗西丝问,“是指谁?”
    
    “我自己,”他说,“以及其他愿意跟我走的人,其他和我一样相信医生的责任是采取行动,是到疾病的来源那儿去的人。”他越发兴奋了。他一心要说服她。
    
    “但是你到哪儿去找这种人呢?”
    
    “教会是在哪儿找到传教士的呢?”他反问道,“如果有人能够响应教会的号召,放弃一切世俗的财物,而去过修道院的俭朴生活,一定也有医生愿意抛开个人利益,去当为人民健康服务的传教士。医生们将联合起来,组成一个医疗工作者的团体。我们要到贫民区去,到需求最迫切的地方去。”弗朗西丝可知道,魁北克省农业区的疾病比全国任何其他地区都要多?她可知道,离蒙特利尔二十英里\footnote{〔英里〕英制长度单位。一英里约为1.609公里。},在拉辛,每三个婴儿里就有一个夭折?她可知道,在蒙特利尔和魁北克城\footnote{〔魁北克城〕加拿大东部的港口城市,魁北克省的省会。}出生时死去的婴儿几乎比世界上任何地方都多,除去孟买和马德拉斯那样落后的城市\footnote{〔孟买和马德拉斯……〕都是印度的城市。}?
    
    “但是,”她打断了他的话,“这一切所需要的钱从哪儿来呢?姑且假设你说服了别人跟你走吧——你们靠什么过活呢?”
    
    “呃,那个——钱……”他不耐烦地一挥手,把钱的问题撇开了。“我们自己什么也不需要。我们靠人民给的钱过活……五分也好,一角也好,一元也好。那些就足够了。人民出得起多少钱,就用多少钱养活我们。药品、器材、设备——这些全是技术性的细节。必要的话,我们就让政府惭愧,逼他们养我们。”
    
    他踱来踱去,滔滔不绝地讲着他的计划,然后在她面前停下来,抓住她的手。“你认为怎么样?”
    
    “我不知道,”她慢吞吞地说,“听你这么讲,我觉得我从来没听过比这更美妙的事——或者更不切实际的事。”
    
\end{normalsize}


\newpage

\textbf{注释}:

\vspace{-1em}

\begin{itemize}
    \setlength\itemsep{-0.2em}
    \item 〔大腹便便〕肚子肥大凸出的样子,形容人体型臃肿,含贬义。便便:肥大的样子。
    \item 〔趋炎附势〕奉承依附有权势的人。炎:热,比喻权势。
    \item 〔萧条〕寂寥冷清,草木凋零。比喻经济活动冷清,民不聊生。
    \item 〔玩世不恭〕以轻藐不拘礼法的游戏态度处世待人。
    \item 〔滔滔不绝〕像流水一样不断绝。滔滔:形容水势浩大,奔流不息。
\end{itemize}

\chapter{车库里的龙}

\begin{normalsize}
    
    “一条会喷火的龙住在我家的车库里。”
    
    假设我一本正经地对你说出这句话,你一定会想亲自去看一眼。有关龙的故事流传了成百上千年,但是从来都没有证据证明龙真的存在,正所谓“机不可失,时不再来”。
    
    “让我看看。”你说道。于是我把你领进我的车库里。你看到里面有架梯子,几个空的涂料桶,还有一辆老旧的三轮车,然而看不到龙。
    
    “龙在哪?”你问。
    
    “她就在这呢。”我会这么回答,大概指一指前面,“我忘了告诉你,她是条会隐身的龙。”
    
    你建议在地上撒些面粉,这样我们就能看到龙的脚印了。
    
    “建议是不错,”我说,“但是这条龙悬浮在空中。”
    
    然后你建议用红外线传感器\footnote{〔红外线传感器〕通过感应物体产生的红外线探测物体温度的仪器。}探测我所说的看不见的火焰。
    
    “好想法。”我说,“但这条龙的火焰是常温的,与四周的空气没有区别。”
    
    你打算用喷漆\footnote{〔喷漆〕将涂料喷涂在物体表面,形成黏附薄层。漆:泛指涂料。}让龙现形。
    
    “对于看不见的物质,这确实是个办法。”我又说,“可惜她是条非物质的龙,漆不会留在她身上的。”就这样一直下去,你所有使用物理手段探测的建议,我都能用一个特别的解释反驳你,告诉你为什么你的建议行不通。
    
    那么,一只看不见的、非物质的、浮在空中、能吐出常温火焰的龙,和根本就没有龙的区别究竟在哪里呢?如果我的说法无法被驳倒,无法构想出合理的实验来证实或证伪,那么说我的龙存在有什么意义呢?我的说法无法被推翻,并不表示我的说法是真的。不管一个主张多么令人振奋,多么让人惊叹,如果没有检验、推翻它的可能,那么这个主张实际上没有任何价值。我之前说的,归根到底只是要你无根据地相信我说的话,我的一面之词。即便我说得天花乱坠,你了解到的,实际上仅仅是“这人的脑子里有这么个古怪的想法”。
    
    你也许会想,如果没有确实的检验结果,究竟是什么说服了我?你会觉得我是没睡醒,把梦境当成了现实,或者产生了幻觉。但即使这样,我为什么要这么严肃地说这件事呢?也许我脑子有问题吧?至少,我低估了我自己犯错的能力。
    
    想象一下,尽管你提议的实验没一个能得出一丁点儿结论,但是你这人思想特别开放,喜欢接纳新的观点,所以你没有立即否定我这个“车库里有条会喷火的龙”的想法。尽管你至今看到的事实都和我的说法严重冲突,但你仅仅把我的说法暂时搁置。如果有新的证据出现,你会欣然同意检验新的证据,看看是否能说服你。这个时候,如果我还因为你下的判断是“尚未证实”而对你大发脾气,批评你古板守旧,缺乏接受新事物的能力,那么显然对你不公平。
    
    再想象一下故事发展的另一个走向。龙确实是隐身的,但是在你的提议下,我们在地面上撒了面粉,而真的有巨大的脚印在铺满面粉的地面上出现了。你的红外线传感器有强烈反应,数值惊人。向空气喷漆后,你面前突然显出来一大块染色的鳞片。不管你之前有多么怀疑这条龙的存在,你现在总得承认,车库里有东西,并且看上去和“一条会喷火的隐身的龙”的描述挺吻合的。
    
    让我们再换一个场景:假设不仅仅是我,假设你的许多熟人,即便他们互相之间不认识,都跟你说,他们家的车库里住着条龙。然而,每家的龙的证据都含糊不清,无法令人满意。我们所有人都承认,我们也不懂,自己为什么会有这么怪异的、毫无事实根据支撑的信仰。我们都没有疯。让我们来想想,如果世界上到处都有隐形的龙藏在车库里,而我们人类刚刚发现这个事实,这意味着什么?老实说,我当然不希望这是真的。不过,也许欧洲和中国那些关于龙的神话,并不仅仅是神话而已。
    
    似乎有报道说,有些人撒面粉的时候,观察到了巨大的脚印。可怀疑这事的人去看的时候,又找不到了。而且经过仔细观察,那些脚印的照片似乎是伪造的。
    
    有痴迷车库巨龙的狂热爱好者站出来,声称他的手指被龙吐出的烈焰灼伤了。但仔细想想,能把手指灼伤的,可不止巨龙喷的火焰呢。
    
    这些所谓的证据,不管巨龙爱好者认为有多么重要,还远不足以让我们信服。说来说去,只要时刻对可能出现的新证据抱着开放的态度,那末,试着否定“车库里有巨龙”的假说,反而更合理。同时我们可以好好想想,是什么让这么多看起来精神正常、头脑清醒的人看到了同样的奇异幻象。
    
\end{normalsize}


\newpage

\textbf{注释}:

\vspace{-1em}

\begin{itemize}
    \setlength\itemsep{-0.2em}
    \item 〔一本正经〕本义是符合道德规范的经典。多指显出很规矩、很郑重的举止或外表。有时含有反讽的意思。
    \item 〔场景〕电影、戏剧中的场面。泛指情景、情况。
    \item 〔常温〕泛指日常生活环境的温度。
    \item 〔搁置〕把事物放在一边不管、不考虑。
    \item 〔天花乱坠〕本为佛教用语,形容佛祖讲经时感动天神,有各种香花从天而降。引申指话讲得动听诱人而不实在、虚妄。
    \item 〔古板〕固执守旧,不肯变通、接受新事物。
    \item 〔一面之词〕辩论、争执中双方或多方中的一方所说的。
    \item 〔证据〕用来确认、表明某种说法、观点真伪的凭据。
    \item 〔实验〕用实际结果检验某种说法、观点的活动。
    \item 〔证明〕用确实的证据和推理说明。
    \item 〔证实〕用证据和推理肯定某种说法或观点,证明它是真实的、正确的。
    \item 〔证伪〕用证据和推理否定某种说法或观点,证明它是虚假的、错误的。
\end{itemize}

\chapter{聪明人和傻子和奴才}

\begin{normalsize}
    
    奴才总不过是寻人诉苦。只要这样,也只能这样。有一日,他遇到一个聪明人。
    
    “先生!”他悲哀地说,眼泪联成一线,就从眼角上直流下来。“你知道的。我所过的简直不是人的生活。吃的是一天未必有一餐,这一餐又不过是高粱皮,连猪狗都不要吃的,尚且只有一小碗……”
    
    “这实在令人同情。”聪明人也惨然说。
    
    “可不是么!”他高兴了。“可是做工是昼夜无休息的:清早担水晚烧饭,上午跑街夜磨面,晴洗衣裳雨张伞,冬烧汽炉夏打扇。半夜要煨银耳,侍候主人要钱;头钱\footnote{〔头钱〕旧社会里提供赌博场所的人向参与赌博者抽取一定数额的钱,也称“抽头”。}从来没分,有时还挨皮鞭……”
    
    “唉唉……。”聪明人叹息着,眼圈有些发红,似乎要下泪。
    
    “先生!我这样是敷衍不下去的。我总得另外想法子。可是什么法子呢?……”
    
    “我想,你总会好起来……”
    
    “是么?但愿如此。可是我对先生诉了冤苦,又得你的同情和慰安,已经舒坦得不少了。可见天理没有灭绝……”
    
    但是,不几日,他又不平起来了,仍然寻人去诉苦。“先生!”他流着眼泪说,“你知道的。我住的简直比猪窠还不如。主人并不将我当人;他对他的叭儿狗\footnote{〔叭儿狗〕即“哈巴狗”,宠物狗的一个品种,体小,腿短,毛长。}还要好到几万倍……”
    
    “混帐!”那人大叫起来,使他吃惊了。那人是一个傻子。“先生,我住的只是一间破小屋,又湿,又阴,满是臭虫,睡下去就咬得真可以。秽气冲着鼻子,四面又没有一个窗……”
    
    “你不会要你的主人开一个窗的么?”
    
    “这怎么行?……”
    
    “那么,你带我去看去!”
    
    傻子跟奴才到他屋外,动手就砸那泥墙。
    
    “先生!你干什么?”他大惊地说。
    
    “我给你打开一个窗洞来。”
    
    “这不行!主人要骂的!”
    
    “管他呢!”他仍然砸。
    
    “人来呀!强盗在毁咱们的屋子了!快来呀!迟一点可要打出窟窿来了!……”他哭嚷着,在地上团团地打滚。一群奴才都出来了,将傻子赶走。
    
    听到了喊声,慢慢地最后出来的是主人。
    
    “有强盗要来毁咱们的屋子,我首先叫喊起来,大家一同把他赶走了。”他恭敬而得胜地说。
    
    “你不错。”主人这样夸奖他。
    
    这一天就来了许多慰问的人,聪明人也在内。
    
    “先生。这回因为我有功,主人夸奖了我了。你先前说我总会好起来;实在是有先见之明……”他大有希望似的高兴地说。
    
    “可不是么……”聪明人也代为高兴似的回答他。
    
    \hfill 一九二五年十二月二十六日
    
\end{normalsize}


\newpage

\textbf{注释}:

\vspace{-1em}

\begin{itemize}
    \setlength\itemsep{-0.2em}
    \item 〔煨〕用微火慢慢地煮。
\end{itemize}

\chapter{在《人民报》创刊周年纪念会上的演说}

\begin{normalsize}
    
    \begin{center}1856年4月14日,伦敦\end{center}
    
    所谓的1848年革命,只是几个微不足道的事件,欧洲社会干硬外壳上的几处小裂口、小缝隙。然而,它们暴露了一个无底深渊。那貌似坚固的外表之下,现出了一片汪洋大海,只要它动荡起来,就能把这磐石般的大陆撞得粉碎。噪聒又懵懂地,它们揭露了19世纪的秘密,19世纪革命的秘密——无产阶级的解放。
    
    1848年那场革命,并不是什么新发明。相比巴尔贝斯、拉斯拜尔和布朗基\footnote{〔巴尔贝斯、拉斯拜尔和布朗基〕三人都是19世纪的社会革命家,积极参与、领导法国工人运动。},蒸汽、电力和自动纺纱机才是危险得多的革命家。然而,即便你们知道,这个世界每个人身上都承受着两万磅\footnote{〔磅〕英制重量单位,一磅约为453克。}的大气压力,你们能感觉到吗?同样,在1848年之前,欧洲社会也没有感觉到那从四面八方包裹着它、压抑着它的革命气氛。19世纪发生了一件大事,大到可以作为这个世纪的标志,任何人都不敢否认。
    
    一方面,19世纪诞生了人类历史上任何时代都无法想象的科技和工业伟力;另一方面,19世纪显露出的衰颓征兆,远远超过史书中罗马帝国衰亡时期的可怕情景。我们这个时代的每一种事物,都好像在孕育自己的反面。机器有着缩短工时、增加产能的神奇力量,却引发了过劳和饥荒。本该是财富的新源泉,却仿佛受了奇怪的诅咒,源源不断地制造贫困。每一次工艺的进步,仿佛都要用道德的败坏来交换。人类对自然的驾驭每进一步,受到的奴役就更深一层——不论是作为他人还是自身卑劣天性的奴隶。就连科学的纯洁光辉,在无知愚昧铺成的黑暗背景中也显得微不足道。看起来,我们的一切发明进步为物质力量\footnote{〔物质力量〕有能力改变事物的运动状态的力量。}赋予了灵性,却把人消磨成了物质力量。
    
    现代的工业与科技与现代的贫困和崩坏,当前时代的生产力和当前时代的社会关系,鲜明对立着,格格不入。这个事实就像洪流袭来,近在眼前,毋庸置疑,无可抵挡。有的人为此痛苦哀号;有的人宁愿抛弃现代科技来摆脱现代的社会冲突;还有人设想,工业上如此巨大的进步,自然要以政治上同样巨大的倒退来补全。对我们来说,我们自然不会认不出这种种矛盾中隐约能见的狡诈身影。我们清楚地认识到,要使社会的新生力量发挥好作用,就只能由新生的人来掌握它们,这些新生的人就是工人。工人也同机器本身一样,是现代的产物。
    
    在那些使中产阶级\footnote{〔中产阶级〕在19世纪上半叶,中产阶级被马克思用来指尚未完全夺取旧贵族阶级的权力,社会地位处于旧贵族转型的大资产阶级和无产阶级之间的新生资产阶级,当时也称“布尔乔亚”。}、贵族和蹩脚的衰退预言家惊慌失措的征兆中,我们认出了我们勇敢的朋友——就如好人儿罗宾\footnote{〔好人儿罗宾〕英国传说故事中的淘气鬼角色。这里指被统治阶级诋毁的无产阶级革命。},那个刨土一流的老鼹鼠一样,光荣的开拓者——劳工革命。英国工人是现代工业的长子。因此,在参与现代工业革命所孕育的社会革命时,自然也不落人后。因为这革命是为了它们所在的阶级在全世界的解放,这革命同资本统治和工薪奴隶制同样必然到来。我知道英国工人阶级从上世纪中叶以来进行了多么英勇的斗争。中产阶级的历史学家把这些斗争掩盖起来,隐瞒不说,因此其英勇光辉少为人知。在中世纪的德国,为了报复统治阶级的罪行,曾有过一种叫作“维末法庭”的秘密法庭。如果某一所房子上画了一个红十字,大家就知道,这所房子的主人逃不过“维末”的判决了。而现在,神秘的红十字已经画在了欧洲所有的房子上。
    
    历史本身就是审判官,而无产阶级就是执刑者。
    
\end{normalsize}


\newpage

\textbf{注释}:

\vspace{-1em}

\begin{itemize}
    \setlength\itemsep{-0.2em}
    \item 〔消磨〕逐渐消耗磨灭。
    \item 〔噪聒〕嘈杂刺耳。
    \item 〔懵懂〕头脑不清楚,无法明辨事物的状态。
    \item 〔毋庸置疑〕指事实非常明显或理由非常充足,没有必要持怀疑态度。毋庸:无须,不必。
\end{itemize}

\chapter{记念刘和珍君}

\begin{normalsize}
    
    \begin{center}\textbf{一}\end{center}
    
    中华民国十五年三月二十五日,就是国立北京女子师范大学为十八日在段祺瑞执政府\footnote{〔段祺瑞执政府〕1924年第二次直奉战争,直系军阀失败,奉系军阀推举段祺瑞为北洋政府临时执政。段祺瑞:北洋军阀皖系首领,数次把持北洋军阀中央政权,1926年4月被冯玉祥驱逐下台。}前遇害的刘和珍杨德群两君开追悼会的那一天,我独在礼堂外徘徊,遇见程君,前来问我道,“先生可曾为刘和珍写了一点什么没有?”我说“没有”。她就正告\footnote{〔正告〕郑重地说。}我,“先生还是写一点罢;刘和珍生前就很爱看先生的文章。”
    
    这是我知道的,凡我所编辑的期刊,大概是因为往往有始无终之故罢,销行一向就甚为寥落,然而在这样的生活艰难中,毅然预定了《莽原》\footnote{〔《莽原》〕鲁迅任编辑的一个文艺期刊。}全年的就有她。我也早觉得有写一点东西的必要了,这虽然于死者毫不相干,但在生者,却大抵只能如此而已。倘使我能够相信真有所谓“在天之灵”,那自然可以得到更大的安慰,——但是,现在,却只能如此而已。
    
    可是我实在无话可说。我只觉得所住的并非人间。四十多个青年的血,洋溢在我的周围,使我艰于呼吸视听,那里还能有什么言语?长歌当哭\footnote{〔长歌当哭〕放声歌诵代替哭泣。这里引申为写文章抒发悲愤。},是必须在痛定之后的。而此后几个所谓学者文人的阴险的论调\footnote{〔几个……〕指陈西滢等人。陈西滢在3月27日出版的《现代评论》上发表评论“三·一八”惨案的《闲话》,污蔑爱国学生“莫名其妙”“没有审判力”,因而被人盲目地引入“死地”,并把杀人责任推到他所说的“民众领袖”身上,说他们“犯了故意引人去死地”的嫌疑。鲁迅在回应的《死地》一文中:“但各种评论中,我觉得有一些可以比刀枪更惊心动魄者在。这就是几个论客,以为学生们本不应当自蹈死地,前去送死的。”},尤使我觉得悲哀。我已经出离愤怒了。我将深味这非人间的浓黑的悲凉;以我的最大哀痛显示于非人间,使它们快意于我的苦痛,就将这作为后死者的菲薄\footnote{〔菲薄〕这里指“微薄”。}的祭品,奉献于逝者的灵前。
    
    \begin{center}\textbf{二}\end{center}
    
    真的猛士,敢于直面惨淡的人生,敢于正视淋漓的鲜血。这是怎样的哀痛者和幸福者?然而造化又常常为庸人设计,以时间的流驶,来洗涤旧迹,仅使留下淡红的血色和微漠的悲哀。在这淡红的血色和微漠的悲哀中,又给人暂得偷生,维持着这似人非人的世界。我不知道这样的世界何时是一个尽头!
    
    我们还在这样的世上活着;我也早觉得有写一点东西的必要了。离三月十八日也已有两星期,忘却的救主快要降临了罢\footnote{〔忘却的……〕指有些人寄希望于大众忘却这件事,这里是反讽的说法。救主:基督教用语,指约定会来拯救众生的神。这里指“大众忘却这件事”是某些人的救主。},我正有写一点东西的必要了。
    
    \begin{center}\textbf{三}\end{center}
    
    在四十余被害的青年之中,刘和珍君是我的学生。学生云者\footnote{〔学生云者〕“学生”这个称呼。云:说,表示“这样”“这种说法”,指代前语。},我向来这样想,这样说,现在却觉得有些踌躇了,我应该对她奉献我的悲哀与尊敬。她不是“苟活到现在的我”的学生,是为了中国而死的中国的青年。
    
    她的姓名第一次为我所见,是在去年夏初杨荫榆\footnote{〔杨荫榆〕1924年起任国立北京女子师范大学校长,反对学生参与社会运动,被鲁迅认为阻碍进步。}女士做女子师范大学校长,开除校中六个学生自治会职员的时候。其中的一个就是她;但是我不认识。直到后来,也许已经是刘百昭率领男女武将,强拖出校之后了,才有人指着一个学生告诉我,说:这就是刘和珍。其时我才能将姓名和实体联合起来,心中却暗自诧异。我平素想,能够不为势利所屈,反抗一广有羽翼的校长的学生,无论如何,总该是有些桀骜锋利的,但她却常常微笑着,态度很温和。待到偏安于宗帽胡同,赁屋授课之后\footnote{〔偏安于……〕指女师大学生被赶出校后,只能在宗帽胡同租房作为临时校舍。鲁迅和一些进步教师曾去义务授课,表示支持。},她才始来听我的讲义,于是见面的回数就较多了,也还是始终微笑着,态度很温和。待到学校恢复旧观\footnote{〔学校恢复旧观〕指女师大复校。},往日的教职员以为责任已尽,准备陆续引退\footnote{〔引退〕辞去官职。这里指辞职。}的时候,我才见她虑及母校前途,黯然至于泣下。此后似乎就不相见。总之,在我的记忆上,那一次就是永别了。
    
    \begin{center}\textbf{四}\end{center}
    
    我在十八日早晨,才知道上午有群众向执政府请愿的事;下午便得到噩耗,说卫队居然开枪,死伤至数百人,而刘和珍君即在遇害者之列。但我对于这些传说,竟至于颇为怀疑。我向来是不惮以最坏的恶意,来推测中国人的,然而我还不料,也不信竟会下劣凶残到这地步。况且始终微笑着的和蔼的刘和珍君,更何至于无端在府门前喋血呢?
    
    然而即日证明是事实了,作证的便是她自己的尸骸。还有一具,是杨德群君的。而且又证明着这不但是杀害,简直是虐杀,因为身体上还有棍棒的伤痕。
    
    但段政府就有令,说她们是“暴徒”!
    
    但接着就有流言,说她们是受人利用的。
    
    惨象,已使我目不忍视了;流言,尤使我耳不忍闻。我还有什么话可说呢?我懂得衰亡民族之所以默无声息的缘由了。沉默呵,沉默呵!不在沉默中爆发,就在沉默中灭亡。
    
    \begin{center}\textbf{五}\end{center}
    
    但是,我还有要说的话。
    
    我没有亲见;听说,她,刘和珍君,那时是欣然前往的。自然,请愿而已,稍有人心者,谁也不会料到有这样的罗网。但竟在执政府前中弹了,从背部入,斜穿心肺,已是致命的创伤,只是没有便死。同去的张静淑君想扶起她,中了四弹,其一是手枪,立仆\footnote{〔立仆〕立刻倒下。仆:卧倒。};同去的杨德群君又想去扶起她,也被击,弹从左肩入,穿胸偏右出,也立仆。但她还能坐起来,一个兵在她头部及胸部猛击两棍,于是死掉了。
    
    始终微笑的和蔼的刘和珍君确是死掉了,这是真的,有她自己的尸骸为证;沉勇而友爱的杨德群君也死掉了,有她自己的尸骸为证;只有一样沉勇而友爱的张静淑君还在医院里呻吟。当三个女子从容地转辗于文明人所发明的枪弹的攒射中的时候,这是怎样的一个惊心动魄的伟大呵!中国军人的屠戮妇婴的伟绩,八国联军的惩创学生的武功,不幸全被这几缕血痕抹杀了。
    
    但是中外的杀人者却居然昂起头来,不知道个个脸上有着血污……
    
    \begin{center}\textbf{六}\end{center}
    
    时间永是流驶,街市依旧太平,有限的几个生命,在中国是不算什么的,至多,不过供无恶意的闲人以饭后的谈资,或者给有恶意的闲人作“流言”的种子。至于此外的深的意义,我总觉得很寥寥,因为这实在不过是徒手的请愿。人类的血战前行的历史,正如煤的形成,当时用大量的木材,结果却只是一小块,但请愿是不在其中的,更何况是徒手。
    
    然而既然有了血痕了,当然不觉要扩大。至少,也当浸渍了亲族;师友,爱人的心,纵使时光流驶,洗成绯红,也会在微漠的悲哀中永存微笑的和蔼的旧影。陶潜\footnote{〔陶潜〕陶渊明,东晋末年著名诗人。}说过,“亲戚或余悲,他人亦已歌,死去何所道,托体同山阿。”倘能如此,这也就够了。
    
    \begin{center}\textbf{七}\end{center}
    
    我已经说过:我向来是不惮以最坏的恶意来推测中国人的。但这回却很有几点出于我的意外。一是当局者竟会这样地凶残,一是流言家竟至如此之下劣,一是中国的女性临难竟能如是之从容。
    
    我目睹中国女子的办事,是始于去年的,虽然是少数,但看那干练坚决,百折不回的气概,曾经屡次为之感叹。至于这一回在弹雨中互相救助,虽殒身不恤的事实,则更足为中国女子的勇毅,虽遭阴谋秘计,压抑至数千年,而终于没有消亡的明证了。倘要寻求这一次死伤者对于将来的意义,意义就在此罢。
    
    苟活者在淡红的血色中,会依稀看见微茫的希望;真的猛士,将更奋然而前行。
    
    呜呼,我说不出话,但以此记念刘和珍君!
    
    \hfill 四月一日
    
\end{normalsize}


\newpage

\textbf{注释}:

\vspace{-1em}

\begin{itemize}
    \setlength\itemsep{-0.2em}
    \item 〔寥落〕稀少,冷落。
    \item 〔销行〕销售发行。
    \item 〔深味〕深深地体会。
    \item 〔微漠〕依稀、淡薄。
    \item 〔出离〕超出
    \item 〔桀骜〕倔强凶暴。
    \item 〔造化〕大自然。
    \item 〔羽翼〕鸟的翅膀和羽毛,指依附随从的势力。
    \item 〔黯然〕心情低落沮丧的样子。
    \item 〔喋血〕血流满地,本为“蹀血”。蹀:顿足、踏。
    \item 〔沉勇〕沉着勇敢。
    \item 〔惩创〕惩罚,惩治。
    \item 〔赁〕租。
    \item 〔浸渍〕浸润,渗透。
    \item 〔殒身不恤〕牺牲生命也不顾惜。陨:死亡。恤:顾虑,在乎。
\end{itemize}

\chapter{《呐喊》自序}

\begin{normalsize}
    
    我在年青时候也曾经做过许多梦,后来大半忘却了,但自己也并不以为可惜。所谓回忆者,虽说可以使人欢欣,有时也不免使人寂寞,使精神的丝缕还牵着己逝的寂寞的时光,又有什么意味呢,而我偏苦于不能全忘却,这不能全忘的一部分,到现在便成了《呐喊》的来由。
    
    我有四年多,曾经常常,——几乎是每天,出入于质铺\footnote{〔质铺〕当铺。质:典当,抵押。}和药店里,年纪可是忘却了,总之是药店的柜台正和我一样高,质铺的是比我高一倍,我从一倍高的柜台外送上衣服或首饰去,在侮蔑里接了钱,再到一样高的柜台上给我久病的父亲去买药。回家之后,又须忙别的事了,因为开方的医生是最有名的,以此所用的药引\footnote{〔药引〕中医认为可以引导药力到达治疗部位的药。主要用途有激发增强疗效、解毒、保护胃肠道等等。}也奇特:冬天的芦根,经霜三年的甘蔗,蟋蟀要原对的,结子的平地木,……多不是容易办到的东西。然而我的父亲终于日重一日的亡故了。
    
    有谁从小康人家而坠入困顿的么,我以为在这途路\footnote{〔途路〕路途。}中,大概可以看见世人的真面目;我要到N进K学堂去了\footnote{〔我要到……〕N指南京,K学堂指江南水师学堂。作者于1898年到南京江南水师学堂肄业,第二年改入江南陆师学堂附设的矿务铁路学堂。},仿佛是想走异路,逃异地,去寻求别样的人们。我的母亲没有法,办了八元的川资,说是由我的自便;然而伊哭了,这正是情理中的事,因为那时读书应试是正路,所谓学洋务\footnote{〔学洋务〕在洋务派办的学堂里学习西方科学知识和军事技术。},社会上便以为是一种走投无路的人,只得将灵魂卖给鬼子,要加倍的奚落而且排斥的,而况伊又看不见自己的儿子了。然而我也顾不得这些事,终于到N去进了K学堂了,在这学堂里,我才知道世上还有所谓格致,算学,地理,历史,绘图和体操。生理学并不教,但我们却看到些木版的《全体新论》\footnote{〔《全体新论》〕英国合信写的生理学著作,1851年翻译为中文出版。}和《化学卫生论》\footnote{〔《化学卫生论》〕英国甄司腾写的营养学著作,1879年翻译为中文出版。}之类了。我还记得先前的医生的议论和方药\footnote{〔方药〕指中医开的药方和药物。},和现在所知道的比较起来,便渐渐的悟得中医不过是一种有意的或无意的骗子\footnote{〔中医……〕指当时中医行当充斥着行骗或盲信的成分。},同时又很起了对于被骗的病人和他的家族的同情;而且从译出的历史上,又知道了日本维新是大半发端于西方医学的事实。
    
    因为这些幼稚的知识,后来便使我的学籍列在日本一个乡间的医学专门学校里了\footnote{〔乡间的……〕指日本仙台的医学专门学校。鲁迅1904年到1906年在此学习医学。}。我的梦很美满,预备卒业回来,救治象我父亲似的被误的病人的疾苦,战争时候便去当军医,一面又促进了国人对于维新的信仰。我已不知道教授微生物学的方法,现在又有了怎样的进步了,总之那时是用了电影,来显示微生物的形状的,因此有时讲义的一段落已完,而时间还没有到,教师便映些风景或时事的画片给学生看,以用去这多余的光阴。其时正当日俄战争的时候,关于战事的画片自然也就比较的多了,我在这一个讲堂中,便须常常随喜\footnote{〔随喜〕佛教用语,指跟随别人参拜佛殿或做善事。这里指跟着别人一起活动。}我那同学们的拍手和喝采。有一回,我竟在画片上忽然会见我久违的许多中国人了,一个绑在中间,许多站在左右,一样是强壮的体格,而显出麻木的神情。据解说,则绑着的是替俄国做了军事上的侦探,正要被日军砍下头颅来示众,而围着的便是来赏鉴\footnote{〔赏鉴〕欣赏鉴别。这里有反讽的意味。}这示众的盛举的人们。
    
    这一学年没有完毕,我已经到了东京了,因为从那一回以后,我便觉得医学并非一件紧要事,凡是愚弱的国民,即使体格如何健全,如何茁壮,也只能做毫无意义的示众的材料和看客,病死多少是不必以为不幸的。所以我们的第一要著,是在改变他们的精神,而善于改变精神的是,我那时以为当然要推文艺,于是想提倡文艺运动了。在东京的留学生很有学法政理化以至警察工业的,但没有人治\footnote{〔治〕研究。}文学和美术;可是在冷淡的空气中,也幸而寻到几个同志了,此外又邀集了必须的几个人,商量之后,第一步当然是出杂志,名目是取“新的生命”的意思,因为我们那时大抵带些复古的倾向,所以只谓之《新生》。
    
    《新生》的出版之期接近了,但最先就隐去了若干担当文字的人,接着又逃走了资本,结果只剩下不名一钱的三个人。创始时候既己背时,失败时候当然无可告语,而其后却连这三个人也都为各自的运命所驱策,不能在一处纵谈将来的好梦了,这就是我们的并未产生的《新生》的结局。
    
    我感到未尝经验的无聊,是自此以后的事。我当初是不知其所以然的;后来想,凡有一人的主张,得了赞和,是促其前进的,得了反对,是促其奋斗的,独有叫喊于生人中,而生人并无反应,既非赞同,也无反对,如置身毫无边际的荒原,无可措手的了,这是怎样的悲哀呵,我于是以我所感到者为寂寞。
    
    这寂寞又一天一天的长大起来,如大毒蛇,缠住了我的灵魂了。
    
    然而我虽然自有无端的悲哀,却也并不愤懑,因为这经验使我反省,看见自己了:就是我决不是一个振臂一呼应者云集的英雄。
    
    只是我自己的寂寞是不可不驱除的,因为这于我太痛苦。我于是用了种种法,来麻醉自己的灵魂,使我沉入于国民中,使我回到古代去,后来也亲历或旁观过几样更寂寞更悲哀的事,都为我所不愿追怀,甘心使他们和我的脑一同消灭在泥土里的,但我的麻醉法却也似乎已经奏了功,再没有青年时候的慷慨激昂的意思了。
    
    S会馆\footnote{〔S会馆〕指绍兴县馆,在北京宣武门外。作者1912年5月到1919年11月间住在这里。}里有三间屋,相传是往昔曾在院子里的槐树上缢死过一个女人的,现在槐树已经高不可攀了,而这屋还没有人住;许多年,我便寓在这屋里钞古碑。客中少有人来,古碑中也遇不到什么问题和主义\footnote{〔问题和主义〕这里讽刺当时胡适说的“多研究些问题,少谈些主义”。},而我的生命却居然暗暗的消去了,这也就是我惟一的愿望。夏夜,蚊子多了,便摇着蒲扇坐在槐树下,从密叶缝里看那一点一点的青天,晚出的槐蚕又每每冰冷的落在头颈上。
    
    那时偶或来谈的是一个老朋友金心异\footnote{〔金心异〕指钱玄同,当时《新青年》的编辑委员之一。},将手提的大皮夹放在破桌上,脱下长衫,对面坐下了,因为怕狗,似乎心房还在怦怦的跳动。
    
    “你钞\footnote{〔钞〕抄。}了这些有什么用?”有一夜,他翻着我那古碑的钞本,发了研究的质问了。
    
    “没有什么用。”
    
    “那么,你钞他是什么意思呢?”
    
    “没有什么意思。”
    
    “我想,你可以做点文章……”
    
    我懂得他的意思了,他们正办《新青年》\footnote{〔《新青年》〕“五四”时期倡导新文化运动、传播马克思主义的重要刊物。1915年9月创刊于上海,由陈独秀主编。鲁迅在“五四”时期是该刊的重要撰稿人,曾参加该刊编辑会议。},然而那时仿佛不特没有人来赞同,并且也还没有人来反对,我想,他们许是感到寂寞了,但是说:
    
    “假如一间铁屋子,是绝无窗户而万难破毁的,里面有许多熟睡的人们,不久都要闷死了,然而是从昏睡入死灭,并不感到就死的悲哀。现在你大嚷起来,惊起了较为清醒的几个人,使这不幸的少数者来受无可挽救的临终的苦楚,你倒以为对得起他们么?”
    
    “然而几个人既然起来,你不能说决没有毁坏这铁屋的希望。”
    
    是的,我虽然自有我的确信,然而说到希望,却是不能抹杀的,因为希望是在于将来,决不能以我之必无的证明,来折服了他之所谓可有,于是我终于答应他也做文章了,这便是最初的一篇《狂人日记》。从此以后,便一发而不可收,每写些小说模样的文章,以敷衍朋友们的嘱托,积久了就有了十余篇。
    
    在我自己,本以为现在是已经并非一个切迫而不能已于言\footnote{〔不能已于言〕不能不说话,不说话就心里不安。}的人了,但或者也还未能忘怀于当日自己的寂寞的悲哀罢,所以有时候仍不免呐喊几声,聊以慰藉那在寂寞里奔驰的猛士,使他不惮于前驱。至于我的喊声是勇猛或是悲哀,是可憎或是可笑,那倒是不暇顾及的;但既然是呐喊,则当然须听将令的了,所以我往往不恤用了曲笔\footnote{〔曲笔〕指没有完全按照真实情况来写。},在《药》的瑜儿的坟上平空添上一个花环,在《明天》里也不叙单四嫂子竟没有做到看见儿子的梦,因为那时的主将是不主张消极的。至于自己,却也并不愿将自以为苦的寂寞,再来传染给也如我那年青时候似的正做着好梦的青年。
    
    这样说来,我的小说和艺术的距离之远,也就可想而知了,然而到今日还能蒙着小说的名,甚而至于且有成集的机会,无论如何总不能不说是一件侥幸的事,但侥幸虽使我不安于心,而悬揣人间暂时还有读者,则究竟也仍然是高兴的。
    
    所以我竟将我的短篇小说结集起来,而且付印了,又因为上面所说的缘由,便称之为《呐喊》。
    
    \hfill 一九二二年十二月三日,鲁迅记于北京
    
\end{normalsize}


\newpage

\textbf{注释}:

\vspace{-1em}

\begin{itemize}
    \setlength\itemsep{-0.2em}
    \item 〔小康〕指家庭经济条件相对宽裕,不愁吃穿。
    \item 〔困顿〕生计艰难,生活窘迫。
    \item 〔川资〕旅费。
    \item 〔奚落〕用尖刻的话说人坏处,使人难堪。
    \item 〔格致〕格物致知的简称,意思是穷究事物的原理而获得知识。清末用格致统称物理、化学等学科。
    \item 〔久违〕很久没见。
    \item 〔慷慨激昂〕语调高亢有力,情绪激奋昂扬。
    \item 〔不名一钱〕一文不值,没有价值。名:占有。
    \item 〔侥幸〕意外获得成功或免除灾害。
    \item 〔悬揣〕估摸,估计。
    \item 〔临终〕即将死亡。
\end{itemize}

\chapter{药}

\begin{normalsize}
    
    \begin{center}\textbf{一}\end{center}
    
    秋天的后半夜,月亮下去了,太阳还没有出,只剩下一片乌蓝的天;除了夜游的东西,什么都睡着。华老栓忽然坐起身,擦着火柴,点上遍身油腻的灯盏,茶馆的两间屋子里,便弥满了青白的光。
    
    “小栓的爹,你就去么?”是一个老女人的声音。里边的小屋子里,也发出一阵咳嗽。
    
    “唔。”老栓一面听,一面应,一面扣上衣服;伸手过去说,“你给我罢”。
    
    华大妈在枕头底下掏了半天,掏出一包洋钱\footnote{〔洋钱〕指银圆(圆片的银币)。圆片的银币来自外国,因此叫洋钱。},交给老栓,老栓接了,抖抖的装入衣袋,又在外面按了两下;便点上灯笼,吹熄灯盏,走向里屋子去了。那屋子里面,正在悉悉窣窣的响,接着便是一通咳嗽。老栓候他平静下去,才低低的叫道,“小栓……你不要起来。……店么?你娘会安排的”。
    
    老栓听得儿子不再说话,料他安心睡了;便出了门,走到街上。街上黑沉沉的一无所有,只有一条灰白的路,看得分明。灯光照着他的两脚,一前一后的走。有时也遇到几只狗,可是一只也没有叫。天气比屋子里冷多了;老栓倒觉爽快,仿佛一旦变了少年,得了神通,有给人生命的本领似的,跨步格外高远。而且路也愈走愈分明,天也愈走愈亮了。
    
    老栓正在专心走路,忽然吃了一惊,远远里看见一条丁字街,明明白白横着。他便退了几步,寻到一家关着门的铺子,蹩进檐下,靠门立住了。好一会,身上觉得有些发冷。
    
    “哼,老头子。”
    
    “倒高兴……”
    
    老栓又吃一惊,睁眼看时,几个人从他面前过去了。一个还回头看他,样子不甚分明,但很像久饿的人见了食物一般,眼里闪出一种攫取的光。老栓看看灯笼,已经熄了。按一按衣袋,硬硬的还在。仰起头两面一望,只见许多古怪的人,三三两两,鬼似的在那里徘徊;定睛再看,却也看不出什么别的奇怪。
    
    没有多久,又见几个兵,在那边走动;衣服前后的一个大白圆圈,远地里也看得清楚,走过面前的,并且看出号衣\footnote{〔号衣〕指清朝士兵的军衣。}上暗红的镶边。——一阵脚步声响,一眨眼,已经拥过了一大簇人。那三三两两的人,也忽然合作一堆,潮一般向前进;将到丁字街口,便突然立住,簇成一个半圆。
    
    老栓也向那边看,却只见一堆人的后背;颈项都伸得很长,仿佛许多鸭,被无形的手捏住了的,向上提着。静了一会,似乎有点声音,便又动摇起来,轰的一声,都向后退;一直散到老栓立着的地方,几乎将他挤倒了。
    
    “喂!一手交钱,一手交货!”一个浑身黑色的人,站在老栓面前,眼光正像两把刀,刺得老栓缩小了一半。那人一只大手,向他摊着;一只手却撮着一个鲜红的馒头\footnote{〔鲜红的馒头〕即蘸有人血的馒头。旧时迷信认为人血可以治肺痨,刽子手便借此卖人血馒头,骗取钱财。},那红的还是一点一点的往下滴。
    
    老栓慌忙摸出洋钱,抖抖的想交给他,却又不敢去接他的东西。那人便焦急起来,嚷道,“怕什么?怎的不拿!”老栓还踌躇着;黑的人便抢过灯笼,一把扯下纸罩,裹了馒头,塞与老栓;一手抓过洋钱,捏一捏,转身去了。嘴里哼着说,“这老东西……”
    
    “这给谁治病的呀?”老栓也似乎听得有人问他,但他并不答应;他的精神,现在只在一个包上,仿佛抱着一个十世单传\footnote{〔十世单传〕连续十代人只有一个儿子传宗接代。比喻十分珍稀。}的婴儿,别的事情,都已置之度外了。他现在要将这包里的新的生命,移植到他家里,收获许多幸福。太阳也出来了;在他面前,显出一条大道,直到他家中,后面也照见丁字街头破匾上“古囗亭口”\footnote{〔“古囗亭口”〕暗指“古轩亭口”,秋瑾就义的地方。“囗”表示缺字。}这四个黯淡的金字。
    
    \begin{center}\textbf{二}\end{center}
    
    老栓走到家,店面早经收拾干净,一排一排的茶桌,滑溜溜的发光。但是没有客人;只有小栓坐在里排的桌前吃饭,大粒的汗,从额上滚下,夹袄也帖住了脊心,两块肩胛骨高高凸出,印成一个阳文\footnote{〔阳文〕阳刻的文字。雕刻手法中,雕刻的形象凸出背景的叫阳刻,凹入背景的叫阴刻。}的“八”字。老栓见这样子,不免皱一皱展开的眉心。他的女人,从灶下急急走出,睁着眼睛,嘴唇有些发抖。
    
    “得了么?”
    
    “得了。”
    
    两个人一齐走进灶下,商量了一会;华大妈便出去了,不多时,拿着一片老荷叶回来,摊在桌上。老栓也打开灯笼罩,用荷叶重新包了那红的馒头。小栓也吃完饭,他的母亲慌忙说:“小栓——你坐着,不要到这里来。”一面整顿了灶火,老栓便把一个碧绿的包,一个红红白白的破灯笼,一同塞在灶里;一阵红黑的火焰过去时,店屋里散满了一种奇怪的香味。
    
    “好香!你们吃什么点心呀?”这是驼背五少爷到了。这人每天总在茶馆里过日,来得最早,去得最迟,此时恰恰蹩到临街的壁角的桌边,便坐下问话,然而没有人答应他。“炒米粥\footnote{〔炒米粥〕用炒过的大米煮成的粥。}么?”仍然没有人应。老栓匆匆走出,给他泡上茶。
    
    “小栓进来罢!”华大妈叫小栓进了里面的屋子,中间放好一条凳,小栓坐了。他的母亲端过一碟乌黑的圆东西,轻轻说:
    
    “吃下去罢,——病便好了。”
    
    小栓撮起这黑东西,看了一会,似乎拿着自己的性命一般,心里说不出的奇怪。十分小心的拗开了,焦皮里面窜出一道白气,白气散了,是两半个白面的馒头。——不多工夫,已经全在肚里了,却全忘了什么味;面前只剩下一张空盘。他的旁边,一面立着他的父亲,一面立着他的母亲,两人的眼光,都仿佛要在他身上注进什么又要取出什么似的;便禁不住心跳起来,按着胸膛,又是一阵咳嗽。
    
    “睡一会罢,——便好了。”
    
    小栓依他母亲的话,咳着睡了。华大妈候他喘气平静,才轻轻的给他盖上了满幅补钉\footnote{〔补钉〕补丁。}的夹被。
    
    \begin{center}\textbf{三}\end{center}
    
    店里坐着许多人,老栓也忙了,提着大铜壶,一趟一趟的给客人冲茶;两个眼眶,都围着一圈黑线。
    
    “老栓,你有些不舒服么?——你生病么?”一个花白胡子的人说。
    
    “没有。”
    
    “没有?——我想笑嘻嘻的,原也不像……”花白胡子便取消了自己的话。
    
    “老栓只是忙。要是他的儿子……”驼背五少爷话还未完,突然闯进了一个满脸横肉的人,披一件玄色布衫,散着纽扣,用很宽的玄色腰带,胡乱捆在腰间。刚进门,便对老栓嚷道:
    
    “吃了么?好了么?老栓,就是运气了你!你运气,要不是我信息灵……”
    
    老栓一手提了茶壶,一手恭恭敬敬的垂着;笑嘻嘻的听。满座的人,也都恭恭敬敬的听。华大妈也黑着眼眶,笑嘻嘻的送出茶碗茶叶来,加上一个橄榄,老栓便去冲了水。
    
    “这是包好!这是与众不同的。你想,趁热的拿来,趁热的吃下。”横肉的人只是嚷。
    
    “真的呢,要没有康大叔照顾,怎么会这样……”华大妈也很感激的谢他。
    
    “包好,包好!这样的趁热吃下。这样的人血馒头,什么痨病都包好!”
    
    华大妈听到“痨病”这两个字,变了一点脸色,似乎有些不高兴;但又立刻堆上笑,搭赸着走开了。这康大叔却没有觉察,仍然提高了喉咙只是嚷,嚷得里面睡着的小栓也合伙咳嗽起来。
    
    “原来你家小栓碰到了这样的好运气了。这病自然一定全好;怪不得老栓整天的笑着呢。”花白胡子一面说,一面走到康大叔面前,低声下气的问道,“康大叔——听说今天结果的一个犯人,便是夏家的孩子,那是谁的孩子?究竟是什么事?”
    
    “谁的?不就是夏四奶奶的儿子么?那个小家伙!”康大叔见众人都耸起耳朵听他,便格外高兴,横肉块块饱绽\footnote{〔饱绽〕饱满得像要绽开。},越发大声说,“这小东西不要命,不要就是了。我可是这一回一点没有得到好处;连剥下来的衣服,都给管牢的红眼睛阿义拿去了。——第一要算我们栓叔运气;第二是夏三爷赏了二十五两雪白的银子,独自落腰包,一文不花。”
    
    小栓慢慢的从小屋子里走出,两手按了胸口,不住的咳嗽;走到灶下,盛出一碗冷饭,泡上热水,坐下便吃。华大妈跟着他走,轻轻的问道,“小栓,你好些么?——你仍旧只是肚饿?……”
    
    “包好,包好!”康大叔瞥了小栓一眼,仍然回过脸,对众人说,“夏三爷真是乖角儿\footnote{〔乖角儿〕机警、机敏的人。},要是他不先告官,连他满门抄斩。现在怎样?银子!——这小东西也真不成东西!关在牢里,还要劝牢头造反。”
    
    “阿呀,那还了得。”坐在后排的一个二十多岁的人,很现出气愤模样。
    
    “你要晓得红眼睛阿义是去盘盘底细的,他却和他攀谈了。他说:这大清的天下是我们大家的。你想:这是人话么?红眼睛原知道他家里只有一个老娘,可是没有料到他竟会这么穷,榨不出一点油水,已经气破肚皮了。他还要老虎头上搔痒,便给他两个嘴巴!”
    
    “义哥是一手好拳棒,这两下,一定够他受用了。”壁角的驼背忽然高兴起来。
    
    “他这贱骨头打不怕,还要说可怜可怜哩。”
    
    花白胡子的人说,“打了这种东西,有什么可怜呢?”
    
    康大叔显出看他不上的样子,冷笑着说,“你没有听清我的话;看他神气,是说阿义可怜哩!”
    
    听着的人的眼光,忽然有些板滞;话也停顿了。小栓已经吃完饭,吃得满头流汗,头上都冒出蒸气来。
    
    “阿义可怜——疯话,简直是发了疯了。”花白胡子恍然大悟似的说。
    
    “发了疯了。”二十多岁的人也恍然大悟的说。
    
    店里的坐客,便又现出活气,谈笑起来。小栓也趁着热闹,拚命咳嗽;康大叔走上前,拍他肩膀说:
    
    “包好!小栓——你不要这么咳。包好!”
    
    “疯了!”驼背五少爷点着头说。
    
    \begin{center}\textbf{四}\end{center}
    
    西关外靠着城根的地面,本是一块官地;中间歪歪斜斜一条细路,是贪走便道的人,用鞋底造成的,但却成了自然的界限。路的左边,都埋着死刑和瘐毙的人,右边是穷人的丛冢\footnote{〔丛冢〕乱坟堆。冢,坟墓。}。两面都已埋到层层叠叠,宛然阔人家里祝寿时的馒头。
    
    这一年的清明,分外寒冷;杨柳才吐出半粒米大的新芽。天明未久,华大妈已在右边的一坐新坟前面,排出四碟菜,一碗饭,哭了一场。化过纸\footnote{〔化过纸〕指烧过了纸钱。纸:纸钱,旧时迷信认为把纸钱火化,可以给死人在阴间用。后面“纸锭”指用纸或者锡箔折成的元宝。},呆呆的坐在地上;仿佛等候什么似的,但自己也说不出等候什么。微风起来,吹动他短发,确乎比去年白得多了。
    
    小路上又来了一个女人,也是半白头发,褴褛的衣裙;提一个破旧的朱漆圆篮,外挂一串纸锭,三步一歇的走。忽然见华大妈坐在地上看她,便有些踌躇,惨白的脸上,现出些羞愧的颜色;但终于硬着头皮,走到左边的一坐坟前,放下了篮子。
    
    那坟与小栓的坟,一字儿排着,中间只隔一条小路。华大妈看他排好四碟菜,一碗饭,立着哭了一通,化过纸锭;心里暗暗地想,“这坟里的也是儿子了。”那老女人徘徊观望了一回,忽然手脚有些发抖,跄跄踉踉退下几步,瞪着眼只是发怔。
    
    华大妈见这样子,生怕她伤心到快要发狂了;便忍不住立起身,跨过小路,低声对他说,“你这位老奶奶不要伤心了,——我们还是回去罢。”
    
    那人点一点头,眼睛仍然向上瞪着;也低声痴痴的说道,“你看,——看这是什么呢?”
    
    华大妈跟了他指头看去,眼光便到了前面的坟,这坟上草根还没有全合,露出一块一块的黄土,煞是难看。再往上仔细看时,却不觉也吃一惊;——分明有一圈红白的花,围着那尖圆的坟顶。
    
    他们的眼睛都已老花多年了,但望这红白的花,却还能明白看见。花也不很多,圆圆的排成一个圈,不很精神,倒也整齐。华大妈忙看他儿子和别人的坟,却只有不怕冷的几点青白小花,零星开着;便觉得心里忽然感到一种不足和空虚,不愿意根究。那老女人又走近几步,细看了一遍,自言自语的说,“这没有根,不像自己开的。——这地方有谁来呢?孩子不会来玩;——亲戚本家早不来了。——这是怎么一回事呢?”他想了又想,忽又流下泪来,大声说道:
    
    “瑜儿,他们都冤枉了你,你还是忘不了,伤心不过,今天特意显点灵,要我知道么?”他四面一看,只见一只乌鸦,站在一株没有叶的树上,便接着说,“我知道了。——瑜儿,可怜他们坑了你,他们将来总有报应,天都知道;你闭了眼睛就是了。——你如果真在这里,听到我的话,——便教这乌鸦飞上你的坟顶,给我看罢。”
    
    微风早经停息了;枯草支支直立,有如铜丝。一丝发抖的声音,在空气中愈颤愈细,细到没有,周围便都是死一般静。两人站在枯草丛里,仰面看那乌鸦;那乌鸦也在笔直的树枝间,缩着头,铁铸一般站着。
    
    许多的工夫过去了;上坟\footnote{〔上坟〕到坟墓拜祭死者。}的人渐渐增多,几个老的小的,在土坟间出没。
    
    华大妈不知怎的,似乎卸下了一挑重担,便想到要走;一面劝着说,“我们还是回去罢”。
    
    那老女人叹一口气,无精打采的收起饭菜;又迟疑了一刻,终于慢慢地走了。嘴里自言自语的说,“这是怎么一回事呢?……”
    
    他们走不上二三十步远,忽听得背后“哑——”的一声大叫;两个人都竦然的回过头,只见那乌鸦张开两翅,一挫身\footnote{〔一挫身〕为了发力而收缩身体。},直向着远处的天空,箭也似的飞去了。
    
    \hfill 一九一九年四月二十五日
    
\end{normalsize}


\newpage

\textbf{注释}:

\vspace{-1em}

\begin{itemize}
    \setlength\itemsep{-0.2em}
    \item 〔报应〕旧时迷信认为做善事或恶事有相应的后果回报。
    \item 〔玄色〕稍微带红的黑色。
    \item 〔搭赸〕搭讪,为了跟人接近或把尴尬的局面应付过去而没话找话说。
    \item 〔拗开〕用手掰开。拗,用手折断。
    \item 〔满门抄斩〕抄没财产,杀戮全家。
    \item 〔板滞〕呆板,停滞不动。
    \item 〔瘐毙〕关在牢狱里的人因受刑或饥寒、疾病而死亡。
    \item 〔竦然〕悚然,惊惧的样子。“竦”通“悚”。
\end{itemize}

\chapter{阿Q正传}

\begin{normalsize}
    
    \begin{center}\textbf{第一章~序}\end{center}
    
    我要给阿Q做正传,已经不止一两年了。但一面要做,一面又往回想,这足见我不是一个“立言”\footnote{〔“立言”〕我国古代所谓“三不朽”之一。《左传·襄公二十四年》中鲁国大夫叔孙豹说:“太上有立德,其次有立功,其次有立言,虽久不废,此之谓不朽。”}的人,因为从来不朽之笔,须传不朽之人,于是人以文传,文以人传——究竟谁靠谁传,渐渐的不甚了然起来,而终于归接到传阿Q,仿佛思想里有鬼似的。
    
    然而要做这一篇速朽的文章,才下笔,便感到万分的困难了。第一是文章的名目。孔子曰,“名不正则言不顺”\footnote{〔“名不正则言不顺”〕出自《论语·子路》。}。这原是应该极注意的。传的名目很繁多:列传,自传,内传\footnote{〔内传〕小说体传记的一种。},外传,别传,家传,小传……,而可惜都不合。“列传”么,这一篇并非和许多阔人排在“正史”\footnote{〔“正史”〕官方撰修或认可的史书。清代乾隆时规定自《史记》至《明史》历代二十四部纪传体史书为“正史”。“正史”中的“列传”部分,一般都是著名人物的传记。}里;“自传”么,我又并非就是阿Q。说是“外传”,“内传”在那里呢?倘用“内传”,阿Q又决不是神仙。“别传”呢,阿Q实在未曾有大总统上谕宣付国史馆立“本传”\footnote{〔宣付国史馆立“本传”〕旧时效忠于统治阶级的重要人物或名人,死后由政府明令褒扬,令文末常有“宣付国史馆立传”的话。国史馆:清代编纂史书的机构名称。辛亥革命后,北洋军阀及国民党政府都曾沿用。}——虽说英国正史上并无“博徒列传”,而文豪迭更司\footnote{〔迭更司〕通译查尔斯·狄更斯,19世纪英国小说家。著有《大卫·科波菲尔》、《双城记》等。《博徒别传》原名《劳特奈·斯吞》,英国小说家柯南·道尔著。此处为鲁迅记错。}也做过《博徒别传》这一部书,但文豪则可,在我辈却不可。其次是“家传”,则我既不知与阿Q是否同宗,也未曾受他子孙的拜托;或“小传”,则阿Q又更无别的“大传”了。总而言之,这一篇也便是“本传”,但从我的文章着想,因为文体卑下,是“引车卖浆者流”所用的话\footnote{〔“引车卖浆者流”所用的话〕指白话文。旧时严肃文章必须用文言写作。},所以不敢僭称,便从不入三教九流的小说家\footnote{〔三教九流〕三教:指儒教、佛教、道教;九流:指先秦的九大学术流派。《汉书·艺文志》:“小说家者流,盖出于稗官。街谈巷语,道听途说者之所造也。……是以君子弗为也。”}所谓“闲话休题言归正传”这一句套话里,取出“正传”两个字来,作为名目,即使与古人所撰《书法正传》\footnote{〔《书法正传》〕清代冯武写的一部关于书法的书。这里“正传”指“正确的传授”。}的“正传”字面上很相混,也顾不得了。
    
    第二,立传的通例,开首大抵该是“某,字某,某地人也”,而我并不知道阿Q姓什么。有一回,他似乎是姓赵,但第二日便模糊了。那是赵太爷的儿子进了秀才的时候,锣声镗镗的报到村里来,阿Q正喝了两碗黄酒,便手舞足蹈的说,这于他也很光采,因为他和赵太爷原来是本家,细细的排起来他还比秀才长三辈呢。其时几个旁听人倒也肃然的有些起敬了。那知道第二天,地保\footnote{〔地保〕地方保甲,旧时替官府办杂事的差役,演变为地方士绅扶持的流民打手。}便叫阿Q到赵太爷家里去;太爷一见,满脸溅朱\footnote{〔满脸溅朱〕满脸通红。},喝道:
    
    “阿Q,你这浑小子!你说我是你的本家么?”
    
    阿Q不开口。
    
    赵太爷愈看愈生气了,抢进几步说:“你敢胡说!我怎么会有你这样的本家?你姓赵么?”
    
    阿Q不开口,想往后退了;赵太爷跳过去,给了他一个嘴巴。
    
    “你怎么会姓赵!——你那里配姓赵!”
    
    阿Q并没有抗辩他确凿姓赵,只用手摸着左颊,和地保退出去了;外面又被地保训斥了一番,谢了地保二百文酒钱。知道的人都说阿Q太荒唐,自己去招打;他大约未必姓赵,即使真姓赵,有赵太爷在这里,也不该如此胡说的。此后便再没有人提起他的氏族来,所以我终于不知道阿Q究竟什么姓。
    
    第三,我又不知道阿Q的名字是怎么写的。他活着的时候,人都叫他阿Quei,死了以后,便没有一个人再叫阿Quei了,那里还会有“著之竹帛”\footnote{〔“著之竹帛”〕出自《吕氏春秋·仲春纪》:“著乎竹帛,传乎后世。”}的事。若论“著之竹帛”,这篇文章要算第一次,所以先遇着了这第一个难关。我曾仔细想:阿Quei,阿桂还是阿贵呢?倘使他号月亭,或者在八月间做过生日,那一定是阿桂了;而他既没有号——也许有号,只是没有人知道他,——又未尝散过生日征文的帖子:写作阿桂,是武断的。又倘使他有一位老兄或令弟叫阿富,那一定是阿贵了;而他又只是一个人:写作阿贵,也没有佐证的。其余音Quei的偏僻字样,更加凑不上了。先前,我也曾问过赵太爷的儿子茂才\footnote{〔茂才〕即秀才。东汉时,因为避光武帝刘秀的名讳,改秀才为茂才;后来有时也沿用作秀才的别称。}先生,谁料博雅如此公,竟也茫然,但据结论说,是因为陈独秀办了《新青年》提倡洋字\footnote{〔陈独秀办了《新青年》提倡洋字〕指1918年前后钱玄同等人在《新青年》杂志上开展关于废除汉字、改用罗马字母拼音的讨论一事。此处鲁迅误记为陈独秀提倡。},所以国粹沦亡,无可查考了。我的最后的手段,只有托一个同乡去查阿Q犯事的案卷,八个月之后才有回信,说案卷里并无与阿Quei的声音相近的人。我虽不知道是真没有,还是没有查,然而也再没有别的方法了。生怕注音字母还未通行,只好用了“洋字”,照英国流行的拼法写他为阿Quei,略作阿Q。这近于盲从《新青年》,自己也很抱歉,但茂才公尚且不知,我还有什么好办法呢。
    
    第四,是阿Q的籍贯了。倘他姓赵,则据现在好称郡望的老例,可以照《郡名百家姓》\footnote{〔《郡名百家姓》〕《百家姓》是以前学塾所用的识字课本之一,宋初人编纂。为便于诵读,将姓氏连缀为四言韵语。《郡名百家姓》则在每一姓上都附注郡(古代地方区域的名称)名,表示某姓望族曾居古代某地,如赵为“天水”、钱为“彭城”之类。}上的注解,说是“陇西天水人也”,但可惜这姓是不甚可靠的,因此籍贯也就有些决不定。他虽然多住未庄,然而也常常宿在别处,不能说是未庄人,即使说是“未庄人也”,也仍然有乖史法的。
    
    我所聊以自慰的,是还有一个“阿”字非常正确,绝无附会假借的缺点,颇可以就正于通人。至于其余,却都非浅学所能穿凿,只希望有“历史癖与考据癖”的胡适之\footnote{〔胡适之〕即胡适,安徽绩溪人,20世纪买办资产阶级文人、政客。他在1920年7月所作《〈水浒传〉考证》中自称“有历史癖与考据癖”。}先生的门人们,将来或者能够寻出许多新端绪来,但是我这《阿Q正传》到那时却又怕早经消灭了。
    
    以上可以算是序。
    
    \begin{center}\textbf{第二章~优胜记略}\end{center}
    
    阿Q不独是姓名籍贯有些渺茫,连他先前的“行状”\footnote{〔“行状”〕原指封建时代记述死者世系、籍贯、生卒、事迹的文字,一般由其家属撰写。这里泛指经历。}也渺茫。因为未庄的人们之于阿Q,只要他帮忙,只拿他玩笑,从来没有留心他的“行状”的。而阿Q自己也不说,独有和别人口角的时候,间或瞪着眼睛道:
    
    “我们先前——比你阔的多啦!你算是什么东西!”
    
    阿Q没有家,住在未庄的土谷祠\footnote{〔土谷祠〕即土地庙。土谷,指土地神和五谷神。}里;也没有固定的职业,只给人家做短工,割麦便割麦,舂米便舂米,撑船便撑船。工作略长久时,他也或住在临时主人的家里,但一完就走了。所以,人们忙碌的时候,也还记起阿Q来,然而记起的是做工,并不是“行状”;一闲空,连阿Q都早忘却,更不必说“行状”了。只是有一回,有一个老头子颂扬说:“阿Q真能做!”这时阿Q赤着膊,懒洋洋的瘦伶仃的正在他面前,别人也摸不着这话是真心还是讥笑,然而阿Q很喜欢。
    
    阿Q又很自尊,所有未庄的居民,全不在他眼神里,甚而至于对于两位“文童”\footnote{〔“文童”〕也称“童生”,指科举时代准备科举而尚未考取秀才的人。}也有以为不值一笑的神情。夫文童者,将来恐怕要变秀才者也;赵太爷钱太爷大受居民的尊敬,除有钱之外,就因为都是文童的爹爹,而阿Q在精神上独不表格外的崇奉,他想:我的儿子会阔得多啦!加以进了几回城,阿Q自然更自负,然而他又很鄙薄城里人,譬如用三尺三寸宽的木板做成的凳子,未庄人叫“长凳”,他也叫“长凳”,城里人却叫“条凳”,他想:这是错的,可笑!油煎大头鱼,未庄都加上半寸长的葱叶,城里却加上切细的葱丝,他想:这也是错的,可笑!然而未庄人真是不见世面的可笑的乡下人呵,他们没有见过城里的煎鱼!
    
    阿Q“先前阔”,见识高,而且“真能做”,本来几乎是一个“完人\footnote{〔完人〕品行优秀毫无缺点的人。}”了,但可惜他体质上还有一些缺点。最恼人的是在他头皮上,颇有几处不知于何时的癞疮疤。这虽然也在他身上,而看阿Q的意思,倒也似乎以为不足贵的,因为他讳说“癞”以及一切近于“赖”的音,后来推而广之,“光”也讳,“亮”也讳,再后来,连“灯”“烛”都讳了。一犯讳,不问有心与无心,阿Q便全疤通红的发起怒来,估量了对手,口讷\footnote{〔口讷〕嘴笨,口头表达能力差。}的他便骂,气力小的他便打;然而不知怎么一回事,总还是阿Q吃亏的时候多。于是他渐渐的变换了方针,大抵改为怒目而视了。
    
    谁知道阿Q采用怒目主义之后,未庄的闲人们便愈喜欢玩笑他。一见面,他们便假作吃惊的说:
    
    “哙,亮起来了。”
    
    阿Q照例的发了怒,他怒目而视了。
    
    “原来有保险灯在这里!”他们并不怕。
    
    阿Q没有法,只得另外想出报复的话来:
    
    “你还不配……”这时候,又仿佛在他头上的是一种高尚的光容的癞头疮,并非平常的癞头疮了;但上文说过,阿Q是有见识的,他立刻知道和“犯忌”有点抵触,便不再往底下说。
    
    闲人还不完,只撩他,于是终而至于打。阿Q在形式上打败了,被人揪住黄辫子,在壁上碰了四五个响头,闲人这才心满意足的得胜的走了,阿Q站了一刻,心里想,“我总算被儿子打了,现在的世界真不像样……”于是也心满意足的得胜的走了。
    
    阿Q想在心里的,后来每每说出口来,所以凡是和阿Q玩笑的人们,几乎全知道他有这一种精神上的胜利法,此后每逢揪住他黄辫子的时候,人就先一着对他说:
    
    “阿Q,这不是儿子打老子,是人打畜生。自己说:人打畜生!”
    
    阿Q两只手都捏住了自己的辫根,歪着头,说道:
    
    “打虫豸\footnote{〔虫豸〕小虫子。这里是骂人的话。},好不好?我是虫豸——还不放么?”
    
    但虽然是虫豸,闲人也并不放,仍旧在就近什么地方给他碰了五六个响头,这才心满意足的得胜的走了,他以为阿Q这回可遭了瘟。然而不到十秒钟,阿Q也心满意足的得胜的走了,他觉得他是第一个能够自轻自贱的人,除了“自轻自贱”不算外,余下的就是“第一个”。状元\footnote{〔状元〕科举中经皇帝殿试取中的第一名进士。}不也是“第一个”么?“你算是什么东西”呢!?
    
    阿Q以如是等等妙法克服怨敌之后,便愉快的跑到酒店里喝几碗酒,又和别人调笑一通,口角一通,又得了胜,愉快的回到土谷祠,放倒头睡着了。假使有钱,他便去押牌宝\footnote{〔押牌宝〕一种赌博。赌局中为主的人叫“桩家”;下文的“青龙”、“天门”、“穿堂”等都是押牌宝的用语,指押赌注的位置;“四百”、“一百五十”是押赌注的钱数。},一推人蹲在地面上,阿Q即汗流满面的夹在这中间,声音他最响:
    
    “青龙四百!”
    
    “咳~开~啦!”桩家揭开盒子盖,也是汗流满面的唱。“天门啦~角回啦~~!人和穿堂空在那里啦~~!阿Q的铜钱拿过来~~!”
    
    “穿堂一百——一百五十!”
    
    阿Q的钱便在这样的歌吟之下,渐渐的输入别个汗流满面的人物的腰间。他终于只好挤出堆外,站在后面看,替别人着急,一直到散场,然后恋恋的回到土谷祠,第二天,肿着眼睛去工作。
    
    但真所谓“塞翁失马安知非福”罢,阿Q不幸而赢了一回,他倒几乎失败了。
    
    这是未庄赛神\footnote{〔赛神〕即迎神赛会,旧时的一种迷信习俗。以鼓乐仪仗和杂戏等迎神出庙,周游街巷,以酬神祈福。}的晚上。这晚上照例有一台戏,戏台左近,也照例有许多的赌摊。做戏的锣鼓,在阿Q耳朵里仿佛在十里之外;他只听得桩家的歌唱了。他赢而又赢,铜钱变成角洋,角洋变成大洋,大洋又成了叠。他兴高采烈得非常:
    
    “天门两块!”
    
    他不知道谁和谁为什么打起架来了。骂声打声脚步声,昏头昏脑的一大阵,他才爬起来,赌摊不见了,人们也不见了,身上有几处很似乎有些痛,似乎也挨了几拳几脚似的,几个人诧异的对他看。他如有所失的走进土谷祠,定一定神,知道他的一堆洋钱不见了。赶赛会的赌摊多不是本村人,还到那里去寻根柢呢?
    
    很白很亮的一堆洋钱!而且是他的——现在不见了!说是算被儿子拿去了罢,总还是忽忽不乐;说自己是虫豸罢,也还是忽忽不乐:他这回才有些感到失败的苦痛了。
    
    但他立刻转败为胜了。他擎起右手,用力的在自己脸上连打了两个嘴巴,热剌剌的有些痛;打完之后,便心平气和起来,似乎打的是自己,被打的是别一个自己,不久也就仿佛是自己打了别个一般,——虽然还有些热剌剌,——心满意足的得胜的躺下了。
    
    他睡着了。
    
    \begin{center}\textbf{第三章~续优胜记略}\end{center}
    
    然而阿Q虽然常优胜,却直待蒙赵太爷打他嘴巴之后,这才出了名。
    
    他付过地保二百文酒钱,愤愤的躺下了,后来想:“现在的世界太不成话,儿子打老子……”于是忽而想到赵太爷的威风,而现在是他的儿子了,便自己也渐渐的得意起来,爬起身,唱着《小孤孀上坟》\footnote{〔《小孤孀上坟》〕当时流行的一出绍兴地方戏。}到酒店去。这时候,他又觉得赵太爷高人一等了。
    
    说也奇怪,从此之后,果然大家也仿佛格外尊敬他。这在阿Q,或者以为因为他是赵太爷的父亲,而其实也不然。未庄通例,倘如阿七打阿八,或者李四打张三,向来本不算口碑。一上口碑,则打的既有名,被打的也就托庇有了名。至于错在阿Q,那自然是不必说。所以者何?\footnote{〔所以者何?〕原因是什么呢?}就因为赵太爷是不会错的。但他既然错,为什么大家又仿佛格外尊敬他呢?这可难解,穿凿起来说,或者因为阿Q说是赵太爷的本家,虽然挨了打,大家也还怕有些真,总不如尊敬一些稳当。否则,也如孔庙里的太牢\footnote{〔太牢〕按古代祭礼,原指牛、羊、豕三牲,但后来单称牛为太牢。}一般,虽然与猪羊一样,同是畜生,但既经圣人下箸,先儒们便不敢妄动了。
    
    阿Q此后倒得意了许多年。
    
    有一年的春天,他醉醺醺的在街上走,在墙根的日光下,看见王胡在那里赤着膊捉虱子,他忽然觉得身上也痒起来了。这王胡,又癞又胡,别人都叫他王癞胡,阿Q却删去了一个癞字,然而非常渺视\footnote{〔渺视〕藐视。}他。阿Q的意思,以为癞是不足为奇的,只有这一部络腮胡子,实在太新奇,令人看不上眼。他于是并排坐下去了。倘是别的闲人们,阿Q本不敢大意坐下去。但这王胡旁边,他有什么怕呢?老实说:他肯坐下去,简直还是抬举他。
    
    阿Q也脱下破夹袄来,翻检了一回,不知道因为新洗呢还是因为粗心,许多工夫,只捉到三四个。他看那王胡,却是一个又一个,两个又三个,只放在嘴里毕毕剥剥的响。
    
    阿Q最初是失望,后来却不平了:看不上眼的王胡尚且那么多,自己倒反这样少,这是怎样的大失体统的事呵!他很想寻一两个大的,然而竟没有,好容易才捉到一个中的,恨恨的塞在厚嘴唇里,狠命一咬,劈的一声,又不及王胡的响。
    
    他癞疮疤块块通红了,将衣服摔在地上,吐一口唾沫,说:
    
    “这毛虫!”
    
    “癞皮狗,你骂谁?”王胡轻蔑的抬起眼来说。
    
    阿Q近来虽然比较的受人尊敬,自己也更高傲些,但和那些打惯的闲人们见面还胆怯,独有这回却非常武勇了。这样满脸胡子的东西,也敢出言无状\footnote{〔出言无状〕说话没有分寸,超越了本人身份地位,显得无礼。}么?
    
    “谁认便骂谁!”他站起来,两手叉在腰间说。
    
    “你的骨头痒了么?”王胡也站起来,披上衣服说。
    
    阿Q以为他要逃了,抢进去就是一拳。这拳头还未达到身上,已经被他抓住了,只一拉,阿Q跄跄踉踉的跌进去,立刻又被王胡扭住了辫子,要拉到墙上照例去碰头。
    
    “‘君子动口不动手’!”阿Q歪着头说。
    
    王胡似乎不是君子,并不理会,一连给他碰了五下,又用力的一推,至于阿Q跌出六尺多远,这才满足的去了。
    
    在阿Q的记忆上,这大约要算是生平第一件的屈辱,因为王胡以络腮胡子的缺点,向来只被他奚落,从没有奚落他,更不必说动手了。而他现在竟动手,很意外,难道真如市上所说,皇帝已经停了考\footnote{〔皇帝已经停了考〕指1905年起清朝政府停止科举考试。},不要秀才和举人了,因此赵家减了威风,因此他们也便小觑了他么?
    
    阿Q无可适从的站着。
    
    远远的走来了一个人,他的对头又到了。这也是阿Q最厌恶的一个人,就是钱太爷的大儿子。他先前跑上城里去进洋学堂,不知怎么又跑到东洋\footnote{〔东洋〕指日本。}去了,半年之后他回到家里来,腿也直了,辫子也不见了,他的母亲大哭了十几场,他的老婆跳了三回井。后来,他的母亲到处说,“这辫子是被坏人灌醉了酒剪去了。本来可以做大官,现在只好等留长再说了。”然而阿Q不肯信,偏称他“假洋鬼子”\footnote{〔“假洋鬼子”〕“假洋鬼子”对举止西化的中国人的蔑称。“洋鬼子”:对外国人的蔑称。},也叫作“里通外国的人”,一见他,一定在肚子里暗暗的咒骂。
    
    阿Q尤其“深恶而痛绝之”的,是他的一条假辫子。辫子而至于假,就是没有了做人的资格;他的老婆不跳第四回井,也不是好女人。
    
    这“假洋鬼子”近来了。
    
    “秃儿。驴……”阿Q历来本只在肚子里骂,没有出过声,这回因为正气忿,因为要报仇,便不由的轻轻的说出来了。
    
    不料这秃儿却拿着一支黄漆的棍子——就是阿Q所谓哭丧棒\footnote{〔哭丧棒〕旧时在为父母送殡时,儿子须手拄“孝杖”,以表示悲痛难支。}——大蹋步走了过来。阿Q在这刹那,便知道大约要打了,赶紧抽紧筋骨,耸了肩膀等候着,果然,“啪”的一声,似乎确凿打在自己头上了。
    
    “我说他!”阿Q指着近旁的一个孩子,分辩说。
    
    “啪!啪啪!”
    
    在阿Q的记忆上,这大约要算是生平第二件的屈辱。幸而拍拍的响了之后,于他倒似乎完结了一件事,反而觉得轻松些,而且“忘却”这一件祖传的宝贝也发生了效力,他慢慢的走,将到酒店门口,早已有些高兴了。
    
    但对面走来了静修庵里的小尼姑\footnote{〔尼姑〕佛教用语,指出家修行的女子。}。阿Q便在平时,看见伊\footnote{〔伊〕她。}也一定要唾骂,而况在屈辱之后呢?他于是发生了回忆,又发生了敌忾了。
    
    “我不知道我今天为什么这样晦气,原来就因为见了你!”他想。
    
    他迎上去,大声的吐一口唾沫:
    
    “咳,呸!”
    
    小尼姑全不睬,低了头只是走。阿Q走近伊身旁,突然伸出手去摩着伊新剃的头皮,呆笑着,说:
    
    “秃儿!快回去,和尚等着你……”
    
    “你怎么动手动脚……”尼姑满脸通红的说,一面赶快走。
    
    酒店里的人大笑了。阿Q看见自己的勋业\footnote{〔勋业〕功勋伟业。}得了赏识,便愈加兴高采烈起来:
    
    “和尚动得,我动不得?”他扭住伊的面颊。
    
    酒店里的人大笑了。阿Q更得意,而且为了满足那些赏鉴家起见,再用力的一拧,才放手。
    
    他这一战,早忘却了王胡,也忘却了假洋鬼子,似乎对于今天的一切“晦气”都报了仇;而且奇怪,又仿佛全身比拍拍的响了之后轻松,飘飘然的似乎要飞去了。
    
    “这断子绝孙的阿Q!”远远地听得小尼姑的带哭的声音。
    
    “哈哈哈!”阿Q十分得意的笑。
    
    “哈哈哈!”酒店里的人也九分得意的笑。
    
\end{normalsize}


\newpage

\textbf{注释}:

\vspace{-1em}

\begin{itemize}
    \setlength\itemsep{-0.2em}
    \item 〔小觑〕小看。觑:眯着眼看。
    \item 〔托庇〕依赖(某人、某势力的)庇护。
    \item 〔舂〕把糙米捣成精米。
    \item 〔气忿〕气愤。
    \item 〔敌忾〕对敌人的愤恨。
    \item 〔晦气〕坏运气。旧时迷信认为沾了“晦气”会导致运气不好。
\end{itemize}

\chapter{范进中举}

\begin{normalsize}
    
    范进进学\footnote{〔进学〕指考中秀才。}回家,母亲、妻子俱各欢喜。正待烧锅做饭,只见他丈人胡屠户,手里拿着一副大肠和一瓶酒,走了进来。范进向他作揖,坐下。胡屠户道:“我自倒运,把个女儿嫁与你这现世宝\footnote{〔现世宝〕丢脸的家伙。现世,出丑,丢脸。},历年以来,不知累了我多少。如今不知因我积了甚么德,带挈你中了个相公\footnote{〔相公〕古时对秀才的称呼。},我所以带个酒来贺你。”范进唯唯连声\footnote{〔唯唯连声〕连连答应。唯唯,答应的声音。},叫浑家把肠子煮了,烫起酒来,在茅草棚下坐着。母亲自和媳妇在厨下做饭。胡屠户又吩咐女婿道:“你如今既中了相公,凡事要立起个体统来。比如我这行事\footnote{〔行事〕行当,行业。}里,都是些正经有脸面的人,又是你的长亲\footnote{〔长亲〕长辈。},你怎敢在我们跟前装大?若是家门口这些做田的,扒粪的,不过是平头百姓,你若同他拱手作揖,平起平坐,这就是坏了学校规矩,连我脸上都无光了。你是个烂忠厚没用的人,所以这些话我不得不教导你,免得惹人笑话。”范进道:“岳父见教的是。”胡屠户又道:“亲家母也来这里坐着吃饭。老人家每日小菜饭,想也难过。我女孩儿也吃些。自从进了你家门,这十几年,不知猪油可曾吃过两三回哩!可怜!可怜!”说罢,婆媳两个都来坐着吃了饭。吃到日西时分,胡屠户吃的醺醺的。这里母子两个,千恩万谢。屠户横披了衣服,腆着肚子去了。
    
    次日,范进少不得拜拜乡邻。魏好古又约了一班同案\footnote{〔同案〕同考取秀才叫做同案。}的朋友,彼此来往。因是乡试年\footnote{〔乡试年〕科举制度,每三年举行一次全省的考试,叫“乡试”,由秀才去应试。轮到乡试这一年就叫乡试年。},做了几个文会\footnote{〔文会〕旧时读书人为了准备应试,在一起写文章、互相观摩的集会。}。不觉到了六月尽间,这些同案的人约范进去乡试。范进因没有盘费,走去同丈人商议,被胡屠户一口啐在脸上,骂了一个狗血喷头,道:“不要失了你的时了!你自己只觉得中了一个相公,就‘癞蛤蟆想吃起天鹅肉’来!我听见人说,就是中相公时,也不是你的文章,还是宗师\footnote{〔宗师〕对一省总管教育的学官的称呼。}看见你老,不过意,舍与你的。如今痴心就想中起老爷\footnote{〔老爷〕这里是对举人的称呼。}来!这些中老爷的都是天上的‘文曲星’!你不看见城里张府上那些老爷,都有万贯家私,一个个方面大耳?像你这尖嘴猴腮,也该撒抛\footnote{〔抛〕量词,现在写作“泡”。}尿自己照照!不三不四,就想天鹅屁吃!趁早收了这心,明年在我们行事里替你寻一个馆\footnote{〔馆〕这里指教书的地方。},每年寻几两银子,养活你那老不死的老娘和你老婆是正经!你问我借盘缠,我一天杀一个猪还赚不得钱把银子,都把与\footnote{〔把与〕拿给。}你去丢在水里,叫我一家老小嗑西北风!”一顿夹七夹八,骂的范进摸不着门\footnote{〔摸不着门〕摸不着门路,意思是不知从何说起。}。辞了丈人回来,自心里想:“宗师说我火候\footnote{〔火候〕这里指写文章的功夫。}已到,自古无场外的举人,如不进去考他一考,如何甘心?”因向几个同案商议,瞒着丈人,到城里乡试。出了场,即便回家。家里已是饿了两三天。被胡屠户知道,又骂了一顿。
    
    到出榜那日,家里没有早饭的米,母亲吩咐范进道:“我有一只生蛋的母鸡,你快拿集上去卖了,买几升米来煮餐粥吃,我已是饿的两眼都看不见了。”范进慌忙抱了鸡,走出门去。才去不到两个时候,只听得一片声的锣响,三匹马闯将来。那三个人下了马,把马拴在茅草棚上,一片声叫道:“快请范老爷出来,恭喜高中了!”母亲不知是甚事,吓得躲在屋里;听见中了,方敢伸出头来,说道:“诸位请坐,小儿方才出去了。”那些报录人道:“原来是老太太。”大家簇拥着要喜钱。正在吵闹,又是几匹马,二报、三报到了,挤了一屋的人,茅草棚地下都坐满了。邻居都来了,挤着看。老太太没奈何,只得央及一个邻居去寻他儿子。
    
    那邻居飞奔到集上,一地里\footnote{〔一地里〕一路上,到处。}寻不见;直寻到集东头,见范进抱着鸡,手里插个草标\footnote{〔草标〕在集市上卖东西,把一根草插在出卖的物品上或拿在手里,作为标志,这草就叫做“草标”。},一步一踱的,东张西望,在那里寻人买。邻居道:“范相公,快些回去!你恭喜中了举人,报喜人挤了一屋里。”范进当是哄他,只装不听见,低着头往前走。邻居见他不理,走上来,就要夺他手里的鸡。范进道:“你夺我的鸡怎的?你又不买。”邻居道:“你中了举了,叫你家去打发报子哩。”范进道:“高邻\footnote{〔高邻〕对邻居的尊称。},你晓得我今日没有米,要卖这鸡去救命,为甚么拿这话来混我?我又不同你顽\footnote{〔顽〕同“玩”,嬉闹,玩笑。},你自回去罢,莫误了我卖鸡。”邻居见他不信,劈手把鸡夺了,掼在地下,一把拉了回来。报录人见了道:“好了,新贵人回来了。”正要拥着他说话,范进三两步走进屋里来,见中间报帖已经升挂起来,上写道:“捷报贵府老爷范讳高中广东乡试第七名亚元。京报连登黄甲。”
    
    范进不看便罢,看了一遍,又念一遍,自己把两手拍了一下,笑了一声,道:“噫!好了!我中了!”说着,往后一跤跌倒,牙关咬紧,不省人事。老太太慌了,慌将几口开水灌了过来。他爬将起来,又拍着手大笑道:“噫!好!我中了!”笑着,不由分说,就往门外飞跑,把报录人和邻居都吓了一跳。走出大门不多路,一脚踹在塘里,挣起来,头发都跌散了,两手黄泥,淋淋漓漓一身的水。众人拉他不住,拍着笑着,一直走到集上去了。众人大眼望小眼,一齐道:“原来新贵人欢喜疯了。”老太太哭道:“怎生这样苦命的事!中了一个甚么举人,就得了这个拙病\footnote{〔拙病〕倒霉的病。}!这一疯了,几时才得好?”娘子胡氏道:“早上好好出去,怎的就得了这样的病!却是如何是好?”众邻居劝道:“老太太不要心慌。我们而今且派两个人跟定了范老爷。这里众人家里拿些鸡蛋酒米,且管待了报子上的老爹们,再为商酌。”
    
    当下众邻居有拿鸡蛋来的,有拿白酒来的,也有背了斗米来的,也有捉两只鸡来的。娘子哭哭啼啼,在厨下收拾齐了,拿在草棚下。邻居又搬些桌凳,请报录的坐着吃酒,商议他这疯了,如何是好。报录的内中有一个人道:“在下倒有一个主意,不知可以行得行不得?”众人问:“如何主意?”那人道:“范老爷平日可有最怕的人?他只因欢喜狠了,痰涌上来,迷了心窍。如今只消他怕的这个人来打他一个嘴巴,说:‘这报录的话都是哄你,你并不曾中。’他吃这一吓,把痰吐了出来,就明白了。”众邻都拍手道:“这个主意好得紧,妙得紧!范老爷怕的,莫过于肉案子上胡老爹。好了!快寻胡老爹来。他想是还不知道,在集上卖肉哩。”又一个人道:“在集上卖肉,他倒好知道了;他从五更鼓就往东头集上迎猪,还不曾回来。快些迎着去寻他。”
    
    一个人飞奔去迎,走到半路,遇着胡屠户来,后面跟着一个烧汤的二汉\footnote{〔二汉〕佣工,伙计。},提着七八斤肉,四五千钱,正来贺喜。进门见了老太太,老太太大哭着告诉了一番。胡屠户诧异道:“难道这等没福?”外边人一片声请胡老爹说话。胡屠户把肉和钱交与女儿,走了出来。众人如此这般,同他商议。胡屠户作难道:“虽然是我女婿,如今却做了老爷,就是天上的星宿。天上的星宿是打不得的!我听得斋公们说:打了天上的星宿,阎王就要拿去打一百铁棍,发在十八层地狱,永不得翻身。我却是不敢做这样的事!”邻居内一个尖酸人说道:“罢么!胡老爹,你每日杀猪的营生,白刀子进去,红刀子出来,阎王也不知叫判官在簿子上记了你几千条铁棍;就是添上这一百棍,也打甚么要紧?只恐把铁棍子打完了,也算不到这笔帐上来。或者你救好了女婿的病,阎王叙功,从地狱里把你提上第十七层来,也不可知。”报录的人道:“不要只管讲笑话。胡老爹,这个事须是这般,你没奈何,权变一权变。”屠户被众人局不过\footnote{〔局不过〕碍于情面,虽然自己不愿意,也只好屈从。},只得连斟两碗酒喝了,壮一壮胆,把方才这些小心收起,将平日的凶恶样子拿出来,卷一卷那油晃晃的衣袖,走上集去。众邻居五六个都跟着走。老太太赶出来叫道:“亲家,你只可吓他一吓,却不要把他打伤了!”众邻居道:“这自然,何消\footnote{〔何消〕哪用得着。}吩咐。”说着,一直去了。
    
    来到集上,见范进正在一个庙门口站着,散着头发,满脸污泥,鞋都跑掉了一只,兀自\footnote{〔兀自〕只管。}拍着掌,口里叫道:“中了!中了!”胡屠户凶神似的走到跟前,说道:“该死的畜生!你中了甚么?”一个嘴巴打将去。众人和邻居见这模样,忍不住的笑。不想胡屠户虽然大着胆子打了一下,心里到底还是怕的,那手早颤起来,不敢打到第二下。范进因这一个嘴巴,却也打晕了,昏倒于地。众邻居一齐上前,替他抹胸口,捶背心,舞了半日,渐渐喘息过来,眼睛明亮,不疯了。众人扶起,借庙门口一个外科郎中的板凳上坐着。胡屠户站在一边,不觉那只手隐隐的疼将起来;自己看时,把个巴掌仰着,再也弯不过来。自己心里懊恼道:“果然天上‘文曲星’是打不得的,而今菩萨计较起来了。”想一想,更疼的狠了,连忙问郎中讨了个膏药贴着。
    
    范进看了众人,说道:“我怎么坐在这里?”又道:“我这半日,昏昏沉沉,如在梦里一般。”众邻居道:“老爷,恭喜高中了。适才欢喜的有些引动了痰,方才吐出几口痰来,好了。快请回家去打发报录人。”范进说道:“是了。我也记得是中的第七名。”范进一面自绾了头发,一面问郎中借了一盆水洗洗脸。一个邻居早把那一只鞋寻了来,替他穿上。见丈人在跟前,恐怕又要来骂。胡屠户上前道:“贤婿老爷,方才不是我敢大胆,是你老太太的主意,央我来劝你的。”邻居内一个人道:“胡老爹方才这个嘴巴打的亲切,少顷范老爷洗脸,还要洗下半盆猪油来!”又一个道:“老爹,你这手明日杀不得猪了。”胡屠户道:“我那里还杀猪!有我这贤婿,还怕后半世靠不着也怎的?我每常说,我的这个贤婿,才学又高,品貌又好,就是城里头那张府、周府这些老爷,也没有我女婿这样一个体面的相貌。你们不知道,得罪你们说,我小老这一双眼睛,却是认得人的。想着先年,我小女在家里长到三十多岁,多少有钱的富户要和我结亲,我自己觉得女儿像有些福气的,毕竟要嫁与个老爷,今日果然不错!”说罢,哈哈大笑。众人都笑起来。看着范进洗了脸,郎中又拿茶来吃了,一同回家。范举人先走,屠户和邻居跟在后面。屠户见女婿衣裳后襟滚皱了许多,一路低着头替他扯了几十回。
    
    到了家门,屠户高声叫道:“老爷回府了!”老太太迎着出来,见儿子不疯,喜从天降。众人问报录的,已是家里把屠户送来的几千钱打发他们去了。范进拜了母亲,也拜谢丈人。胡屠户再三不安道:“些须几个钱,不够你赏人。”范进又谢了邻居。正待坐下,早看见一个体面的管家,手里拿着一个大红全帖,飞跑了进来:“张老爷来拜新中的范老爷。”说毕,轿子已是到了门口。胡屠户忙躲进女儿房里,不敢出来。邻居各自散了。
    
    范进迎了出去,只见那张乡绅下了轿进来,头戴纱帽,身穿葵花色圆领,金带、皂靴。他是举人出身,做过一任知县的,别号静斋,同范进让了进来,到堂屋内平磕了头,分宾主坐下。张乡绅先攀谈道:“世先生同在桑梓,一向有失亲近。”范进道:“晚生久仰老先生,只是无缘,不曾拜会。”张乡绅道:“适才看见题名录,贵房师高要县汤公,就是先祖的门生,我和你是亲切的世弟兄。”范进道:“晚生侥幸,实是有愧。却幸得出老先生门下,可为欣喜。”张乡绅四面将眼睛望了一望,说道:“世先生果是清贫。”随在跟的家人手里拿过一封银子来,说道:“弟却也无以为敬,谨具贺仪五十两,世先生权且收着。这华居\footnote{〔华居〕对对方住宅的客气说法。}其实住不得,将来当事拜往,俱不甚便。弟有空房一所,就在东门大街上,三进三间,虽不轩敞,也还干净,就送与世先生;搬到那里去住,早晚也好请教些。”范进再三推辞,张乡绅急了,道:“你我年谊世好,就如至亲骨肉一般;若要如此,就是见外了。”范进方才把银子收下,作揖谢了。又说了一会,打躬作别。胡屠户直等他上了轿,才敢走出堂屋来。
    
    范进即将这银子交与浑家打开看,一封一封雪白的细丝锭子\footnote{〔细丝锭子〕铸有细条纹的银块。},即便包了两锭,叫胡屠户进来,递与他道:“方才费老爹的心,拿了五千钱来。这六两多银子,老爹拿了去。”屠户把银子攥在手里紧紧的,把拳头舒过来,道:“这个,你且收着。我原是贺你的,怎好又拿了回去?”范进道:“眼见得我这里还有这几两银子,若用完了,再来问老爹讨来用。”屠户连忙把拳头缩了回去,往腰里揣,口里说道:“也罢,你而今相与\footnote{〔相与〕结交。}了这个张老爷,何愁没有银子用?他家里的银子,说起来比皇帝家还多些哩!他家就是我卖肉的主顾,一年就是无事,肉也要用四五千斤,银子何足为奇!”又转回头来望着女儿,说道:“我早上拿了钱来,你那该死行瘟的\footnote{〔该死行瘟的〕该生瘟病死的。}兄弟还不肯,我说:‘姑老爷今非昔比,少不得有人把银子送上门来给他用,只怕姑老爷还不稀罕。’今日果不其然!如今拿了银子家去,骂这死砍头短命的奴才!”说了一会,千恩万谢,低着头,笑迷迷的去了。
    
    自此以后,果然有许多人来奉承他:有送田产的;有人送店房的;还有那些破落户,两口子来投身为仆,图荫庇的。到两三个月,范进家奴仆、丫鬟都有了,钱、米是不消说了。张乡绅家又来催着搬家。搬到新房子里,唱戏、摆酒、请客,一连三日。
    
\end{normalsize}


\newpage

\textbf{注释}:

\vspace{-1em}

\begin{itemize}
    \setlength\itemsep{-0.2em}
    \item 〔见教〕指教(我)。“见”字用在动词前面表示对“我”怎么样。
    \item 〔今非昔比〕多指形势、自然面貌等发生了巨大的变化。
    \item 〔带挈〕带领,照顾,提携。”
    \item 〔体统〕规矩。
    \item 〔央及〕恳求、请求。
    \item 〔盘费〕旅费。
    \item 〔浑家〕妻子。
    \item 〔尖酸〕说话尖刻。
    \item 〔平头百姓〕平民百姓。
    \item 〔劈手〕形容手的动作异常迅速。
    \item 〔万贯家私〕大量的家财。万贯:上万贯铜钱。贯:古时穿钱的绳子,既钱穿,也指一串钱,一千文为一串,称一贯。形容家产很多,非常富有。
    \item 〔舍与〕施舍给,赏给。
    \item 〔先年〕先前。
    \item 〔腆着〕挺着。
    \item 〔啐〕吐唾沫,表示唾弃、斥责或辱骂。
    \item 〔星宿〕我国古代天文学将天空分为二十八星宿。宿:居所。
    \item 〔桑梓〕家乡。
    \item 〔商酌〕反复仔细地商量。
    \item 〔些须〕很少。
    \item 〔轩敞〕宽敞。
    \item 〔果不其然〕多用来强调不出所料。
\end{itemize}

\chapter{包身工}

\begin{normalsize}
    
    已经是旧历四月中旬了,凌晨四点过一刻,拂晓的星星才从慢慢推移的淡云中消去,蜂房般的格子铺里,就有生物在蠕动了。
    
    “拆铺啦!起来!”穿着一身和时节不相称的拷绸\footnote{〔拷绸〕也叫莨绸、香云纱,指用薯莨液染的一种丝绸,往往用来制作夏天穿的衣服。}衫裤的男子,生气地呼喊,“芦柴棒,去烧火!妈的,还躺着,猪猡!”
    
    七尺阔、十二尺深的工房楼下,横七竖八地躺满了十六七个“猪猡”。随着威吓声,在充满了汗臭、粪臭和湿气的空气里面,她们很快就骚动起来,像被搅动了的蜂窝一般。打呵欠,叹气,寻衣服,穿错了别人的鞋子,胡乱地踏在别人身上,叫喊,在离别人的头不到一尺的马桶上很响地小便。成年女孩本该有的羞耻感,在这些被叫做“猪猡”的生物中间,已经很淡薄了。半裸体地起来开门,拎着裤子争夺马桶;身体稍稍背转一下,就是公然在男人面前换衣服。
    
    那男人虎虎地在起得慢一点的“猪猡”身上踢了几脚,回转身来站在不满二尺阔的楼梯上面,向着楼上的另一群生物呼喊:
    
    “揍你的!再不起来?懒虫!等太阳上山吗?”
    
    蓬头、赤脚,一边扣着纽扣,几个睡眼惺松的“懒虫”从楼上冲下来了。自来水龙头边挤满了人,用手捧些水来浇在脸上。“芦柴棒”急着要将大锅里的稀饭烧滚,倒冒出来的青烟让她好一阵猛咳。她约莫十五六岁,除了老板之外,大概很少人知道她的姓名。她的手脚瘦得像芦柴棒一样,于是大家就拿“芦柴棒”当做她的名字了。
    
    四点过一刻,鸽子笼一般的住房里,包身工起床,开始了一天非人的生活。
    
    这是杨树浦福临路东洋\footnote{〔东洋〕指日本。}纱厂的工房。一条水门汀\footnote{〔水门汀〕水泥,有时也用来指混凝土。}的弄堂\footnote{〔弄堂〕上海方言,居民区中的小巷。}路,把由红砖墙严密封锁着的长方形区域,划成狭长的两块。每边八排,每排五户,一共八十户一楼一底的房屋,分得像鸽子笼一般均匀。每间工房的楼上楼下,住着三十二三个“懒虫”和“猪猡”。所以,除了“带工”老板、老板娘、他们的家族亲戚和穿拷绸衣服的打杂、请愿警\footnote{〔请愿警〕由民间团体向警察局申请批准,在特定范围内执勤的警察。}之外,这墙圈里面住着两千多个衣服褴褛、替别人制造衣料的“猪猡”。
    
    但是,她们正式的名称却是“包身工”。她们的身体,已经以一种奇妙的方式包给了叫做“带工”的老板。每年特别是水灾、旱灾的时候,这些在东洋厂里有“脚路”\footnote{〔“脚路”〕指门路、关系。}的带工,就亲自或者派人到他们家乡或者灾荒的地方,用他们多年熟练的、能把稻草讲成金条的嘴巴,去游说那些不忍让儿女饿死的同乡。
    
    “还用说?住的是洋式的公司房子。吃的是鱼肉荤腥。一个月休息两天,咱们带着到马路上去玩耍。嘿,几十层楼的房子,两层楼的汽车,各种各样好看好用的洋货。老乡!人生一世,你也得去见识一下啊!——做满三年,以后赚的钱就归你啦。块把钱一天的工钱,嘿,别人给我叩头,也不替她写进去!咱们是同乡,有交情。——交给我带去,有什么三差二错,我还有脸回家乡吗?”
    
    这样说着,咬着草根树皮的女孩子自不必说,就是她们的父母,也只恨自己没这个福份了。于是,在预备好的“包身契”上画一个十字,包身费大洋二十元,期限三年。三年之内,由带工的供给住食,介绍工作,赚钱归带工者收用,生死疾病一听天命,先付包洋十元,人银两交,“恐后无凭,立此包身契据是实!”
    
    福临路工房的两千多个包身工,隶属在五十多个“带工”头手下,她们是顺从地替带工赚钱的“机器”。所以,每个“带工”带人的数量,也就表示了他们的手面\footnote{〔手面〕手上可用的现金。引申为交际的手段、手腕、排场。}和财产。少一点的,三十五十,多一点的带着一百五十个以上。手面宽一点的“带工”,不仅可以放债、买田、起屋,还能兼营茶楼、浴室、理发铺一类的买卖。
    
    东洋厂家将这些红砖墙围着的工房以每月五元的代价租给“带工”,“带工”就在这鸽子笼一般的“洋式”楼房里装进三十几部没有固定车脚\footnote{〔车脚〕大型机器接地的构件。车:泛指大件的机器。}的活动机器。这种工房没有普通弄堂房子的“前门”,它们的前门恰和普通房子的后门一样。每扇门楹上,一律钉着一块三寸长的木牌,上面用东洋笔法的汉字写着:“陈永田泰洲”“许富达维扬”等等带工头的籍贯和名字。门上,大大小小地贴着褪了色的红纸春联。中间,大都是红纸剪的元宝、如意、八卦,或者木版印的“姜太公在此,百无禁忌”的图像。春联的文字,大都是“积德前程远”“存仁后步宽”之类。这些春联贴在这种地方,好像是在对别人骄傲,又像是在对自己讽刺。
    
    四点半之后,没有线条和影子的晨光胆怯地显出来的时候,水门汀路上和弄堂里面,已被这些赤脚的乡下姑娘挤满了。凉爽而带一点湿气的晨风,大约就是这些生活在死水一般的空气里面的人们仅有的天惠。她们嘈杂起来,有的在公共自来水龙头边舀水,有的用断了齿的木梳梳掉执拗地粘在头发里的棉絮,陆续地两个一组地用扁担抬着平满的马桶,吆喝着从人们身边擦过。带工的老板或打杂的拿着一叠叠的“打印子簿子”,懒散地站在正门出口——好像火车站轧票处\footnote{〔轧票处〕检票处。轧:碾压。}一般的木栅子的前面。楼下的那些席子、破被之类收拾掉之后,晚上倒挂在墙壁上的两张饭桌放下来了。几十只碗,一把竹筷,胡乱地放在桌上,轮值烧稀饭的就将一洋铅\footnote{〔洋铅〕又叫白铁皮,指镀锌铁皮。当时只能从欧美进口。}桶浆糊一般的薄粥放在板桌中央。她们的定食是两粥一饭,早晚吃粥,中午的干饭由老板差人给她们送进工厂里去。粥!它的成分并不和一般通用的意义一样,里面是较少的籼米、锅焦、碎米和较多的乡下人用来喂猪的豆腐渣!粥菜?是不可能有的。有几个“慈祥”的老板到小菜场去收集一些莴苣的菜叶,用盐一浸,这就是她们难得的佳肴。
    
    只有两条板凳,——其实,即使有更多的板凳,这屋子里面也没有同时容纳三十个人吃粥的地方。她们一窝蜂地抢一般地盛了一碗,歪着头用舌舔着淋漓在碗边外的粥汁,就四散地蹲伏或者站立在路上和门口。添粥的机会除了特殊的日子——譬如老板、老板娘的生日,或者发工钱的日子——通常是很难有的。轮着揩地板、倒马桶的日子,也有连一碗也轮不到的时候。洋铅桶空了,轮不到盛第一碗的人们还捧着一只空碗,于是老板娘拿起铅桶到锅子里去刮一下锅焦、残粥,再到自来水龙头边去冲一些清水,用她那双才在梳头的油手搅拌一下,气哄哄地放在这些廉价的、不需要更多维持费的“机器”们面前。
    
    “死懒!躺着死不起来,活该!”
    
    十一年前内外棉的顾正红事件\footnote{〔顾正红事件〕指1925年上海日商内外棉纱厂发生罢工潮,中方工人代表顾正红被日本工头枪杀的事件。该事件成为“五卅”运动导火索。},尤其是五年前的“一·二八”战争\footnote{〔“一·二八”战争〕指1932年淞沪抗战,也叫“一·二八”事变。}之后,东洋厂对于这种特殊的廉价“机器”的需要突然地增加起来。据说,这是一种极合经济原理和经营原则的方法。有引号的机器,终究还是血肉之躯。所以当超过了“外头工人”忍耐的最大限度的时候,他们往往会自然地想起一种久已遗忘了的人类本就有的力量。有时候愚蠢的奴隶会领悟到一束箭折不断的道理。再消极一点,他们也还可以拼着饿死不干。一个有殖民地经验的“温情主义者”,在一本著作的序文上说:“在这次斗争中,警察没有任何的威权,在民众的结合力前面,什么权力都不中用了!”可是,结论呢?是温情主义吗?不,不!他们所采用的对策,只是用廉价而没有“结合力”的“包身工”来替代“外头工人”而已。
    
    第一,包身工的身体是属于带工老板的,所以她们根本就没有“做”或者“不做”的自由。她们每天的工资就是老板的利润,所以即使在生病的时候,老板也会很可靠地替厂家服务,用拳头、棍棒或者冷水来强制她们工作。就拿上面讲到过的芦柴棒为例吧,——其实,这种情况每个包身工都会遇到:有一次,在一个很冷的清晨,芦柴棒害了急性的重伤风而躺在“床”上了。她们躺的地方,到了一定的时间是非让出来做吃粥的地方不可的,可是在那一天,芦柴棒可真的挣扎不起来了,她很见机地将身体慢慢地移到屋子的角上,缩做一团,尽可能地不占地方。可是在这种工房里面,生病躺着休养的例子是不能任你开的,一个打杂的很快地走过来了。干这种职务的人,大半是带工头的亲戚,或者在地方上有一点势力的流氓。所以在这种法律的触手达不到的地方,他们几乎掌有生杀大权。芦柴棒的喉咙早已哑了,用手做着手势,表示身体没力,请求他的怜悯。
    
    “假病,老子给你医!”
    
    一手抓住了头发,狠命地往上一摔,芦柴棒手脚着地,像一只肢体上附有吸盘的乌贼。一脚踢在她的腿上,照例第二、第三脚是不会少的,可是打杂的很快就停止了——后来听说,是因为芦柴棒“露骨”地突出的腿骨,碰痛了他的足趾!打杂的恼了,顺手夺过一盆冷水,迎头泼在芦柴棒的头上。这是冬天,外面在刮寒风,芦柴棒遭了这意外的一泼,反射似的跳起身来,于是在门口刷牙的老板娘笑了:
    
    “瞧!还不是假病!好好地会爬起来,一盆冷水就医好了。”
    
    这只是无数例子中的一个。
    
    第二,包身工都是新从乡下出来,而且她们大半都是老板娘的乡邻,这一点,在“管理”上是极有利的条件。厂家在工房周围造一条围墙,门房里置一个请愿警和门外钉一块“工房重地,闲人莫入”的木牌,不但使这些“乡下小姑娘”和外界隔绝,更完全将管理权交给了带工的老板。这样,早晨五点钟由打杂的或者老板自己送进工厂,晚上六点钟接领回来,她们就永没有和外头人接触的机会。所以包身工是一种“罐装了的劳动力”,可以“安全地”保藏,自由地使用,绝没有因为和空气接触而起变化的危险。
    
    第三,那当然是工价的低廉。包身工由“带工”带进厂里,她们的集合名词又变了。在厂方那里,她们有“试验工”和“养成工”两种称号。试验工就表示“生手”,要养成为“熟手”。最初的钱是每天十二小时大洋一角至一角五分,最初的工作范围是不需要任何技术的扫地、开花衣、扛原棉、松花衣之类。一两个礼拜之后就调到钢丝车间、条子间、粗纱间去工作。在这种工厂所有者的本国,拆包间、弹花间、钢丝车间的工作,通例是男工做的,可是在半殖民地,不必顾虑到社会的纠缠和官厅的监督,就将这种不是女性所能担任的工作,加到工资不及男工三分之一的包身工们身上去了。
    
    五点钟,上工的汽笛声响了。红砖罐头的盖子——那一扇铁门一推开,就好像鸡鸭一般地无秩序地冲出一大群没有锁链的奴隶。每人手里都拿着一本打印子的簿子,不很讲话,即使讲话也没有什么生气。一出门,这人的河流就分开了,第一厂的朝东,二三五六厂的朝西;走不到一百步,她们就和另一种河流——同在东洋厂工作的“外头工人”们汇在一起。但是,住在这地域附近的人,很容易分得出,这河流里面的不同成分。外头工人的衣服多少地整洁一点,很多穿着旗袍,黄色或者淡蓝的橡皮鞋子;十七八岁的小姑娘们有时爱搽些粉,甚至也有人烫过头发。包身工就没有这种福气了。她们没有例外地穿着短衣,上面是褪色和油脏了的湖绿乃至莲青的短衫,下面是玄色或者条纹的裤子,长头发,很多还梳着辫子,破脏的粗布鞋,缠过未放大的脚,走路也就有点蹒跚的样子。在路上走,这两种人很少有谈话的机会。脏,乡下气,土头土脑,言语不通,这都是她们不亲近的原因,过分看高自己,不必要地看低别人,这种心理是在“外头工人”心里下意识地存在着的。她们想:我们比你们多一种自由,多一种权利——宁愿饿肚子的自由,随时可以调厂和不做的权利。
    
    红砖头的怪物,已经张着嘴巴在等待它的滋养物了。她们经过红头鬼把守着的铁门,在门房间交出准许她们贡献劳力的凭证。包身工只交一本打印子的簿子,外头工人在这簿子之外还有一张贴着照片的入厂凭证。这凭证,已经有十一年的历史了。顾正红事件以后,内外棉摇班\footnote{〔摇班〕方言,指罢工。}了,可其他的东洋厂还有一部分在工作,于是,在沪西的丰田厂\footnote{〔丰田厂〕丰田纺织厂,1921年在上海设立。},有许多内外棉的工人冒险混进去,做了一次里应外合的英勇的工作,从这时候起,由丰田提议,工人入厂之前就需要这种有照片的凭证。这种制度,是东洋厂所特有的。
    
    织成衣服的一缕缕纱,编成袜子的一根根线,穿在身上都是光滑舒适而愉快的。可是从原棉制成纱线的过程,就不像穿衣服那样的愉快了。纱厂工人终日面临着音响、尘埃和湿气三大威胁。
    
    到杨树浦去的电车经过齐齐哈尔路的时候,你就可以听到一种“沙沙”的急雨和“隆隆”的雷响混合在一起的声音。一进厂,猛烈的噪音,就会消灭——不,麻痹了你的听觉,马达的吼叫,皮带的拍击,锭子\footnote{〔锭子〕纺纱机的主要部件,把纱绕在筒管上成一定形状。}的转动,齿轮的轧轹……一切使人难受的声音,好像被压缩了,紧装在这红砖墙的厂房里面,分辨不出这是什么声音,也决没有使你听觉有分别这些音响的余裕。纺纱间里的“落纱”\footnote{〔落纱〕纺纱工序之一。这里指专管落纱的熟练工。}和“荡管”\footnote{〔荡管〕指巡回管理的上级女工,日本人叫做“见回”。},命令工人的时候,不用言语,不用手势,而用经常衔在嘴里的口哨,因为只有口哨锐厉的高音才能突破这紧张的空气。
    
    尘埃,那种使人难受的程度,更在意料之外了。精纺和粗纺间的空气中,肉眼也可看出飞扬着无数的“棉絮”。扫地的女工经常将扫帚的一端按在地上,像揩地板一样的推着,一个人在一条“弄堂”\footnote{〔“弄堂”〕指两部纺机中间的空隙。}中间反复地走着,细雪一般的棉絮依旧可以看出积在地上。弹花间、拆包间和钢丝车间更不必讲了。拆包间的工作,是将打成包捆的原棉拆开,用手扯松,拣去里面的夹杂成分;这种工作,现在的东洋厂差不多已经完全派给包身工去做了,因为她们“听话”,肯做别的工人不愿做的工作。在那种车间里,不论你穿什么衣服,一刻儿就会一律变成灰白。爱作弄人的小恶魔一般,在室中飞舞的花絮,“无孔不入”地钻进她们的五官。鼻孔、睫毛和每一个毛孔,都是这些纱花寄托的场所;要知道这些花絮粘在身上的感觉,那你可以假想一下——正像当你工作到出汗的时候,有人在你面前拆散一个木棉絮的枕芯,而使这枕芯的灰絮遍粘在你的身上!纱厂女工没有一个有健康的颜色,据调查,做十二小时的工,每人平均要吸入0.15克的棉絮!
    
    湿气的压迫,也是纱厂工人——尤其是织布间工人最大的威胁。她们每天过着黄霉\footnote{〔过着黄霉〕指处于充满黄霉的空气中。},每天接触着饱和着水蒸气的热气。按照棉纱的特性,张力和湿度是成正比例的。说得直白一点,棉纱在潮湿状态比较不容易扯断,所以车间里必须有喷雾的装置。在织布间,每部织机的头上就有一个不断地放射蒸气的喷口,伸手不见五指,对面不见他人!身上有一点被蚊虱咬开或者机器碰伤而破皮的时候,很快地就会引起溃烂。盛夏四十多摄氏度下工作的情景,那就决不是“外面人”所能想象的了。
    
    这大概是自然现象吧,一种生物在这三种威胁下面工作,更加容易疲劳。但是在做夜班的时候,打瞌睡是不会有的。因为野兽一般的铁的暴君监视着你,只要断了线不接,锭壳轧坏,皮辊摆错方向,乃至车板上有什么堆积,就有遭到“拿莫温”\footnote{〔“拿莫温”〕工头。旧时英国工厂工头的编号一般是1号,因此以英文中“1号”的谐音“拿莫温”称呼工头。}和“小荡管”毒骂和殴打的危险。这几年来,一般地讲,殴打的事情已经渐渐地少了,可是这种“幸福”只局限在外头工人身上。拿莫温和小荡管打人,很容易引起同车间工人的反对,即使当场不致发作,散工之后往往会有“喊朋友评理”和“打相打”\footnote{〔打相打〕上海方言,指“打架”。}的危险。但是,包身工是没有“朋友”和帮手的!什么人都可以欺侮,什么人都看不起她们,她们是最下层的一类人,是拿莫温和小荡管们发脾气、使威风的对象。在纱厂,活儿做得不好的罚规,大约是殴打、罚工钱和“停生意”三种。那么,带工老板的立场来看,后面的两种当然是很不利了。罚工钱就是减少他们的利润,停生意不仅不能赚钱,还要贴她二粥一饭,于是带工头不假思索地爱上了殴打这办法。每逢端午重阳、年头年尾,带工头总要对拿莫温们送礼,那时候他们总会谄媚地讲:
    
    “总得你帮忙,照应照应。咱的小姑娘有什么事情,尽管打,打死不干事,只要不是罚工钱停生意!”
    
    打死不干事,在这种情形之下,包身工当然是“人人得而欺之”了。有一次,一个叫做小福子的包身工整好了的烂纱没有装起,就遭了拿莫温的殴打,恰恰运气坏,一个“东洋婆”走过来了,拿莫温为着要在主子面前显出他的威风,和对东洋婆表示他管督的严厉,打得比平常格外着力。东洋婆望了一会儿,也许是她不喜欢这种不文明的殴打,也许是她要介绍一种更合理的惩戒方法,走近身来,揪住小福子的耳朵,将她扯到太平龙头\footnote{〔太平龙头〕消防用的水龙头。}前面,叫她向着墙壁立着;拿莫温跟着过来,很懂得东洋婆的意思似的,拿起一个丢在地上的皮带盘心子,不怀好意地叫她顶在头上。东洋婆会心地笑了:
    
    “这个小姑娘坏得很,懒惰!”
    
    拿莫温学着同样生硬的调子说:
    
    “这样她就打不成瞌睡了!”
    
    这种“文明”的惩罚,有时候会持续到两小时以上。两小时不做工作,赶不出一天该做的“生活”,那么工资减少又会招致带工老板的殴打,也就是分内的事了。殴打之外还有饿饭、吊起、关黑房间等等方法。
    
    实际上,拿莫温对待外头工人,也并不怎样客气,因为除了打骂之外,还有更巧妙的方法,譬如派给你难做的“生活”\footnote{〔“生活”〕指不熟悉的工作。},或者调你去做不愿意去做的工作。所以,外头工人里面的狡猾分子,就常常用送节礼巴结拿莫温的手段,来保障自己的安全。拿出血汗换的钱来孝敬工头,于她们当然是一种难堪的负担,但是对包身工来说,那是连这种送礼的权利也没有的!外头工人在抱怨这种额外的负担,而包身工却在羡慕这种可以自主地拿出钱来贿赂工头的权利!
    
    在特殊优惠的保护之下,吸收着廉价劳动力的滋养,中国的东洋厂飞跃地庞大了。单就这福临路的东洋厂讲,光绪28年\footnote{〔光绪28年〕即公元1902年。},三井系\footnote{〔三井系〕指三井财团。1902年起在上海收购多个纱厂。}的资本收买大纯纱厂而创立第一厂的时候,锭子还不到两万,可是三十年之后,他们已经有了六个纱厂,五个布厂,二十五万锭子,三千张布机,八千工人和一千二百万元的资本。美国一位作家索洛曾在一本书上说过,美国铁路的每一根枕木下面,都横卧着一个爱尔兰工人的尸骸\footnote{〔索洛……〕亨利·梭罗,19世纪美国作家。此句出自《瓦尔登湖》第2章。原句为:“你有没有想过美国铁路下的枕木?每一根都是一个人,爱尔兰人,或北方佬。”}。那么,我也这样联想,东洋厂的每一个锭子上面,都附托着一个中国奴隶的冤魂!
    
    两粥一饭,十二小时工作,劳动强化,工房和老板家庭的义务服役,猪猡一般的生活,泥土一般的作践——血肉造成的“机器”终究和钢铁造成的不一样,包身契上写明的三年期限,能够做满的不到三分之二。工作,工作,衰弱到不能走路还是工作,手脚像芦柴棒一般的瘦,身体像弓一样的弯,面色像死人一样的惨!咳着,喘着,淌着冷汗,还是被逼着在做工。譬如讲芦柴棒吧,她的身体实在瘦得太可怕了,放工的时候,厂门口的“抄身婆”\footnote{〔“抄身婆”〕指在下班出厂时对工人搜身,防止工人将值钱物件带出厂的人。}也不愿意去接触她的身体:
    
    “让她扎一两根油线绳吧!骷髅一样,摸着她的骨头会做噩梦!”
    
    但是带工老板是不怕做噩梦的!有人觉得太难看了,对她的老板说:
    
    “譬如做好事吧,放了她!”
    
    “放了她?行!还我二十块钱,两年间的伙食、房钱。”他随便地说,回转头来对她一瞪:
    
    “不还钱,可别做梦!宁愿赔棺材,要她做到死!”
    
    芦柴棒现在的工钱是每天三角八,拿去年的工钱三角二做平均,两年来在她身上已经收入了二百三十块了!
    
    还有一个,什么名字记不起了,她熬不住这种生活,用了许多工夫,在上午的十五分钟休息时间里,偷偷托一个在补习学校念书的外头工人写了一封给她父母的家信,邮票大概是那位同情她的女工捐助的了。一个月没有回信,她在焦灼,她在希望,也许,她的父亲会到上海来接她回去。可是,回信是捏在老板的手里了。散工回来的时候,老板和两个打杂的站在门口,横肉脸上在发火了,一把扭住她的头发,踢,打,掷,和爆炸一般的听不清的嚷骂:
    
    “死娼妓,你倒有本领,打断我的家乡路!”
    
    “猪猡,一天三餐将你喂昏了!”
    
    “揍死你,给大家做个榜样!”
    
    “信谁给你写的?讲,讲!”
    
    鲜血和惨叫使整个工房的人都怔住了,大家都在发抖,这好像真是一个榜样。打倦了之后,再在老板娘的亭子楼里吊了一晚。这一晚,整屋子除了快要断气的呻吟一般的呼喊之外,再没有别的声音。屏着气,睁着眼,百千个奴隶在黑夜中叹息她们的命运。
    
    人类的身体构造,有时候觉得确实有一点神奇。长得结实肥胖的往往会像折断一根麻梗一般的很快地死亡,而像芦柴棒一般的却偏能一天一天地磨难下去。每一分钟都有死的可能,可是她还有韧性地在那儿支撑。两粥一饭、十二小时噪音、尘埃和湿气中的工作,默默地,可是规则地反复着,直到榨完了残留在她皮骨里的最后一滴血汗为止。
    
    看着这种圈养小姑娘营利的制度,我禁不住想起小时候看到过的船户养墨鸭\footnote{〔墨鸭〕一种鱼鹰,被驯养来捕鱼。}捕鱼的事了。和乌鸦很相像的那种怪样子的墨鸭,整排地停在舷上,它们的脚是用绳子吊住了的,下水捕鱼,起水的时候船户就在它的颈子上轻轻一挤!吐了再捕,捕了再吐,墨鸭整天地捕鱼,卖鱼得钱的却是养墨鸭的船户。但是,从我们孩子的眼里看来,船户对墨鸭并没有怎样虐待,用船户养墨鸭捕鱼的事,比喻帝国主义及其买办\footnote{〔买办〕指为帝国主义在殖民地经营产业的本地中间人、经纪人。}们与包身工的剥削与被剥削的关系,十分精当,有力地控诉了吃人的包身工制度。而现在,将这种关系转移到人和人中间,便连这一点施与的温情也已经不存在了!
    
    在这千万被圈养者中间,没有光,没有热,没有温情,没有希望……没有法律,没有人道。这儿有的是20世纪的烂熟了的技术、机械、体制和对这种体制忠实服役的16世纪封建制度下的奴隶!
    
    黑夜,静寂的、死一般的长夜。表面上,这儿似乎还没有自觉,还没有团结,还没有反抗。她们住在一个伟大的锻冶场里面,闪烁的火花常常在她们身边擦过。可是,对这些被强压强榨着的生物来说,好像连那可以引火、可以燃烧的火种,也已经消散掉了。
    
    但是,黎明的到来,是无法抗拒的。索洛警告美国人当心枕木下的尸骸,我也想警告某一些人,当心那些锭子上呻吟着的冤魂!
    
\end{normalsize}


\newpage

\textbf{注释}:

\vspace{-1em}

\begin{itemize}
    \setlength\itemsep{-0.2em}
    \item 〔拂晓〕天即将亮的时候。
    \item 〔睡眼惺松〕刚睡醒眼睛迷糊不清的状态。
    \item 〔作践〕糟蹋,摧残。
    \item 〔天惠〕上天的恩惠。
    \item 〔譬如〕比如。
    \item 〔怜悯〕同情不幸的人、遭受苦难的人。
    \item 〔里应外合〕外部攻打时,里面接应,与外部配合。
    \item 〔照应〕照顾,配合。
    \item 〔轧轹〕滚压。
\end{itemize}

\chapter{求雨}

\begin{normalsize}
    
    金斗坪村的龙王庙,建筑在村北头河西边的高岸上。这岸的底部是村西边山脚下的崖石。据老人们说,要不是有这一段崖石,金斗坪早被大河冲得没有影了。
    
    在解放以前,每逢天旱了的时候,金斗坪的人便集中在这庙里求雨。求雨的组织,是把全村一百来户人家每八人编成一班,轮流跪祷。第一班焚上香之后,跪在地上等一炷香着完了,然后第二班接着焚香跪守……该不着上班的人,随便在一旁敲钟打鼓,希望引起龙王注意。这样周而复始地轮流着,直到下了雨为止。
    
    组织领导这事的人常是地主,在解放前不久是本村地主周伯元。周伯元怎样领导这事,只要引土地改革时候老贫农于天佑在斗争周伯元大会上说的一段话就可以明白。于天佑那段话是这样说的:“在求雨时候,你把你的名字排在第一班第一名,可是跪香时候你可以打发长工替你跪。别人误了跪香,按你立的规矩是罚一斤灯油,你的长工误了替你跪香,连罚的灯油也得他替你出。大家饿着肚子跪香,你屯着粮食不出放,反而只用一斗米一亩地的价钱买我们好地,求了十来次雨,就把金斗坪一半土地都买成你的了。有一次你和你亲家说:‘我这领导求雨不过是个样子,其实下不下都好——因为一半金斗坪都是我的,下了雨自然数我打的粮食多,不下雨我可以用一斗米一亩地的价钱慢慢把另一半也买过来。’你长的是什么心?要不是解放,那就只有你活的了……”
    
    土地改革后,金斗坪的全部土地又都回到农民手里,可是这午夏天不幸就又遇上了旱灾。这时候,政府号召开渠、打井、担水保苗,想尽一切方法和旱灾作斗争。金斗坪就在河边,开渠有条件,党支部书记于长水和村政委员会商量了开渠的计划,又请人测量了地势,就召开动员大会,动员开渠。
    
    因为这渠要经过龙王庙下边的石崖,估计至少得二十天。有人说:“要是二十天不下雨的话,渠还没有开成,苗早就晒干了!”于长水说:“只要把渠开成,苗干了还能再种晚粮;要是不开渠白等二十天,苗干了不是白干吗?只要我们大家有信心,我们就能克服灾荒。要是开成了这条渠,以后就再不会受旱灾的威胁了!”经过这一番加油打气,金斗坪的渠便开工了。
    
    不幸在动工这一天,又出了点小事:大家正在画好了石灰线的地方挖土,忽然听见龙王庙里敲钟打鼓。一听这个,大家就议论开了:“谁还这么封建?”“不要管他,咱们干咱们的!”“去叫他们停止了!不要让他们咚咚当当扰乱人心!”“叫人家求吧!能求得雨来不更好吗?”“开渠的开渠,求雨的求雨,谁也妨碍不了谁!”……各有各的主张。村长和党支部书记都去计划石工去了,党员们虽有自己的主张可是也说不服大家,最后都同意派个人去看看是些什么人在庙里,一个青年接受了这个使命。
    
    这青年跑到庙里一看,庙里有八个老头,最想不到的是土地改革时候的积极分子于天佑也在内。青年问于天佑:“你怎么也来了?”“我怎么不能来?”“你不是亲自说过龙王爷是被周伯元利用着发财的吗?”“那是周伯元坏,不是龙王爷不好!“原来你也是个老封建!”说了个“老封建”就把老头们惹恼了。有个老头是这青年的本家爷爷。他骂青年说:“你给我滚!不是你们得罪了龙王爷爷的话,早下雨了!你们长的是什么心肝,天旱得跟火熬一样还不让别人求雨!”这青年没法,只好回到河边去报告。晌午,党员把这情况反映到支部,支部书记于长水想出的对付办法是一方面说服他们,一方面加紧开渠——只要渠开成了,自然就没人求雨了。
    
    可是钟鼓不断地敲着,把一些心里还没有和龙王爷完全断绝关系的人又敲活动了:庙里又增加了好几个老头子,青壮年也有被家里老人们逼到庙里去的。庙里又定出轮班跪香制,参加开渠的人,凡是和龙王有点感情的,在上下工时候也绕到庙里磕个头。
    
    于长水一边发动党团员加紧挖土搬石头,一边帮着石匠钻炮眼崩石崖。土渠开得快,给人们增加了信心;石头崩得响,压倒了庙里的钟鼓。跪香的青壮年在不值班的时候,也溜出庙来参加开渠;老头们说他们心不诚,妨碍了求雨的效果。
    
    两天之后,开渠遇上了新困难:上半截土渠已经挖到庙下边的石崖边,可是石崖上的石头太硬,两天才崩了一排鸡窝窝。原来的估计不正确,光这一段五十尺长五尺深的石渠,一个月也开不过去。这时候退坡的,说闲话的慢慢多起来,也有装病的,也有说家里没吃的不能动的,也有不声不响走开不来的;剩下的人,有的说“一年也开不过去”,有的说:“现在旱得人心慌,还不如等到冬天再开”……原来在庙里跪香的仍回去跪香,原来只在上下工时候去磕个头的也正式编人跪香的班次。
    
    河边人少了,崩开了的石头没人搬,炮声暂且停下来。于长水一边仍叫党团员们搬着石头支持场面不让冷了场,一边脱了鞋,卷起裤管,过到河的对岸,坐在一块石头上,对着这讨厌的石崖想主意。这时候,田里的苗白白地干着,河里的水白白地流着,庙里的钟鼓无用地响着,他觉着实在不是个好滋味。他下了个决心说:“要不能把这么现成的水引到地里去,就算金斗坪没有党!”在火海一样的太阳下,他坐在几乎能烫焦了裤子的石头上,攒着眉头,两眼死盯在这段石崖上,好像想用他的眼光把这段石崖烧化了一样,大约有点把钟没有转眼睛,新办法就被他想出来了。他想要是从石崖离顶五尺高的腰里,凿上一排窟窿,钉上橛子,架上木槽,就可以把水接过去。他这样想着,好像已经看见有好几段连在一起的木槽横在这石崖的腰里,水从木槽里平平地流过去,就泻在村北头的平地上。他的眉头展开了。他站起来向对岸搬石头的人们喊:“同志们,不要搬了!有了好办法了!”说了就又过河来和大家商量。石匠对他的办法又加了点补充,说再把崖上钉了一排竖橛子,用铁绳把横橛子的外边那一头吊在竖橛子上管保成功。
    
    午上又开过群众大会通过了这个办法,退了坡的人听见有门道又都回来参加工作,党团员自然更加了劲,找木匠的、搬木头的、搭架的、拉锯的……七手八脚忙起来。
    
    庙里跪香的人又少了,气得于天佑拼命地敲钟。
    
    一天过去了,河边的木槽已经成形,庙里跪香的人偷跑了三分之二。
    
    两天过去了,木槽已经上了架,跪香的人,不但后来参加的全部退出,连原来的八个老头也少了三个。
    
    石崖腰里架木槽是个新玩意,全村男女老少都来看新鲜,吵嚷得比赶集还热闹。这声音,在庙里的五个老头听起来心里很不安,连钟鼓也无心敲了。于天佑说:“人们这样没有诚心,恐怕要惹得龙王爷一年也不给下雨!”其余四个老头撇了撇嘴,随后五个人商量了一下,一齐跪到地上祷告。于天佑说:“龙王爷呀!不论别人怎么样,我们几个的心是真诚的!求你老人家可怜可怜吧!”就在这时候,忽听得外边的人群像疯了一样齐声大喊起来,喊得比崩石崖的炮声还惊人。一个老头说:“这一定是出了什么事故了!”说了便爬起来跑出去,其余四个也都侧着耳朵听。
    
    出去的那个老头跑回来喊:“快去看!接过水来了!大着哩!”地上跪着的四个老头,除了于天佑也都爬起来要出去。于天佑说:“难道我们也不能诚心到底吗?”一个老头说:“抢水救苗要紧!龙王爷会原谅!”说着便都走出去。
    
    最后剩下于天佑。于天佑给龙王磕了个头说:“龙王爷!我也请你原谅!我房背后的二亩谷子也赶紧得浇一浇水了!”说罢也爬起来,跟着别的老头往外走去。
    
\end{normalsize}



\chapter{雨巷}

\begin{normalsize}
    
    \begin{verse}[0.5\linewidth]
        撑着油纸伞,独自 \\
        彷徨在悠长,悠长 \\
        又寂寥的雨巷, \\
        我希望逢着 \\
        一个丁香一样地 \\
        结着愁怨的姑娘。
    \end{verse}
    
    
    \begin{verse}[0.5\linewidth]
        她是有 \\
        丁香一样的颜色, \\
        丁香一样的芬芳, \\
        丁香一样的忧愁, \\
        在雨中哀怨, \\
        哀怨又彷徨;
    \end{verse}
    
    
    \begin{verse}[0.5\linewidth]
        她彷徨在这寂寥的雨巷, \\
        撑着油纸伞 \\
        像我一样, \\
        像我一样地 \\
        默默彳亍着, \\
        冷漠,凄清,又惆怅。
    \end{verse}
    
    
    \begin{verse}[0.5\linewidth]
        她静默地走近 \\
        走近,又投出 \\
        太息一般的眼光, \\
        她飘过 \\
        像梦一般地, \\
        像梦一般地凄惋迷茫。
    \end{verse}
    
    
    \begin{verse}[0.5\linewidth]
        像梦中飘过 \\
        一枝丁香地, \\
        我身旁飘过这女郎; \\
        她静默地远了,远了, \\
        到了颓圯的篱墙, \\
        走尽这雨巷。
    \end{verse}
    
    
    \begin{verse}[0.5\linewidth]
        在雨的哀曲里, \\
        消了她的颜色, \\
        散了她的芬芳, \\
        消散了,甚至她的 \\
        太息般的眼光, \\
        她丁香般的惆怅。
    \end{verse}
    
    
    \begin{verse}[0.5\linewidth]
        撑着油纸伞,独自 \\
        彷徨在悠长,悠长 \\
        又寂寥的雨巷, \\
        我希望飘过 \\
        一个丁香一样地 \\
        结着愁怨的姑娘。
    \end{verse}
    
\end{normalsize}



\chapter{麦琪的礼物}

\begin{normalsize}
    
    一元八角七。全都在这儿了。其中六十二分是一分一分的铜板。这些铜板是从杂货店、菜摊和肉店老板那儿一分两分地抠下来的。每个铜板背后都是一番软磨硬泡,锱铢必较,直到教人羞惭难当的地步。德拉反复数了三次,还是一元八角七,而第二天就是圣诞节\footnote{〔圣诞节〕庆祝耶稣诞生的节日,在阳历12月25日。}了。
    
    除了倒在那张破旧的小睡椅上大哭一场之外,显然没有别的办法了。于是德拉就这么办了。生活是由号哭、抽噎和微笑组成的,其中绝大多数是抽噎。
    
    在女主人从号哭慢慢转入抽噎的这段时间里,让我们看看这个家吧。一套带家具的公寓,每周房租八美元。尽管难以用笔墨形容,确实与贫民窟\footnote{〔贫民窟〕城市中贫困人口密集聚居的区域,居住条件恶劣,生活条件极差。}也相差无几了。
    
    楼下的门道里有个信箱,可从来没有装过信;还有一个门铃电钮,鬼才按的响;按钮旁边还有一张名片,写着“詹姆斯·狄林汉·杨先生”。
    
    “狄林汉”这个名号是主人先前手面\footnote{〔手面〕手上可用的现金。引申为排场、派头。}还阔绰的时候,一时兴起加上去的,那时候他每星期挣三十元。现在每周的收入缩减到二十元了,“狄林汉”的字母也显得模糊不清,似乎慎重考虑着,是否缩写成一个谦逊朴实的“狄”。不过,每当詹姆斯·狄林汉·杨回家,走进楼上的房间时,詹姆斯·狄林汉·杨太太,就是刚介绍给诸位的德拉,总是把他叫做“吉姆”,而且热烈地拥抱他。那当然是再好不过的了。
    
    德拉哭完了,小心地用破粉扑往面颊上抹了抹粉。她站在窗前,痴痴地瞅着灰蒙蒙的后院。后院里,一只灰色的猫在灰色的篱笆上走着。明天就是圣诞节,她只有一元八角七给吉姆买一份礼物。几个月来,她尽可能地省下每一分钱,结果也只不过如此。一周二十美元,实在经不起花,钱比她预计的更不经用了。只有一元八角七给吉姆买礼物。她心爱的吉姆啊!她耗费了多少幸福的时光,筹划着要送他一件可心的礼物,一件精致、珍奇、贵重的礼物——多少要有点儿配得上吉姆的东西才成啊。
    
    房间的两扇窗子之间有一面壁镜。也许你见过每周房租八美元的公寓壁镜吧。一个非常瘦小而灵巧的人,通过观察自己一连串的纵条映像,也许可以对自己的容貌得到一个大致不错的概念。德拉全靠身材纤细,才精通了这门艺术。
    
    骤然间,她从窗口旋风般地转过身来,站到壁镜前。她两眼晶莹透亮,但不到二十秒钟,她的脸上就失去了光彩。她很快地把头发解开,披落下来。
    
    詹姆斯·狄林汉·杨夫妇俩各有一件特别引以自豪的东西。一件是吉姆的金表,是他祖父传给父亲,父亲又传给他的传家宝;另一件则是德拉的秀发。如果示巴女王\footnote{〔示巴女王〕古代埃塞俄比亚传说中的统治者,曾带着香料、宝石和金子前往耶路撒冷拜见所罗门王,寻求智慧。}也住在天井对面的公寓里,总有一天德拉会把头发披散,悬在窗外晾干,好教那女王的珍宝首饰黯然失色;如果所罗门王\footnote{〔所罗门王〕古代犹太民族传说中的统治者,被神赐予无尽的财富和无穷的智慧。}把公寓的地下室用来存放金银财宝,每当吉姆路过那儿,准会摸出金表,好让那所罗门王忌妒得吹胡子瞪眼睛。
    
    此时此刻,德拉的秀发泼撒在身上,波浪起伏,光芒闪耀,仿佛一股褐色的瀑布。她的美发一直垂到膝下,仿佛一件长袍。接着,她又神经质\footnote{〔神经质〕心理学中的一种人格特质,感受力强,更容易产生激烈的情绪,并为情绪左右。这里指像神经质的人一样。}地赶紧把头发梳好。她踌躇了一分钟,静静地立在那儿,几滴眼泪落在破旧的红地毯上。
    
    她穿上那件褐色的旧外衣,戴上褐色的旧帽子,眼睛里残留着晶莹的泪花,裙子一摆,便飘出房门,下楼来到街上。
    
    她走到一块招牌前停下来,招牌上写着:“娑芙罗妮夫人——专营各式头发用品”。德拉奔上楼梯,喘着气,定了定神。娑芙罗妮夫人身躯肥大,过于苍白,冷若冰霜,和她雅致的名字完全不相称。
    
    “你要买我的头发吗?”德拉问。
    
    “我买头发,”夫人说。“揭掉帽子,让我看看头发的样子。”
    
    那褐色的瀑布倾泻下来。
    
    “二十美元。”夫人熟练地抓起头发。
    
    “快给我钱。”
    
    嗬!接着而至的两个小时犹如长了玫瑰色的翅膀,愉快地飞掠而过——请不用理会这杂凑的比喻。她正在彻底搜寻各家店铺,为吉姆买礼物。
    
    她终于找到了,那准是专为吉姆特制的,决非为别人。她找遍了各家商店,哪儿也没有这样的东西,一条朴素的白金表链,式样简单朴素。正如真正的上等货那样,只以材质,而不以庸俗的雕饰来宣示自己的价值。它正配得上那只金表。她一见这条表链,就知道它应该为吉姆所有。它就像吉姆一样,文静而高尚。她用二十一美元买下了表链,匆匆赶回家,只剩下八角七分钱。配上这条链子,在任何场合,吉姆都可以毫无顾忌地掏出表看钟点了。那只表虽然精美,可因为他只用一根旧皮条当作表链,他很少看表,至多偷偷拿出来瞥上一眼。
    
    回到家里,她的狂喜逐渐让位于审慎和理智。她找出烫发铁钳,点燃煤气,着手修补因爱情加慷慨所造成的破坏,这自然是件艰巨的任务,亲爱的朋友们——简直是件了不起的任务呵。
    
    不出四十分钟,她的头上满是紧贴头皮的一绺绺小发卷,教她看起来活像个逃学的小男孩。她仔细而苛刻地对着镜子照了又照。
    
    “吉姆一看到我,不把我杀死才怪呢!”她自言自语,“他一定会说我像科尼岛游戏场里的卖唱姑娘。但是我能怎么办呢——唉,只有一元八角七,我能有什么办法呢?”
    
    七点钟,她煮好了咖啡,把煎锅放到炉子上,随时准备煎肉排。
    
    吉姆一贯准时回家。德拉把表链对叠,握在手心里,坐在他进门必经之处,靠门的桌子角上。接着,她听见下面楼梯上响起了他的脚步声,她紧张得脸发白了。她习惯于为了最简单的日常事物默默祈祷,此刻,她悄声道:“天主保佑,让他觉得我还是漂亮的吧。”
    
    门开了,吉姆走进来,随手关上了门。他显得瘦削而又非常严肃。可怜的人儿,他才二十二岁,就挑起了家庭的重担!他的大衣该换了,手套也没有。
    
    吉姆一进门就站住了,好像猎犬嗅到了鹌鹑的气味似的纹丝不动。他的目光定在德拉身上,带着一种使她无法理解的神情,使她大为惊慌。那既不是愤怒,也不是惊讶,又不是不满,更不是嫌恶,不是她所预料的任何一种神情。他只是用这种神情死死地盯着她。
    
    德拉一扭腰,从桌上跳了下来,向他走过去。
    
    “吉姆,亲爱的,”她喊道,“别那样盯着我。我把头发剪掉卖了,因为不送你一件礼物,我无法过圣诞节。头发会再长起来的——你不会介意的,是吗?我非这么做不可。我的头发长得快极了。说‘圣诞快乐’吧!吉姆,让我们开开心心的。你肯定猜不着我给你买了一件多么好——多么美丽精致的礼物啊!”
    
    “你把头发剪掉了?”吉姆吃力地问道,似乎他绞尽脑汁也没把这显然的事实弄明白。
    
    “剪掉了,卖掉了,”德拉说。“不管怎么说,你还是那么爱我——不是吗?没了长发,我还是我——不是吗?”
    
    吉姆古怪地四下望望这房间。
    
    “你说你的头发没有了吗?”他带着近似白痴的神情问道。
    
    “别找啦,”德拉说。“告诉你,我已经卖了——卖掉了,没有啦。这是圣诞夜\footnote{〔圣诞夜〕圣诞节的前夜。},亲爱的。好好待我,这是为了你呀。也许我的头发数得清,”突然她特别温柔地接下去,“可谁也数不清我对你的爱。我把肉排烧上,好吗,吉姆?”
    
    吉姆好像忽然从恍惚之中醒来了,把德拉紧紧地搂在怀里。现在,别着急,先让我们花十秒钟,从另一个角度好好想一下某些无关紧要的东西吧。每周八美元的房子,或者一百万美元——有什么差别呢?数学家或耍嘴皮子的也许会给你错误的答案。麦琪\footnote{〔麦琪〕基督教《新约》中记载耶稣诞生时有三个智者从东方来敬拜,并赠送黄金、乳香、没药作为礼物。}带来了珍贵的礼物,但它不是其中任何一个。这句晦涩的话,下文自有分说。
    
    吉姆从大衣口袋里掏出一个小包,扔在桌上。
    
    “别对我产生误会,德儿\footnote{〔德儿〕詹姆斯对德拉的昵称。},”他说道,“无论剪发、修脸,还是洗头,我对我的姑娘的爱不会减低一分。不过,你只要打开那包东西,就会明白为什么刚才我愣住了。”
    
    白皙的手指灵巧地解开绳子,打开包装纸。紧接着是欣喜若狂的尖叫。紧接着,哎呀!突然变成了女性神经质的泪水和号哭,需要男主人立马尽心安慰。
    
    因为摆在桌上是一整套梳子,包括两鬓用的,后面用的,样样俱全。很久以前,德拉在百老汇\footnote{〔百老汇〕百老汇大街是纽约市的文化和娱乐中心,有著名的剧院区和时代广场。}的一个橱窗里见过,羡慕得要死。这套美妙的发梳是纯玳瑁\footnote{〔玳瑁〕一种海龟,其甲壳用于制作饰品,十分珍贵。}做的,边上镶着珠宝——这色彩这光泽,正好同她失去的美发相匹配。她知道这套梳子有多昂贵。她心中自然神往已久,但从不抱拥有它的希望。现在,这心仪多时的饰物居然为她所有了,可惜那头美丽长发,它本应装饰的长发,已经不在了。
    
    不过,她还是把发梳紧紧抱在怀里,隔了好久,才抬起泪水迷蒙的双眼,微笑着说:“我的头发长得飞快,吉姆!”
    
    随后,德拉像一只被烫到的小猫似的跳起来,叫道,“噢!噢!”
    
    吉姆还没看她送给他的礼物哩。她急不可耐地把手掌摊开,伸到他面前,那无知无觉的贵重金属似乎反映着她的欢快和热忱。
    
    “漂亮吗,吉姆?我搜遍了全城才找到了它。现在,你每天可以把表看上一百次了。把表给我,我要看看它配在表上的样子。”
    
    吉姆并没有照她的话去做,反而倒在睡椅上,头枕在双手上,微微发笑。
    
    “德儿,”他说,“让我们暂且把圣诞礼物先放在一边吧。我们现在还用不上。我把表卖了,用卖表的钱为你买了发梳。现在把肉排烧上吧。”
    
    各位读者也知道,麦琪是有智慧的人——非常有智慧的人,他们从东方带来珍贵的礼物,送给刚诞生在马槽\footnote{〔马槽〕基督教传说中耶稣诞生在马厩中的马槽里。}里的耶稣。这就是圣诞礼物的由来。麦琪是有智慧的人,毫无疑问,他们的礼物也是智慧的礼物。即便两件礼物一模一样,也必然能相互交换。我的拙笔在此向你们描述了一个平平无奇的故事。住在一间公寓里的两个傻孩子,愚蠢地为了对方放弃了家里最珍贵的宝物。不过,最后我还是想告诉现今的聪明人:这两个人,在所有互赠礼物的人里,是最有智慧的。送礼的是最有智慧的,收礼的也是。天下没有比他俩更有智慧的人了。他们就是麦琪。
    
\end{normalsize}


\newpage

\textbf{注释}:

\vspace{-1em}

\begin{itemize}
    \setlength\itemsep{-0.2em}
    \item 〔锱铢必较〕为很少的钱或很小的事计较,不肯退让。锱、铢:古代重量单位,一斤等于十六两,一两等于四锱,一锱等于六铢。
    \item 〔羞惭难当〕因为羞愧而难为情,难以面对别人。
    \item 〔抽噎〕一吸一顿、呼吸断断续续地哭泣。
    \item 〔阔绰〕有钱,能花钱。
    \item 〔晦涩〕文辞隐晦,不流畅,难懂。
    \item 〔心仪〕心中向往。仪:向往。
    \item 〔拙笔〕谦称自己的文字或书画。
    \item 〔黯然失色〕指相比之下,事物仿佛失去原有的色泽和光彩。
    \item 〔骤然〕突然,极短时间。
\end{itemize}

\end{document}
