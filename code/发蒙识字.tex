\documentclass[12pt,UTF-8,openany]{ctexbook}
\usepackage{ctex}
\usepackage{titlesec}
\usepackage{xeCJK}
\usepackage{verse}
\usepackage{fontspec,xunicode,xltxtra}
\usepackage{xpinyin}
\usepackage{hanzibox}
\usepackage{geometry}
\usepackage{indentfirst}
\usepackage{pifont}
\usepackage{enumitem}
\usepackage[perpage,symbol*]{footmisc}
\usepackage[table,dvipsnames]{xcolor}

\geometry{a5paper,left=1.4cm,right=1.4cm,top=2.3cm,bottom=2.3cm}
\renewcommand{\footnotesize}{\fontsize{8.5pt}{10.5pt}\selectfont}
\setmainfont{Mona Sans Light}
\setCJKmainfont[BoldFont=STZhongsong]{汉字之美仿宋GBK 免费}
\xeCJKDeclareCharClass{CJK}{`0 -> `9}
\xeCJKsetup{AllowBreakBetweenPuncts=true}
\DefineFNsymbols{circled}{{\ding{192}}{\ding{193}}{\ding{194}}{\ding{195}}{\ding{196}}{\ding{197}}{\ding{198}}{\ding{199}}{\ding{200}}{\ding{201}}}
\setfnsymbol{circled}
\xpinyinsetup{ratio=0.5,hsep={.6em plus .6em},vsep={1em}}
\hanziboxset{frametype=咪,framelinewidth=0.5pt,width=0.9cm,resize=real,pinyinline=true,framecolor=red,charf={\kaishu\huge},pinyinf=\scriptsize,pinyincolor=green!30!black,charcolor=green!30!black}
\definecolor{script-1-0}{RGB}{28,120,180}
\definecolor{script-1-1}{RGB}{70,170,40}

\titleformat{\chapter}{\zihao{-1}\bfseries}{ }{16pt}{}
\titleformat{\section}{\zihao{-2}\bfseries}{ }{0pt}{}
\title{\zihao{0} \bfseries 发蒙识字}
\setlength{\lineskip}{24pt}
\setlength{\parskip}{6pt}
\author{}
\date{}
\begin{document}
\maketitle
\tableofcontents
\newpage

\chapter{我是中国人}

\begin{large}
    
    我上学了
    
    我学中文
    
    我是中国人
    
    我爱中国
    
\end{large}





\chapter{一二三四五}

\begin{large}
    
    \begin{verse}[0.5\linewidth]
        一二三四五 \\
        金木水火土 \\
        天地分上下 \\
        日月见今古
    \end{verse}
    
\end{large}





\chapter{人}

\begin{large}
    
    \begin{verse}[0.5\linewidth]
        头面身手足 \\
        口牙目耳心
    \end{verse}
    
\end{large}





\chapter{田}

\begin{large}
    
    \begin{verse}[0.5\linewidth]
        山川风云雨 \\
        田力禾苗实
    \end{verse}
    
\end{large}





\chapter{比大小}

\begin{large}
    
    一二三四五
    
    六七八九十
    
    六比三大,二比五小
    
    九比七多,一比四少
    
\end{large}





\chapter{开门}

\begin{large}
    
    有人吗?有人吗?开门!开门!
    
    爸爸不在,妈妈不在。
    
    只有我一个人。我不开门。
    
\end{large}





\chapter{山村}

\begin{large}
    
    \begin{verse}[0.5\linewidth]
        山下青石路,田中牛马肥。 \\
        只闻鸡犬叫,不见农人归。
    \end{verse}
    
\end{large}





\chapter{你我他}

\begin{large}
    
    我是学生,你也是学生,他不是学生。
    
    我们是中国人,你们也是中国人。
    
    他们不是中国人。
    
\end{large}





\chapter{田鸟}

\begin{large}
    
    \begin{verse}[0.5\linewidth]
        田里黄黄麦,飞来小小鸟。 \\
        风吹麦点头,小鸟飞去了。
    \end{verse}
    
\end{large}





\chapter{鸟鱼虫}

\begin{large}
    
    小鸟有爪子,小鱼有尾巴。
    
    小鱼没有爪子,小鸟有尾巴。
    
    虫子没有爪子,虫子没有尾巴。
    
    小鸟有羽毛,小鱼没有毛。
    
    小鸟吃虫子,小鱼不吃虫子。
    
\end{large}





\chapter{山羊}

\begin{large}
    
    \begin{verse}[0.5\linewidth]
        大羊大,小羊小, \\
        山上山下吃青草。 \\
        山下牛羊多, \\
        山上牛羊少。 \\
        比一比,谁吃的多,谁吃的少。
    \end{verse}
    
    
    \begin{verse}[0.5\linewidth]
        大羊跳,小羊跑, \\
        山上山下吃青草。 \\
        大牛前面走, \\
        小牛后边叫。 \\
        比一比,谁吃的多,谁吃的少。
    \end{verse}
    
\end{large}





\chapter{方向}

\begin{large}
    
    今天星期一,一早上学去。
    
    东方有太阳,面向太阳走。
    
    前面是东,后面是西。
    
    左边是北,右边是南。
    
\end{large}





\chapter{我有一个家}

\begin{large}
    
    我有一个家,爸爸和妈妈。
    
    我是儿子。我是女儿。
    
    我是男的。我是女的。
    
    我是哥哥,我有弟弟。
    
    我是弟弟,我有哥哥。
    
    我有姐姐,我是妹妹。
    
    我有妹妹,我是姐姐。
    
    我爱我家。
    
\end{large}





\chapter{看地图}

\begin{large}
    
    上北下南,左西右东。
    
    上面下面,前方后方。
    
    上来下去,左入右出。
    
    向左走,向右走,向前走,向后走。
    
    向上走,向下走,向里走,向外走。
    
    向北走,向南走,向东走,向西走。
    
\end{large}





\chapter{工农兵}

\begin{large}
    
    工人力量大。
    
    农民力量大。
    
    士兵力量大。
    
    好好学习,天天向上。
    
    团结起来,保卫国家。
    
    中国人民大团结,世界人民大团结。
    
\end{large}





\chapter{张开口}

\begin{large}
    
    张开口,看一看。你有几颗牙?
    
    上面有牙,下面有牙。
    
    舌头在中间,嘴唇在外面。
    
\end{large}





\chapter{这是什么}

\begin{large}
    
    这是什么?那是什么?
    
    这是一头羊,那是一只鸟。
    
    这羊有多大?那鸟有多小?
    
    这羊有这么大。那鸟有这么小。
    
    这里有多少头羊?那里有多少只鸟?
    
    这里有三头羊,那儿有一只鸟。
    
    这儿有点多,那儿有点少。
    
\end{large}





\chapter{你吃什么}

\begin{large}
    
    小白兔,小黄鸡,一个吃草,一个捉虫。
    
    小白兔不捉虫,小黄鸡不吃草。
    
    小猴子,小花猫,一个吃桃,一个捉鱼。
    
    小猴子不捉鱼,小花猫不吃桃。
    
\end{large}





\chapter{谁比我高}

\begin{large}
    
    牛比马高,马比羊高,羊比狗高,狗比鸡高。
    
    鸡飞到屋子上说:“谁比我高?”
    
\end{large}





\chapter{时间}

\begin{large}
    
    一年有四季。一年有十二个月。
    
    一天有二十四小时。一小时有六十分,一分有六十秒。
    
    现在是几点?现在是九点四十分。
    
    他们几点到?他们十点到。
    
    现在几点了?现在是十点半。
    
    他们到了吗?他们还没到。
    
\end{large}





\chapter{问路}

\begin{large}
    
    \begin{description}[itemsep=1ex,leftmargin=2.5em,labelwidth=2em]
    
    \item[{\color{script-1-1} 路人}]你好!请问去东大门怎么走?
    
    \item[{\color{script-1-0} 我}]先往前走,过了那个路口往右,下一个路口再往左,就是东大门了。
    
    \item[{\color{script-1-1} 路人}]谢谢!
    
    \item[{\color{script-1-0} 我}]不客气!
    
    \noindent ~
    
    \item[{\color{script-1-1} 路人}]你好!请问到南方公园怎么走?
    
    \item[{\color{script-1-0} 我}]往东直走三个路口,再往左走一个路口,再往右一直走就到了。
    
    \item[{\color{script-1-1} 路人}]谢谢!
    
    \item[{\color{script-1-0} 我}]不客气!
    
    \end{description}
    
    
\end{large}





\chapter{过马路}

\begin{large}
    
    过马路,左右看。不在路上跑和玩。
    
    人行道,有白线。红绿灯,在前面。
    
\end{large}





\chapter{你叫什么名字}

\begin{large}
    
    你好!请问你叫什么名字?
    
    我叫王二小。
    
    张刘吴李,王赵陈杨。
    
    你姓王,名二小。我姓李,名四光。
    
\end{large}





\chapter{画}

\begin{large}
    
    \begin{verse}[0.5\linewidth]
        远看山有色,近听水无声。 \\
        春去花还在,人来鸟不惊。
    \end{verse}
    
\end{large}





\chapter{下雨啦}

\begin{large}
    
    下雨啦,下雨啦。
    
    种子说:“下吧,下吧,我要发芽。”
    
    禾苗说:“下吧,下吧,我要长大。”
    
    果树说:“下吧,下吧,我要开花。”
    
\end{large}





\chapter{小猫读书}

\begin{large}
    
    \begin{verse}[0.5\linewidth]
        小猫你在干什么?怎么把书打开了? \\
        小猫你来读一读,这页书在说什么? \\
        小猫只会喵喵叫,一脚把书合上了。
    \end{verse}
    
\end{large}





\chapter{今天天气好}

\begin{large}
    
    \begin{verse}[0.5\linewidth]
        今天天气好,太阳当空照。 \\
        白云两三朵,还有几只鸟。 \\
        背上小书包,我去上学了。
    \end{verse}
    
\end{large}





\chapter{车子吃什么油}

\begin{large}
    
    你家的车子吃什么油?
    
    我家的车子吃石油。
    
    车子不吃石油。你家的车子吃什么油?
    
    我家的车子吃花生油。
    
    车子不吃花生油。你家的车子吃什么油?
    
    我家的车子吃汽油。你家的车子吃什么油?
    
    我家的车子不用吃油。我家开的是电车!
    
\end{large}





\chapter{小舟}

\begin{large}
    
    弯弯的月儿小小的舟,
    
    小小的舟儿两头尖。
    
    我在小小的舟中坐,
    
    只看见闪闪的星星蓝蓝的天。
    
\end{large}





\chapter{我有一张床}

\begin{large}
    
    我有一张床,床上有席子,有被子,还有一个大枕头。
    
    床头有一个小台子,台子上有台灯。床边有一个大柜子,柜子里有衣服。
    
\end{large}





\chapter{我的笔盒}

\begin{large}
    
    我的笔盒里面有四支笔。
    
    一支是铅笔,一只是钢笔。
    
    一支是黑笔,一支是红笔。
    
    笔盒里还有直尺和橡皮。
    
\end{large}





\chapter{找朋友}

\begin{large}
    
    找呀找呀找朋友,找到一个好朋友。
    
    笑一笑,点点头。敬个礼,握握手。
    
    你是我的好朋友。
    
\end{large}





\chapter{你开心吗?}

\begin{large}
    
    小鸟开心叽咕叫,小猫开心上下跳。
    
    小狗开心尾巴摇,小朋友开心拍手笑。
    
    你要是觉得开心了,张开嘴巴笑一笑。
    
\end{large}





\chapter{切西瓜}

\begin{large}
    
    \begin{verse}[0.5\linewidth]
        拿起大长刀,我要切西瓜。 \\
        一刀分两半,两刀成四块。 \\
        切成一片片,请你吃西瓜。
    \end{verse}
    
\end{large}





\chapter{丢手绢}

\begin{large}
    
    丢手绢,丢手绢,
    
    轻轻地放在小朋友的后面,
    
    大家不要告诉他。
    
    快点快点捉住他,
    
    快点快点捉住他。
    
\end{large}





\chapter{秋天}

\begin{large}
    
    天气凉了,树叶黄了,一片片叶子从树上落下来。
    
    天空那么蓝,那么高。一群大雁往南飞,一会儿排成个“人”字,一会儿排成个“一”字。
    
    啊!秋天来了。
    
\end{large}





\chapter{画彩虹}

\begin{large}
    
    张开大画布,
    
    打开颜料盒,
    
    我想画彩虹。
    
    天上彩虹有七色,
    
    红橙黄绿青蓝紫。
    
    红加黄得到橙,
    
    黄加蓝得到绿,
    
    蓝加绿得到青,
    
    红加蓝得到紫。
    
\end{large}



\chapter{风筝}

\begin{large}
    
    竹做的骨头纸做的背,
    
    清风送它们往天上飞。
    
    我们在地上边笑边跑,
    
    它们在天上越飞越高。
    
\end{large}





\chapter{大桥}

\begin{large}
    
    江上有一座大桥。
    
    汽车火车在桥上跑。
    
    轮船从桥下过。
    
    以前,江上没有大桥。
    
    车子开到江边,
    
    要下车再坐船过江。
    
    自从有了大桥,
    
    汽车火车就能过江了。
    
\end{large}





\chapter{中秋}

\begin{large}
    
    月团圆来人团圆,中秋时节月最明。
    
    走到巷头看花灯,坐在月下吃月饼。
    
\end{large}





\chapter{拍皮球}

\begin{large}
    
    公园里,草地多,找一片草地拍皮球。
    
    你来拍,我来数。一二三,一二三,一二三四五。
    
    夏天太阳大,玩出一身汗。拿出大毛巾,喝口冰汽水。
    
\end{large}





\chapter{你知不知道}

\begin{large}
    
    你知不知道,地有多广?
    
    你知不知道,天有多高?
    
    你知不知道,海有多深?
    
    你知不知道,天上的星星有多少?
    
    海水为什么蓝?花儿为什么香?
    
    老虎为什么吃肉?马儿为什么吃草?
    
    大自然里学问多,一起来把答案找。
    
\end{large}





\chapter{雪地里的小画家}

\begin{large}
    
    下雪啦,下雪啦!
    
    雪地里来了一群小画家。
    
    小鸡画竹叶,小狗画梅花。
    
    小鸭画枫叶,小马画月牙。
    
    不用颜料不用笔, 几步就成一副画。
    
    青蛙为什么没参加?它在洞里睡着啦。
    
\end{large}





\chapter{二月二}

\begin{large}
    
    二月二,龙抬头,大家小户使耕牛。
    
    春雷一声惊万物,细雨绵绵似美酒。
    
\end{large}





\chapter{菜市场}

\begin{large}
    
    买菜了,买菜了!菜市场里有什么?
    
    青菜,花菜,白菜,还有空心菜。
    
    黄瓜,冬瓜,丝瓜,还有大南瓜。
    
    土豆,大豆,扁豆,还有四季豆。
    
    小葱萝卜西红柿,生姜茄子西兰花。
    
    你那儿的市场里有什么菜?
    
\end{large}





\chapter{大扫除}

\begin{large}
    
    大扫除,开始了!人人动手来干活。
    
    我用扫把扫地板,你用抹布抹窗户。
    
    大家一齐来拖地,我拿拖把,你提水桶。
    
    屋子里外亮堂堂,干干净净真快乐。
    
\end{large}





\chapter{东海龙宫}

\begin{large}
    
    村头树下一口井,井口看去黑乌乌。
    
    水井下面有什么?井水流到东海出。
    
    东海里有水晶宫,水晶宫里龙王住。
    
    龙王手里宝贝多,红花玉树夜明珠。
    
\end{large}





\chapter{江南}

\begin{large}
    
    \begin{verse}[0.5\linewidth]
        江南可采莲,莲叶何田田。 \\
        鱼戏莲叶间。 \\
        鱼戏莲叶东,鱼戏莲叶西, \\
        鱼戏莲叶南,鱼戏莲叶北。
    \end{verse}
    
\end{large}





\chapter{鹅}

\begin{large}
    
    \begin{verse}[0.5\linewidth]
        \phantom{鹅}鹅鹅鹅\phantom{鹅},曲项向天歌。 \\
        白毛浮绿水,红掌拨清波。
    \end{verse}
    
\end{large}





\chapter{春天在哪里}

\begin{large}
    
    \begin{verse}[0.5\linewidth]
        春天在哪里呀, \\
        春天在哪里? \\
        春天在那青翠的山林里。 \\
        这里有红花呀, \\
        这里有绿草, \\
        还有那会唱歌的小黄鹂。
    \end{verse}
    
    
    \begin{verse}[0.5\linewidth]
        春天在哪里呀, \\
        春天在哪里? \\
        春天在那湖水的倒影里。 \\
        映出红的花呀, \\
        映出绿的草, \\
        还有那会唱歌的小黄鹂。
    \end{verse}
    
    
    \begin{verse}[0.5\linewidth]
        春天在哪里呀, \\
        春天在哪里? \\
        春天在那小朋友眼睛里。 \\
        看见红的花呀, \\
        看见绿的草, \\
        还有那会唱歌的小黄鹂。
    \end{verse}
    
\end{large}





\chapter{看星星}

\begin{large}
    
    夜晚,我坐在草地上,
    
    抬头,看见点点星光。
    
    还有浅浅的天河,
    
    横流过晚空中央。
    
    那天河的两边,
    
    是织女和牛郎。
    
    北斗七星,好像一把勺子。
    
    秋天来了,勺把指着西方。
    
\end{large}





\chapter{小燕子}

\begin{large}
    
    \begin{verse}[0.5\linewidth]
        小燕子,穿花衣, \\
        年年春天到这里。 \\
        我问燕子你为啥来, \\
        燕子说:“这里的春天最美丽!”
    \end{verse}
    
    
    \begin{verse}[0.5\linewidth]
        小燕子,告诉你, \\
        今年这里更美丽! \\
        我们盖起了大工厂, \\
        还装上了新机器。 \\
        欢迎你,长长久久住在这里!
    \end{verse}
    
\end{large}





\chapter{冬天的兴安岭}

\begin{large}
    
    天空灰蒙蒙的;大山白茫茫的。
    
    到处是厚厚的雪,小河也结冰了。
    
    林子里走出一只小鹿,东张西望。
    
    背上洒满了雪,好像大山一样。
    
\end{large}





\end{document}
