\documentclass[12pt,UTF-8]{ctexbook}
\usepackage{ctex}
\usepackage{titlesec}
\usepackage[BoldFont]{xeCJK}
\usepackage{fontspec,xunicode,xltxtra}
\usepackage{hanzibox}
\usepackage{xpinyin}

\setCJKmainfont[BoldFont=NSimSun]{SimSun}
\setmainfont{CMU Serif}
\xpinyinsetup{ratio=0.4}
\titleformat{\section}{\zihao{-2}\bfseries}{ \thesection }{16pt}{}
\setlength{\lineskip}{12pt}
\setlength{\parskip}{20pt}
\hanziboxset{tran=false,frametype=none,framelinewidth=0pt} % fillcolor=yellow!10,charcolor=red,xscale=1,yscale=1,resize=real,framecolor=blue
% 封面
\title{\zihao{0} \bfseries 标题}
\author{}
\date{}
\begin{document}
\maketitle
\newpage



{\zihao{4}
    \xpinyin*{我总以为大兴安岭奇峰怪石,高不可攀。这回有机会看到它,并且走进原始森林,脚落在积得几尺厚的松针上,手摸到那些古木,才证实了这个悦耳的名字是那种亲切与舒服。}
}

{\zihao{4}\bfseries
    \xpinyin*{我总以为大兴安岭奇峰怪石,高不可攀。这回有机会看到它,并且走进原始森林,脚落在积得几尺厚的松针上,手摸到那些古木,才证实了这个悦耳的名字是那种亲切与舒服。}
}

\begin{pinyinscope}
列位看官:你道此书从何而来?说起根由,虽近荒唐,细按则深有趣味。
待在下将此来历注明,方使阅者\xpinyin{了}{liao3}然不惑。
\end{pinyinscope}

    
\end{document}