《农村调查》序言

年级:9

作者:毛泽东

现在党的农村政策,不是十年内战时期那样的土地革命政策,而是抗日民族统一战线的政策。全党应该执行一九四零年七月七日和十二月二十五日的中央指示,应该执行即将到来的七次大会的指示。所以印这个材料,是为了帮助同志们找一个研究问题的方法。现在我们很多同志,还保存着一种粗枝大叶、不求甚解的作风,甚至全然不了解下情,却在那里担负指导工作,这是异常危险的现象。对于中国各个社会阶级的实际情况,没有真正具体的了解,真正好的领导是不会有的。
要了解情况,唯一的方法是向社会作调查,调查社会各阶级的生动情况。对于担负指导工作的人来说,有计划地抓住几个城市、几个乡村,用马克思主义的基本观点,即阶级分析的方法,作几次周密的调查,乃是了解情况的最基本的方法。只有这样,才能使我们具有对中国社会问题的最基础的知识。
要做这件事,第一是眼睛向下,不要只是昂首望天。没有眼睛向下的兴趣和决心,是一辈子也不会真正懂得中国的事情的。
第二是开调查会。东张西望,道听途说,决然得不到什么完全的知识。我用开调查会的方法得来的材料,湖南的几个,井冈山的几个,都失掉了。这里印的,主要的是一个《兴国调查》,一个《长冈乡调查》和一个《才溪乡调查》。开调查会,是最简单易行又最忠实可靠的方法,我用这个方法得了很大的益处,这是比较什么大学还要高明的学校。到会的人,应是真正有经验的中级和下级的干部,或老百姓。我在湖南五县调查和井冈山两县调查,找的是各县中级负责干部;寻乌调查找的是一部分中级干部,一部分下级干部,一个穷秀才,一个破产了的商会会长,一个在知县衙门管钱粮的已经失了业的小官吏。他们都给了我很多闻所未闻的知识。使我第一次懂得中国监狱全部腐败情形的,是在湖南衡山县作调查时该县的一个小狱吏。兴国调查和长冈、才溪两乡调查,找的是乡级工作同志和普通农民。这些干部、农民、秀才、狱吏、商人和钱粮师爷,就是我的可敬爱的先生,我给他们当学生是必须恭谨勤劳和采取同志态度的,否则他们就不理我,知而不言,言而不尽。开调查会每次人不必多,三五个七八个人即够。必须给予时间,必须有调查纲目,还必须自己口问手写,并同到会人展开讨论。因此,没有满腔的热忱,没有眼睛向下的决心,没有求知的渴望,没有放下臭架子、甘当小学生的精神,是一定不能做,也一定做不好的。必须明白:群众是真正的英雄,而我们自己则往往是幼稚可笑的,不了解这一点,就不能得到起码的知识。
我再度申明:出版这个参考材料的主要目的,在于指出一个如何了解下层情况的方法,而不是要同志们去记那些具体材料及其结论。一般地说,中国幼稚的资产阶级还没有来得及也永远不可能替我们预备关于社会情况的较完备的甚至起码的材料,如同欧美日本的资产阶级那样,所以我们自己非做搜集材料的工作不可。特殊地说,实际工作者须随时去了解变化着的情况,这是任何国家的共产党也不能依靠别人预备的。所以,一切实际工作者必须向下作调查。对于只懂得理论不懂得实际情况的人,这种调查工作尤有必要,否则他们就不能将理论和实际相联系。“没有调查就没有发言权”,这句话,虽然曾经被人讥为“狭隘经验论”的,我却至今不悔;不但不悔,我仍然坚持没有调查是不可能有发言权的。有许多人,“下车伊始”,就哇喇哇喇地发议论,提意见,这也批评,那也指责,其实这种人十个有十个要失败。因为这种议论或批评,没有经过周密调查,不过是无知妄说。我们党吃所谓“钦差大臣”的亏,是不可胜数的。而这种“钦差大臣”则是满天飞,几乎到处都有。斯大林的话说得对:“理论若不和革命实践联系起来,就会变成无对象的理论。”当然又是他的话对:“实践若不以革命理论为指南,就会变成盲目的实践。”除了盲目的、无前途的、无远见的实际家,是不能叫做“狭隘经验论”的。
我现在还痛感有周密研究中国事情和国际事情的必要,这是和我自己对于中国事情和国际事情依然还只是一知半解这种事实相关联的,并非说我是什么都懂得了,只是人家不懂得。和全党同志共同一起向群众学习,继续当一个小学生,这就是我的志愿。