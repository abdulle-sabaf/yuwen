麦琪的礼物

年级:8

作者:欧·亨利

一元八角七。全都在这儿了。其中六十二分是一分一分的铜板。这些铜板是从杂货店、菜摊和肉店老板那儿一分两分地抠下来的。每个铜板背后都是一番软磨硬泡,锱铢必较,直到教人羞惭难当的地步。德拉反复数了三次,还是一元八角七,而第二天就是圣诞节了。 
除了倒在那张破旧的小睡椅上大哭一场之外,显然没有别的办法了。于是德拉就这么办了。生活是由啜泣、抽噎和微笑组成的,其中绝大多数是抽噎。 
在女主人从啜泣慢慢转入抽噎的这段时间里,让我们看看这个家吧。一套带家具的公寓,每周房租八美元。尽管难以用笔墨形容,确实与贫民窟也相差无几了。 
楼下的门道里有个信箱,可从来没有装过信;还有一个门铃电钮,鬼才按的响;按钮旁边还有一张名片,写着“詹姆斯·狄林汉·杨先生”。 
“狄林汉”这个名号是主人先前手面还阔绰的时候,一时兴起加上去的,那时候他每星期挣三十元。现在每周的收入缩减到二十元了,“狄林汉”的字母也显得模糊不清,似乎慎重考虑着,是否缩写成一个谦逊朴实的“狄”。不过,每当詹姆斯·狄林汉·杨回家,走进楼上的房间时,詹姆斯·狄林汉·杨太太,就是刚介绍给诸位的德拉,总是把他叫做“吉姆”,而且热烈地拥抱他。那当然是再好不过的了。 
德拉哭完了,小心地用破粉扑往面颊上抹了抹粉。她站在窗前,痴痴地瞅着灰蒙蒙的后院。后院里,一只灰色的猫在灰色的篱笆上走着。明天就是圣诞节,她只有一元八角七给吉姆买一份礼物。几个月来,她尽可能地省下每一分钱,结果也只不过如此。一周二十美元,实在经不起花,钱比她预计的更不经用了。只有一元八角七给吉姆买礼物。她心爱的吉姆啊!她耗费了多少幸福的时光,筹划着要送他一件可心的礼物,一件精致、珍奇、贵重的礼物——多少要有点儿配得上吉姆的东西才成啊。 
房间的两扇窗子之间有一面壁镜。也许你见过每周房租八美元的公寓壁镜吧。一个非常瘦小而灵巧的人,通过观察自己一连串的纵条映像,也许可以对自己的容貌得到一个大致不错的概念。德拉全靠身材纤细,才精通了这门艺术。 
骤然间,她从窗口旋风般地转过身来,站到壁镜前。她两眼晶莹透亮,但不到二十秒钟,她的脸上就失去了光彩。她很快地把头发解开,披落下来。 
詹姆斯·狄林汉·杨夫妇俩各有一件特别引以自豪的东西。一件是吉姆的金表,是他祖父传给父亲,父亲又传给他的传家宝;另一件则是德拉的秀发。如果示巴女王也住在天井对面的公寓里,总有一天德拉会把头发披散,悬在窗外晾干,好教那女王的珍宝首饰黯然失色;如果所罗门王把公寓的地下室用来存放金银财宝,每当吉姆路过那儿,准会摸出金表,好让那所罗门王忌妒得吹胡子瞪眼睛。 
此时此刻,德拉的秀发泼撒在身上,波浪起伏,光芒闪耀,仿佛一股褐色的瀑布。她的美发一直垂到膝下,仿佛一件长袍。接着,她又神经质地赶紧把头发梳好。她踌躇了一分钟,静静地立在那儿,几滴眼泪落在破旧的红地毯上。 
她穿上那件褐色的旧外衣,戴上褐色的旧帽子,眼睛里残留着晶莹的泪花,裙子一摆,便飘出房门,下楼来到街上。 
她走到一块招牌前停下来,招牌上写着:“娑芙罗妮夫人——专营各式头发用品”。德拉奔上楼梯,喘着气,定了定神。娑芙罗妮夫人身躯肥大,过于苍白,冷若冰霜,和她雅致的名字完全不相称。 
“你要买我的头发吗?”德拉问。 
“我买头发,”夫人说。“揭掉帽子,让我看看头发的样子。” 
那褐色的瀑布泼潵了下来。 
“二十美元。”夫人熟练地抓起头发。 
“快给我钱。”
嗬!接着而至的两个小时犹如长了玫瑰色的翅膀,愉快地飞掠而过——请不用理会这杂凑的比喻。她正在彻底搜寻各家店铺,为吉姆买礼物。 
她终于找到了,那准是专为吉姆特制的,决非为别人。她找遍了各家商店,哪儿也没有这样的东西,一条朴素的白金表链,式样简单朴素。正如真正的上等货那样,只以材质,而不以庸俗的雕饰来宣示自己的价值。它正配得上那只金表。她一见这条表链,就知道它应该为吉姆所有。它就像吉姆一样,文静而高尚。她用二十一美元买下了表链,匆匆赶回家,只剩下八角七分钱。配上这条链子,在任何场合,吉姆都可以毫无顾忌地掏出表看钟点了。那只表虽然精美,可因为他只用一根旧皮条当作表链,他很少看表,至多偷偷拿出来瞥上一眼。 
回到家里,她的狂喜逐渐让位于审慎和理智。她找出烫发铁钳,点燃煤气,着手修补因爱情加慷慨所造成的破坏,这自然是件艰巨的任务,亲爱的朋友们——简直是件了不起的任务呵。 
不出四十分钟,她的头上满是紧贴头皮的一绺绺小发卷,教她看起来活像个逃学的小男孩。她仔细而苛刻地对着镜子照了又照。 
“吉姆一看到我,不把我杀死才怪呢!”她自言自语,“他一定会说我像科尼岛游戏场里的卖唱姑娘。但是我能怎么办呢——唉,只有一元八角七,我能有什么办法呢?” 
七点钟,她煮好了咖啡,把煎锅置于热炉上,随时准备煎肉排。 
吉姆一贯准时回家。德拉把表链对叠,握在手心里,坐在离他进门必经之处,靠门的桌子角上。接着,她听见下面楼梯上响起了他的脚步声,她紧张得脸发白了。她习惯于为了最简单的日常事物而默默祈祷,此刻,她悄声道:“求求上帝,让他觉得我还是漂亮的吧。” 
门开了,吉姆走进来,随手关上了门。他显得瘦削而又非常严肃。可怜的人儿,他才二十二岁,就挑起了家庭的重担!他需要买件新大衣了,手套也没有。 
吉姆一进门就站住了,好像猎犬嗅到了鹌鹑的气味似的纹丝不动。他的目光定在德拉身上,带着一种使她无法理解的神情,使她大为惊慌。那既不是愤怒,也不是惊讶,又不是不满,更不是嫌恶,不是她所预料的任何一种神情。他只是用这种神情死死地盯着她。 
德拉一扭腰,从桌上跳了下来,向他走过去。 
“吉姆,亲爱的,”她喊道,“别那样盯着我。我把头发剪掉卖了,因为不送你一件礼物,我无法过圣诞节。头发会再长起来的——你不会介意,是吗?我非这么做不可。我的头发长得快极了。说‘圣诞快乐’吧!吉姆,让我们开开心心的。你肯定猜不着我给你买了一件多么好——多么美丽精致的礼物啊!” 
“你把头发剪掉了?”吉姆吃力地问道,似乎他绞尽脑汁也没把这显然的事实弄明白。 
“剪掉了,卖掉了,”德拉说。“不管怎么说,你还是那么爱我——不是吗?没了长发,我还是我——不是吗?” 
吉姆古怪地四下望望这房间。 
“你说你的头发没有了吗?”他带着近似白痴的神情问道。 
“别找啦,”德拉说。“告诉你,我已经卖了——卖掉了,没有啦。这是圣诞前夜,亲爱的。好好待我,这是为了你呀。也许我的头发数得清,”突然她特别温柔地接下去,“可谁也数不清我对你的爱。我把肉排烧上,好吗,吉姆?” 
吉姆好像忽然从恍惚之中醒来了,把德拉紧紧地搂在怀里。现在,别着急,先让我们花十秒钟,从另一个角度好好想一下某些无关紧要的东西吧。每周八美元的房子,或者一百万美元——那有什么差别呢?数学家或耍嘴皮子的也许会给你错误的答案。麦琪带来了珍贵的礼物,但它不是其中任何一个。这句晦涩的话,下文将有所交待。 
吉姆从大衣口袋里掏出一个小包,扔在桌上。 
“别对我产生误会,德儿,”他说道,“无论剪发、修脸,还是洗头,我对我的姑娘的爱不会减低一分。不过,你只要打开那包东西,就会明白为什么刚才我愣住了。” 
白皙的手指灵巧地解开绳子,打开包装纸。紧接着是欣喜若狂的尖叫,哎呀!突然变成了女性神经质的泪水和哭泣,急需男主人千方百计的慰藉。 
因为摆在桌上是一整套梳子,包括两鬓用的,后面用的,样样俱全。很久以前,德拉在百老汇的一个橱窗里见过,羡慕得要死。这套美妙的发梳是纯玳瑁做的,边上镶着珠宝——这色彩这光泽,正好同她失去的美发相匹配。她知道这套梳子有多昂贵。她心中自然神往已久,但从不抱拥有它的希望。现在,这心仪多时的饰物居然为她所有了,可惜那头美丽长发,它本应装饰的长发,已经不在了。 
不过,她还是把发梳紧紧抱在怀里,隔了好久,才抬起泪水迷蒙的双眼,微笑着说:“我的头发长得飞快,吉姆!” 
随后,德拉像一只被烫到的小猫似的跳起来,叫道,“噢!噢!” 
吉姆还没看她送给他的礼物哩。她急不可耐地把手掌摊开,伸到他面前,那无知无觉的贵重金属似乎反映着她的欢快和热忱。 
“漂亮吗,吉姆?我搜遍了全城才找到了它。现在,你每天可以把表看上一百次了。把表给我,我要看看它配在表上的样子。” 
吉姆并没有照她的话去做,反而倒在睡椅上,头枕在双手上,微微发笑。 
“德儿,”他说,“让我们暂且把圣诞礼物先放在一边吧。我们现在还用不上。我把表卖了,用卖表的钱为你买了发梳。现在把肉排烧上吧。” 
各位读者也知道,麦琪是有智慧的人——非常有智慧的人,他们从东方带来珍贵的礼物,送给刚诞生在马槽里的耶稣。这就是圣诞礼物的由来。麦琪是有智慧的人,毫无疑问,他们的礼物也是智慧的礼物。即便两件礼物一模一样,也必然能相互交换。我的拙笔在此向你们描述了一个平平无奇的故事。住在一间公寓里的两个傻孩子,愚蠢地为了对方放弃了家里最珍贵的宝物。不过,最后我还是想告诉现今的聪明人:这两个人,在所有互赠礼物的人里,是最有智慧的。送礼的是最有智慧的,收礼的也是。天下没有比他俩更有智慧的人了。他们就是麦琪。