阿长与山海经

年级:7

作者:鲁迅

长妈妈,已经说过,是一个一向带领着我的女工,说得阔气一点,就是我的保姆。我的母亲和许多别的人都这样称呼她,似乎略带些客气的意思。只有祖母叫她阿长。我平时叫她“阿妈”,连“长”字也不带;但到憎恶她的时候,——例如知道了谋死我那隐鼠\footnote{1}的却是她的时候,就叫她阿长。
我们那里没有姓长的;她生得黄胖而矮,“长”也不是形容词。又不是她的名字,记得她自己说过,她的名字是叫作什么姑娘的。什么姑娘,我现在已经忘却了,总之不是长姑娘;也终于不知道她姓什么。记得她也曾告诉过我这个名称的来历:先前的先前,我家有一个女工,身材生得很高大,这就是真阿长。后来她回去了,我那什么姑娘才来补她的缺,然而大家因为叫惯了,没有再改口,于是她从此也就成为长妈妈了。
虽然背地里说人长短不是好事情,但倘使要我说句真心话,我可只得说:我实在不大佩服她。最讨厌的是常喜欢切切察察,向人们低声絮说些什么事。还竖起第二个手指,在空中上下摇动,或者点着对手或自己的鼻尖。我的家里一有些小风波,不知怎的我总疑心和这“切切察察”有些关系。又不许我走动,拔一株草,翻一块石头,就说我顽皮,要告诉我的母亲去了。一到夏天,睡觉时她又伸开两脚两手,在床中间摆成一个“大”字,挤得我没有余地翻身,久睡在一角的席子上,又已经烤得那么热。推她呢,不动;叫她呢,也不闻。
“长妈妈生得那么胖,一定很怕热罢?晚上的睡相,怕不见得很好罢?……”
母亲听到我多回诉苦之后,曾经这样地问过她。我也知道这意思是要她多给我一些空席。她不开口。但到夜里,我热得醒来的时候,却仍然看见满床摆着一个“大”字,一条臂膊还搁在我的颈子上。我想,这实在是无法可想了。
但是她懂得许多规矩;这些规矩,也大概是我所不耐烦的。一年中最高兴的时节,自然要数除夕了。辞岁\footnote{2}之后,从长辈得到压岁钱,红纸包着,放在枕边,只要过一宵,便可以随意使用。睡在枕上,看着红包,想到明天买来的小鼓、刀枪、泥人、糖菩萨……。然而她进来,又将一个福橘\footnote{3}放在床头了。
“哥儿,你牢牢记住!”她极其郑重地说。“明天是正月初一,清早一睁开眼睛,第一句话就得对我说:‘阿妈,恭喜恭喜!’记得么?你要记着,这是一年的运气的事情。不许说别的话!说过之后,还得吃一点福橘。”她又拿起那橘子来在我的眼前摇了两摇,“那么,一年到头,顺顺流流……。”
梦里也记得元旦的,第二天醒得特别早,一醒,就要坐起来。她却立刻伸出臂膊,一把将我按住。我惊异地看她时,只见她惶急地看着我。
她又有所要求似的,摇着我的肩。我忽而记得了——
“阿妈,恭喜……。”
恭喜恭喜!大家恭喜!真聪明!恭喜恭喜!”她于是十分欢喜似的,笑将起来,同时将一点冰冷的东西,塞在我的嘴里。我大吃一惊之后,也就忽而记得,这就是所谓福橘,元旦辟头的磨难,总算已经受完,可以下床玩耍去了。
她教给我的道理还很多,例如说人死了,不该说死掉,必须说“老掉了”;死了人,生了孩子的屋子里,不应该走进去;饭粒落在地上,必须拣起来,最好是吃下去;晒裤子用的竹竿底下,是万不可钻过去的……。此外,现在大抵忘却了,只有元旦的古怪仪式记得最清楚。总之:都是些烦琐\footnote{4}之至,至今想起来还觉得非常麻烦的事情。
然而我有一时也对她发生过空前的敬意。她常常对我讲“长毛”。她之所谓“长毛”者,不但洪秀全军\footnote{5},似乎连后来一切土匪强盗都在内,但除却革命党\footnote{6},因为那时还没有。她说得长毛非常可怕,他们的话就听不懂。她说先前长毛进城的时候,我家全都逃到海边去了,只留一个门房\footnote{7}和年老的煮饭老妈子看家。后来长毛果然进门来了,那老妈子便叫他们“大王”,——据说对长毛就应该这样叫,——诉说自己的饥饿。长毛笑道:“那么,这东西就给你吃了罢!”将一个圆圆的东西掷了过来,还带着一条小辫子,正是那门房的头。煮饭老妈子从此就骇破了胆,后来一提起,还是立刻面如土色,自己轻轻地拍着胸埔道:“阿呀,骇死我了,骇死我了……。”
我那时似乎倒并不怕,因为我觉得这些事和我毫不相干的,我不是一个门房。但她大概也即觉到了,说道:“象你似的小孩子,长毛也要掳的,掳去做小长毛。还有好看的姑娘,也要掳。”
“那么,你是不要紧的。”我以为她一定最安全了,既不做门房,又不是小孩子,也生得不好看,况且颈子上还有许多炙疮疤。
“那里的话?!”她严肃地说。“我们就没有用处?我们也要被掳去。城外有兵来攻的时候,长毛就叫我们脱下裤子,一排一排地站在城墙上,外面的大炮就放不出来;再要放,就炸了!”
这实在是出于我意想之外的,不能不惊异。我一向只以为她满肚子是麻烦的礼节罢了,却不料她还有这样伟大的神力。从此对于她就有了特别的敬意,似乎实在深不可测;夜间的伸开手脚,占领全床,那当然是情有可原的了,倒应该我退让。
这种敬意,虽然也逐渐淡薄起来,但完全消失,大概是在知道她谋害了我的隐鼠之后。那时就极严重地诘问,而且当面叫她阿长。我想我又不真做小长毛,不去攻城,也不放炮,更不怕炮炸,我惧惮她什么呢!
但当我哀悼隐鼠,给它复仇的时候,一面又在渴慕着绘图的《山海经》了。这渴慕是从一个远房的叔祖惹起来的。他是一个胖胖的,和蔼的老人,爱种一点花木,如珠兰、茉莉之类,还有极其少见的,据说从北边带回去的马缨花\footnote{}。他的太太却正相反,什么也莫名其妙,曾将晒衣服的竹竿搁在珠兰的枝条上,枝折了,还要愤愤地咒骂道:“死尸!”这老人是个寂寞者,因为无人可谈,就很爱和孩子们往来,有时简直称我们为“小友”。在我们聚族而居的宅子里,只有他书多,而且特别。制艺和试帖诗,自然也是有的;但我却只在他的书斋里,看见过陆玑的《毛诗草木鸟兽虫鱼疏》\footnote{},还有许多名目很生的书籍。我那时最爱看的是《花镜》\footnote{},上面有许多图。他说给我听,曾经有过一部绘图的《山海经》\footnote{},画着人面的兽,九头的蛇,三脚的鸟,生着翅膀的人,没有头而以两乳当作眼睛的怪物,……可惜现在不知道放在那里了。
很愿意看看这样的图画,但不好意思力逼他去寻找,他是很疏懒的。问别人呢,谁也不肯真实地回答我。压岁钱还有几百文,买罢,又没有好机会。有书买的大街离我家远得很,我一年中只能在正月间去玩一趟,那时候,两家书店都紧紧地关着门。
玩的时候倒是没有什么的,但一坐下,我就记得绘图的《山海经》。
大概是太过于念念不忘了,连阿长也来问《山海经》是怎么一回事。这是我向来没有和她说过的,我知道她并非学者,说了也无益;但既然来问,也就都对她说了。
过了十多天,或者一个月罢,我还记得,是她告假回家以后的四五天,她穿着新的蓝布衫回来了,一见面,就将一包书递给我,高兴地说道:——“哥儿,有画儿的‘三哼经’,我给你买来了!”
我似乎遇着了一个霹雳,全体\footnote{}都震悚起来;赶紧去接过来,打开纸包,是四本小小的书,略略一翻,人面的兽,九头的蛇,……果然都在内。
又使我发生新的敬意了,别人不肯做,或不能做的事,她却能够做成功。她确有伟大的神力。谋害隐鼠的怨恨,从此完全消灭了。
这四本书,乃是我最初得到,最为心爱的宝书。
书的模样,到现在还在眼前。可是从还在眼前的模样来说,却是一部刻印都十分粗拙的本子。纸张很黄;图象也很坏,甚至于几乎全用直线凑合,连动物的眼睛也都是长方形的。但那是我最为心爱的宝书,看起来,确是人面的兽;九头的蛇;一脚的牛;袋子似的帝江\footnote{};没有头而“以乳为目,以脐为口”,还要“执干戚而舞”的刑天\footnote{}。
此后我就更其搜集绘图的书,于是有了石印\footnote{}的《尔雅音图》\footnote{}和《毛诗品物图考》\footnote{},又有了《点石斋丛画》\footnote{}和《诗画舫》\footnote{}。《山海经》也另买了一部石印的,每卷都有图赞,绿色的画,字是红的,比那木刻的精致得多了。这一部直到前年还在,木刻的却已经记不清是什么时候失掉了。
我的保姆,长妈妈即阿长,辞了这人世,大概也有了三十年了罢。我终于不知道她的姓名,她的经历;仅知道有一个过继的儿子,她大约是青年守寡的孤孀。仁厚黑暗的地母\footnote{}呵,愿在你怀里永安她的魂灵!

尾注:〔憎恶〕不顺滑。

尾注:〔惶急〕从旁劝说,使想做。

尾注:〔骇〕惊吓。

尾注:〔霹雳〕又急又响的雷。

尾注:〔震悚〕震惊惶恐。

尾注:〔掷〕扔,抛。

尾注:〔掳〕抢走,抓走。

尾注:〔孀〕称呼丧夫的寡妇。

尾注:〔图赞〕写在画面或图页上的赞美诗文。
