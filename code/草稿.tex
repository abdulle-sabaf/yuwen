项链

年级:11

作者:纪·德·莫泊桑

注释:〔布列塔尼〕布列塔尼是法国西北部的省,当时是经济不发达的落后地区。

注释:〔东方样式〕指波斯特色的。

注释:〔南泰尔平原〕当时巴黎市郊的一个地方。

注释:〔威尼斯式〕指从威尼斯流传出来的镶嵌工艺。

注释:〔皮草〕指动物毛皮制成的衣物。

注释:〔金路易〕法国货币,一金路易等于20法郎。

注释:〔铜板〕指面值一分钱的货币。

她也是一个美丽动人的姑娘,大概由于命运的差错,生在了一个小职员的家里。她没有陪嫁的资产,也没有什么法子让有钱的体面人结识她、了解她、爱上她、娶她。最终她嫁给了教育部门的一个小科员。

她没钱讲究打扮,因此穿着朴素,这令她深感痛苦,仿佛受了贬谪,降了身份。因为女性并无种姓种族之分。对女性而言,美貌、娇艳、丰韵就是门第出身。天生的聪敏,本能的优雅,自然的裕如,就决定了高下,让贫家女也能与贵妇相提并论。

她觉得自己生来就该属于那种高贵奢华的生活,现实于她就是无尽的痛苦。寒碜的住所,黯淡的四壁,破旧的桌椅,粗陋的衣料,这一切都令她痛苦。像她这个级别的美人,甚至不应该看到这些东西。这一切都折磨着她,让她愤慨。看着替她料理家务的布列塔尼小女仆,她心底就不由得泛起感伤悔恨,生起狂乱的迷梦来。她梦想寂静的偏厅装饰着东方样式的帷幔和高脚的青铜灯,两个穿着短裤的高大仆人躺在宽大的扶手椅上,被暖气管散出的沉重热气熏得昏昏欲睡。她梦想偌大的客厅四壁蒙着仿古花样的丝缎,精美的橱柜里摆着珍贵的古玩;还有那些雅致的带熏香的小客厅,用来在下午五点与最亲密的好友闲聊,与那些广受欢迎的人物、那些所有女子都垂涎不已、渴盼青睐的男子倾谈。

每当她坐在圆桌旁准备吃晚餐的时候,看着铺在桌上的三天没洗的桌布,看着对面坐着的丈夫揭开盆盖,心满意足地赞叹:“啊!多么好吃的炖肉!再没有比这更好的了!”她就会梦想那些奢华的晚餐,闪亮的银餐具,那屏风上的挂毯,上面织有古代的人物和仙境一般的园林里的异鸟珍禽。她还梦想到盛在名贵的碗碟中的珍馐佳肴,想象自己一边吃着粉红的鳟鱼或松鸡翅膀,一边带着高深莫测的微笑,倾听宾客之间兴致盎然的喁喁低语。

她没有漂亮的衣装,没有珠宝首饰,什么也没有。可她偏偏只喜爱这些;她觉得自己就是为此而生的。她最希望的就是能讨男子欢心,让女子羡慕,风流动人,处处受欢迎。
她有一个有钱的朋友,是她在女校读书时的同学。她每次与她聚会回来都痛苦无比,以至于再也不愿去见她了。每次见完她,她总陷入伤心、懊悔、绝望、彷徨,哭上好几天。
一天晚上,她丈夫得意洋洋地回家来,手里拿着一个大信封。
“给!”他说,“这是给你的一点东西。”
她连忙拆开信封,里面是一张请柬,上边印着:
教育部部长乔治·朗伯诺及夫人恭请乐瓦赛先生及夫人光临教育部礼堂,参加一月十八日(星期一)之晚会。
她并不像她丈夫预料的那样欢喜,反而恼恨地把请柬往桌上一丢,咕哝着说:“你这是要让我做什么?”
“怎么了,亲爱的?我以为你一定会喜欢的!你总不出门交际,这可是一个机会,千载难逢的机会!我费了好大劲才弄到这张请柬。大家都想要,可不容易了。这请柬一般不大发给科员的。你到了那儿,什么达官贵人都能见到。”
她瞪着他,眼中冒火,最后不耐烦地喊道:“你叫我穿什么去呢?”
他没料到这点,于是支支吾吾地说:“你去戏院时穿的那件衣服,依我看,就很不错……”
他住了口,他看见妻子已经在哭了,他惊呆了,张徨失措。两大滴眼泪从他妻子的眼角慢慢地向嘴角流下来;他吃力地问道:“你怎么啦?你怎么啦?”
她费了很大劲才抑制住悲痛,擦了擦被泪水沾湿的两颊,用平静的声音说道:“没什么。但既然我没有像样的衣饰,当然无法赴会了。要是您哪位同事的太太有比我更好的衣饰,就把请柬送给他吧。”
他难受了,于是说道:“好吧,玛蒂尔德,咱们来商量一下。一套过得去的衣服,一套在别的场合还可以穿的、足够朴素的衣服,得用多少钱?”她想了几秒钟,合计了一下,考虑着怎样的数目才不致当场遭到这个俭朴的科员拒绝,不至于把他吓得叫出来。最终,她带着迟疑说道:“我也说不好一个准数;不过我想,要是有四百法郎,大概就可以了。”
他脸色有点发白,因为他正巧积攒下这样一笔款子,打算买一杆猎枪,好和几个朋友一道去打猎,在夏天的星期日到南泰尔平原去打云雀。
不过他还是回答道:“好吧。我给你四百法郎。可是你得好好想法子,做件漂漂亮亮的衣服。”晚会的日子快到了,乐瓦赛太太却显得伤心焦虑,很是忧愁。她的衣服可是已经做好了。一天晚上,她丈夫不禁问她:“你怎么啦?这几天你总是神不守舍的。”
“我心里愁着呢。我没有首饰,也没有珠宝,没有一点能穿戴的,太寒酸了。我简直不想参加这次晚会了。”
他说:“你可以在头上戴几朵鲜花呀。这个时候的鲜花很漂亮。十个法郎就能买到两三朵非常好看的玫瑰花了。”
她丝毫不依。
“不行……在那些阔太太中间,显露着一副穷酸相,再难堪也没有了。”
他丈夫突然喊道:“你可真傻!你可以去找你的朋友福雷斯杰太太嘛!向她借几件珠宝就好了。你跟她很有交情,这点小事不难办到。”
她惊喜地叫了出来:“对呀!我竟然完全没想到。”
第二天,她就到了她朋友家里,把自己的苦恼讲给她听。福雷斯杰太太立刻走到她那镶镜子的大立柜跟前,取出一个硕大的珠宝匣子,到乐瓦赛太太面前打开,对她说:“亲爱的,挑吧!”
首先映入眼帘的是几只手镯,随后是一串珍珠项链,一个威尼斯式的镶嵌珠宝的金十字架,做工极其精细。她戴上这些首饰,对着镜子试了又试,犹豫不决,舍不得摘下。她嘴里不停地问:“你再没有别的了吗?”
“当然有。你自己找吧。我不知道哪样合你的意。”

忽然,她发现了。那是一串异常美丽的钻石项链,躺在一个青色缎子的盒子里。一股异常强烈的欲望涌上心头,让她的心怦怦直跳,连拿起它的手都在颤抖。她把它戴在颈子上,让它在衣领上舒展开来,对着镜中的自己,看得出了神。

她按捺着心中焦急,迟疑地问道:“你可以把这个借给我吗?就这一样?”
“当然可以。”
她几乎跳将起来,一把搂住她朋友的脖子,给了她一个热烈的吻,然后带着那宝贝逃也似地离开了。

晚会的日子到了。乐瓦赛太太大获成功。她比晚会上的其他女子都美丽,又高雅又迷人,面上总带着微笑,兴致高昂。所有的男子的目光都离不开她,都在打听她是谁,求人给介绍。部里机要处的人全都想跟她合舞。部长也注意到了她。她陶醉在欢乐中,什么也不想了,只是兴奋地、狂热地跳舞。她的美貌战胜了一切,她的成功充满了光辉,所有这些人都对自己殷勤献媚、阿谀赞扬、垂涎欲滴。任何女子能想象的最甜美的胜利,已完完全全握在手中,她便在这一片幸福的云中舞着。
她是凌晨四点钟左右离开的。她的丈夫从午夜起就和另外三个男宾在一间偏僻的小客厅里睡着了。这几位的太太都还在尽情欢舞。
他为她披上外套,那是怕她在室外受寒,从家里带来的,是平日穿的家常衣服。那寒酸气和华丽的长裙晚礼服格格不入。她马上察觉到这一点。为了不叫旁近那些裹在华贵皮草里的太太们注意到,她急着要奔出大门去。
乐瓦赛先生把她拉住,说:“等一等,你这么出去要着凉的!我去叫一辆马车来。”
但是她一点也不听,匆匆下了楼梯。到了街上,一辆马车也没有。他们只得到处找,远远地看见车夫就喊。他们沿着塞纳河一路走一路找,浑身哆嗦,满心失望。最后在河边找到了一辆做夜里生意的旧马车。这种马车只有天黑之后才会见到,也许是因为太破,羞于在白天出来揽客。
马车把他们送到殉难者街,到了他们家门口。他们凄凄凉凉地爬上楼,回到家里。于她来说,一切都结束了。他呢?他想的是十点钟还要到部里办公。
她对着更衣镜,褪下披在肩上的外套,为的是再端详一次还笼罩在荣光中的自己。然而,她猛地大喊一声:颈上的项链不见了。
她丈夫的衣服已经脱了一半,见状问道:“怎么了?”
她吓得要昏过去了,转过身来,说:“我……我……不见了福雷斯杰太太的项链……”
他挺直了身子,惊惶地说:“什么!……怎么!……怎么会有这样的事?”
他们在礼服、大衣的褶层里翻来找去,所有口袋都寻了一遍,哪儿也没有找到。他问:“你确定你离开舞会的时候还戴着它吗?”
“是的,在部里的衣帽间里我还摸过它呢。”
“不过,要是掉在街上了,总该听见掉落的响声啊!大概是掉在马车里了。”
“没错,一定是这样。你还记得车牌号吗?”
“不记得了。你呢?你没注意吗?”
“没有。”
他们面面相觑,满脸惊恐。末了,乐瓦赛先生重新穿好衣服。
“我去。”他说,“把我们回来的路重走一遍。看看能不能找着。”
他出去了。她穿着那件舞会上的礼服,连上床睡觉的力气也没有,只是倒在一把椅子里发呆,毫无神气,什么也没法想。
七点钟左右,她丈夫回来了。什么也没找着。
后来,他又去了警察厅,去了各个报馆,悬赏招寻;又去了所有的车行寻找。总之,但凡有一线希望的地方,他都去过了。
而她只有等待,整日等候,在这可怖的灾祸造成的惊恐中,麻木地等候着。
晚上,乐瓦赛先生带着瘦削苍白的脸回来了。一无所获。
“你得给你的朋友写信。”他说,“就说你把项链的搭钩弄坏了,要拿去修理。这样我们才有周转的时间。”
她照他说的写了信。
过了一个星期,所有的希望都断绝了。
乐瓦赛先生好像老了五年。他毅然说:“该想办法赔这件首饰了。”
第三天,他们拿着装项链的盒子,照着上面的招牌字号,找到了卖项链的珠宝店。老板查看了许久账簿,说:“太太,这项链不是我们卖出的。我只卖了这个盒子。”
于是他们一家一家珠宝店地找过去,凭着记忆去找一串同样的项链。两人都愁得要病倒了。
终于,在王宫街的一家铺子里,他们看到了一串和他们要找的一模一样的项链,标价四万法郎。老板让了价,只要三万六千。
于是他们恳求老板三天内不要卖出去。他们还跟老板约定,如果原来的项链能在二月底之前找回来,老板可以用三万四千法郎把这串项链收回去。
乐瓦赛先生有父亲留给他的遗产,一共一万八千法郎。剩下的钱要去借。
他借钱了。向这个借一千,向那个借五百。从这儿借五个金路易,那里借三个金路易。他签了好些欠条,答应了不少足以让他破产的条件。他跟很多放高利贷的人、各种各样的放债人打交道。他已经顾不得下半辈子了,冒险到处签名,不管能否偿还。他害怕将来的忧患,也害怕即将加身的极端贫困,物质上的缺乏,精神上的折磨。他满怀这着诸多恐惧,把三万六千法郎放到了商铺的柜台上,取到了那串新项链。

乐瓦赛太太上门送还项链时,福雷斯杰太太带着不满意的神情对她说:“你该早点还我的。我也许要戴呢!”
她并没有打开盒子来看。乐瓦赛太太就怕她当面打开盒子检查,因为如果她发现项链不是原物,她会怎么想呢?她会怎么说呢?难道不会把她当作盗贼吗?

乐瓦赛太太尝到了穷人的艰难生活的滋味了。幸而她一下子就显出了英雄的气概。这笔骇人听闻的债务是必须偿还的。她必须把债还清。他们辞退了女仆,搬了家,租了一间紧挨屋顶的阁楼。家里的粗笨活、厨房里的腻人的杂活,都包在了她身上。锅碗瓢盆都得自己洗刷,油腻的盆子和锅底磨坏了她玫瑰色的手指甲。脏衣服、衬衫、抹布也都得自己洗,洗了晾在绳子上。每天早上,她要把垃圾搬到街上,又把水提到楼上,每上一层楼都要停一停喘口气。她习惯了穷苦妇女的穿着,胳膊上挎着篮子上水果店,上杂货店,上肉店,讨价还价,顶着讥嘲,一个铜板一个铜板地省下一点可怜的钱。
每个月都要还一批旧债,借一批新债,延长偿债的期限。她丈夫晚上还要替商贾誊写账目,到深夜还在抄写五分钱一页的书稿。
这样的生活持续了十年。
第十年年底,债全还清了。不但那高额的利息,就是利滚利的利息也还清了。
乐瓦赛太太现在显得老多了。她成了穷苦家庭里的粗壮耐劳的妇人。她不再讲究梳妆,头发随意地挽着,围裙歪斜地系在身上,两手通红,用大嗓门说话,用大桶水洗地板。不过有时候,当她丈夫去上班的时候,她会坐到窗前,回想起当年的那次晚会。那晚的舞会上,她是多么美丽,多么令人倾倒啊。
如果她当年没有弄丢那串项链,今天又该是什么样子?谁知道呢?谁知道呢?人生是多么古怪,多么变幻无常啊!极细小的事就可以断送你的人生,或者拯救你的人生!
又是一个星期天,她到大街上散步。劳累了一个星期,她要消遣一下。就在此时,她看见一个妇人带着孩子在散步。那夫人正是福雷斯杰太太。她还是那样年轻、那样美丽、那样动人。乐瓦赛太太十分感慨。她要上前向福雷斯杰太太搭话吗?当然要去。既然债务都还清了,她完全可以对她坦白了。为什么不呢?
于是她走上前去。
“你好,让那。”
对方完全认不得她了。一个平民妇人这样亲密地叫她,让她诧异,她吞吞吐吐地说:“可是……太太!……我不知道……您大概是认错人了。”
“您没有认错。我是玛蒂尔德·乐瓦赛!”
她的朋友喊了起来:“天呐!……是我可怜的玛蒂尔德吗?你怎么变成这样儿了?……”
“是的,自从那一次见了你,我过了多少年艰难日子啊,忍受了多少苦楚……而这一切都是因为你!……”
“因为我?……这是怎么回事?”
“你还记得你借了我一串钻石项链吗?我戴了去参加部里晚会的那串?”
“记得,那又怎么样呢?”
“怎么样?我把它丢了。”
“怎么会呢?你不是已经还给我了吗?”
“我还给你的是另一串一模一样的。你知道吗?为此我们借了好多钱,整整花了十年才还清!你知道,对于我们这样没什么资产的人,这可不是件容易的事啊!……现在,事情总算了结了,我可高兴了。”
福雷斯杰太太停下脚步:“你刚才说,你买了一串钻石项链赔给我?”
“对啊,你当时没看出来?和你原来的那串一模一样!”她露出了微笑,笑里带着一股自傲而天真的快乐。
福雷斯杰太太突然激动起来,抓住了她的双手说道:“天呐!我可怜的玛蒂尔德!我那串项链是假的呀。顶多也就值五百法郎!”