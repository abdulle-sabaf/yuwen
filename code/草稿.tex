种金子

作者:赵世杰

年级:5

蓝蓝的天上没有半丝云,阳光照在集市的小路上。巴依老爷拿着账本,又来催账了。

巴依老爷敲着账本,对艾尔肯大爷说:“你去年向我借了两个鸡蛋,如今你要还我五十个银币。”

“天呐!五十个银币,够买一头羊了。两个鸡蛋,哪里值这么多钱?”大爷委屈地嚷嚷起来。

“你不懂!你自个儿好好算算。这两个鸡蛋要不是给你吃了,早就变成公鸡和母鸡了。母鸡又能下蛋,蛋又能孵出鸡来。你算算,你拿了我的蛋,让我少了几只鸡?”巴依老爷说着,眼泪都要掉下来了,仿佛自己才是那个受了委屈的人。

巴依老爷又对着古丽小姑娘说:“还有你!你之前向我借了一口铁锅,如今你要还我一百个银币。”

“那口锅还好着呢,我把锅还给你不就成了。一百个银币,够买好几口铁锅了。”小姑娘也不高兴了。

巴依老爷眼珠转了转,说:“你不懂!我这口锅是母锅。要是不借给你,不知能生几口小锅呢!你要是还我的锅,就把小锅也还上。”

“天呐!我简直一点儿也不明白!”巴依老爷走了之后,大爷大声叫了起来。

路过的阿凡提听到了,就过来问:“大爷,您遇到什么事了?”

“巴依老爷财迷心窍,想钱想疯了!”大爷就把巴依老爷的说法告诉了阿凡提。阿凡提听得皱起了眉头。

过了两天,阿凡提也到巴依老爷家做客。巴依老爷的锅里正在煮肉。巴依老爷指着锅里的肉,问阿凡提:“这肉可真香啊,你也闻到了吧?”

阿凡提说:“当然,这肉可真香啊,恐怕连真主也闻到了。”

“既然闻到了,就该付钱了吧?”巴依老爷得意地伸出了手掌。

“闻个香味也要付钱吗?”阿凡提问。

“那当然!香味也是肉的一部分嘛!品尝了肉香,就是享用了我的肉。”巴依老爷眨着眼睛说。

“那好吧,如您所愿。”阿凡提把钱袋子拿出来,在巴依老爷面前晃了晃,里面的银币哐当作响。

“听到了吗?”阿凡提问。

“听到了,听到了!是银币的响声!”巴依老爷听得心欢喜,伸手就要拿钱袋,阿凡提又把钱袋收回去了。

“怎么不给钱?”

“既然您听过钱的响声了,那就够了。”

“怎么能这么说!听个响也算付钱吗?”

“那当然!响声也是钱的一部分嘛!我让您欣赏了钱的响声,就算给肉香付钱了。”

巴依老爷没占到便宜。他看着阿凡提的钱袋,突然想到:这个阿凡提,怎么突然有钱了?怕不是偷来的,还是骗来的。我要好好查查这事,不能让这些刁民占了便宜。

太阳落山了,巴依老爷偷偷跟着阿凡提回家。只见阿凡提把钱袋子里的银币倒出来,埋在后院一棵歪脖子树下挖好的坑里。阿凡提一边埋银币,一边口里念念有词。

巴依老爷记下了埋银币的地方。每天晚上,他都去查看。第三天的晚上,阿凡提又来了,从树下挖出了昨晚埋下的银币,居然装满了两个钱袋子!

巴依老爷惊呆了。他想:“好家伙,这个阿凡提,不知道从哪里学来了这妖术,种下一袋银币,居然能收获两袋银币!”他又想:“要是我家的金银财宝也能这么种,那该有多好!”他从藏身的地方走出来,对阿凡提说:“阿凡提,你从哪里学来这种钱的法子?”

阿凡提吓了一跳,说:“哪有什么种钱的法子?这是我昨天埋下的。”

巴依老爷说:“别骗我了,我都看见了。种下十枚银币,收获二十枚银币。你得教我这种钱的法子。我给你一箱金子,你给我种出两箱金子来。”

阿凡提说:“这可不行。”

巴依老爷生气了:“你要是不教我,我就告到阿訇那里去,说你不守教规,偷窃真主的财富!”

阿凡提吓住了,只好说:“那好吧,如您所愿。您把金子给我,三天之后收成。这法子见不得光,见不得人。您切不可对别人说起,也不能在白天里挖开来看,否则这法子就不灵了。”

巴依老爷高兴极了,连忙从家里运来一大箱金子,看着阿凡提在树下挖了一个大坑,把金子都埋进去了。

想到三天之后就能收获两箱金子,巴依老爷就高兴得睡不着觉,每天都想着大树底下埋着的金子。到了第三天,他实在不放心,就到阿凡提家里来。阿凡提不在家,巴依老爷跑到后院,悄悄把土翻开来看:金子不见了!

巴依老爷的心好像被锤子砸碎了。他失魂落魄地找到阿凡提,问:“金子怎么没了?你还我的金子!”

阿凡提说:“我的好老爷。我告诉过您,可不能在白天里挖开来看呀!您这么一弄,金子都死了!”

巴依老爷气得吹胡子:“胡说!金子怎么会死呢?”

阿凡提说:“您既然相信金子能种,怎么就不相信金子会死呢?”

巴依老爷的金子没了,就把阿凡提抓到官府里问罪。官府宣判,阿凡提骗了巴依老爷的金子,把阿凡提砍了脑袋。可也有人说,阿凡提并没有死,因为不久之后的一个晚上,城里每户穷人家的门前都出现了一小块金子。人们都说,那是阿凡提种出来的金子。

脚注:〔真主〕伊斯兰教崇拜的神。

脚注:〔阿訇〕穆斯林对主持宗教事务人员的称呼。

