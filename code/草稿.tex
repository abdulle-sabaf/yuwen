在《人民报》创刊周年纪念会上的演说

年级:10

作者:卡尔·马克思

所谓的1848年革命,只是几个微不足道的事件,欧洲社会干硬外壳上的几处小裂口、小缝隙。然而,它们暴露了一个无底深渊。那貌似坚固的外表之下,现出了一片汪洋大海,只要它动荡起来,就能把这磐石般的大陆撞得粉碎。噪聒又懵懂地,它们揭露了19世纪的秘密,19世纪革命的秘密——无产阶级的解放。
1848年那场革命,并不是什么新发明。相比巴尔贝斯、拉斯拜尔和布朗基,蒸汽、电力和自动纺纱机才是危险得多的革命家。然而,即便你们知道,这个世界每个人身上都承受着两万磅的大气压力,你们能感觉到吗?同样,在1848年之前,欧洲社会也没有感觉到那从四面八方包裹着它、压抑着它的革命气氛。19世纪发生了一件大事,大到可以作为这个世纪的标志,任何政党都不敢否认。
一方面,19世纪诞生了人类历史上任何时代都无法想象的科技和工业伟力;另一方面,19世纪显露出的衰颓征兆,远远超过史书中罗马帝国末期的可怕情景。我们这个时代的每一种事物,都好像在孕育自己的反面。机器有着缩短工时、增加产能的神奇力量,却引发了过劳和饥荒。本该是财富的新源泉,却仿佛受了奇怪的诅咒,源源不断地制造贫困。每一次工艺的进步,仿佛都要用道德的败坏来交换。人类对自然的驾驭每进一步,受到的奴役就更深一层——不论是作为他人还是自身劣根性的奴隶。就连科学的纯洁光辉,在无知愚昧铺成的黑暗背景中也显得微不足道。看起来,我们的一切发明进步为物质力量赋予了灵性,却把人消磨成了物质力量。
现代的工业与科技与现代的贫困和崩坏,当前时代的生产力和当前时代的社会关系,鲜明对立着,格格不入。这个事实就像洪流袭来,近在眼前,毋庸置疑,无可抵挡。有的人为此痛苦哀号;有的人宁愿抛弃现代科技来摆脱现代的社会冲突;还有人设想,工业上如此巨大的进步,自然要以政治上同样巨大的倒退来补全。对我们来说,我们自然不会认不出这种种矛盾中隐约能见的狡诈身影。我们清楚地认识到,要使社会的新生力量发挥好作用,就只能由新生的人来掌握它们,这些新生的人就是工人。工人也同机器本身一样,是现代的产物。

在那些使中产阶级、贵族和蹩脚的衰退预言家惊慌失措的征兆中,我们认出了我们勇敢的朋友:好人儿罗宾,这个刨土一流的老鼹鼠、光荣的开拓者——革命。英国工人是现代工业的长子。因此,在参与现代工业革命所孕育的社会革命时,自然也不落人后。因为这革命是为了它们所在的阶级在全世界的解放,这革命同资本统治和工薪奴隶制同样必然到来。我知道英国工人阶级从上世纪中叶以来进行了多么英勇的斗争。中产阶级的历史学家把这些斗争掩盖起来,隐瞒不说,因此其英勇光辉少为人知。在中世纪的德国,为了报复统治阶级的罪行,曾有过一种叫作“维末法庭”的秘密法庭。如果某一所房子上画了一个红十字,大家就知道,这所房子的主人逃不过“维末”的判决了。而现在,欧洲所有的房子都画上了神秘的红十字。
历史本身就是审判官,而无产阶级就是执刑者。