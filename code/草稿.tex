山地回忆
 
作者:孙犁

年级:11

注释:〔阜平〕河北省保定市下辖县,地处保定市西部,太行山中北部东麓。

注释:〔工业展览会〕1949年11月4日至12月5日在天津举办的博览会。

注释:〔北平〕北京的旧称。

注释:〔“踢倒山”〕。

注释:〔曲阳〕河北省保定市下辖县,地处保定市西南部,在阜平县东南。

注释:〔碛口〕山西省吕梁市临县下的古镇,位于黄河晋陕峡谷中部,吕梁山西麓。

从阜平乡下来了一位农民代表,参观天津的工业展览会。我们是老交情,已经快有十年不见面了。我陪他去参观展览,他对于中纺的织纺,对于那些改良的新农具特别感到兴趣。临走的时候,我一定要送点东西给他,我想买几尺布。 
为什么我偏偏想起买布来?因为他身上穿的还是那样一种浅蓝的土靛染的粗布裤褂。这种蓝的颜色,不知道该叫什么蓝,可是它使我想起很多事情,想起在阜平穷山恶水之间度过的三年战斗的岁月,使我记起很多人。这种颜色,我就叫它“阜平蓝”或是“山地蓝”吧。 
他这身衣服的颜色,在天津是很显得突出,也觉得土气。但是在阜平,这样一身衣服,织染既是不容易,穿上也就觉得鲜亮好看了。阜平土地很少,山上都是黑石头,雨水很多很暴,有些泥土就冲到冀中平原上来了——冀中是我的家乡。阜平的农民没有见过大的地块,他们所有的,只是象炕台那样大,或是象锅台那样大的一块土地。在这小小的、不规整的,有时是尖形的,有时是半圆形的,有时是梯形的小块土地上,他们费尽心思,全力经营。他们用石块垒起,用泥土包住,在边沿栽上枣树,在中间种上玉黍。 
阜平的天气冷,山地不容易见到太阳。那里不种棉花,我刚到那里的时候,老大娘们手里搓着线锤。很多活计用麻代线,连袜底也是用麻纳的。 
就是因为袜子,我和这家人认识了,并且成了老交情。那是个冬天,该是一九四一年的冬天,我打游击打到了这个小村庄,情况缓和了,部队决定休息两天。 
我每天到河边去洗脸,河里结了冰,我登在冰冻的石头上,把冰砸破,浸湿毛巾,等我擦完脸,毛巾也就冻挺了。有一天早晨,刮着冷风,只有一抹阳光,黄黄的落在河对面的山坡上。我又登在那块石头上去,砸开那个冰口,正要洗脸,听见在下水流有人喊: 
“你看不见我在这里洗菜吗?洗脸到下边洗去!” 
这声音是那么严厉,我听了很不高兴。这样冷天,我来砸冰洗脸,反倒妨碍了人。心里一时挂火,就也大声说: 
“离着这么远,会弄脏你的菜!” 
我站在上风头,狂风吹送着我的愤怒,我听见洗菜的人也恼了,那人说: 
“菜是下口的东西呀!你在上流洗脸洗屁股,为什么不脏?” 
“你怎么骂人?”我站立起来转过身去,才看见洗菜的是个女孩子,也不过十六七岁。风吹红了她的脸,象带霜的柿叶,水冻肿了她的手,象上冻的红萝卜。她穿的衣服很单薄,就是那种蓝色的破袄裤。 
十月严冬的河滩上,敌人往返烧毁过几次的村庄的边沿,在寒风里,她抱着一篮子水沤的杨树叶,这该是早饭的食粮。 
不知道为什么,我一时心平气和下来。我说: 
“我错了,我不洗了,你在这块石头上来洗吧!” 
她冷冷地望着我,过了一会才说: 
“你刚在那石头上洗了脸,又叫我站上去洗菜!” 
我笑着说: 
“你看你这人,我在上水洗,你说下水脏,这么一条大河,哪里就能把我脸上的泥土冲到你的菜上去?现在叫你到上水来,我到下水去,你还说不行,那怎么办哩?” 
“怎么办,我还得往上走!” 
她说着,扭着身子逆着河流往上去了。登在一块尖石上,把菜篮浸进水里,把两手插在袄襟底下取暖,望着我笑了。 
我哭不的,也笑不的,只好说: 
“你真讲卫生呀!” 
“我们是真卫生,你是装卫生!你们尽笑我们,说我们山沟里的人不讲卫生,住在我们家里,吃了我们的饭,还刷嘴刷牙,我们的菜饭再不干净,难道还会弄脏了你们的嘴?为什么不连肠子都刷刷干净!”说着就笑的弯下腰去。 
我觉得好笑。可也看见,在她笑着的时候,她的整齐的牙齿洁白的放光。 
“对,你卫生,我们不卫生。”我说。 
“那是假话吗?你们一个饭缸子,也盛饭,也盛菜,也洗脸,也洗脚,也喝水,也尿泡,那是讲卫生吗?”她笑着用两手在冷水里刨抓。 
“这是物质条件不好,不是我们愿意不卫生。等我们打败了日本,占了北平,我们就可以吃饭有吃饭的家伙,喝水有喝水的家伙了,我们就可以一切齐备了。” 
“什么时候,才能打败鬼子?”女孩子望着我,“我们的房,叫他们烧过两三回了!” 
“也许三年,也许五年,也许十年八年。可是不管三年五年,十年八年,我们总是要打下去,我们不会悲观的。”我这样对她讲,当时觉得这样讲了以后,心里很高兴了。 
“光着脚打下去?”女孩子转脸望了我脚上一下,就又低下头去洗菜了。 
我一时没弄清是怎么回事,就问: 
“你说什么?” 
“说什么?”女孩子也装没有听见,“我问你为什么不穿袜子,脚不冷吗?也是卫生吗?” 
“咳!”我也笑了,“这是没有法子么,什么卫生!从九月里就反‘扫荡’,可是我们八路军,是非到十月底不发袜子的。这时候,正在打仗,哪里去找袜子穿呀?” 
“不会买一双?”女孩子低声说。 
“哪里去买呀,尽住小村,不过镇店。”我说。 
“不会求人做一双?” 
“哪里有布呀?就是有布,求谁做去呀?” 
“我给你做。”女孩子洗好菜站起来,“我家就住在那个坡子上,“她用手一指,“你要没有布,我家里有点,还够做一双袜子。” 
她端着菜走了,我在河边上洗了脸。我看了看我那只穿着一双“踢倒山”的鞋子,冻的发黑的脚,一时觉得我对于面前这山,这水,这沙滩,永远不能分离了。 
我洗过脸,回到队上吃了饭,就到女孩子家去。她正在烧火,见了我就说: 
“你这人倒实在,叫你来你就来了。” 
我既然摸准了她的脾气,只是笑了笑,就走进屋里。屋里蒸气腾腾,等了一会,我才看见炕上有一个大娘和一个四十多岁的大伯,围着一盆火坐着。在大娘背后还有一位雪白头发的老大娘。一家人全笑着让我炕上坐。女孩子说: 
“明儿别到河里洗脸去了,到我们这里洗吧,多添一瓢水就够了!” 
大伯说: 
“我们妞儿刚才还笑话你哩!” 
白发老大娘瘪着嘴笑着说: 
“她不会说话,同志,不要和她一样呀!” 
“她很会说话!”我说,“要紧的是她心眼儿好,她看见我光着脚,就心疼我们八路军!” 
大娘从炕角里扯出一块白粗布,说: 
“这是我们妞儿纺了半年线赚的,给我做了一条棉裤,剩下的说给她爹做双袜子,现在先给你做了穿上吧。” 
我连忙说: 
“叫大伯穿吧!要不,我就给钱!” 
“你又装假了,”女孩子烧着火抬起头来,“你有钱吗?” 
大娘说: 
“我们这家人,说了就不能改移。过后再叫她纺,给她爹赚袜子穿。早先,我们这里也不会纺线,是今年春天,家里住了一个女同志,教会了她。还说再过来了,还教她织布哩!你家里的人,会纺线吗?” 
“会纺!”我说,“我们那里是穿洋布哩,是机器织纺的。大娘,等我们打败日本……” 
“占了北平,我们就有洋布穿,就一切齐备!”女孩子接下去,笑了。 
可巧,这几天情况没有变动,我们也不转移。每天早晨,我就到女孩子家里去洗脸。第二天去,袜子已经剪裁好,第三天她已经纳底子了,用的是细细的麻线。她说: 
“你们那里是用麻用线?” 
“用线。”我摸了摸袜底,“在我们那里,鞋底也没有这么厚!” 
“这样坚实。”女孩子说,“保你穿三年,能打败日本不?” 
“能够。”我说。 
第五天,我穿上了新袜子。 
和这一家人熟了,就又成了我新的家,这一家人身体都健壮,又好说笑,女孩子的母亲,看起来比女孩子的父亲还要健壮。女孩子的姥姥九十岁了,还那么结实,耳朵也不聋,我们说话的时候,她不插言,只是微微笑着,她说:她很喜欢听人们说闲话。 
女孩子的父亲是个生产的好手,现在地里没活了,他正计划贩红枣到曲阳去卖,问我能不能帮他的忙。部队重视民运工作,上级允许我帮老乡去作运输,每天打早起,我同大伯背上一百多斤红枣,顺着河滩,爬山越岭,送到曲阳去。女孩子早起晚睡给我们做饭,饭食很好,一天,大伯说: 
“同志,你知道我是沾你的光吗?” 
“怎么沾了我的光?” 
“往年,我一个人背枣,我们妞儿是不会给我吃这么好的!” 
我笑了。女孩子说: 
“沾他什么,他穿了我们的袜子,就该给我们做活了!” 
又说: 
“你们跑了快半月,赚了多少钱?” 
“你看,她来查账了,”大伯说,“真是,我们也该计算计算了!”他打开放在被垒底下的一个小包袱,“我们这叫包袱账,赚了赔了,反正都在这里面。” 
我们一同数了票子,一共赚了五千多块钱,女孩子说: 
“够了。” 
“够干什么了?”大伯问。 
“够给我买张织布机子了!这一趟,你们在曲阳给我买架织布机子回来吧!” 
无论姥姥、母亲、父亲和我,都没人反对女孩子这个正义的要求。我们到了曲阳,把枣卖了,就去买了一架机子。大伯不怕多花钱,一定要买一架好的,把全部盈余都用光了。我们分着背了回来,累的浑身流汗。 
这一天,这一家人最高兴,也该是女孩子最满意的一天。这象要了几亩地,买回一头牛;这象制好了结婚前的陪送。 
以后,女孩子就学习纺织的全套手艺了:纺,拐,浆,落,经,镶,织。 
当她卸下第一匹布的那天,我出发了。从此以后,我走遍山南塞北,那双袜子,整整穿了三年也没有破绽。一九四五年,我们战胜了日本强盗,我从延安回来,在碛口地方,跳到黄河里去洗了一个澡,一时大意,奔腾的黄水,冲走了我的全部衣物,也冲走了那双袜子。黄河的波浪激荡着我关于敌后几年生活的回忆,激荡着我对于那女孩子的纪念。 
开国典礼那天,我同大伯一同到百货公司去买布,送他和大娘一人一身蓝士林布,另外,送给女孩子一身红色的。大伯没见过这样鲜艳的红布,对我说: 
“多买上几尺,再买点黄色的!” 
“干什么用?”我问。 
“这里家家门口挂着新旗,咱那山沟里准还没有哩!你给了我一张国旗的样子,一块带回去,叫妞儿给做一个,开会过年的时候,挂起来!” 
他说妞儿已经有两个孩子了,还象小时那样,就是喜欢新鲜东西,说什么也要学会。 

\hfill 1949年12月

备注:1950年2月,孙犁的短篇小说《山地回忆》发表于《小说》杂志第3卷第4期。