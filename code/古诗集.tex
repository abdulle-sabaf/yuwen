\documentclass[12pt,UTF-8,openany]{ctexbook}
\usepackage{ctex}
\usepackage{titlesec}
\usepackage{xeCJK}
\usepackage{fontspec,xunicode,xltxtra}
\usepackage{xpinyin}
\usepackage{geometry}
\usepackage{indentfirst}
\usepackage{pifont}
\usepackage[perpage,symbol*]{footmisc}

\geometry{a5paper,left=1.4cm,right=1.4cm,top=2.4cm,bottom=2.4cm}
\setmainfont{Arial}
\setCJKmainfont[BoldFont=STZhongsong]{汉字之美仿宋GBK 免费}
\xeCJKDeclareCharClass{CJK}{`0 -> `9}
\xeCJKsetup{AllowBreakBetweenPuncts=true}
\DefineFNsymbols{circled}{{\ding{192}}{\ding{193}}{\ding{194}}{\ding{195}}{\ding{196}}{\ding{197}}{\ding{198}}{\ding{199}}{\ding{200}}{\ding{201}}}
\setfnsymbol{circled}
\xpinyinsetup{ratio=0.44,hsep={.6em plus .6em},vsep={1em}}
\titleformat{\chapter}{\zihao{-1}\bfseries}{ }{16pt}{}
\titleformat{\section}{\zihao{-2}\bfseries}{ }{0pt}{}
\title{\zihao{0} \bfseries 小学语文古诗集}
\setlength{\lineskip}{24pt}
\setlength{\parskip}{6pt}
\author{}
\date{}
\begin{document}
\maketitle
\tableofcontents
\newpage

\chapter{第一层}

\section{江南}

\begin{center}
    \vspace{10pt}
    
    \begin{normalsize}
        
        〔无名氏〕
        
    \end{normalsize}
    
    \vspace{8pt}
    
    \begin{large}
        
        \xpinyin*{江南可采莲,}
        
        \xpinyin*{莲叶何田田,}
        
        \xpinyin*{鱼戏莲叶间。}
        
        \xpinyin*{鱼戏莲叶东,}
        
        \xpinyin*{鱼戏莲叶西,}
        
        \xpinyin*{鱼戏莲叶南,}
        
        \xpinyin*{鱼戏莲叶北。}
        
    \end{large}
    
\end{center}

\vspace{8pt}


\chapter{第二层}

\section{画}

\begin{center}
    \vspace{10pt}
    
    \begin{normalsize}
        
        〔无名氏〕
        
    \end{normalsize}
    
    \vspace{8pt}
    
    \begin{large}
        
        \xpinyin*{远看山有色,近听水无声。}
        
        \xpinyin*{春去花还在,人来鸟不惊。}
        
    \end{large}
    
\end{center}

\vspace{8pt}


\section{春晓}

\begin{center}
    \vspace{10pt}
    
    \begin{normalsize}
        
        〔孟浩然〕
        
    \end{normalsize}
    
    \vspace{8pt}
    
    \begin{large}
        
        \xpinyin*{春眠不觉晓,处处闻啼鸟。}
        
        \xpinyin*{夜来风雨声,花落知多少。}
        
    \end{large}
    
\end{center}

\vspace{8pt}


\section{静夜思}

\begin{center}
    \vspace{10pt}
    
    \begin{normalsize}
        
        〔李白〕
        
    \end{normalsize}
    
    \vspace{8pt}
    
    \begin{large}
        
        \xpinyin*{床前明月光,疑是地上霜。}
        
        \xpinyin*{举头望明月,低头思故乡。}
        
    \end{large}
    
\end{center}

\vspace{8pt}


\section{登鹳雀楼}

\begin{center}
    \vspace{10pt}
    
    \begin{normalsize}
        
        〔王之涣〕
        
    \end{normalsize}
    
    \vspace{8pt}
    
    \begin{large}
        
        \xpinyin*{白日依山尽,黄河入海流。}
        
        \xpinyin*{欲穷千里目,更上一层楼。}
        
    \end{large}
    
\end{center}

\vspace{8pt}


\chapter{第三层}

\section{草}

\begin{center}
    \vspace{10pt}
    
    \begin{normalsize}
        
        〔白居易〕
        
    \end{normalsize}
    
    \vspace{8pt}
    
    \begin{large}
        
        \xpinyin*{离离原上草,一岁一枯荣。}
        
        \xpinyin*{野火烧不尽,春风吹又生。}
        
    \end{large}
    
\end{center}

\vspace{8pt}


\section{敕勒歌}

\begin{center}
    \vspace{10pt}
    
    \begin{normalsize}
        
        〔作者不详〕
        
    \end{normalsize}
    
    \vspace{8pt}
    
    \begin{large}
        
        \xpinyin*{敕勒川,阴山下。}
        
        \xpinyin*{天似穹庐,笼盖四野。}
        
        \xpinyin*{天苍苍,野茫茫。}
        
        \xpinyin*{风吹草低见牛羊}
        
    \end{large}
    
\end{center}

\vspace{8pt}


\section{早发白帝城}

\begin{center}
    \vspace{10pt}
    
    \begin{normalsize}
        
        〔李白〕
        
    \end{normalsize}
    
    \vspace{8pt}
    
    \begin{large}
        
        \xpinyin*{朝辞白帝彩云间,千里江陵一日还。}
        
        \xpinyin*{两岸猿声啼不住,轻舟已过万重山。}
        
    \end{large}
    
\end{center}

\vspace{8pt}


\section{山行}

\begin{center}
    \vspace{10pt}
    
    \begin{normalsize}
        
        〔杜牧〕
        
    \end{normalsize}
    
    \vspace{8pt}
    
    \begin{large}
        
        \xpinyin*{远上寒山石径斜,白云深处有人家。}
        
        \xpinyin*{停车坐爱枫林晚,霜叶红于二月花。}
        
    \end{large}
    
\end{center}

\vspace{8pt}


\section{咏柳}

\begin{center}
    \vspace{10pt}
    
    \begin{normalsize}
        
        〔贺知章〕
        
    \end{normalsize}
    
    \vspace{8pt}
    
    \begin{large}
        
        \xpinyin*{碧玉妆成一树高,万条垂下绿丝绦。}
        
        \xpinyin*{不知细叶谁裁出?二月春风似剪刀。}
        
    \end{large}
    
\end{center}

\vspace{8pt}


\section{大林寺桃花}

\begin{center}
    \vspace{10pt}
    
    \begin{normalsize}
        
        〔白居易〕
        
    \end{normalsize}
    
    \vspace{8pt}
    
    \begin{large}
        
        \xpinyin*{人间四月芳菲尽,山寺桃花始盛开。}
        
        \xpinyin*{长恨春归无觅处,不知转入此中来。}
        
    \end{large}
    
\end{center}

\vspace{8pt}


\section{鸟鸣涧}

\begin{center}
    \vspace{10pt}
    
    \begin{normalsize}
        
        〔王维〕
        
    \end{normalsize}
    
    \vspace{8pt}
    
    \begin{large}
        
        \xpinyin*{人闲桂花落,夜静春山空。}
        
        \xpinyin*{月出惊山鸟,时鸣春涧中。}
        
    \end{large}
    
\end{center}

\vspace{8pt}


\section{咏华山}

\begin{center}
    \vspace{10pt}
    
    \begin{normalsize}
        
        〔寇准〕
        
    \end{normalsize}
    
    \vspace{8pt}
    
    \begin{large}
        
        \xpinyin*{只有天在上,更无山与齐。}
        
        \xpinyin*{举头红日近,回首白云低。}
        
    \end{large}
    
\end{center}

\vspace{8pt}


\section{鹿柴}

\begin{center}
    \vspace{10pt}
    
    \begin{normalsize}
        
        〔王维〕
        
    \end{normalsize}
    
    \vspace{8pt}
    
    \begin{large}
        
        \xpinyin*{空山不见人,但闻人语响。}
        
        \xpinyin*{返景入深林,复照青苔上。}
        
    \end{large}
    
\end{center}

\vspace{8pt}


\section{江雪}

\begin{center}
    \vspace{10pt}
    
    \begin{normalsize}
        
        〔柳宗元〕
        
    \end{normalsize}
    
    \vspace{8pt}
    
    \begin{large}
        
        \xpinyin*{千山鸟飞绝,万径人踪灭。}
        
        \xpinyin*{孤舟蓑笠翁,独钓寒江雪。}
        
    \end{large}
    
\end{center}

\vspace{8pt}


\section{寻隐者不遇}

\begin{center}
    \vspace{10pt}
    
    \begin{normalsize}
        
        〔贾岛〕
        
    \end{normalsize}
    
    \vspace{8pt}
    
    \begin{large}
        
        \xpinyin*{松下问童子,言师采药去。}
        
        \xpinyin*{只在此山中,云深不知处。}
        
    \end{large}
    
\end{center}

\vspace{8pt}


\section{悯农(二)}

\begin{center}
    \vspace{10pt}
    
    \begin{normalsize}
        
        〔李绅〕
        
    \end{normalsize}
    
    \vspace{8pt}
    
    \begin{large}
        
        \xpinyin*{锄禾日当午,汗滴禾下土。}
        
        \xpinyin*{谁知盘中餐,粒粒皆辛苦。}
        
    \end{large}
    
\end{center}

\vspace{8pt}


\section{舟夜书所见}

\begin{center}
    \vspace{10pt}
    
    \begin{normalsize}
        
        〔查慎行〕
        
    \end{normalsize}
    
    \vspace{8pt}
    
    \begin{large}
        
        \xpinyin*{月黑见渔灯,孤光一点萤。}
        
        \xpinyin*{微微风簇浪,散作满河星。}
        
    \end{large}
    
\end{center}

\vspace{8pt}


\section{江上渔者}

\begin{center}
    \vspace{10pt}
    
    \begin{normalsize}
        
        〔范仲淹〕
        
    \end{normalsize}
    
    \vspace{8pt}
    
    \begin{large}
        
        \xpinyin*{江上往来人,但爱鲈鱼美。}
        
        \xpinyin*{君看一叶舟,出没风波里。}
        
    \end{large}
    
\end{center}

\vspace{8pt}


\section{蚕妇}

\begin{center}
    \vspace{10pt}
    
    \begin{normalsize}
        
        〔杜荀鹤〕
        
    \end{normalsize}
    
    \vspace{8pt}
    
    \begin{large}
        
        \xpinyin*{昨日入城市,归来泪满巾。}
        
        \xpinyin*{遍身罗绮者,不是养蚕人。}
        
    \end{large}
    
\end{center}

\vspace{8pt}


\section{送孟浩然之广陵}

\begin{center}
    \vspace{10pt}
    
    \begin{normalsize}
        
        〔李白〕
        
    \end{normalsize}
    
    \vspace{8pt}
    
    \begin{large}
        
        \xpinyin*{故人西辞黄鹤楼,烟花三月下扬州。}
        
        \xpinyin*{孤帆远影碧空尽,唯见长江天际流。}
        
    \end{large}
    
\end{center}

\vspace{8pt}


\chapter{第四层}

\section{望庐山瀑布}

\begin{center}
    \vspace{10pt}
    
    \begin{normalsize}
        
        〔李白〕
        
    \end{normalsize}
    
    \vspace{8pt}
    
    \begin{large}
        
        \xpinyin*{日照香炉生紫烟,遥看瀑布挂前川。}
        
        \xpinyin*{飞流直下三千尺,疑是银河落九天。}
        
    \end{large}
    
\end{center}

\vspace{8pt}


\section{绝句}

\begin{center}
    \vspace{10pt}
    
    \begin{normalsize}
        
        〔杜甫〕
        
    \end{normalsize}
    
    \vspace{8pt}
    
    \begin{large}
        
        \xpinyin*{两个黄鹂鸣翠柳,一行白鹭上青天。}
        
        \xpinyin*{窗含西岭千秋雪,门泊东吴万里船。}
        
    \end{large}
    
\end{center}

\vspace{8pt}


\section{凉州词其一}

\begin{center}
    \vspace{10pt}
    
    \begin{normalsize}
        
        〔王之涣〕
        
    \end{normalsize}
    
    \vspace{8pt}
    
    \begin{large}
        
        \xpinyin*{黄河远上白云间,一片孤城万仞山。}
        
        \xpinyin*{羌笛何须怨杨柳,春风不度玉门关。}
        
    \end{large}
    
\end{center}

\vspace{8pt}


\section{滁州西涧}

\begin{center}
    \vspace{10pt}
    
    \begin{normalsize}
        
        〔韦应物〕
        
    \end{normalsize}
    
    \vspace{8pt}
    
    \begin{large}
        
        \xpinyin*{独怜幽草涧边生,上有黄鹂深树鸣。}
        
        \xpinyin*{春潮带雨晚来急,野渡无人舟自横。}
        
    \end{large}
    
\end{center}

\vspace{8pt}


\section{宿新市徐公店}

\begin{center}
    \vspace{10pt}
    
    \begin{normalsize}
        
        〔杨万里〕
        
    \end{normalsize}
    
    \vspace{8pt}
    
    \begin{large}
        
        \xpinyin*{篱落疏疏一径深,树头新绿未成阴。}
        
        \xpinyin*{儿童急走追黄蝶,飞入菜花无处寻。}
        
    \end{large}
    
\end{center}

\vspace{8pt}


\section{小儿垂钓}

\begin{center}
    \vspace{10pt}
    
    \begin{normalsize}
        
        〔胡令能〕
        
    \end{normalsize}
    
    \vspace{8pt}
    
    \begin{large}
        
        \xpinyin*{蓬头稚子学垂纶,侧坐莓苔草映身。}
        
        \xpinyin*{路人借问遥招手,怕得鱼惊不应人。}
        
    \end{large}
    
\end{center}

\vspace{8pt}


\section{夜宿山寺}

\begin{center}
    \vspace{10pt}
    
    \begin{normalsize}
        
        〔李白〕
        
    \end{normalsize}
    
    \vspace{8pt}
    
    \begin{large}
        
        \xpinyin*{危楼高百尺,手可摘星辰。}
        
        \xpinyin*{不敢高声语,恐惊天上人。}
        
    \end{large}
    
\end{center}

\vspace{8pt}


\section{独坐敬亭山}

\begin{center}
    \vspace{10pt}
    
    \begin{normalsize}
        
        〔李白〕
        
    \end{normalsize}
    
    \vspace{8pt}
    
    \begin{large}
        
        \xpinyin*{众鸟高飞尽,孤云独去闲。}
        
        \xpinyin*{相看两不厌,只有敬亭山。}
        
    \end{large}
    
\end{center}

\vspace{8pt}


\section{迟日江山丽}

\begin{center}
    \vspace{10pt}
    
    \begin{normalsize}
        
        〔杜甫〕
        
    \end{normalsize}
    
    \vspace{8pt}
    
    \begin{large}
        
        \xpinyin*{迟日江山丽,春风花草香。}
        
        \xpinyin*{泥融飞燕子,沙暖睡鸳鸯。}
        
    \end{large}
    
\end{center}

\vspace{8pt}


\section{宿建德江}

\begin{center}
    \vspace{10pt}
    
    \begin{normalsize}
        
        〔孟浩然〕
        
    \end{normalsize}
    
    \vspace{8pt}
    
    \begin{large}
        
        \xpinyin*{移舟泊烟渚,日暮客愁新。}
        
        \xpinyin*{野旷天低树,江清月近人。}
        
    \end{large}
    
\end{center}

\vspace{8pt}


\section{游子吟}

\begin{center}
    \vspace{10pt}
    
    \begin{normalsize}
        
        〔孟郊〕
        
    \end{normalsize}
    
    \vspace{8pt}
    
    \begin{large}
        
        \xpinyin*{慈母手中线,游子身上衣。}
        
        \xpinyin*{临行密密缝,意恐迟迟归。}
        
        \xpinyin*{谁言寸草心,报得三春晖。}
        
    \end{large}
    
\end{center}

\vspace{8pt}


\section{梅花}

\begin{center}
    \vspace{10pt}
    
    \begin{normalsize}
        
        〔王安石〕
        
    \end{normalsize}
    
    \vspace{8pt}
    
    \begin{large}
        
        \xpinyin*{墙角数枝梅,凌寒独自开。}
        
        \xpinyin*{遥知不是雪,为有暗香来。}
        
    \end{large}
    
\end{center}

\vspace{8pt}


\section{竹里馆}

\begin{center}
    \vspace{10pt}
    
    \begin{normalsize}
        
        〔王维〕
        
    \end{normalsize}
    
    \vspace{8pt}
    
    \begin{large}
        
        \xpinyin*{独坐幽篁里,弹琴复长啸。}
        
        \xpinyin*{深林人不知,明月来相照。}
        
    \end{large}
    
\end{center}

\vspace{8pt}


\section{芙蓉楼送辛渐}

\begin{center}
    \vspace{10pt}
    
    \begin{normalsize}
        
        〔王昌龄 〕
        
    \end{normalsize}
    
    \vspace{8pt}
    
    \begin{large}
        
        \xpinyin*{寒雨连江夜入吴,平明送客楚山孤。}
        
        \xpinyin*{洛阳亲友如相问,一片冰心在玉壶。}
        
    \end{large}
    
\end{center}

\vspace{8pt}


\section{小池}

\begin{center}
    \vspace{10pt}
    
    \begin{normalsize}
        
        〔杨万里〕
        
    \end{normalsize}
    
    \vspace{8pt}
    
    \begin{large}
        
        \xpinyin*{泉眼无声惜细流,树阴照水爱晴柔。}
        
        \xpinyin*{小荷才露尖尖角,早有蜻蜓立上头。}
        
    \end{large}
    
\end{center}

\vspace{8pt}


\section{夏日田园杂兴其一}

\begin{center}
    \vspace{10pt}
    
    \begin{normalsize}
        
        〔范成大〕
        
    \end{normalsize}
    
    \vspace{8pt}
    
    \begin{large}
        
        \xpinyin*{梅子金黄杏子肥,麦花雪白菜花稀。}
        
        \xpinyin*{日长篱落无人过,唯有蜻蜓蛱蝶飞。}
        
    \end{large}
    
\end{center}

\vspace{8pt}


\chapter{第五层}

\section{塞下曲其二}

\begin{center}
    \vspace{10pt}
    
    \begin{normalsize}
        
        〔卢纶〕
        
    \end{normalsize}
    
    \vspace{8pt}
    
    \begin{large}
        
        \xpinyin*{林暗草惊风,将军夜引弓。}
        
        \xpinyin*{平明寻白羽,没在石棱中。}
        
    \end{large}
    
\end{center}

\vspace{8pt}


\section{望天门山}

\begin{center}
    \vspace{10pt}
    
    \begin{normalsize}
        
        〔李白〕
        
    \end{normalsize}
    
    \vspace{8pt}
    
    \begin{large}
        
        \xpinyin*{天门中断楚江开,碧水东流至此回。}
        
        \xpinyin*{两岸青山相对出,孤帆一片日边来。}
        
    \end{large}
    
\end{center}

\vspace{8pt}


\section{题西林壁}

\begin{center}
    \vspace{10pt}
    
    \begin{normalsize}
        
        〔苏轼〕
        
    \end{normalsize}
    
    \vspace{8pt}
    
    \begin{large}
        
        \xpinyin*{横看成岭侧成峰,远近高低各不同。}
        
        \xpinyin*{不识庐山真面目,只缘身在此山中。}
        
    \end{large}
    
\end{center}

\vspace{8pt}


\section{秋浦歌}

\begin{center}
    \vspace{10pt}
    
    \begin{normalsize}
        
        〔李白〕
        
    \end{normalsize}
    
    \vspace{8pt}
    
    \begin{large}
        
        \xpinyin*{白发三千丈,缘愁似个长。}
        
        \xpinyin*{不知明镜里,何处得秋霜?}
        
    \end{large}
    
\end{center}

\vspace{8pt}


\section{望湖楼醉书}

\begin{center}
    \vspace{10pt}
    
    \begin{normalsize}
        
        〔苏轼〕
        
    \end{normalsize}
    
    \vspace{8pt}
    
    \begin{large}
        
        \xpinyin*{黑云翻墨未遮山,白雨跳珠乱入船。}
        
        \xpinyin*{卷地风来忽吹散,望湖楼下水如天。}
        
    \end{large}
    
\end{center}

\vspace{8pt}


\section{惠崇春江晓景}

\begin{center}
    \vspace{10pt}
    
    \begin{normalsize}
        
        〔苏轼〕
        
    \end{normalsize}
    
    \vspace{8pt}
    
    \begin{large}
        
        \xpinyin*{竹外桃花三两枝,春江水暖鸭先知。}
        
        \xpinyin*{蒌蒿满地芦芽短,正是河豚欲上时。}
        
    \end{large}
    
\end{center}

\vspace{8pt}


\section{早春呈水部张十八员外}

\begin{center}
    \vspace{10pt}
    
    \begin{normalsize}
        
        〔韩愈〕
        
    \end{normalsize}
    
    \vspace{8pt}
    
    \begin{large}
        
        \xpinyin*{天街小雨润如酥,草色遥看近却无。 }
        
        \xpinyin*{最是一年春好处,绝胜烟柳满皇都。}
        
    \end{large}
    
\end{center}

\vspace{8pt}


\section{江畔独步寻花}

\begin{center}
    \vspace{10pt}
    
    \begin{normalsize}
        
        〔杜甫〕
        
    \end{normalsize}
    
    \vspace{8pt}
    
    \begin{large}
        
        \xpinyin*{黄四娘家花满蹊,千朵万朵压枝低。}
        
        \xpinyin*{流连戏蝶时时舞,自在娇莺恰恰啼。}
        
    \end{large}
    
\end{center}

\vspace{8pt}


\section{游园不值}

\begin{center}
    \vspace{10pt}
    
    \begin{normalsize}
        
        〔叶绍翁〕
        
    \end{normalsize}
    
    \vspace{8pt}
    
    \begin{large}
        
        \xpinyin*{应怜屐齿印苍苔,小扣柴扉久不开。}
        
        \xpinyin*{春色满园关不住,一枝红杏出墙来。}
        
    \end{large}
    
\end{center}

\vspace{8pt}


\section{书湖阴先生壁}

\begin{center}
    \vspace{10pt}
    
    \begin{normalsize}
        
        〔王安石〕
        
    \end{normalsize}
    
    \vspace{8pt}
    
    \begin{large}
        
        \xpinyin*{茅檐长扫净无苔,花木成畦手自栽。}
        
        \xpinyin*{一水护田将绿绕,两山排闼送青来。}
        
    \end{large}
    
\end{center}

\vspace{8pt}


\section{元日}

\begin{center}
    \vspace{10pt}
    
    \begin{normalsize}
        
        〔王安石〕
        
    \end{normalsize}
    
    \vspace{8pt}
    
    \begin{large}
        
        \xpinyin*{爆竹声中一岁除,春风送暖入屠苏。}
        
        \xpinyin*{千门万户曈曈日,总把新桃换旧符。}
        
    \end{large}
    
\end{center}

\vspace{8pt}


\section{晓出净慈寺送林子方}

\begin{center}
    \vspace{10pt}
    
    \begin{normalsize}
        
        〔杨万里〕
        
    \end{normalsize}
    
    \vspace{8pt}
    
    \begin{large}
        
        \xpinyin*{毕竟西湖六月中,风光不与四时同。}
        
        \xpinyin*{接天莲叶无穷碧,映日荷花别样红。}
        
    \end{large}
    
\end{center}

\vspace{8pt}


\section{逢雪宿芙蓉山主人}

\begin{center}
    \vspace{10pt}
    
    \begin{normalsize}
        
        〔刘长卿〕
        
    \end{normalsize}
    
    \vspace{8pt}
    
    \begin{large}
        
        \xpinyin*{日暮苍山远,天寒白屋贫。}
        
        \xpinyin*{柴门闻犬吠,风雪夜归人。}
        
    \end{large}
    
\end{center}

\vspace{8pt}


\section{村居}

\begin{center}
    \vspace{10pt}
    
    \begin{normalsize}
        
        〔高鼎〕
        
    \end{normalsize}
    
    \vspace{8pt}
    
    \begin{large}
        
        \xpinyin*{草长莺飞二月天,拂堤杨柳醉春烟。}
        
        \xpinyin*{儿童散学归来早,忙趁东风放纸鸢。}
        
    \end{large}
    
\end{center}

\vspace{8pt}


\chapter{第六层}

\section{暮江吟}

\begin{center}
    \vspace{10pt}
    
    \begin{normalsize}
        
        〔白居易〕
        
    \end{normalsize}
    
    \vspace{8pt}
    
    \begin{large}
        
        \xpinyin*{一道残阳铺水中,半江瑟瑟半江红。}
        
        \xpinyin*{可怜九月初三夜,露似真珠月似弓。}
        
    \end{large}
    
\end{center}

\vspace{8pt}


\section{江碧鸟逾白}

\begin{center}
    \vspace{10pt}
    
    \begin{normalsize}
        
        〔杜甫〕
        
    \end{normalsize}
    
    \vspace{8pt}
    
    \begin{large}
        
        \xpinyin*{江碧鸟逾白,山青花欲燃。}
        
        \xpinyin*{今春看又过,何日是归年。}
        
    \end{large}
    
\end{center}

\vspace{8pt}


\section{饮湖上初晴后雨}

\begin{center}
    \vspace{10pt}
    
    \begin{normalsize}
        
        〔苏轼〕
        
    \end{normalsize}
    
    \vspace{8pt}
    
    \begin{large}
        
        \xpinyin*{水光潋滟晴方好,山色空濛雨亦奇。}
        
        \xpinyin*{欲把西湖比西子,淡妆浓抹总相宜。}
        
    \end{large}
    
\end{center}

\vspace{8pt}


\section{春夜喜雨}

\begin{center}
    \vspace{10pt}
    
    \begin{normalsize}
        
        〔杜甫〕
        
    \end{normalsize}
    
    \vspace{8pt}
    
    \begin{large}
        
        \xpinyin*{好雨知时节,当春乃发生。}
        
        \xpinyin*{随风潜入夜,润物细无声。}
        
        \xpinyin*{野径云俱黑,江船火独明。}
        
        \xpinyin*{晓看红湿处,花重锦官城。}
        
    \end{large}
    
\end{center}

\vspace{8pt}


\section{赋得古原草送别}

\begin{center}
    \vspace{10pt}
    
    \begin{normalsize}
        
        〔白居易〕
        
    \end{normalsize}
    
    \vspace{8pt}
    
    \begin{large}
        
        \xpinyin*{离离原上草,一岁一枯荣。}
        
        \xpinyin*{野火烧不尽,春风吹又生。}
        
        \xpinyin*{远芳侵古道,晴翠接荒城。}
        
        \xpinyin*{又送王孙去,萋萋满别情。}
        
    \end{large}
    
\end{center}

\vspace{8pt}


\section{关山月}

\begin{center}
    \vspace{10pt}
    
    \begin{normalsize}
        
        〔李白〕
        
    \end{normalsize}
    
    \vspace{8pt}
    
    \begin{large}
        
        \xpinyin*{明月出天山,苍茫云海间。}
        
        \xpinyin*{长风几万里,吹度玉门关。}
        
        \xpinyin*{汉下白登道,胡窥青海湾。}
        
        \xpinyin*{由来征战地,不见有人还。}
        
        \xpinyin*{戍客望边色,思归多苦颜。}
        
        \xpinyin*{高楼当此夜,叹息未应闲。}
        
    \end{large}
    
\end{center}

\vspace{8pt}


\section{十一月四日风雨大作}

\begin{center}
    \vspace{10pt}
    
    \begin{normalsize}
        
        〔陆游〕
        
    \end{normalsize}
    
    \vspace{8pt}
    
    \begin{large}
        
        \xpinyin*{僵卧孤村不自哀,尚思为国戍轮台。}
        
        \xpinyin*{夜阑卧听风吹雨,铁马冰河入梦来。}
        
    \end{large}
    
\end{center}

\vspace{8pt}


\section{别董大}

\begin{center}
    \vspace{10pt}
    
    \begin{normalsize}
        
        〔陆游〕
        
    \end{normalsize}
    
    \vspace{8pt}
    
    \begin{large}
        
        \xpinyin*{千里黄云白日曛,北风吹雁雪纷纷。}
        
        \xpinyin*{莫愁前路无知己,天下谁人不识君?}
        
    \end{large}
    
\end{center}

\vspace{8pt}


\section{泊船瓜洲}

\begin{center}
    \vspace{10pt}
    
    \begin{normalsize}
        
        〔王安石〕
        
    \end{normalsize}
    
    \vspace{8pt}
    
    \begin{large}
        
        \xpinyin*{京口瓜洲一水间,钟山只隔数重山。}
        
        \xpinyin*{春风又绿江南岸,明月何时照我还。}
        
    \end{large}
    
\end{center}

\vspace{8pt}


\section{秋思}

\begin{center}
    \vspace{10pt}
    
    \begin{normalsize}
        
        〔张籍〕
        
    \end{normalsize}
    
    \vspace{8pt}
    
    \begin{large}
        
        \xpinyin*{洛阳城里见秋风,欲作家书意万重。}
        
        \xpinyin*{复恐匆匆说不尽,行人临发又开封。}
        
    \end{large}
    
\end{center}

\vspace{8pt}


\section{八阵图}

\begin{center}
    \vspace{10pt}
    
    \begin{normalsize}
        
        〔杜甫〕
        
    \end{normalsize}
    
    \vspace{8pt}
    
    \begin{large}
        
        \xpinyin*{功盖三分国,名成八阵图。}
        
        \xpinyin*{江流石不转,遣恨失吞吴。}
        
    \end{large}
    
\end{center}

\vspace{8pt}


\section{渭城曲}

\begin{center}
    \vspace{10pt}
    
    \begin{normalsize}
        
        〔王维〕
        
    \end{normalsize}
    
    \vspace{8pt}
    
    \begin{large}
        
        \xpinyin*{渭城朝雨浥轻尘,客舍青青柳色新。}
        
        \xpinyin*{劝君更尽一杯酒,西出阳关无故人。}
        
    \end{large}
    
\end{center}

\vspace{8pt}


\section{清明}

\begin{center}
    \vspace{10pt}
    
    \begin{normalsize}
        
        〔杜牧〕
        
    \end{normalsize}
    
    \vspace{8pt}
    
    \begin{large}
        
        \xpinyin*{清明时节雨纷纷,路上行人欲断魂。}
        
        \xpinyin*{借问酒家何处有,牧童遥指杏花村。}
        
    \end{large}
    
\end{center}

\vspace{8pt}


\chapter{第七层}

\section{塞下曲其一}

\begin{center}
    \vspace{10pt}
    
    \begin{normalsize}
        
        〔卢纶〕
        
    \end{normalsize}
    
    \vspace{8pt}
    
    \begin{large}
        
        \xpinyin*{月黑雁飞高,单于夜遁逃。}
        
        \xpinyin*{欲将轻骑逐,大雪满弓刀。}
        
    \end{large}
    
\end{center}

\vspace{8pt}


\section{送元二使安西}

\begin{center}
    \vspace{10pt}
    
    \begin{normalsize}
        
        〔王维〕
        
    \end{normalsize}
    
    \vspace{8pt}
    
    \begin{large}
        
        \xpinyin*{渭城朝雨浥轻尘,客舍青青柳色新。}
        
        \xpinyin*{劝君更尽一杯酒,西出阳关无故人。}
        
    \end{large}
    
\end{center}

\vspace{8pt}


\section{江村即事}

\begin{center}
    \vspace{10pt}
    
    \begin{normalsize}
        
        〔司空曙〕
        
    \end{normalsize}
    
    \vspace{8pt}
    
    \begin{large}
        
        \xpinyin*{钓罢归来不系船,江村月落正堪眠。}
        
        \xpinyin*{纵然一夜风吹去,只在芦花浅水边。}
        
    \end{large}
    
\end{center}

\vspace{8pt}


\chapter{第八层}

\section{出塞}

\begin{center}
    \vspace{10pt}
    
    \begin{normalsize}
        
        〔王昌龄〕
        
    \end{normalsize}
    
    \vspace{8pt}
    
    \begin{large}
        
        \xpinyin*{秦时明月汉时关,万里长征人未还。}
        
        \xpinyin*{但使龙城飞将在,不教胡马渡阴山。}
        
    \end{large}
    
\end{center}

\vspace{8pt}


\section{春日偶成}

\begin{center}
    \vspace{10pt}
    
    \begin{normalsize}
        
        〔程颢〕
        
    \end{normalsize}
    
    \vspace{8pt}
    
    \begin{large}
        
        \xpinyin*{云淡风轻近午天,傍花随柳过前川。}
        
        \xpinyin*{时人不识余心乐,将谓偷闲学少年。}
        
    \end{large}
    
\end{center}

\vspace{8pt}


\section{九月九日忆山东兄弟}

\begin{center}
    \vspace{10pt}
    
    \begin{normalsize}
        
        〔王维〕
        
    \end{normalsize}
    
    \vspace{8pt}
    
    \begin{large}
        
        \xpinyin*{独在异乡为异客,每逢佳节倍思亲。}
        
        \xpinyin*{遥知兄弟登高处,遍插茱萸少一人。}
        
    \end{large}
    
\end{center}

\vspace{8pt}


\section{观书有感}

\begin{center}
    \vspace{10pt}
    
    \begin{normalsize}
        
        〔朱熹〕
        
    \end{normalsize}
    
    \vspace{8pt}
    
    \begin{large}
        
        \xpinyin*{半亩方塘一鉴开,天光云影共徘徊。}
        
        \xpinyin*{问渠哪得清如许?为有源头活水来。}
        
    \end{large}
    
\end{center}

\vspace{8pt}


\chapter{第九层}

\section{凉州词}

\begin{center}
    \vspace{10pt}
    
    \begin{normalsize}
        
        〔王翰〕
        
    \end{normalsize}
    
    \vspace{8pt}
    
    \begin{large}
        
        \xpinyin*{葡萄美酒夜光杯,欲饮琵琶马上催。}
        
        \xpinyin*{醉卧沙场君莫笑,古来征战几人回。}
        
    \end{large}
    
\end{center}

\vspace{8pt}


\section{从军行}

\begin{center}
    \vspace{10pt}
    
    \begin{normalsize}
        
        〔王昌龄〕
        
    \end{normalsize}
    
    \vspace{8pt}
    
    \begin{large}
        
        \xpinyin*{青海长云暗雪山,孤城遥望玉门关。}
        
        \xpinyin*{黄沙百战穿金甲,不破楼兰终不还。}
        
    \end{large}
    
\end{center}

\vspace{8pt}


\section{悯农(一)}

\begin{center}
    \vspace{10pt}
    
    \begin{normalsize}
        
        〔李绅〕
        
    \end{normalsize}
    
    \vspace{8pt}
    
    \begin{large}
        
        \xpinyin*{春种一粒粟,秋收万颗子。}
        
        \xpinyin*{四海无闲田,农夫犹饿死。}
        
    \end{large}
    
\end{center}

\vspace{8pt}


\section{回乡偶书}

\begin{center}
    \vspace{10pt}
    
    \begin{normalsize}
        
        〔贺知章〕
        
    \end{normalsize}
    
    \vspace{8pt}
    
    \begin{large}
        
        \xpinyin*{少小离家老大回,乡音无改鬓毛衰。}
        
        \xpinyin*{儿童相见不相识,笑问客从何处来。}
        
    \end{large}
    
\end{center}

\vspace{8pt}


\section{长歌行}

\begin{center}
    \vspace{10pt}
    
    \begin{normalsize}
        
        〔汉·无名氏〕
        
    \end{normalsize}
    
    \vspace{8pt}
    
    \begin{large}
        
        \xpinyin*{青青园中葵,朝露待日晞。}
        
        \xpinyin*{阳春布德泽,万物生光辉。}
        
        \xpinyin*{常恐秋节至,焜黄华叶衰。}
        
        \xpinyin*{百川东到海,何时复西归?}
        
        \xpinyin*{少壮不努力,老大徒伤悲。}
        
    \end{large}
    
\end{center}

\vspace{8pt}


\section{赠汪伦}

\begin{center}
    \vspace{10pt}
    
    \begin{normalsize}
        
        〔李白〕
        
    \end{normalsize}
    
    \vspace{8pt}
    
    \begin{large}
        
        \xpinyin*{李白乘舟将欲行,忽闻岸上踏歌声。}
        
        \xpinyin*{桃花潭水深千尺,不及汪伦送我情。}
        
    \end{large}
    
\end{center}

\vspace{8pt}


\chapter{第十层}

\section{夏日绝句}

\begin{center}
    \vspace{10pt}
    
    \begin{normalsize}
        
        〔李清照〕
        
    \end{normalsize}
    
    \vspace{8pt}
    
    \begin{large}
        
        \xpinyin*{生当作人杰,死亦为鬼雄。}
        
        \xpinyin*{至今思项羽,不肯过江东。}
        
    \end{large}
    
\end{center}

\vspace{8pt}


\section{墨梅}

\begin{center}
    \vspace{10pt}
    
    \begin{normalsize}
        
        〔王冕〕
        
    \end{normalsize}
    
    \vspace{8pt}
    
    \begin{large}
        
        \xpinyin*{我家洗砚池头树,朵朵花开淡墨痕。}
        
        \xpinyin*{不要人夸好颜色,只留清气满乾坤。}
        
    \end{large}
    
\end{center}

\vspace{8pt}


\section{竹石}

\begin{center}
    \vspace{10pt}
    
    \begin{normalsize}
        
        〔郑燮〕
        
    \end{normalsize}
    
    \vspace{8pt}
    
    \begin{large}
        
        \xpinyin*{咬定青山不放松,立根原在破岩中。}
        
        \xpinyin*{千磨万击还坚劲,任尔东西南北风。}
        
    \end{large}
    
\end{center}

\vspace{8pt}


\section{乐游原}

\begin{center}
    \vspace{10pt}
    
    \begin{normalsize}
        
        〔李商隐〕
        
    \end{normalsize}
    
    \vspace{8pt}
    
    \begin{large}
        
        \xpinyin*{向晚意不适,驱车登古原。}
        
        \xpinyin*{夕阳无限好,只是近黄昏。}
        
    \end{large}
    
\end{center}

\vspace{8pt}


\section{渡汉江}

\begin{center}
    \vspace{10pt}
    
    \begin{normalsize}
        
        〔宋之问〕
        
    \end{normalsize}
    
    \vspace{8pt}
    
    \begin{large}
        
        \xpinyin*{岭外音书断,经冬复历春。}
        
        \xpinyin*{近乡情更怯,不敢问来人。}
        
    \end{large}
    
\end{center}

\vspace{8pt}


\section{题临安邸}

\begin{center}
    \vspace{10pt}
    
    \begin{normalsize}
        
        〔林升〕
        
    \end{normalsize}
    
    \vspace{8pt}
    
    \begin{large}
        
        \xpinyin*{山外青山楼外楼,西湖歌舞几时休。}
        
        \xpinyin*{暖风熏得游人醉,直把杭州作汴州。}
        
    \end{large}
    
\end{center}

\vspace{8pt}


\section{枫桥夜泊}

\begin{center}
    \vspace{10pt}
    
    \begin{normalsize}
        
        〔张继〕
        
    \end{normalsize}
    
    \vspace{8pt}
    
    \begin{large}
        
        \xpinyin*{月落乌啼霜满天,江枫渔火对愁眠。}
        
        \xpinyin*{姑苏城外寒山寺,夜半钟声到客船。}
        
    \end{large}
    
\end{center}

\vspace{8pt}


\section{石灰吟}

\begin{center}
    \vspace{10pt}
    
    \begin{normalsize}
        
        〔于谦〕
        
    \end{normalsize}
    
    \vspace{8pt}
    
    \begin{large}
        
        \xpinyin*{千锤万凿出深山,烈火焚烧若等闲。}
        
        \xpinyin*{粉身碎骨浑不怕,要留清白在人间。}
        
    \end{large}
    
\end{center}

\vspace{8pt}


\section{乌衣巷}

\begin{center}
    \vspace{10pt}
    
    \begin{normalsize}
        
        〔刘禹锡〕
        
    \end{normalsize}
    
    \vspace{8pt}
    
    \begin{large}
        
        \xpinyin*{朱雀桥边野草花,乌衣巷口夕阳斜。}
        
        \xpinyin*{旧时王谢堂前燕,飞入寻常百姓家。}
        
    \end{large}
    
\end{center}

\vspace{8pt}


\section{江南春}

\begin{center}
    \vspace{10pt}
    
    \begin{normalsize}
        
        〔杜牧〕
        
    \end{normalsize}
    
    \vspace{8pt}
    
    \begin{large}
        
        \xpinyin*{千里莺啼绿映红,水村山郭酒旗风。}
        
        \xpinyin*{南朝四百八十寺,多少楼台烟雨中。}
        
    \end{large}
    
\end{center}

\vspace{8pt}


\section{示儿}

\begin{center}
    \vspace{10pt}
    
    \begin{normalsize}
        
        〔陆游〕
        
    \end{normalsize}
    
    \vspace{8pt}
    
    \begin{large}
        
        \xpinyin*{死去元知万事空,但悲不见九州同。}
        
        \xpinyin*{王师北定中原日,家祭无忘告乃翁。}
        
    \end{large}
    
\end{center}

\vspace{8pt}


\section{闻官军收河南河北}

\begin{center}
    \vspace{10pt}
    
    \begin{normalsize}
        
        〔杜甫〕
        
    \end{normalsize}
    
    \vspace{8pt}
    
    \begin{large}
        
        \xpinyin*{剑外忽传收蓟北,忽闻涕泪满衣裳。}
        
        \xpinyin*{去看妻子愁何在,漫卷诗书喜欲狂。}
        
        \xpinyin*{白日放歌须纵酒,青春作伴好还乡。}
        
        \xpinyin*{即从巴峡穿巫峡,便下襄阳向洛阳。}
        
    \end{large}
    
\end{center}

\vspace{8pt}


\section{寒食}

\begin{center}
    \vspace{10pt}
    
    \begin{normalsize}
        
        〔韩翃〕
        
    \end{normalsize}
    
    \vspace{8pt}
    
    \begin{large}
        
        \xpinyin*{春城无处不飞花,寒食东风御柳斜。}
        
        \xpinyin*{日暮汉宫传蜡烛,轻烟散入五侯家。}
        
    \end{large}
    
\end{center}

\vspace{8pt}


\chapter{其他}

\section{己亥杂诗}

\begin{center}
    \vspace{10pt}
    
    \begin{normalsize}
        
        〔龚自珍〕
        
    \end{normalsize}
    
    \vspace{8pt}
    
    \begin{large}
        
        \xpinyin*{九州生气恃风雷,万马齐喑究可哀。}
        
        \xpinyin*{我劝天公重抖擞,不拘一格降人才。}
        
    \end{large}
    
\end{center}

\vspace{8pt}


\section{秋夜将晓出篱门迎凉有感}

\begin{center}
    \vspace{10pt}
    
    \begin{normalsize}
        
        〔陆游〕
        
    \end{normalsize}
    
    \vspace{8pt}
    
    \begin{large}
        
        \xpinyin*{三万里河东入海,五千仞岳上摩天。}
        
        \xpinyin*{遗民泪尽胡尘里,南望王师又一年。}
        
    \end{large}
    
\end{center}

\vspace{8pt}


\section{望月怀远}

\begin{center}
    \vspace{10pt}
    
    \begin{normalsize}
        
        〔张九龄〕
        
    \end{normalsize}
    
    \vspace{8pt}
    
    \begin{large}
        
        \xpinyin*{海上生明月,天涯共此时。}
        
        \xpinyin*{情人怨遥夜,竟夕起相思。}
        
        \xpinyin*{灭烛怜光满,披衣觉露滋。}
        
        \xpinyin*{不堪盈手赠,还寝梦佳期。}
        
    \end{large}
    
\end{center}

\vspace{8pt}


\section{迢迢牵牛星}

\begin{center}
    \vspace{10pt}
    
    \begin{normalsize}
        
        〔无名氏〕
        
    \end{normalsize}
    
    \vspace{8pt}
    
    \begin{large}
        
        \xpinyin*{迢迢牵牛星,皎皎河汉女。}
        
        \xpinyin*{纤纤擢素手,札札弄机杼。}
        
        \xpinyin*{终日不成章,泣涕零如雨。}
        
        \xpinyin*{河汉清且浅,相去复几许。}
        
        \xpinyin*{盈盈一水间,脉脉不得语。}
        
    \end{large}
    
\end{center}

\vspace{8pt}


\section{送杜少府之任蜀州}

\begin{center}
    \vspace{10pt}
    
    \begin{normalsize}
        
        〔王勃〕
        
    \end{normalsize}
    
    \vspace{8pt}
    
    \begin{large}
        
        \xpinyin*{城阙辅三秦,风烟望五津。}
        
        \xpinyin*{与君离别意,同是宦游人。}
        
        \xpinyin*{海内存知己,天涯若比邻。}
        
        \xpinyin*{无为在岐路,儿女共沾巾。}
        
    \end{large}
    
\end{center}

\vspace{8pt}


\section{春望}

\begin{center}
    \vspace{10pt}
    
    \begin{normalsize}
        
        〔杜甫〕
        
    \end{normalsize}
    
    \vspace{8pt}
    
    \begin{large}
        
        \xpinyin*{国破山河在,城春草木深。}
        
        \xpinyin*{感时花溅泪,恨别鸟惊心。}
        
        \xpinyin*{烽火连三月,家书抵万金。}
        
        \xpinyin*{白头搔更短,浑欲不胜簪。}
        
    \end{large}
    
\end{center}

\vspace{8pt}


\section{登岳阳楼}

\begin{center}
    \vspace{10pt}
    
    \begin{normalsize}
        
        〔杜甫〕
        
    \end{normalsize}
    
    \vspace{8pt}
    
    \begin{large}
        
        \xpinyin*{昔闻洞庭水,今上岳阳楼。}
        
        \xpinyin*{吴楚东南坼,乾坤日夜浮。}
        
        \xpinyin*{亲朋无一字,老病有孤舟。}
        
        \xpinyin*{戎马关山北,凭轩涕泗流。}
        
    \end{large}
    
\end{center}

\vspace{8pt}


\section{黄鹤楼}

\begin{center}
    \vspace{10pt}
    
    \begin{normalsize}
        
        〔崔颢〕
        
    \end{normalsize}
    
    \vspace{8pt}
    
    \begin{large}
        
        \xpinyin*{昔人已乘黄鹤去,此地空余黄鹤楼。}
        
        \xpinyin*{黄鹤一去不复返,白云千载空悠悠。}
        
        \xpinyin*{晴川历历汉阳树,芳草萋萋鹦鹉洲。}
        
        \xpinyin*{日暮乡关何处是,烟波江上使人愁。}
        
    \end{large}
    
\end{center}

\vspace{8pt}


\section{题都城南庄}

\begin{center}
    \vspace{10pt}
    
    \begin{normalsize}
        
        〔崔护〕
        
    \end{normalsize}
    
    \vspace{8pt}
    
    \begin{large}
        
        \xpinyin*{去年今日此门中,人面桃花相映红。}
        
        \xpinyin*{人面不知何处去,桃花依旧笑春风。}
        
    \end{large}
    
\end{center}

\vspace{8pt}


\section{长征}

\begin{center}
    \vspace{10pt}
    
    \begin{normalsize}
        
        〔毛泽东〕
        
    \end{normalsize}
    
    \vspace{8pt}
    
    \begin{large}
        
        \xpinyin*{红军不怕远征难,万水千山只等闲。}
        
        \xpinyin*{五岭逶迤腾细浪,乌蒙磅礴走泥丸。}
        
        \xpinyin*{金沙水拍云崖暖,大渡桥横铁索寒。}
        
        \xpinyin*{更喜岷山千里雪,三军过后尽开颜。}
        
    \end{large}
    
\end{center}

\vspace{8pt}


\end{document}
