\documentclass[12pt,UTF8]{ctexbook}
\usepackage{ctex}
\usepackage{array}
\usepackage{graphicx}
\usepackage{wrapfig}
\usepackage[table,dvipsnames]{xcolor}
\usepackage{tabularx}
\usepackage{amsmath}
\usepackage{amssymb}
\usepackage{xfrac}
\usepackage{eucal}
\usepackage{titlesec}
\usepackage{amsthm}
\usepackage{tikz-cd}
\usepackage{enumitem}
\usepackage{verbatim}
\usepackage{fontspec,xunicode,xltxtra}
\usepackage{xeCJK} 

\definecolor{gl}{RGB}{246, 252, 240}
\definecolor{gd}{RGB}{236, 244, 230}
\definecolor{bg}{RGB}{242, 244, 228}


\setCJKmainfont[BoldFont=STZhongsong]{STSong}
\setCJKmonofont{simkai.ttf} % for \texttt
\setCJKsansfont{simfang.ttf} % for \textsf
\setlength\parskip{8pt}
\setlength{\fboxsep}{12pt}
\renewcommand\thesection{\arabic{chapter}.\arabic{section}}
\newtheorem{df}{定义}[section] 
\newtheorem{pp}{命题}[section]
\newtheorem{tm}{定理}[section]
\newtheorem{ex}{例子}[section]
\newtheorem{et}{例题}[section]
\newtheorem{sk}{思考}[section]
\newtheorem{po}{公理}
\newtheorem*{so}{解答}
\newenvironment{proof2}{\paragraph{\textbf{证明:}}}{\hfill$\square$}
\newtheorem{xt}{习题}[section]
\newtheorem{cor}{推论}[pp]
% 列举环境的行间距
\setenumerate[1]{itemsep=0pt,partopsep=0pt,parsep=0pt,topsep=0pt}
\setitemize[1]{itemsep=0pt,partopsep=0pt,parsep=0pt,topsep=0pt}
\setdescription{itemsep=0pt,partopsep=0pt,parsep=0pt,topsep=0pt}
% 章节字体大小
\titleformat{\section}{\zihao{-2}\bfseries}{ \thesection }{16pt}{}
% 封面
\title{\zihao{0} \bfseries 第一册}
\author{\zihao{2} \texttt{大青花鱼}}
% \date{\bfseries\today}
\date{}
% 正文
\begin{document}
\maketitle
\tableofcontents
\newpage

\chapter{第一层}

\section{江南}

江南可采莲,莲叶何田田。
鱼戏莲叶间。
鱼戏莲叶东,
鱼戏莲叶西,
鱼戏莲叶南,
鱼戏莲叶北。

\section{画}

远看山有色,
近听水无声。
春去花还在,
人来鸟不惊。

\chapter{第二层}

\section{春晓}

孟浩然

春眠不觉晓,
处处闻啼鸟。
夜来风雨声,
花落知多少。

\section{静夜思}

李白

床前明月光,
疑是地上霜。
举头望明月,
低头思故乡。

\section{}

\section{登鹳雀楼}

白日依山尽,
黄河入海流。
欲穷千里目,
更上一层楼。

\section{草}

离离原上草,
一岁一枯荣。
野火烧不尽,
春风吹又生。

\chapter{第三层}

\section{望庐山瀑布}

李白

日照香炉生紫烟,遥看瀑布挂前川。
飞流直下三千尺,疑是银河落九天。

\section{绝句}

杜甫

两个黄鹂鸣翠柳,一行白鹭上青天
窗含西岭千秋雪,门泊东吴万里船

\section{早发白帝城}

朝辞白帝彩云间,
千里江陵一日还。
两岸猿声啼不住,
轻舟已过万重山。

\section{山行}

远上寒山石径斜,
白云深处有人家。
停车坐爱枫林晚,
霜叶红于二月花。

\section{咏柳}

碧玉妆成一树高,
万条垂下绿丝绦。
不知细叶谁裁出?
二月春风似剪刀。

\section{大林寺桃花}

白居易

人间四月芳菲尽,
山寺桃花始盛开。
长恨春归无觅处,
不知转入此中来。

\section{鸟鸣涧}

王维

人闲桂花落,
夜静春山空。
月出惊山鸟,
时鸣春涧中。

\section{咏华山}

寇准

只有天在上,
更无山与齐。
举头红日近,
回首白云低。

\section{塞下曲}

卢纶

林暗鸟惊风,
将军夜引弓。
平明寻白羽,
没在石棱中。

\section{望天门山}

天门中断楚江开,
碧水东流至此回。
两岸青山相对出,
孤帆一片日边来。

\section{鹿柴}

王维

空山不见人,
但闻人语响。
返景入深林,
复照青苔上。

\section{江雪}

柳宗元

千山鸟飞绝,
万径人踪灭。
孤舟蓑笠翁,
独钓寒江雪。

\section{寻隐者不遇}

贾岛

松下问童子,
言师采药去。
只在此山中,
云深不知处。

\section{凉州词}

王之涣

黄河远上白云间,
一片孤城万仞山。
羌笛何须怨杨柳,
春风不度玉门关。

\section{出塞}

王昌龄

秦时明月汉时关,
万里长征人未还。
但使龙城飞将在,
不教胡马度阴山。

\section{题西林壁}

苏轼

横看成岭侧成峰,
远近高低各不同。
不识庐山真面目,
只缘身在此山中。

\section{暮江吟}

白居易

一道残阳铺水中,
半江瑟瑟半江红。
可怜九月初三夜,
露似真珠月似弓。

\section{悯农(一)}

李绅

春种一粒粟,
秋收万颗子。
四海无闲田,
农夫犹饿死。

\section{悯农(二)}

李绅

锄禾日当午,
汗滴禾下土。
谁知盘中餐,
粒粒皆辛苦。

\section{舟夜书所见} 

查慎行

月黑见渔灯,
孤光一点萤。
微微风簇浪,
散作满河星。

\section{江上渔者}

范仲淹

江上往来人,
但爱鲈鱼美。
君看一叶舟,
出没风波里。

\section{蚕妇}

昨日入城市,
归来泪满巾。
遍身罗绮者,
不是养蚕人。

\section{早发白帝城}

朝辞白帝彩云间,
千里江陵一日还。
两岸猿声啼不住,
轻舟已过万重山。

\section{山行}

远上寒山石径斜,
白云深处有人家。
停车坐爱枫林晚,
霜叶红于二月花。

\section{送元二使安西}

王维

渭城朝雨浥轻尘,
客舍青青柳色新。
劝君更尽一杯酒,
西出阳关无故人。

\section{送孟浩然之广陵}

李白

故人西辞黄鹤楼,
烟花三月下扬州。
孤帆远影碧空尽,
唯见长江天际流。

\section{滁州西涧}

韦应物

独怜幽草涧边生,
上有黄鹂深树鸣。
春潮带雨晚来急,
野渡无人舟自横。

\chapter{第四层}

\section{宿新市徐公店}

杨万里

篱落疏疏一径深,
树头新绿未成阴。
儿童急走追黄蝶,
飞入菜花无处寻。

\section{小儿垂钓}

胡令能

蓬头稚子学垂纶,
侧坐莓苔草映身。
路人借问遥招手,
怕得鱼惊不应人。

\section{望庐山瀑布}

李白

日照香炉生紫烟,遥看瀑布挂前川。
飞流直下三千尺,疑是银河落九天。

\section{绝句}

杜甫

两个黄鹂鸣翠柳,一行白鹭上青天
窗含西岭千秋雪,门泊东吴万里船

\section{夜宿山寺}

危楼高百尺,
手可摘星辰。
不敢高声语,
恐惊天上人。

\section{宿建德江}

孟浩然

移舟泊烟渚,
日暮客愁新。
野旷天低树,
江清月近人。

\section{游子吟}

孟郊

慈母手中线,游子身上衣。
临行密密缝,意恐迟迟归。
谁言寸草心,报得三春晖。

\section{梅花}

王安石

墙角数枝梅,
凌寒独自开。
遥知不是雪,
为有暗香来。

\section{芙蓉楼送辛渐}

王昌龄 

寒雨连江夜入吴,
平明送客楚山孤。
洛阳亲友如相问,
一片冰心在玉壶。

\section{墨梅}

王冕

我家洗砚池头树,
朵朵花开淡墨痕。
不要人夸好颜色,
只留清气满乾坤。

\chapter{第五层}

\section{早春呈水部张十八员外}

韩愈

天街小雨润如酥,
草色遥看近却无。 
最是一年春好处,
绝胜烟柳满皇都。

\section{江畔独步寻花}

杜甫

黄四娘家花满蹊,
千朵万朵压枝低。
流连戏蝶时时舞,
自在娇莺恰恰啼。

\section{游园不值}

叶绍翁

应怜屐齿印苍苔,
小扣柴扉久不开。
春色满园关不住,
一枝红杏出墙来。

\section{书湖阴先生壁}

王安石

茅檐长扫净无苔,花木成畦手自栽。
一水护田将绿绕,两山排闼送青来。

\section{关山月}

李白

明月出天山,苍茫云海间。
长风几万里,吹度玉门关。
汉下白登道,胡窥青海湾。
由来征战地,不见有人还。
戍客望边色,思归多苦颜。
高楼当此夜,叹息未应闲。

\chapter{第六层}

\section{十一月四日风雨大作}

陆游

僵卧孤村不自哀,
尚思为国戍轮台。
夜阑卧听风吹雨,
铁马冰河入梦来。

\section{泊船瓜洲}

王安石

京口瓜洲一水间,
钟山只隔数重山。
春风又绿江南岸,
明月何时照我还。

\section{秋思}

张籍

洛阳城里见秋风,
欲作家书意万重。
复恐匆匆说不尽,
行人临发又开封。

\section{八阵图}

杜甫

功盖三分国,名成八阵图。
江流石不转,遣恨失吞吴。

\section{渭城曲}

王维

渭城朝雨浥轻尘,客舍青青柳色新。
劝君更尽一杯酒,西出阳关无故人。

\section{清明}

杜牧

清明时节雨纷纷,路上行人欲断魂。
借问酒家何处有,牧童遥指杏花村。

\section{塞下曲}

卢纶

月黑雁飞高,
单于夜遁逃。
欲将轻骑逐,
大雪满弓刀。

\chapter{第七层}

\section{江村即事}

司空曙

钓罢归来不系船,江村月落正堪眠。
纵然一夜风吹去,只在芦花浅水边。

\section{出塞}

王昌龄

秦时明月汉时关,万里长征人未还。
但使龙城飞将在,不教胡马渡阴山。

\chapter{第八层}

\section{春日偶成}

程颢

云淡风轻近午天,
傍花随柳过前川。
时人不识余心乐,
将谓偷闲学少年。

\section{九月九日忆山东兄弟}

王维

独在异乡为异客,
每逢佳节倍思亲。
遥知兄弟登高处,
遍插茱萸少一人。

\section{赠汪伦}

李白

李白乘舟将欲行,
忽闻岸上踏歌声。
桃花潭水深千尺,
不及汪伦送我情。

\chapter{第九层}

\section{凉州词}

王翰

葡萄美酒夜光杯,欲饮琵琶马上催。
醉卧沙场君莫笑,古来征战几人回。

\section{枫桥夜泊}

张继

月落乌啼霜满天,江枫渔火对愁眠。
姑苏城外寒山寺,夜半钟声到客船。

\chapter{第十层}

\section{石灰吟}

于谦

千锤万凿出深山,
烈火焚烧若等闲。
粉身碎骨浑不怕,
要留清白在人间。

\section{乌衣巷}

刘禹锡

朱雀桥边野草花,乌衣巷口夕阳斜。
旧时王谢堂前燕,飞入寻常百姓家。

\section{江南春}

杜牧

千里莺啼绿映红,
水村山郭酒旗风。
南朝四百八十寺,
多少楼台烟雨中。

\section{示儿}

陆游

死去元知万事空,
但悲不见九州同。
王师北定中原日,
家祭无忘告乃翁。

\section{闻官军收河南河北}

杜甫

剑外忽传收蓟北,
忽闻涕泪满衣裳。
去看妻子愁何在,
漫卷诗书喜欲狂。
白日放歌须纵酒,
青春作伴好还乡。
即从巴峡穿巫峡,
便下襄阳向洛阳。

\section{寒食}

韩翃

春城无处不飞花,
寒食东风御柳斜。
日暮汉宫传蜡烛,
轻烟散入五侯家。

\section{秋夜将晓出篱门迎凉有感}

陆游

三万里河东入海,
五千仞岳上摩天。
遗民泪尽胡尘里,
南望王师又一年。

\chapter{其他}

\section{春夜喜雨} 

杜甫 

好雨知时节,当春乃发生。
随风潜入夜,润物细无声。
野径云俱黑,江船火独明。
晓看红湿处,花重锦官城。

\section{望月怀远}

张九龄

海上生明月,天涯共此时。
情人怨遥夜,竟夕起相思。
灭烛怜光满,披衣觉露滋。
不堪盈手赠,还寝梦佳期。

\section{迢迢牵牛星}

迢迢牵牛星,皎皎河汉女。
纤纤擢素手,札札弄机杼。
终日不成章,泣涕零如雨。
河汉清且浅,相去复几许。
盈盈一水间,脉脉不得语。

\section{送杜少府之任蜀州}

王勃

城阙辅三秦,风烟望五津。
与君离别意,同是宦游人。
海内存知己,天涯若比邻。
无为在岐路,儿女共沾巾。

\section{春望}

杜甫

国破山河在,城春草木深。
感时花溅泪,恨别鸟惊心。
烽火连三月,家书抵万金。
白头搔更短,浑欲不胜簪。


\section{登岳阳楼}

杜甫

昔闻洞庭水,今上岳阳楼。
吴楚东南坼,乾坤日夜浮。
亲朋无一字,老病有孤舟。
戎马关山北,凭轩涕泗流。

\section{黄鹤楼}

崔颢

昔人已乘黄鹤去,此地空余黄鹤楼。
黄鹤一去不复返,白云千载空悠悠。
晴川历历汉阳树,芳草萋萋鹦鹉洲。
日暮乡关何处是,烟波江上使人愁。

\section{题都城南庄}

崔护

去年今日此门中,人面桃花相映红。
人面不知何处去,桃花依旧笑春风。

\section{望月怀远}

张九龄

海上生明月,天涯共此时。
情人怨遥夜,竟夕起相思。
灭烛怜光满,披衣觉露滋。
不堪盈手赠,还寝梦佳期。

\section{七律·长征}

红军不怕远征难,
万水千山只等闲。
五岭逶迤腾细浪,
乌蒙磅礴走泥丸。
金沙水拍云崖暖,
大渡桥横铁索寒。
更喜岷山千里雪,
三军过后尽开颜。

\end{document}
