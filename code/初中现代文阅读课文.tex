\documentclass[12pt,UTF-8,openany]{ctexbook}
\usepackage{ctex}
\usepackage{titlesec}
\usepackage{xeCJK}
\usepackage{fontspec,xunicode,xltxtra}
\usepackage{xpinyin}
\usepackage{geometry}
\usepackage{indentfirst}
\usepackage{pifont}
\usepackage[perpage,symbol*]{footmisc}

\geometry{a5paper,left=1.4cm,right=1.4cm,top=2.4cm,bottom=2.4cm}
\setmainfont{Arial}
\setCJKmainfont[BoldFont=STZhongsong]{汉字之美仿宋GBK 免费}
\xeCJKDeclareCharClass{CJK}{`0 -> `9}
\xeCJKsetup{AllowBreakBetweenPuncts=true}
\DefineFNsymbols{circled}{{\ding{192}}{\ding{193}}{\ding{194}}{\ding{195}}{\ding{196}}{\ding{197}}{\ding{198}}{\ding{199}}{\ding{200}}{\ding{201}}}
\setfnsymbol{circled}
\xpinyinsetup{ratio=0.5,hsep={.7em plus .7em},vsep={1em}}
\titleformat{\chapter}{\zihao{-1}\bfseries}{ }{16pt}{}
\titleformat{\section}{\zihao{-2}\bfseries}{ }{0pt}{}
\title{\zihao{0} \bfseries 初中语文课文集萃}
\setlength{\lineskip}{24pt}
\setlength{\parskip}{6pt}
\author{}
\date{}
\begin{document}
\maketitle
\tableofcontents
\newpage

\chapter{匆匆}

\begin{large}
    
    燕子去了,有再来的时候;杨柳枯了,有再青的时候;桃花谢了,有再开的时候。但是,聪明的,你告诉我,我们的日子为什么一去不复返呢?——是有人偷了他们吧:那是谁?又藏在何处呢?是他们自己逃走了吧:现在又到了哪里呢?
    
    我不知道他们给了我们多少日子,但我的手确乎是渐渐空虚了。在默默里算着,八千多日子已经从我手中溜去;像针尖上一滴水滴在大海里,我的日子滴在时间的流里,没有声音,也没有影子。我不禁头涔涔而泪潸潸了。
    
    去的尽管去了,来的尽管来着,去来的中间,又怎样地匆匆呢?早上我起来的时候,小屋里射进两三方斜斜的太阳。太阳他有脚啊,轻轻悄悄地挪移了,我也茫茫然跟着旋转。于是——洗手的时候,日子从水盆里过去;吃饭的时候,日子从饭碗里过去;默默时,便从凝然的双眼前过去;我觉察他去得匆匆了,伸出手遮挽时,他又从遮挽的手边过去;天黑时,我躺在床上,他便伶伶俐俐地从我身上跨过,从我脚边飞去了;等我睁开眼和太阳再见,这算又溜走了一日;我掩面叹息,但是新来的日子的影儿又开始在叹息里闪过了。
    
    在逃去如飞的日子里,在千门万户的世界里的我能做什么呢?只有徘徊罢了,只有匆匆罢了。在八千多日的匆匆里,除徘徊外,又剩些什么呢?过去的日子如轻烟,被微风吹散了,如薄雾,被初阳蒸融了。我留着些什么痕迹呢?我何曾留着像游丝样的痕迹呢?我赤裸裸来到这世界,转眼间也将赤裸裸地回去吧?但不能平的,为什么偏要白白走这一遭啊?
    
    你聪明的,告诉我,我们的日子为什么一去不复返呢?
    
\end{large}



\chapter{冬天的济南}

\begin{large}
    
    对于一个在北平\footnote{〔北平〕北京的旧称。}住惯的人,像我,冬天要是不刮风,便觉得是奇迹;济南的冬天是没有风声的。对于一个刚由伦敦\footnote{〔伦敦〕英国首都。}回来的人,像我,冬天要能看得见日光,便觉得是怪事;济南的冬天是响晴的。自然,在热带\footnote{〔热带〕地球南北回归线之间的地带。四季炎热。}的地方,日光是永远那么毒,响亮的天气,反有点叫人害怕。可是,在北中国的冬天,而能有温晴的天气,济南真得算个宝地。
    
    设若\footnote{〔设若〕如果、假若。}单单是有阳光,那也算不了出奇。请闭上眼睛想:一个老城,有山有水,全在天底下晒着阳光,暖和安适地睡着,只等春风来把它们唤醒,这是不是个理想的境界?小山整把济南围了个圈儿,只有北边缺着点口儿。这一圈小山在冬天特别可爱,好像是把济南放在一个小摇篮里,它们安静不动地低声地说:“你们放心吧,这儿准保暖和。”真的,济南的人们在冬天是面上含笑的。他们一看那些小山,心中便觉得有了着落,有了依靠。他们由天上看到山上,便不知不觉地想起:“明天也许就是春天了吧?这样的温暖,今天夜里山草也许就绿起来了吧?”就是\footnote{〔就是〕这里同“就算”。}这点幻想不能一时实现,他们也并不着急,因为有这样慈善的冬天,干啥还希望别的呢!
    
    最妙的是下点小雪呀。看吧,山上的矮松越发的青黑,树尖上顶着一髻儿白花\footnote{〔髻〕原指头顶或脑后盘成的各种形状的头发。这里比喻树顶上的积雪如发髻一般。},仿佛雪地里的灰松鼠。山尖全白了,给蓝天镶上一道银边。山坡上,有的地方雪厚点,有的地方草色还露着;这样,一道儿白,一道儿暗黄,给山们穿上一件带水纹的花衣;看着看着,这件花衣好像被风儿吹动,叫你希望看见一点更美的山的肌肤。等到快日落的时候,微黄的阳光斜射在山腰上,那点薄雪好像忽然害了羞,微微露出点粉色。就是下小雪吧,济南是受不住大雪的,那些小山太秀气!
    
    古老的济南,城里那么狭窄,城外又那么宽敞,山坡上卧着些小村庄,小村庄的房顶上卧着点雪,对,这是张小水墨画,或者是唐代的名手画的吧。
    
    那水呢,不但不结冰,倒反在绿藻上冒着点热气。水藻真绿,把终年贮蓄的绿色全拿出来了。天儿越晴,水藻越绿,就凭这些绿的精神,水也不忍得冻上;况且那长枝的垂柳还要在水里照个影儿呢!看吧,由澄清的河水慢慢往上看吧,空中,半空中,天上,自上而下全是那么清亮,那么蓝汪汪的,整个的是块空灵的蓝水晶。这块水晶里,包着红屋顶,黄草山,像地毯上的小团花的小灰色树影;这就是冬天的济南。
    
\end{large}


\newpage

\textbf{注释}:

\vspace{-1em}

\begin{itemize}
    \setlength\itemsep{-0.2em}
    \item 〔\xpinyin*{响晴}〕晴朗无云。
    \item 〔\xpinyin*{水墨画}〕用水、墨而不用彩色颜料的国画。
    \item 〔\xpinyin*{名手}〕这里指有名的画家。手:擅长做某事的人。
    \item 〔\xpinyin*{贮蓄}〕储存,积聚。
    \item 〔\xpinyin*{空灵}〕空静而又灵动,难以捉摸。
\end{itemize}

\chapter{紫藤萝瀑布}

\begin{large}
    
    我不由得停住了脚步。
    
    从未见过开得这样盛的藤萝,只见一片辉煌的淡紫色,像一条瀑布,从空中垂下,不见其发端,也不见其终极。只是深深浅浅的紫,仿佛在流动,在欢笑,在不停地生长。紫色的大条幅上,泛着点点银光,就像迸溅的水花。仔细看时,才知道那是每一朵紫花中的最浅淡的部分,在和阳光互相挑逗。
    
    这里春红已谢,没有赏花的人群,也没有蜂围蝶阵。有的就是这一树闪光的、盛开的藤萝。花朵儿一串挨着一串,一朵接着一朵,彼此推着挤着,好不活泼热闹!
    
    “我在开花!”它们在笑。
    
    “我在开花!”它们嚷嚷。
    
    每一穗花都是上面的盛开、下面的待放 。颜色便上浅下深,好像那紫色沉淀下来了,沉淀在最嫩最小的花苞里。每一朵盛开的花就像是一个小小的张满了的帆,帆下带着尖底的舱,船舱鼓鼓的;又像一个忍俊不禁的笑容,就要绽开似的。那里装的是什么仙露琼浆?我凑上去,想摘一朵。
    
    但是我没有摘。我没有摘花的习惯。我只是伫立凝望,觉得这一条紫藤萝瀑布不只在我眼前,也在我心上缓缓流过。流着流着,它带走了这些时一直压在我心上的焦虑和悲痛,那是关于生死谜、手足情的。我沉浸在这繁密的花朵的光辉中,别的一切暂时都不存在,有的只是精神的宁静和生的喜悦。
    
    这里除了光彩,还有淡淡的芳香,香气似乎也是浅紫色的,梦幻一般轻轻地笼罩着我。忽然记起十多年前家门外也曾有过一大株紫藤萝,它依傍一株枯槐爬得很高,但花朵从来都稀落,东一穗西一串伶仃地挂在树梢,好像在试探什么。后来索性连那稀零的花串也没有了。园中别的紫藤花架也都拆掉,改种了果树。那时的说法是,花和生活腐化有什么必然关系。我曾遗憾地想:这里再也看不见藤萝花了。
    
    过了这么多年,藤萝又开花了,而且开得这样盛,这样密,紫色的瀑布遮住了粗壮的盘虬卧龙般的枝干,不断地流着,流着,流向人的心底。
    
    花和人都会遇到各种各样的不幸,但是生命的长河是无止境的。我抚摸了一下那小小的紫色的花舱,那里满装生命的酒酿,它张满了帆,在这闪光的花的河流上航行。它是万花中的一朵,也正是一朵朵花,组成了万花灿烂的流动的瀑布。
    
    在这浅紫色的光辉和浅紫色的芳香中,我不觉加快了脚步。
    
\end{large}



\chapter{太空一日}

\begin{large}
    
    我以为自己要牺牲了
    
    9时整,火箭尾部发出巨大的轰鸣声,数百吨高能燃料开始燃烧,八台发动机同时喷出炽热的火焰,高温高速的气体,几秒钟就把发射台下的上千吨水化为蒸汽。
    
    火箭起飞了。
    
    我全神贯注,肌肉紧绷,整个人收得像一块铁,准备执行动作。
    
    飞船缓缓升起,非常平稳,甚至比电梯还平稳。我感到压力远不像训练时想象的那么大,稍稍释然,全身绷紧的肌肉也渐渐放松下来。
    
    “逃逸塔\footnote{〔逃逸塔〕飞船顶端的逃生装置。可在火箭升空期间出现危急状况时,让航天员迅速脱离危险区域。}分离”,“助推器分离”……
    
    火箭逐渐加速,我感到压力渐渐增强。这种负荷我们训练时承受过,变化幅度甚至比训练时还小些,所以我的身体感受还挺好,觉得没啥问题。
    
    然而,就在火箭上升到三四十公里的高度时,火箭和飞船产生了共振\footnote{〔共振〕物体受外界振动刺激时,产生特别强烈的振动的现象。},开始急剧振动。这让我非常痛苦。
    
    人体对10赫兹\footnote{〔赫兹〕每秒振动的次数。10赫兹表示每秒振动10次。}以下的振动非常敏感。它会让人的内脏产生共振。不仅如此,当时的负荷大约有六倍重力加速度\footnote{〔重力加速度〕重力导致的加速度。六倍重力加速度相当于身体重量变为六倍,感觉如同自身五倍的重量压在全身。},两者叠加,实在太可怕了,我们从来没有进行过这种训练。
    
    意外出现了。
    
    共振时强时弱,痛苦越来越强烈,我异常清醒,只觉得五脏六腑似乎都要碎了。我几乎难以承受,觉得自己快不行了。
    
    当时,我以为飞船起飞时就是这样的。其实,起飞阶段发生的共振并非正常现象。
    
    共振持续26秒后,慢慢减轻。我从极度难受的状态解脱出来,一切不适都不见了,只感到从未有过的轻松和舒服,如释千钧重负,如同重生。我甚至觉得这个过程很耐人寻味。但在痛苦的极点,就在那短短一刹那,我真的以为自己要牺牲了。
    
    飞行回来后,我详细描述了这段难受的过程。经过分析研究,工作人员认为,飞船的共振主要来自火箭的振动。随后他们改进工艺,解决了这个问题。“神舟六号”飞行时,情况有了很大改善;后来的航天飞行中再没有出现过这种问题。聂海胜\footnote{〔聂海胜〕中国航天员。2005年10月,他和费俊龙成功执行“神舟六号”载人航天飞行任务。}说:“我们乘坐的火箭、飞船都非常舒适,几乎感觉不到振动。”
    
    在空中度过那难以承受的26秒时,不仅我感觉特别漫长,地面的工作人员也陷入了空前的紧张中。因为通过大屏幕,飞船传回来的画面是定格的,我整个人一动不动,眼睛也不眨。大家都担心我是不是出了什么事故。
    
    后来,整流罩\footnote{〔整流罩〕套在飞行器上的保护罩。用于减少空气阻力,免除飞行时气流、热流的影响。}打开,外面的光线透过舷窗一下子照射进来,阳光很刺眼,我的眼睛忍不住眨了一下。
    
    就这一下,指挥大厅有人大声喊道:“快看啊,他眨眼了,利伟还活着!”所有的人都鼓掌欢呼起来。
    
    这是回到地面后,我看了升空时指挥大厅的录像才知道的。那一刻,所有的人都在流泪。看到这里的时候,我感动得说不出话来。
    
    我看到了什么
    
    此后一切顺利。升空后10分钟左右,飞船仿佛一下子跳进了轨道。我突然有了失重的感觉。
    
    好容易等到地面指挥人员下达指令,我迫不及待地摘下束缚带,飘到舷窗边上。
    
    哈!太空和地球一下子出现在我眼前。
    
    我先望向地球。从飞船上看到的地球,只是一段弧面,不是完整的球体。因为地球的半径有六千多公里,而飞船距离地面343公里左右。我们平常在地理书上看到的地球照片,是由飞行轨道更高的同步卫星拍摄而来。
    
    地球真的太漂亮了。她散发着柔和的光芒,仿佛披着蓝色纱裙和白色飘带的仙女,款款而行。蓝色的弧面之外,是深远幽黑的宇宙。
    
    飞船每90分钟就绕地球一圈,一共飞行了14圈。我也看了14次日出和日落。我曾在新疆的天山上,也曾站在家乡的大海边看日出,但都无法与太空中的日出相比。一条亮白的金弧不断延伸,太阳就是镶在中间的宝珠,发出炫目的光。金弧逐渐扩散开来,把光明涂抹在广袤的弧面上,一切都清晰起来。日落时,一切又追随着太阳涌去,汇成一条光弧,再彻底消失。
    
    在太空中,我可以准确判断各大洲和各个国家的方位。因为飞船有预定的飞行轨迹,显示屏上实时标示着飞船走到哪个位置,投影到地球上是哪一点。有图可依,一目了然。
    
    即使不借助仪器和地图,以我们航天课程中学到的知识,从山脉的轮廓,海岸线的走向以及河流的形状,我也基本可以判断出飞船正经过哪个洲的上空,正在经过哪个国家。
    
    经过亚洲,特别是到中国上空时,我就仔细辨别大概到哪个省了。飞船经过中国上空的时间很短,每一次飞过后,我都期待着下一次。
    
    飞船的轨迹大都是不重复的,在距离地面三百多公里的高度上俯瞰,视野广阔,祖国的各个省份我大都看到了。
    
    我曾俯瞰我们的首都北京。白天它是燕山山脉边的一片灰白色,分辨不清;夜晚则呈现一片红晕。那里有我的战友和亲人。
    
    我看到中国东部优美的海岸线、长白山脉,那里是辽宁,我的家乡;我看到甘肃、新疆,披着积雪的昆仑山脉和大片沙漠,我曾在那里驾机飞行,也从那里乘火箭升空;我看到了曲折的黄河横穿陕西、山西、山东数省;我看到了西藏和青藏高原,我看到了四川、安徽、江苏、上海,蜿蜒的长江奔向大海;我看到了东南方向的台湾岛,看上去它与大陆几乎没有间隔;我看到了宽广的内蒙古一片平阔,而我将在那里降落……
    
    神秘的敲击声
    
    作为首飞航天员,除了一些小难题,我还遇到了许多突发的、原因不明的、不在预案中的情况。
    
    比如,当飞船刚刚入轨,进入失重状态时,百分之八九十的航天员都会产生一种“本末倒置”的错觉。这种错觉很难受,明明是朝上坐的,却感觉脑袋朝下。如果不消除这种倒悬的错觉,就会觉得自己一直在倒着飞,很难受,严重时还可能诱发空间运动病\footnote{〔空间运动病〕人的空间平衡感失调导致的疾病。晕车、晕船、晕飞机,都属于空间运动病。},影响任务的完成。
    
    在地面时,没人提到过这种情况。即使知道,训练也无法模拟。估计在我之前遨游太空的国外航天员有类似体会,但他们从未对我说起过。
    
    在这个情况下,没别的办法,只能靠意志力克服这种错觉。我想像自己在地面训练的情景,眼睛闭着猛想,不停地想,给身体一个适应的过程。几十分钟后,我终于调整过来了。
    
    “神舟六号”和“神舟七号”升空后,航天员都产生了这种错觉,但他们已有心理准备,因为我跟他们仔细讲过。而且,飞船舱体也经过改进,内壁上下刷了不同的颜色:天花板是白色的,地板是褐色的。这样有助于航天员迅速调整感觉。
    
    我在太空还遇到另一个至今仍然原因不明的情况,那就是时不时出现的敲击声。这个声音是突然出现的。并不一直响,而是一阵一阵的。不管白天还是黑夜,毫无规律,说不准什么时候就响几声。既不是外面传进来的声音,也不是飞船里面的声音,仿佛谁在外面敲飞船的船体。很难准确描述它:不是叮叮的,也不是当当的,更像是用一把木头锤子敲铁桶,咚……咚咚……咚……
    
    鉴于飞船的运行一直很正常,我并没有向地面报告这一情况。但我自己还是很紧张,因为第一次飞行,生怕哪里出了问题。每当响声传来的时候,我就趴在舷窗那里,边听边看,试图找出响声所在,但什么也没能发现。
    
    回到地面后,人们对这个神秘的声音做过许多猜测。技术人员想弄清它到底来自哪里,就用各种办法模拟它,拿着录音让我一次又一次听,我却总是觉得不像。对航天员的最基本要求是严谨,不是当时的声音,我就不能签字,所以他们就让我反复听各种声音,断断续续听了一年多。但是直到现在,那个神秘的声音也没有在我耳边准确地再现过。
    
    在“神舟六号”和“神舟七号”飞行时,这个声音又出现了,但我告诉航天员:“出现这个声音别害怕,是正常现象。”
    
    归途如此惊心动魄
    
    5时35分,北京航天指挥中心向飞船发出“返回”指令。飞船开始在343公里高的轨道上制动,就像刹车一样。
    
    飞船先是在轨道上进行180度调姿——返回时要让推进舱在前,这就需要“掉头”。
    
    “制动发动机关机!”5时58分,飞船的速度减到一定数值,开始脱离原来的轨道,进入无动力飞行状态。
    
    6时4分,飞船下降至距地100公里,进入逐渐稠密的大气层。
    
    这时飞船的飞行速度仍然很快,遇到空气阻力后,它急剧减速,产生了近四倍重力加速度的过载\footnote{〔过载〕过大的加速度(比重力加速度更大的加速度)。}。我的前胸和后背都承受着很大的压力。我们平时已经训练过如何应对这种情况,因此身体应付自如,也没有紧张。
    
    让我紧张以至于惊慌的另有原因。
    
    飞船进入了“黑障”区\footnote{〔“黑障”〕航天飞行中出现的现象。在距离地面数十公里的高空高速飞行时,飞行器和大气摩擦产生的高温,使气体分子电离,并在飞行器表面形成离子层,阻碍电磁波通过。飞行器无法用电磁波与外界联系,因此称为“黑障”。},距地大约80公里到40公里。首先是快速行进的飞船与大气摩擦,产生的高温把舷窗外面烧得一片通红;接着在映红的舷窗外,有红的白的碎片不停划过。飞船的外表面有防烧蚀层,它是耐高温的,随着温度升高,开始剥落,并在剥落的过程中会带走一部分热量。我学习过这方面的知识,看到这种情形,知道是怎么回事。
    
    但随后发生的情况让我非常紧张——右边的舷窗开始出现裂纹。窗外烧得跟炼钢炉一样,而窗上出现裂纹。那纹路就跟强化玻璃被打碎之后的小碎纹一样。这种细细的碎纹,眼看着越来越多……说不恐惧那是假话。你想啊,窗外边可是有1600至1800摄氏度!
    
    我的汗水出来了。这时舱内的温度也在升高,但并没到让我瞬间出汗的程度,主要还是因为紧张。
    
    我现在还能清楚地记起当时的情形:飞船急速下降,跟空气摩擦产生激波\footnote{〔激波〕气流的速度超过了气体扰动传播的速度,使气流突然压缩变稠密,产生高温高热的现象。},不仅带来极高的温度,还伴随着尖利的呼啸声;飞船带着不小的过载,不停振动着,里面咯吱咯吱乱响。外面高温,不怕!有碎片划过,不怕!过载,也能承受!但是,看到舷窗玻璃开始出现裂缝,我紧张了,心想:完了,这个舷窗不行了。美国的“哥伦比亚号”航天飞机,不就是这样出事的吗?先是一块防热板出现裂缝,然后高热就使飞机解体了。这么大一个舷窗坏了,那还得了!
    
    右边的舷窗裂到一半的时候,左边的舷窗也开始出现裂纹。这反倒让我稍微放心了:哦——可能没什么问题!因为如果是故障,重复出现的概率并不高。
    
    回来之后,我才知道,飞船的舷窗外做了一层防烧涂层,是这个涂层烧裂了,而不是窗玻璃本身出现了问题。为什么两边没有同时出现裂纹呢?因为两边用了不同的材料。以前每次进行飞船发射与返回的实验,返回的飞船舱体经过高温烧灼,舷窗黑乎乎的,工作人员看不到这些裂纹。如果不是在飞船内亲眼所见,谁都不会想到有这种情况。
    
    距离地面还有40公里,飞船出了“黑障”区,速度已经降下来。一个关键的操作——抛伞,即将开始。这时舷窗已经烧得黑乎乎的,我抱着操作盒,屏息凝神,等待着配合程序:到哪里该做什么,该发什么指令,判断和操作都必须准确无误。
    
    6时14分,飞船距地10公里。飞船抛开降落伞盖,并迅速带出引导伞。
    
    这是一个剧烈的动作,能听到“砰”的一声,非常响。我在里边感觉被狠狠地一拽,瞬间过载很大,对身体的冲击也非常厉害。接下来是一连串的快速动作。引导伞出来后,紧跟着把减速伞也带出来,减速伞让飞船减速下落,16秒之后再把主伞带出来。
    
    其实最折磨人的就是这段过程了。随着一声巨响,你会感到突然一减速;引导伞一开,使劲一提,这个劲很大,会把人吓一跳;减速伞一开,又往那边一拽;主伞开时又把你拉到另一边。每次力量都相当大。飞船晃荡得很厉害,让人不知道是怎么回事。
    
    我们航天员是很重视这段过程的:伞开得好等于安全有保障,至少不会丢了性命。所以我被七七八八地拽了一通,平稳下来后心里却真是踏实——数据出来了,速度控制在规定范围内。我知道,这伞肯定是开好了!
    
    离地面5公里的时候,飞船抛掉防热大底,露出缓冲发动机。同时主伞也变成双点吊挂,让飞船摆正姿态,在风中晃悠着落向地面。
    
    飞船离地面1.2米时,缓冲发动机点火。接着,飞船“嗵”的一下落地了。
    
    我感觉落地很重,飞船弹了起来。在它第二次落地时,我迅速按下了切伞开关\footnote{〔切伞〕将飞船与降落伞分离。}。飞船停住了。此时是2003年10月16日6时23分。而这一时刻,正好是天安门当天升国旗的时刻,这是一个无法设计的巧合。
    
    那一刻四周寂静无声。舷窗黑乎乎的,看不到外面。
    
    过了几分钟,我隐约听到了叫喊声,手电的光从烧黑的舷窗上隐约照进来。他们找到飞船了!我听到外面插上钥匙的声音,舱门动弹了……
    
\end{large}



\chapter{从百草园到三味书屋}

\begin{large}
    
    我家的后面有一个很大的园,相传叫作百草园。现在是早已并屋子一起卖给朱文公的子孙了\footnote{〔朱文公〕南宋学者朱熹死后的谥号。这里指把屋子卖给一个姓朱的人。},连那最末次的相见也已经隔了七八年,其中似乎确凿只有一些野草;但那时却是我的乐园。
    
    不必说碧绿的菜畦,光滑的石井栏,高大的皂荚树\footnote{〔皂荚树〕一种乔木,果实像扁豆,长约20厘米,捣碎了泡水,可以洗衣服。},紫红的桑椹\footnote{〔桑椹〕桑树的果实,又叫桑葚。};也不必说鸣蝉在树叶里长吟,肥胖的黄蜂伏在菜花上,轻捷的叫天子\footnote{〔叫天子〕一种小鸟,又叫云雀。体长约20厘米,叫声响亮。}忽然从草间直窜向云霄里去了。单是周围的短短的泥墙根一带,就有无限趣味。油蛉\footnote{〔油蛉〕一种昆虫,俗名金钟儿,形似西瓜子,黑色,昼夜都鸣。}在这里低唱,蟋蟀们在这里弹琴。翻开断砖来,有时会遇见蜈蚣;还有斑蝥\footnote{〔斑蝥〕一种昆虫,能飞,翅上有黄黑色斑纹。这里是指类似斑蝥的“行夜虫”,俗称“放屁虫”。},倘若用手指按住它的脊梁,便会拍的一声,从后窍\footnote{〔后窍〕这里指昆虫的肛门。}喷出一阵烟雾。何首乌\footnote{〔何首乌〕一种多年生蔓草,根粗大,可入药。}藤和木莲\footnote{〔木莲〕一种蔓生的常绿灌木。}藤缠络着,木莲有莲房\footnote{〔莲房〕莲蓬。}一般的果实,何首乌有臃肿的根。有人说,何首乌根是有像人形的,吃了便可以成仙,我于是常常拔它起来,牵连不断地拔起来,也曾因此弄坏了泥墙,却从来没有见过有一块根像人样。如果不怕刺,还可以摘到覆盆子\footnote{〔覆盆子〕一种多年生草,茎长,有刺,夏天结果实。},象小珊瑚珠攒成的小球,又酸又甜,色味都比桑椹要好得远。
    
    长的草里是不去的,因为相传这园里有一条很大的赤练蛇\footnote{〔赤练蛇〕一种无毒蛇。体长一米左右,有黑红相间的斑纹。}。
    
    长妈妈\footnote{〔长妈妈〕鲁迅小时候家里的女工,下文的阿长也指她。}曾经讲给我一个故事听:先前,有一个读书人住在古庙里用功,晚间,在院子里纳凉的时候,突然听到有人在叫他。答应着,四面看时,却见一个美女的脸露在墙头上,向他一笑,隐去了。他很高兴;但竟给那走来夜谈的老和尚识破了机关\footnote{〔机关〕这里指周密而巧妙的计谋。}。说他脸上有些妖气,一定遇见“美女蛇”了;这是人首蛇身的怪物,能唤人名,倘一答应,夜间便要来吃这人的肉的。他自然吓得要死,而那老和尚却道无妨,给他一个小盒子,说只要放在枕边,便可高枕而卧。他虽然照样办,却总是睡不着,——当然睡不着的。到半夜,果然来了,沙沙沙!门外象是风雨声。他正抖作一团时,却听得豁的一声,一道金光从枕边飞出,外面便什么声音也没有了,那金光也就飞回来,敛在盒子里。后来呢?后来,老和尚说,这是飞蜈蚣,它能吸蛇的脑髓,美女蛇就被它治死了。
    
    结末的教训是:所以倘有陌生的声音叫你的名字,你万不可答应他。
    
    这故事很使我觉得做人之险,夏夜乘凉,往往有些担心,不敢去看墙上,而且极想得到一盒老和尚那样的飞蜈蚣。走到百草园的草丛旁边时,也常常这样想。但直到现在,总还没有得到,但也没有遇见过赤练蛇和美女蛇。叫我名字的陌生声音自然是常有的,然而都不是美女蛇。
    
    冬天的百草园比较的无味;雪一下,可就两样了。拍雪人和塑雪罗汉需要人们鉴赏,这是荒园,人迹罕至,所以不相宜,只好来捕鸟。薄薄的雪,是不行的;总须积雪盖了地面一两天,鸟雀们久已无处觅食的时候才好。扫开一块雪,露出地面,用一支短棒支起一面大的竹筛来,下面撒些秕谷,棒上系一条长绳,人远远地牵着,看鸟雀下来啄食,走到竹筛底下的时候,将绳子一拉,便罩住了。但所得的是麻雀居多,也有白颊的“张飞鸟”\footnote{〔张飞鸟〕鹡鸰。头部像戏台上张飞的脸谱,所以浙东也有叫张飞鸟。},性子很躁,养不过夜的。
    
    这是闰土的父亲\footnote{〔闰土〕作者在小说《故乡》中写到的儿时朋友。}所传授的方法,我却不大能用。明明见它们进去了,拉了绳,跑去一看,却什么都没有,费了半天力,捉住的不过三四只。闰土的父亲是小半天便能捕获几十只,装在叉袋\footnote{〔叉袋〕一种装粮食的布袋或者麻袋,袋口有叉角,可以打结。}里叫着撞着的。我曾经问他得失的缘由,他只静静地笑道:你太性急,来不及等它走到中间去。
    
    我不知道为什么家里的人要将我送进书塾里去了,而且还是全城中称为最严厉的书塾。也许是因为拔何首乌毁了泥墙罢,也许是因为将砖头抛到间壁的梁家去了罢,也许是因为站在石井栏上跳下来罢,……都无从知道。总而言之:我将不能常到百草园了。\texttt{Ade}\footnote{〔\texttt{Ade}〕德语,再见的意思。},我的蟋蟀们!\texttt{Ade},我的覆盆子们和木莲们!……
    
    出门向东,不上半里,走过一道石桥,便是我的先生的家了。从一扇黑油的竹门进去,第三间是书房。中间挂着一块匾额:三味书屋\footnote{〔三味书屋〕在绍兴城内作者故家附近。解放后辟为鲁迅纪念馆的一部分。};匾下面是一幅画,画着一只很肥大的梅花鹿伏在古树下。没有孔子牌位,我们便对着那匾和鹿行礼。第一次算是拜孔子,第二次算是拜先生。
    
    第二次行礼时,先生\footnote{〔先生〕作者的启蒙老师,姓寿,名怀鉴,字镜吾,是一个老秀才。}便和蔼地在一旁答礼。他是一个高而瘦的老人,须发都花白了,还戴着大眼镜。我对他很恭敬,因为我早听到,他是本城中极方正,质朴,博学的人。
    
    不知从哪里听来的,东方朔\footnote{〔东方朔〕西汉文学家,善辞赋,性格诙谐滑稽。关于他的民间传说很多。}也很渊博,他认识一种虫,名曰“怪哉”,冤气所化,用酒一浇,就消释了。我很想详细地知道这故事,但阿长是不知道的,因为她毕竟不渊博。现在得到机会了,可以问先生。
    
    “先生,‘怪哉”\footnote{〔怪哉〕据《太平广记》,汉武帝巡视时发现的人面怪虫。询问东方朔,东方朔回答说,过去秦朝拘押无辜的人太多,百姓纷纷感叹:“怪哉怪哉!”愤怒感动上天,产生这种虫。汉武帝问他有什么办法可以去掉,东方朔回答:喝酒可以消愁。用酒浇灌,可以让这虫子消失。}这虫,是怎么一回事?……”我上了生书\footnote{〔生书〕未读过的书,新课。},将要退下来的时候,赶忙问。
    
    “不知道!”他似乎很不高兴,脸上还有怒色了。
    
    我才知道做学生是不应该问这些事的,只要读书,因为他是渊博的宿儒\footnote{〔宿〕长久从事某种工作。儒:信奉孔孟之道的知识分子。},决不至于不知道,所谓不知道者,乃是不愿意说。年纪比我大的人,往往如此,我遇见过好几回了。
    
    我就只读书,正午习字,晚上对课\footnote{〔对课〕即对对联,旧时学习词句、准备作诗的一种练习。一般老师出上联,学生对下联。三言、五言,即三字一句、五字一句。字数越多越难。}。先生最初这几天对我很严厉,后来却好起来了,不过给我读的书渐渐加多,对课也渐渐地加上字去,从三言到五言,终于到七言。
    
    三味书屋后面也有一个园,虽然小,但在那里也可以爬上花坛去折腊梅花,在地上或桂花树上寻蝉蜕\footnote{〔蝉蜕〕蝉的幼虫变为蝉时脱去的外壳,可入药。}。最好的工作是捉了苍蝇喂蚂蚁,静悄悄地没有声音。然而同窗们到园里的太多,太久,可就不行了,先生在书房里便大叫起来:
    
    “人都到哪里去了?”
    
    人们便一个一个陆续走回去;一同回去,也不行的。他有一条戒尺,但是不常用,也有罚跪的规矩,但也不常用,普通总不过瞪几眼,大声道:
    
    “读书!”
    
    于是大家放开喉咙读一阵书,真是人声鼎沸。有念“仁远乎哉我欲仁斯仁至矣\footnote{见《论语·述而》,应读为“仁远乎哉?我欲仁,斯仁至矣!”}”的,有念“笑人齿缺曰狗窦大开”\footnote{见《幼学琼林·身体》,原句是:“笑人缺齿,狗窦胡为大开?”}的,有念“上九潜龙勿用”\footnote{见《周易》,原句是:“初九,潜龙勿用。”}的,有念“厥土下上上错厥贡苞茅橘柚”\footnote{见《尚书·禹贡》,原句是:“厥土惟涂泥。厥田惟下下,厥赋下上,上错。……厥包橘柚锡贡。”}的……先生自己也念书。后来,我们的声音便低下去,静下去了,只有他还大声朗读着:
    
    “铁如意,指挥倜傥,一座皆惊呢;金叵罗\footnote{〔叵罗〕古代饮酒用的一种敞口的浅杯。},颠倒淋漓噫,千杯未醉嗬……”
    
    我疑心这是极好的文章,因为读到这里,他总是微笑起来,而且将头仰起,摇着,向后面拗过去,拗过去。
    
    先生读书入神的时候,于我们是很相宜的。有几个便用纸糊的盔甲套在指甲上做戏。
    
    我是画画儿,用一种叫作“荆川纸”\footnote{〔荆川纸〕一种竹纸,薄而略透明。}的,蒙在小说的绣像\footnote{〔绣像〕明清以来,通俗小说前面往往附有书中人物的图像,称为绣像。}上一个个描下来,象习字时候的影写一样。读的书多起来,画的画也多起来;书没有读成,画的成绩却不少了,最成片断\footnote{〔片断〕片段。}的是《荡寇志》\footnote{〔《荡寇志》〕清朝俞万春著的一部反《水浒传》、诬蔑歪曲梁山起义的小说。}和《西游记》的绣像,都有一大本。后来,因为要钱用,卖给一个有钱的同窗了。他的父亲是开锡箔\footnote{〔锡箔〕这里指锡箔纸,附着一层薄锡的纸,旧时多用于祭奠死去的人。}店的;听说现在自己已经做了店主,而且快要升到绅士\footnote{〔绅士〕地方上有影响力、有威望的人,辅助官府统治、维持秩序。}的地位了。这东西早已没有了罢。
    
    \hfill 九月十八日
    
\end{large}


\newpage

\textbf{注释}:

\vspace{-1em}

\begin{itemize}
    \setlength\itemsep{-0.2em}
    \item 〔\xpinyin*{菜畦}〕种植蔬菜的一排排整齐的小块田地。四周围着土埂,便于管理和浇灌。
    \item 〔\xpinyin*{臃肿}〕胖大。
    \item 〔\xpinyin*{秕谷}〕不饱满的谷粒。
    \item 〔\xpinyin*{敛}〕收拢。
    \item 〔\xpinyin*{脑髓}〕脑浆。
    \item 〔\xpinyin*{无从}〕没有方法、门路(做某事)。
    \item 〔\xpinyin*{同窗}〕在同一窗下读书的人,指同学。
    \item 〔\xpinyin*{人迹罕至}〕少有人来。迹:足迹。罕:稀少。
    \item 〔\xpinyin*{戒尺}〕老师用来责罚学生的长条形木板。
    \item 〔\xpinyin*{书塾}〕旧时家庭、宗族或教师自己设立的教学场所。
    \item 〔\xpinyin*{倜傥}〕洒脱不受拘束的样子。
    \item 〔\xpinyin*{淋漓}〕濡湿流淌的样子。也形容酣畅、痛快。
    \item 〔\xpinyin*{拗}〕弯屈,弯转。
    \item 〔\xpinyin*{盔甲}〕古代战士的护身装备。头戴的称为“盔”,身穿的称为“甲”。
    \item 〔\xpinyin*{影写}〕把纸蒙在字帖上照着描。
    \item 〔\xpinyin*{人声鼎沸}〕许多人吵闹,声音像大锅里沸腾的水。
    \item 〔\xpinyin*{渊博}〕精深而广博。
\end{itemize}

\chapter{国王的新衣}

\begin{large}
    
    很久以前有一位国王,他非常喜欢穿新衣服。为了穿得漂亮,他把所有的钱都花到衣服上去了。他一点也不关心他的军队,不愿意坐马车出游,甚至不爱去戏院看戏,除非是为了炫耀一下新衣服。他要求每天每个钟点都准备好新衣服。人们一提到国王,总是说:“王上\footnote{〔王上〕古代对国王的尊称。}在更衣室里。”
    
    由于国王喜爱新衣服,很多裁缝、织工、鞋匠都到王宫来,希望为国王做衣服。一天,两个年轻人来到王宫,自称是来自外国的织工。他们说,他们能织出世上最美丽的布。这种布不仅美丽无比,还有一个奇妙的作用:凡是愚蠢的人,都看不见用这布做成的衣服。
    
    “这可不是最适合我的衣服嘛!”国王心想,“我穿了这样的衣服,就可以看出我的王国里哪些人不称职;我就可以辨别出哪些人是聪明人,哪些人是傻子。没错,我要叫他们马上织出这样的布来!”他付了一大笔钱给这两个人,赐给他们带庭院的住宅,叫他们马上开始工作。
    
    外国的织工摆出两架织机来,装做在工作的样子,可是他们的织机上什么东西也没有。他们接二连三地请求国王赏赐最好的蚕丝和金线给他们。他们把这些好东西都装进自己的腰包,却假装在那两架空空的织机上忙碌地工作,一直忙到深夜。
    
    国王等了好几天。他太想知道究竟织得怎样了。不过,他立刻就想起来,愚蠢的人是看不见这布的,这让他心里不太舒服。他相信他自己是用不着害怕的。虽然如此,他还是觉得,先派一个人去看看比较妥当。
    
    “先派老宫相\footnote{〔宫相〕欧洲古代掌管宫廷政务,辅佐国王的大臣。}去看看,”国王想,“他这个人很有头脑,而且不会说谎。”
    
    因此这位正直的老宫相就到那两个外国织工家去。两个骗子正在空空的织机上忙碌地工作着。
    
    “这是怎么一回事儿?”老宫相心想。他把眼睛睁得大大的。
    
    “我什么东西也没有看见!”但是他不敢把这句话说出来。
    
    两个骗子请他走近一点,指着那两架空空的织机问他,布的纹理是不是很精细,色彩是不是很漂亮。
    
    可怜的老宫相的眼睛越睁越大,可还是看不见什么东西,因为的确没有什么东西可看。
    
    “我的老天爷!”他想,“难道我是一个愚蠢的人吗?我从来没有怀疑过我自己。我决不能让人知道这件事——不成,我决不能让人知道我看不见布料。”
    
    “哎,您一点意见也没有吗?”一个正在织布的织工说。
    
    “啊,美极了!真是美妙极了!”老宫相说。他戴着眼镜仔细地看。“多么精细的纹理!多么美的色彩!是的,我会呈报王上,说我对于这布非常满意。”
    
    “听到您的话,我们就放心了。”两个织工一起说。他们把这些丰富的色彩和纹理仔细描述了一番。这位老大臣注意地听着,以便回到国王那里去时,可以照样背得出来。事实上他也这样办了。
    
    两个骗子又要了很多的钱,更多的蚕丝和金线,他们说这是为了织布的需要。他们把这些东西全装进腰包里,连一根线也没有放到织机上去。不过他们还是继续在空空的机架上工作。
    
    过了不久,国王又派了另一位正直可信的大臣,去看布是不是很快就可以织好。这位大臣也遇到了同样的事:他看了又看,但是那两架空空的织机上什么也没有,他什么东西也看不出来。
    
    “您看这段布美不美?”两个骗子问。他们指出各种漂亮的花纹,仔细解释。事实上并没有什么花纹。
    
    “我并不愚蠢!”这位大臣想,“难道我是个蠢货吗?这也真够滑稽,但是我决不能让人看出来!”因此他就把他完全没有看见的布称赞了一番,同时对他们说,他非常喜欢这些美丽的色彩和精致的花纹。“是的,那真是太美了。”他回去对国王说。
    
    城里所有的人都在谈论这神奇的布料。
    
    国王很想亲自去看一次。他特别指定了一批随员\footnote{〔随员〕随同出行的人。}——包括已经去看过的那两位正直的大臣。两个狡猾的骗子正卖力地织布,但是一根线的影子也看不见。“各位,这不漂亮吗?”那两位正直的大臣说,“陛下\footnote{〔陛下〕古代对国王的尊称。}请看,多么精致的花纹!多么美丽的色彩!”他们指着空空的织机,因为他们以为别人一定看得见。
    
    “这是怎么一回事儿呢?”国王心想,“我什么也没有看见!这真是荒唐!难道我是一个愚蠢的人吗?难道我不配做国王吗?这太可怕了。我从没有遇过这样的事情。”
    
    “啊,它真是美极了!”国王说,“我十分满意!”
    
    于是他点头表示满意。他装作很仔细地看着织机的样子,因为他不愿意说出他什么也没有看见。全体随员也仔细地看了又看,可是他们也没有看出任何东西。不过,他们也照着国王的话说:“啊,真是美极了!”他们提出,这种新奇的布料,正适合即将举行的游行庆典\footnote{〔游行庆典〕在大街上行进的庆祝活动。}。国王穿着用它做的新衣服游行,再好不过了。
    
    “真美丽!真精致!真是好极了!”每人都随声附和着,每个人都显得开心极了。国王赐给骗子每人一个爵士的头衔,一枚可以挂在扣眼\footnote{〔扣眼〕上衣胸前用来别纽扣的孔,也可以用来佩挂饰物。}上的勋章。
    
    第二天早上就是游行庆典了。这两个骗子整夜不睡,点起十六支蜡烛。你可以看到他们在连夜赶工,要完成国王的新衣。他们装做把布料从织机上取下来。他们用两把大剪刀在空中裁了一阵子,同时又用没有穿线的针缝了一通。最后,他们齐声说:“请看!新衣服做好了!”
    
    国王带着一群贵族,亲自到来了。两个骗子每人举起一只手,好像他们拿着什么东西。“请看吧,这是裤子,这是上衣!这是披风!”他们指出每一件衣服的名称。“这衣服轻柔得像蜘蛛网一样:穿着它的人会觉得身上什么也没有似的——这也正是这衣服的妙处。”
    
    “一点也不错。”所有的贵族都赞同。可是他们什么也没有看见,因为实际上什么东西也没有。
    
    “现在请陛下脱下衣服,”两个骗子说,“我们要在这个大镜子前为陛下换上新衣。”
    
    国王把身上的衣服统统脱了。这两个骗子装作把刚才缝好的新衣服一件一件给他穿上。他们在他的腰上弄了一阵子,仿佛在系上什么东西:这就是后裙摆\footnote{〔后裙摆〕欧洲古代贵族的一种装束,是拖在身后的很长的一块布。}。国王在镜子面前转了转身子,扭了扭腰肢。
    
    “天啊,这衣服多么合身啊!这剪裁、这式样,多么好看啊!”大家都说,“多么精致的花纹!多么美妙的色彩!这真是一套前所未见、令人惊叹的衣服!”
    
    “华盖\footnote{〔华盖〕国王出行时遮阳遮雨的大伞。}已经准备好了,只等陛下穿好新衣服,就可以开始游行了!”典礼官说。
    
    “对,我已经穿好了。”国王说,“这衣服合我的身么?”于是他又在镜子面前扭动身子,因为他要叫大家看出,他在认真地欣赏他美丽的新衣服。侍从们都把手在地上东摸西摸,仿佛真的在拾起裙摆。他们抬起手来,手中托着空气——他们不敢让人瞧出,他们其实什么也没有看见。
    
    这么着,国王就在华盖下游行起来了。街上看见的人、街旁窗户里望见的人都说:“王上的新衣服真是漂亮!长长的后裙摆多么美丽!衣服多么合身!”谁也不愿意让人知道自己看不见,因为这样就会暴露自己是个愚蠢的家伙。国王所有的衣服从来没有得到过这样一致的称赞。
    
    “可是他什么衣服也没有穿呀!”突然,一个小孩子叫出声来。
    
    “天呐,你听这个天真的声音!”爸爸说。于是大家把这孩子讲的话低声地传开来。
    
    “他并没有穿什么衣服!有一个小孩子说他并没有穿什么衣服呀!”
    
    “他实在是没有穿什么衣服呀!”最终,所有的老百姓都这么说了。这话终于传到了国王耳中。国王有点儿发抖,因为他似乎觉得老百姓讲的是对的。“但是,我决不能让人看出来,否则这庆典就无法收场了。”因此他摆出一副更高傲的神气。他的侍从们跟在他后面,手中托着并不存在的裙摆。
    
\end{large}



\chapter{海燕}

\begin{large}
    
    在苍茫的大海上,狂风卷集着乌云。在乌云和大海之间,海燕像黑色的闪电,在高傲地飞翔。
    
    一会儿翅膀碰着波浪,一会儿箭一般地直冲向乌云,它叫喊着,──就在这鸟儿勇敢的叫喊声里,乌云听出了欢乐。
    
    在这叫喊声里──充满着对暴风雨的渴望!在这叫喊声里,乌云听出了愤怒的力量、热情的火焰和胜利的信心。
    
    海鸥在暴风雨来临之前呻吟着,──呻吟着,它们在大海上飞窜,想把自己对暴风雨的恐惧,掩藏到大海深处。
    
    海鸭也在呻吟着,──它们这些海鸭啊,享受不了生活的战斗的欢乐:轰隆隆的雷声就把它们吓坏了。
    
    蠢笨的企鹅,胆怯地把肥胖的身体躲藏到悬崖底下……只有那高傲的海燕,勇敢地,自由自在地,在泛起白沫的大海上飞翔!
    
    乌云越来越暗,越来越低,向海面直压下来,而波浪一边歌唱,一边冲向高空,去迎接那雷声。
    
    雷声轰响。波浪在愤怒的飞沫中呼叫,跟狂风争鸣。看吧,狂风紧紧抱起一层层巨浪,恶狠狠地把它们甩到悬崖上,把这些大块的翡翠摔成尘雾和碎末。
    
    海燕叫喊着,飞翔着,像黑色的闪电,箭一般地穿过乌云,翅膀掠起波浪的飞沫。
    
    看吧,它飞舞着,像个精灵,──高傲的、黑色的暴风雨的精灵,——它在大笑,它又在号叫……它笑那些乌云,它因为欢乐而号叫!
    
    这个敏感的精灵,——它从雷声的震怒里,早就听出了困乏,它深信,乌云遮不住太阳,──是的,遮不住的!
    
    狂风吼叫……雷声轰响……
    
    一堆堆乌云,像青色的火焰,在无底的大海上燃烧。大海抓住闪电的箭光,把它们熄灭在自己的深渊里。这些闪电的影子,活像一条条火蛇,在大海里蜿蜒游动,一晃就消失了。
    
    ——暴风雨!暴风雨就要来啦!
    
    这是勇敢的海燕,在怒吼的大海上,在闪电中间,高傲地飞翔;这是胜利的预言家在叫喊:
    
    ——让暴风雨来得更猛烈些吧!
    
\end{large}


\newpage

\textbf{注释}:

\vspace{-1em}

\begin{itemize}
    \setlength\itemsep{-0.2em}
    \item 〔\xpinyin*{深渊}〕非常深的水。
    \item 〔\xpinyin*{呻吟}〕因痛苦而发出声音。
    \item 〔\xpinyin*{胆怯}〕胆小害怕。
    \item 〔\xpinyin*{蜿蜒}〕蛇屈折爬行的样子。
\end{itemize}

\chapter{春}

\begin{large}
    
    盼望着,盼望着,东风来了,春天的脚步近了。
    
    一切都像刚睡醒的样子,欣欣然张开了眼。山朗润起来了,水涨起来了,太阳的脸红起来了。
    
    小草偷偷地从土里钻出来,嫩嫩的,绿绿的。园子里,田野里,瞧去,一大片一大片满是的。坐着,躺着,打两个滚,踢几脚球,赛几趟跑,捉几回迷藏。风轻悄悄的,草绵软软的。
    
    桃树、杏树、梨树,你不让我,我不让你,都开满了花赶趟儿。红的像火,粉的像霞,白的像雪。花里带着甜味儿;闭了眼,树上仿佛已经满是桃儿、杏儿、梨儿。花下成千成百的蜜蜂嗡嗡地闹着,大小的蝴蝶飞来飞去。野花遍地是:杂样儿,有名字的,没名字的,散在草丛里,像眼睛,像星星,还眨呀眨的。
    
    “吹面不寒杨柳风”,不错的,像母亲的手抚摸着你。风里带来些新翻的泥土的气息,混着青草味儿,还有各种花的香,都在微微润湿的空气里酝酿。鸟儿将窠巢安在繁花嫩叶当中,高兴起来了,呼朋引伴地卖弄清脆的喉咙,唱出宛转的曲子,与轻风流水应和着。牛背上牧童的短笛,这时候也成天在嘹亮地响。
    
    雨是最寻常的,一下就是三两天。可别恼。看,像牛毛,像花针,像细丝,密密地斜织着,人家屋顶上全笼着一层薄烟。树叶子却绿得发亮,小草也青得逼你的眼。傍晚时候,上灯了,一点点黄晕的光,烘托出一片安静而和平的夜。乡下去,小路上,石桥边,有撑起伞慢慢走着的人;还有地里工作的农夫,披着蓑,戴着笠的。他们的草屋,稀稀疏疏的,在雨里静默着。
    
    天上风筝渐渐多了,地上孩子也多了。城里乡下,家家户户,老老小小,他们也赶趟儿似的,一个个都出来了。舒活舒活筋骨,抖擞抖擞精神,各做各的一份事去。“一年之计在于春”,刚起头儿,有的是工夫,有的是希望。
    
    春天像刚落地\footnote{〔落地〕这里指婴儿出生。}的娃娃,从头到脚都是新的,他生长着。
    
    春天像小姑娘,花枝招展的,笑着,走着。
    
    春天像健壮的青年,有铁一般的胳膊和腰脚,他领着我们上前去。
    
\end{large}


\newpage

\textbf{注释}:

\vspace{-1em}

\begin{itemize}
    \setlength\itemsep{-0.2em}
    \item 〔\xpinyin*{朗润}〕明亮滋润。朗:明亮。润:滋润、润泽。
    \item 〔\xpinyin*{赶趟儿}〕时间赶得上。这里指众多果树争先恐后地开花。
    \item 〔\xpinyin*{酝酿}〕造酒的发酵过程。这里指各种气息在空气里,像发酵似的,越来越浓。
    \item 〔\xpinyin*{窠巢}〕鸟兽昆虫的窝。
    \item 〔\xpinyin*{宛转}〕形容声音抑扬动听。现在多写作“婉转”。
    \item 〔\xpinyin*{花针}〕绣花用的细针。
    \item 〔\xpinyin*{黄晕}〕昏黄,不明亮。
    \item 〔\xpinyin*{笠}〕用竹篾或棕皮编制的遮阳挡雨的帽子。
    \item 〔\xpinyin*{花枝招展}〕形容女子打扮得十分艳丽。这里比喻姿态优美。
    \item 〔\xpinyin*{抖擞}〕振作(精神)。
\end{itemize}

\chapter{纪念白求恩}

\begin{large}
    
    白求恩同志是加拿大共产党员,五十多岁了,为了帮助中国的抗日战争,受加拿大共产党和美国共产党的派遣,不远万里,来到中国。去年春上到延安,后来到五台山工作,不幸以身殉职。一个外国人,毫无利己的动机,把中国人民的解放事业当作他自己的事业,这是什么精神?这是国际主义的精神,这是共产主义的精神,每一个中国共产党员都要学习这种精神。列宁主义认为:资本主义国家的无产阶级要拥护殖民地半殖民地人民的解放斗争,殖民地半殖民地的无产阶级要拥护资本主义国家的无产阶级的解放斗争,世界革命才能胜利。白求恩同志是实践了这一条列宁主义路线的。我们中国共产党员也要实践这一条路线。我们要和一切资本主义国家的无产阶级联合起来,要和日本的、英国的、美国的、德国的、意大利的以及一切资本主义国家的无产阶级联合起来,才能打倒帝国主义,解放我们的民族和人民,解放世界的民族和人民。这就是我们的国际主义,这就是我们用以反对狭隘民族主义和狭隘爱国主义的国际主义。
    
    白求恩同志毫不利己专门利人的精神,表现在他对工作的极端的负责任,对同志对人民的极端的热忱。每个共产党员都要学习他。不少的人对工作不负责任,拈轻怕重,把重担子推给人家,自己挑轻的。一事当前,先替自己打算,然后再替别人打算。出了一点力就觉得了不起,喜欢自吹,生怕人家不知道。对同志对人民不是满腔热忱,而是冷冷清清,漠不关心,麻木不仁。这种人其实不是共产党员,至少不能算一个纯粹的共产党员。从前线回来的人说到白求恩,没有一个不佩服,没有一个不为他的精神所感动。晋察冀边区的军民,凡亲身受过白求恩医生的治疗和亲眼看过白求恩医生的工作的,无不为之感动。每一个共产党员,一定要学习白求恩同志的这种真正共产主义者的精神。
    
    白求恩同志是个医生,他以医疗为职业,对技术精益求精;在整个八路军医务系统中,他的医术是很高明的。这对于一班见异思迁的人,对于一班鄙薄技术工作以为不足道、以为无出路的人,也是一个极好的教训。
    
    我和白求恩同志只见过一面。后来他给我来过许多信。可是因为忙,仅回过他一封信,还不知他收到没有。对于他的死,我是很悲痛的。现在大家纪念他,可见他的精神感人之深。我们大家要学习他毫无自私自利之心的精神。从这点出发,就可以变为大有利于人民的人。一个人能力有大小,但只要有这点精神,就是一个高尚的人,一个纯粹的人,一个有道德的人,一个脱离了低级趣味的人,一个有益于人民的人。
    
\end{large}


\newpage

\textbf{注释}:

\vspace{-1em}

\begin{itemize}
    \setlength\itemsep{-0.2em}
    \item 〔\xpinyin*{殉职}〕因本职工作死亡。
    \item 〔\xpinyin*{实践}〕实际去做。践:踩,踏。
    \item 〔\xpinyin*{狭隘}〕狭小。狭义。气量小。
    \item 〔\xpinyin*{派遣}〕正式命令、委任下级去干某事,通常到别的地方。
    \item 〔\xpinyin*{精益求精}〕已经很好了,还要求更好。
    \item 〔\xpinyin*{见异思迁}〕看到别的地方就想搬迁过去。看到别的目标就改变主意。
    \item 〔\xpinyin*{鄙薄}〕看不起,认为没有价值。
\end{itemize}

\chapter{猫}

\begin{large}
    
    我家养了好几次猫,结局总是失踪或死亡。三妹是最喜欢猫的,她常在课后回家时,逗着猫玩。有一次,从隔壁要了一只新生的猫来。花白的毛,很活泼,如带着泥土的白雪球似的,常在廊前太阳光里滚来滚去。三妹常常取了一条红带,或一根绳子,在它面前来回的拖摇着,它便扑过来抢,又扑过去抢。我坐在藤椅上看着他们,可以微笑着消耗过一二小时的光阴,那时太阳光暖暖的照着,心上感着生命的新鲜与快乐。后来这只猫不知怎地忽然消瘦了,也不肯吃东西,光泽的毛也污涩了,终日躺在厅上的椅下,不肯出来。三妹想着种种方法逗它,它都不理会。我们都很替它忧郁。三妹特地买了一个很小很小的铜铃,用红绫带穿了,挂在它颈下,但只显得不相称,它只是毫无生意的,懒惰的,郁闷的躺着。有一天中午,我从编译所\footnote{〔编译所〕1921年4月,在茅盾介绍下,郑振铎进入商务印书馆编译所工作。}回来,三妹很难过的说道:“哥哥,小猫死了!”
    
    我心里也感着一缕的酸辛,可怜这两月来相伴的小侣!当时只得安慰着三妹道:“不要紧,我再向别处要一只来给你。”
    
    隔了几天,二妹从虹口\footnote{〔虹口〕上海市辖区。在黄浦江西北岸。}舅舅家里回来,她道,舅舅那里有三四只小猫,很有趣,正要送给人家。三妹便怂恿着她去拿一只来。礼拜天,母亲回来了,却带了一只浑身黄色的小猫同来。立刻三妹一部分的注意,又被这只黄色小猫吸引去了。这只小猫较第一只更有趣、更活泼。它在园中乱跑,又会爬树,有时蝴蝶安详地飞过时,它也会扑过去捉。它似乎太活泼了,一点也不怕生人,有时由树上跃到墙上,又跑到街上,在那里晒太阳。我们都很为它提心吊胆,一天都要“小猫呢?小猫呢?”查问得好几次。每次总要寻找了一回,方才寻到。三妹常指它笑着骂道:“你这小猫呀,要被乞丐捉去后才不会乱跑呢!”我回家吃中饭,总看见它坐在铁门外边,一见我进门,便飞也似地跑进去了。饭后的娱乐,是看它在爬树。隐身在阳光隐约里的绿叶中,好像在等待着要捉捕什么似的。把它抱了下来。一放手,又极快地爬上去了。过了二三个月,它会捉鼠了。有一次,居然捉到一只很肥大的鼠,自此,夜间便不再听见讨厌的吱吱的声了。
    
    某一日清晨,我起床来,披了衣下楼,没有看见小猫,在小园里找了一遍,也不见。心里便有些亡失的预警。
    
    “三妹,小猫呢?”
    
    她慌忙地跑下楼来,答道:“我刚才也寻了一遍,没有看见。”
    
    家里的人都忙乱的在寻找,但终于不见。
    
    李嫂道;“我一早起来开门,还见它在厅上。烧饭时,才不见了它。”
    
    大家都不高兴,好像亡失了一个亲爱的同伴,连向来不大喜欢它的张婶也说;“可惜,可惜,这样好的一只小猫。”
    
    我心里还有一线希望,以为它偶然跑到远处去,也许会认得归途的。
    
    午饭时,张婶诉说道:“刚才遇到隔壁周家的丫头,她说,早上看见我家的小猫在门外,被一个过路的人捉去了。”
    
    于是这个亡失证实了。三妹很不高兴的咕噜着道:“他们看见了,为什么不出来阻止?他们明晓得它是我家的!”
    
    我也怅然的,愤恨的,在诅骂着那个不知名的夺去我们所爱的东西的人。
    
    自此,我家好久不养猫。
    
    冬天的早晨,门口蜷伏着一只很可怜的小猫。毛色是花白,但并不好看,又很瘦。它伏着不去。我们如不取来留养,至少也要为冬寒与饥饿所杀。张婶把它拾了进来,每天给它饭吃。但大家都不大喜欢它,它不活泼,也不像别的小猫之喜欢顽游,好像是具着天生的忧郁性似的,连三妹那样爱猫的,对于它也不加注意。如此的,过了几个月,它在我家仍是一只若有若无的动物。它渐渐的肥胖了,但仍不活泼。大家在廊前晒太阳闲谈着时,它也常来蜷伏在母亲或三妹的足下。三妹有时也逗着它玩,但没有对于前几只小猫那样感兴趣。有一天,它因夜里冷,钻到火炉底下去,毛被烧脱好几块,更觉得难看了。
    
    春天来了,它成了一只壮猫了,却仍不改它的忧郁性,也不去捉鼠,终日懒惰的伏着,吃得胖胖的。
    
    这时,妻买了一对黄色的芙蓉鸟来,挂在廊前,叫得很好听。妻常常叮嘱着张婶换水,加鸟粮,洗刷笼子。那只花白猫对于这一对黄鸟,似乎也特别注意,常常跳在桌上,对鸟笼凝望着。
    
    妻道:“张婶,留心猫,它会吃鸟呢。”
    
    张婶便跑来把猫捉了去。隔一会,它又跳上桌子对鸟笼凝望着了。
    
    一天,我下楼时,听见张婶在叫道:“鸟死了一只,一条腿被咬去了,笼扳上都是血。是什么东西把它咬死的?”
    
    我匆匆跑下去看,果然一只鸟是死了,羽毛松散着,好像它曾与它的敌人挣扎了许久。
    
    我很愤怒,叫道:“一定是猫,一定是猫!”于是立刻便去找它。
    
    妻听见了,也匆匆地跑下来,看了死鸟,很难过,便道:“不是这猫咬死的还有谁?它常常对鸟笼望着,我早就叫张婶要小心了。张婶!你为什么不小心?”
    
    张婶默默无言,不能有什么话来辩护。
    
    于是猫的罪状证实了。大家都去找这可厌的猫,想给它以一顿惩戒。找了半天,却没找到。我以为它真是“畏罪潜逃”了。
    
    三妹在楼上叫道:“猫在这里了。”
    
    它躺在露台板上晒太阳,态度很安详,嘴里好像还在吃着什么。我想,它一定是在吃着这可怜的鸟的腿了,一时怒气冲天,拿起楼门旁倚着的一根木棒,追过去打了一下。它很悲楚地叫了一声“咪呜!”便逃到屋瓦上了。
    
    我心里还愤愤的,以为惩戒得还没有快意。
    
    隔了几天,李嫂在楼下叫道:“猫,猫!又来吃鸟了。”同时我看见一只黑猫飞快的逃过露台,嘴里衔着一只黄鸟。我开始觉得我是错了!
    
    我心里十分的难过,真的,我的良心受伤了,我没有判断明白,便妄下断语,冤苦了一只不能说话辩诉的动物。想到它的无抵抗的逃避,益使我感到我的暴怒,我的虐待,都是针,刺我的良心的针!
    
    我很想补救我的过失,但它是不能说话的,我将怎样的对它表白我的误解呢?
    
    两个月后,我们的猫忽然死在邻家的屋脊上。我对于它的亡失,比以前的两只猫的亡失,更难过得多。
    
    我永无改正我的过失的机会了!
    
    自此,我家永不养猫。
    
\end{large}


\newpage

\textbf{注释}:

\vspace{-1em}

\begin{itemize}
    \setlength\itemsep{-0.2em}
    \item 〔\xpinyin*{涩}〕不顺滑。
    \item 〔\xpinyin*{怂恿}〕从旁劝说,使想做。
    \item 〔\xpinyin*{提心吊胆}〕不放心,心里不安。
    \item 〔\xpinyin*{乞丐}〕靠要饭要钱过活的人。
    \item 〔\xpinyin*{倚}〕斜靠。
    \item 〔\xpinyin*{虐待}〕用狠毒残忍的手段对待人。
\end{itemize}

\chapter{阿长与山海经}

\begin{large}
    
    长妈妈,已经说过,是一个一向带领着我的女工,说得阔气一点,就是我的保姆。我的母亲和许多别的人都这样称呼她,似乎略带些客气的意思。只有祖母叫她阿长。我平时叫她“阿妈”,连“长”字也不带;但到憎恶她的时候,——例如知道了谋死我那隐鼠\footnote{〔隐鼠〕鼹鼠的别称。}的却是她的时候,就叫她阿长。
    
    我们那里没有姓长的;她生得黄胖而矮,“长”也不是形容词。又不是她的名字,记得她自己说过,她的名字是叫作什么姑娘的。什么姑娘,我现在已经忘却了,总之不是长姑娘;也终于不知道她姓什么。记得她也曾告诉过我这个名称的来历:先前的先前,我家有一个女工,身材生得很高大,这就是真阿长。后来她回去了,我那什么姑娘才来补她的缺,然而大家因为叫惯了,没有再改口,于是她从此也就成为长妈妈了。
    
    虽然背地里说人长短不是好事情,但倘使要我说句真心话,我可只得说:我实在不大佩服她。最讨厌的是常喜欢切切察察,向人们低声絮说些什么事。还竖起第二个手指,在空中上下摇动,或者点着对手或自己的鼻尖。我的家里一有些小风波,不知怎的我总疑心和这“切切察察”有些关系。又不许我走动,拔一株草,翻一块石头,就说我顽皮,要告诉我的母亲去了。一到夏天,睡觉时她又伸开两脚两手,在床中间摆成一个“大”字,挤得我没有余地翻身,久睡在一角的席子上,又已经烤得那么热。推她呢,不动;叫她呢,也不闻。
    
    “长妈妈生得那么胖,一定很怕热罢?晚上的睡相,怕不见得很好罢?……”
    
    母亲听到我多回诉苦之后,曾经这样地问过她。我也知道这意思是要她多给我一些空席。她不开口。但到夜里,我热得醒来的时候,却仍然看见满床摆着一个“大”字,一条臂膊还搁在我的颈子上。我想,这实在是无法可想了。
    
    但是她懂得许多规矩;这些规矩,也大概是我所不耐烦的。一年中最高兴的时节,自然要数除夕了。辞岁\footnote{〔辞岁〕新年开始。旧年最后一夜叫做“除夕”,度过后迎来新年,称为“辞旧岁,迎新春”。}之后,从长辈得到压岁钱\footnote{〔压岁钱〕过年的习俗。长辈要给小辈压岁钱,保佑平安过年。},红纸包着,放在枕边,只要过一宵,便可以随意使用。睡在枕上,看着红包,想到明天买来的小鼓、刀枪、泥人、糖菩萨\footnote{〔糖菩萨〕一种小吃。把糖用模具做成菩萨样子。}……。然而她进来,又将一个福橘\footnote{〔福橘〕过年的习俗。橘音近“吉”,因此过年吃橘子,称为“福橘”。}放在床头了。
    
    “哥儿,你牢牢记住!”她极其郑重地说。“明天是正月初一,清早一睁开眼睛,第一句话就得对我说:‘阿妈,恭喜恭喜!’记得么?你要记着,这是一年的运气的事情。不许说别的话!说过之后,还得吃一点福橘。”她又拿起那橘子来在我的眼前摇了两摇,“那么,一年到头,顺顺流流……。”
    
    梦里也记得元旦的,第二天醒得特别早,一醒,就要坐起来。她却立刻伸出臂膊,一把将我按住。我惊异地看她时,只见她惶急地看着我。
    
    她又有所要求似的,摇着我的肩。我忽而记得了——
    
    “阿妈,恭喜……。”
    
    恭喜恭喜!大家恭喜!真聪明!恭喜恭喜!”她于是十分欢喜似的,笑将起来,同时将一点冰冷的东西,塞在我的嘴里。我大吃一惊之后,也就忽而记得,这就是所谓福橘,元旦辟头\footnote{〔辟头〕开头。}的磨难,总算已经受完,可以下床玩耍去了。
    
    她教给我的道理还很多,例如说人死了,不该说死掉,必须说“老掉了”;死了人,生了孩子的屋子里,不应该走进去;饭粒落在地上,必须拣起来,最好是吃下去;晒裤子用的竹竿底下,是万不可钻过去的……。此外,现在大抵忘却了,只有元旦的古怪仪式记得最清楚。总之:都是些烦琐\footnote{〔烦琐〕繁琐。}之至,至今想起来还觉得非常麻烦的事情。
    
    然而我有一时也对她发生过空前的敬意。她常常对我讲“长毛”。她之所谓“长毛”者,不但洪秀全\footnote{〔洪秀全〕清晚期太平天国运动的发起者和领袖。}军,似乎连后来一切土匪强盗都在内,但除却革命党\footnote{〔革命党〕清末以兴中会为首、意图推翻帝制的革命团体。},因为那时还没有。她说得长毛非常可怕,他们的话就听不懂。她说先前长毛进城的时候,我家全都逃到海边去了,只留一个门房\footnote{〔门房〕大门口内侧的小房,有专人看守。也指看守门房的人。}和年老的煮饭老妈子看家。后来长毛果然进门来了,那老妈子便叫他们“大王”,——据说对长毛就应该这样叫,——诉说自己的饥饿。长毛笑道:“那么,这东西就给你吃了罢!”将一个圆圆的东西掷了过来,还带着一条小辫子,正是那门房的头。煮饭老妈子从此就骇破了胆,后来一提起,还是立刻面如土色,自己轻轻地拍着胸埔道:“阿呀,骇死我了,骇死我了……。”
    
    我那时似乎倒并不怕,因为我觉得这些事和我毫不相干的,我不是一个门房。但她大概也即觉到了,说道:“象你似的小孩子,长毛也要掳的,掳去做小长毛。还有好看的姑娘,也要掳。”
    
    “那么,你是不要紧的。”我以为她一定最安全了,既不做门房,又不是小孩子,也生得不好看,况且颈子上还有许多炙疮疤。
    
    “那里的话?!”她严肃地说。“我们就没有用处?我们也要被掳去。城外有兵来攻的时候,长毛就叫我们脱下裤子,一排一排地站在城墙上,外面的大炮就放不出来;再要放,就炸了!”
    
    这实在是出于我意想之外的,不能不惊异。我一向只以为她满肚子是麻烦的礼节罢了,却不料她还有这样伟大的神力。从此对于她就有了特别的敬意,似乎实在深不可测;夜间的伸开手脚,占领全床,那当然是情有可原的了,倒应该我退让。
    
    这种敬意,虽然也逐渐淡薄起来,但完全消失,大概是在知道她谋害了我的隐鼠之后。那时就极严重地诘问,而且当面叫她阿长。我想我又不真做小长毛,不去攻城,也不放炮,更不怕炮炸,我惧惮她什么呢!
    
    但当我哀悼隐鼠,给它复仇的时候,一面又在渴慕着绘图的《山海经》\footnote{〔《山海经》〕古代流传的地理志,传说是大禹所作。有各种神怪奇物图解。}了。这渴慕是从一个远房的叔祖\footnote{〔叔祖〕祖父的弟弟。}惹起来的。他是一个胖胖的,和蔼的老人,爱种一点花木,如珠兰、茉莉之类,还有极其少见的,据说从北边带回去的马缨花\footnote{〔马缨花〕杜鹃花品种,因花朵状似马缨而得名。产于云南、贵州、广西等地的山区。}。他的太太却正相反,什么也莫名其妙,曾将晒衣服的竹竿搁在珠兰的枝条上,枝折了,还要愤愤地咒骂道:“死尸!”这老人是个寂寞者,因为无人可谈,就很爱和孩子们往来,有时简直称我们为“小友”。在我们聚族而居的宅子里,只有他书多,而且特别。制艺和试帖诗,自然也是有的;但我却只在他的书斋里,看见过陆玑的《毛诗草木鸟兽虫鱼疏》\footnote{〔《毛诗草木鸟兽虫鱼疏》〕三国时期吴国的陆玑编著,对《毛诗》中的动植物进行考注说明。《毛诗》指战国末年毛亨、毛苌辑注的《诗经》。},还有许多名目很生的书籍。我那时最爱看的是《花镜》\footnote{〔《花镜》〕即《秘传花镜》,清代陈淏子所著。主要讲藤木花草的分类,以及栽培花卉、饲养禽鸟兽畜昆虫的方法。},上面有许多图。他说给我听,曾经有过一部绘图的《山海经》,画着人面的兽,九头的蛇,三脚的鸟,生着翅膀的人,没有头而以两乳当作眼睛的怪物,……可惜现在不知道放在那里了。
    
    很愿意看看这样的图画,但不好意思力逼他去寻找,他是很疏懒的。问别人呢,谁也不肯真实地回答我。压岁钱还有几百文,买罢,又没有好机会。有书买的大街离我家远得很,我一年中只能在正月间去玩一趟,那时候,两家书店都紧紧地关着门。
    
    玩的时候倒是没有什么的,但一坐下,我就记得绘图的《山海经》。
    
    大概是太过于念念不忘了,连阿长也来问《山海经》是怎么一回事。这是我向来没有和她说过的,我知道她并非学者,说了也无益;但既然来问,也就都对她说了。
    
    过了十多天,或者一个月罢,我还记得,是她告假回家以后的四五天,她穿着新的蓝布衫回来了,一见面,就将一包书递给我,高兴地说道:——“哥儿,有画儿的‘三哼经’,我给你买来了!”
    
    我似乎遇着了一个霹雳,全体\footnote{〔全体〕全身。}都震悚起来;赶紧去接过来,打开纸包,是四本小小的书,略略一翻,人面的兽,九头的蛇,……果然都在内。
    
    又使我发生新的敬意了,别人不肯做,或不能做的事,她却能够做成功。她确有伟大的神力。谋害隐鼠的怨恨,从此完全消灭了。
    
    这四本书,乃是我最初得到,最为心爱的宝书。
    
    书的模样,到现在还在眼前。可是从还在眼前的模样来说,却是一部刻印都十分粗拙的本子。纸张很黄;图象也很坏,甚至于几乎全用直线凑合,连动物的眼睛也都是长方形的。但那是我最为心爱的宝书,看起来,确是人面的兽;九头的蛇;一脚的牛;袋子似的帝江\footnote{〔帝江〕《山海经》记载的天山的山神。赤红如火,六足四翼,没有面目,能歌舞。};没有头而“以乳为目,以脐为口”,还要“执干戚而舞”的刑天\footnote{〔刑天〕《山海经》记载的巨人,与帝争神,被断头,葬于常羊之山。}。
    
    此后我就更其搜集绘图的书,于是有了石印\footnote{〔石印〕石版油墨印刷技术。18世纪末出现,19世纪传入中国,比称为“木刻”的雕版印刷更方便更好,广受欢迎。}的《尔雅音图》\footnote{〔《尔雅音图》〕《尔雅》是我国最早的辞典。西晋郭璞为《尔雅》注音、作图,内有大量插图。近代出版的画谱由清代姚之麟绘画。}和《毛诗品物图考》\footnote{〔《毛诗品物图考》〕18世纪日本汉学家对《毛诗》中动植物的图释著作。由冈元凤纂辑,橘国雄绘画。},又有了《点石斋丛画》\footnote{〔《点石斋丛画》〕点石斋书局印刷的画册,汇集了数百首诗及明清画家的插画。点石斋,1876年由英国商人厄内斯特·美查(Ernest Major)在上海创办。}和《诗画舫》\footnote{〔《诗画舫》〕汇集明晚期画家的唐诗插画谱,收录近五百首诗。}。《山海经》也另买了一部石印的,每卷都有图赞,绿色的画,字是红的,比那木刻的精致得多了。这一部直到前年还在,木刻的却已经记不清是什么时候失掉了。
    
    我的保姆,长妈妈即阿长,辞了这人世,大概也有了三十年了罢。我终于不知道她的姓名,她的经历;仅知道有一个过继的儿子,她大约是青年守寡的孤孀。仁厚黑暗的地母\footnote{〔地母〕中国古代信仰中的大地之神,又称后土。《山海经》中也有记载。与玉皇大帝合称皇天后土。民间认为人死后灵魂归于后土。}呵,愿在你怀里永安她的魂灵!
    
\end{large}


\newpage

\textbf{注释}:

\vspace{-1em}

\begin{itemize}
    \setlength\itemsep{-0.2em}
    \item 〔\xpinyin*{憎恶}〕厌恶仇恨。
    \item 〔\xpinyin*{惶急}〕从旁劝说,使想做。
    \item 〔\xpinyin*{骇}〕惊吓。
    \item 〔\xpinyin*{菩萨}〕佛教中指修行有成的大觉悟者。也指心地慈善的人。
    \item 〔\xpinyin*{霹雳}〕又急又响的雷。
    \item 〔\xpinyin*{震悚}〕震惊惶恐。
    \item 〔\xpinyin*{掷}〕扔,抛。
    \item 〔\xpinyin*{掳}〕抢走,抓走。
    \item 〔\xpinyin*{孀}〕称呼丧夫的寡妇。
    \item 〔\xpinyin*{图赞}〕写在画面或图页上的赞美诗文。
    \item 〔\xpinyin*{保姆}〕帮忙带小孩的妇女。
\end{itemize}

\end{document}
