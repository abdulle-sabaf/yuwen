\documentclass[12pt,UTF-8,openany]{ctexbook}
\usepackage{ctex}
\usepackage{titlesec}
\usepackage{xeCJK}
\usepackage{verse}
\usepackage{fontspec,xunicode,xltxtra}
\usepackage{xpinyin}
\usepackage{geometry}
\usepackage{indentfirst}
\usepackage{pifont}
\usepackage[perpage,symbol*]{footmisc}

\geometry{a5paper,left=1.4cm,right=1.4cm,top=2.4cm,bottom=2.4cm}
\setmainfont{Arial}
\setCJKmainfont[BoldFont=STZhongsong]{汉字之美仿宋GBK 免费}
\xeCJKDeclareCharClass{CJK}{`0 -> `9}
\xeCJKsetup{AllowBreakBetweenPuncts=true}
\DefineFNsymbols{circled}{{\ding{192}}{\ding{193}}{\ding{194}}{\ding{195}}{\ding{196}}{\ding{197}}{\ding{198}}{\ding{199}}{\ding{200}}{\ding{201}}}
\setfnsymbol{circled}
\xpinyinsetup{ratio=0.5,hsep={.6em plus .6em},vsep={1em}}
\titleformat{\chapter}{\zihao{-1}\bfseries}{ }{16pt}{}
\titleformat{\section}{\zihao{-2}\bfseries}{ }{0pt}{}
\title{\zihao{0} \bfseries 初中语文课文集萃}
\setlength{\lineskip}{24pt}
\setlength{\parskip}{6pt}
\author{}
\date{}
\begin{document}
\maketitle
\tableofcontents
\newpage

\chapter{匆匆}

\begin{normalsize}
    
    燕子去了,有再来的时候;杨柳枯了,有再青的时候;桃花谢了,有再开的时候。但是,聪明的,你告诉我,我们的日子为什么一去不复返呢?——是有人偷了他们吧:那是谁?又藏在何处呢?是他们自己逃走了吧:现在又到了哪里呢?
    
    我不知道他们给了我们多少日子,但我的手确乎是渐渐空虚了。在默默里算着,八千多日子已经从我手中溜去;像针尖上一滴水滴在大海里,我的日子滴在时间的流里,没有声音,也没有影子。我不禁头涔涔而泪潸潸了。
    
    去的尽管去了,来的尽管来着,去来的中间,又怎样地匆匆呢?早上我起来的时候,小屋里射进两三方斜斜的太阳。太阳他有脚啊,轻轻悄悄地挪移了,我也茫茫然跟着旋转。于是——洗手的时候,日子从水盆里过去;吃饭的时候,日子从饭碗里过去;默默时,便从凝然的双眼前过去;我觉察他去得匆匆了,伸出手遮挽时,他又从遮挽的手边过去;天黑时,我躺在床上,他便伶伶俐俐地从我身上跨过,从我脚边飞去了;等我睁开眼和太阳再见,这算又溜走了一日;我掩面叹息,但是新来的日子的影儿又开始在叹息里闪过了。
    
    在逃去如飞的日子里,在千门万户的世界里的我能做什么呢?只有徘徊罢了,只有匆匆罢了。在八千多日的匆匆里,除徘徊外,又剩些什么呢?过去的日子如轻烟,被微风吹散了,如薄雾,被初阳蒸融了。我留着些什么痕迹呢?我何曾留着像游丝样的痕迹呢?我赤裸裸来到这世界,转眼间也将赤裸裸地回去吧?但不能平的,为什么偏要白白走这一遭啊?
    
    你聪明的,告诉我,我们的日子为什么一去不复返呢?
    
\end{normalsize}



\chapter{济南的冬天}

\begin{normalsize}
    
    对于一个在北平\footnote{〔北平〕北京的旧称。}住惯的人,像我,冬天要是不刮风,便觉得是奇迹;济南的冬天是没有风声的。对于一个刚由伦敦\footnote{〔伦敦〕英国首都。}回来的人,像我,冬天要能看得见日光,便觉得是怪事;济南的冬天是响晴的。自然,在热带\footnote{〔热带〕地球南北回归线之间的地带。四季炎热。}的地方,日光是永远那么毒,响亮的天气,反有点叫人害怕。可是,在北中国的冬天,而能有温晴的天气,济南真得算个宝地。
    
    设若\footnote{〔设若〕如果、假若。}单单是有阳光,那也算不了出奇。请闭上眼睛想:一个老城,有山有水,全在天底下晒着阳光,暖和安适地睡着,只等春风来把它们唤醒,这是不是个理想的境界?小山整把济南围了个圈儿,只有北边缺着点口儿。这一圈小山在冬天特别可爱,好像是把济南放在一个小摇篮里,它们安静不动地低声地说:“你们放心吧,这儿准保暖和。”真的,济南的人们在冬天是面上含笑的。他们一看那些小山,心中便觉得有了着落,有了依靠。他们由天上看到山上,便不知不觉地想起:“明天也许就是春天了吧?这样的温暖,今天夜里山草也许就绿起来了吧?”就是\footnote{〔就是〕这里同“就算”。}这点幻想不能一时实现,他们也并不着急,因为有这样慈善的冬天,干啥还希望别的呢!
    
    最妙的是下点小雪呀。看吧,山上的矮松越发的青黑,树尖上顶着一髻儿白花\footnote{〔髻〕原指头顶或脑后盘成的各种形状的头发。这里比喻树顶上的积雪如发髻一般。},仿佛雪地里的灰松鼠。山尖全白了,给蓝天镶上一道银边。山坡上,有的地方雪厚点,有的地方草色还露着;这样,一道儿白,一道儿暗黄,给山们穿上一件带水纹的花衣;看着看着,这件花衣好像被风儿吹动,叫你希望看见一点更美的山的肌肤。等到快日落的时候,微黄的阳光斜射在山腰上,那点薄雪好像忽然害了羞,微微露出点粉色。就是下小雪吧,济南是受不住大雪的,那些小山太秀气!
    
    古老的济南,城里那么狭窄,城外又那么宽敞,山坡上卧着些小村庄,小村庄的房顶上卧着点雪,对,这是张小水墨画,或者是唐代的名手画的吧。
    
    那水呢,不但不结冰,倒反在绿藻上冒着点热气。水藻真绿,把终年贮蓄的绿色全拿出来了。天儿越晴,水藻越绿,就凭这些绿的精神,水也不忍得冻上;况且那长枝的垂柳还要在水里照个影儿呢!看吧,由澄清的河水慢慢往上看吧,空中,半空中,天上,自上而下全是那么清亮,那么蓝汪汪的,整个的是块空灵的蓝水晶。这块水晶里,包着红屋顶,黄草山,像地毯上的小团花的小灰色树影;这就是冬天的济南。
    
\end{normalsize}


\newpage

\textbf{注释}:

\vspace{-1em}

\begin{itemize}
    \setlength\itemsep{-0.2em}
    \item 〔响晴〕晴朗无云。
    \item 〔水墨画〕用水、墨而不用彩色颜料的国画。
    \item 〔名手〕这里指有名的画家。手:擅长做某事的人。
    \item 〔贮蓄〕储存,积聚。
    \item 〔空灵〕空静而又灵动,难以捉摸。
\end{itemize}

\chapter{紫藤萝瀑布}

\begin{normalsize}
    
    我不由得停住了脚步。
    
    从未见过开得这样盛的藤萝,只见一片辉煌的淡紫色,像一条瀑布,从空中垂下,不见其发端,也不见其终极。只是深深浅浅的紫,仿佛在流动,在欢笑,在不停地生长。紫色的大条幅上,泛着点点银光,就像迸溅的水花。仔细看时,才知道那是每一朵紫花中的最浅淡的部分,在和阳光互相挑逗。
    
    这里春红已谢,没有赏花的人群,也没有蜂围蝶阵。有的就是这一树闪光的、盛开的藤萝。花朵儿一串挨着一串,一朵接着一朵,彼此推着挤着,好不活泼热闹!
    
    “我在开花!”它们在笑。
    
    “我在开花!”它们嚷嚷。
    
    每一穗花都是上面的盛开、下面的待放 。颜色便上浅下深,好像那紫色沉淀下来了,沉淀在最嫩最小的花苞里。每一朵盛开的花就像是一个小小的张满了的帆,帆下带着尖底的舱,船舱鼓鼓的;又像一个忍俊不禁的笑容,就要绽开似的。那里装的是什么仙露琼浆?我凑上去,想摘一朵。
    
    但是我没有摘。我没有摘花的习惯。我只是伫立凝望,觉得这一条紫藤萝瀑布不只在我眼前,也在我心上缓缓流过。流着流着,它带走了这些时一直压在我心上的焦虑和悲痛,那是关于生死谜、手足情的。我沉浸在这繁密的花朵的光辉中,别的一切暂时都不存在,有的只是精神的宁静和生的喜悦。
    
    这里除了光彩,还有淡淡的芳香,香气似乎也是浅紫色的,梦幻一般轻轻地笼罩着我。忽然记起十多年前家门外也曾有过一大株紫藤萝,它依傍一株枯槐爬得很高,但花朵从来都稀落,东一穗西一串伶仃地挂在树梢,好像在试探什么。后来索性连那稀零的花串也没有了。园中别的紫藤花架也都拆掉,改种了果树。那时的说法是,花和生活腐化有什么必然关系。我曾遗憾地想:这里再也看不见藤萝花了。
    
    过了这么多年,藤萝又开花了,而且开得这样盛,这样密,紫色的瀑布遮住了粗壮的盘虬卧龙般的枝干,不断地流着,流着,流向人的心底。
    
    花和人都会遇到各种各样的不幸,但是生命的长河是无止境的。我抚摸了一下那小小的紫色的花舱,那里满装生命的酒酿,它张满了帆,在这闪光的花的河流上航行。它是万花中的一朵,也正是一朵朵花,组成了万花灿烂的流动的瀑布。
    
    在这浅紫色的光辉和浅紫色的芳香中,我不觉加快了脚步。
    
\end{normalsize}



\chapter{太空一日}

\begin{normalsize}
    
    我以为自己要牺牲了
    
    9时整,火箭尾部发出巨大的轰鸣声,数百吨高能燃料开始燃烧,八台发动机同时喷出炽热的火焰,高温高速的气体,几秒钟就把发射台下的上千吨水化为蒸汽。
    
    火箭起飞了。
    
    我全神贯注,肌肉紧绷,整个人收得像一块铁,准备执行动作。
    
    飞船缓缓升起,非常平稳,甚至比电梯还平稳。我感到压力远不像训练时想象的那么大,稍稍释然,全身绷紧的肌肉也渐渐放松下来。
    
    “逃逸塔\footnote{〔逃逸塔〕飞船顶端的逃生装置。可在火箭升空期间出现危急状况时,让航天员迅速脱离危险区域。}分离”,“助推器分离”……
    
    火箭逐渐加速,我感到压力渐渐增强。这种负荷我们训练时承受过,变化幅度甚至比训练时还小些,所以我的身体感受还挺好,觉得没啥问题。
    
    然而,就在火箭上升到三四十公里的高度时,火箭和飞船产生了共振\footnote{〔共振〕物体受外界振动刺激时,产生特别强烈的振动的现象。},开始急剧振动。这让我非常痛苦。
    
    人体对10赫兹\footnote{〔赫兹〕每秒振动的次数。10赫兹表示每秒振动10次。}以下的振动非常敏感。它会让人的内脏产生共振。不仅如此,当时的负荷大约有六倍重力加速度\footnote{〔重力加速度〕重力导致的加速度。六倍重力加速度相当于身体重量变为六倍,感觉如同自身五倍的重量压在全身。},两者叠加,实在太可怕了,我们从来没有进行过这种训练。
    
    意外出现了。
    
    共振时强时弱,痛苦越来越强烈,我异常清醒,只觉得五脏六腑似乎都要碎了。我几乎难以承受,觉得自己快不行了。
    
    当时,我以为飞船起飞时就是这样的。其实,起飞阶段发生的共振并非正常现象。
    
    共振持续26秒后,慢慢减轻。我从极度难受的状态解脱出来,一切不适都不见了,只感到从未有过的轻松和舒服,如释千钧重负,如同重生。我甚至觉得这个过程很耐人寻味。但在痛苦的极点,就在那短短一刹那,我真的以为自己要牺牲了。
    
    飞行回来后,我详细描述了这段难受的过程。经过分析研究,工作人员认为,飞船的共振主要来自火箭的振动。随后他们改进工艺,解决了这个问题。“神舟六号”飞行时,情况有了很大改善;后来的航天飞行中再没有出现过这种问题。聂海胜\footnote{〔聂海胜〕中国航天员。2005年10月,他和费俊龙成功执行“神舟六号”载人航天飞行任务。}说:“我们乘坐的火箭、飞船都非常舒适,几乎感觉不到振动。”
    
    在空中度过那难以承受的26秒时,不仅我感觉特别漫长,地面的工作人员也陷入了空前的紧张中。因为通过大屏幕,飞船传回来的画面是定格的,我整个人一动不动,眼睛也不眨。大家都担心我是不是出了什么事故。
    
    后来,整流罩\footnote{〔整流罩〕套在飞行器上的保护罩。用于减少空气阻力,免除飞行时气流、热流的影响。}打开,外面的光线透过舷窗一下子照射进来,阳光很刺眼,我的眼睛忍不住眨了一下。
    
    就这一下,指挥大厅有人大声喊道:“快看啊,他眨眼了,利伟还活着!”所有的人都鼓掌欢呼起来。
    
    这是回到地面后,我看了升空时指挥大厅的录像才知道的。那一刻,所有的人都在流泪。看到这里的时候,我感动得说不出话来。
    
    我看到了什么
    
    此后一切顺利。升空后10分钟左右,飞船仿佛一下子跳进了轨道。我突然有了失重的感觉。
    
    好容易等到地面指挥人员下达指令,我迫不及待地摘下束缚带,飘到舷窗边上。
    
    哈!太空和地球一下子出现在我眼前。
    
    我先望向地球。从飞船上看到的地球,只是一段弧面,不是完整的球体。因为地球的半径有六千多公里,而飞船距离地面343公里左右。我们平常在地理书上看到的地球照片,是由飞行轨道更高的同步卫星拍摄而来。
    
    地球真的太漂亮了。她散发着柔和的光芒,仿佛披着蓝色纱裙和白色飘带的仙女,款款而行。蓝色的弧面之外,是深远幽黑的宇宙。
    
    飞船每90分钟就绕地球一圈,一共飞行了14圈。我也看了14次日出和日落。我曾在新疆的天山上,也曾站在家乡的大海边看日出,但都无法与太空中的日出相比。一条亮白的金弧不断延伸,太阳就是镶在中间的宝珠,发出炫目的光。金弧逐渐扩散开来,把光明涂抹在广袤的弧面上,一切都清晰起来。日落时,一切又追随着太阳涌去,汇成一条光弧,再彻底消失。
    
    在太空中,我可以准确判断各大洲和各个国家的方位。因为飞船有预定的飞行轨迹,显示屏上实时标示着飞船走到哪个位置,投影到地球上是哪一点。有图可依,一目了然。
    
    即使不借助仪器和地图,以我们航天课程中学到的知识,从山脉的轮廓,海岸线的走向以及河流的形状,我也基本可以判断出飞船正经过哪个洲的上空,正在经过哪个国家。
    
    经过亚洲,特别是到中国上空时,我就仔细辨别大概到哪个省了。飞船经过中国上空的时间很短,每一次飞过后,我都期待着下一次。
    
    飞船的轨迹大都是不重复的,在距离地面三百多公里的高度上俯瞰,视野广阔,祖国的各个省份我大都看到了。
    
    我曾俯瞰我们的首都北京。白天它是燕山山脉边的一片灰白色,分辨不清;夜晚则呈现一片红晕。那里有我的战友和亲人。
    
    我看到中国东部优美的海岸线、长白山脉,那里是辽宁,我的家乡;我看到甘肃、新疆,披着积雪的昆仑山脉和大片沙漠,我曾在那里驾机飞行,也从那里乘火箭升空;我看到了曲折的黄河横穿陕西、山西、山东数省;我看到了西藏和青藏高原,我看到了四川、安徽、江苏、上海,蜿蜒的长江奔向大海;我看到了东南方向的台湾岛,看上去它与大陆几乎没有间隔;我看到了宽广的内蒙古一片平阔,而我将在那里降落……
    
    神秘的敲击声
    
    作为首飞航天员,除了一些小难题,我还遇到了许多突发的、原因不明的、不在预案中的情况。
    
    比如,当飞船刚刚入轨,进入失重状态时,百分之八九十的航天员都会产生一种“本末倒置”的错觉。这种错觉很难受,明明是朝上坐的,却感觉脑袋朝下。如果不消除这种倒悬的错觉,就会觉得自己一直在倒着飞,很难受,严重时还可能诱发空间运动病\footnote{〔空间运动病〕人的空间平衡感失调导致的疾病。晕车、晕船、晕飞机,都属于空间运动病。},影响任务的完成。
    
    在地面时,没人提到过这种情况。即使知道,训练也无法模拟。估计在我之前遨游太空的国外航天员有类似体会,但他们从未对我说起过。
    
    在这个情况下,没别的办法,只能靠意志力克服这种错觉。我想像自己在地面训练的情景,眼睛闭着猛想,不停地想,给身体一个适应的过程。几十分钟后,我终于调整过来了。
    
    “神舟六号”和“神舟七号”升空后,航天员都产生了这种错觉,但他们已有心理准备,因为我跟他们仔细讲过。而且,飞船舱体也经过改进,内壁上下刷了不同的颜色:天花板是白色的,地板是褐色的。这样有助于航天员迅速调整感觉。
    
    我在太空还遇到另一个至今仍然原因不明的情况,那就是时不时出现的敲击声。这个声音是突然出现的。并不一直响,而是一阵一阵的。不管白天还是黑夜,毫无规律,说不准什么时候就响几声。既不是外面传进来的声音,也不是飞船里面的声音,仿佛谁在外面敲飞船的船体。很难准确描述它:不是叮叮的,也不是当当的,更像是用一把木头锤子敲铁桶,咚……咚咚……咚……
    
    鉴于飞船的运行一直很正常,我并没有向地面报告这一情况。但我自己还是很紧张,因为第一次飞行,生怕哪里出了问题。每当响声传来的时候,我就趴在舷窗那里,边听边看,试图找出响声所在,但什么也没能发现。
    
    回到地面后,人们对这个神秘的声音做过许多猜测。技术人员想弄清它到底来自哪里,就用各种办法模拟它,拿着录音让我一次又一次听,我却总是觉得不像。对航天员的最基本要求是严谨,不是当时的声音,我就不能签字,所以他们就让我反复听各种声音,断断续续听了一年多。但是直到现在,那个神秘的声音也没有在我耳边准确地再现过。
    
    在“神舟六号”和“神舟七号”飞行时,这个声音又出现了,但我告诉航天员:“出现这个声音别害怕,是正常现象。”
    
    归途如此惊心动魄
    
    5时35分,北京航天指挥中心向飞船发出“返回”指令。飞船开始在343公里高的轨道上制动,就像刹车一样。
    
    飞船先是在轨道上进行180度调姿——返回时要让推进舱在前,这就需要“掉头”。
    
    “制动发动机关机!”5时58分,飞船的速度减到一定数值,开始脱离原来的轨道,进入无动力飞行状态。
    
    6时4分,飞船下降至距地100公里,进入逐渐稠密的大气层。
    
    这时飞船的飞行速度仍然很快,遇到空气阻力后,它急剧减速,产生了近四倍重力加速度的过载\footnote{〔过载〕过大的加速度(比重力加速度更大的加速度)。}。我的前胸和后背都承受着很大的压力。我们平时已经训练过如何应对这种情况,因此身体应付自如,也没有紧张。
    
    让我紧张以至于惊慌的另有原因。
    
    飞船进入了“黑障”区\footnote{〔“黑障”〕航天飞行中出现的现象。在距离地面数十公里的高空高速飞行时,飞行器和大气摩擦产生的高温,使气体分子电离,并在飞行器表面形成离子层,阻碍电磁波通过。飞行器无法用电磁波与外界联系,因此称为“黑障”。},距地大约80公里到40公里。首先是快速行进的飞船与大气摩擦,产生的高温把舷窗外面烧得一片通红;接着在映红的舷窗外,有红的白的碎片不停划过。飞船的外表面有防烧蚀层,它是耐高温的,随着温度升高,开始剥落,并在剥落的过程中会带走一部分热量。我学习过这方面的知识,看到这种情形,知道是怎么回事。
    
    但随后发生的情况让我非常紧张——右边的舷窗开始出现裂纹。窗外烧得跟炼钢炉一样,而窗上出现裂纹。那纹路就跟强化玻璃被打碎之后的小碎纹一样。这种细细的碎纹,眼看着越来越多……说不恐惧那是假话。你想啊,窗外边可是有1600至1800摄氏度!
    
    我的汗水出来了。这时舱内的温度也在升高,但并没到让我瞬间出汗的程度,主要还是因为紧张。
    
    我现在还能清楚地记起当时的情形:飞船急速下降,跟空气摩擦产生激波\footnote{〔激波〕气流的速度超过了气体扰动传播的速度,使气流突然压缩变稠密,产生高温高热的现象。},不仅带来极高的温度,还伴随着尖利的呼啸声;飞船带着不小的过载,不停振动着,里面咯吱咯吱乱响。外面高温,不怕!有碎片划过,不怕!过载,也能承受!但是,看到舷窗玻璃开始出现裂缝,我紧张了,心想:完了,这个舷窗不行了。美国的“哥伦比亚号”航天飞机,不就是这样出事的吗?先是一块防热板出现裂缝,然后高热就使飞机解体了。这么大一个舷窗坏了,那还得了!
    
    右边的舷窗裂到一半的时候,左边的舷窗也开始出现裂纹。这反倒让我稍微放心了:哦——可能没什么问题!因为如果是故障,重复出现的概率并不高。
    
    回来之后,我才知道,飞船的舷窗外做了一层防烧涂层,是这个涂层烧裂了,而不是窗玻璃本身出现了问题。为什么两边没有同时出现裂纹呢?因为两边用了不同的材料。以前每次进行飞船发射与返回的实验,返回的飞船舱体经过高温烧灼,舷窗黑乎乎的,工作人员看不到这些裂纹。如果不是在飞船内亲眼所见,谁都不会想到有这种情况。
    
    距离地面还有40公里,飞船出了“黑障”区,速度已经降下来。一个关键的操作——抛伞,即将开始。这时舷窗已经烧得黑乎乎的,我抱着操作盒,屏息凝神,等待着配合程序:到哪里该做什么,该发什么指令,判断和操作都必须准确无误。
    
    6时14分,飞船距地10公里。飞船抛开降落伞盖,并迅速带出引导伞。
    
    这是一个剧烈的动作,能听到“砰”的一声,非常响。我在里边感觉被狠狠地一拽,瞬间过载很大,对身体的冲击也非常厉害。接下来是一连串的快速动作。引导伞出来后,紧跟着把减速伞也带出来,减速伞让飞船减速下落,16秒之后再把主伞带出来。
    
    其实最折磨人的就是这段过程了。随着一声巨响,你会感到突然一减速;引导伞一开,使劲一提,这个劲很大,会把人吓一跳;减速伞一开,又往那边一拽;主伞开时又把你拉到另一边。每次力量都相当大。飞船晃荡得很厉害,让人不知道是怎么回事。
    
    我们航天员是很重视这段过程的:伞开得好等于安全有保障,至少不会丢了性命。所以我被七七八八地拽了一通,平稳下来后心里却真是踏实——数据出来了,速度控制在规定范围内。我知道,这伞肯定是开好了!
    
    离地面5公里的时候,飞船抛掉防热大底,露出缓冲发动机。同时主伞也变成双点吊挂,让飞船摆正姿态,在风中晃悠着落向地面。
    
    飞船离地面1.2米时,缓冲发动机点火。接着,飞船“嗵”的一下落地了。
    
    我感觉落地很重,飞船弹了起来。在它第二次落地时,我迅速按下了切伞开关\footnote{〔切伞〕将飞船与降落伞分离。}。飞船停住了。此时是2003年10月16日6时23分。而这一时刻,正好是天安门当天升国旗的时刻,这是一个无法设计的巧合。
    
    那一刻四周寂静无声。舷窗黑乎乎的,看不到外面。
    
    过了几分钟,我隐约听到了叫喊声,手电的光从烧黑的舷窗上隐约照进来。他们找到飞船了!我听到外面插上钥匙的声音,舱门动弹了……
    
\end{normalsize}



\chapter{从百草园到三味书屋}

\begin{normalsize}
    
    我家的后面有一个很大的园,相传叫作百草园。现在是早已并屋子一起卖给朱文公的子孙了\footnote{〔朱文公〕南宋学者朱熹死后的谥号。这里指把屋子卖给一个姓朱的人。},连那最末次的相见也已经隔了七八年,其中似乎确凿只有一些野草;但那时却是我的乐园。
    
    不必说碧绿的菜畦,光滑的石井栏,高大的皂荚树\footnote{〔皂荚树〕一种乔木,果实像扁豆,长约20厘米,捣碎了泡水,可以洗衣服。},紫红的桑椹\footnote{〔桑椹〕桑树的果实,又叫桑葚。};也不必说鸣蝉在树叶里长吟,肥胖的黄蜂伏在菜花上,轻捷的叫天子\footnote{〔叫天子〕一种小鸟,又叫云雀。体长约20厘米,叫声响亮。}忽然从草间直窜向云霄里去了。单是周围的短短的泥墙根一带,就有无限趣味。油蛉\footnote{〔油蛉〕一种昆虫,俗名金钟儿,形似西瓜子,黑色,昼夜都鸣。}在这里低唱,蟋蟀们在这里弹琴。翻开断砖来,有时会遇见蜈蚣;还有斑蝥\footnote{〔斑蝥〕一种昆虫,能飞,翅上有黄黑色斑纹。这里是指类似斑蝥的“行夜虫”,俗称“放屁虫”。},倘若用手指按住它的脊梁,便会拍的一声,从后窍\footnote{〔后窍〕这里指昆虫的肛门。}喷出一阵烟雾。何首乌\footnote{〔何首乌〕一种多年生蔓草,根粗大,可入药。}藤和木莲\footnote{〔木莲〕一种蔓生的常绿灌木。}藤缠络着,木莲有莲房\footnote{〔莲房〕莲蓬。}一般的果实,何首乌有臃肿的根。有人说,何首乌根是有像人形的,吃了便可以成仙,我于是常常拔它起来,牵连不断地拔起来,也曾因此弄坏了泥墙,却从来没有见过有一块根像人样。如果不怕刺,还可以摘到覆盆子\footnote{〔覆盆子〕一种多年生草,茎长,有刺,夏天结果实。},象小珊瑚珠攒成的小球,又酸又甜,色味都比桑椹要好得远。
    
    长的草里是不去的,因为相传这园里有一条很大的赤练蛇\footnote{〔赤练蛇〕一种无毒蛇。体长一米左右,有黑红相间的斑纹。}。
    
    长妈妈\footnote{〔长妈妈〕鲁迅小时候家里的女工,下文的阿长也指她。}曾经讲给我一个故事听:先前,有一个读书人住在古庙里用功,晚间,在院子里纳凉的时候,突然听到有人在叫他。答应着,四面看时,却见一个美女的脸露在墙头上,向他一笑,隐去了。他很高兴;但竟给那走来夜谈的老和尚识破了机关\footnote{〔机关〕这里指周密而巧妙的计谋。}。说他脸上有些妖气,一定遇见“美女蛇”了;这是人首蛇身的怪物,能唤人名,倘一答应,夜间便要来吃这人的肉的。他自然吓得要死,而那老和尚却道无妨,给他一个小盒子,说只要放在枕边,便可高枕而卧。他虽然照样办,却总是睡不着,——当然睡不着的。到半夜,果然来了,沙沙沙!门外象是风雨声。他正抖作一团时,却听得豁的一声,一道金光从枕边飞出,外面便什么声音也没有了,那金光也就飞回来,敛在盒子里。后来呢?后来,老和尚说,这是飞蜈蚣,它能吸蛇的脑髓,美女蛇就被它治死了。
    
    结末的教训是:所以倘有陌生的声音叫你的名字,你万不可答应他。
    
    这故事很使我觉得做人之险,夏夜乘凉,往往有些担心,不敢去看墙上,而且极想得到一盒老和尚那样的飞蜈蚣。走到百草园的草丛旁边时,也常常这样想。但直到现在,总还没有得到,但也没有遇见过赤练蛇和美女蛇。叫我名字的陌生声音自然是常有的,然而都不是美女蛇。
    
    冬天的百草园比较的无味;雪一下,可就两样了。拍雪人和塑雪罗汉需要人们鉴赏,这是荒园,人迹罕至,所以不相宜,只好来捕鸟。薄薄的雪,是不行的;总须积雪盖了地面一两天,鸟雀们久已无处觅食的时候才好。扫开一块雪,露出地面,用一支短棒支起一面大的竹筛来,下面撒些秕谷,棒上系一条长绳,人远远地牵着,看鸟雀下来啄食,走到竹筛底下的时候,将绳子一拉,便罩住了。但所得的是麻雀居多,也有白颊的“张飞鸟”\footnote{〔张飞鸟〕鹡鸰。头部像戏台上张飞的脸谱,所以浙东也有叫张飞鸟。},性子很躁,养不过夜的。
    
    这是闰土的父亲\footnote{〔闰土〕作者在小说《故乡》中写到的儿时朋友。}所传授的方法,我却不大能用。明明见它们进去了,拉了绳,跑去一看,却什么都没有,费了半天力,捉住的不过三四只。闰土的父亲是小半天便能捕获几十只,装在叉袋\footnote{〔叉袋〕一种装粮食的布袋或者麻袋,袋口有叉角,可以打结。}里叫着撞着的。我曾经问他得失的缘由,他只静静地笑道:你太性急,来不及等它走到中间去。
    
    我不知道为什么家里的人要将我送进书塾里去了,而且还是全城中称为最严厉的书塾。也许是因为拔何首乌毁了泥墙罢,也许是因为将砖头抛到间壁的梁家去了罢,也许是因为站在石井栏上跳下来罢,……都无从知道。总而言之:我将不能常到百草园了。\texttt{Ade}\footnote{〔\texttt{Ade}〕德语,再见的意思。},我的蟋蟀们!\texttt{Ade},我的覆盆子们和木莲们!……
    
    出门向东,不上半里,走过一道石桥,便是我的先生的家了。从一扇黑油的竹门进去,第三间是书房。中间挂着一块匾额:三味书屋\footnote{〔三味书屋〕在绍兴城内作者故家附近。解放后辟为鲁迅纪念馆的一部分。};匾下面是一幅画,画着一只很肥大的梅花鹿伏在古树下。没有孔子牌位,我们便对着那匾和鹿行礼。第一次算是拜孔子,第二次算是拜先生。
    
    第二次行礼时,先生\footnote{〔先生〕作者的启蒙老师,姓寿,名怀鉴,字镜吾,是一个老秀才。}便和蔼地在一旁答礼。他是一个高而瘦的老人,须发都花白了,还戴着大眼镜。我对他很恭敬,因为我早听到,他是本城中极方正,质朴,博学的人。
    
    不知从哪里听来的,东方朔\footnote{〔东方朔〕西汉文学家,善辞赋,性格诙谐滑稽。关于他的民间传说很多。}也很渊博,他认识一种虫,名曰“怪哉”,冤气所化,用酒一浇,就消释了。我很想详细地知道这故事,但阿长是不知道的,因为她毕竟不渊博。现在得到机会了,可以问先生。
    
    “先生,‘怪哉”\footnote{〔怪哉〕据《太平广记》,汉武帝巡视时发现的人面怪虫。询问东方朔,东方朔回答说,过去秦朝拘押无辜的人太多,百姓纷纷感叹:“怪哉怪哉!”愤怒感动上天,产生这种虫。汉武帝问他有什么办法可以去掉,东方朔回答:喝酒可以消愁。用酒浇灌,可以让这虫子消失。}这虫,是怎么一回事?……”我上了生书\footnote{〔生书〕未读过的书,新课。},将要退下来的时候,赶忙问。
    
    “不知道!”他似乎很不高兴,脸上还有怒色了。
    
    我才知道做学生是不应该问这些事的,只要读书,因为他是渊博的宿儒\footnote{〔宿〕长久从事某种工作。儒:信奉孔孟之道的知识分子。},决不至于不知道,所谓不知道者,乃是不愿意说。年纪比我大的人,往往如此,我遇见过好几回了。
    
    我就只读书,正午习字,晚上对课\footnote{〔对课〕即对对联,旧时学习词句、准备作诗的一种练习。一般老师出上联,学生对下联。三言、五言,即三字一句、五字一句。字数越多越难。}。先生最初这几天对我很严厉,后来却好起来了,不过给我读的书渐渐加多,对课也渐渐地加上字去,从三言到五言,终于到七言。
    
    三味书屋后面也有一个园,虽然小,但在那里也可以爬上花坛去折腊梅花,在地上或桂花树上寻蝉蜕\footnote{〔蝉蜕〕蝉的幼虫变为蝉时脱去的外壳,可入药。}。最好的工作是捉了苍蝇喂蚂蚁,静悄悄地没有声音。然而同窗们到园里的太多,太久,可就不行了,先生在书房里便大叫起来:
    
    “人都到哪里去了?”
    
    人们便一个一个陆续走回去;一同回去,也不行的。他有一条戒尺,但是不常用,也有罚跪的规矩,但也不常用,普通总不过瞪几眼,大声道:
    
    “读书!”
    
    于是大家放开喉咙读一阵书,真是人声鼎沸。有念“仁远乎哉我欲仁斯仁至矣\footnote{见《论语·述而》,应读为“仁远乎哉?我欲仁,斯仁至矣!”}”的,有念“笑人齿缺曰狗窦大开”\footnote{见《幼学琼林·身体》,原句是:“笑人缺齿,狗窦胡为大开?”}的,有念“上九潜龙勿用”\footnote{见《周易》,原句是:“初九,潜龙勿用。”}的,有念“厥土下上上错厥贡苞茅橘柚”\footnote{见《尚书·禹贡》,原句是:“厥土惟涂泥。厥田惟下下,厥赋下上,上错。……厥包橘柚锡贡。”}的……先生自己也念书。后来,我们的声音便低下去,静下去了,只有他还大声朗读着:
    
    “铁如意,指挥倜傥,一座皆惊呢;金叵罗\footnote{〔叵罗〕古代饮酒用的一种敞口的浅杯。},颠倒淋漓噫,千杯未醉嗬……”
    
    我疑心这是极好的文章,因为读到这里,他总是微笑起来,而且将头仰起,摇着,向后面拗过去,拗过去。
    
    先生读书入神的时候,于我们是很相宜的。有几个便用纸糊的盔甲套在指甲上做戏。
    
    我是画画儿,用一种叫作“荆川纸”\footnote{〔荆川纸〕一种竹纸,薄而略透明。}的,蒙在小说的绣像\footnote{〔绣像〕明清以来,通俗小说前面往往附有书中人物的图像,称为绣像。}上一个个描下来,象习字时候的影写一样。读的书多起来,画的画也多起来;书没有读成,画的成绩却不少了,最成片断\footnote{〔片断〕片段。}的是《荡寇志》\footnote{〔《荡寇志》〕清朝俞万春著的一部反《水浒传》、诬蔑歪曲梁山起义的小说。}和《西游记》的绣像,都有一大本。后来,因为要钱用,卖给一个有钱的同窗了。他的父亲是开锡箔\footnote{〔锡箔〕这里指锡箔纸,附着一层薄锡的纸,旧时多用于祭奠死去的人。}店的;听说现在自己已经做了店主,而且快要升到绅士\footnote{〔绅士〕地方上有影响力、有威望的人,辅助官府统治、维持秩序。}的地位了。这东西早已没有了罢。
    
    \hfill 九月十八日
    
\end{normalsize}


\newpage

\textbf{注释}:

\vspace{-1em}

\begin{itemize}
    \setlength\itemsep{-0.2em}
    \item 〔菜畦〕种植蔬菜的一排排整齐的小块田地。四周围着土埂,便于管理和浇灌。
    \item 〔臃肿〕胖大。
    \item 〔秕谷〕不饱满的谷粒。
    \item 〔敛〕收拢。
    \item 〔脑髓〕脑浆。
    \item 〔无从〕没有方法、门路(做某事)。
    \item 〔同窗〕在同一窗下读书的人,指同学。
    \item 〔人迹罕至〕少有人来。迹:足迹。罕:稀少。
    \item 〔戒尺〕老师用来责罚学生的长条形木板。
    \item 〔书塾〕旧时家庭、宗族或教师自己设立的教学场所。
    \item 〔倜傥〕洒脱不受拘束的样子。
    \item 〔淋漓〕濡湿流淌的样子。也形容酣畅、痛快。
    \item 〔拗〕弯屈,弯转。
    \item 〔盔甲〕古代战士的护身装备。头戴的称为“盔”,身穿的称为“甲”。
    \item 〔影写〕把纸蒙在字帖上照着描。
    \item 〔人声鼎沸〕许多人吵闹,声音像大锅里沸腾的水。
    \item 〔渊博〕精深而广博。
\end{itemize}

\chapter{国王的新衣}

\begin{normalsize}
    
    很久以前有一位国王,他非常喜欢穿新衣服。为了穿得漂亮,他把所有的钱都花到衣服上去了。他一点也不关心他的军队,不愿意坐马车出游,甚至不爱去戏院看戏,除非是为了炫耀一下新衣服。他要求每天每个钟点都准备好新衣服。人们一提到国王,总是说:“王上\footnote{〔王上〕古代对国王的尊称。}在更衣室里。”
    
    由于国王喜爱新衣服,很多裁缝、织工、鞋匠都到王宫来,希望为国王做衣服。一天,两个年轻人来到王宫,自称是来自外国的织工。他们说,他们能织出世上最美丽的布。这种布不仅美丽无比,还有一个奇妙的作用:凡是愚蠢的人,都看不见用这布做成的衣服。
    
    “这可不是最适合我的衣服嘛!”国王心想,“我穿了这样的衣服,就可以看出我的王国里哪些人不称职;我就可以辨别出哪些人是聪明人,哪些人是傻子。没错,我要叫他们马上织出这样的布来!”他付了一大笔钱给这两个人,赐给他们带庭院的住宅,叫他们马上开始工作。
    
    外国的织工摆出两架织机来,装做在工作的样子,可是他们的织机上什么东西也没有。他们接二连三地请求国王赏赐最好的蚕丝和金线给他们。他们把这些好东西都装进自己的腰包,却假装在那两架空空的织机上忙碌地工作,一直忙到深夜。
    
    国王等了好几天。他太想知道究竟织得怎样了。不过,他立刻就想起来,愚蠢的人是看不见这布的,这让他心里不太舒服。他相信他自己是用不着害怕的。虽然如此,他还是觉得,先派一个人去看看比较妥当。
    
    “先派老宫相\footnote{〔宫相〕欧洲古代掌管宫廷政务,辅佐国王的大臣。}去看看,”国王想,“他这个人很有头脑,而且不会说谎。”
    
    因此这位正直的老宫相就到那两个外国织工家去。两个骗子正在空空的织机上忙碌地工作着。
    
    “这是怎么一回事儿?”老宫相心想。他把眼睛睁得大大的。
    
    “我什么东西也没有看见!”但是他不敢把这句话说出来。
    
    两个骗子请他走近一点,指着那两架空空的织机问他,布的纹理是不是很精细,色彩是不是很漂亮。
    
    可怜的老宫相的眼睛越睁越大,可还是看不见什么东西,因为的确没有什么东西可看。
    
    “我的老天爷!”他想,“难道我是一个愚蠢的人吗?我从来没有怀疑过我自己。我决不能让人知道这件事——不成,我决不能让人知道我看不见布料。”
    
    “哎,您一点意见也没有吗?”一个正在织布的织工说。
    
    “啊,美极了!真是美妙极了!”老宫相说。他戴着眼镜仔细地看。“多么精细的纹理!多么美的色彩!是的,我会呈报王上,说我对于这布非常满意。”
    
    “听到您的话,我们就放心了。”两个织工一起说。他们把这些丰富的色彩和纹理仔细描述了一番。这位老大臣注意地听着,以便回到国王那里去时,可以照样背得出来。事实上他也这样办了。
    
    两个骗子又要了很多的钱,更多的蚕丝和金线,他们说这是为了织布的需要。他们把这些东西全装进腰包里,连一根线也没有放到织机上去。不过他们还是继续在空空的机架上工作。
    
    过了不久,国王又派了另一位正直可信的大臣,去看布是不是很快就可以织好。这位大臣也遇到了同样的事:他看了又看,但是那两架空空的织机上什么也没有,他什么东西也看不出来。
    
    “您看这段布美不美?”两个骗子问。他们指出各种漂亮的花纹,仔细解释。事实上并没有什么花纹。
    
    “我并不愚蠢!”这位大臣想,“难道我是个蠢货吗?这也真够滑稽,但是我决不能让人看出来!”因此他就把他完全没有看见的布称赞了一番,同时对他们说,他非常喜欢这些美丽的色彩和精致的花纹。“是的,那真是太美了。”他回去对国王说。
    
    城里所有的人都在谈论这神奇的布料。
    
    国王很想亲自去看一次。他特别指定了一批随员\footnote{〔随员〕随同出行的人。}——包括已经去看过的那两位正直的大臣。两个狡猾的骗子正卖力地织布,但是一根线的影子也看不见。“各位,这不漂亮吗?”那两位正直的大臣说,“陛下\footnote{〔陛下〕古代对国王的尊称。}请看,多么精致的花纹!多么美丽的色彩!”他们指着空空的织机,因为他们以为别人一定看得见。
    
    “这是怎么一回事儿呢?”国王心想,“我什么也没有看见!这真是荒唐!难道我是一个愚蠢的人吗?难道我不配做国王吗?这太可怕了。我从没有遇过这样的事情。”
    
    “啊,它真是美极了!”国王说,“我十分满意!”
    
    于是他点头表示满意。他装作很仔细地看着织机的样子,因为他不愿意说出他什么也没有看见。全体随员也仔细地看了又看,可是他们也没有看出任何东西。不过,他们也照着国王的话说:“啊,真是美极了!”他们提出,这种新奇的布料,正适合即将举行的游行庆典\footnote{〔游行庆典〕在大街上行进的庆祝活动。}。国王穿着用它做的新衣服游行,再好不过了。
    
    “真美丽!真精致!真是好极了!”每人都随声附和着,每个人都显得开心极了。国王赐给骗子每人一个爵士的头衔,一枚可以挂在扣眼\footnote{〔扣眼〕上衣胸前用来别纽扣的孔,也可以用来佩挂饰物。}上的勋章。
    
    第二天早上就是游行庆典了。这两个骗子整夜不睡,点起十六支蜡烛。你可以看到他们在连夜赶工,要完成国王的新衣。他们装做把布料从织机上取下来。他们用两把大剪刀在空中裁了一阵子,同时又用没有穿线的针缝了一通。最后,他们齐声说:“请看!新衣服做好了!”
    
    国王带着一群贵族,亲自到来了。两个骗子每人举起一只手,好像他们拿着什么东西。“请看吧,这是裤子,这是上衣!这是披风!”他们指出每一件衣服的名称。“这衣服轻柔得像蜘蛛网一样:穿着它的人会觉得身上什么也没有似的——这也正是这衣服的妙处。”
    
    “一点也不错。”所有的贵族都赞同。可是他们什么也没有看见,因为实际上什么东西也没有。
    
    “现在请陛下脱下衣服,”两个骗子说,“我们要在这个大镜子前为陛下换上新衣。”
    
    国王把身上的衣服统统脱了。这两个骗子装作把刚才缝好的新衣服一件一件给他穿上。他们在他的腰上弄了一阵子,仿佛在系上什么东西:这就是后裙摆\footnote{〔后裙摆〕欧洲古代贵族的一种装束,是拖在身后的很长的一块布。}。国王在镜子面前转了转身子,扭了扭腰肢。
    
    “天啊,这衣服多么合身啊!这剪裁、这式样,多么好看啊!”大家都说,“多么精致的花纹!多么美妙的色彩!这真是一套前所未见、令人惊叹的衣服!”
    
    “华盖\footnote{〔华盖〕国王出行时遮阳遮雨的大伞。}已经准备好了,只等陛下穿好新衣服,就可以开始游行了!”典礼官说。
    
    “对,我已经穿好了。”国王说,“这衣服合我的身么?”于是他又在镜子面前扭动身子,因为他要叫大家看出,他在认真地欣赏他美丽的新衣服。侍从们都把手在地上东摸西摸,仿佛真的在拾起裙摆。他们抬起手来,手中托着空气——他们不敢让人瞧出,他们其实什么也没有看见。
    
    这么着,国王就在华盖下游行起来了。街上看见的人、街旁窗户里望见的人都说:“王上的新衣服真是漂亮!长长的后裙摆多么美丽!衣服多么合身!”谁也不愿意让人知道自己看不见,因为这样就会暴露自己是个愚蠢的家伙。国王所有的衣服从来没有得到过这样一致的称赞。
    
    “可是他什么衣服也没有穿呀!”突然,一个小孩子叫出声来。
    
    “天呐,你听这个天真的声音!”爸爸说。于是大家把这孩子讲的话低声地传开来。
    
    “他并没有穿什么衣服!有一个小孩子说他并没有穿什么衣服呀!”
    
    “他实在是没有穿什么衣服呀!”最终,所有的老百姓都这么说了。这话终于传到了国王耳中。国王有点儿发抖,因为他似乎觉得老百姓讲的是对的。“但是,我决不能让人看出来,否则这庆典就无法收场了。”因此他摆出一副更高傲的神气。他的侍从们跟在他后面,手中托着并不存在的裙摆。
    
\end{normalsize}



\chapter{海燕}

\begin{normalsize}
    
    在苍茫的大海上,狂风卷集着乌云。在乌云和大海之间,海燕像黑色的闪电,在高傲地飞翔。
    
    一会儿翅膀碰着波浪,一会儿箭一般地直冲向乌云,它叫喊着,──就在这鸟儿勇敢的叫喊声里,乌云听出了欢乐。
    
    在这叫喊声里──充满着对暴风雨的渴望!在这叫喊声里,乌云听出了愤怒的力量、热情的火焰和胜利的信心。
    
    海鸥在暴风雨来临之前呻吟着,──呻吟着,它们在大海上飞窜,想把自己对暴风雨的恐惧,掩藏到大海深处。
    
    海鸭也在呻吟着,──它们这些海鸭啊,享受不了生活的战斗的欢乐:轰隆隆的雷声就把它们吓坏了。
    
    蠢笨的企鹅,胆怯地把肥胖的身体躲藏到悬崖底下……只有那高傲的海燕,勇敢地,自由自在地,在泛起白沫的大海上飞翔!
    
    乌云越来越暗,越来越低,向海面直压下来,而波浪一边歌唱,一边冲向高空,去迎接那雷声。
    
    雷声轰响。波浪在愤怒的飞沫中呼叫,跟狂风争鸣。看吧,狂风紧紧抱起一层层巨浪,恶狠狠地把它们甩到悬崖上,把这些大块的翡翠摔成尘雾和碎末。
    
    海燕叫喊着,飞翔着,像黑色的闪电,箭一般地穿过乌云,翅膀掠起波浪的飞沫。
    
    看吧,它飞舞着,像个精灵,──高傲的、黑色的暴风雨的精灵,——它在大笑,它又在号叫……它笑那些乌云,它因为欢乐而号叫!
    
    这个敏感的精灵,——它从雷声的震怒里,早就听出了困乏,它深信,乌云遮不住太阳,──是的,遮不住的!
    
    狂风吼叫……雷声轰响……
    
    一堆堆乌云,像青色的火焰,在无底的大海上燃烧。大海抓住闪电的箭光,把它们熄灭在自己的深渊里。这些闪电的影子,活像一条条火蛇,在大海里蜿蜒游动,一晃就消失了。
    
    ——暴风雨!暴风雨就要来啦!
    
    这是勇敢的海燕,在怒吼的大海上,在闪电中间,高傲地飞翔;这是胜利的预言家在叫喊:
    
    ——让暴风雨来得更猛烈些吧!
    
\end{normalsize}


\newpage

\textbf{注释}:

\vspace{-1em}

\begin{itemize}
    \setlength\itemsep{-0.2em}
    \item 〔深渊〕非常深的水。
    \item 〔呻吟〕因痛苦而发出声音。
    \item 〔胆怯〕胆小害怕。
    \item 〔蜿蜒〕蛇屈折爬行的样子。
\end{itemize}

\chapter{春}

\begin{normalsize}
    
    盼望着,盼望着,东风来了,春天的脚步近了。
    
    一切都像刚睡醒的样子,欣欣然张开了眼。山朗润起来了,水涨起来了,太阳的脸红起来了。
    
    小草偷偷地从土里钻出来,嫩嫩的,绿绿的。园子里,田野里,瞧去,一大片一大片满是的。坐着,躺着,打两个滚,踢几脚球,赛几趟跑,捉几回迷藏。风轻悄悄的,草绵软软的。
    
    桃树、杏树、梨树,你不让我,我不让你,都开满了花赶趟儿。红的像火,粉的像霞,白的像雪。花里带着甜味儿;闭了眼,树上仿佛已经满是桃儿、杏儿、梨儿。花下成千成百的蜜蜂嗡嗡地闹着,大小的蝴蝶飞来飞去。野花遍地是:杂样儿,有名字的,没名字的,散在草丛里,像眼睛,像星星,还眨呀眨的。
    
    “吹面不寒杨柳风”,不错的,像母亲的手抚摸着你。风里带来些新翻的泥土的气息,混着青草味儿,还有各种花的香,都在微微润湿的空气里酝酿。鸟儿将窠巢安在繁花嫩叶当中,高兴起来了,呼朋引伴地卖弄清脆的喉咙,唱出宛转的曲子,与轻风流水应和着。牛背上牧童的短笛,这时候也成天在嘹亮地响。
    
    雨是最寻常的,一下就是三两天。可别恼。看,像牛毛,像花针,像细丝,密密地斜织着,人家屋顶上全笼着一层薄烟。树叶子却绿得发亮,小草也青得逼你的眼。傍晚时候,上灯了,一点点黄晕的光,烘托出一片安静而和平的夜。乡下去,小路上,石桥边,有撑起伞慢慢走着的人;还有地里工作的农夫,披着蓑,戴着笠的。他们的草屋,稀稀疏疏的,在雨里静默着。
    
    天上风筝渐渐多了,地上孩子也多了。城里乡下,家家户户,老老小小,他们也赶趟儿似的,一个个都出来了。舒活舒活筋骨,抖擞抖擞精神,各做各的一份事去。“一年之计在于春”,刚起头儿,有的是工夫,有的是希望。
    
    春天像刚落地\footnote{〔落地〕这里指婴儿出生。}的娃娃,从头到脚都是新的,他生长着。
    
    春天像小姑娘,花枝招展的,笑着,走着。
    
    春天像健壮的青年,有铁一般的胳膊和腰脚,他领着我们上前去。
    
\end{normalsize}


\newpage

\textbf{注释}:

\vspace{-1em}

\begin{itemize}
    \setlength\itemsep{-0.2em}
    \item 〔朗润〕明亮滋润。朗:明亮。润:滋润、润泽。
    \item 〔赶趟儿〕时间赶得上。这里指众多果树争先恐后地开花。
    \item 〔酝酿〕造酒的发酵过程。这里指各种气息在空气里,像发酵似的,越来越浓。
    \item 〔窠巢〕鸟兽昆虫的窝。
    \item 〔宛转〕形容声音抑扬动听。现在多写作“婉转”。
    \item 〔花针〕绣花用的细针。
    \item 〔黄晕〕昏黄,不明亮。
    \item 〔笠〕用竹篾或棕皮编制的遮阳挡雨的帽子。
    \item 〔花枝招展〕形容女子打扮得十分艳丽。这里比喻姿态优美。
    \item 〔抖擞〕振作(精神)。
\end{itemize}

\chapter{纪念白求恩}

\begin{normalsize}
    
    白求恩同志是加拿大共产党员,五十多岁了,为了帮助中国的抗日战争,受加拿大共产党和美国共产党的派遣,不远万里,来到中国。去年春上到延安,后来到五台山工作,不幸以身殉职。一个外国人,毫无利己的动机,把中国人民的解放事业当作他自己的事业,这是什么精神?这是国际主义的精神,这是共产主义的精神,每一个中国共产党员都要学习这种精神。列宁主义认为:资本主义国家的无产阶级要拥护殖民地半殖民地人民的解放斗争,殖民地半殖民地的无产阶级要拥护资本主义国家的无产阶级的解放斗争,世界革命才能胜利。白求恩同志是实践了这一条列宁主义路线的。我们中国共产党员也要实践这一条路线。我们要和一切资本主义国家的无产阶级联合起来,要和日本的、英国的、美国的、德国的、意大利的以及一切资本主义国家的无产阶级联合起来,才能打倒帝国主义,解放我们的民族和人民,解放世界的民族和人民。这就是我们的国际主义,这就是我们用以反对狭隘民族主义和狭隘爱国主义的国际主义。
    
    白求恩同志毫不利己专门利人的精神,表现在他对工作的极端的负责任,对同志对人民的极端的热忱。每个共产党员都要学习他。不少的人对工作不负责任,拈轻怕重,把重担子推给人家,自己挑轻的。一事当前,先替自己打算,然后再替别人打算。出了一点力就觉得了不起,喜欢自吹,生怕人家不知道。对同志对人民不是满腔热忱,而是冷冷清清,漠不关心,麻木不仁。这种人其实不是共产党员,至少不能算一个纯粹的共产党员。从前线回来的人说到白求恩,没有一个不佩服,没有一个不为他的精神所感动。晋察冀边区的军民,凡亲身受过白求恩医生的治疗和亲眼看过白求恩医生的工作的,无不为之感动。每一个共产党员,一定要学习白求恩同志的这种真正共产主义者的精神。
    
    白求恩同志是个医生,他以医疗为职业,对技术精益求精;在整个八路军医务系统中,他的医术是很高明的。这对于一班见异思迁的人,对于一班鄙薄技术工作以为不足道、以为无出路的人,也是一个极好的教训。
    
    我和白求恩同志只见过一面。后来他给我来过许多信。可是因为忙,仅回过他一封信,还不知他收到没有。对于他的死,我是很悲痛的。现在大家纪念他,可见他的精神感人之深。我们大家要学习他毫无自私自利之心的精神。从这点出发,就可以变为大有利于人民的人。一个人能力有大小,但只要有这点精神,就是一个高尚的人,一个纯粹的人,一个有道德的人,一个脱离了低级趣味的人,一个有益于人民的人。
    
\end{normalsize}


\newpage

\textbf{注释}:

\vspace{-1em}

\begin{itemize}
    \setlength\itemsep{-0.2em}
    \item 〔殉职〕因本职工作死亡。
    \item 〔实践〕实际去做。践:踩,踏。
    \item 〔热忱〕真诚的热情。忱:真挚的情意。
    \item 〔狭隘〕狭小。狭义。气量小。
    \item 〔派遣〕正式命令、委任下级去干某事,通常到别的地方。
    \item 〔精益求精〕已经很好了,还要求更好。
    \item 〔见异思迁〕看到别的地方就想搬迁过去。看到别的目标就改变主意。
    \item 〔鄙薄〕看不起,认为没有价值。
\end{itemize}

\chapter{猫}

\begin{normalsize}
    
    我家养了好几次猫,结局总是失踪或死亡。三妹是最喜欢猫的,她常在课后回家时,逗着猫玩。有一次,从隔壁要了一只新生的猫来。花白的毛,很活泼,如带着泥土的白雪球似的,常在廊前太阳光里滚来滚去。三妹常常取了一条红带,或一根绳子,在它面前来回的拖摇着,它便扑过来抢,又扑过去抢。我坐在藤椅上看着他们,可以微笑着消耗过一二小时的光阴,那时太阳光暖暖的照着,心上感着生命的新鲜与快乐。后来这只猫不知怎地忽然消瘦了,也不肯吃东西,光泽的毛也污涩了,终日躺在厅上的椅下,不肯出来。三妹想着种种方法逗它,它都不理会。我们都很替它忧郁。三妹特地买了一个很小很小的铜铃,用红绫带穿了,挂在它颈下,但只显得不相称,它只是毫无生意的,懒惰的,郁闷的躺着。有一天中午,我从编译所\footnote{〔编译所〕1921年4月,在茅盾介绍下,郑振铎进入商务印书馆编译所工作。}回来,三妹很难过的说道:“哥哥,小猫死了!”
    
    我心里也感着一缕的酸辛,可怜这两月来相伴的小侣!当时只得安慰着三妹道:“不要紧,我再向别处要一只来给你。”
    
    隔了几天,二妹从虹口\footnote{〔虹口〕上海市辖区。在黄浦江西北岸。}舅舅家里回来,她道,舅舅那里有三四只小猫,很有趣,正要送给人家。三妹便怂恿着她去拿一只来。礼拜天,母亲回来了,却带了一只浑身黄色的小猫同来。立刻三妹一部分的注意,又被这只黄色小猫吸引去了。这只小猫较第一只更有趣、更活泼。它在园中乱跑,又会爬树,有时蝴蝶安详地飞过时,它也会扑过去捉。它似乎太活泼了,一点也不怕生人,有时由树上跃到墙上,又跑到街上,在那里晒太阳。我们都很为它提心吊胆,一天都要“小猫呢?小猫呢?”查问得好几次。每次总要寻找了一回,方才寻到。三妹常指它笑着骂道:“你这小猫呀,要被乞丐捉去后才不会乱跑呢!”我回家吃中饭,总看见它坐在铁门外边,一见我进门,便飞也似地跑进去了。饭后的娱乐,是看它在爬树。隐身在阳光隐约里的绿叶中,好像在等待着要捉捕什么似的。把它抱了下来。一放手,又极快地爬上去了。过了二三个月,它会捉鼠了。有一次,居然捉到一只很肥大的鼠,自此,夜间便不再听见讨厌的吱吱的声了。
    
    某一日清晨,我起床来,披了衣下楼,没有看见小猫,在小园里找了一遍,也不见。心里便有些亡失的预警。
    
    “三妹,小猫呢?”
    
    她慌忙地跑下楼来,答道:“我刚才也寻了一遍,没有看见。”
    
    家里的人都忙乱的在寻找,但终于不见。
    
    李嫂道;“我一早起来开门,还见它在厅上。烧饭时,才不见了它。”
    
    大家都不高兴,好像亡失了一个亲爱的同伴,连向来不大喜欢它的张婶也说;“可惜,可惜,这样好的一只小猫。”
    
    我心里还有一线希望,以为它偶然跑到远处去,也许会认得归途的。
    
    午饭时,张婶诉说道:“刚才遇到隔壁周家的丫头,她说,早上看见我家的小猫在门外,被一个过路的人捉去了。”
    
    于是这个亡失证实了。三妹很不高兴的咕噜着道:“他们看见了,为什么不出来阻止?他们明晓得它是我家的!”
    
    我也怅然的,愤恨的,在诅骂着那个不知名的夺去我们所爱的东西的人。
    
    自此,我家好久不养猫。
    
    冬天的早晨,门口蜷伏着一只很可怜的小猫。毛色是花白,但并不好看,又很瘦。它伏着不去。我们如不取来留养,至少也要为冬寒与饥饿所杀。张婶把它拾了进来,每天给它饭吃。但大家都不大喜欢它,它不活泼,也不像别的小猫之喜欢顽游,好像是具着天生的忧郁性似的,连三妹那样爱猫的,对于它也不加注意。如此的,过了几个月,它在我家仍是一只若有若无的动物。它渐渐的肥胖了,但仍不活泼。大家在廊前晒太阳闲谈着时,它也常来蜷伏在母亲或三妹的足下。三妹有时也逗着它玩,但没有对于前几只小猫那样感兴趣。有一天,它因夜里冷,钻到火炉底下去,毛被烧脱好几块,更觉得难看了。
    
    春天来了,它成了一只壮猫了,却仍不改它的忧郁性,也不去捉鼠,终日懒惰的伏着,吃得胖胖的。
    
    这时,妻买了一对黄色的芙蓉鸟来,挂在廊前,叫得很好听。妻常常叮嘱着张婶换水,加鸟粮,洗刷笼子。那只花白猫对于这一对黄鸟,似乎也特别注意,常常跳在桌上,对鸟笼凝望着。
    
    妻道:“张婶,留心猫,它会吃鸟呢。”
    
    张婶便跑来把猫捉了去。隔一会,它又跳上桌子对鸟笼凝望着了。
    
    一天,我下楼时,听见张婶在叫道:“鸟死了一只,一条腿被咬去了,笼扳上都是血。是什么东西把它咬死的?”
    
    我匆匆跑下去看,果然一只鸟是死了,羽毛松散着,好像它曾与它的敌人挣扎了许久。
    
    我很愤怒,叫道:“一定是猫,一定是猫!”于是立刻便去找它。
    
    妻听见了,也匆匆地跑下来,看了死鸟,很难过,便道:“不是这猫咬死的还有谁?它常常对鸟笼望着,我早就叫张婶要小心了。张婶!你为什么不小心?”
    
    张婶默默无言,不能有什么话来辩护。
    
    于是猫的罪状证实了。大家都去找这可厌的猫,想给它以一顿惩戒。找了半天,却没找到。我以为它真是“畏罪潜逃”了。
    
    三妹在楼上叫道:“猫在这里了。”
    
    它躺在露台板上晒太阳,态度很安详,嘴里好像还在吃着什么。我想,它一定是在吃着这可怜的鸟的腿了,一时怒气冲天,拿起楼门旁倚着的一根木棒,追过去打了一下。它很悲楚地叫了一声“咪呜!”便逃到屋瓦上了。
    
    我心里还愤愤的,以为惩戒得还没有快意。
    
    隔了几天,李嫂在楼下叫道:“猫,猫!又来吃鸟了。”同时我看见一只黑猫飞快的逃过露台,嘴里衔着一只黄鸟。我开始觉得我是错了!
    
    我心里十分的难过,真的,我的良心受伤了,我没有判断明白,便妄下断语,冤苦了一只不能说话辩诉的动物。想到它的无抵抗的逃避,益使我感到我的暴怒,我的虐待,都是针,刺我的良心的针!
    
    我很想补救我的过失,但它是不能说话的,我将怎样的对它表白我的误解呢?
    
    两个月后,我们的猫忽然死在邻家的屋脊上。我对于它的亡失,比以前的两只猫的亡失,更难过得多。
    
    我永无改正我的过失的机会了!
    
    自此,我家永不养猫。
    
\end{normalsize}


\newpage

\textbf{注释}:

\vspace{-1em}

\begin{itemize}
    \setlength\itemsep{-0.2em}
    \item 〔涩〕不顺滑。
    \item 〔怂恿〕从旁劝说,使想做。
    \item 〔提心吊胆〕不放心,心里不安。
    \item 〔乞丐〕靠要饭要钱过活的人。
    \item 〔倚〕斜靠。
    \item 〔虐待〕用狠毒残忍的手段对待人。
\end{itemize}

\chapter{阿长与山海经}

\begin{normalsize}
    
    长妈妈,已经说过,是一个一向带领着我的女工,说得阔气一点,就是我的保姆。我的母亲和许多别的人都这样称呼她,似乎略带些客气的意思。只有祖母叫她阿长。我平时叫她“阿妈”,连“长”字也不带;但到憎恶她的时候,——例如知道了谋死我那隐鼠\footnote{〔隐鼠〕鼹鼠的别称。}的却是她的时候,就叫她阿长。
    
    我们那里没有姓长的;她生得黄胖而矮,“长”也不是形容词。又不是她的名字,记得她自己说过,她的名字是叫作什么姑娘的。什么姑娘,我现在已经忘却了,总之不是长姑娘;也终于不知道她姓什么。记得她也曾告诉过我这个名称的来历:先前的先前,我家有一个女工,身材生得很高大,这就是真阿长。后来她回去了,我那什么姑娘才来补她的缺,然而大家因为叫惯了,没有再改口,于是她从此也就成为长妈妈了。
    
    虽然背地里说人长短不是好事情,但倘使要我说句真心话,我可只得说:我实在不大佩服她。最讨厌的是常喜欢切切察察,向人们低声絮说些什么事。还竖起第二个手指,在空中上下摇动,或者点着对手或自己的鼻尖。我的家里一有些小风波,不知怎的我总疑心和这“切切察察”有些关系。又不许我走动,拔一株草,翻一块石头,就说我顽皮,要告诉我的母亲去了。一到夏天,睡觉时她又伸开两脚两手,在床中间摆成一个“大”字,挤得我没有余地翻身,久睡在一角的席子上,又已经烤得那么热。推她呢,不动;叫她呢,也不闻。
    
    “长妈妈生得那么胖,一定很怕热罢?晚上的睡相,怕不见得很好罢?……”
    
    母亲听到我多回诉苦之后,曾经这样地问过她。我也知道这意思是要她多给我一些空席。她不开口。但到夜里,我热得醒来的时候,却仍然看见满床摆着一个“大”字,一条臂膊还搁在我的颈子上。我想,这实在是无法可想了。
    
    但是她懂得许多规矩;这些规矩,也大概是我所不耐烦的。一年中最高兴的时节,自然要数除夕了。辞岁\footnote{〔辞岁〕新年开始。旧年最后一夜叫做“除夕”,度过后迎来新年,称为“辞旧岁,迎新春”。}之后,从长辈得到压岁钱\footnote{〔压岁钱〕过年的习俗。长辈要给小辈压岁钱,保佑平安过年。},红纸包着,放在枕边,只要过一宵,便可以随意使用。睡在枕上,看着红包,想到明天买来的小鼓、刀枪、泥人、糖菩萨\footnote{〔糖菩萨〕一种小吃。把糖用模具做成菩萨样子。}……。然而她进来,又将一个福橘\footnote{〔福橘〕过年的习俗。橘音近“吉”,因此过年吃橘子,称为“福橘”。}放在床头了。
    
    “哥儿,你牢牢记住!”她极其郑重地说。“明天是正月初一,清早一睁开眼睛,第一句话就得对我说:‘阿妈,恭喜恭喜!’记得么?你要记着,这是一年的运气的事情。不许说别的话!说过之后,还得吃一点福橘。”她又拿起那橘子来在我的眼前摇了两摇,“那么,一年到头,顺顺流流……。”
    
    梦里也记得元旦的,第二天醒得特别早,一醒,就要坐起来。她却立刻伸出臂膊,一把将我按住。我惊异地看她时,只见她惶急地看着我。
    
    她又有所要求似的,摇着我的肩。我忽而记得了——
    
    “阿妈,恭喜……。”
    
    恭喜恭喜!大家恭喜!真聪明!恭喜恭喜!”她于是十分欢喜似的,笑将起来,同时将一点冰冷的东西,塞在我的嘴里。我大吃一惊之后,也就忽而记得,这就是所谓福橘,元旦辟头\footnote{〔辟头〕开头。}的磨难,总算已经受完,可以下床玩耍去了。
    
    她教给我的道理还很多,例如说人死了,不该说死掉,必须说“老掉了”;死了人,生了孩子的屋子里,不应该走进去;饭粒落在地上,必须拣起来,最好是吃下去;晒裤子用的竹竿底下,是万不可钻过去的……。此外,现在大抵忘却了,只有元旦的古怪仪式记得最清楚。总之:都是些烦琐\footnote{〔烦琐〕繁琐。}之至,至今想起来还觉得非常麻烦的事情。
    
    然而我有一时也对她发生过空前的敬意。她常常对我讲“长毛”。她之所谓“长毛”者,不但洪秀全\footnote{〔洪秀全〕清晚期太平天国运动的发起者和领袖。}军,似乎连后来一切土匪强盗都在内,但除却革命党\footnote{〔革命党〕清末以兴中会为首、意图推翻帝制的革命团体。},因为那时还没有。她说得长毛非常可怕,他们的话就听不懂。她说先前长毛进城的时候,我家全都逃到海边去了,只留一个门房\footnote{〔门房〕大门口内侧的小房,有专人看守。也指看守门房的人。}和年老的煮饭老妈子看家。后来长毛果然进门来了,那老妈子便叫他们“大王”,——据说对长毛就应该这样叫,——诉说自己的饥饿。长毛笑道:“那么,这东西就给你吃了罢!”将一个圆圆的东西掷了过来,还带着一条小辫子,正是那门房的头。煮饭老妈子从此就骇破了胆,后来一提起,还是立刻面如土色,自己轻轻地拍着胸埔道:“阿呀,骇死我了,骇死我了……。”
    
    我那时似乎倒并不怕,因为我觉得这些事和我毫不相干的,我不是一个门房。但她大概也即觉到了,说道:“象你似的小孩子,长毛也要掳的,掳去做小长毛。还有好看的姑娘,也要掳。”
    
    “那么,你是不要紧的。”我以为她一定最安全了,既不做门房,又不是小孩子,也生得不好看,况且颈子上还有许多炙疮疤。
    
    “哪里的话?!”她严肃地说。“我们就没有用处?我们也要被掳去。城外有兵来攻的时候,长毛就叫我们脱下裤子,一排一排地站在城墙上,外面的大炮就放不出来;再要放,就炸了!”
    
    这实在是出于我意想之外的,不能不惊异。我一向只以为她满肚子是麻烦的礼节罢了,却不料她还有这样伟大的神力。从此对于她就有了特别的敬意,似乎实在深不可测;夜间的伸开手脚,占领全床,那当然是情有可原的了,倒应该我退让。
    
    这种敬意,虽然也逐渐淡薄起来,但完全消失,大概是在知道她谋害了我的隐鼠之后。那时就极严重地诘问,而且当面叫她阿长。我想我又不真做小长毛,不去攻城,也不放炮,更不怕炮炸,我惧惮她什么呢!
    
    但当我哀悼隐鼠,给它复仇的时候,一面又在渴慕着绘图的《山海经》\footnote{〔《山海经》〕古代流传的地理志,传说是大禹所作。有各种神怪奇物图解。}了。这渴慕是从一个远房的叔祖\footnote{〔叔祖〕祖父的弟弟。}惹起来的。他是一个胖胖的,和蔼的老人,爱种一点花木,如珠兰、茉莉之类,还有极其少见的,据说从北边带回去的马缨花\footnote{〔马缨花〕杜鹃花品种,因花朵状似马缨而得名。产于云南、贵州、广西等地的山区。}。他的太太却正相反,什么也莫名其妙,曾将晒衣服的竹竿搁在珠兰的枝条上,枝折了,还要愤愤地咒骂道:“死尸!”这老人是个寂寞者,因为无人可谈,就很爱和孩子们往来,有时简直称我们为“小友”。在我们聚族而居的宅子里,只有他书多,而且特别。制艺和试帖诗,自然也是有的;但我却只在他的书斋里,看见过陆玑的《毛诗草木鸟兽虫鱼疏》\footnote{〔《毛诗草木鸟兽虫鱼疏》〕三国时期吴国的陆玑编著,对《毛诗》中的动植物进行考注说明。《毛诗》指战国末年毛亨、毛苌辑注的《诗经》。},还有许多名目很生的书籍。我那时最爱看的是《花镜》\footnote{〔《花镜》〕即《秘传花镜》,清代陈淏子所著。主要讲藤木花草的分类,以及栽培花卉、饲养禽鸟兽畜昆虫的方法。},上面有许多图。他说给我听,曾经有过一部绘图的《山海经》,画着人面的兽,九头的蛇,三脚的鸟,生着翅膀的人,没有头而以两乳当作眼睛的怪物,……可惜现在不知道放在哪里了。
    
    很愿意看看这样的图画,但不好意思力逼他去寻找,他是很疏懒的。问别人呢,谁也不肯真实地回答我。压岁钱还有几百文,买罢,又没有好机会。有书买的大街离我家远得很,我一年中只能在正月间去玩一趟,那时候,两家书店都紧紧地关着门。
    
    玩的时候倒是没有什么的,但一坐下,我就记得绘图的《山海经》。
    
    大概是太过于念念不忘了,连阿长也来问《山海经》是怎么一回事。这是我向来没有和她说过的,我知道她并非学者,说了也无益;但既然来问,也就都对她说了。
    
    过了十多天,或者一个月罢,我还记得,是她告假回家以后的四五天,她穿着新的蓝布衫回来了,一见面,就将一包书递给我,高兴地说道:——“哥儿,有画儿的‘三哼经’,我给你买来了!”
    
    我似乎遇着了一个霹雳,全体\footnote{〔全体〕全身。}都震悚起来;赶紧去接过来,打开纸包,是四本小小的书,略略一翻,人面的兽,九头的蛇,……果然都在内。
    
    又使我发生新的敬意了,别人不肯做,或不能做的事,她却能够做成功。她确有伟大的神力。谋害隐鼠的怨恨,从此完全消灭了。
    
    这四本书,乃是我最初得到,最为心爱的宝书。
    
    书的模样,到现在还在眼前。可是从还在眼前的模样来说,却是一部刻印都十分粗拙的本子。纸张很黄;图象也很坏,甚至于几乎全用直线凑合,连动物的眼睛也都是长方形的。但那是我最为心爱的宝书,看起来,确是人面的兽;九头的蛇;一脚的牛;袋子似的帝江\footnote{〔帝江〕《山海经》记载的天山的山神。赤红如火,六足四翼,没有面目,能歌舞。};没有头而“以乳为目,以脐为口”,还要“执干戚而舞”的刑天\footnote{〔刑天〕《山海经》记载的巨人,与帝争神,被断头,葬于常羊之山。}。
    
    此后我就更其搜集绘图的书,于是有了石印\footnote{〔石印〕石版油墨印刷技术。18世纪末出现,19世纪传入中国,比称为“木刻”的雕版印刷更方便更好,广受欢迎。}的《尔雅音图》\footnote{〔《尔雅音图》〕《尔雅》是我国最早的辞典。西晋郭璞为《尔雅》注音、作图,内有大量插图。近代出版的画谱由清代姚之麟绘画。}和《毛诗品物图考》\footnote{〔《毛诗品物图考》〕18世纪日本汉学家对《毛诗》中动植物的图释著作。由冈元凤纂辑,橘国雄绘画。},又有了《点石斋丛画》\footnote{〔《点石斋丛画》〕点石斋书局印刷的画册,汇集了数百首诗及明清画家的插画。点石斋,1876年由英国商人厄内斯特·美查(Ernest Major)在上海创办。}和《诗画舫》\footnote{〔《诗画舫》〕汇集明晚期画家的唐诗插画谱,收录近五百首诗。}。《山海经》也另买了一部石印的,每卷都有图赞,绿色的画,字是红的,比那木刻的精致得多了。这一部直到前年还在,木刻的却已经记不清是什么时候失掉了。
    
    我的保姆,长妈妈即阿长,辞了这人世,大概也有了三十年了罢。我终于不知道她的姓名,她的经历;仅知道有一个过继的儿子,她大约是青年守寡的孤孀。仁厚黑暗的地母\footnote{〔地母〕中国古代信仰中的大地之神,又称后土。《山海经》中也有记载。与玉皇大帝合称皇天后土。民间认为人死后灵魂归于后土。}呵,愿在你怀里永安她的魂灵!
    
\end{normalsize}


\newpage

\textbf{注释}:

\vspace{-1em}

\begin{itemize}
    \setlength\itemsep{-0.2em}
    \item 〔憎恶〕厌恶仇恨。
    \item 〔惶急〕从旁劝说,使想做。
    \item 〔骇〕惊吓。
    \item 〔菩萨〕佛教中指修行有成的大觉悟者。也指心地慈善的人。
    \item 〔霹雳〕又急又响的雷。
    \item 〔震悚〕震惊惶恐。
    \item 〔掷〕扔,抛。
    \item 〔掳〕抢走,抓走。
    \item 〔孀〕称呼丧夫的寡妇。
    \item 〔图赞〕写在画面或图页上的赞美诗文。
    \item 〔保姆〕帮忙带小孩的妇女。
\end{itemize}

\chapter{谁是最可爱的人}

\begin{normalsize}
    
    在朝鲜的每一天,我都被一些东西感动着;我的思想感情的潮水,在放纵奔流着;我想把一切东西都告诉给我祖国的朋友们。但我最急于告诉你们的,是我思想感情的一段重要经历,这就是:我越来越深刻地感觉到,谁是我们最可爱的人!
    
    谁是我们最可爱的人呢?我们的战士,我感到他们是最可爱的人。
    
    也许还有人心里隐隐约约地说:你说的就是那些“兵”吗?他们看来是很平凡、很简单的哩,既看不出他们有什么高深的知识,又看不出他们有什么丰富的感情。可是,我要说,这是由于他跟我们的战士接触太少,还没有了解我们的战士:他们的品质是那样的纯洁和高尚,他们的意志是那样的坚韧和刚强,他们的气质是那样的淳朴和谦逊,他们的胸怀是那样的美丽和宽广!
    
    让我还是来说一段故事吧。
    
    还是在二次战役\footnote{〔二次战役〕1950年11月至12月,云山战役之后,中国人民志愿军和朝鲜人民军在朝鲜北部发起的一次围歼战。}的时候,有一支志愿军的部队向敌后猛插,去切断军隅里\footnote{〔军隅里〕朝鲜平安南道西北部,清川江下游南岸。}敌人的逃路。当他们赶到书堂站时,逃敌也恰恰赶到那里,眼看就要从汽车路上开过去。这支部队的先头边就匆匆占领了汽车路边一个很低的光光的小山冈,阻住敌人。一场壮烈的搏斗就开始了。敌人为了逃命,用了32架飞机、十多辆坦克发起集团冲锋,向这个连的阵地汹涌卷来,整个山顶的土都被打翻了,汽油弹的火焰把这个阵地烧红了。但是,勇士们在这烟与火的山冈上,高喊着口号,一次又一次把敌人打死在阵地前面。敌人的死尸像谷个子\footnote{〔谷个子〕收割下来、一捆一捆的谷子。}似的在山前堆满了,血也把这山冈流红了。可是敌人还是要拼死争夺,好使自己的主力不致覆灭。这场激战整整持续了八个小时。最后,勇士们的了弹打光了。蜂拥上来的敌人占领了山头,把他们压到山脚。飞机掷下的汽油弹把他们的身上烧着了火。这时候,勇士们是仍然不会后退的呀,他们把枪一摔,向敌人扑去,身上帽子上呼呼地冒着火苗,把敌人抱住,让身上的火,也把占领阵地的敌人烧死。……据这个营的营长告诉我,战后,这个连的阵地上,枪支完全摔碎了,机枪零件扔得满山都是。烈士们的遗体,保留着各种各样的姿势,。有抱住敌人腰的,有抱住敌人头的,有掐住敌人脖子把敌人摁倒在地上的,和敌人倒在一起,烧在一起。有一个战士,他手里还紧握着一个手榴弹,弹体上沾满脑浆;和他死在一起的美国鬼子,脑浆迸裂,涂了一地。另一个战士,嘴里还衔着敌人的半块耳朵。在掩埋烈士遗体的时候,由于他们两手扣着,把敌人抱得那样紧,分都分不开,以致把有些人的手指都掰断了。……这个连虽然伤亡很大,他们却打死了三百多个敌人,更重要的,他们使得我们部队的主力赶上来,聚歼了敌人。
    
    这就是朝鲜战场上一次最壮烈的战头——松骨峰战斗\footnote{〔松骨峰战斗〕现称松骨峰阻击战。参战部队为第38军112师335团一营三连,记集体特等功,授予“英雄部队”称号。},或者叫书堂站战斗。假若需要立纪念碑的话,让我把带火扑敌和用刺刀跟敌人拼死在一起的烈士们的名字记下吧。他们的名字是:王金传、邢玉堂、王文英、熊官全、王金侯、赵锡杰、隋金山、李玉安\footnote{〔李玉安〕后获救生还,1997年去世。}、丁振岱、张贵生、崔玉亮、李树国。还有一个战士,已经不可能知道他的名字了。让我们的烈士们千载万世永垂不朽吧!
    
    这个营的营长向我叙说了以上的情形,他的声调是缓慢的,他的感情是沉重的。他说在阵地上掩埋烈士的时候,他掉了眼泪。但是,他接着说:“你不要以为我是为他们伤心,不,我是为他们骄傲!我觉得我们的战士太伟大了,太可爱了,我不能不被他们感动得掉下泪来。”
    
    朋友,当你听到这段英雄事迹的时候,你的感想如何呢?你不觉得我们的战士是可爱的吗?你不以我们的祖国有着这样的英雄而自豪吗?
    
    我们的战士,对敌人这样狠,而对朝鲜人民却是那样的爱,充满国际主义的深厚热情。
    
    在汉江\footnote{〔汉江〕朝鲜半岛的河流,在汉朝设的四郡内。}北岸,我遇到一个青年战士,他今年才21岁,名叫马玉祥,是黑龙江青冈县人。他长着一副微黑透红的脸膛,高高的个儿,站在那儿,像秋天田野里一株红高粱那样淳朴可爱。不过因为他才从阵地上下来,显得稍微疲劳些,眼里的红丝还没有退净。他原来是炮兵连的。有一天夜里,他被一阵哭声惊醒了,出去一看,是一个朝鲜老妈妈坐在山冈上哭。原来她的房子被炸毁了,她在山里搭了个窝棚,窝棚又被炸毁了。回来,他马上到连部要求调到步兵连去,正好步兵连也需要人,就批准了他。我说:“在炮兵连不是一样打敌人吗?”“那,不同!”他说,“离敌人越近,越觉着打得过瘾,越觉着打得解恨!”
    
    在汉江南岸阻击敌人的日子里,有一天他从阵地上下来做饭。刚一进村,有几架敌机袭过来,打了一阵机关炮,接着就扔下了两个大燃烧弹。有几间房子着了火,火又盛,烟又大,使人不敢到跟前去。这时候,他听见烟火里有一个小孩子哇哇哭叫的声音。他马上穿过浓烟到近处一看,一个朝鲜的中年男人在院子里倒着,小孩子的哭声还在屋里。他走到屋门口,屋门口的火苗呼呼的,已经进不去人,门窗的纸已经烧着。小孩子的哭声随着那滚滚的浓烟传出来,听得真真切切。当他叙述到这里的时候,他说:“我能够不进去吗?我不能!我想,要在祖国遇见这种情形,我能够进去,那么,在朝鲜我就可以不进去吗?朝鲜人民和我们祖国的人民不是一样的吗?我就踹开门,扑了进去。呀!满屋子灰洞洞的烟,只能听见小孩哭,看不见人。我的眼也睁不开,脸烫得像刀割一般。我也不知道自己的身上着了火没有,我也不管它了,只是在地上乱摸。先摸着一个大人,拉了拉没拉动;又向大人的身后摸,才摸着小孩的腿,我就一把抓着抱起来,跳出门去。我一看小孩子,是挺好的一个小孩儿啊。他穿着小短褂儿,光着两条小腿儿,小腿儿乱蹬着,哇哇地哭。我心想:‘不管你哭不哭,不救活你家大人,谁养活你哩!’这时候,火更大了,屋子里的家具什物\footnote{〔什物〕杂物。}也烧着了。我就把他往地上一放,就又从那火门里钻了进去一拉那个大人,她哼了一声,我就使劲往外拉,见她又不动了。凑近一看,见她脸上流下来的血已经把她胸前的白衣染红了,眼睛已经闭上。我知道她不行了,才赶忙跳出门外,扑灭身上的火苗,抱起这个无父无母的孩子。……”
    
    朋友,当你听到这段事迹的时候,你的感觉又是如何呢?你不觉得我们的战士是最可爱的人吗?
    
    谁都知道,朝鲜战场是艰苦些。但战士们是怎样想的呢?有一次,我见到一个战士,在防空洞里,吃一口炒面\footnote{〔炒面〕用小麦粉等面粉炒制的干粮。},就一口雪。我问他:“你不觉得苦吗?”他把正送往嘴里的一勺雪收回来,笑了笑,说:“怎么能不觉得?我们革命军队又不是个怪物。不过我们的光荣也就在这里。”他把小勺儿干脆放下,兴奋地说,“就拿吃雪来说吧。我在这里吃雪,正是为了我们祖国的人民不吃雪。他们可以坐在挺豁亮的屋子里,泡上一壶茶,守住个小火炉子,想吃点什么就做点什么。”他又指了指狭小潮湿的防空洞说,“再比如蹲防空洞吧,多憋闷得慌哩,眼看着外面好好的太阳不能晒,光光的马路不能走。可是我在这里蹲防空洞,祖国的人民就可以不蹲防空洞啊,他们就可以在马路上不慌不忙地走啊。他们想骑车子也行,想走路也行,边遛达边说话也行。只要能使人民得到幸福,就是我们最大的幸福。所以,”他又把雪放到嘴里,像总结似的说“我在这里流点血不算什么,吃这点苦又算什么哩!”我又问:“你想不想祖国啊?”他笑起来:“谁不想哩,说不想,那是假话,可是我不愿意回去。如果回去,祖国的老百姓问,‘我们托付给你们的任务完成得怎么样啦?’我怎么答呢?我说‘朝鲜半边红,半边黑’,这算什么话呢?”我接着问:“你们经历了这么多危险,吃了这么多苦,你们对祖国对朝鲜有什么要求吗?”他想了一下,才回答我:“我们什么也不要。可是说心里话,——我这话可不一定恰当啊,我们是想要这么大的一个东西……”他笑着,用手指比个铜子儿大小,怕我不明白,“一块‘朝鲜解放纪念章’,我们愿意戴在胸脯上,回到咱们的祖国去。”
    
    朋友们,用不着多举例,你们已经可以了解我们的战士是怎样一种人,这种人有一种什么品质,他们的灵魂多么地美丽和宽广。他们是历史上、世界上第一流的战士,第一流的人!他们是世界上一切伟大人民的优秀之花!是我们值得骄傲,我们以我们的祖国有这样的英雄而骄傲,我们以生在这个英雄的国度而自豪!
    
    亲爱的朋友们,当你坐上早晨第一列电车驰向工厂的时候,当你扛上犁耙走向田野的时候,当你喝完一杯豆浆、提着书包走向学校的时候,当你坐到办公桌前开始这一天工作的时候,当你往孩子口里塞苹果的时候,当你和爱人一起散步的时候……朋友,你是否意识到你是在幸福之中呢?你也许很惊讶地说:“这是很平常的呀!”可是,从朝鲜归来的人,会知道你正生活在幸福中。请你意识到这是一种幸福吧,因为只有你意识到这一点,你才能更深刻了解我们的战士在朝鲜奋不顾身的原因。朋友!你是这么爱我们的祖国,爱我们的伟大领袖毛主席,你一定会深深地爱我们的战士,——他们确实是我们最可爱的人!
    
\end{normalsize}


\newpage

\textbf{注释}:

\vspace{-1em}

\begin{itemize}
    \setlength\itemsep{-0.2em}
    \item 〔淳朴〕忠厚朴实。
    \item 〔犁耙〕农具。犁用来耕地,耙用来平整土地。
    \item 〔豁亮〕宽敞明亮。
    \item 〔遛达〕闲逛,散步。
    \item 〔奋不顾身〕奋勇而不顾自身安危。
\end{itemize}

\chapter{致杨振宁}

\begin{normalsize}
    
    \noindent 振宁:
    
    \vspace{24pt}
    
    你这次回到祖国来,老师们和同学们见到你真是感到非常高兴。我这次从外地到北京来看见你,也确实感到非常高兴。在你离京之后,我也就要回到工作岗位上去。
    
    关于你要打听的事,我已向组织上了解,寒春\footnote{〔寒春〕原名Joan Hinton,美国核物理学家。1948年到中国与男友阳早(Sid Engst)结婚并定居中国。}确实没有参加过我国任何有关制造核武器\footnote{〔核武器〕利用原子核内的结合能的武器。当时指裂变弹(原子弹)和聚变弹(氢弹)。}的事,我特地写这封信告诉你。
    
    你这次回来能见到总理\footnote{〔总理〕指周恩来。},总理这样的高龄,能在百忙中用这么长的时间和你亲切地谈话,关怀地询问你各方面的情况,使我们在座的人都受到很大的教育,希望你能经常地想起这次亲切的接见。
    
    你这次回来能看见祖国各方面的革命和建设的情况,这真是难得的机会。希望你能了解到祖国的解放是来之不易的,是无数先烈流血牺牲换来的。毛主席说:“成千成万的先烈,为着人民的利益,在我们的前头英勇地牺牲了,让我们高举起他们的旗帜,踏着他们的血迹前进吧!”你谈到人生的意义应该明确,我想人生的意义就应该遵照毛主席所说的这句话去做。我的世界观改得也很差,许多私心杂念随时冒出来,像在工作中,顺利时就沾沾自喜,不顺利时就气馁,怕负责任等等。但我愿意引用毛主席这句话,与振宁共勉。希望你在国外时能经常想到我们的祖国。
    
    这次在北京见到你,时间虽然不长,但每天晚上回来后心情总是不很平静,从小在一起,各个时期的情景,总是涌上心头。这次送你走后,心里自然有些惜别之感。和你见面几次,心里总觉得缺点什么东西似的,细想起来心里总是有“友行千里心担忧”的感觉。因此心里总是盼望着“但愿人长久,千里共同途”。
    
    夜深了,不多谈了。代我向你父母问安。祝两位老人家身体健康。祝你一路顺风。
    
    \hfill 稼先
    
    \hfill 8.13/71
    
    \vspace{36pt}
    
    \begin{flushright}
        
    \end{flushright}
    
    
    
\end{normalsize}


\newpage

\textbf{注释}:

\vspace{-1em}

\begin{itemize}
    \setlength\itemsep{-0.2em}
    \item 〔气馁〕失去信心和勇气,灰心丧气。
    \item 〔世界观〕对世界、社会的根本看法。
    \item 〔共勉〕相互鼓励、激励。
    \item 〔惜别〕舍不得离别。
\end{itemize}

\chapter{批评与自我批评}

\begin{normalsize}
    
    有无认真的自我批评,也是我们和其他政党互相区别的显着的标志之一。
    
    我们曾经说过,房子是应该经常打扫的,不打扫就会积满了灰尘;脸是应该经常洗的,不洗也就会灰尘满面。我们同志的思想,我们党的工作,也会沾染灰尘的,也应该打扫和洗涤。“流水不腐,户枢不蠹\footnote{〔流水不腐,户枢不蠹〕流动的水不会腐臭,经常转动的门轴不会被虫蛀。}”,是说它们在不停的运动中抵抗了微生物或其他生物的侵蚀。
    
    对于我们,经常地检讨工作,在检讨中推广民主作风,不惧怕批评和自我批评,实行“知无不言,言无不尽”“言者无罪,闻者足戒”“有则改之,无则加勉”这些中国人民的有益的格言,正是抵抗各种政治灰尘和政治微生物侵蚀我们同志的思想和我们党的肌体的唯一有效的方法。以“惩前毖后,治病救人”为宗旨的整风运动\footnote{〔整风运动〕指1941年至1945年的延安整风运动。}之所以发生了很大的效力,就是因为我们在这个运动中展开了正确的而不是歪曲的、认真的而不是敷衍的批评和自我批评。
    
    以中国最广大人民的最大利益为出发点的中国共产党人,相信自己的事业是完全合乎正义的,不惜牺牲自己个人的一切,随时准备拿出自己的生命去殉我们的事业,难道还有什么不适合人民需要的思想、观点、意见、办法,舍不得丢掉的吗?难道我们还欢迎任何政治的灰尘、政治的微生物来玷污我们的清洁的面貌和侵蚀我们的健全的肌体吗?无数革命先烈为了人民的利益牺牲了他们的生命,使我们每个活着的人想起他们就心里难过,难道我们还有什么个人利益不能牺牲,还有什么错误不能抛弃吗?
    
    我们很快就要在全国胜利了。这个胜利将冲破帝国主义的东方战线,具有伟大的国际意义。夺取这个胜利,已经是不要很久的时间和不要花费很大的气力了;巩固这个胜利,则是需要很久的时间和要花费很大的气力的事情。
    
    资产阶级怀疑我们的建设能力。帝国主义者估计我们终久会要向他们讨乞才能活下去。因为胜利,党内的骄傲情绪,以功臣自居的情绪,停顿起来不求进步的情绪,贪图享乐不愿再过艰苦生活的情绪,可能生长。因为胜利,人民感谢我们,资产阶级也会出来捧场。敌人的武力是不能征服我们的,这点已经得到证明了。资产阶级的捧场则可能征服我们队伍中的意志薄弱者。
    
    可能有这样一些共产党人,他们是不曾被拿枪的敌人征服过的,他们在这些敌人面前不愧英雄的称号;但是经不起人们用糖衣裹着的炮弹的攻击,他们在糖弹面前要打败仗。我们必须预防这种情况。
    
    夺取全国胜利,这只是万里长征走完了第一步。如果这一步也值得骄傲,那是比较渺小的,更值得骄傲的还在后头。在过了几十年之后来看中国人民民主革命的胜利,就会使人们感觉那好像只是一出长剧的一个短小的序幕。剧是必须从序幕开始的,但序幕还不是高潮。中国的革命是伟大的,但革命以后的路程更长,工作更伟大,更艰苦。
    
    这一点现在就必须向党内讲明白,务必使同志们继续地保持谦虚、谨慎、不骄、不躁的作风,务必使同志们继续地保持艰苦奋斗的作风。我们有批评和自我批评这个马克思列宁主义的武器。我们能够去掉不良作风,保持优良作风。我们能够学会我们原来不懂的东西。我们不但善于破坏一个旧世界,我们还将善于建设一个新世界。中国人民不但可以不要向帝国主义者讨乞也能活下去,而且还将活得比帝国主义国家要好些。
    
\end{normalsize}


\newpage

\textbf{注释}:

\vspace{-1em}

\begin{itemize}
    \setlength\itemsep{-0.2em}
    \item 〔惩前毖后〕纠正以前的过错,今后小心不重犯。惩:停止。毖:谨慎小心。
    \item 〔玷污〕弄脏,污损。
\end{itemize}

\chapter{反对自由主义}

\begin{normalsize}
    
    我们主张积极的思想斗争,因为它是达到党内和革命团体内的团结使之利于战斗的武器。每个共产党员和革命分子,应该拿起这个武器。
    
    但是自由主义取消思想斗争,主张无原则的和平,结果是腐朽庸俗的作风发生,使党和革命团体的某些组织和某些个人在政治上腐化起来。
    
    自由主义有各种表现。
    
    因为是熟人、同乡、同学、知心朋友、亲爱者、老同事、老部下,明知不对,也不同他们作原则上的争论,任其下去,求得和平和亲热。或者轻描淡写地说一顿,不作彻底解决,保持一团和气。结果是有害于团体,也有害于个人。这是第一种。
    
    不负责任的背后批评,不是积极地向组织建议。当面不说,背后乱说;开会不说,会后乱说。心目中没有集体生活的原则,只有自由放任。这是第二种。
    
    事不关己,高高挂起;明知不对,少说为佳;明哲保身,但求无过。这是第三种。
    
    命令不服从,个人意见第一。只要组织照顾,不要组织纪律。这是第四种。
    
    不是为了团结,为了进步,为了把事情弄好,向不正确的意见斗争和争论,而是个人攻击,闹意气,泄私愤,图报复。这是第五种。
    
    听了不正确的议论也不争辩,甚至听了反革命分子的话也不报告,泰然处之,行若无事。这是第六种。
    
    见群众不宣传,不鼓动,不演说,不调查,不询问,不关心其痛痒,漠然置之,忘记了自己是一个共产党员,把一个共产党员混同于一个普通的老百姓。这是第七种。
    
    见损害群众利益的行为不愤恨,不劝告,不制止,不解释,听之任之。这是第八种。
    
    办事不认真,无一定计划,无一定方向,敷衍了事,得过且过,做一天和尚撞一天钟。这是第九种。
    
    自以为对革命有功,摆老资格,大事做不来,小事又不做,工作随便,学习松懈。这是第十种。
    
    自己错了,也已经懂得,又不想改正,自己对自己采取自由主义。这是第十一种。
    
    还可以举出一些。主要的有这十一种。
    
    所有这些,都是自由主义的表现。
    
    革命的集体组织中的自由主义是十分有害的。它是一种腐蚀剂,使团结涣散,关系松懈,工作消极,意见分歧。它使革命队伍失掉严密的组织和纪律,政策不能贯彻到底,党的组织和党所领导的群众发生隔离。这是一种严重的恶劣倾向。
    
    自由主义的来源,在于小资产阶级的自私自利性,以个人利益放在第一位,革命利益放在第二位,因此产生思想上、政治上、组织上的自由主义。
    
    自由主义者以抽象的教条看待马克思主义的原则。他们赞成马克思主义,但是不准备实行之,或不准备完全实行之,不准备拿马克思主义代替自己的自由主义。这些人,马克思主义是有的,自由主义也是有的:说的是马克思主义,行的是自由主义;对人是马克思主义,对己是自由主义。两样货色齐备,各有各的用处。这是一部分人的思想方法。
    
    自由主义是机会主义的一种表现,是和马克思主义根本冲突的。它是消极的东西,客观上起着援助敌人的作用,因此敌人是欢迎我们内部保存自由主义的。自由主义的性质如此,革命队伍中不应该保留它的地位。
    
    我们要用马克思主义的积极精神,克服消极的自由主义。一个共产党员,应该是襟怀坦白,忠实,积极,以革命利益为第一生命,以个人利益服从革命利益;无论何时何地,坚持正确的原则,同一切不正确的思想和行为作不疲倦的斗争,用以巩固党的集体生活,巩固党和群众的联系;关心党和群众比关心个人为重,关心他人比关心自己为重。这样才算得一个共产党员。
    
    一切忠诚、坦白、积极、正直的共产党员团结起来,反对一部分人的自由主义的倾向,使他们改变到正确的方面来。这是思想战线的任务之一。
    
\end{normalsize}



\chapter{论雷峰塔的倒掉}

\begin{normalsize}
    
    听说,杭州西湖上的雷峰塔倒掉了,听说而已,我没有亲见。但我却见过未倒的雷峰塔,破破烂烂的映掩于湖光山色之间,落山的太阳照着这些四近的地方,就是“雷峰夕照”,西湖十景之一。“雷峰夕照”的真景我也见过,并不见佳,我以为。
    
    然而一切西湖胜迹的名目之中,我知道得最早的却是这雷峰塔。我的祖母曾经常常对我说,白蛇娘娘就被压在这塔底下!有个叫做许仙的人救了两条蛇,一青一白,后来白蛇便化作女人来报恩,嫁给许仙了;青蛇化作丫鬟,也跟着。一个和尚,法海禅师,得道的禅师,看见许仙脸上有妖气,——凡讨妖怪作老婆的人,脸上就有妖气的,但只有非凡的人才看得出——便将他藏在金山寺的法座后,白蛇娘娘来寻夫,于是就“水漫金山”。我的祖母讲起来还要有趣得多,大约是出于一部弹词叫作《义妖传》里的,但我没有看过这部书,所以也不知道“许仙”“法海”究竟是否这样写。总而言之,白蛇娘娘终于中了法海的计策,被装在一个小小的钵盂里了。钵盂埋在地里,上面还造起一座镇压的塔来,这就是雷峰塔。此后似乎事情还很多,如“白状元祭塔”之类,但我现在都忘记了。
    
    那时我惟一的希望,就在这雷峰塔的倒掉。后来我长大了,到杭州,看见这破破烂烂的塔,心里就不舒服。后来我看看书,说杭州人又叫这塔作“保叔塔”,其实应该写作“保俶塔”,是钱王的儿子造的。那么,里面当然没有白蛇娘娘了,然而我心里仍然不舒服,仍然希望他倒掉。
    
    现在,他居然倒掉了,则普天之下的人民,其欣喜为何如?
    
    这是有事实可证的。试到吴越的山间海滨,探听民意去。凡有田夫野老,蚕妇村氓,除了几个脑髓里有点贵恙的之外,可有谁不为白娘娘抱不平,不怪法海太多事的?
    
    和尚本应该只管自己念经。白蛇自迷许仙,许仙自娶妖怪,和别人有什么相干呢?他偏要放下经卷,横来招是搬非,大约是怀着嫉妒罢,——那简直是一定的。
    
    听说,后来玉皇大帝也就怪法海多事,以至荼毒生灵,想要拿办他了。他逃来逃去,终于逃在蟹壳里避祸,不敢再出来,到现在还如此。我对于玉皇大帝所作的事,腹诽的非常多,独于这一件却很满意,因为“水漫金山”一案,的确应该由法海负责;他实在办得很不错的。只可惜我那时没有打听这话的出处,或者不在《义妖传》中,却是民间的传说罢。
    
    秋高稻熟时节,吴越间所多的是螃蟹,煮到通红之后,无论取哪一只,揭开背壳来,里面就有黄,有膏;倘是雌的,就有石榴子一般鲜红的子。先将这些吃完,即一定露出一个圆锥形的薄膜,再用小刀小心地沿着锥底切下,取出,翻转,使里面向外,只要不破,便变成一个罗汉模样的东西,有头脸,身子,是坐着的,我们那里的小孩子都称他“蟹和尚”,就是躲在里面避难的法海。
    
    当初,白蛇娘娘压在塔底下,法海禅师躲在蟹壳里。现在却只有这位老禅师独自静坐了,非到螃蟹断种的那一天为止出不来。莫非他造塔的时候,竟没有想到塔是终究要倒的么?
    
    活该。
    
\end{normalsize}



\chapter{“友邦惊诧”论}

\begin{normalsize}
    
    只要略有知觉\footnote{〔知觉〕这里指了解情况。}的人就都知道:这回学生的请愿\footnote{〔学生的请愿〕指1931年12月间全国各地学生为反对蒋介石的不抵抗政策到南京请愿的事件。对于这次学生爱国行动,国民党政府于12月5日通令全国,禁止请愿;17日当各地学生联合向国民党中央党部请愿时,又命令军警逮捕和枪杀请愿学生,当场打死二十余人,打伤百余人;18日还电令各地军政当局紧急处置请愿事件。},是因为日本占据了辽吉\footnote{〔辽吉〕辽宁和吉林。},南京政府束手无策,单会去哀求国联\footnote{〔哀求国联〕“九一八”事变后,国民党政府多次向国联申诉,11月22日当日军进攻锦州时,又向国联提议划锦州为中立区,以中国军队退入关内为条件请求日军停止进攻;12月15日在日军继续进攻锦州时再度向国联申诉,请求它出面干涉,阻止日本帝国主义扩大侵华战争。},而国联却正和日本是一伙。读书呀,读书呀,不错,学生是应该读书的,但一面也要大人老爷们不至于葬送土地,这才能够安心读书。报上不是说过,东北大学\footnote{〔东北大学〕奉系军阀张作霖推动创办的一所大学,1923年在沈阳成立,1931年“九一八”事变后流亡。}逃散,冯庸大学\footnote{〔冯庸大学〕奉系军阀冯庸创办的一所大学,1927年在沈阳成立,1931年“九一八”事变后停办。}逃散,日本兵看见学生模样的就枪毙吗?放下书包来请愿,真是已经可怜之至。不道\footnote{〔不道〕不料,想不到。}国民党政府却在十二月十八日通电各地军政当局文里,又加上他们“捣毁机关,阻断交通,殴伤中委,拦劫汽车,横击路人及公务人员,私逮刑讯,社会秩序,悉被破坏”的罪名,而且指出结果,说是“友邦人士,莫名惊诧,长此以往,国将不国”了!
    
    好个“友邦人士”!日本帝国主义的兵队强占了辽吉,炮轰机关,他们不惊诧;阻断铁路,追炸客车,捕禁官吏,枪毙人民,他们不惊诧。中国国民党治下的连年内战,空前水灾,卖儿救穷,砍头示众,秘密杀戮,电刑逼供,他们也不惊诧。在学生的请愿中有一点纷扰,他们就惊诧了!
    
    好个国民党政府的“友邦人士”!是些什么东西!即使所举的罪状是真的罢,但这些事情,是无论那一个“友邦”也都有的,他们的维持他们的“秩序”的监狱,就撕掉了他们的“文明”的面具。摆什么“惊诧”的臭脸孔呢?
    
    可是“友邦人士”一惊诧,我们的国府就怕了,“长此以往,国将不国”了,好像失了东三省,党国倒愈像一个国,失了东三省谁也不响,党国倒愈像一个国,失了东三省只有几个学生上几篇“呈文”,党国倒愈像一个国,可以博得“友邦人士”的夸奖,永远“国”下去一样。
    
    几句电文,说得明白极了:怎样的党国,怎样的“友邦”。“友邦”要我们人民身受宰割,寂然无声,略有“越轨”,便加屠戮;党国是要我们遵从这“友邦人士”的希望,否则,他就要“通电各地军政当局”,“即予紧急处置,不得于事后借口无法劝阻,敷衍塞责”了!
    
    因为“友邦人士”是知道的:日兵“无法劝阻”,学生们怎会“无法劝阻”?每月一千八百万的军费,四百万的政费,作什么用的呀,“军政当局”呀?
    
    写此文后刚一天,就见二十一日《申报》登载南京专电\footnote{〔专电〕记者专门为本报社、电台发来的电讯。}云:“考试院部员张以宽,盛传前日为学生架去重伤。兹据张自述,当时因车夫误会,为群众引至中大\footnote{〔中大〕南京中央大学。},旋出校回寓,并无受伤之事。至行政院某秘书被拉到中大,亦当时出来,更无失踪之事。”而“教育消息”栏内,又记本埠一小部分学校赴京请愿学生死伤的确数,则云:“中公\footnote{〔中公〕中公,中国公学;复旦,复旦大学;复旦附中,复旦大学附属实验中学;东亚,东亚体育专科学校;上中,上海中学;文生氏,文生氏高等英文学校。这些都是当时上海的私立学校。}死二人,伤三十人,复旦伤二人,复旦附中伤十人,东亚失踪一人(系女性),上中失踪一人,伤三人,文生氏死一人,伤五人……”可见学生并未如国府通电所说,将“社会秩序,破坏无余”,而国府则不但依然能够镇压,而且依然能够诬陷,杀戮。“友邦人士”,从此可以不必“惊诧莫名”,只请放心来瓜分就是了。
    
\end{normalsize}


\newpage

\textbf{注释}:

\vspace{-1em}

\begin{itemize}
    \setlength\itemsep{-0.2em}
    \item 〔惊诧〕惊讶诧异。
    \item 〔敷衍〕表面上应付,其实并不认真对待。
    \item 〔兹〕现在。
    \item 〔旋〕不久后,经过很短时间。
    \item 〔本埠〕指上海。埠:码头,引申指与外国通商的城市。
\end{itemize}

\chapter{看云识天气}

\begin{normalsize}
    
    天上的云,真是姿态万千,变化无常。它们有的像羽毛,轻轻地飘在空中;有的像鱼鳞,一片片整整齐齐地排列着;有的像羊群,来来去去;有的像一床大棉被,严严实实地盖住了天空;还有的像峰峦,像河流,像雄狮,像奔马……它们有时把天空点缀得很美丽,有时又把天空笼罩得很阴森。刚才还是白云朵朵,阳光灿烂;一霎间却又是乌云密布,大雨倾盆。云就像是天气的“招牌”:天上挂什么云,就将出现什么样的天气。
    
    经验告诉我们:天空的薄云,往往是天气晴朗的象征;那些低而厚密的云层,常常是阴雨风雪的预兆。
    
    那最轻盈、站得最高的云,叫卷云。这种云很薄,阳光可以透过云层照到地面,房屋和树木的光与影依然很清晰。卷云丝丝缕缕地飘浮着,有时像一片白色的羽毛,有时像一块洁白的绫纱。如果卷云成群成行地排列在空中,好像微风吹过水面引起的鳞波,这就成了卷积云。卷云和卷积云都很高,那里水分少,它们一般不会带来雨雪。还有一种像棉花团似的白云,叫积云。它们常在两千米左右的天空,一朵朵分散着,映着灿烂的阳光,云块四周散发出金黄的光辉。积云都在上午出现,午后最多,傍晚渐渐消散。在晴天,我们还会偶见一种高积云。高积云是成群的扁球状的云块,排列很匀称,云块间露出碧蓝的天幕,远远望去,就像草原上雪白的羊群。卷云、卷积云、积云和高积云,都是很美丽的。
    
    当那连绵的雨雪将要来临的时候,卷云在聚集着,天空渐渐出现一层薄云,仿佛蒙上了白色的绸幕。这种云叫卷层云。卷层云慢慢地向前推进,天气就将转阴。接着,云层越来越低,越来越厚,隔了云看太阳或月亮,就像隔了一层毛玻璃,朦胧不清。这时卷层云已经改名换姓,该叫它高层云了。出现了高层云,往往在几个钟头内便要下雨或者下雪。最后,云压得更低,变得更厚,太阳和月亮都躲藏了起来,天空被暗灰色的云块密密层层地布满了。这种云叫雨层云。雨层云一形成,连绵不断的雨雪也就降临了。
    
    夏天,雷雨到来之前,在天空先会看到积云。积云如果迅速地向上凸起,形成高大的云山,群峰争奇,耸入天顶,就变成了积雨云。积雨云越长越高,云底慢慢变黑,云峰渐渐模糊,不一会,整座云山崩塌了,乌云弥漫了天空,顷刻间,雷声隆隆,电光闪闪,马上就会哗啦哗啦地下起暴雨,有时竟会带来冰雹或者龙卷风。
    
    我们还可以根据云上的光彩现象,推测天气的情况。在太阳和月亮的周围,有时会出现一种美丽的七彩光圈,里层是红色的,外层是紫色的。这种光圈叫做晕。日晕和月晕常常产生在卷层云上,卷层云后面的大片高层云和雨层云,是大风雨的征兆。所以有“日晕三更雨,月晕午时风”的说法。说明出现卷层云,并且伴有晕,天气就会变坏。另有一种比晕小的彩色光环,叫做“华”。颜色的排列是里紫外红,跟晕刚好相反。日华和月华大多产生在高积云的边缘部分。华环由小变大,天气趋向晴好。华环由大变小,天气可能转为了阴雨。夏天,雨过天晴,太阳对面的云幕上,常会挂上一条彩色的圆弧,这就是虹。人们常说:“东虹轰隆西虹雨。”意思是说,虹在东方,就有雷无雨;虹在西方,将有大雨。还有一种云彩常出现在清晨或傍晚。太阳照到天空,使云层变成红色,这种云彩叫做霞。朝霞在西,表明阴雨天气在向我们进袭;晚霞在东,表示最近几天里天气晴朗。所以有“朝霞不出门,晚霞行千里”的谚语。
    
    云,能够帮助我们识别阴晴风雨,预知天气变化,这对工农业生产有着重要的意义。我们要学会看云识天气,就要虚心向有经验的人学习,留心观察云的变化,在反复的观察中掌握规律。但是,天气变化异常复杂,看云识天气毕竟有一定的限度。要准确掌握天气变化的情况,还得依靠天气预报。
    
\end{normalsize}



\chapter{老山界}

\begin{normalsize}
    
    我们决定要爬一座三十里高\footnote{〔三十里高〕这里指上山的路程,不是海拔高度。}的瑶山\footnote{〔瑶山〕泛指瑶族生活地区的山。},地图上叫越城岭\footnote{〔越城岭〕五岭之一,位于广西湖南交界。五岭:也称南岭,在湖南与两广交界。南岭以南称为岭南。},土名叫老山界。
    
    下午才动身,沿着山沟向上走。前面不知道为什么走不动,等了好久才走了几步,又要停下来等。队伍挤得紧紧的,站累了,就在路旁坐下来,等前头喊着“走,走,走”,就站起来再走。满望可以多走一段,可是走不了几次又要停下来。天色晚了,肚子饿了,许多人烦得叫起来,骂起来。我们偷了个空儿,跑到前面去。地势渐渐更加陡起来。我们已经超过自己的纵队\footnote{〔自己的纵队〕指“红章”纵队。“红章”“红星”纵队是长征时中央机关工作人员编成的两个纵队。作者当时在“红章”纵队政治部宣传部工作。},跑到“红星”纵队的尾巴上,恰好在转弯地方发现路旁有一间房子,我们就进去歇一下。
    
    这是一家瑶民\footnote{〔瑶民〕对瑶族人的称呼。},住着母女二人;男人大概是因为听到过队伍,照着习惯,到什么地方去躲起来了。
    
    “大嫂,借你这里歇歇脚儿。”
    
    “请到里边坐。”她带着些惊惶的神情说。队伍还是极迟慢地向前行动。我们就跟瑶民攀谈起来。照我们一路上的经验,不论是谁,不论他们开始怎样怕我们,只要我们对他们说清楚了红军是什么,没有不变忧为喜,同我们十分亲热起来的。今天对瑶民,我们也要试一试。
    
    我们谈到红军,谈到苛捐杂税,谈到广西军阀禁止瑶民信仰自己的宗教,残杀瑶民,谈到她住在这里的生活情形。那女人哭起来了。
    
    她说她原来也有过地,但是汉人把他们从自己的地上赶跑了。现在住到这荒山上来,种人家的地,每年要缴特别重的租。她说:“广西的苛捐杂税对瑶民特别重,广西军阀特别欺侮瑶民。你们红军早些来就好了,我们就不会吃这样的苦了。”
    
    她问我们饿了没有。这一问正问中了我们的心事。她拿出仅有的一点米,放在房中间木头架成的一个灶上煮粥。她对我们道歉,说没有多的米,也没有大锅,要不就多煮些给部队吃。我们给她钱,她不要。好容易来了一个认识的同志,带来一袋米,够吃三天的粮食,虽然明知道前面粮食缺乏,我们还是把这整袋子米送给她。她非常欢喜地接受了。
    
    部队今天非夜里行军不可,她的房子和篱笆都是枯竹编成的,我们生怕有人拆下来当火把点,就写了几条标语,用米汤贴在外面显眼的地方,告知我们的部队不准拆篱笆当火把。我们问了瑶民,知道前面还有竹林,可以砍来作火把,就派人到前面竹林去准备。
    
    粥吃起来十分香甜,因为确是饿了。我们也拿碗盛给瑶民母女吃。打听前面的路程,知道前面有一个地方叫雷公岩,很陡,上山三十里,下山十五里,再前面才是塘坊边。我们现在还没到山脚下呢。
    
    自己的队伍来了,我们饶了些水给大家喝。一路前进,天黑了才到山脚,果然有许多竹林。
    
    满天都是星光,火把也亮起来了。从山脚向上望,只见火把排成许多“之”字形,一直连到天上,跟星光按起来,分不出是火把还是星星。达真是我生平没见过的奇观。
    
    大家都知道这座山是怎样地陡了,不由浑身紧张,前后呼喊起来,都想努一把力,好快些翻过山去。
    
    “不要掉队呀!”
    
    “不要落后做乌龟呀!”
    
    “我们顶着天啦!”
    
    大家听了,哈哈地笑起来。
    
    在“之”字拐的路上一步一步地上去。向上看,火把在头顶上一点点排到天空;向下看,简直是绝壁,火把照着人的脸,就在脚底下。
    
    走了半天,忽然前面又走不动了。传来的话说,前面又有一段路在峭壁上,马爬不上去。又等了一点多钟,传下命令来说,就在这里睡,明天一早登山。
    
    就在这里睡觉?怎么行呢?下去到竹林里睡是不可能的。但就在路上睡么?路只有一尺来宽,半夜里一个翻身不就骨碌\footnote{〔骨碌〕指翻滚。}下去了么?而且路上的石头又非常不平,睡一晚准会疼死人。
    
    但这是没有办法的,只得裹一条毯子,横着心躺下去。因为实在太疲倦,一会儿就酣然入梦了。
    
    半夜里,忽然醒来,才觉得寒气逼人,刺入肌骨,浑身打着颤。把毯子卷得更紧些把身子蜷起来,还是睡不着。天上闪烁的星星好象黑色幕上缀着的宝石,它跟我们这样地接近哪!黑的山峰象巨人一样矗立在面前。四围的山把这山谷包围得象一口井。上边和下边有几堆火没有熄;冻醒了的同志们围着火堆小声地谈着话。除此以外,就是寂静。耳朵里有不可捉摸的声响,极远的又是极近的,极洪大的又是极细切的,像春蚕在咀嚼桑叶,像野马在平原上奔驰,像山泉在呜咽,像波涛在澎湃。不知什么时候又睡着了。
    
    黎明的时候被人推醒,说是准备出发。山下有人送饭上来,不管三七二十一,抢了一碗就吃。
    
    又传下命令来,要队伍今天无论如何爬过这座山。因为山路很难走,一路上需要督促前进。我们几个人又停下来,立刻写标语,分配人到山下山上各段去喊口号,演说,帮助病员和运输员。忙了一会,再向前进。
    
    走了不多远,看见昨晚所说的峭壁上的路,也就是所谓雷公岩的,果然陡极了,几乎是垂直的石梯,只有一尺多宽;旁边就是悬崖,虽然不很深,但也够怕人的。崖下已经聚集了很多马匹,都是昨晚不能过去、要等今天全纵队过完了再过去的。有几匹曾经从崖上跌下来,脚骨都断了。
    
    很小心地过了这个石梯。上面的路虽然还是陡,但并不陡得那么厉害了。一路走,一路检查标语。我渐渐地掉了队,顺便做些鼓动工作。
    
    这很陡的山爬完了。我以为三十里的山就是那么一点;恰巧来了一个瑶民,同他谈谈,知道还差得远,还有二十多里很陡的山。
    
    昨天的晚饭,今天的早饭,都没吃饱。肚子很饿,气力不够,但是必须鼓着勇气前进。一路上,看见以前送上去的标语用完了,就一路写着标语贴。累得走不动的时候,索性在地上躺一会儿。
    
    快要到山顶,我已经落得很远了。许多运输员都走到前头去了,剩下来的是医务人员和掩护部队。医务人员真是辛苦,因为山陡,伤员病员都下了担架走,旁边需要有人搀扶着。医务人员中的女同志们英勇得很,她们还是处处在慰问和帮助伤员病员,一点也不知道疲倦。回头向来路望去,那些小山都成了“矮子”。机关枪声很密,大概是在我们昨天出发的地方,五、八军团\footnote{〔五、八军团〕指红军第五和第八军团。长征时这两个军团负责断后。}正跟敌人开火。远远地还听见敌人飞机的叹息,大概是在叹息自己的命运:为什么不到抗日的战线上去显显身手呢?
    
    到了山顶,已经是下午两点多钟。我忽然想起:将来要在这里立个纪念碑,写上某年某月某日,红军北上抗日,路过此处。我长长地吐了一口气,坐在山顶上休息一会。回头看队伍,还没有过山的只有不多的几个人了。我们完成了任务,把一个坚强的意志灌输到整个纵队每个人心中,饥饿,疲劳甚至受伤的痛苦都被这个意志克服了。难翻的老山界被我们这样笨重的队伍战胜了。
    
    下山十五里,也是很倾斜的。我们一口气儿跑下去,跑得真快。路上有几处景致很好,浓密的树林里,银子似的泉水流下山去,清得透底。在每条溪流的旁边,有很多战士们用脸盆、饭盒子、茶缸煮粥吃。我们虽然也很饿,但仍旧一气儿跑下山去,一直到宿营地。
    
    这回翻山使部队开始养成一种新的习惯:那就是用脸盆、饭盒子、茶缸煮饭吃,煮东西吃。这种习惯一直保持了很久。
    
    老山界是我们长征中所过的第一座难走的山。但是我们走过了金沙江、大渡河、雪山、草地以后,才觉得老山界的困难,比起这些地方来,还是小得很。
    
\end{normalsize}


\newpage

\textbf{注释}:

\vspace{-1em}

\begin{itemize}
    \setlength\itemsep{-0.2em}
    \item 〔满望〕十分希望。
    \item 〔攀谈〕搭话聊天。
    \item 〔苛捐杂税〕繁重的税。捐:泛指巧立名目变相收税。
    \item 〔酣然入梦〕陷入熟睡。
    \item 〔细切〕细密。
    \item 〔呜咽〕低声哭泣。
    \item 〔澎湃〕波涛猛烈产生的巨大响声。
    \item 〔咀嚼〕在嘴里反复细细咬碎
    \item 〔搀扶〕在旁挽臂扶助。
    \item 〔景致〕景色。
\end{itemize}

\chapter{土地的誓言}

\begin{normalsize}
    
    对于广大的关东\footnote{〔关东〕旧称东北三省,以位于山海关之东而得名。亦称“关外”}原野,我心里怀着炽痛的热爱。我无时无刻不听见她呼唤我的名字,我无时无刻不听见她召唤我回去。我有时把手放在我的胸膛上,我知道我的心还是跳动的,我的心还在喷涌着热血,因为我常常感到它在泛滥着一种热情。
    
    当我躺在土地上的时候,当我仰望天上的星星,手里握着一把泥土的时候,或者当我回想起儿时的往事的时候,我想起那参天碧绿的白桦林,标直漂亮的白桦树在原野上呻吟;我看见奔流似的马群,深夜嗥鸣的蒙古狗,我听见皮鞭滚落在山涧里的脆响;我想起红布似的高粱,金黄的豆粒,黑色的土地,红玉的脸庞,黑玉的眼睛,斑斓的山雕,奔驰的鹿群,带着松香气味的煤块,带着赤色的足金;我想起幽远的车铃,晴天里马儿戴着串铃在溜直的大道上跑着,狐仙姑\footnote{〔狐仙姑〕东北地区民间传说中的神仙。}深夜的谰语\footnote{〔谰语〕没有根据的话。},原野上怪诞的狂风……这时我听到故乡在召唤我,故乡有一种声音在召唤着我。她低低地呼唤着我的名字,声音是那样的急切,使我不得不回去。
    
    我总是被这种声音所缠绕,不管我走到哪里,即使我睡得很沉,或者在睡梦中突然惊醒的时候,我都会突然想到是我应该回去的时候了。我必须回去,我从来没想过离开她。这种声音是不可阻止的,是不能选择的。这种声音已经和我的心取得了永远的沟通。当我记起故乡的时候,我便能看见那大地的深层,在翻滚着一种红熟的浆液,这声音便是从那里来的。在那亘古的地层里,有着一股燃烧的洪流,像我的心喷涌着血液一样。这个我是知道的,我常常把手放在大地上,我会感到她在跳跃,和我的心的跳跃是一样的。它们从来没有停息,它们的热血一直在流,在热情的默契里它们彼此呼唤着,终有一天它们要汇合在一起。
    
    土地是我的母亲,我的每一寸皮肤,都有着土粒;我的手掌一接近土地,心就变得平静。我是土地的族系\footnote{〔族系〕家族的世系。有密切关联的同类。},我不能离开她。在故乡的土地上,我印下我无数的脚印。在那田垄里埋葬过我 的欢笑,在那稻棵上我捉过蚱蜢,在那沉重的镐头\footnote{〔镐头〕刨土用的工具。}上留着我的手印。我吃过我自己种的白菜。故乡的土壤是香的。在春天,东风吹起的时候,土壤的香气便在田野里飘扬。河流浅浅地流过,柳条像一阵烟雨似的窜出来,空气里都有一种欢喜的声音。原野到处有一种鸣叫,天空清亮透明,劳动的声音从这头响到那头。秋天,银线似的蛛丝在牛角上挂着,粮车拉粮回来,麻雀吃厌了,这里那里到处飞。稻禾的香气是强烈的,碾着新谷的场院辘辘地响着,多么美丽,多么丰饶……没有人能够忘记她。我必定为她而战斗到底。
    
    土地,原野,我的家乡,你必须被解放!你必须站立!夜夜我听见马蹄奔驰的声音,草原的儿子在黎明的天边呼唤。这时我起来,找寻天空中北方的大熊\footnote{〔大熊〕指大熊星座。},在它金色的光芒之下,乃是我的家乡。我向那边注视着,注视着,直到天边破晓。我永不能忘记,因为我答应过她,我要回到她的身边,我答应过我一定会回去。为了她,我愿付出一切。我必须看见一个更美丽的故乡出现在我的面前——或者我的坟前。而我将用我的泪水,洗去她一切的污秽和耻辱。
    
    \hfill “九一八”十周年写
    
\end{normalsize}


\newpage

\textbf{注释}:

\vspace{-1em}

\begin{itemize}
    \setlength\itemsep{-0.2em}
    \item 〔炽痛〕烧得疼痛。
    \item 〔标直〕笔直。
    \item 〔嗥鸣〕犬类叫。
    \item 〔斑斓〕色彩错杂灿烂。
    \item 〔亘古〕远古。
    \item 〔污秽〕脏东西。
\end{itemize}

\chapter{天上的街市}

\begin{normalsize}
    
    \begin{verse}[0.5\linewidth]
        远远的街灯明了 \\
        好像闪着无数的明星 \\
        天上的星星现了 \\
        好像点着无数的明灯
    \end{verse}
    
    
    \begin{verse}[0.5\linewidth]
        我想那缥缈的空中 \\
        定然有美丽繁华的街道 \\
        道旁橱窗里陈列的 \\
        定然是世间没有的珍宝
    \end{verse}
    
    
    \begin{verse}[0.5\linewidth]
        你看那浅浅的天河 \\
        定然是不甚宽广 \\
        那隔河相望的人儿 \\
        定能骑着牛儿来往
    \end{verse}
    
    
    \begin{verse}[0.5\linewidth]
        我想他们此刻 \\
        定然在天街闲游 \\
        不信,请看那朵流星 \\
        是他们提着灯笼在走
    \end{verse}
    
\end{normalsize}


\newpage

\textbf{注释}:

\vspace{-1em}

\begin{itemize}
    \setlength\itemsep{-0.2em}
    \item 〔缥缈〕隐隐约约,若有若无。遥不可及。
    \item 〔橱窗〕商店临街的玻璃窗,用来展览样品。
\end{itemize}

\chapter{未选择的路}

\begin{normalsize}
    
    \begin{verse}[0.5\linewidth]
        黄叶林里分出了两条路, \\
        可惜我不能同时去涉足。 \\
        在那路口我伫立无言, \\
        向其中一条极目望去, \\
        直到它消失在树丛深处。
    \end{verse}
    
    
    \begin{verse}[0.5\linewidth]
        但我选了另外一条路, \\
        路上十分幽寂,荒草萋萋, \\
        显得更加诱人、更加美丽。 \\
        其实两条路并无差别, \\
        都很少有行人的足迹。
    \end{verse}
    
    
    \begin{verse}[0.5\linewidth]
        虽然那天清晨落叶满地, \\
        虽然两条路都罕有人迹。 \\
        第一条路,改日再见吧! \\
        但我知道长路无尽头, \\
        一去之后重来谈何容易。
    \end{verse}
    
    
    \begin{verse}[0.5\linewidth]
        也许有一天,某处的我, \\
        将轻声叹息把往事回顾。 \\
        黄叶林里分出了两条路, \\
        我选了更少人走过的, \\
        而踏上了不同的旅途。
    \end{verse}
    
\end{normalsize}


\newpage

\textbf{注释}:

\vspace{-1em}

\begin{itemize}
    \setlength\itemsep{-0.2em}
    \item 〔伫立〕久立,长时间地站着。
    \item 〔极目〕尽力远望。
\end{itemize}

\chapter{驿路梨花}

\begin{normalsize}
    
    山,好大的山啊!起伏的青山一座挨一座,延伸到远方,消失在迷茫的暮色中。
    
    这是哀牢山\footnote{〔哀牢山〕山名,在云南省南部,元江和阿墨江的分水岭,云岭南延分支之一。}南段的最高处。这么陡峭的山,这么茂密的树林,走上一天,路上也难得遇见几个人。夕阳西下,我们有点着急了,今夜要是赶不到山那边的太阳寨,只有在这深山中露宿了。
    
    同行老余是在边境地区生活过多年的人。正走着,他突然指着前面叫了起来:“看,梨花!”
    
    白色梨花开满枝头,多么美丽的一片梨树林啊!
    
    老余说:“这里有梨树,前边就会有人家。”
    
    一弯新月升起了,我们借助淡淡的月光,在忽明忽暗的梨树林里走着。山间的夜风吹得人脸上凉凉的,梨花的白色花瓣轻轻飘落在我们身上。
    
    “快看,有人家了。”
    
    一座草顶、竹篾泥墙的小屋出现在梨树林边。屋里漆黑,没有灯也没有人声。这是什么人的房子呢?
    
    老余打着电筒走过去,发现门是从外扣着的。白水门板上用黑炭写着两个字:“请进!”
    
    我们推开门进去。火塘\footnote{〔火塘〕室内地上挖的小坑,四周垒上砖石,中间生火取暖。}里的灰是冷的,显然,好多天没人住过了。一张简陋的大竹床铺着厚厚的稻草。倚在墙边的大竹筒里装满了水,我尝了一口,水清凉可口。我们走累了,决定在这里过夜。
    
    老余用电筒在屋里上上下下扫射了一圈,又发现墙上写着几行粗大的字:“屋后边有干柴,梁上竹筒里有米,有盐巴,有辣子。”
    
    我们开始烧火做饭。温暖的火、喷香的米饭和滚热的洗脚水,把我们身上的疲劳、饥饿都撵走了。我们躺在软软的干草铺上,对小茅屋的主人有说不尽的感激。我问老余:“你猜这家主人是干什么的?”老余说:“可能是一位守山护林的老人。”
    
    正说着,门被推开了。一个须眉花白的瑶族老人站在门前,手里提着一杆明火枪,肩上打着一袋米。
    
    “主人”回来了。我和老余同时抓住老人的手,抢着说感谢的话;老人眼睛瞪得大大的,几次想说话插不上嘴。直到我们不作声了,老人才笑道:“我不是主人,也是过路人呢!”
    
    我们把老人请到火塘前坐下,看他也是又累又饿,赶紧给他端来了热水、热饭。老人笑了笑:“多谢,多谢,说了半天还得多谢你们。”
    
    看来他是个很有穿山走林经验的人。吃完饭,他燃起一袋旱烟笑着说:“我是给主人家送粮食来的。”
    
    “主人家是谁?”
    
    “不晓得。”
    
    “粮食交给谁呢?”
    
    “挂在屋梁上。”
    
    “老人家,你真会开玩笑。”
    
    他悠闲地吐着烟,说:“我不是开玩笑。”停了一会,又接着说:“我是红河\footnote{〔红河〕中南半岛大河,发源于云南省西部。}边上过山岩的瑶家\footnote{〔瑶家〕指瑶族人。},平常爱打猎。上个月,我追赶一群麂子\footnote{〔麂子〕一种小型鹿类动物,腿细而有力,善于跳跃。},在老林里东转西转迷失了方向,不知怎么插到这个山头来了。那时候,人走累了,干粮也吃完了,想找个寨子歇歇,偏偏这一带没有人家。我正失望的时候,突然看到了这片梨花林和这小屋,屋里有柴、有米、有水,就是没有主人。吃了用了人家的东西,不说清楚还行?我只好撕了片头巾上的红布、插了根羽毛在门上,告诉主人,有个瑶家人来打扰了,过几天再来道谢……”
    
    说到这里,他用手指了指门背后:“你们看,那东西还在呢!”
    
    一根白羽毛钉在红布上,红白相衬很好看。老人家说到这里,停了一会,又接着说下去:“我到处打听小茅屋的主人是哪个,好不容易才从一个赶马人那里知道个大概,原来对门山头上有个名叫梨花的哈尼\footnote{〔哈尼〕哈尼族,我国少数民族之一,主要居住在云南省红河哈尼族彝族自治州等几个州县。}小姑娘,她说这大山坡上,前不着村后不挨寨,她要用为人民服务的精神来帮助过路人。”
    
    我们这才明白,屋里的米、水、干柴,以及那充满了热情的“请进”二字,都是出自那哈尼小姑娘的手。多好的梨花啊!
    
    瑶族老人又说:“过路人受到照料,都很感激,也都尽力把用了的柴、米补上,好让后来人方便。我这次是专门送粮食来的。”
    
    这天夜里,我睡得十分香甜,梦中恍惚在那香气四溢的梨花林里漫步,还看见一个身穿着花衫的哈尼小姑娘在梨花丛中歌唱……
    
    第二天早上,我们没有立即上路,老人也没有离开,我们决定把小茅屋修葺一下,给屋顶加点草,把房前屋后的排水沟再挖深一些。一个哈尼小姑娘都能为群众着想,我们真应该向她学习。
    
    我们正在劳动,突然梨树丛中闪出了一群哈尼小姑娘。走在前边的约莫十四五岁,红润的脸上有两道弯弯的修长的眉毛和一对晶莹的大眼睛。我想:“她一定是梨花。”
    
    瑶族老人立即走到她们面前,深深弯下腰去,行了个大礼,吓得小姑娘们像小雀似的蹦开了,接着就哈哈大笑起来:“老爷爷,你给我们行这样大的礼,不怕折损我们吗?”老人严肃地说:“我感谢你们盖了这间小草房。”
    
    为头的那个小姑娘赶紧插手:“不要谢我们!不要谢我们!房子是解放军叔叔盖的。”
    
    接着,小姑娘向我们讲述了房子的来历。十多年前,有一队解放军路过这里,在树林里过夜,半夜淋了大雨。他们想,这里要有一间给过路人避风雨的小屋就好了,第二天早上就砍树割草盖起了房子。她姐姐恰好过这边山上来拾菌子\footnote{〔菌子〕指蘑菇。},好奇地问解放军叔叔:“你们要在这里长住?”解放军说:“不,我们是为了方便过路人。是雷锋同志教我们这样做的。”她姐姐很受感动。从那以后,常常趁砍柴、拾菌子、找草药的机会来照料这小茅屋。
    
    原来她还不是梨花。我问:“梨花呢?”
    
    “前几年出嫁到山那边了。”
    
    不用说,姐姐出嫁后,是小姑娘接过任务,常来照管这小茅屋。
    
    我望着这群充满朝气的哈尼小姑娘和那洁白的梨花,不由得想起了一句诗:“驿路梨花处处开\footnote{〔驿路梨花处处开〕出自宋代陆游《闻武均州报已复西京》。驿路:又叫驿道。古时传递公文的公路,沿途有换马或供休息的驿站。}”。
    
\end{normalsize}


\newpage

\textbf{注释}:

\vspace{-1em}

\begin{itemize}
    \setlength\itemsep{-0.2em}
    \item 〔竹篾〕劈成薄片的竹条。
    \item 〔寨子〕防守用的栅栏。有围栏的村子。
    \item 〔修葺〕修理(建筑物)。葺:修理,修建。
\end{itemize}

\chapter{假如生活欺骗了你}

\begin{normalsize}
    
    \begin{verse}[0.5\linewidth]
        假如生活欺骗了你, \\
        不要悲伤,不要心急! \\
        忧郁的日子里须要镇静: \\
        相信吧,快乐的日子将会来临。
    \end{verse}
    
    
    \begin{verse}[0.5\linewidth]
        心儿永远向往着未来; \\
        现在却常是忧郁: \\
        一切都是瞬息,一切都将会过去; \\
        而那过去了的,就会成为亲切的怀恋。
    \end{verse}
    
\end{normalsize}


\newpage

\textbf{注释}:

\vspace{-1em}

\begin{itemize}
    \setlength\itemsep{-0.2em}
    \item 〔瞬息〕极短的时间。
    \item 〔怀恋〕怀念热爱。
\end{itemize}

\chapter{繁星}

\begin{normalsize}
    
    我爱月夜,但我也爱星天。从前在家乡七、八月的夜晚在庭院里纳凉的时候,我最爱看天上密密麻麻的繁星。望着星天,我就会忘记一切,仿佛回到了母亲的怀里似的。
    
    三年前在南京我住的地方有一道后门,每晚我打开后门,便看见一个静寂的夜。下面是一片菜园,上面是星群密布的蓝天。星光在我们的肉眼里虽然微小,然而它使我们觉得光明无处不在。那时候我正在读一些关于天文学的书,也认得一些星星,好像它们就是我的朋友,它们常常在和我谈话一样。
    
    如今在海上,每晚和繁星相对,我把它们认得很熟了。我躺在舱面上,仰望天空。深蓝色的天空里悬着无数半明半昧的星。船在动,星也在动,它们是这样低,真是摇摇欲坠呢!渐渐地我的眼睛模糊了,我好像看见无数萤火虫在我的周围飞舞。海上的夜是柔和的,是静寂的,是梦幻的。我望着那许多认识的星,我仿佛看见它们在对我霎眼,我仿佛听见它们在小声说话。这时我忘记了一切。在星的怀抱中我微笑着,我沉睡着。我觉得自己是一个小孩子,现在睡在母亲的怀里了。
    
\end{normalsize}


\newpage

\textbf{注释}:

\vspace{-1em}

\begin{itemize}
    \setlength\itemsep{-0.2em}
    \item 〔纳凉〕为避热而在阴凉处歇息。
    \item 〔昧〕暗,不明。
\end{itemize}

\chapter{静寂的园子}

\begin{normalsize}
    
    没有听见房东家的狗的声音。现在园子里非常静。那棵不知名的五瓣的白色小花仍然寂寞地开着。阳光照在松枝和盆中的花树上,给那些绿叶涂上金黄色。天是晴朗的,我不用抬起眼睛就知道头上是晴空万里。
    
    忽然我听见洋铁\footnote{〔洋铁〕镀锡或镀锌的铁皮。}瓦沟上有铃子响声,抬起头,看见两只松鼠正从瓦上溜下来,这两只小生物在松枝上互相追逐取乐。它们的绒线球似的大尾巴,它们的可爱的小黑眼睛,它们颈项上的小铃子吸引了我的注意。我索性不转睛地望着窗外。但是它们跑了两三转,又从藤萝\footnote{〔藤萝〕紫藤萝。}架回到屋瓦上,一瞬间就消失了,依旧把这个静寂的园子留给我。
    
    我刚刚埋下头,又听见小鸟的叫声。我再看,桂树枝上立着一只青灰色的白头小鸟,昂起头得意地歌唱。屋顶的电灯线上,还有一对麻雀在吱吱喳喳地讲话。
    
    我不了解这样的语言。但是我在鸟声里听出了一种安闲的快乐。它们要告诉我的一定是它们的喜悦的感情。可惜我不能回答它们。我把手一挥,它们就飞走了。我的话不能使它们留住,它们留给我一个园子的静寂。不过我知道它们过一阵又会回来的。
    
    现在我觉得我是这个园子里唯一的生物了。我坐在书桌前俯下头写字,没有一点声音来打扰我。我正可以把整个心放在纸上。但是我渐渐地烦躁起来。这静寂像一只手慢慢地挨近我的咽喉。我感到呼吸不畅快了。这是不自然的静寂。这是一种灾祸的预兆,就像暴雨到来前那种沉闷静止的空气一样。
    
    我似乎在等待什么东西。我有一种不安定的感觉,我不能够静下心来。我一定是在等待什么东西。我在等待空袭警报;或者我在等待房东家的狗吠声,这就是说,预行警报已经解除,不会有空袭警报响起来,我用不着准备听见凄厉的汽笛声\footnote{〔汽笛声〕指空袭警报。}就锁门出去。近半月来晴天有警报差不多成了常例。
    
    可是我的等待并没有结果。小鸟回来后又走了;松鼠们也来过一次,但又追逐地跑上屋顶,我不知道它们消失在什么地方。从我看不见的正面楼房屋顶上送过来一阵的乌鸦叫。这些小生物不知道人间的事情,它们不会带给我什么信息。
    
    我写到上面的一段,空袭警报就响了。我的等待果然没有落空。这时我觉得空气在动了。我听见巷外大街上汽车的叫声。我又听见飞机的发动机声,这大概是民航机飞出去躲警报。有时我们的驱逐机\footnote{〔驱逐机〕歼击机。}也会在这种时候排队飞出,等着攻击敌机。我不能再写了,便拿了一本书锁上园门,匆匆地走到外面去。
    
    在城门口经过一阵可怕的拥挤后,我终于到了郊外。在那里耽搁了两个多钟头,和几个朋友在一起,还在草地上吃了他们带出去的午餐。警报解除后,我回来,打开锁,推开园门,迎面扑来的仍然是一个园子的静寂。
    
    我回到房间,回到书桌前面,打开玻璃窗,在继续执笔前还看看窗外。树上,地上,满个园子都是阳光。墙角一丛观音竹微微地在飘动它们的尖叶。一只大苍蝇带着嗡嗡声从开着的窗飞进房来,在我的头上盘旋。一两只乌鸦在我看不见的地方叫。一只黄色小蝴蝶在白色小花间飞舞。忽然一阵奇怪的声音在对面屋瓦上响起来,又是那两只松鼠从高墙沿着洋铁滴水管溜下来。它们跑到那个支持松树的木架上,又跑到架子脚边有假山的水池的石栏杆下,在那里追逐了一回,又沿着木架跑上松枝,隐在松叶后面了。松叶动起来,桂树的小枝也动了,一只绿色小鸟刚刚歇在那上面。
    
    狗的声音还是听不见。我向右侧着身子去看那条没有阳光的窄小过道。房东家的小门紧紧地闭着。这些时候那里就没有一点声音。大概这家人大清早就到城外躲警报去了,现在还不曾回来。他们回来恐怕在太阳落坡的时候。那条肥壮的黄狗一定也跟着他们“疏散”了,否则会有狗抓门的声音送进我的耳里来。
    
    我又坐在窗前写了这许多字。还是只有乌鸦和小鸟的叫声陪伴我。苍蝇的嗡嗡声早已寂灭了。现在在屋角又响起了老鼠啃东西的声音。都是响一回又静一回的,在这个受着轰炸威胁的城市里我感到了寂寞。
    
    然而像一把刀要划破万里晴空似的,嘹亮的机声突然响起来。这是我们自己的飞机。声音多么雄壮,它扫除了这个园子的静寂。我要放下笔到庭院中去看天空,看那些背负着金色阳光在蓝空里闪耀的灰色大蜻蜓。那是多么美丽的景象。
    
    \hfill 1940年10月11日在昆明
    
\end{normalsize}


\newpage

\textbf{注释}:

\vspace{-1em}

\begin{itemize}
    \setlength\itemsep{-0.2em}
    \item 〔耽搁〕停留。
\end{itemize}

\chapter{社戏}

\begin{normalsize}
    
    我们鲁镇的习惯,本来是凡有出嫁的女儿,倘自己还未当家,夏间便大抵回到母家去消夏\footnote{〔消夏〕过夏天。}。那时我的祖母虽然还康健,但母亲也已分担了些家务,所以夏期便不能多日的归省\footnote{〔归省〕回家探望(父母)。}了,只得在扫墓完毕之后,抽空去住几天,这时我便每年跟了我的母亲住在外祖母的家里。那地方叫平桥村,是一个离海边不远,极偏僻的,临河的小村庄;住户不满三十家,都种田,打鱼,只有一家很小的杂货店。但在我是乐土:因为我在这里不但得到优待,又可以免念“秩秩斯干幽幽南山\footnote{〔秩秩斯干幽幽南山〕《诗经·小雅·斯干》头两句。这里泛指难懂的古书。}”了。
    
    和我一同玩的是许多小朋友,因为有了远客,他们也都从父母那里得了减少工作的许可,伴我来游戏。在小村里,一家的客,几乎也就是公共的。我们年纪都相仿,但论起行辈\footnote{〔行辈〕辈分。}12来,却至少是叔子,有几个还是太公,因为他们合村都同姓,是本家。然而我们是朋友,即使偶尔吵闹起来,打了太公,一村的老老少少,也决没有一个会想出“犯上”这两个字来,而他们也百分之九十九不识字。
    
    我们每天的事情大概是掘蚯蚓,掘来穿在铜丝做的小钩上,伏在河沿上去钓虾。虾是水世界里的呆子,决不惮用了自己的两个钳捧着钩尖送到嘴里去的,所以不半天便可以钓到一大碗。这虾照例是归我吃的。其次便是一同去放牛,但或者因为高等动物了的缘故罢,黄牛水牛都欺生\footnote{〔欺生〕欺负新来的人。},敢于欺侮我,因此我也总不敢走近身,只好远远地跟着,站着。这时候,小朋友们便不再原谅我会读“秩秩斯干”,却全都嘲笑起来了。
    
    至于我在那里所第一盼望的,却在到赵庄去看戏。赵庄是离平桥村五里的较大的村庄;平桥村太小,自己演不起戏,每年总付给赵庄多少钱,算作合做的。当时我并不想到他们为什么年年要演戏。现在想,那或者是春赛\footnote{〔春赛〕春季酬谢神灵的祭礼。},是社戏\footnote{〔社戏〕农村一些地方春秋酬神祈福的戏,一般在庙台或野外搭台演出。在绍兴,社是一种区划名称,社戏就是每个社中每年所演的“年规戏”。}了。
    
    就在我十一二岁时候的这一年,这日期也看看等到了。不料这一年真可惜,在早上就叫不到船。平桥村只有一只早出晚归的航船\footnote{〔航船〕江浙一带定期往来城镇间的载客运货的木船。}是大船,决没有留用的道理。其余的都是小船,不合用;央人到邻村去问,也没有,早都给别人定下了。外祖母很气恼,怪家里的人不早定,絮叨起来。母亲便宽慰伊\footnote{〔伊〕她。早期白话文用法。},说我们鲁镇的戏比小村里的好得多,一年看几回,今天就算了。只有我急得要哭,母亲却竭力的嘱咐我,说万不能装模装样,怕又招外祖母生气,又不准和别人一同去,说是怕外祖母要担心。
    
    总之,是完了。到下午,我的朋友都去了,戏已经开场了,我似乎听到锣鼓的声音,而且知道他们在戏台下买豆浆喝。
    
    这一天我不钓虾,东西也少吃。母亲很为难,没有法子想。到晚饭时候,外祖母也终于觉察了,并且说我应当不高兴,他们太怠慢,是待客的礼数里从来没有的。吃饭之后,看过戏的少年们也都聚拢来了,高高兴兴的来讲戏。只有我不开口;他们都叹息而且表同情。忽然间,一个最聪明的双喜大悟似的提议了,他说,“大船?八叔的航船不是回来了么?”十几个别的少年也大悟,立刻撺掇起来,说可以坐了这航船和我一同去。我高兴了。然而外祖母又怕都是孩子,不可靠;母亲又说是若叫大人一同去,他们白天全有工作,要他熬夜,是不合情理的。在这迟疑之中,双喜可又看出底细来了,便又大声的说道,“我写包票\footnote{〔写包票〕打包票,担保。}!船又大;迅哥儿向来不乱跑;我们又都是识水性的!”
    
    诚然!这十多个少年,委实没有一个不会凫水的,而且两三个还是弄潮\footnote{〔弄潮〕在潮水里搏击嬉戏。}的好手。
    
    外祖母和母亲也相信,便不再驳回,都微笑了。我们立刻一哄的出了门。
    
    我的很重的心忽而轻松了,身体也似乎舒展到说不出的大。一出门,便望见月下的平桥内泊着一只白篷的航船,大家跳下船,双喜拔\footnote{〔拔〕拨。}前篙,阿发拔后篙,年幼的都陪我坐在舱中,较大的聚在船尾。母亲送出来吩咐“要小心”的时候,我们已经点开船,在桥石上一磕,退后几尺,即又上前出了桥。于是架起两支橹,一支两人,一里一换,有说笑的,有嚷的,夹着潺潺的船头激水的声音,在左右都是碧绿的豆麦田地的河流中,飞一般径向赵庄前进了。
    
    两岸的豆麦和河底的水草所发散出来的清香,夹杂在水气中扑面的吹来;月色便朦胧在这水气里。淡黑的起伏的连山,仿佛是踊跃的铁的兽脊似的,都远远地向船尾跑去了,但我却还以为船慢。他们换了四回手,渐望见依稀的赵庄,而且似乎听到歌吹\footnote{〔歌吹〕歌声和乐器声。}了,还有几点火,料想便是戏台,但或者也许是渔火。
    
    那声音大概是横笛,宛转,悠扬,使我的心也沉静,然而又自失\footnote{〔自失〕失去自我。出神而忘记了自身存在。}起来,觉得要和他弥散在含着豆麦蕴藻\footnote{〔蕴藻〕一种水草,也叫蕰藻。}之香的夜气里。
    
    那火接近了,果然是渔火;我才记得先前望见的也不是赵庄。那是正对船头的一丛松柏林,我去年也曾经去游玩过,还看见破的石马倒在地下,一个石羊蹲在草里呢。过了那林,船便弯进了汊港,于是赵庄便真在眼前了。
    
    最惹眼的是屹立在庄外临河的空地上的一座戏台,模糊在远处的月夜中,和空间几乎分不出界限,我疑心画上见过的仙境,就在这里出现了。这时船走得更快,不多时,在台上显出人物来,红红绿绿的动,近台的河里一望乌黑的是看戏的人家的船篷。
    
    “近台没有什么空了,我们远远的看罢。”阿发说。
    
    这时船慢了,不久就到,果然近不得台旁,大家只能下了篙,比那正对戏台的神棚\footnote{〔神棚〕为演戏搭的供奉神的棚子。}还要远。其实我们这白篷的航船,本也不愿意和乌篷的船在一处,而况并没有空地呢……
    
    在停船的匆忙中,看见台上有一个黑的长胡子的背上插着四张旗,捏着长枪,和一群赤膊的人正打仗。双喜说,那就是有名的铁头老生\footnote{〔老生〕戏曲角色行当,扮演中老年男性,多为正面人物。},能连翻八十四个筋斗,他日\footnote{〔他日〕另一天。}里亲自数过的。
    
    我们便都挤在船头上看打仗,但那铁头老生却又并不翻筋斗,只有几个赤膊的人翻,翻了一阵,都进去了,接着走出一个小旦\footnote{〔小旦〕戏曲角色行当,扮演年轻女子。}来,咿咿呀呀的唱。双喜说,“晚上看客少,铁头老生也懈了,谁肯显本领给白地\footnote{〔白地〕空地。}看呢?”我相信这话对,因为其时台下已经不很有人,乡下人为了明天的工作,熬不得夜,早都睡觉去了,疏疏朗朗的站着的不过是几十个本村和邻村的闲汉。乌篷船里的那些土财主的家眷固然在,然而他们也不在乎看戏,多半是专到戏台下来吃糕饼水果和瓜子的。所以简直可以算白地。
    
    然而我的意思却也并不在乎看翻筋斗。我最愿意看的是一个人蒙了白布,两手在头上捧着一支棒似的蛇头的蛇精,其次是套了黄布衣跳老虎。但是等了许多时都不见,小旦虽然进去了,立刻又出来了一个很老的小生\footnote{〔小生〕戏曲角色行当,扮演年轻男子。}。我有些疲倦了,托桂生买豆浆去。他去了一刻,回来说:“没有。卖豆浆的聋子也回去了。日里倒有,我还喝了两碗呢。现在去舀一瓢水来给你喝罢。”
    
    我不喝水,支撑着仍然看,也说不出见了些什么,只觉得戏子\footnote{〔戏子〕旧称职业戏曲演员,有轻蔑意味。}的脸都渐渐的有些稀奇\footnote{〔稀奇〕这里指怪异。}了,那五官渐不明显,似乎融成一片的再没有什么高低。年纪小的几个多打呵欠了,大的也各管自己谈话。忽而一个红衫的小丑被绑在台柱子上,给一个花白胡子的用马鞭打起来了,大家才又振作精神的笑着看。在这一夜里,我以为这实在要算是最好的一折\footnote{〔一折〕一场戏。}。
    
    然而老旦\footnote{〔老旦〕戏曲角色行当,扮演年老女子。}终于出台了。老旦本来是我所最怕的东西,尤其是怕他坐下了唱。这时候,看见大家也都很扫兴,才知道他们的意见是和我一致的。那老旦当初还只是踱来踱去的唱,后来竟在中间的一把交椅上坐下了。我很担心;双喜他们却就破口喃喃的骂。我忍耐的等着,许多工夫,只见那老旦将手一抬,我以为就要站起来了,不料他却又慢慢的放下在原地方,仍旧唱。全船里几个人不住的吁气,其余的也打起哈欠来。双喜终于熬不住了,说道,怕他会唱到天明还不完,还是我们走的好罢。大家立刻都赞成,和开船时候一样踊跃,三四人径奔船尾,拔了篙,点退几丈,回转船头,驾起橹,骂着老旦,又向那松柏林前进了。
    
    月还没有落,仿佛看戏也并不很久似的,而一离赵庄,月光又显得格外的皎洁。回望戏台在灯火光中,却又如初来未到时候一般,又缥缈得像一座仙山楼阁,满被红霞罩着了。吹到耳边来的又是横笛,很悠扬;我疑心老旦已经进去了,但也不好意思说再回去看。
    
    不多久,松柏林早在船后了,船行也并不慢,但周围的黑暗只是浓,可知已经到了深夜。他们一面议论着戏子,或骂,或笑,一面加紧的摇船。这一次船头的激水声更其响亮了,那航船,就像一条大白鱼背着一群孩子在浪花里蹿,连夜渔\footnote{〔夜渔〕夜间捕鱼。}的几个老渔父,也停了艇子看着喝彩起来。
    
    离平桥村还有一里模样,船行却慢了,摇船的都说很疲乏,因为太用力,而且许久没有东西吃。这回想出来的是桂生,说是罗汉豆\footnote{〔罗汉豆〕蚕豆的别称。}正旺相\footnote{〔旺相〕旺盛。},柴火又现成,我们可以偷一点来煮吃。大家都赞成,立刻近岸停了船;岸上的田里,乌油油的便都是结实的罗汉豆。
    
    “呵呵,阿发,这边是你家的,这边是老六一家的,我们偷哪一边的呢?”双喜先跳下去了,在岸上说。
    
    我们也都跳上岸。阿发一面跳,一面说道,“且慢,让我来看一看罢,”他于是往来的摸了一回,直起身来说道,“偷我们的罢,我们的大得多呢。”一声答应,大家便散开在阿发家的豆田里,各摘了一大捧,抛入船舱中。双喜以为再多偷,倘给阿发的娘知道是要哭骂的,于是各人便到六一公公的田里又各偷了一大捧。
    
    我们中间几个年长的仍然慢慢的摇着船,几个到后舱去生火,年幼的和我都剥豆。不久豆熟了,便任凭航船浮在水面上,都围起来用手撮着吃。吃完豆,又开船,一面洗器具,豆荚豆壳全抛在河水里,什么痕迹也没有了。双喜所虑的是用了八公公船上的盐和柴,这老头子很细心,一定要知道,会骂的。然而大家议论之后,归结是不怕。他如果骂,我们便要他归还去年在岸边拾去的一枝枯桕树\footnote{〔桕树〕即乌桕,也叫木油树。落叶乔木。},而且当面叫他“八癞子”。
    
    “都回来了!哪里会错。我原说过写包票的!”双喜在船头上忽而大声的说。
    
    我向船头一望,前面已经是平桥。桥脚上站着一个人,却是我的母亲,双喜便是对伊说着话。我走出前舱去,船也就进了平桥了,停了船,我们纷纷都上岸。母亲颇有些生气,说是过了三更了,怎么回来得这样迟,但也就高兴了,笑着邀大家去吃炒米。
    
    大家都说已经吃了点心,又渴睡\footnote{〔渴睡〕想睡觉,困倦。},不如及早睡的好,各自回去了。
    
    第二天,我向午才起来,并没有听到什么关系八公公盐柴事件的纠葛,下午仍然去钓虾。
    
    “双喜,你们这班小鬼,昨天偷了我的豆了罢?又不肯好好的摘,踏坏了不少。”我抬头看时,是六一公公棹着小船,卖了豆回来了,船肚里还有剩下的一堆豆。
    
    “是的。我们请客。我们当初还不要你的呢。你看,你把我的虾吓跑了!”双喜说。
    
    六一公公看见我,便停了楫,笑道,“请客?——这是应该的。”于是对我说,“迅哥儿,昨天的戏可好么?”
    
    我点一点头,说道,“好。”
    
    “豆可中吃呢?”
    
    我又点一点头,说道,“很好。”
    
    不料六一公公竟非常感激起来,将大拇指一翘,得意的说道,“这真是大市镇里出来的读过书的人才识货!我的豆种是粒粒挑选过的,乡下人不识好歹,还说我的豆比不上别人的呢。我今天也要送些给我们的姑奶奶\footnote{〔姑奶奶〕娘家称已经出嫁的女儿。这里指鲁迅母亲。}尝尝去……”他于是打着楫子过去了。
    
    待到母亲叫我回去吃晚饭的时候,桌上便有一大碗煮熟了的罗汉豆,就是六一公公送给母亲和我吃的。听说他还对母亲极口夸奖我,说“小小年纪便有见识,将来一定要中状元。姑奶奶,你的福气是可以写包票的了”。但我吃了豆,却并没有昨夜的豆那么好。
    
    真的,一直到现在,我实在再没有吃到那夜似的好豆,——也不再看到那夜似的好戏了。
    
    \hfill 一九二二年十月
    
\end{normalsize}


\newpage

\textbf{注释}:

\vspace{-1em}

\begin{itemize}
    \setlength\itemsep{-0.2em}
    \item 〔礼数〕礼节。
    \item 〔撺掇〕怂恿,从旁鼓动。
    \item 〔凫水〕游水。
    \item 〔渔火〕渔船上的灯火。
    \item 〔弥散〕扩散,满布。
    \item 〔汊港〕与河流连通的小河道。
    \item 〔家眷〕妻子和儿女。
    \item 〔交椅〕有靠背,椅脚交叉能折叠的椅子。
    \item 〔喃喃〕连续不断的低语声。
    \item 〔皎洁〕明亮洁白。
    \item 〔喝彩〕博彩时呼喝作势,希望中彩,后来表示大声叫好。
    \item 〔撮〕用两三个指头捉取。
    \item 〔向午〕临近中午。
    \item 〔纠葛〕纠缠的葛蔓。比喻纠缠不清的事情。
    \item 〔篙〕行船工具。用长竹竿撑到水底使船前进。也写作“槁”。
    \item 〔橹〕一种桨。中部定在船尾,摇动上柄,下片拨水使船行进。
    \item 〔棹〕一种桨。中部定在船边,摇动上柄,下片拨水使船行进。
    \item 〔楫〕楫、棹都是桨。短的叫楫,长的叫棹。
    \item 〔状元〕科举考试第一名。
\end{itemize}

\chapter{我的老师}

\begin{normalsize}
    
    最使我难忘的,是我小学时候的女教师蔡芸芝先生。
    
    回想起来,她那时有十八九岁。嘴角右边有榆钱\footnote{〔榆钱〕榆树的果实。形状像铜钱,因此叫榆钱。}大小一块黑痣。在我的记忆里,她是一个温柔和美丽的人。
    
    她从来不打骂我们。仅仅有一次,她的教鞭好像要落下来,我用石板\footnote{〔石板〕学习用具,可以用石笔在上面写字。}一迎,教鞭轻轻地敲在石板边上,大伙笑了,她也笑了。我用儿童的狡猾的眼光察觉,她爱我们,并没有存心要打的意思。
    
    在课外的时候,她教我们跳舞,我还记得她把我扮成女孩子表演跳舞的情景。
    
    在假日里,她把我们带到她的家里和朋友的家里。在她的朋友的园子里,她还让我们观察蜜蜂,也是在那时候,我认识了蜂王,并且平生第一次吃了蜂蜜。
    
    她爱诗,并且爱用歌唱的音调教我们读诗。直到现在我还记得她读诗的音调,还能背诵她教我们的诗:
    
    \begin{verse}[0.5\linewidth]
    
    圆天盖着大海,\\
    
    黑水托着孤舟,\\
    
    远看不见山,\\
    
    那天边只有云头,\\
    
    也看不见树,\\
    
    那水上只有海鸥……\footnote{〔圆天盖着大海……〕出自周太玄的《过印度洋》。这首诗1919年发表于《少年中国》。赵元任将它谱成歌曲,流行一时。}\\
    
    \end{verse}
    
    今天想来,她对我的接近文学和爱好文学,是有着多么有益的影响!
    
    像这样的教师,我们怎么会不喜欢她,怎么会不愿意和她亲近呢?我们见了她不由地就围上去。即使她写字的时候,我们也默默地看着她,连她握铅笔的姿势都急于模仿。
    
    有一件小事,我不知道还值不值得提它,但回想起来,在那时却占据过我的心灵。我父亲那时候在军阀部队里,好几年没有回来,我跟母亲非常牵挂他,不知道他的死活。我的母亲常常站在一张褪了色的神像面前焚起香来,把两个有象征记号的字条卷着埋在香炉里,然后磕了头,抽出一个来卜问\footnote{〔卜问〕向神灵询问未来或未知的事。}吉凶。我虽不像母亲那样,也略略懂了些事。可是在孩子群中,我的那些小“反对派”们,常常在我的耳边猛喊:“哎哟哟,你爹回不来了哟,他吃了炮子儿\footnote{〔炮子儿〕枪弹或炮弹。“吃了炮子儿”指中弹身亡。}啰!”那时的我,真好像父亲死了似的那么悲伤。这时候,蔡老师援助了我,批评了我的“反对派”们,还写了一封信劝慰我,说我是“心清如水的学生”。一个老师排除孩子世界里的一件小小的纠纷,是多么平常,可是回想起来,那时候我却觉得是给了我莫大的支持!在一个孩子的眼睛里,他的老师是多么慈爱,多么公平,多么伟大的人啊!
    
    每逢放假的时候,我们就更不愿离开她。我还记得,放假前我默默地站在她的身边,看她收拾东西的情景。蔡老师!我不知道你当时是不是察觉,一个孩子站在那里,对你是多么的依恋!至于暑假,对于一个喜欢他的老师的孩子来说,又是多么漫长!记得在一个夏季的夜里,席子铺在当屋,旁边燃着蚊香,我睡熟了。不知道睡了多久,也不知道是夜里的什么时候,我忽然爬起来,迷迷糊糊地往外就走。母亲喊住我:
    
    “你要去干什么?”
    
    “找蔡老师……”我模模糊糊地回答。
    
    “不是放暑假了么?”
    
    哦,我才醒了。看看那块席子,我已经走出六七尺远。母亲把我拉回来,劝说了一会,我才睡熟了。我是多么想念我的蔡老师啊!至今回想起来,我还觉得这是我记忆中的珍宝之一。一个孩子的纯真的心,就是那些在热恋中的人们也难比啊!什么时候,我能再见一见我的蔡老师呢?
    
    可惜我没有上完初小\footnote{〔初小〕当时的学制把小学分为初小和高小。初小为前四年。},就转到县立五小上学去了,从此,我就和蔡老师分别了。
    
\end{normalsize}


\newpage

\textbf{注释}:

\vspace{-1em}

\begin{itemize}
    \setlength\itemsep{-0.2em}
    \item 〔褪色〕颜色不再鲜艳,变得暗淡。
    \item 〔平生〕有生以来。
    \item 〔依恋〕依靠留恋,舍不得离开。
    \item 〔劝慰〕劝说安慰。
    \item 〔莫大〕没有比这个更大,极大。
    \item 〔牵挂〕挂念。
    \item 〔迷迷糊糊〕神志不清。
\end{itemize}

\chapter{东方红一号发射}

\begin{normalsize}
    
    1965年初,我大学毕业,分配到七机部一院\footnote{〔七机部一院〕第七机械工业部第一研究院,现中国航天科技集团公司第一研究院,也称“中国运载火箭技术研究院”。}工作。不久,“651”工程\footnote{〔“651”工程〕研制“东方红一号”卫星的工程代称。以1965年1月周恩来总理批示赵九章建议书的时间为名。}正式启动,将要研制发射我国第一颗人造卫星。它的名字决定为“东方红一号”,运载火箭则命名为“长征一号”。我与一批同学有幸参与。
    
    我当时所在的十二所六室一组负责火箭上自动控制系统的综合设计。从方案论证、设计实验到发射成功全过程,我都参与了。
    
    1970年3月,“651”工程即将进入到酒泉基地的试验阶段。一天,我接到通知:带上图纸资料及日常生活用品,到南苑东高地\footnote{〔南苑东高地〕一院本部所在地。}开会。赶到后,我才得知是要随专列出发去基地。
    
    这是一趟保密程度极高的军运专列,相关场地及设施都采取严密的安保措施,甚至开行命令都是派专人送达,没有用电信手段,以防泄密。列车上会有一个班战士负责警卫,列车运行途中铁路沿线还有战士站岗。沿线每一段铁路局的首席军代表随车到停靠站,与下一段铁路局的军代表交接。临时停靠客站时,车站都会采取戒严措施,旅客及无关人员不准入内。
    
    东高地的一院院内专列停车场提前由警卫部队检查并派重兵警卫。上车后,我关注了一下专列的情况。机车后面只挂7节车,除了装载火箭及卫星的几节安有铁轨的厢式货车、卧铺车及餐车外,还有一节上面只有两条铁轨的平板车,这是专门用来装卸火箭这样的大型物件的。装车前货车及平板车沿铁路线被推入总装车间,在这里,火箭已被分解成3节,分别固定在装有轮子的专用支架上,先用吊车吊起一节放到平板车铁轨上,再沿与其相连的货车铁轨推入货车并固定好,全部装好再运至专列停车场。专列带上平板车便于到达目的地卸车。
    
    乘车人员很少。有警卫班、军代表、技术人员和总装车间外厂组的几位师傅。这几位师傅相当于火箭和卫星的“保姆”,负责把它们从总装车间护送到基地,沿途还要对装运设备的车厢内部情况进行检查。
    
    经过几天几夜,专列才到达基地。到达后,火箭及卫星就由基地接管了,一切检测及最后发射工作都由基地的同志完成,而设计制造方在基地的代表一定会坚守在相关的现场,但并不去操纵设备,只在仪器设备或系统出了问题时,他们才会出面协助解决。
    
    基地先将星箭运到一部\footnote{〔一部〕“长征”系列运载火箭总体设计单位。}的一个厂房。在这里,火箭处于水平状态,也未加注燃料。然后就对各仪器及系统做全面详细的检测,并对能否向发射场转运作出评估。业内把这里称为水平阵地,也叫技术阵地,对火箭测试就叫水平测试。在发射场,火箭是垂直竖立的,那里叫发射阵地,也叫垂直阵地。
    
    测试过程中,我发现火箭第一、二级的级间分离线路有点问题。在一个陀螺仪测试时,又发现其中的漂移数据\footnote{〔漂移数据〕漂移指陀螺仪测量中的累积误差。已知陀螺仪的漂移数据,可以反过来校准误差。漂移数据错误将导致测量出现误差。}不正常,经查证是生产厂家调错了。问题很快由相关单位解决了。
    
    陀螺仪是火箭上对制造环境要求极高、制造工艺极复杂、造价昂贵的仪器。它是利用其惯性对火箭飞行进行定位的,即发射前就定了飞行方向和轨迹,还能测量飞行中偏离此“定位”的程度,以便及时纠正。若调错了漂移数据,火箭飞行就有偏差,可想而知后果有多严重。
    
    4月中旬,水平阵地全部检测工作完成。4月14日下午,主要负责人及一线技术人员代表抵达北京向周总理领导的中央专委汇报。我受派参加了汇报小组。
    
    我们在人民大会堂见到了周总理。听取汇报的还有李先念副总理、国家计委副主任余秋里、总政治部主任李德生、李福泽、七机部副部长钱学森及当时中央军委办事组的几位成员。
    
    作汇报的主要是“651”工程一院负责人任新民,汇报过程中涉及到的具体问题,就由相关的技术人员作答。我作答的是控制系统计算机可靠性问题。任新民还就此问题向总理作了简要说明,总理认可了。
    
    运载火箭系统非常复杂。发射时又加注了大量燃料,其中还有剧毒物质。如果发射失败,造成的严重后果是多方面的。一、有损国家声誉;二、掉下来有的东西可能还比较完整,能够识别,造成技术泄密;三、对人民生命财产造成损失;四、若发射失败,但离卫星入轨的条件差得不多,火箭第三级及卫星会飞很远才掉下来,掉在公海大洋里没啥问题。若掉在外国,是友好国家还好办,若掉在有敌对情绪的国家,则成国际事件。
    
    这就要求精确计算,预判出各种可能,报政府做到心中有数,到时从容应对。相关人员为此,又采取了多种举措。
    
    一、合理选择航区\footnote{〔航区〕火箭飞行轨迹下方地域。}。详细了解航区情况,尽量避开城市、人口稠密区及重要地面设施。为此一院一部派专员到预定航区调查,确保一切情况在掌握中。
    
    二、在火箭上设销毁装置。一旦火箭飞行失常就启动,让火箭爆炸后在空中解体成小碎片。这样对地面危害比较小,也能避免技术泄密。我们设计了两套系统,一套是自毁系统,一套是遥控系统。无论火箭控制系统检测到问题,还是地面人员判断到问题,都能销毁火箭。设计两套系统是为保险,一套没起作用,还有一套。
    
    不过,还有一个问题,让设计人员左右为难,就是过载开关的问题。很多人认为,一旦火箭出故障,卫星不能正常入轨,此时若播放《东方红》乐曲,恐怕会造成不好的政治影响。为此,火箭上设计安装了过载开关,以确保只在发射成功时才播放乐曲。但反过来想,过载开关自身也会出故障。要是发射成功了,却没有播放乐曲,就太可惜了。总理问:“你们认为火箭、卫星到底可靠不可靠啊?”几位负责人都说,从测试检查情况来看,火箭和卫星质量是可靠的。总理听了,说:“既然你们认为可靠,那我个人认为这个开关可以不要。不过,我得先向中央报告之后,再正式通知你们。”
    
    16日深夜,汇报小组接到通知,中央政治局同意了发射方案。过载开关截断\footnote{〔截断〕指不启用。},不投入使用。
    
    17日凌晨4时,我们返回基地,钱学森也同机过来,坐镇指挥。星箭转入垂直阵地。发射前的最后准备工作开始了。期间,上级专门安排大家做事故预想,为的是把所有可能出现的问题都发现和解决在发射前。
    
    4月24日21时35分,火箭发射升空。21时48分,地面指挥所确认星箭分离成功,卫星顺利入轨。90分钟后,卫星环绕地球一圈经过喀什\footnote{〔喀什〕新疆南部城市,有当时我国最西端的地面卫星跟踪站。}上空,酒泉卫星发射中心的收音机里响起了《东方红》的歌声。发射过程完美!
    
    东方红一号卫星发射成功,说明我们能把卫星准确送上天,也就能研发出射程更远的导弹。有了“两弹一星”,我们再也不怕核大国的恐吓,就能够独立自主,专心建设我们的国家。
    
\end{normalsize}


\newpage

\textbf{注释}:

\vspace{-1em}

\begin{itemize}
    \setlength\itemsep{-0.2em}
    \item 〔左右为难〕左也不好,右也不是。形容无论怎样做都有难处。
\end{itemize}

\chapter{最后一课}

\begin{normalsize}
    
    那天早晨上学,我去得很晚,心里很怕韩麦尔先生骂我,况且他说过要问我们分词\footnote{〔分词〕欧洲语言中的语法现象。},可是我连一个字也说不上来。我想就别上学了,到野外去玩玩吧。
    
    天气那么暖和,那么晴朗!
    
    画眉在树林边宛转地唱歌;锯木厂后边草地上,普鲁士兵\footnote{〔普鲁士〕指德国。1871年,普鲁士王国击败法国后,成立德意志帝国。}正在操练。这些景象,比分词用法有趣多了;可是我还能管住自己,急忙向学校跑去。
    
    我走过镇公所的时候,看见许多人站在布告牌前边。最近两年来,我们的一切坏消息都是从那里传出来的:败仗啦,征发啦,司令部的各种命令啦,我也不停步,只在心里思量:“又出了什么事啦?”
    
    铁匠华希特带着他的徒弟也挤在那里看布告,他看见我在广场上跑过,就向我喊:“用不着那么快呀,孩子,你反正是来得及赶到学校的!”
    
    我想他在拿我开玩笑,就上气不接下气地赶到韩麦尔先生的小院子里。
    
    平常日子,学校开始上课的时候,总有一阵喧闹,就是在街上也能听到。开课桌啦,关课桌啦,大家怕吵捂着耳朵大声背书啦……还有老师拿着大铁戒尺在桌子上紧敲着,“静一点,静一点……”
    
    我本来打算趁那一阵喧闹偷偷地溜到我的座位上去;可是那一天,一切偏安安静静的,跟星期日的早晨一样。我从开着的窗子望进去,看见同学们都在自己的座位上了;韩麦尔先生呢,踱来踱去,胳膊底下夹着那怕人的铁戒尺。我只好推开门,当着大家的面走进静悄悄的教室。你们可以想像,我那时脸多么红,心多么慌!
    
    可是一点儿也没有什么。韩麦尔先生见了我,很温和地说:“快坐好,小弗郎士,我们就要开始上课,不等你了。”
    
    我一纵身跨过板凳就坐下。我的心稍微平静了一点儿,我才注意到,我们的老师今天穿上了他那件挺漂亮的绿色礼服,打着皱边的领结,戴着那顶绣边的小黑丝帽。这套衣帽,他只在督学来视察或者发奖的日子才穿戴。而且整个教室有一种不平常的严肃的气氛。最使我吃惊的是,后边几排一向空着的板凳上坐着好些镇上的人,他们也跟我们一样肃静。其中有郝叟老头儿,戴着他那顶三角帽,有从前的镇长,从前的邮递员,还有些旁的人。个个看来都很忧愁。郝叟还带着一本书边破了的初级读本,他把书翻开,摊在膝头上,书上横放着他那副大眼镜。
    
    我看见这些情形,正在诧异,韩麦尔先生已经坐上椅子,像刚才对我说话那样,又柔和又严肃地对我们说:“我的孩子们,这是我最后一次给你们上课了。柏林已经来了命令,阿尔萨斯和洛林的学校只许教德语了。新老师明天就到。今天是你们最后一堂法语课,我希望你们多多用心学习。”
    
    我听了这几句话,心里万分难过。啊,那些坏家伙,他们贴在镇公所布告牌上的,原来就是这么一回事!
    
    我的最后一堂法语课!
    
    我几乎还不会作文呢!我再也不能学法语了!难道这样就算了吗?我从前没好好学习,旷了课去找鸟窝,到萨尔河上去溜冰……想起这些,我多么懊悔!我这些课本,语法啦,历史啦,刚才我还觉得那么讨厌,带着又那么沉重,现在都好像是我的老朋友,舍不得跟它们分手了。还有韩麦尔先生也一样。他就要离开了,我再也不能看见他了!想起这些,我忘了他给我的惩罚,忘了我挨的戒尺。
    
    可怜的人!
    
    他穿上那套漂亮的礼服,原来是为了纪念这最后一课!现在我明白了,镇上那些老年人为什么来坐在教室里。这好像告诉我,他们也懊悔当初没常到学校里来。他们像是用这种方式来感谢我们老师四十年来忠诚的服务,来表示对就要失去的国土的敬意。
    
    我正想着这些的时候,忽然听见老师叫我的名字。轮到我背书了。天啊,如果我能把那条出名难学的分词用法从头到尾说出来,声音响亮,口齿清楚,又没有一点儿错误,那么任何代价我都愿意拿出来的。可是开头几个字我就弄糊涂了,我只好站在那里摇摇晃晃,心里挺难受,头也不敢抬起来。我听见韩麦尔先生对我说:
    
    “我也不责备你,小弗郎士,你自己一定够难受的了。这就是了。大家天天都这么想:‘算了吧,时间有的是,明天再学也不迟。’现在看看我们的结果吧。唉,总要把学习拖到明天,这正是阿尔萨斯人最大的不幸。现在那些家伙就有理由对我们说了:‘怎么?你们还自己说是法国人呢,你们连自己的语言都不会说,不会写!……’不过,可怜的小弗郎士,也并不是你一个人的过错,我们大家都有许多地方应该责备自己呢。
    
    “你们的爹妈对你们的学习不够关心。他们为了多赚一点儿钱,宁可叫你们丢下书本到地里,到纱厂里去干活儿。我呢,我难道就没有应该责备自己的地方吗?我不是常常让你们丢下功课替我浇花吗?我去钓鱼的时候,不是干脆就放你们一天假吗?……”
    
    接着,韩麦尔先生从这一件事谈到那一件事,谈到法国语言上来了。他说,法国语言是世界上最美的语言最明白,最精确;又说,我们必须把它记在心里,永远别忘了它,亡了国当了奴隶的人民,只要牢牢记住他们的语言,就好像拿着一把打开监狱大门的钥匙。说到这里,他就翻开书讲语法。真奇怪,今天听讲,我全都懂。他讲的似乎挺容易,挺容易。我觉得我从来没有这样细心听讲过,他也从来没有这样耐心讲解过。这可怜的人好像恨不得把自己知道的东西在他离开之前全教给我们,一下子塞进我们的脑子里去。
    
    语法课完了,我们又上习字课。那一天,韩麦尔先生发给我们新的字帖,帖上都是美丽的圆体字:“法兰西”“阿尔萨斯”“法兰西”“阿尔萨斯”。这些字帖挂在我们课桌的铁杆上,就好像许多面小国旗在教室里飘扬。个个都那么专心,教室里那么安静!只听见钢笔在纸上沙沙地响。有时候一些金甲虫飞进来,但是谁都不注意,连最小的孩子也不分心,他们正在专心画“杠子”,好像那也算是法国字。屋顶上鸽子咕咕咕咕地低声叫着,我心里想:“他们该不会强迫这些鸽子也用德国话唱歌吧!”
    
    我每次抬起头来,总看见韩麦尔先生坐在椅子里,一动也不动,瞪着眼看周围的东西,好像要把这小教室里的东西都装在眼睛里带走似的。只要想想:四十年来,他一直在这里,窗外是他的小院子,面前是他的学生;用了多年的课桌和椅子,擦光了,磨损了;院子里的胡桃树长高了;他亲手栽的紫藤,如今也绕着窗口一直爬到屋顶了。可怜的人啊,现在要他跟这一切分手,叫他怎么不伤心呢?何况又听见他的妹妹在楼上走来走去收拾行李!他们明天就要永远离开这个地方了。
    
    可是他有足够的勇气把今天的功课坚持到底。习字课完了,他又教了一堂历史。接着又教初级班拼他们的ba,be,bi,bo,bu。在教室后排座位上,郝叟老头儿已经戴上眼镜,两手捧着他那本初级读本,跟他们一起拼这些字母。他感情激动,连声音都发抖了。听到他古怪的声音,我们又想笑,又难过。啊!这最后一课,我真永远忘不了!
    
    忽然教堂的钟敲了十二下。祈祷的钟声也响了。窗外又传来普鲁士兵的号声他们已经收操了。韩麦尔先生站起来,脸色惨白,我觉得他从来没有这么高大。
    
    “我的朋友们啊,”他说,“我——我——”
    
    但是他哽住了,他说不下去了。
    
    他转身朝着黑板,拿起一支粉笔,使出全身的力量,写了两个大字:
    
    “法兰西万岁!”
    
    然后他呆在那儿,头靠着墙壁,话也不说,只向我们做了一个手势:“都结束了,你们走吧。”
    
\end{normalsize}


\newpage

\textbf{注释}:

\vspace{-1em}

\begin{itemize}
    \setlength\itemsep{-0.2em}
    \item 〔诧异〕惊讶,觉得不寻常。
\end{itemize}

\chapter{生命的意义}

\begin{normalsize}
    
    保尔沿着小镇上冷冷清清的街道踱着步子,不知不觉走到了松树林前,在岔路口停住了脚步。岔路口右面是从前的监狱,阴森森的,和松林只隔着一道挺高的尖木栅栏。监狱后面是医院的白色楼房。
    
    就在这里,瓦莉亚和她的同志们被送上了绞架\footnote{〔绞架〕执行绞刑的工具。},牺牲在这空寂的广场上。保尔在当年竖立绞架的地方默默站了很一会儿,然后他走下路边的陡坡,进了埋葬烈士的墓地。
    
    也不知是哪个有心人,用冷杉枝条把那一排坟墓装饰了起来,给这片小小的墓地围上了一圈绿色的栅栏。
    
    陡坡外高耸着挺拔的青松,谷地里满铺着如茵的嫩草。这儿是小镇的尽头,阴郁而冷清。只有松树林轻声的低语,只有复苏的大地散发出新春的气息。
    
    就在这里,故乡的同志们英勇地牺牲了。为了改变那些生来就一无所有、生来就得做奴隶的人们的命运,为了使他们的生活变得美好,他们献出了自己年轻的生命。
    
    保尔缓缓摘下军帽。悲愤,深深的悲愤充满了他的心。
    
    人最宝贵的是生命。生命,每个人只有一次。
    
    人的一生应当这样度过:每当回忆往事的时候,他不会因为虚度年华而悔恨,也不会因为碌碌无为而羞耻;在临死的时候,他能够说:“我的整个生命和全部精力,都已经献给了世界上最壮丽的事业——为人类解放而斗争。”人应当充分利用每一天,因为意外的疾病或事故随时可能终结他的生命。
    
\end{normalsize}


\newpage

\textbf{注释}:

\vspace{-1em}

\begin{itemize}
    \setlength\itemsep{-0.2em}
    \item 〔茵〕铺垫的东西,垫子、褥子、毯子的通称。
    \item 〔踱〕慢慢地走。
    \item 〔冷冷清清〕不热闹,凄凉萧条。
    \item 〔烈士〕为正义事业而牺牲的人。
    \item 〔阴郁〕树木繁盛而幽深。
    \item 〔虚度〕度过(时间)而什么也没做。
    \item 〔碌碌无为〕无能,没有做成什么事。
\end{itemize}

\chapter{花市}

\begin{normalsize}
    
    今天城里逢集,街上还很安静的时候,花市上就摆满了一片花草。紫竹、刺梅、石榴、绣球、倒挂金钟、四季海棠,真是花团锦簇,千丽百俏,半条街飘满了清淡的花香。
    
    一个小小的县城里,为什么出现了这么多卖花的人?有的人说,栽培花卉不但可以供人观赏,美化环境,而且许多花卉具有药用、食用和其他用途,可以增加社会财富;也有人说农民们见钱眼开,只要能赚钱,什么生意都想做一做;还有一种简单但是富有哲理的说法,那就是:“如今买花的人多了,卖花的人自然也就多了。”
    
    “老大爷,你买了这盆三叶梅吧,这花便宜,好活,你看它开得多么鲜艳!”
    
    花市东头,一个卖花的乡下姑娘在和一个看花的乡下老头谈生意。这个姑娘集集来卖花,经常赶集的人都认识她,但不知道她叫什么名字。姑娘不过二十一二岁,生得细眉细眼,爱笑,薄薄的嘴唇很会谈生意。
    
    那老头蹲在她的花摊前面,摇摇头,对那盆开满粉红色零星小花的三叶梅表示不感兴趣。姑娘又说:
    
    “那就买了这盆兰花吧,古人说,它是‘香祖’……”
    
    “那一盆多少钱?”老头抬起下巴朝花车儿上一指,打断她的话。
    
    那是一盆令箭荷花\footnote{〔令箭荷花〕仙人掌科植物。茎扁平状如令箭,花似睡莲,因此叫令箭荷花。令箭:军中传令用的小旗,杆头加箭头,因此叫令箭。}。在今天的花市上,这是独一份儿。葱翠的令箭似的叶状枝上,四朵花竞相开放,那花朵大,花瓣儿层层叠叠,光洁鲜亮,一层紫红,一层桃红,一层粉红,花丝弯曲嫩黄,阳光一照,整个花朵就像薄薄的彩色玻璃做的一样。
    
    姑娘说:“老大爷,那是令箭荷花。”
    
    “我要的就是令箭荷花!”
    
    “它贵。”
    
    “有价儿没有?”
    
    姑娘听他口气很大,把他仔细打量了一遍。老头瘦瘦的,大约60多岁,白布褂子,紫花裤子,敞着怀,露着黑黑的结实的胸脯,不像是养种花草的人。姑娘问:
    
    “老大爷,你是哪村的?”
    
    “严村的。”
    
    “哪村?”
    
    “严村,城北的严村。”
    
    “晓得晓得。”一个看花的小伙子打趣说,“严村,好地方啊,那里的人们身上不缺‘胡萝卜素’……”
    
    看花的人们一齐笑了,姑娘笑得弯下腰去。严村是个苦地方,多少年来,那里的人们每年分的口粮只能吃七八个月,不足部分,就用胡萝卜接济\footnote{〔接济〕物资上援助。这里有接替的意思。}。这一带人们教育自己不爱做活的姑娘时,总是这么说:“懒吧,懒吧,捉不住针,拿不起线,长大了看到哪里找个婆家。拙手笨脚没人要,就把你嫁到严村吃胡萝卜去!”这个卖花的姑娘,小时候一定也受到过大人的这种警告吧?
    
    在人们的笑声中,老头红了脸,好像受了莫大羞辱。他一横眉,冲着姑娘说:“笑!你是来做买卖的,还是来笑的!”
    
    姑娘一点也不急,反倒觉得这个老头很可爱,依然笑着说:“老大爷,如今村里怎样啊?”
    
    “不怎样!”
    
    “去年,工值多少?”
    
    老头没有回答,看看买花的人多起来了,就又指着那盆令箭荷花说:“多少钱,有价儿没有?”
    
    “十五。”姑娘止住笑说。
    
    “多少?”人们睁大眼睛。
    
    “十五。”姑娘重复道。
    
    “坑人哩!”老头站起身。
    
    “太贵了,太贵了。”人们也说。
    
    姑娘看看众人,又笑了说:“是贵。这东西不能吃,不能喝,一块钱一盆也不便宜。可是老大爷,人各一爱,自己心爱的东西,讲什么贵贱呀?想便宜买胡萝卜去,十五块钱买一大车,一冬天吃不完。——你又不买,偏偏想来挨坑,那怨谁呢?”
    
    姑娘的巧嘴儿又把人们逗笑了。老头也咧着大嘴笑了说:“不买不买,太贵太贵。”
    
    “你给多少?”姑娘赶了一句。
    
    “十块钱。”老头鼓鼓肚子。
    
    “再添两块,十二块钱叫你搬走。”姑娘最后表示慷慨。
    
    老头用手捻\footnote{〔捻〕用手指捏着搓转。}着胡子,斜着眼珠望着那盆令箭荷花,牙疼似的咂起嘴唇儿。人们说:“姑娘,自家出产的,让他两块吧!”
    
    “老头,买了吧,值!”
    
    “十块,多一分钱也不买。”老头坚定地说。
    
    “十二,少一分钱也不卖。”姑娘也不相让。
    
    “不卖,你留着自己欣赏吧!”老头白了姑娘一眼,终于走了,但他不住回头望一望那盆令箭荷花。
    
    上午十点钟,集上热闹起来,花市上也站满了人。那些买花的,看花的,和猪市、兔市、木器市上一样,大半是头上戴草帽或扎手巾的乡下人。原来乡下人除了吃饭穿衣,他们的生活中也是需要一点花香的。
    
    姑娘的生意很好,转眼工夫,就卖了许多花。她正忙着,听见人群里有人嚷道:“姑娘,拿来,买了!”抬头一看,那老头又回来了,脸上红红的,好像刚刚喝了酒。
    
    “十二。”姑娘说。
    
    “给你!”老头忍疼说,“你说得对,人各一爱。我只当耽误了八天工,只当闺女少包了半垄棉花,只当又割资本主义尾巴\footnote{〔割资本主义尾巴〕指“文革”中把农民个体户正常的农副作业生产当“资本主义尾巴”砍掉。}呢,割了我两只老母鸡!”
    
    姑娘笑了笑,把那盆令箭荷花搬到他跟前去。正要付钱,一个眉目清秀的干部打扮的年轻人挤上来说:“多少钱?多少钱?”
    
    “十二。”姑娘答。
    
    “我买我买!”年轻干部去掏钱包。
    
    “我买了,我买了!”老头胳膊一乍,急忙护住那盆花。
    
    年轻干部手里摇着黑色纸扇,上下看了老头一眼,似笑非笑地说:“老头,你晓得这是什么花?”
    
    “令箭荷花!”
    
    “原产哪里?”
    
    “原产……原产姑娘家里!”
    
    年轻干部哈哈大笑。笑罢,用扇子照老头的肩上拍了两拍,说:“墨西哥。——让给我吧,老头。”
    
    “我买的东西,为什么让给你?”
    
    “唉,你买它做什么!”
    
    “你买它做什么?”
    
    “我看。”
    
    年轻干部笑了一下,弯腰去搬那盆花。老头大手一伸,急忙捉住他的手,向后一扔,也给他笑了一下:“我也看!”
    
    人群里爆发了一片笑声。姑娘没有笑,手拿着一块小花手绢,在怀里扇着风,冷冷地注视着年轻干部的行动。年轻干部无可奈何,用扇子挡着嘴,对老头嘀咕了几句什么。老头立刻冷着脸说:“不行不行,明天也是我的生日,我也爱花!”
    
    “你这个人真难说话!这么贵,你吃它喝它?”
    
    “咦,我不吃它喝它,你那个上级吃它喝它?”
    
    人们听得明白,就又笑起来了。年轻干部不知出于一种什么心理,陡地变了脸色说:“你是哪个村的?”
    
    “严村的。”
    
    “你们村的支书是谁?”
    
    老头眨眨眼睛,向众人说:“你们看这个人怪不怪,我买一盆花,他问我们村的支书是谁做什么?”
    
    这一回,人们没有笑。乡下人自有乡下人的经验,他们望着年轻干部的脸色,猜测着他的身份、来历,纷纷说:
    
    “老头,让给他吧,与人方便自己方便。”
    
    “是啊,让给他吧,只当是学雷锋哩……”
    
    老头听人劝说,心里好像活动了一点。他望着那盆令箭荷花,用手捻着胡子,又咂起嘴唇儿。年轻干部冷冷一笑,乘势说:“就是嘛,你们乡下人,还缺花看吗?高粱花、棒子花、打破碗碗花,野花野草遍地都是。姑娘,我出十三块钱买了!”
    
    说着,把钱送到姑娘脸前。
    
    姑娘不接他的钱,手拿着小花手绢,依然那么扇着,冷冷地盯着他。他还想说什么,那老头一跳脚,从怀里掏出一把崭新的票子,扯着嗓子嚷道:
    
    “你要那么说,我出十四块钱!”
    
    “我出十五块钱!”
    
    “我出……”
    
    “你这个人真是自不量力!”姑娘好像生了很大的气,瞪了老头一眼说,“你干一天活,挣几个钱,充什么大肚汉子呢!
    
    十五不要,十四不要,十二也不要了,看在你来得早,凭着你那票子新鲜,依你,十块钱搬走吧!记住,原产墨西哥,免得叫人再拿扇子拍你!”
    
    “多少多少?”年轻干部睁大眼睛。
    
    “十块钱,我们谈好了的。”姑娘轻轻一笑,对他倒很和气。
    
    老头愣了一下,呵呵地笑了,赶快付了钱,搬起那盆令箭荷花就走。年轻干部气得脸色发白,用扇子指着姑娘的脸,一时不知说什么好:
    
    “你你……”
    
    “我叫蒋小玉,南关的,我们支书叫蒋大河,还问我们治保主任是谁吗?”
    
    人们明白姑娘的心思,一齐仰着脖子大笑起来。在笑声中,人们都去摸自己的钱包,都想买姑娘一盆花,姑娘就忙起来了。她笑微微地站在百花丛中,也像一枝花,像一枝挺秀淡雅的兰花吧。
    
\end{normalsize}


\newpage

\textbf{注释}:

\vspace{-1em}

\begin{itemize}
    \setlength\itemsep{-0.2em}
    \item 〔花卉〕花草。卉:草。
    \item 〔葱翠〕草木茂盛青翠。
    \item 〔千丽百俏〕俏丽多姿。
    \item 〔无可奈何〕不情愿又没办法。
    \item 〔陡地〕陡然,突然。
    \item 〔羞辱〕使羞耻,使耻辱。
    \item 〔花团锦簇〕形容花色繁多,美艳亮丽。
    \item 〔横眉〕愤怒地皱眉头。
    \item 〔垄〕田里农作物的行,或行与行之间的空地。
\end{itemize}

\chapter{大自然的语言}

\begin{normalsize}
    
    立春过后,大地渐渐从沉睡中苏醒过来。冰雪融化,草木萌发,各种花次第开放。再过两个月,燕子翩然归来。不久,布谷鸟也来了。于是转入炎热的夏季,这是植物孕育果实的时期。到了秋天,果实成熟,植物的叶子渐渐变黄,在秋风中簌簌地落下来。北雁南飞,活跃在田间草际的昆虫也都销声匿迹。到处呈现一片衰草连天的景象,准备迎接风雪载途的寒冬。在地球上温带和亚热带区域里,年年如是,周而复始。
    
    几千年来,劳动人民注意了草木荣枯、候鸟去来等自然现象同气候的关系,据以安排农事。杏花开了,就好像大自然在传语要赶快耕地;桃花开了,又好像在暗示要赶快种谷子。布谷鸟开始唱歌,劳动人民懂得它在唱什么:“阿公阿婆,割麦插禾\footnote{〔阿公阿婆,割麦插禾〕这是把布谷鸟的叫声想象成催促农民及时耕作的话。禾,这里指稻秧。}。”这样看来,花香鸟语,草长莺飞,都是大自然的语言。
    
    这些自然现象,我国古代劳动人民称它为物候。物候知识在我国起源很早。古代流传下来的许多农谚就包含了丰富的物候知识。到了近代,利用物候知识来研究农业生产,已经发展为一门科学,就是物候学。物候学记录植物的生长荣枯,动物的养育往来,如桃花开、燕子来等自然现象,从而了解随着时节推移的气候变化和这种变化对动植物的影响.
    
    物候观测使用的是“活的仪器”,是活生生的生物。它比气象仪器复杂得多,灵敏得多。物候观测的数据反映气温、湿度等气候条件的综合,也反映气候条件对于生物的影响。应用在农事活动里,比较简便,容易掌握。物候对于农业的重要性就在这里。下面是一个例子。
    
    北京的物候记录,1962年的山桃、杏花、苹果、榆叶梅\footnote{〔榆叶梅〕落叶灌木或小乔木,花粉红色,核果球形、红色。可供观赏。}、西府海棠、丁香、刺槐的花期比1961年迟十天左右,比1960年迟五六天。根据这些物候观测资料,可以判断北京地区1962年农业季节来得较晚。而那年春初种的花生等作物仍然是按照往年日期播种的,结果受到低温的损害。如果能注意到物候延迟,选择适宜的播种日期,这种损失就可能避免。
    
    物候现象的来临决定于哪些因素呢?
    
    首先是纬度。越往北桃花开得越迟,候鸟也来得越晚。值得指出的是物候现象南北差异的日数因季节的差别而不同。中国大陆性气候显著,冬冷夏热。冬季南北温度悬殊,夏季却相差不大。在春天,早春跟晚春也不相同。如在早春三四月间,南京桃花要比北京早开20天,但是到晚春五月初,南京刺槐开花只比北京早10天。所以在华北常感觉到春季短促,冬天结束,夏天就到了。
    
    经度的差异是影响物候的第二个因素。凡是近海的地方,比同纬度的内陆,冬天温和,春天反而寒冷。所以沿海地区的春天的来临比内陆要迟若干天。如大连纬度在北京以南约1°,但是在大连,连翘\footnote{〔连翘〕落叶灌木,春季开鲜黄色花,果实可以入药。}和榆叶梅的盛开都比北京要迟一个星期。又如济南苹果开花在四月中或谷雨节,烟台要到立夏。两地纬度相差无几,但烟台靠海,春天便来得迟了。
    
    影响物候的第三个因素是高下的差异。植物的抽青、开花等物候现象在春夏两季越往高处越迟,而到秋天乔木的落叶则越往高处越早。不过研究这个因素要考虑到特殊的情况。例如秋冬之交,天气晴朗的空中,在一定高度上气温反比低处高。这叫逆温层。由于冷空气比较重,在无风的夜晚,冷空气便向低处流。这种现象在山地秋冬两季,特别是这两季的早晨,极为显著,常会发现山脚有霜而山腰反无霜。在华南丘陵区把热带作物引种在山腰很成功,在山脚反不适宜,就是这个道理。
    
    此外,物候现象来临的迟早还有古今的差异。根据英国南部物候的一种长期记录,拿1741到1750年十年平均的春初七种乔木油青和开花日期同1921到1930年十年的平均值相比较,可以看出后者比前者早九天。就是说,春天提前九天。
    
    物候学这门科学接近生物学中的生态学和气象学中的农业气象学。物候学的研究首先是为了预报农时,选择播种日期。此外还有多方面的意义。物候资料对于安排农作物区划,确定造林和采集树木种子的日期,很有参考价值,还可以利用来引种植物到物候条件相同的地区,也可以利用来避免或减轻害虫的侵害。中国有很大面积的山区土地可以耕种,而山区的气候、土壤对农作物的适应情况,有很多地方还有待调查。为了便利山区的农业发展,开展山区物候观测是必要的。
    
    物候学是关系到农业丰产的科学,我们要进一步加强物候观测,懂得大自然的语言,争取农业更大的丰收。
    
\end{normalsize}


\newpage

\textbf{注释}:

\vspace{-1em}

\begin{itemize}
    \setlength\itemsep{-0.2em}
    \item 〔次第〕一个挨着一个地,接连地。
    \item 〔翩然〕动作轻快的样子。
    \item 〔孕育〕怀孕时在体内养育胎儿,用来比喻酝酿着新事物。
    \item 〔簌簌〕多而杂乱地接连落下的样子,纷纷。
    \item 〔销声匿迹〕躲藏起来不发声不露面。
    \item 〔风雪载途〕风雪连接不断。载途:满路,形容多。
    \item 〔农谚〕有关农业生产的谚语。谚:因有道理而在群众中流传的固定语句。
    \item 〔纬度〕地理学名词。南北为纬,纬度即衡量南北的度数。以赤道为0度,北极为90度,称为北纬。越靠北,北纬度越大。
    \item 〔经度〕地理学名词。东西为经,经度即衡量东西的度数。以英国伦敦为0度,至太平洋中央为180度,称为东经。越靠东,东经度越大。
    \item 〔抽青〕发芽。抽:拔,比喻植物生长。
\end{itemize}

\chapter{一}

\begin{normalsize}
    
    \begin{verse}[0.5\linewidth]
        天际,于褐色的群峦间, \\
        太阳,这光辉无限的花, \\
        斜倚向大地,也将睡了。 \\
        不起眼的一朵雏菊,开在田野旁, \\
        坍倒的灰墙上,四下麦草疯长。 \\
        洁白无瑕,绽放纯真晕光, \\
        这小花,独立残垣, \\
        凝视那永恒青苍。 \\
        巨大的恒星倾洒着无尽流晖。 \\
        “可我,我也有光芒!”
    \end{verse}
    
\end{normalsize}


\newpage

\textbf{注释}:

\vspace{-1em}

\begin{itemize}
    \setlength\itemsep{-0.2em}
    \item 〔峦〕小而尖的山。
    \item 〔雏菊〕欧洲常见野花,春季开花。
    \item 〔绽放〕(花)开。绽:衣服裂开。
    \item 〔坍〕竖立的建筑物(墙、堤岸)倒塌。
    \item 〔残垣〕残破的建筑。垣:矮墙。
    \item 〔恒星〕自身能发出光和热的天体。这里指太阳。
\end{itemize}

\chapter{三年以后}

\begin{normalsize}
    
    \begin{verse}[0.5\linewidth]
        推开这摇摇欲坠的窄门, \\
        我徜徉在小小的花园里。 \\
        露珠上闪动着晨光温嫩, \\
        遍撒在花瓣上晶莹欲滴。
    \end{verse}
    
    
    \begin{verse}[0.5\linewidth]
        眼前一切如旧:葡萄藤蔓 \\
        缠满凉棚,棚里数把藤椅。 \\
        泉水低语,银光清亮依然, \\
        老杨树的悲怨永不停息。
    \end{verse}
    
    
    \begin{verse}[0.5\linewidth]
        蔷薇轻颤,犹同往日;犹同 \\
        往日,百合傲然,随风摇动。 \\
        往来的云雀\footnote{〔云雀〕百灵科鸟类。大体砂棕色,头后羽毛稍长,略成羽冠状。},都与我相熟。
    \end{verse}
    
    
    \begin{verse}[0.5\linewidth]
        甚至,还有薇莉达\footnote{〔薇莉达〕女先知,传说在日耳曼部落反抗古罗马统治时,曾预言了起义初期的胜利。}的雕塑。 \\
        石膏剥落,散落走道尽处。 \\
        纤影伫立在木犀\footnote{〔木犀〕草本植物,一尺高,花白或淡黄白色。}暗香中。
    \end{verse}
    
\end{normalsize}


\newpage

\textbf{注释}:

\vspace{-1em}

\begin{itemize}
    \setlength\itemsep{-0.2em}
    \item 〔徜徉〕自由自在地行走。
    \item 〔伫立〕久久站立。
    \item 〔石膏〕硫酸盐矿物,粉末调水后可做雕塑材料。
\end{itemize}

\chapter{播种季的傍晚}

\begin{normalsize}
    
    \begin{verse}[0.5\linewidth]
        黄昏时分,我坐在大门口 \\
        用憧憬的目光,欣赏 \\
        白日剩余的时间,怎样由 \\
        最后的农活来照亮
    \end{verse}
    
    
    \begin{verse}[0.5\linewidth]
        夜色沁入田里,我凝望着 \\
        一个老人,衣衫褴褛 \\
        将未来的收成,一把把地 \\
        撒入犁沟,感动难已
    \end{verse}
    
    
    \begin{verse}[0.5\linewidth]
        他的背影高大深沉,俯临 \\
        这深耕的土地。可以 \\
        感受到,他深深相信 \\
        每分每秒流过,皆有意义
    \end{verse}
    
    
    \begin{verse}[0.5\linewidth]
        他行走在无垠的大地上 \\
        来来去去,远撒谷粒 \\
        手掌张开,合拢,又再扬起 \\
        暗中远觑,我心徜徉
    \end{verse}
    
    
    \begin{verse}[0.5\linewidth]
        和着喧嚣,夜将他的影子 \\
        铺展开来,直到天边 \\
        也将播种者神圣的身姿 \\
        放大,遍布星辰之间
    \end{verse}
    
\end{normalsize}


\newpage

\textbf{注释}:

\vspace{-1em}

\begin{itemize}
    \setlength\itemsep{-0.2em}
    \item 〔憧憬〕向往。
    \item 〔沁〕渗入,浸润。
    \item 〔褴褛〕(衣服)破烂。
    \item 〔俯临〕俯身下看。临:从上向下看。
    \item 〔无垠〕广阔无边。
    \item 〔觑〕偷看。
    \item 〔喧嚣〕声音杂乱,不清静。
    \item 〔徜徉〕陶醉于某事物之中。
\end{itemize}

\chapter{刘胡兰}

\begin{normalsize}
    
    1945年,日本投降了。在中国的日军听到了投降的消息,仍然不死心,图谋东山再起。山西的军阀阎锡山想要拥兵自重,做土皇帝。在山西的日军就和阎锡山勾结,假投降,真合作,仍在山西横行无忌。直到1949年,山西人民仍然生活在军阀和日寇的阴影下。
    
    1946年下半年,阎锡山调集近万兵力,对晋中地区进行“扫荡”。“扫荡”后,文水县云周西村迎来了伪村长石佩怀。
    
    他积极为阎锡山军队派粮派款,抓壮丁、抢粮食,成了当地一害。我党决定消灭石佩怀,为民除害。
    
    处决石佩怀的行动中,负责侦查放哨的是14岁的刘胡兰。
    
    刘胡兰也是云周西村人。她从小积极参加革命,已经是“妇救会”的秘书。石佩怀的问题,也是她向区县干部反映的。
    
    “扫荡”中,文水县委让刘胡兰跟着部分干部转移上山。刘胡兰坚决要求留下来,坚持斗争。
    
    处决石佩怀后,阎锡山军队和镇上的地主伪官惊怒交加,派出“复仇队”到云周西村搜捕抢掠,要捉拿“凶手”。
    
    1947年1月12日上午,“复仇队”包围了云周西村。由于叛徒的出卖,刘胡兰等七名同志被捕了。
    
    为了给自己的反动统治“立威”,敌人打算让刘胡兰“自白”。审问开始了。敌人问:“你就是刘胡兰?”
    
    刘胡兰说:“我就是刘胡兰。”
    
    敌人又说:“已经有人供出,你是共产党。”
    
    刘胡兰说:“我就是共产党。”
    
    敌人问:“村里还有谁是共产党?”
    
    刘胡兰说:“就是我一个。”
    
    敌人问:“你认得谁是八路军?”
    
    刘胡兰说:“谁也不认得。”
    
    敌人见她年纪小,哄她说:“今天你跟着我走,我就放你一马,还给你一块土地。”
    
    刘胡兰不理睬。
    
    敌人恼羞成怒,大叫:“小小年纪,好嘴硬!你不怕死吗?”
    
    刘胡兰说:“怕死不当共产党!”
    
    敌人要刘胡兰屈服。要她保证“今后不再给八路军办事”。刘胡兰说:“那可保不住。”
    
    敌人无计可施,决定杀死刘胡兰。刘胡兰走到铡刀前,自己躺下去,敌人就把她杀害了。
    
    在刘胡兰牺牲后,毛主席为她写了八个字:“生的伟大,死的光荣”。
    
\end{normalsize}



\chapter{挖荠菜}

\begin{normalsize}
    
    我对荠菜\footnote{〔荠菜〕草本植物,花白色,茎叶嫩时可以吃。},有着一种特殊的感情……
    
    小的时候,我是那么馋!刚抽出嫩条还没打花苞的蔷薇枝,把皮一剥,我就能吃下去;刚割下来的蜂蜜,我会连蜂房一起放进嘴巴里;更别说什么青玉米棒子、青枣、青豌豆啰。所以,只要我一出门儿,碰上财主家的胖儿子,他就总要跟在我身后,拍着手、跳着脚地叫着:“馋丫头!馋丫头!”羞得我连头也不敢回。
    
    我感到又羞恼,又冤屈!七八岁的姑娘家,谁愿意落下这么个名声?可是有什么办法呢?我饿啊!我真不记得什么时候,那种饥饿的感觉曾经离开过我,就是现在,每当我回忆起那个时候的情景,留在我记忆里最鲜明的感觉,也还是一片饥饿……
    
    吃那些没收进主人家仓房里的东西,我还一次也没有被人家抓到过。倒不是因为我的运气格外好,而是人们多半并不想认真地惩罚一个饥饿的孩子。可有一次,我在财主家的地里掰玉米棒子,被他的大管家发现了,他立刻拿着一根又粗又直的木头棒子,毫不留情地紧紧向我追来。我没命地逃着。我想我一定跑得飞快,因为风在我的耳朵旁边呼呼直响。不知是我被吓昏了,还是平时很熟悉的那些田间小路有意捉弄我,为什么面前偏偏横着一条小河?追赶我的人越来越近了。我害怕到了极点,便不顾一切地纵身跳进那条河。
    
    河水并不很深,但是足以没过我那矮小的身子。我一声不响地挣扎着,扑腾着,身子失去了平衡。冰凉的河水呛得我好难受,我几乎背过气去,而河水却依旧在我身边不停地流着,流着……在由于恐怖而变得混乱的意识里,却出奇清晰地反映出岸上那个追赶我的人的残酷笑声。
    
    我简直不知道我是怎么样才爬上对岸的。更使我丧气的是脚上的鞋子不知什么时候掉了一只。我实在没有勇气重新回头去找那只丢失了的鞋子,可我也不敢回家,我怕妈妈知道。不,我并不是怕她打我。我是怕看见她那双被贫困的生活折磨得失去了光彩的、哀愁的眼睛。那双眼睛,会因为我丢失了鞋子而更加暗淡。
    
    我独自一人游荡在田野里。太阳落山了,琥珀色的晚霞渐渐地从天边退去。远处,庙里的钟声在薄暮中响起来。羊儿咩咩地叫着,由放羊的孩子赶着回圈了;乌鸦也呱呱地叫着回巢去了。夜色越来越浓了,村落啦,树林子啦,坑洼啦,沟渠啦,好像一下子全都掉进了神秘的沉寂里。我听见妈妈在村口焦急地呼唤着我的名字,只是不敢答应。一种比饥饿更可怕的东西平生头一次潜入了我那童稚的心……
    
    说过了这些,人们也许会理解我为什么对荠菜有着那么特殊的感情。
    
    经过一个没有什么吃食可以寻觅、因而显得更加饥饿的冬天,大地春回、万物复苏的日子重新来临了!田野里长满了各种野菜:雪蒿、马齿苋、灰灰菜、野葱……最好吃的是荠菜。把它下在玉米糊糊里,再放上点盐花,真是无上的美味啊!而挖荠菜时的那种坦然的心情,更可以称得上是一种享受:提着篮子,迈着轻捷的步子,向广阔无垠的田野里奔去。嫩生生的荠菜,在微风中挥动它们绿色的手掌,招呼我,欢迎我。我再也不必担心有谁会拿着大棒子凶神恶煞似地追赶我,我甚至可以不时地抬头看看天上吱吱喳喳飞过去的小鸟,树上绽开的花儿和蓝天上白色的云朵。那时,我的心里便会不由地升起一个热切的愿望:巴不得这个世界上的一切,都像荠菜一样是属于我们每一个人的。
    
    解放以后,我进了城。偶然,在大菜场里,也可以看到人工培植的荠菜出售。长得肥肥大大的,总有半尺来长,洗得干干净净,水灵灵的。一小扎,一小扎,码得整整齐齐地摆在菜摊子上,价钱也不贵。可我,总还是怀念那长在野地里的荠菜,就像怀念那些与自己共过患难的老朋友一样。
    
    多少年来,每到春天,我总要挑个风和日丽的日子,带上孩子们到郊区的野地里去挖荠菜。我明白,孩子们之所以在我的身旁跳着,跑着,尖声地打着唿哨\footnote{〔唿哨〕把手指放在嘴里用力吹,发出尖锐的哨子一样的声音。},多半因为这对他们来说,是一种有趣的游戏——和煦的阳光,绿色的田野,就像一幅优美的风景画似的展现在他们面前,使他们的身心全都感到愉快。他们长大一些之后,陪同我去挖荠菜,似乎就变成了对我的一种迁就了,正像那些恭顺的年轻人,迁就他们那些因为上了年纪而变得有点怪癖的长辈一样。这时,我深感遗憾:他们多半不能体会我当年挖荠菜的心情!
    
    等到我把一盘用精盐、麻油、味精、白糖精心调配好的荠菜放到餐桌上去的时候(小的时候,我可是做梦也没有想到我那可爱的荠菜会享受到今天这样的“荣华富贵”),他们也还是带着那种迁就的微笑,漫不经心地用筷子挑上几根荠菜……
    
    看着他们那双懒洋洋的筷子,我的心里就像翻倒了的五味瓶,什么滋味都有。因为我知道,这种赏光似的迁就,并不只是表现在对挖荠菜这一桩事情上,它还表现在对我们这一代人的一些见解和行为上。在他们看来,我们的有些见解和行为,都像陈列在博物馆里的出土文物——离他们的现实生活太远了,不顶用了。自然,我也并不认为我们的见解和行为就完全正确。只要他们不觉得厌烦,我甚至愿意跟他们谈谈我们在探索人生方面曾经走过的弯路,以便他们少付出一些不必要的代价。我真希望我们之间不要成为隔膜很深的两代人,而是心灵相通的朋友。
    
    孩子,让我们多谈谈心吧,让妈妈多讲讲当“馋丫头”时的故事给你们听吧。想想你们妈妈当年挖荠莱的情景,你们就会珍爱荠菜,珍爱生活。你们就会懂得什么是幸福,怎样才会得到幸福。
    
\end{normalsize}


\newpage

\textbf{注释}:

\vspace{-1em}

\begin{itemize}
    \setlength\itemsep{-0.2em}
    \item 〔馋〕贪吃,想吃好吃的。
    \item 〔蔷薇〕落叶灌木,茎通常有皮刺。
    \item 〔蜂房〕蜜蜂用蜜蜡做的六角形的巢。
    \item 〔童稚〕幼小,儿童的。
    \item 〔见解〕看法,观点。
    \item 〔和煦〕(风、阳光)温暖的。
    \item 〔患难〕灾祸和苦难。患:灾祸。
    \item 〔薄暮〕傍晚,黄昏。
    \item 〔凶神恶煞〕凶恶的神怪。煞:作恶的鬼怪。
    \item 〔迁就〕委屈自己,曲意将就。
    \item 〔恭顺〕恭敬顺从(长辈)。
    \item 〔隔膜〕观念上有分歧,情感上缺少关连,没有亲密感、亲切感,仿佛隔了一层膜。
\end{itemize}

\chapter{人民英雄永垂不朽}

\begin{normalsize}
    
    从东长安街向天安门广场走去,还未进入广场,就能望见高耸的人民英雄纪念碑。
    
    它如同顶天立地的巨人,屹立在广场南部,和天安门遥遥相对。在远处就可以看到毛主席亲笔题写的“人民英雄永垂不朽”八个金色大字。
    
    花岗石石道直铺到纪念碑宽广的台阶前。沿着台阶可以登上月台。月台有两层,每层用汉白玉\footnote{〔汉白玉〕纯白的大理石,从汉代起,用于宫殿中的阶级护栏。}雕栏围起。纪念碑就在第二层月台中央的大小两层碑座上。
    
    这座纪念碑是根据1949年9月30日中国人民政治协商会议第一届全体会议的决议兴建的。当天傍晚,毛主席率领全体政协委员为纪念碑举行了庄严隆重的奠基礼。毛主席亲自执锨铲土,为纪念碑奠定基石。1958年4月22日,人民英雄纪念碑建成。
    
    这是中国自古以来最大的纪念碑。它从地面到碑顶高达三十七点九四公尺,有十层楼那么高,比纪念碑对面的天安门还高三点二四公尺。
    
    纪念碑是用一万七千块坚硬的花岗石和洁白的汉白玉砌成的。整个纪念碑的造型既保留了传统风格,又有鲜明的新时代精神。碑顶是上有卷云下有重幔的小庑殿顶,这是民族传统的建筑形式。碑身东西两侧上部,刻着以红星、松柏和旗帜组成的装饰花纹,象征着革命精神万年长存。小碑座的四周,雕刻着以牡丹花、荷花、菊花等组成的八个大花圈,这些花朵象征着品质高贵、纯洁,表示全国人民对英雄们的怀念和敬仰。
    
    碑的正面朝北。一块六十吨重、十四点七公尺高的碑心石\footnote{〔碑心石〕即碑身主体,为一整块大理石。}上,雕刻着毛主席题写的“人民英雄永垂不朽”八个镏金大字。这八个字是纪念碑的主题。碑身背面,一行行镏金字整齐地排列着,这是毛主席亲自起草、周总理亲笔写的碑文:
    
    “三年以来在人民解放战争和人民革命中牺牲的人民英雄们永垂不朽。”
    
    “三十年以来在人民解放战争和人民革命中牺牲的人民英雄们永垂不朽。”
    
    “由此上溯到一千八百四十年,从那时起,为了反对内外敌人,争取民族独立和人民自由幸福,在历次斗争中牺牲的人民英雄们永垂不朽。”
    
    十块汉白玉的大浮雕,镶嵌在大碑座的四周。这些大浮雕高二公尺,合在一起共长四十点六八公尺。每幅浮雕里有二十个左右英雄人物,每个人物都和真人一样大小,他们的面貌、性格、表情和姿态形象都不相同。
    
    从东面起,按着历史顺序,浮雕的主题分别是:东面的“虎门销烟”、“太平天国”,南面的“武昌起义”、“五四运动”、“五卅惨案”,西面的“南昌起义”、“敌后抗日”。
    
    北面,也是碑的正面,有三块大浮雕。中央是人民解放军百万雄师“胜利渡长江,解放全中国”的浮雕,这是十块浮雕中最大的一块。两旁还各有一块浮雕。左边是渡江前夕,工农妇女支援前线的场面;右边是全国人民欢迎解放军的情景。
    
    人民英雄纪念碑展现了中国革命的艰难道路。百余年来,中国人民争取自由幸福的斗争中,诞生了无数的英雄。他们带领着中国人民走向了最终的胜利。中国人民的英雄永垂不朽。
    
\end{normalsize}


\newpage

\textbf{注释}:

\vspace{-1em}

\begin{itemize}
    \setlength\itemsep{-0.2em}
    \item 〔屹立〕像山峰一样稳固地直立。
    \item 〔锨〕一种掘土工具。
    \item 〔上溯〕追求根源。溯:逆着水流的方向走。
    \item 〔镏金〕一种镀金手法,把溶解在水银里的金子涂刷在金属表面,再加热让水银蒸发,留下金子。也写作鎏金。
    \item 〔浮雕〕一种雕刻技法。雕刻形象突出周围平坦的表面,仿佛浮在表面上。
    \item 〔前线〕战争中指战斗发生的地方。
    \item 〔永垂不朽〕长久流传,永不磨灭。垂:指记录为文字而流传后世。朽:木头腐烂。
\end{itemize}

\chapter{愚公移山}

\begin{normalsize}
    
    我们开了一个很好的大会。我们做了三件事:第一,决定了党的路线,这就是放手发动群众,壮大人民力量,在我党的领导下,打败日本侵略者,解放全国人民,建立一个新民主主义的中国。第二,通过了新的党章。第三,选举了党的领导机关——中央委员会。今后的任务就是领导全党实现党的路线。我们开了一个胜利的大会,一个团结的大会。代表们对三个报告发表了很好的意见。许多同志作了自我批评,从团结的目标出发,经过自我批评,达到了团结。这次大会是团结的模范,是自我批评的模范,又是党内民主的模范。
    
    大会闭幕以后,很多同志将要回到自己的工作岗位上去,将要分赴各个战场。同志们到各地去,要宣传大会的路线,并经过全党同志向人民作广泛的解释。
    
    我们宣传大会的路线,就是要使全党和全国人民建立起一个信心,即革命一定要胜利。首先要使先锋队觉悟,下定决心,不怕牺牲,排除万难,去争取胜利。但这还不够,还必须使全国广大人民群众觉悟,甘心情愿和我们一起奋斗,去争取胜利。要使全国人民有这样的信心:中国是中国人民的,不是反动派的。中国古代有个寓言,叫做“愚公移山”。说的是古代有一位老人,住在华北,名叫北山愚公。他的家门南面有两座大山挡住他家的出路,一座叫做太行山,一座叫做王屋山。愚公下决心率领他的儿子们要用锄头挖去这两座大山。有个老头子名叫智叟的看了发笑,说是你们这样干未免太愚蠢了,你们父子数人要挖掉这样两座大山是完全不可能的。愚公回答说:我死了以后有我的儿子,儿子死了,又有孙子,子子孙孙是没有穷尽的。这两座山虽然很高,却是不会再增高了,挖一点就会少一点,为什么挖不平呢?愚公批驳了智叟的错误思想,毫不动摇,每天挖山不止。这件事感动了上天,他就派了两个神仙下凡,把两座山背走了。现在也有两座压在中国人民头上的大山,一座叫做帝国主义,一座叫做封建主义。中国共产党早就下了决心,要挖掉这两座山。我们一定要坚持下去,一定要不断地工作,我们也会感动上帝的。这个上帝不是别人,就是全中国的人民大众。全国人民大众一齐起来和我们一道挖这两座山,有什么挖不平呢?
    
    昨天有两个美国人要回美国去,我对他们讲了,美国政府要破坏我们,这是不允许的。我们反对美国政府扶蒋反共的政策。但是我们第一要把美国人民和他们的政府相区别,第二要把美国政府中决定政策的人们和下面的普通工作人员相区别。我对这两个美国人说:告诉你们美国政府中决定政策的人们,我们解放区\footnote{〔解放区〕建立了人民政权的地区。}禁止你们到那里去,因为你们的政策是扶蒋反共,我们不放心。假如你们是为了打日本,要到解放区是可以去的,但要订一个条约。倘若你们偷偷摸摸到处乱跑,那是不许可的。赫尔利\footnote{〔赫尔利〕指帕特里克·赫尔利,美国外交家。1944年11月任美国驻华大使,试图斡旋国共关系。1945年8月陪同毛泽东赴重庆谈判。1945年11月辞职返美。}已经公开宣言不同中国共产党合作,既然如此,为什么还要到我们解放区去乱跑呢?
    
    美国政府的扶蒋反共政策,说明了美国反动派的猖狂。但是一切中外反动派的阻止中国人民胜利的企图,都是注定要失败的。现在的世界,民主是主流,反民主的反动只是一股逆流。目前反动的逆流企图压倒民族独立和人民民主的主流,但反动的逆流终究不会变为主流。现在依然如斯大林很早就说过的一样,旧世界有三个大矛盾:第一个是帝国主义国家中的无产阶级和资产阶级的矛盾,第二个是帝国主义国家之间的矛盾,第三个是殖民地半殖民地国家和帝国主义宗主国之间的矛盾。这三种矛盾不但依然存在,而且发展得更尖锐了,更扩大了。由于这些矛盾的存在和发展,所以虽有反苏反共反民主的逆流存在,但是这种反动逆流总有一天会要被克服下去。
    
    现在中国正在开着两个大会,一个是国民党的第六次代表大会,一个是共产党的第七次代表大会。两个大会有完全不同的目的:一个要消灭共产党和中国民主势力,把中国引向黑暗;一个要打倒日本帝国主义和它的走狗中国封建势力,建设一个新民主主义的中国,把中国引向光明。这两条路线在互相斗争着。我们坚决相信,中国人民将要在中国共产党领导之下,在中国共产党第七次大会的路线的领导之下,得到完全的胜利,而国民党的反革命路线必然要失败。
    
    \hfill 一九四五年六月十一日
    
\end{normalsize}



\chapter{雄伟的人民大会堂}

\begin{normalsize}
    
    天安门广场西边,巍然耸立着一座雄伟壮丽的大厦,这就是人民大会堂。全国各族人民的代表在这里共商国是。
    
    庄严的人民大会堂,是首都最宏伟的建筑之一,建筑面积达十七万一千八百平方米,体积有一百五十九万六千九百立方米。一条黄绿相间的琉璃屋檐,把巍峨的大会堂的轮廓从蔚蓝的天空中勾画出来。那壮丽的柱廊,淡雅的色调,以及四周层次繁多的建筑立面\footnote{〔立面〕建筑与外部环境直接接触的界面,及其展现方式。},组成了一副庄严绚丽的图画。
    
    我们在建筑师的陪同下,参观了人民大会堂。在天安门广场,远远就能看见正门顶上的国徽的闪闪金光。踏上一层楼高的花岗石大台阶,迎面是十二根浅灰色的大理石门柱。门柱有二十五米高,柱身要四个人才能合抱。柱距采用我国柱廊的传统样式,明间\footnote{〔明间〕建筑中指居中四柱围成的矩形空间。两旁的柱间空间称为次间。}最宽,紧邻左右的次间较窄,再往两旁的又更窄。这样高大而有力的柱廊,是建筑师吸收了中外古今门柱造型的优点创造出来的。
    
    迈进金色大铜门,穿过宽阔的风门厅\footnote{〔风门厅〕为挡风而加设的隔厅。}和衣帽厅\footnote{〔衣帽厅〕为换衣帽而加设的隔厅。},就到了大会堂建筑的枢纽部分——中央大厅。建筑师站在这里,指着四周向我们介绍了整个建筑的布局:朝西直入万人大礼堂;往北通宴会厅;向南穿过长长的廊道,是全国人民代表大会常务委员会的办公大楼。整个建筑就是由这三部分组成的。
    
    万人大礼堂,里面宽七十六米,深六十米,中部高三十三米,体积达八万六千立方米,几乎能容下一座大楼。但是由于设计师处理得巧妙,走进大礼堂的人放眼一看,从屋顶到地面,上下浑然一体,并不感到怎样空旷。穹窿形的天花\footnote{〔天花〕屋顶梁架下、室内顶上的层面。室内顶饰的一种。}上纵横排着近五百个灯孔。灯火齐明的时候,就像满天星斗。中心的大灯是红宝石般的五角星,周围是七十条瑰丽的光芒线和四十瓣镏金的向日葵花瓣,象征着全国各族人民万众一心,紧密团结在中国共产党的周围。再往外是三环层次分明的水波形暗灯槽\footnote{〔暗灯槽〕使灯光不直射,间接照明的工具。},同周围装贴的淡青色塑料板相映,形成“水天一色”的奇观。
    
    大礼堂是椭圆形\footnote{〔椭圆形〕扁圆形,类似蛋形。}的,有两层挑台\footnote{〔挑台〕从墙壁伸出的悬空台。挑:支起横的东西。},像两弯新月,围拱着主席台,层次分明,错落有致。两层挑台连地面共三层座席,有九千六百多个席位。主席台像个小会场,能容纳三百多人。底层席位的桌柜上都装有接收同声传译\footnote{〔同声传译〕在不打断讲话者的情况下,不间断地将内容同时翻译后讲给听众的一种翻译方式。听众同时听到讲话者的内容和翻译后的内容。}的耳机,每四个席位还有一个即席发言的话筒。下层挑台最前排也装有话筒,其余席位都有扬声喇叭。屋顶和挑台下的灯光,能够把礼堂的每个角落照得通明。
    
    尽管上下三层席位高低差距很大,底层面积达三千多平方米,最远处距离主席台有六十米,但是高大的礼堂中间没有一根柱子。建筑师画了一张草图,告诉我们,大礼堂顶上藏着比北京新扩建的长安街路面还要宽的十二榀\footnote{〔榀〕屋架的单位。}钢屋架\footnote{〔屋架〕用于承托屋顶的架构。多用于内部空间较大的建筑。}。其中六榀,一端压在一个九米高的钢筋混凝土横梁上,所有这些重量又一起压在主席台台口的两根柱子上,每根柱子都能承受三千多吨的重量。这样庞大而复杂的结构,该是一项多么艰巨的工程啊!说到这里,建筑师极力推崇建筑工人的伟大智慧和创造力。是他们在短短九个月的时间内,完成了这复杂的工程,还安装了声、电、冷热风、电视转播等各种现代化设备。
    
    人民大会堂的北翼是宴会厅,面临长安街。从人民大会堂北门进去,穿过大理石柱廊、风门厅、衣帽厅,就进入宴会厅底层大厅。这是宴前休息的场所。往前走,是五组六十二级的汉白玉大台阶,迎面墙壁上镶嵌着以毛主席《沁园春·雪》为主题的巨幅国画。画的一边是一片白茫茫的江山,“山舞银蛇,原驰蜡象”;另一边,云海茫茫中旭日东升,照耀大地,显得“江山如此多娇”。从这里经过东西两侧的走马廊\footnote{〔走马廊〕建筑物外围的宽阔走廊。},就进入宴会厅。
    
    有五千个席位的宴会厅,又是另一番景象。它的面积有七千平方米,比一个足球场还大,设计的精巧也是罕见的。天花和回廊\footnote{〔回廊〕环回的走廊。}圆柱装饰精美,雍容典雅。大厅的高度只有十五米多,但天花上运用了方井\footnote{〔方井〕一种室内顶饰手法。房顶四周向中央层层上凹,如倒过来的井,也叫藻井、天井、斗八。}的手法,显得明朗宽敞。
    
    建筑师还领我们参观了设置在大厅北面东西两角的厨房。厨房直通大厅两侧的回廊,开宴的时候,服务员可以从廊道进出宴席之间。厨房里的设备都是现代化的,上部厨房与地下室冷藏间和食品加工间等,都有专用电梯和楼梯上下运输。生冷和熟食,未洗和洗净的餐具,各有专线输送。
    
    人民大会堂的南翼是人大常委会办公楼。这是一座口字形的大楼,中间有六千平方米的庭院,里面一片草坪,是理想的集体摄影场地,也是幽静的休息场所。从这庭院穿过一座拱形的洞门,就到了人民大会堂的外面。
    
    看完这座大厦,一整天已经过去了。走出人民大会堂,只见万道霞光洒在苍翠的树丛上,洒在杏黄色的墙壁上,洒在天安门的红墙黄瓦上,溅出灿烂的异彩。
    
\end{normalsize}


\newpage

\textbf{注释}:

\vspace{-1em}

\begin{itemize}
    \setlength\itemsep{-0.2em}
    \item 〔大厦〕高大的房屋。
    \item 〔巍然〕高大的样子。
    \item 〔耸立〕高高地直立,矗立。
    \item 〔共商国是〕共同商议国家大计。国是:国家的重大问题。
    \item 〔枢纽〕枢:门户的转轴。纽:器物上用来提系的部分。比喻事物的关键部分,连起各方面的中心。
    \item 〔即席〕在座位上,不用离开座位。
    \item 〔雍容〕从容大方。
    \item 〔浑然一体〕完全融合,仿佛本就同属一个身体,不可分开。
    \item 〔推崇〕推重崇敬。推:指出优点,使成为优先考虑的对象。
    \item 〔穹窿〕中间高、四周下垂的形状。
    \item 〔错落有致〕交错纷杂而有美感、有趣。
    \item 〔溅〕液体受冲击向四周飞射。
\end{itemize}

\chapter{中国石拱桥}

\begin{normalsize}
    
    石拱桥的桥洞成弧形,就像虹。古代神话里说,雨后彩虹是“人间天上的桥”,通过彩虹就能上天。我国的诗人爱把拱桥比作虹,说拱桥是“卧虹”“飞虹”,把水上拱桥形容为“长虹卧波”。
    
    石拱桥在世界桥梁史上出现得比较早。这种桥不但形式优美,而且结构坚固,能几十年几百年甚至上千年雄跨在江河之上,在交通方面发挥作用。
    
    我国的石拱桥有悠久的历史。《水经注》里提到的“旅人桥”,大约建成于公元282年,可能是有记载的最早的石拱桥了。我国的石拱桥几乎到处都有。这些桥大小不一,形式多样,有许多是惊人的杰作。其中最著名的当推河北省赵县的赵州桥,还有北京丰台区的卢沟桥。
    
    赵州桥横跨在洨河上,是世界著名的古代石拱桥,也是造成后一直使用的最古的石桥。这座桥修建于公元605年左右,到1962年已经1300多年了,还保持着原来的雄姿。到解放的时候,桥身有些残损了,在人民政府的领导下,经过彻底整修,这座古桥又恢复了青春。
    
    赵州桥非常雄伟,全长50.82米,两端宽9.6米,中部略窄,宽约9米。桥的设计完全合乎科学原理,施工技术更是巧妙绝伦。唐朝的张嘉贞说它“制造奇特,人不知其所以为”。这座桥的特点是:一、全桥只有一个大拱,长达37.4米,在当时可算是世界上最长的石拱。桥洞不是普通半圆形,而是像一张弓,因而大拱上面的道路没有陡坡,便于车马上下。二、大拱的两肩上,各有两个小拱。这是创造性的设计,不但节约了石料,减轻了桥身的重量,而且在河水暴涨的时候,还可以增加桥洞的过水量,减轻洪水对桥身的冲击。同时,拱上加拱,桥身也更美观。三、大拱由28道拱圈拼成,就像这么多同样形状的弓合拢在一起,作成了一个弧形的桥洞。每道拱圈都能独立支撑上面的重量,一道坏了,其他各道不致受到影响。四、全桥结构匀称,和四周景色配合得十分和谐;桥上的石栏石板也雕刻得古朴美观。唐朝的张𬸦说,远望这座桥就像“初月出云,长虹饮涧”。赵州桥高度的技术水平和不朽的艺术价值,充分显示出了我国劳动人民的智慧和力量。桥的主要设计者李春就是一位杰出的工匠,在桥头的碑文里刻着他的名字。
    
    永定河上的卢沟桥,修建于公元1189到1192年间。桥长265米,由11个半圆形的石拱组成,每个石拱长度不一,自16米到21.6米。桥宽约8米,桥面平坦,几乎与河面平行。每两个石拱之间有石砌桥墩,把11个石拱联成一个整体。由于各拱相联,所以这种桥叫做联拱石桥。永定河发水时,来势很猛,以前两岸河堤常被冲毁,但是这座桥极少出事,足见它的坚固。桥面用石板铺砌,两旁有石栏石柱。每个柱头上都雕刻着不同姿态的狮子。这些石刻狮子,有的母子相抱,有的交头接耳,有的像倾听水声,有的像注视行人,千态万状,惟妙惟肖。
    
    卢沟桥地处入都要道,而且建筑优美,历来为人们所称赞。“卢沟晓月”很早就成为北京的胜景之一。
    
    卢沟桥在我国人民反抗帝国主义侵略战争的历史上,也是值得纪念的。1937年7月7日中国军队在此抗击日本帝国主义的侵略,揭开了中国人民全面抗战的序幕。
    
    为什么我国的石拱桥会有这样光辉的成就呢?首先,在于我国劳动人民的勤劳和智慧。他们制作石料的工艺极其精巧,能把石料切成整块大石碑,又能把石块雕刻成各种形象。在建筑技术上有很多创造,在起重吊装方面更有意想不到的办法。如福建漳州的江东桥,修建于700多年前,有的石梁一块就有200来吨重,究竟是怎样安装上去的,还不完全知道。其次,我国石拱桥的设计有优良传统,建成的桥,用料省,结构巧,强度高。再其次,我国富有建筑用的各种石料,便于就地取材,这也为修造石桥提供了有利条件。
    
    两千年来,我国修建了无数的石拱桥。解放后,全国大规模兴建起各种形式的公路桥和铁路桥。其中就有不少石拱桥。1961年,云南省建成了一座世界最长的独拱石桥,名叫“长虹大桥”,石拱长达112.5米。在传统的石拱桥的基础上,我们还造了大量的钢筋混凝土拱桥,其中“双曲拱桥”是我国劳动人民的新创造,是世界上所仅有的。这几年来,全国造了总长二十余万米的这种拱桥,其中最大的一孔,长达150米。我国桥梁事业的飞跃发展,表明了我国社会主义制度的无比优越。
    
\end{normalsize}



\chapter{漫谈无理数}

\begin{normalsize}
    
    传说在两千多年前,古希腊\footnote{〔古希腊〕公元前欧洲巴尔干半岛南部、爱琴海诸岛和小亚细亚沿岸的一些城邦的合称。}有个智者叫毕达哥拉斯。他对数的研究很深,认为世间万物都能用数来解释。这里的数,指的是1、2、3……这样的整数。他用数与数的比例解释了很多自然现象。他的弟子和崇拜者创建了以他为名的教派\footnote{〔教派〕这里指因信仰形成的团体。},提出“万物皆数”,万物按照数和数的比例构成完美和谐的秩序。
    
    有一天,教派中的一名弟子希帕索斯思考了一个问题:已知一个正方形,如何画一个面积是它两倍的正方形?这个问题并不难。把已知正方形的对角线作为边长的正方形,面积就是它的两倍。希帕索斯进一步思考:这个新正方形的边长是多少呢?
    
    如果把原来的正方形的边长记作1,希帕索斯要找的就是平方等于2的数。按毕达哥拉斯的思想,这个数必定是整数或整数之比。但是,希帕索斯证明了:这个数无法表示成任何两个整数的比!
    
    这个结论让毕达哥拉斯教派大为恐慌。他们怕希帕索斯泄露这个秘密,就把他沉到海里杀害了。
    
    远古的传说已经难以考证,但是,发现“无法表示成整数之比的数”,确实是数学史的一个里程碑。它说明了,仅仅用整数的加减乘除,无法处理现实中图形的问题。要计算关于三角形、长方形的问题,数学家就必须引入一种新的数。毕达哥拉斯教派认为数是万物的道理,能用整数之比表示的数叫做有理数,于是“平方等于2”这样不能表示为整数之比的数就被称为无理数。
    
    无理数多吗?我们把“平方等于整数的数”称为整数的平方根。如果仅仅考虑整数的平方根,我们可能会觉得无理数不算多。比如,“平方等于4的数”就是2。也就是说,不是所有整数都会生出无理数。这样看来,无理数似乎比有理数少。
    
    然而事实并非如此。首先,无理数可不仅仅包括“平方等于2”这样的数。我们还有“立方等于2的数”、“5次方等于2的数”等等。我们把“乘方\footnote{〔乘方〕数自乘的结果。}等于整数的数”称为方根。除了平方根,还有立方根、4次方根等等。比如4的立方根就是“立方等于4的数”,5的4次方根就是“4次方等于5的数”,等等。这些方根几乎都是无理数。
    
    此外,还有不是方根的无理数。圆的周长与直径之比叫做圆周率,它也是无理数。而且,圆周率的平方、立方、4次方……都不是有理数。19世纪的数学家康托\footnote{〔康托〕戈奥格·康托,19世纪德国数学家,现代集合论创立者。}严格证明了,无理数比有理数“多得多”。
    
    和有理数相比,无理数不仅多,而且杂乱无章,难以理解。方根是最容易理解的无理数。它和自身相乘,就能得到整数。又比如著名的黄金分割率\footnote{〔黄金分割〕把整体分为大小两部分,使整体与较大部分之比等于较大部分与较小部分之比。其中的比率称为黄金分割率或黄金比率。},它的平方减去自己是1。能够通过和自身及整数的加减乘除变成整数,这样的无理数称为代数数。换个角度\footnote{〔角度〕这里指看问题的出发点。}来说,代数数就是整数系数的整式方程的解。
    
    然而,无理数可不全是代数数。比如,圆周率就不是任何整数系数的整式方程的解。换句话说,把圆周率和整数一起做加减乘除,无论做多少次,都无法变成整数。相比2的方根或黄金分割率,圆周率这样的无理数更加难以用整数描述,不可捉摸。我们把这样的无理数称为超越数。
    
    超越数多吗?数学家遗憾地告诉我们,比起超越数,代数数和有理数的数量都可以“忽略不计”。无理数几乎都是超越数。
    
    直到21世纪,大多数数学家对无理数的研究,仍然停留在代数数上。对于超越数,我们只了解其中极少的一部分。
    
    借助无限的概念,我们能够理解关于一些超越数的性质。比如,借助极限\footnote{〔极限〕数学中指无限逼近的结果。}的概念,我们可以定义所有正整数的平方倒数的和\footnote{〔平方倒数〕指平方的倒数,比如2的平方倒数为四分之一。},并且证明它等于圆周率平方的六分之一。
    
    不仅如此,所有正整数的偶数次方倒数的和,都等于圆周率的偶数次方乘以某个有理数。但是,关于所有正整数的奇数次方倒数的和,我们所知甚少。
    
    1978年,法国数学家阿佩里证明了:所有正整数的立方倒数的和是无理数。它是否是超越数呢?这仍是个未解之谜。有人猜想,它等于圆周率的立方乘以某个有理数,但还没人能够证明。
    
    很多时候,我们竟然无法知道,数学研究中发现的数,是否是无理数。比如著名的欧拉常数,它是调和级数与自然对数的差的极限\footnote{〔……差的极限〕调和级数:正整数倒数的和。自然对数:自然科学中的一种基本函数。欧拉常数指前$n$个正整数的倒数的和减去$n$的自然对数的差在$n$逼近无限大时的极限。},大约等于0.577。我们至今不知道它是否是无理数。更有一种叫做“不可定义数”的,它们无法用有限的文字确定,更不用说进行研究了。
    
    直到今天,无理数中仍然埋藏着许多秘密,等着我们去发现。
    
\end{normalsize}


\newpage

\textbf{注释}:

\vspace{-1em}

\begin{itemize}
    \setlength\itemsep{-0.2em}
    \item 〔智者〕聪明的人,有智慧的人。
    \item 〔和谐〕多个声音合成好听的声音。指配合恰当。
    \item 〔秩序〕各个部分有条理有先后的状态和规则。
    \item 〔里程碑〕设置在路旁记录里数的标志。比喻历史发展过程中的重大事件。
    \item 〔杂乱无章〕无条理,无规律。
    \item 〔证明〕用确实的证据和推理说明。
    \item 〔遗憾〕这里指因为无法改变的不足而可惜。
\end{itemize}

\chapter{向沙漠进军}

\begin{normalsize}
    
    沙漠是人类最顽强的自然敌人之一。有史以来,人类就同沙漠不断地斗争。但是从古代的传说和史书的记载看来,过去人类没有能征服沙漠,若干住人的地区反而为沙漠所并吞。
    
    地中海沿岸被称为西方文明的摇篮。古代埃及、巴比伦和希腊\footnote{〔古代埃及……〕亚欧非交界地带的古代文明,现称古埃及、古巴比伦、古希腊文明。}的文明都是在这里产生和发展起来的。但是两三千年来,这个区域不断受到风沙的侵占,有些部分逐渐变成荒漠了。
    
    中国陕西榆林地区,雨量还充沛,在明末清初的时候是个天然草原区,没有多少风沙。到了清朝乾隆年间,陕西和山西北部许多人移居到榆林以北关外去开垦。当时的政府根本不关心农业生产事业,生产技术又不高,垦荒伐木,致使原来的草地露出了泥土,日晒风吹,尘沙就到处飞扬。由于长城外的风沙侵入,榆林城也受袭击,到解放以前,榆林地区关外30公里都变成沙漠了。
    
    沙漠逞强施威,所用的武器是风和沙。风沙的进攻主要有两种方式。一种可以称为“游击战”。狂风一起,沙粒随风飞扬,风愈大,沙的打击力愈强。春天四五月间禾苗刚出土,正是狂风肆虐的时候。一次大风沙袭击,可以把幼苗全部打死,甚至连根拔起。沿长城一带风沙大的地区,农民常常要补种两三次才能有点收获。一种可以称为“阵地战”,就是风推动沙丘,缓缓前进。沙丘的高度一般从几米到几十米,也有高达100米以上的。沙丘的前进并不是整体移动的。当风速达到每秒5米以上的时候,沙丘迎风面的沙粒就成批地随风移动,从沙丘的底部移到顶部,过了顶部,由于风速减弱,就在背风面的坡上落下。所以部分沙粒的移动速度虽然相当快,每天可以移动几米到几十米,可是整个沙丘波浪式地前进,移动速度并不快,每年不过5米到10米。几个沙丘常常联在一起,成为沙丘链。沙丘的移动虽然慢,可是所到之处,森林全被摧毁,田园全被埋葬,城郭变成丘墟。
    
    抵御风沙袭击的方法是培植防护林。防护林的主要作用是减小风的力量。风遇到防护林,速度就减小70\%~80\%。到距离防护林等于林木高度20倍的地方,风又恢复原来的速度。所以防护林必须是并行排列的许多林带,两列之间的距离不要超过林木高度的20倍。其次是培植草皮。有了草皮覆盖地面,即使有风,刮起的沙也不多,这就减少了沙粒的来源。
    
    抵御沙丘进攻的方法是植树种草。中国沙荒地区,有一部分沙丘已经长了草皮和灌木,不再转移阵地了。这种固定的沙丘,只要能妥善保护草皮和灌木,防止过度砍伐和任意放牧,就可以固定下来。根据近年治沙的经验,陕北榆林、内蒙古磴口、甘肃民勤地区的流动沙丘,表面干沙层的厚度一般不超过10厘米。10厘米以下,水分含量逐渐增大,到40厘米的深处,水分含量达到2\%以上,这就是湿沙层了。湿沙层的水分足够供应固定沙丘的植物的需要。所以在流动沙丘上植树种草,是可以成活的。林木和草类成长以后,沙丘就可以固定下来了。
    
    仅仅防御风沙袭击,固定沙丘阵地,还只是采取守势,自然是不够的。征服沙漠的最主要的武器是水。无论植树还是种草,土壤中必须有充足的水分。所以要取得向沙漠进军的胜利,必须有充足的水源。
    
    中国内蒙古东部和陕西、山西北部有足够的雨量。就是西北干旱地区,地面径流\footnote{〔径流〕自然降水沿重力沿地面或地下流动的水流。}和地下潜水也是很大的。有些沙荒地区,如河西走廊\footnote{〔河西走廊〕甘肃省西北部祁连山以北,合黎山、龙首山以南,乌鞘岭以西一带。自古为通往新疆的要道。}、柴达木、新疆北部准噶尔和新疆南部塔里木,都是盆地,周围的高山上有大量的积雪。这样看来,只要能充分利用这些水源,我们向沙漠进军不但有收复失地的把握,而且能在大沙漠里开辟出若干绿洲来。普通河流愈到下游,水量愈多,河流愈大。但在沙漠中,一部分水被蒸发到空中,一部分浸入到土壤岩隙中成为地下水,河流反而愈流愈小,终至于干涸不见。如地质构造\footnote{〔地质构造〕地壳岩石的构成方式、各部分形态及面貌特征。}是一个盆地,则能汇成地下海,可以作为建立绿洲的水源。据中国科学院综合考察委员会的调查,只要有水源,单新疆尚有一亿亩荒地可以开垦。
    
    沙漠是可以治理的。中国在治理沙漠方面已经取得了若干成绩。新疆生产建设兵团在天山\footnote{〔天山〕新疆西北部山脉。}南北建立国营农场,开沟挖渠,种麦种棉植树,那里原是不毛之地,现在一片葱茏,俨然成为绿洲。内蒙古沙荒区的治沙工作也获得不少成绩。
    
    我们向沙漠进军,不但保护了农田,开辟了绿洲,而且对交通线路也起了防护作用。包兰铁路从银川到兰州的一段,要经过腾格里沙漠,其间中卫县沙坡头一带,风沙特别厉害。那里沙多风大,一次大风沙就可以把铁路淹没。有关部门\footnote{〔有关部门〕指中国科学院。}在1956年成立了沙坡头治沙站,进行固沙造林。这一工作已经提前完成。包兰铁路通车以来,火车在沙漠上行驶,从来没有因为风沙的侵袭而发生事故。
    
    风是沙漠向人类进攻的武器,但是也可以为人类造福。沙漠地区地势平坦,风力很强。如新疆的星星峡、托克逊、达坂城都是著名的风口。中国科学院力学研究所在托克逊地方试制了半径两米的风力车,可以供发电、汲水、磨面之用。
    
    沙漠地区空气干燥,日光的照射特别强烈。那里日照时间又特别长,一年达到3000小时,而长江流域只有1500小时,华北地区也不过2500小时。日光可以用来发电,取暖,煮水,做饭。沙漠湖水含盐,日光使水蒸发,可以取得蒸馏水\footnote{〔蒸馏〕加热液体使变成蒸气,,再使蒸气冷却凝成液体,从而除去其中的杂质。}和盐。把日光变为热能和电能的最良好的工具是半导体\footnote{〔半导体〕导电能力介于导体和绝缘体之间的材料。},估计将来有可能在沙漠里用便宜的半导体做屋顶,人住在里边冬天不冷,夏天不热。
    
    从上面介绍的一些情况,可以清楚地认识到,只要我们正确地认识沙漠的危害,找出对付它的办法,沙漠是有可能治理的。
    
\end{normalsize}


\newpage

\textbf{注释}:

\vspace{-1em}

\begin{itemize}
    \setlength\itemsep{-0.2em}
    \item 〔肆虐〕任意干残暴的事情。
    \item 〔治理〕整理、处理好。治:消除水灾。
    \item 〔不毛之地〕长不出作物的荒地。
    \item 〔丘墟〕废墟。
    \item 〔抵御〕挡,抗,防。
    \item 〔地势〕地面高低起伏的情况。
    \item 〔开垦〕把荒地开辟成可以种植的土地。垦:翻土。
    \item 〔干涸〕河流池塘没水。
    \item 〔汲水〕从下往上取水。
    \item 〔葱茏〕植物丛聚茂盛的样子。
    \item 〔俨然〕很像。
\end{itemize}

\chapter{白杨礼赞}

\begin{normalsize}
    
    白杨树实在不是平凡的,我赞美白杨树!
    
    当汽车在望不到边际的高原上奔驰,扑入你的视野的,是黄绿错综的一条大毯子;黄的,是土,未开垦的荒地,几十万年前由伟大的自然力堆积而成的黄土高原的外壳;绿的呢,是人类劳力战胜自然的成果,是麦田,和风吹送,翻起了一轮一轮的绿波——这时你会真心佩服昔人所造的两个字“麦浪”,若不是妙手偶得,便确是经过锤炼的语言的精华。黄与绿主宰着,无边无垠,坦荡如砥,这时如果不是宛若并肩的远山的连峰提醒了你,你会忘记了汽车是在高原上行驶,这时你涌起来的感想也许是“雄壮”,也许是“伟大”,诸如此类的形容词,然而同时你的眼睛也许觉得有点倦怠,你对当前的“雄壮”或“伟大”闭了眼,而另一种味儿在你心头潜滋暗长了—— “单调”。可不是,单调,有一点儿吧?
    
    然而刹那间,要是你猛抬眼看见了前面远远地有一排,——不,或者甚至只是三五株,一二株,傲然地耸立,象哨兵似的树木的话,那你的恹恹欲睡的情绪又将如何?我那时是惊奇地叫了一声的!
    
    那就是白杨树,西北极普通的一种树,然而实在是不平凡的一种树!
    
    那是力争上游的一种树,笔直的干,笔直的枝。它的干,通常是丈把高,像加以人工似的,一丈以内,绝无旁枝;它所有的丫枝一律向上,而且紧紧靠拢,也像加以人工似的,成为一束,绝不旁逸斜出;它的宽大的叶子也是片片向上,几乎没有斜生的,更不用说倒垂了;它的皮光滑而有银色的晕圈,微微泛出淡青色。这是虽在北方风雪的压迫下却保持着倔强挺立的一种树。哪怕只有碗那样粗细,它却努力向上发展,高到丈许,二丈,参天耸立,不折不挠,对抗着西北风。
    
    这就是白杨树,西北极普通的一种树,然而决不是平凡的树!
    
    它没有婆娑的姿态,没有屈曲盘旋的虬枝,也许你要说它不美,如果美是专指“婆娑”或“旁斜逸出”之类而言,那么,白杨树算不得树中的好女子;但是它伟岸,正直,朴质,严肃,也不缺乏温和,更不用提它的坚强不屈与挺拔,它是树中的伟丈夫!当你在积雪初融的高原上走过,看见平坦的大地上傲然挺立这么一株或一排白杨树,难道你觉得树只是树?难道你就不想到它的朴质,严肃,坚强不屈,至少也象征了北方的农民?难道你竟一点也不联想到,在敌后的广大土地上,到处有坚强不屈,就象这白杨树一样傲然挺立的守卫他们家乡的哨兵?难道你又不更远一点想到这样枝枝叶叶靠紧团结,力求上进的白杨树,宛然象征了今天在华北平原纵横决荡用血写出新中国历史的那种精神和意志?
    
    白杨不是平凡的树。它在西北极普遍,不被人重视,就跟北方农民相似;它有极强的生命力,磨折不了,压迫不倒,也跟北方的农民相似。我赞美白杨树,就因为它不但象征了北方的农民,尤其象征了今天我们民族解放斗争中所不可缺的朴质,坚强,力求上进的精神。
    
    让那些看不起民众,贱视民众,顽固的倒退的人们去赞美那贵族化的楠木,去鄙视这极常见,极易生长的白杨吧,我要高声赞美白杨树!
    
\end{normalsize}


\newpage

\textbf{注释}:

\vspace{-1em}

\begin{itemize}
    \setlength\itemsep{-0.2em}
    \item 〔边际〕边界(多指地区和空间)。
    \item 〔视野〕眼睛看到的空间范围;眼界。
    \item 〔错综〕纵横交错。
    \item 〔开垦〕把荒地开辟成可以种植的土地。
    \item 〔妙手偶得〕指文学素养深的人偶然间所得到的。出自陆游《文章》:“文章本天成,妙手偶得之。”妙手:技艺高超的人 。
    \item 〔精华〕最重要、最好的部分。
    \item 〔主宰〕支配,有决定权。
    \item 〔诸如此类〕与此相似的种种事物。
    \item 〔力争上游〕努力奋斗,争取先进。
    \item 〔旁逸斜出〕意思是(树枝)从树干的旁边斜伸出来。逸:逃。
    \item 〔姿态〕姿势,样儿。还可指态度、气度。
    \item 〔屈曲〕弯曲、曲折的意思。
    \item 〔盘旋〕环绕着飞或走。
    \item 〔伟岸〕高大挺拔。
    \item 〔朴质〕质朴。
    \item 〔温和〕(性情)温柔平和。
    \item 〔坚强不屈〕坚毅刚强,不屈服。
    \item 〔宛然〕仿佛,好像。
    \item 〔纵横决荡〕纵横四方,冲杀突击。
    \item 〔磨折〕磨难,挫折。
    \item 〔贱视〕轻视。
    \item 〔秀颀〕美而高。
    \item 〔鄙视〕轻视,看不起。
    \item 〔坦荡如砥〕平坦得像磨刀石磨过的样子。
    \item 〔恹恹〕精神不好,困倦的样子。
\end{itemize}

\chapter{苏州园林}

\begin{normalsize}
    
    苏州园林据说有一百多处,我到过的不过十多处。其他地方的园林我也到过一些。倘若要我说说总的印象,我觉得苏州园林是我国各地园林的标本,各地园林或多或少都受到苏州园林的影响。因此,谁如果要鉴赏我国的园林,苏州园林就不该错过。
    
    设计者和匠师们因地制宜,自出心裁,修建成功的园林当然各个不同。可是苏州各个园林在不同之中有个共同点,似乎设计者和匠师们一致追求的是:务必使游览者无论站在哪个点上,眼前总是一幅完美的图画。为了达到这个目的,他们讲究亭台轩榭\footnote{〔亭台轩榭〕泛指园林建筑。亭:有顶无墙的小屋。台:供登高望远的平楼。轩:敞窗的小屋或长廊。榭:建在土台或水上的木屋。}的布局,讲究假山\footnote{〔假山〕园林中以造景为目的,用土、石等材料构筑的山。}池沼的配合,讲究花草树木的映衬,讲究近景远景的层次。总之,一切都要为构成完美的图画而存在,决不容许有欠美伤美的败笔。他们惟愿游览者得到“如在画图中”的美感,而他们的成绩实现了他们的愿望,游览者来到园里,没有一个不心里想着口头说着“如在画图中”的。
    
    我国的建筑,从古代的宫殿到近代的一般住房,绝大部分是对称的,左边怎么样,右边也怎么样。苏州园林可绝不讲究对称,好像故意避免似的。东边有了一个亭子或者一道回廊,西边决不会来一个同样的亭子或者一道同样的回廊。这是为什么?我想,用图画来比方,对称的建筑是图案画,不是美术画,而园林是美术画,美术画要求自然之趣,是不讲究对称的。
    
    苏州园林里都有假山和池沼。假山的堆叠,可以说是一项艺术而不仅是技术。或者是重峦叠嶂,或者是几座小山配合着竹子花木,全在乎设计者和匠师们生平多阅历,胸中有丘壑,才能使游览者攀登的时候忘却苏州城市,只觉得身在山间。至于池沼,大多引用活水\footnote{〔活水〕有源头长流不断的水。}。有些园林池沼宽敞,就把池沼作为全园的中心,其他景物配合着布置。水面假如成河道模样,往往安排桥梁。假如安排两座以上的桥梁,那就一座一个样,决不雷同。池沼或河道的边沿很少砌齐整的石岸,总是高低屈曲任其自然。还在那儿布置几块玲珑的石头,或者种些花草:这也是为了取得从各个角度看都成一幅画的效果。池沼里养着金鱼或各色鲤鱼,夏秋季节荷花或睡莲开放,游览者看“鱼戏莲叶间”,又是入画的一景。
    
    苏州园林栽种和修剪树木也着眼在画意。高树与低树俯仰生姿。落叶树与常绿树相间,花时不同的多种花树相间,这就一年四季不感到寂寞。没有修剪得像宝塔那样的松柏,没有阅兵式似的道旁树:因为依据中国画的审美观点看,这是不足取的。有几个园里有古老的藤萝,盘曲嶙峋的枝干就是一幅好画。开花的时候满眼的珠光宝气,使游览者感到无限的繁华和欢悦,可是没法说出来。
    
    游览苏州园林必然会注意到花墙和廊子。有墙壁隔着,有廊子界着,层次多了,景致就见得深了\footnote{〔深〕景色深,指景观有层次,移步换景,不会一览无余。}。可是墙壁上有砖砌的各式镂空图案,廊子大多是两边无所依傍的,实际是隔而不隔,界而未界,因而更增加了景致的深度。有几个园林还在适当的位置装上一面大镜子,层次就更多了,几乎可以说把整个园林翻了一番。
    
    游览者必然也不会忽略另外一点,就是苏州园林在每一个角落都注意图画美。阶砌旁边栽几丛书带草\footnote{〔书带草〕多年生草本植物,多见于沟旁及山坡草丛,庭园绿化植物。}。墙上蔓延着爬山虎或者蔷薇木香\footnote{〔蔷薇木〕也叫红木,常用来做家具的名贵木材。}。如果开窗正对着白色墙壁,太单调了,给补上几竿竹子或几棵芭蕉。诸如此类,无非要游览者即使就极小范围的局部看,也能得到美的享受。
    
    苏州园林里的门和窗,图案设计和雕镂琢磨功夫都是工艺美术的上品。大致说来,那些门和窗尽量工细\footnote{〔工细〕指做工细致不马虎,注重细节。}而决不庸俗,即使简朴而别具匠心。四扇,八扇,十二扇,综合起来看,谁都要赞叹这是高度的图案美。摄影家挺喜欢这些门和窗,他们斟酌着光和影,摄成称心满意的照片。
    
    苏州园林与北京的园林不同,极少使用彩绘。梁和柱子以及门窗栏杆大多漆广漆\footnote{〔广漆〕生漆或熟漆中加入桐油制成,棕黑色,又叫金漆,是明清木家具常用涂料。},那是不刺眼的颜色。墙壁白色。有些室内墙壁下半截铺水磨方砖,淡灰色和白色对衬。屋瓦和檐漏一律淡灰色。这些颜色与草木的绿色配合,引起人们安静闲适的感觉。花开时节,更显得各种花明艳照眼。
    
    可以说的当然不止以上这些,这里不再多写了。
    
\end{normalsize}


\newpage

\textbf{注释}:

\vspace{-1em}

\begin{itemize}
    \setlength\itemsep{-0.2em}
    \item 〔因地制宜〕根据不同地方的具体条件,制定相应的妥善措施
    \item 〔自出心裁〕出于自己心中的设计,指构思独到。
    \item 〔嶙峋〕形容山石高直而细,突兀而立。引申为人消瘦或刚直有骨气。
    \item 〔斟酌〕计算着倒酒。引申为反复考虑以后决定取舍。
    \item 〔阅历〕亲身见闻、经历。
    \item 〔檐漏〕屋檐和墙之间的空档。空档中的椽,梁,枋等部分都外露,梁枋上住往做出丰富多彩的装饰,非常漂亮,是讲究的做法。
    \item 〔俯仰生姿〕指高低错落,如人俯仰的姿态。
    \item 〔重峦叠嶂〕山峰连绵不绝。
    \item 〔珠光宝气〕珍珠美玉的光辉。
\end{itemize}

\chapter{万紫千红的花}

\begin{normalsize}
    
    又到了赏花的季节。看着争相开放的花朵,你有没有想过,花为什么有这么多美丽鲜艳的色彩呢?
    
    花怎么会有各种美丽鲜艳的色彩呢?这是由于花瓣的细胞液\footnote{〔细胞液〕细胞内的液体。}中存在着色素\footnote{〔色素〕生物体内使生物呈现各种颜色的物质。}。有一些花的颜色是红的、蓝的或紫的。这些花里含的色素叫花青素。花青素平常是紫色,遇到酸就变红,遇到碱就变蓝。你可以拿一朵喇叭花\footnote{〔喇叭花〕园林花卉,旋花科缠绕草本植物,也叫牵牛花。}来做实验,把红色的喇叭花泡在肥皂水里,它很快就变成蓝色,因为肥皂是碱性的。再把这朵蓝色的花泡到醋里,它又重新变成红色,因为醋是酸性的。
    
    还有一些花的颜色是黄的、橙黄的、橙红的。它们的花瓣里含的色素叫胡萝卜素。胡萝卜素最初是在胡萝卜里发现的,有六十多种。柑橘、南瓜的颜色,也来自胡萝卜素。含有胡萝卜素的花也是五颜六色的。
    
    白色的花含有什么色素呢?白色的花什么色素也没有。它看来是白色的,那是因为花瓣里充满了小气泡的缘故。你拿一朵白花来,用手捏一捏花瓣,把里面的小气泡挤掉,它就成为无色透明的了。
    
    各种花含有的色素和酸、碱的浓度不一样。随着养料、水分、温度等条件经常在变化,花的颜色就有深有浅,有浓有淡,有的还会变色。
    
    会变色的花很多。例如红喇叭花,它初开的时候是红色,败落的时候就变成紫色了。杏花含苞待放的时候是红色,开放后逐渐变淡,最后几乎变成白色了。最有趣的要数“弄色木芙蓉\footnote{〔弄色木芙蓉〕我国传统观赏花卉,锦葵科落叶灌木,花色多变,也叫文官花。}”。它的花初开是白色,第二天变成浅红色,后来又变成深红色,到花落的时候又变成紫色了。这些变化看起来很玄妙,其实都是花内色素随着温度和酸碱浓度变化所玩的把戏。
    
    我国有种樱草\footnote{〔樱草〕我国传统观赏花卉,报春花科草本植物,花色多样。},在普通温度下,花是红色。在30摄氏度的暗室里,就变成白色了。八仙花\footnote{〔八仙花〕也叫绣球花,我国传统观赏花卉,绣球科落叶灌木,花密集成簇,花色多样。}在有些土壤中开蓝色的花,在另一些土壤中开粉红色的花。还有一些花,受精\footnote{〔受精〕花的雌蕊中的卵细胞接受来自雄蕊的精细胞的过程。}以后也会变色。比如海桐花\footnote{〔海桐〕我国传统植物,海桐科常绿灌木,常用于防风绿化。},起初是黄色,受精后就变成白色了。红锦带花\footnote{〔锦带花〕观赏花卉,忍冬科落叶灌木,花红色。}受精后,也会变成白色。
    
    有人统计了4197种花的颜色。做了如下的分类:
    
    \begin{center}\begin{tabular}{| c | c | c | c | c | c | c | c | c | c |}
    
    \hline
    
    颜色 & 白 & 黄 & 红 & 蓝 & 紫 & 绿 & 橙 & 茶 & 黑 \\
    
    \hline
    
    数量 & 1193 & 951 & 923 & 594 & 307 & 153 & 50 & 18 & 8 \\
    
    \hline
    
    \end{tabular}\end{center}
    
    从这个统计可以看出,白色、黄色和红色的花最多。这三种颜色的花有个好处,配着绿叶非常鲜艳,容易惹昆虫注意。
    
    昆虫对花的颜色也是有选择的。比如蜜蜂就不大喜欢黄色,而喜欢红色和蓝色。更有趣的是有些花还选择昆虫。例如金鱼草\footnote{〔金鱼草〕车前科草本植物,常见庭园花卉。},他的花平时闭合着,等到它所喜爱的一种小蜂飞来的时候,花就立即开放了。别的小昆虫来“叩门”,它理也不理。还有待宵草\footnote{〔待宵草〕观赏花卉,柳叶菜科草本植物,花黄色。},它的花到夜间才能张开笑脸。这时候,有一种白天躲在阴暗地方的小蛾,就飞来帮它传送花粉。夜间开的花,大多是白色或黄色的,否则在黑暗中就不容易被昆虫发现。
    
    美丽的花朵对人有很大的吸引力。意大利的诗人但丁\footnote{〔但丁〕但丁·阿利杰里,13世纪意大利诗人,文艺复兴的先驱。主要作品为长诗《神曲》三部曲。}在他的《神曲》中写道:
    
    \begin{quotation}
    
    我向前走,但我一看到花,脚步就慢下来了。
    
    \end{quotation}
    
    世界上恐怕没有人不喜爱花。人们用万紫千红的花来点缀生活环境,用它的形象来装饰衣服和用具,把它作为美丽、纯洁和幸福的象征。
    
\end{normalsize}


\newpage

\textbf{注释}:

\vspace{-1em}

\begin{itemize}
    \setlength\itemsep{-0.2em}
    \item 〔浓度〕度量浓淡的量。
    \item 〔败落〕花开过后衰落。
    \item 〔玄妙〕事物的道理深奥难明。
    \item 〔点缀〕以少量衬托,装饰。缀:将小块布连起来,把小物连到物件边缘。
\end{itemize}

\chapter{在烈日和暴雨下}

\begin{normalsize}
    
    六月十五\footnote{〔六月十五〕这里指农历六月十五日。}那天,天热得发了狂。太阳刚一出来,地上已经像下了火。一些似云非云似雾非雾的灰气低低地浮在空中,使人觉得憋气。一点风也没有。祥子在院子里看了看那灰红的天,喝了瓢凉水就走出去。
    
    街上的柳树像病了似的,叶子挂着层灰土在枝上打着卷;枝条一动也懒得动,无精打采地低垂着。马路上一个水点也没有,干巴巴地发着白光。便道上尘土飞起多高,跟天上的灰气联接起来,结成一片毒恶的灰沙阵,烫着行人的脸。处处干燥,处处烫手,处处憋闷,整个老城像烧透了的砖窑,使人喘不过气来。狗趴在地上吐出红舌头,骡马的鼻孔张得特别大,小贩们不敢吆喝,柏油路晒化了,甚至于铺户门前的铜牌好像也要晒化。街上非常寂静,只有铜铁铺里发出使人焦躁的一些单调的丁丁当当。拉车的人们,只要今天还不至于挨饿,就懒得去张罗买卖:有的把车放在有些阴凉的地方,支起车棚,坐在车上打盹;有的钻进小茶馆去喝茶;有的根本没有拉出车来,只到街上看看有没有出车的可能。那些拉着买卖的,即使是最漂亮的小伙子,也居然甘于丢脸,不敢再跑,只低着头慢慢地走。每一口井都成了他们的救星,不管刚拉了几步,见井就奔过去,赶不上新汲的水,就跟驴马同在水槽里灌一大气。还有的,因为中了暑,或是发痧\footnote{〔痧〕病名。由饮食不干净引起,患者又吐又泻,四肢发凉,严重时会失去知觉。由于常在热天发病,一般认为是中暑。},走着走着,一头栽到地上,永不起来。
    
    祥子有些胆怯了。拉着空车走了几步,他觉出从脸到脚都被热气围着,连手背上都流了汗。可是见了座儿\footnote{〔座儿〕指想要坐车的人。}他还想拉,以为跑起来也许倒能有点风。他拉上了个买卖,把车拉起来,他才晓得天气的厉害已经到了不允许任何人工作的程度。一跑,就喘不上气来,而且嘴唇发焦,明明心里不渴,也见水就想喝。不跑呢,那毒花花的太阳把手和脊背都要晒裂。好歹拉到了地方,他的裤褂全裹在了身上。拿起芭蕉扇扇扇,没用,风是热。他已经不知喝了几气凉水,可是又跑到茶馆去。
    
    两壶热茶喝下去,他心里安静了些。茶从嘴里进去,汗马上从身上出来,好像身子已经是空膛的,不会再储藏一点水分。他不敢再动了。
    
    坐下了好久,他心里腻烦\footnote{〔腻烦〕这里指感觉心烦。}了。既不敢出去,又没事可作,他觉得天气仿佛成心跟他过不去。想出去,可是腿真懒得动,身上非常软,好像洗澡没洗痛快那样,汗虽然出了不少,心里还是不舒畅。又坐了会儿,他再也坐不住了,反正坐着也是出汗,不如爽性出去试试。
    
    一出来,才晓得自己错了。天上的那层灰气已经散开,不很憋闷了,可是阳光也更厉害了:没人敢抬头看太阳在哪里,只觉得到处都闪眼,空中,屋顶上,墙壁上,地上,都白亮亮的,白里透着点红,从上至下整个地像一面极大的火镜,每一条光都像火镜的焦点,晒得东西要发火。在这个白光里,每一个颜色都刺目,每一个声响都难听,每一种气味都搀合着地上蒸发出来的腥臭。街上仿佛没了人,道路好像忽然加宽了许多,空旷而没有一点凉气,白花花的令人害怕。祥子不知怎么是好了,低着头,拉着车,慢腾腾地往前走,没有主意,没有目的,昏昏沉沉的,身上挂着一层粘汗,发着馊臭的味儿。走了会儿,脚心跟鞋袜粘在一块,好像踩着块湿泥,非常难过,本来不想再喝水,可是见了井不由得又过去灌了一气,不为解渴,似乎专为享受井水那点凉气,从口腔到胃里,忽然凉了一下,身上的毛孔猛地一收缩,打个冷战,非常舒服。喝完,他连连地打嗝,水要往上漾\footnote{〔漾〕这里是胃里的东西向上涌的意思。}。
    
    走一会儿,坐一会儿,他始终懒得张罗买卖。一直到了正午,他还觉不出饿来。想去照例地吃点什么,可是看见食物就要恶心。胃里差不多装满了各样的水,有时候里面会轻轻地响,像骡马喝完水那样,肚子里光光光地响动。
    
    正在午后一点的时候,他又拉上个买卖。这是一天里最热的时候,又赶上这一夏里最热的一天。刚走了几步,他觉到一点凉风,就像在极热的屋里从门缝进来一点凉气似的。他不敢相信自己;看看路旁的柳枝,的确微微地动了两下。街上突然加多了人,铺子里的人争着往外跑,都攥着把蒲扇遮着头,四下里找。“有了凉风!有了凉风!凉风下来了!”大家都嚷着,几乎要跳起来。路旁的柳树忽然变成了天使似的,传达着上天的消息。“柳条儿动了!老天爷,多赏点凉风吧!”
    
    还是热,心里可镇定多了。凉风,即使是一点点,也给了人们许多希望。几阵凉风过去,阳光不那么强了,一阵亮,一阵稍暗,仿佛有片飞沙在上面浮动似的。风忽然大起来,那半天没动的柳条像猛地得到什么可喜的事,飘洒地摇摆,枝条都像长出一截儿来。一阵风过去,天暗起来,灰尘全飞到半空。尘土落下一些,北面的天边出现了墨似的乌云。祥子身上没了汗,向北边看了一眼,把车停住,上了雨布,他晓得夏天的雨是说来就来,不容工夫的。
    
    刚上好了雨布,又是一阵风,墨云滚似地遮黑了半边天。地上的热气跟凉风搀合起来,夹杂着腥臊的干土,似凉又热;南边的半个天响晴白日,北边的半个天乌云如墨,仿佛有什么大难来临,一切都惊慌失措。车夫急着上雨布,铺户忙着收幌子\footnote{〔幌子〕商店门外表明所卖商品的标志。},小贩们慌手忙脚地收拾摊子,行路的加紧往前奔。又一阵风。风过去,街上的幌子,小摊,行人,仿佛都被风卷走了,全不见了,只剩下柳枝随着风狂舞。
    
    云还没铺满天,地上已经很黑,极亮极热的晴午忽然变成了黑夜似的。风带着雨星,像在地上寻找什么似的,东一头西一头地乱撞。北边远处一个红闪,像把黑云掀开一块,露出一大片血似的。风小了,可是利飕有劲,使人颤抖。一阵这样的风过去,一切都不知怎么好似的,连柳树都惊疑不定地等着点什么。又一个闪,正在头上,白亮亮的雨点紧跟着落下来,极硬的,砸起许多尘土,土里微带着雨气。几个大雨点砸在祥子的背上,他哆嗦了两下。雨点停了,黑云铺满了天。又一阵风,比以前的更厉害,柳枝横着飞,尘土往四下里走,雨道往下落;风,土,雨,混在一处,联成一片,横着竖着都灰茫茫冷飕飕,一切的东西都裹在里面,辨不清哪是树,哪是地,哪是云,四面八方全乱,全响,全迷糊。风过去了,只剩下直的雨道,扯天扯底地垂落,看不清一条条的,只是那么一片,一阵,地上射起无数的箭头,房屋上落下万千条瀑布。几分钟,天地已经分不开,空中的水往下倒,地上的水到处流,成了灰暗昏黄的,有时又白亮亮的,一个水世界。
    
    祥子的衣服早已湿透,全身没有一点干松的地方;隔着草帽,他的头发已经全湿。地上的水过了脚面,湿裤子裹住他的腿,上面的雨直砸着他的头和背,横扫着他的脸。他不能抬头,不能睁眼,不能呼吸,不能迈步。他像要立定在水里,不知道哪是路,不晓得前后左右都有什么,只觉得透骨凉的水往身上各处浇。他什么也不知道了,只茫茫地觉得心有点热气,耳边有一片雨声。他要把车放下,但是不知放在哪里好。想跑,水裹住他的腿。他就那么半死半活地,低着头一步一步地往前拽。坐车的仿佛死在了车上,一声不出地任凭车夫在水里挣命。
    
    雨小了些,祥子微微直了直脊背,吐出一口气:“先生,避避再走吧!”
    
    “快走!你把我扔在这儿算怎么回事?”坐车的跺着脚喊。
    
    祥子真想硬把车放下,去找个地方避一避。可是,看看浑身上下都流水,他知道一站住就会哆嗦成一团。他咬上了牙,蹚着水,不管高低深浅地跑起来。刚跑出不远,天黑了一阵,紧跟着一亮,雨又迷住他的眼。拉到了,坐车的连一个铜板也没多给。祥子没说什么,他已经顾不过命来。
    
    雨住一会儿,又下一阵儿。比以前小了许多。祥子一气跑回了家。抱着火,烤了一阵,他哆嗦得像风雨中的树叶。虎妞给他冲了碗姜糖水,他傻子似的抱着碗一气喝完。喝完,他钻了被窝,什么也不知道了,似睡非睡,耳中刷刷的一片雨声。
    
\end{normalsize}


\newpage

\textbf{注释}:

\vspace{-1em}

\begin{itemize}
    \setlength\itemsep{-0.2em}
    \item 〔张罗买卖〕指想办法找坐车的客人。
    \item 〔打盹〕打瞌睡。
    \item 〔汲〕从下往上打水。
    \item 〔蹚〕从浅水里走过去。
\end{itemize}

\chapter{背影}

\begin{normalsize}
    
    我与父亲不相见已二年余了,我最不能忘记的是他的背影。
    
    那年冬天,祖母死了,父亲的差使\footnote{〔差使〕旧时指官场中临时委任的职务,后来也泛指职务或官职。}也交卸\footnote{〔交卸〕交付货物时卸货。引申指离职时交接职位事务,比喻离职。此处实指失业。}了,正是祸不单行的日子。我从北京到徐州,打算跟着父亲奔丧回家。到徐州见着父亲,看见满院狼藉的东西,又想起祖母,不禁簌簌地流下眼泪。父亲说:“事已如此,不必难过,好在天无绝人之路!”
    
    回家变卖典质,父亲还了亏空;又借钱办了丧事。这些日子,家中光景很是惨澹\footnote{〔惨澹〕惨淡,光线暗淡。引申指悲惨凄凉。},一半为了丧事,一半为了父亲赋闲\footnote{〔赋闲〕这里指失业在家。晋代潘岳辞官家居,作《闲居赋》,后因称罢官闲居,事业无事为赋闲。}。丧事完毕,父亲要到南京谋事,我也要回北京念书,我们便同行。
    
    到南京时,有朋友约去游逛,勾留\footnote{〔勾留〕短时间停留。}了一日;第二日上午便须渡江到浦口\footnote{〔浦口〕地名,即今南京市浦口区,在南京西北部的长江北岸,是早年津浦地铁的终点。},下午上车北去。父亲因为事忙,本已说定不送我,叫旅馆里一个熟识的茶房\footnote{〔茶房〕旧时称在旅馆、车站等从事供应茶水等杂物的人。}陪我同去。他再三嘱咐茶房,甚是仔细。但他终于不放心,怕茶房不妥帖;颇踌躇了一会。其实我那年已二十岁,北京已来往过两三次,是没有什么要紧的了。他踌躇了一会,终于决定还是自己\footnote{〔己〕停止。这里是控制的意思。}送我去。我再三劝他不必去;他只说:“不要紧,他们去不好!”
    
    我们过了江,进了车站。我买票,他忙着照看行李。行李太多,得向脚夫\footnote{〔脚夫〕旧时对搬运工人的称呼。}行些小费才可过去。他便又忙着和他们讲价钱。我那时真是聪明过分,总觉他说话不大漂亮,非自己插嘴不可,但他终于讲定了价钱;就送我上车。他给我拣定了靠车门的一张椅子;我将他给我做的紫毛大衣铺好座位。他嘱我路上小心,夜里要警醒\footnote{〔警醒〕睡眠时容易醒来。这里是不要睡得太沉的意思。}些,不要受凉。又嘱托茶房好好照应我。我心里暗笑他的迂;他们只认得钱,托他们只是白托!而且我这样大年纪的人,难道还不能料理自己么?我现在想想,我那时真是太聪明了。
    
    我说道:“爸爸,你走吧。”他望车外看了看,说:“我买几个橘子去。你就在此地,不要走动。”我看那边月台的栅栏外有几个卖东西的等着顾客。走到那边月台,须穿过铁道,须跳下去又爬上去。父亲是一个胖子,走过去自然要费事些。我本来要去的,他不肯,只好让他去。我看见他戴着黑布小帽,穿着黑布大马褂,深青布棉袍,蹒跚地走到铁道边,慢慢探身下去,尚不大难。可是他穿过铁道,要爬上那边月台,就不容易了。他用两手攀着上面,两脚再向上缩;他肥胖的身子向左微倾,显出努力的样子。这时我看见他的背影,我的泪很快地流下来了。我赶紧拭干了泪。怕他看见,也怕别人看见。我再向外看时,他已抱了朱红的橘子往回走了。过铁道时,他先将橘子散放在地上,自己慢慢爬下,再抱起橘子走。到这边时,我赶紧去搀他。他和我走到车上,将橘子一股脑儿放在我的皮大衣上。于是扑扑衣上的泥土,心里很轻松似的。过一会儿说:“我走了,到那边来信!”我望着他走出去。他走了几步,回过头看见我,说:“进去吧,里边没人。”等他的背影混入来来往往的人里,再找不着了,我便进来坐下,我的眼泪又来了。
    
    近几年来,父亲和我都是东奔西走,家中光景是一日不如一日。他少年出外谋生,独力支持,做了许多大事。哪知老境却如此颓唐!他触目伤怀,自然情不能自已。情郁于中,自然要发之于外;家庭琐屑便往往触他之怒。他待我渐渐不同往日。但最近两年不见,他终于忘却我的不好,只是惦记着我,惦记着他的儿子。我北来后,他写了一信给我,信中说道:“我身体平安,惟膀子疼痛厉害,举箸提笔,诸多不便,大约大去\footnote{〔大去〕委婉语,指死亡。}之期不远矣。”我读到此处,在晶莹的泪光中,又看见那肥胖的、青布棉袍黑布马褂的背影。唉!我不知何时再能与他相见!
    
\end{normalsize}


\newpage

\textbf{注释}:

\vspace{-1em}

\begin{itemize}
    \setlength\itemsep{-0.2em}
    \item 〔狼藉〕乱七八糟的样子。
    \item 〔簌簌〕纷纷落下的样子。
    \item 〔典质〕把财务等典当,抵押出去。典:典当。质:抵押。
    \item 〔踌躇〕停下来思考往哪里走的样子。引申指犹豫。
    \item 〔迂〕言行守旧,不合时宜。
    \item 〔月台〕站台。
    \item 〔拭〕擦。
    \item 〔颓唐〕衰颓败落。
    \item 〔郁〕(忧愁,气愤等)积聚。
    \item 〔琐屑〕细小而琐碎的事。
    \item 〔膀子〕胳膊。
    \item 〔箸〕筷子。
\end{itemize}

\chapter{故乡}

\begin{normalsize}
    
    我冒了严寒,回到相隔二千余里,别了二十余年的故乡去。
    
    时候既然\footnote{〔既然〕已经。这里和后面的“又”连用,表递进关系。}是深冬;渐近故乡时,天气又阴晦了,冷风吹进船舱中,呜呜的响,从蓬隙向外一望,苍黄的天底下,远近横着几个萧索的荒村,没有一些活气。我的心禁不住悲凉起来了。阿!这不是我二十年来时时记得的故乡?
    
    我所记得的故乡全不如此。我的故乡好得多了。但要我记起他的美丽,说出他的佳处来,却又没有影像,没有言辞了。仿佛也就如此。于是我自己解释说:故乡本也如此,——虽然没有进步,也未必有如我所感的悲凉,这只是我自己心情的改变罢了,因为我这次回乡,本没有什么好心绪。
    
    我这次是专为了别他而来的。我们多年聚族而居的老屋,已经公同卖给别姓了,交屋的期限,只在本年,所以必须赶在正月初一以前,永别了熟识的老屋,而且远离了熟识的故乡,搬家到我在谋食的异地去。
    
    第二日清早晨我到了我家的门口了。瓦楞\footnote{〔瓦楞〕瓦屋屋顶一行一行的瓦铺成的凹凸相间的行列。}上许多枯草的断茎当风抖着,正在说明这老屋难免易主的原因。几房的本家大约已经搬走了,所以很寂静。我到了自家的房外,我的母亲早已迎着出来了,接着便飞出了八岁的侄儿宏儿。
    
    我的母亲很高兴,但也藏着许多凄凉的神情,教我坐下,歇息,喝茶,且不谈搬家的事。宏儿没有见过我,远远的对面站着只是看。
    
    但我们终于谈到搬家的事。我说外间的寓所已经租定了,又买了几件家具,此外须将家里所有的木器卖去,再去增添。母亲也说好,而且行李也略已齐集,木器不便搬运的,也小半卖去了,只是收不起钱来。
    
    “你休息一两天,去拜望亲戚本家一回,我们便可以走了。”母亲说。
    
    “是的。”
    
    “还有闰土,他每到我家来时,总问起你,很想见你一回面。我已经将你到家的大约日期通知他,他也许就要来了。”
    
    这时候,我的脑里忽然闪出一幅神异的图画来:深蓝的天空中挂着一轮金黄的圆月,下面是海边的沙地,都种着一望无际的碧绿的西瓜,其间有一个十一二岁的少年,项带银圈,手捏一柄钢叉,向一匹猹\footnote{〔猹〕作者在1929年5月4日致舒新城的信中说:“‘猹’字是我据乡下人所说的声音,生造出来的,读如‘查’。……现在想起来,也许是獾罢。”}尽力的刺去,那猹却将身一扭,反从他的胯下逃走了。
    
    这少年便是闰土。我认识他时,也不过十多岁,离现在将有三十年了;那时我的父亲还在世,家景也好,我正是一个少爷。那一年,我家是一件大祭祀的值年\footnote{〔大祭祀的值年〕封建社会中的大家族,每年都有祭祀祖先的活动,费用从族中“祭产”收入支取,由各房按年轮流主持,轮到的称为“值年”。}。这祭祀,说是三十多年才能轮到一回,所以很郑重;正月里供祖像,供品很多,祭器很讲究,拜的人也很多,祭器也很要防偷去。我家只有一个忙月(我们这里给人做工的分三种:整年给一定人家做工的叫长工;按日给人做工的叫短工;自己也种地,只在过年过节以及收租时候来给一定人家做工的称忙月),忙不过来,他便对父亲说,可以叫他的儿子闰土来管祭器的。
    
    我的父亲允许了;我也很高兴,因为我早听到闰土这名字,而且知道他和我仿佛年纪,闰月生的,五行缺土\footnote{〔五行缺土〕旧社会所谓算“八字”的迷信说法。即用天干(甲乙丙丁戊己庚辛壬癸)和地支(子丑寅卯辰巳午未申酉戌亥)相配,来记一个人出生的年、月、日、时,各得两字,合为“八字”;又认为它们在五行(金、木、水、火、土)中各有所属,如甲乙寅卯属木,丙丁巳午属火等等,如八个字能包括五者,就是五行俱全。“五行缺土”,就是这八个字中没有属土的字,需用土或土作偏旁的字取名等办法来弥补。},所以他的父亲叫他闰土。他是能装弶捉小鸟雀的。
    
    我于是日日盼望新年,新年到,闰土也就到了。好容易到了年末,有一日,母亲告诉我,闰土来了,我便飞跑的去看。他正在厨房里,紫色的圆脸,头戴一顶小毡帽,颈上套一个明晃晃的银项圈,这可见他的父亲十分爱他,怕他死去,所以在神佛面前许下愿心,用圈子将他套住了。他见人很怕羞,只是不怕我,没有旁人的时候,便和我说话,于是不到半日,我们便熟识了。
    
    我们那时候不知道谈些什么,只记得闰土很高兴,说是上城之后,见了许多没有见过的东西。
    
    第二日,我便要他捕鸟。他说:
    
    “这不能。须大雪下了才好。我们沙地上,下了雪,我扫出一块空地来,用短棒支起一个大竹匾,撒下秕谷,看鸟雀来吃时,我远远地将缚在棒上的绳子只一拉,那鸟雀就罩在竹匾下了。什么都有:稻鸡,角鸡,鹁鸪,蓝背……”
    
    我于是又很盼望下雪。
    
    闰土又对我说:
    
    “现在太冷,你夏天到我们这里来。我们日里到海边捡贝壳去,红的绿的都有,鬼见怕也有,观音手也有\footnote{〔鬼见怕……〕鬼见怕和观音手,都是小贝壳的名称。旧时浙江沿海的人把这种小贝壳用线串在一起,戴在孩子的手腕或脚踝上,认为可以“避邪”。这类名称多是根据“避邪”的意思取的。}。晚上我和爹管西瓜去,你也去。”
    
    “管贼么?”
    
    “不是。走路的人口渴了摘一个瓜吃,我们这里是不算偷的。要管的是獾猪,刺猬,猹。月亮底下,你听,啦啦的响了,猹在咬瓜了。你便捏了胡叉,轻轻地走去……”
    
    我那时并不知道这所谓猹的是怎么一件东西——便是现在也没有知道——只是无端的觉得状如小狗而很凶猛。
    
    “他不咬人么?”
    
    “有胡叉呢。走到了,看见猹了,你便刺。这畜生很伶俐,倒向你奔来,反从胯下窜了。他的皮毛是油一般的滑……”
    
    我素不知道天下有这许多新鲜事:海边有如许五色的贝壳;西瓜有这样危险的经历,我先前单知道他在水果店里出卖罢了。
    
    “我们沙地里,潮汛要来的时候,就有许多跳鱼儿只是跳,都有青蛙似的两个脚……”
    
    阿!闰土的心里有无穷无尽的希奇的事,都是我往常的朋友所不知道的。他们不知道一些事,闰土在海边时,他们都和我一样只看见院子里高墙上的四角的天空。
    
    可惜正月过去了,闰土须回家里去,我急得大哭,他也躲到厨房里,哭着不肯出门,但终于被他父亲带走了。他后来还托他的父亲带给我一包贝壳和几支很好看的鸟毛,我也曾送他一两次东西,但从此没有再见面。
    
    现在我的母亲提起了他,我这儿时的记忆,忽而全都闪电似的苏生\footnote{〔苏生〕苏醒,复活。这里指重新记起来。}过来,似乎看到了我的美丽的故乡了。我应声说:
    
    “这好极!他,——怎样?……”
    
    “他?……他景况也很不如意……”母亲说着,便向房外看,“这些人又来了。说是买木器,顺手也就随便拿走的,我得去看看。”
    
    母亲站起身,出去了。门外有几个女人的声音。我便招宏儿走近面前,和他闲话:问他可会写字,可愿意出门。
    
    “我们坐火车去么?”
    
    “我们坐火车去。”
    
    “船呢?”
    
    “先坐船,……”
    
    “哈!这模样了!胡子这么长了!”一种尖利的怪声突然大叫起来。
    
    我吃了一吓,赶忙抬起头,却见一个凸颧骨,薄嘴唇,五十岁上下的女人站在我面前,两手搭在髀间,没有系裙,张着两脚,正像一个画图仪器里细脚伶仃\footnote{〔细脚伶仃〕身板细长单薄。}的圆规。
    
    我愕然了。
    
    “不认识了么?我还抱过你咧!”
    
    我愈加愕然了。幸而我的母亲也就进来,从旁说:
    
    “他多年出门,统忘却了。你该记得罢,”便向着我说,“这是斜对门的杨二嫂,……开豆腐店的。”
    
    哦,我记得了。我孩子时候,在斜对门的豆腐店里确乎终日坐着一个杨二嫂,人都叫伊“豆腐西施\footnote{〔西施〕春秋时越国的美女,后来用以泛称一般美女。}”。但是擦着白粉,颧骨没有这么高,嘴唇也没有这么薄,而且终日坐着,我也从没有见过这圆规式的姿势。那时人说:因为伊,这豆腐店的买卖非常好。但这大约因为年龄的关系,我却并未蒙着一毫感化,所以竟完全忘却了。然而圆规很不平,显出鄙夷的神色,仿佛嗤笑法国人不知道拿破仑\footnote{〔拿破仑〕即拿破仑·波拿巴,18世纪末19世纪初法国资产阶级革命时期的军事家、政治家。建立法兰西第一帝国。},美国人不知道华盛顿\footnote{〔华盛顿〕即乔治·华盛顿,18世纪末美国政治家。曾领导独立战争,胜利后任美国第一任总统。}似的,冷笑说:
    
    “忘了?这真是贵人眼高……”
    
    “那有这事……我……”我惶恐着,站起来说。
    
    “那么,我对你说。迅哥儿,你阔了,搬动又笨重,你还要什么这些破烂木器,让我拿去罢。我们小户人家,用得着。”
    
    “我并没有阔哩。我须卖了这些,再去……”
    
    “阿呀呀,你放了道台\footnote{〔道台〕清朝官职道员的俗称,分总管一个区域行政职务的道员和专掌某一特定职务的道员。前者是省以下、府州以上的行政长官;后者掌管一省特定事务,如督粮道、兵备道等。辛亥革命后,北洋军阀政府也曾沿用此制,改称道尹。}了,还说不阔?你现在有三房姨太太;出门便是八抬的大轿,还说不阔?吓,什么都瞒不过我。”
    
    我知道无话可说了,便闭了口,默默的站着。
    
    “阿呀阿呀,真是愈有钱,便愈是一毫不肯放松\footnote{〔放松〕松手,放手。},愈是一毫不肯放松,便愈有钱……”圆规一面愤愤的回转身,一面絮絮的说,慢慢向外走,顺便将我母亲的一副手套塞在裤腰里,出去了。
    
    此后又有近处的本家和亲戚来访问我。我一面应酬,偷空便收拾些行李,这样的过了三四天。
    
    一日是天气很冷的午后,我吃过午饭,坐着喝茶,觉得外面有人进来了,便回头去看。我看时,不由的非常出惊,慌忙站起身,迎着走去。
    
    这来的便是闰土。虽然我一见便知道是闰土,但又不是我这记忆上的闰土了。他身材增加了一倍;先前的紫色的圆脸,已经变作灰黄,而且加上了很深的皱纹;眼睛也像他父亲一样,周围都肿得通红,这我知道,在海边种地的人,终日吹着海风,大抵是这样的。他头上是一顶破毡帽,身上只一件极薄的棉衣,浑身瑟索着;手里提着一个纸包和一支长烟管,那手也不是我所记得的红活圆实的手,却又粗又笨而且开裂,像是松树皮了。
    
    我这时很兴奋,但不知道怎么说才好,只是说:
    
    “阿!闰土哥,——你来了?……”
    
    我接着便有许多话,想要连珠一般涌出:角鸡,跳鱼儿,贝壳,猹,……但又总觉得被什么挡着似的,单在脑里面回旋,吐不出口外去。
    
    他站住了,脸上现出欢喜和凄凉的神情;动着嘴唇,却没有作声。他的态度终于恭敬起来了,分明的叫道:
    
    “老爷!……”
    
    我似乎打了一个寒噤\footnote{〔寒噤〕因寒冷或受惊,身体不自主的颤动,寒战。};我就知道,我们之间已经隔了一层可悲的厚障壁了。我也说不出话。
    
    他回过头去说,“水生,给老爷磕头。”便拖出躲在背后的孩子来,这正是一个廿年前的闰土,只是黄瘦些,颈子上没有银圈罢了。“这是第五个孩子,没有见过世面,躲躲闪闪……”
    
    母亲和宏儿下楼来了,他们大约也听到了声音。
    
    “老太太。信是早收到了。我实在喜欢的不得了,知道老爷回来……”闰土说。
    
    “阿,你怎的这样客气起来。你们先前不是哥弟称呼么?还是照旧:迅哥儿。”母亲高兴的说。
    
    “阿呀,老太太真是……这成什么规矩。那时是孩子,不懂事……”闰土说着,又叫水生上来打拱,那孩子却害羞,紧紧的只贴在他背后。
    
    “他就是水生?第五个?都是生人,怕生也难怪的;还是宏儿和他去走走。”母亲说。
    
    宏儿听得这话,便来招水生,水生却松松爽爽同他一路出去了。母亲叫闰土坐,他迟疑了一回,终于就了坐,将长烟管靠在桌旁,递过纸包来,说:
    
    “冬天没有什么东西了。这一点干青豆倒是自家晒在那里的,请老爷……”
    
    我问问他的景况。他只是摇头。
    
    “非常难。第六个孩子也会帮忙了,却总是吃不够……又不太平……什么地方都要钱,没有规定\footnote{〔规定〕定规,定下来不变的规矩。}……收成又坏。种出东西来,挑去卖,总要捐几回钱\footnote{〔捐钱〕这里指清末民初以“捐款”为名义的变相收税盘剥。},折了本\footnote{〔折了本〕亏本。折:亏损。};不去卖,又只能烂掉……”
    
    他只是摇头;脸上虽然刻着许多皱纹,却全然不动,仿佛石像一般。他大约只是觉得苦,却又形容不出,沉默了片时,便拿起烟管来默默的吸烟了。
    
    母亲问他,知道他的家里事务忙,明天便得回去;又没有吃过午饭,便叫他自己到厨下炒饭吃去。
    
    他出去了;母亲和我都叹息他的景况:多子,饥荒,苛税,兵,匪,官,绅,都苦得他像一个木偶人了。母亲对我说,凡是不必搬走的东西,尽可以送他,可以听他自己去拣择。
    
    下午,他拣好了几件东西:两条长桌,四个椅子,一副香炉和烛台,一杆抬秤\footnote{〔抬秤〕一种能秤上百斤东西的大杆秤。}。他又要所有的草灰(我们这里煮饭是烧稻草的,那灰,可以做沙地的肥料),待我们启程的时候,他用船来载去。
    
    夜间,我们又谈些闲天,都是无关紧要的话;第二天早晨,他就领了水生回去了。
    
    又过了九日,是我们启程的日期。闰土早晨便到了,水生没有同来,却只带着一个五岁的女儿管船只。我们终日很忙碌,再没有谈天的工夫。来客也不少,有送行的,有拿东西的,有送行兼拿东西的。待到傍晚我们上船的时候,这老屋里的所有破旧大小粗细东西,已经一扫而空了。
    
    我们的船向前走,两岸的青山在黄昏中,都装成了深黛颜色,连着退向船后梢去。
    
    宏儿和我靠着船窗,同看外面模糊的风景,他忽然问道:
    
    “大伯!我们什么时候回来?”
    
    “回来?你怎么还没有走就想回来了。”
    
    “可是,水生约我到他家玩去咧……”他睁着大的黑眼睛,痴痴的想。
    
    我和母亲也都有些惘然,于是又提起闰土来。母亲说,那豆腐西施的杨二嫂,自从我家收拾行李以来,本是每日必到的,前天伊在灰堆里,掏出十多个碗碟来,议论之后,便定说是闰土埋着的,他可以在运灰的时候,一齐搬回家里去;杨二嫂发见了这件事,自己很以为功,便拿了那狗气杀(这是我们这里养鸡的器具,木盘上面有着栅栏,内盛食料,鸡可以伸进颈子去啄,狗却不能,只能看着气死),飞也似的跑了,亏伊装着这么高低的小脚\footnote{〔装着……小脚〕小脚:封建陋习,用布将女性双脚紧紧缠裹,使之畸形变小,以之为美。这样高低:这么高。},竟跑得这样快。
    
    老屋离我愈远了;故乡的山水也都渐渐远离了我,但我却并不感到怎样的留恋。我只觉得我四面有看不见的高墙,将我隔成孤身,使我非常气闷;那西瓜地上的银项圈的小英雄的影像,我本来十分清楚,现在却忽地模糊了,又使我非常的悲哀。
    
    母亲和宏儿都睡着了。
    
    我躺着,听船底潺潺的水声,知道我在走我的路。我想:我竟与闰土隔绝到这地步了,但我们的后辈还是一气,宏儿不是正在想念水生么。我希望他们不再像我,又大家隔膜起来……然而我又不愿意他们因为要一气,都如我的辛苦辗转而生活,也不愿意他们都如闰土的辛苦麻木而生活,也不愿意都如别人的辛苦恣睢而生活。他们应该有新的生活,为我们所未经生活过的。
    
    我想到希望,忽然害怕起来了。闰土要香炉和烛台的时候,我还暗自以为他总是崇拜偶像,什么时候都不忘却。现在我所谓希望,不也是我自己手制的偶像么?只是他的愿望切近,我的愿望茫远罢了。
    
    我在朦胧中,眼前展开一片海边碧绿的沙地来,上面深蓝的天空中挂着一轮金黄的圆月。我想:希望本是无所谓有,无所谓无的。这正如地上的路;其实地上本没有路,走的人多了,也便成了路。
    
    \hfill 一九二一年一月
    
\end{normalsize}


\newpage

\textbf{注释}:

\vspace{-1em}

\begin{itemize}
    \setlength\itemsep{-0.2em}
    \item 〔潺潺〕流水声。
    \item 〔瑟索〕不由自主地哆嗦
    \item 〔惘然〕心里迷茫的样子。
    \item 〔连珠〕串联在一起的珠子,形容一个接一个不间断。
    \item 〔应酬〕应答酬谢,指与人交际来往。
    \item 〔恣睢〕放纵张狂。
\end{itemize}

\chapter{想和做}

\begin{normalsize}
    
    有些人只会空想,不会做事。他们凭空想了许多念头,滔滔不绝地说了许多空话,可是从来没认真做过一件事。
    
    也有些人只顾做事,不动脑筋。他们一天忙到晚,做他们一向做惯的或者别人要他们做的事。他们做事的方法只是根据自己的习惯,或者别人的命令,或一般人的通例。自己一向这样做,别人要他们这样做,一般人都这样做,他们就“依葫芦画瓢”,照样做去。到底为什么要做这件事,为什么要这样做,有没有更好的办法,他们从来不想一想。
    
    我们瞧不起前一种人,说他们是“空想家”,可是往往赞美后一种人,说他们能够“埋头苦干”。能够苦干固然是好的,但是只顾埋着头,不肯动动脑筋来想想自己做的事,其实并不值得赞美。
    
    这种埋头做事不动脑筋的人简直是——说得不客气一点——跟牛马一样。拉磨的牛成年累月地在鞭子下绕着石磨转,永远不会想一想为什么要做这件事,为什么要这样做,有没有更好的办法。能够这样想的只有人。人在劳动中不断地动脑筋,想办法,才清清楚楚地知道自己做这件事为什么目的,有什么意义,有什么缺点,才渐渐想出节省劳力,提高效率的方法。人类能够这样劳动,能够一面做,一面想,所以文化能够不断地进步。要不,今天的人类就只能像几万年以前的人类一样,过着最原始最简单的生活了。
    
    一事不做,凭空设想,那是“空想”。不动脑筋,埋头苦干,那是“死做”。无论什么事情,工作也好,学习也好,“空想”和“死做”都不会得到进步。想和做是分不开的,一定要联结起来。
    
    想和做怎样才能够联结起来呢?我们常常听说“从实际出发”这句话,这就是想和做联结起来的一条路。想的时候要从实际出发,就不能“空想”,必须去接近实际。怎样才能够接近实际?当然要观察。光靠观察还不够,还得有行动。举个例子来说,人怎样学会游泳的呢?光靠观察各种物体在水中浮沉的现象,光靠观察鱼类和水禽类\footnote{〔水禽类〕生活在水上或近水的禽类,如鸭、鹅、灰雁等。}的动作,那是不够的;一定要自己跳下水去试验,一次,两次,十次,几十次地试验,才学会了游泳。如果只站在水边,先是一阵子呆看,再发一阵子空想,即使能够想出一大堆“道理”来,自己还是不会游泳,对于别的游泳的人也没有好处。这样空想出来的“道理”其实并不算什么道理。真正的道理是在行动中取得的经验,再根据经验想出来的。而且想出来的道理到底对不对,还得拿行动来证明:行得通的就是对的,行不通的就是错的。
    
    一面做,一面想。做,要靠想来指导;想,要靠做来证明。想和做是紧密地联结在一起的。
    
    在学校里,有些同学很“用功”,可是不会用思想。他们学习语文,就硬读\footnote{〔硬〕不讲方法,不顾一切,强行。}课文。因为只读不想,同一个语言文字上的道理,在这一课里老师讲明白了,出现在别一课里,他们又不理解了。他们学习数学,就硬记公式。因为只记不想,用这个公式算出了一道题,碰到同类的第二道题就又不会算了。从旧经验里得到的道理,不能应用在新事物上,这就是不会用思想的缘故。另外也有些同学,他们能想出些省力的有效的方法,拿来记住动植物的分类,弄清历史的年代。我们固然不赞成为了应付考试想出一些投机取巧的办法;但是我们承认,在学习各种功课和训练记忆力上,是可以有一些比较省力的有效的方法的。这些方法也得从学习的经验中取得。假如只是埋头苦读,不动脑筋想一想,那就得不到。除了学习功课以外,做种种课外活动,也要把想和做联结起来。例如开会,演说,办壁报,组织班会和学术团体,这些实际的行动,如果光凭一腔热情,埋头苦干,不根据已有的成绩和经验,想想怎样才能把这些事情做得更好,更有效果,那么,结果常常会劳而无功。
    
    无论什么人,不管他怎样忙,应该抽点功夫\footnote{〔功夫〕做事所费的精力和时间。}来想一想。想什么?想他自己做过的事,想自己做事得到的经验。这样,他脑子里所有的就不是空想,他的行动也就可以不断地得到进步。
    
\end{normalsize}


\newpage

\textbf{注释}:

\vspace{-1em}

\begin{itemize}
    \setlength\itemsep{-0.2em}
    \item 〔空话〕没有意义的话;对现实无用的话;并未实现的诺言。
    \item 〔滔滔不绝〕像流水一样不断绝。滔滔:形容水势浩大,奔流不息。
    \item 〔一腔热情〕多写作一腔热血,表示满怀着热情。腔:动物身体中空的部分,此处指胸腔(肺),比喻想要喊出来。
    \item 〔凭空〕没有凭据,没有依靠。
    \item 〔投机取巧〕利用心机和伪装,指用不正当的手段。
\end{itemize}

\chapter{畏惧错误就是毁灭进步}

\begin{normalsize}
    
    “畏惧错误就是毁灭进步。”——怀特黑德\footnote{〔怀特黑德〕阿尔弗雷德·怀特黑德,20世纪英国哲学家,提出历程哲学。原句为“畏惧错误毁灭进步,对真理的爱守护进步。”}这句名言,蕴含着丰富的哲理,它给人们——特别是热心改革、勇于创新的人们以深刻的启示。
    
    人们从事各项活动,总是希望获得成功,避免失败。可是实际上往往事与愿违。这原因,从认识论\footnote{〔认识论〕对什么是知识和获取知识的讨论。个人对此的观点和信念。}讲,是由于客观事物的本质并不是显露在外,而是潜藏在内的;不仅如此,它有时还会以颠倒的形式——“假象”出现,就像达尔文有一次半认真地说的那样:“大自然是一有机会就要说谎的。”加上人们认识能力、水平、经验的限制,就难免发生错误了。
    
    用辩证\footnote{〔辩证〕通过对对立矛盾的研究,建立对事物真理的认知。}的观点来看,错误并不可怕,叹息、感伤、畏惧是大可不必的。错误同真理,失败同成功,像睡梦同清醒、黑夜同黎明一样紧密相联。一个人从错误的“梦”中醒来,就会以新的智慧和力量奔向真理。经历着失败的黑夜,正预示了成功的黎明即将来临。黑格尔说过:错误本身乃是“达到真理的一个必然的环节”,“由于这种错误,真理才会出现”。人们的知识、能力以至发明创造,并不单单是在总结成功经验,也是在汲取失败教训的基础上产生出来的。戴维谈到自己获得成功,就说过\footnote{〔戴维……〕汉弗里·戴维,18至19世纪英国化学家,发现了钾、钠、钙、锶等多种化学元素。这句话通常被认为是戴维说的,也符合他的学术研究生涯的特点,但事实上并没有相关记录。}:“我的那些最主要的发现是受到失败的启示而作出的。”
    
    一个人若要有发现,有创造,就不应当畏惧错误。倘若你想把一切错误都关在门外,那你也必将永远被关在真理的门外;倘若你想避免任何失败,那你也必定永远得不到成功。有人说:“若不让错误出生,便不会有真理降世。”这是有道理的。奠定电磁学实验基础的法拉第\footnote{〔法拉第〕迈克尔·法拉第,19世纪物理学家,发现电磁感应现象。},正是由于不怕一而再再而三的失败,经过近十年的艰辛努力,才终于使磁铁产生了电流,开拓了电磁学的新领域。欧立希\footnote{〔欧立希〕保罗·欧立希,19世纪德国生物学家、医学家,化学疗法的先驱。}也以惊人的毅力,在失败了数百次之后,才成功地制成了药品坤凡纳明。无怪德国物理学家普朗克\footnote{〔普朗克〕马克斯·普朗克,19至20世纪德国物理学家,量子力学创始人。}在获取诺贝尔奖金\footnote{〔诺贝尔奖〕1901年起根据瑞典化学家、发明家阿尔弗雷德·诺贝尔的遗嘱设立的奖项。分为物理学、化学、生理学或医学、经济学、文学五个奖,由挪威评议颁发。普朗克于1918年获得诺贝尔物理学奖。}时深有感受地说:“回顾……最后通向发现(量子论)的漫长曲折的道路,我对歌德\footnote{〔歌德〕约翰·歌德,18世纪德国戏剧家、诗人、文艺理论家、政治人物。以《浮士德》等作品闻名。}的话记忆犹新。他说,人们若要有所追求就不能不犯错误。”当然,这绝不是说不要努力去防止和减少错误,或者说可以对错误持满不在乎的态度,而是说不要因为惧怕错误而畏首畏尾,缩手缩脚。这也怕那也怕,是成就不了事业的。
    
    目前,我国正在深入进行体制改革。改革,是破旧创新,没有现成的道路可走,没有定型\footnote{〔定型〕已经有固定形制,可以直接拿来用的。}的模式可依,要靠自己的努力去实践,探索,开拓,创造。在这个过程中是难免出现缺陷和错误的。我们既不应当因为出了点错误便偃旗息鼓,悲观泄气,更不应当因为有了错误就否定改革。你毫不动弹,当然是再保险不过的了,不过,正像鲁迅指出的:“这毫不动弹,却也就是一个大错。”对待错误和失败的正确态度应该是:分析原因,总结教训,找到正确的道路。正是从这个意义上,我们也可以这样说:“畏惧错误就是毁灭改革。”
    
\end{normalsize}


\newpage

\textbf{注释}:

\vspace{-1em}

\begin{itemize}
    \setlength\itemsep{-0.2em}
    \item 〔事与愿违〕事情的发展与愿望相违背。
    \item 〔偃旗息鼓〕放倒旗子,停止击鼓。表示停止战斗,比喻停止做事或声势减弱。偃:仰卧,指平放。
    \item 〔蕴含〕包含。
    \item 〔哲理〕深刻的道理。
\end{itemize}

\chapter{理想的阶梯}

\begin{normalsize}
    
    青年最爱谈理想,青年最苦恼的是理想和现实常常有矛盾。
    
    有的青年虽有理想,但刻苦勤奋不足;有的也很想为理想努力,但不能抓紧一点一滴的时间;有的自以为条件差,岗位平凡,无用武之地,不能充分发挥主观能动作用。结果,常常在碌碌无为的苦闷中慨叹\footnote{〔慨叹〕感慨叹息}蹉跎。
    
    奋斗,是实现理想的阶梯。离开奋斗,理想就只能是幻想而已。有理想的青年,都应从眼前的现实起步,以非常艰苦的奋斗,作为通往理想境界的阶梯。
    
    理想的阶梯,属于刻苦勤奋的人。马克思为实现解放全人类的崇高理想奋斗一生。他积极投身于火热的工人运动,研读无数种著作,学会了欧洲好几个国家的语言。他不断在图书馆钻研,数十年如一日,座位下的地面竟然磨掉一层。化学家诺贝尔\footnote{〔诺贝尔〕阿尔弗雷德·诺贝尔,瑞典化学家、发明家、工程师、企业家。以他的遗嘱成立的诺贝尔奖是科学界、文学界的著名奖项。}为减轻工地上挖土工人的繁重劳动,决心发明炸药。废寝忘食,四年里做了几百次试验。最后一次试验时,他聚精会神地盯着燃延的导火线\footnote{〔导火线〕连着爆炸物的线。点燃一端后,将火沿线传导到爆炸物而引爆。比喻直接引发冲突的事物。}。一声巨响,在旁的人们惊叫:“诺贝尔完了!”诺贝尔却从浓烟中跳出来,面孔乌黑,身上还带着血,兴奋地狂呼:“成功了!”那些杰出的人物正是被一种崇高的目标所鼓舞,才产生了惊人的毅力与忘我的精神。是理想的浪涛激励着他们去刻苦奋斗。今天,我们为实现四化\footnote{〔四化〕四个现代化,指工业、农业、国防、科技现代化。1950年代我党提出的国家战略目标。}而奋斗,这是中华民族空前的事业,其任务之艰,难度之大,更需要亿万人民,特别是青年,百折不回地艰苦奋斗。有志于为这一崇高理想而奋斗的青年要敢于面对现实,不怕一切艰难困苦,不怨天尤人,以凌云的壮志,用刻苦勤奋的汗水浇开灿烂的理想之花。
    
    理想的阶梯,属于珍惜时间的人。富兰克林\footnote{〔富兰克林〕本杰明·富兰克林,18世纪政治家、外交家、科学家、发明家,美利坚合众国创建者之一,共济会成员。}有句名言:“你热爱生命吗?那么别浪费时间,因为时间是组成生命的材料。”许多科学家、文艺家都是同时间赛跑的能手。爱迪生\footnote{〔爱迪生〕托马斯·爱迪生,19至20世纪发明家、企业家。}一生有一千多项发明。这无数次试验的时间从哪里来?就是从常常连续工作两天三天的极度紧张中挤出来的。鲁迅以“时间就是生命”的格言律己,从事无产阶级文学艺术事业三十年,视时间如生命,笔耕不辍。巴尔扎克\footnote{〔巴尔扎克〕奥诺雷·德·巴尔扎克,19世纪现实主义作家,著有《人间喜剧》共91部小说,被誉为当时法国社会的“百科全书”。}用如痴如狂的拼劲,每天奋笔疾书十六七个小时,即使累得手臂疼痛,双眼流泪,也不肯浪费一刻时间。他一生留下为人民深深喜爱的巨著《人间喜剧》,共九十多部小说。这些血汗的结晶不正是时间与生命的光辉记录吗?
    
    时间的流逝是无情的,可怕的。人生七十古来稀\footnote{〔人生七十古来稀〕七十岁的老人自古少有。出自唐代杜甫《曲江二首·其二》。},三分之一要睡去,再除去幼年玩耍的时间,学习与工作大约只有三十几年,一万多天。虚度一日就等于耗费生命的万分之一。朱自清在散文《匆匆》中说:“洗手的时候,日子从水盆里过去;吃饭的时候,日子从饭碗里过去;默默时,便从凝然的双眼前过去。我觉察他去的匆匆了,伸出手遮挽时,他又从遮挽着的手边过去,……”可是,有人甚至从未想过遮挽一下时光呢。对时间的态度,可以检验一个人的世界观。没有理想的人,不懂人生的意义,自然不爱惜时间。真正有理想的人,必定珍惜一分一秒,因为每一瞬间的奋斗都关系着目标的实现。
    
    理想的阶梯,属于迎难而上的人。奋斗的必要,正是由于困难的存在。在通往四化的征途上,坎坷、曲折、荆棘、浪涛是不会少的。幻想一切都顺顺利利,就等于在四化面前止步。有的青年埋怨条件差。这也许是事实。但今天的处境,总不致像伽利略\footnote{〔伽利略〕伽利略·伽利莱,16至17世纪意大利物理学家、天文学家、数学家,欧洲科学革命重要人物,现代物理学奠基者之一。}、布鲁诺\footnote{〔布鲁诺〕乔达诺·布鲁诺,16世纪哲学家、数学家。支持日心说和宇宙无限观,被教会裁判为异端并烧死。}那样冒着受宗教极刑的危险,总不致像高尔基\footnote{〔高尔基〕马克西姆·高尔基,19至20世纪社会主义、现实主义作家、诗人,政治活动家。主要著作为《童年》《在人间》《我的大学》等。}那样在老板的皮鞭下学写作吧。艰苦的环境更能激发有理想的人奋发向上。高尔基从小饱尝人间的辛酸,旧社会血泪的鞭笞铸成了他伟大的心灵。他坚持在敌人的明枪暗箭下写作,在饥饿与死亡的威胁中战斗,为了共产主义事业,不在任何艰难困苦中屈服、畏缩,永远像海燕一样在雷鸣电闪中展翅翱翔。相比之下,我们的困难又算什么呢?有的青年埋怨自己的岗位平凡。这也可能是事实。但革命事业需要三百六十行,绝大多数人都要在平凡岗位上工作。无志之人,将使生命比岗位更平凡;有志之人,将在平凡岗位上成功。华罗庚\footnote{〔华罗庚〕20世纪中国数学家,中国现代数学奠基人。}年轻时在一个中学干杂活,夜间在如豆的昏黄油灯下演算,打下牢固的根基,后来才成为著名的数学家。开普勒\footnote{〔开普勒〕约翰尼斯·开普勒,16至17世纪德国天文学家、数学家。总结了描述天体运动的三大定律。}长期操劳杂役,业余苦钻,发现了行星运动三大定律。道尔顿\footnote{〔道尔顿〕约翰·道尔顿,18至19世纪英国化学家、物理学家,近代原子理论的提出者。}是中学教员,爱因斯坦\footnote{〔爱因斯坦〕阿尔伯特·爱因斯坦,20世纪德国物理学家,创立了相对论和量子力学,现代最伟大的科学家之一。}是小职员,那些发明纺织机、蒸汽机、飞机、火车的,他们的职业、岗位不也都很平凡吗?可见问题不在于岗位,而在于有没有真正的崇高理想和为这理想而奋斗不息的顽强精神。一个有理想有抱负的青年,决不应让困难攫住自己的心灵,而要在奋斗中舒展自己的双臂。当为崇高理想而奋斗一生的双臂收拢时,抱住的必将是令人欣慰的硕果。
    
    奋斗,是改变现实的杠杆,是亿万人民共攀四化高峰的坚实阶梯。只有以不懈的韧劲,一级级攀登,才能一步步接近那光辉灿烂的理想高峰。让我们在四化的伟大征途上谱写出自己的奋斗之歌吧。
    
\end{normalsize}


\newpage

\textbf{注释}:

\vspace{-1em}

\begin{itemize}
    \setlength\itemsep{-0.2em}
    \item 〔蹉跎〕任由时间过去而无作为,浪费时间。
    \item 〔废寝忘食〕顾不得睡觉,忘记吃饭。形容非常专注努力。
    \item 〔荆棘〕泛指山野丛生的多刺灌木。比喻阻挡前进的障碍、艰难局面或纷扰。
    \item 〔笔耕不辍〕坚持勤奋写作。笔耕:用笔耕耘,比喻以写作谋生或勤奋写作。辍:中止。
    \item 〔攫住〕快速有力地抓住。
    \item 〔硕果〕巨大的成果。
\end{itemize}

\chapter{记一辆纺车}

\begin{normalsize}
    
    我曾经使用过一辆纺车\footnote{〔纺车〕把原料纺成纱的设备。},离开延安的那年,把它跟一些书籍一起留在蓝家坪了。后来常常想起它。想起它,就像想起旅途的旅伴,战场的战友,心里充满了深深的怀念。
    
    那是一辆普通的纺车。说它普通,一来它的车架,轮叶\footnote{〔轮叶〕纺车部件,用来转动纱线的轮子。},锭子\footnote{〔锭子〕纺车部件,用来把纤维捻成纱并把纱绕在筒管上成一定形状。},跟一般农村用的手摇纺车没有什么两样;二来它是延安上千上万辆纺车中的一辆。的确,那个时候在延安的人,无论是机关的干部,学校的教员和学员,也无论是部队的指挥员和战斗员,在工作,学习或者练兵的间隙里,谁没有使用过纺车呢?纺车跟战斗用的枪,耕田用的犁,学习用的书和笔一样,成为大家亲密的伙伴。
    
    在延安,纺车是作为战斗的武器使用的。那是在抗日战争最艰苦的时候,国民党反动派发动反共高潮\footnote{〔反共高潮〕1939年春至1943年夏国民党多次对日媾和,与日本共同反共的行动。},配合日寇重重封锁陕甘宁边区\footnote{〔陕甘宁边区〕陕西省北部、甘肃省和宁夏省的东部合称,我党的革命抗日根据地。},想困死我们。我们边区军民热烈响应毛泽东同志的伟大号召:“自己动手,丰衣足食”,结果彻底粉粹了敌人围困的阴谋。在延安的人,在所有抗日根据地的人,不但吃得饱,而且穿得暖,坚持了抗战,争取到了抗战的最后胜利。开荒,种庄稼,种蔬菜,是保证足食的战线;纺羊毛,纺棉花,是保证丰衣的战线。
    
    大家用纺的毛线织毛衣,织呢子\footnote{〔呢子〕一种较厚较密的毛织品,多用来做制服、大衣等。};用纺的棉纱合线\footnote{〔合线〕把细纱线拧合成一定粗细的线。},织布。同志们穿的衣服鞋袜,有的就是自己纺线或者跟同志换工劳动做成的。开垦南泥湾\footnote{〔南泥湾〕延安城东南45公里的荒地。1940年,王震率领三五九旅开垦南泥湾,展开大生产运动。}的部队甚至能够在打仗练兵和进行政治、文化学习而外,纺毛线给指战员发军装呢。同志们亲手纺线织布做的衣服,穿着格外舒适,也格外爱惜。那个时候,人们对一身灰布制服,一件本色的粗毛线衣,或者自己打的一副手套,一双草鞋,都很有感情。衣服旧了,也“敝帚自珍”,不舍得丢弃。总是脏了洗洗,破了补补,穿一水又穿一水,穿一年又穿一年。衣服只要整齐干净,越朴素穿着越随心。西装革履,华丽的服饰,只有在演剧的时候作演员的服装,平时不要说穿,就是看看也觉得碍眼。美的概念里是更健康的内容,那就是整洁,朴素,自然。
    
    纺线,劳动量并不太小,纺久了会胳膊疼腰酸;不过在刻苦学习和紧张工作的间隙里纺线,除了经济上对敌斗争的意义而外,也是一种很有兴趣的生活。在纺线的时候,眼看着匀净的毛线或者棉纱从拇指和食指中间的毛卷里或者棉条里抽出来,又细又长,连绵不断,简直会有一种艺术创作的快感。摇动的车轮,旋转的锭子,争着发出嗡嗡嘤嘤的声音,像演奏弦乐,像轻轻地唱歌。那有节奏的乐音和歌声是和谐的,优美的。
    
    纺线也需要技术。车摇慢了,线抽快了,线会断头;车摇快了,线抽慢了,毛卷、棉条会拧成绳,线全打成结。摇车,抽线,配合恰当,成为熟悉的技巧,可不简单,需要用很大的耐心和毅力下一番功夫。初学纺线,往往不知道劲往哪儿使。一会儿毛卷拧成绳了,一会儿棉纱打成结了,纺手急得满头大汗。性子躁一些的人甚至为断头接不好生纺车的气,摔摔打打,恨不得把纺车砸碎。可是那关纺车什么事呢?尽管人急得站起来,坐下去,一点也没有用,纺车总是安安稳稳地待在那里,像露出头角的蜗牛,像着陆停驶的飞机,一声不响,仿佛只是在等待,等待。一直等到使用纺车的人心平气和了,左右手动作协调,用力适当,快慢均匀了,左手拇指和食指间的毛线或者棉纱就会像魔术家帽子里的彩绸一样无穷无尽地抽出来。那仿佛不是用羊毛、棉花纺线,而是从毛卷里或者棉条里往外抽线。线是现成的,早就藏在毛卷里或者棉条里的。熟练的纺手,趁着一豆灯光或者朦胧的月光,也能摇车,抽线,上线,一切做得从容自如。线上在锭子上,线穗子\footnote{〔线穗子〕纺纱时绕在锭子上成团的纱线。}就跟着一层层加大,直到沉甸甸的,像成熟了的肥桃。从锭子上取下穗子,也像从果树上摘下果实,劳动后收获的愉快,那是任何物质享受都不能比拟的。这个时候,就连起初想砸碎纺车的人也对纺车发生了感情。那种感情,是凯旋的骑士对战马的感情,是“仰手接飞猱,俯身散马蹄”\footnote{〔“仰手接飞猱,俯身散马蹄”〕出自曹植的《白马篇》。形容技艺高超。猱:一种猴子。马蹄:箭靶。}的射手对良弓的感情。
    
    纺线有几种姿势:可以坐着蒲团纺,可以坐着矮凳纺,也可以把纺车垫得高高的站着纺。站着纺线,步子有进有退,手臂尽量伸直,像“白鹤晾翅”\footnote{〔“白鹤晾翅”〕太极拳的招式,两手上举,如鹤展开双翅。},一抽线能拉得很长很长。这样气势最开阔,肢体最舒展;兴致高的时候,很难说那是生产,是舞蹈,还是体育锻炼。
    
    为了提高生产率,大家也进行技术改革,运用物理学上轮轴\footnote{〔轮轴〕上有轮子旋转的枢轴。}和摩擦传动的道理,在轮子和锭子中间安装加速轮,加快锭子旋转的速度,把手工生产的工具变成半机械化。大多数纺车是在纺羊毛、纺棉花的劳动实践中培养出来的木工做的;安装加速轮也是在劳动实践中大家摸索出来的创造发明。从劳动实践中还不断总结出一些新的经验。譬如,纺羊毛跟纺棉布常有不同的要求:羊毛要松一些,干一些;棉花要紧一些,潮一些。因此弹过的羊毛要卷成卷,棉花要搓条,烘晒毛卷和阴润棉条都有一定的火候分寸。这些技术经验,不靠实践是一辈子也不知道里边的奥妙的。
    
    为了交流经验,互相提高,纺线也开展竞赛。三五十辆或者百几十辆纺车搬在一起,在同一个时间里比纺线的数量和质量。成绩好的有奖励,譬如,奖一辆纺车,奖手巾、肥皂、笔记本之类。那是很光荣的。更光荣的是被称为纺毛突击手、纺纱突击手。竞赛,有的时候在礼堂,有的时候在窑洞\footnote{〔窑洞〕}前边,更有的时候在山根河边的坪坝\footnote{〔坪坝〕山间平整的场地。}上。在坪坝上竞赛的那种场面最壮阔,“沙场秋点兵”\footnote{〔“沙场秋点兵”〕出自宋代辛弃疾的《破阵子·为陈同甫赋壮词以寄之》。沙场:战场。点兵:检阅军队。}或者能有那种气派?不,阵容相近,热闹不够。那是盛大的节日里赛会的场面。只要想想:天地是厂房,深谷是车间,幕天席地,群山环拱,怕世界上还没有哪个地方哪种轻工业生产有那样的规模哩。你看,整齐的纺车行列,精神饱满的纺手队伍,一声号令,百车齐鸣,别的不说,只那嗡嗡的响声就有点像飞机场上机群起飞,扬子江\footnote{〔扬子江〕长江的旧称。}边船只拔锚。那哪儿是竞赛,那是万马奔腾,在共同完成一项战斗任务。因此竞赛结束,无论是纺得多的还是纺得比较少的,得奖的还是没有得奖的,大家都感到胜利的快乐。
    
    就这样,用劳动的双手,自力更生。纺线,不只在经济上保证了革命根据地的人大家有衣穿,使大家学会了一套生产劳动的本领,而且在思想上还教育了大家认识劳动“本身成了生活的第一需要”的意义;自觉地克服了那种“认为劳动只是一种负担,凡是劳动都应当付给一定报酬的习惯”。劳动为集体,同时也为自己。在劳动的过程里,很少人为了个人的什么去锱铢计较;倒是为集体做了些什么有意义的事情,才感到是真正的幸福。
    
    就因为这些,我常常想起那辆纺车。想起它像想起老朋友,心里充满了深深的怀念。围绕着这种怀念,也想起延安的种种生活。在党中央和毛泽东同志的周围工作,学习,劳动,同志的友谊,革命大家庭的温暖,把大家团结得像一个人。真是既团结,紧张,又严肃,活泼。那个时候,物质生活曾经是艰苦的、困难的吧,但是,比起无限丰富的精神生活来,那算得了什么!凭着崇高的理想、豪迈的气概、乐观的志趣,克服困难不也是一种享受吗?
    
    跟困难作斗争,其乐无穷。
    
\end{normalsize}


\newpage

\textbf{注释}:

\vspace{-1em}

\begin{itemize}
    \setlength\itemsep{-0.2em}
    \item 〔凯旋〕打仗胜利归来。凯:打胜仗回来奏的乐曲。旋:回归。
    \item 〔幕天席地〕把天做幕,把地当席,指露天。
    \item 〔自力更生〕用自己的力量办事、解决问题。
    \item 〔报酬〕
    \item 〔锱铢〕
\end{itemize}

\chapter{生物的入侵者}

\begin{normalsize}
    
    当你在路边草地或自家庭院里发现一两只从未见过的甲虫时,你肯定不会感到惊讶。但在生物学家和生态学家们看来,这或许不是件寻常小事。专家们把这种原本生活在异国他乡、通过非自然途径到新的生态环境中的“移民”称为“生物入侵者”——它们不仅会破坏某个地区原有的生态系统,而且还可能给人类社会造成难以估量的经济损失。
    
    在人类文明的早期,陆路和航海技术尚不发达,自然界中的生态平衡并没有受到太大的破坏。在自然条件下,一颗蒲公英的种子可能随风飘荡几千米后才会落地,如果各种条件适合,它会在那里生根、发芽、成长;山间溪水中的鱼虾可能随着水流游到大江大河中安家落户……凡此种种,都是在没有人为干预的条件下缓慢进行的,时间和空间跨度都非常有限,因此不会造成生态系统的严重失衡。
    
    如果一种物种在新的生存环境中不受同类的食物竞争以及天敌伤害等诸多因素制约,它很可能会无节制地繁殖。1988年,几只原本生活在欧洲大陆的斑贝(一种类似河蚌软体动物)被一艘货物带到北美大陆。当时,这些混杂在仓底货物中的“偷渡者”并没有引起当地人的注意,它们被随便丢弃在五大湖附近的水域中。然而令人始料不及的是,这里竟成了斑贝的“天堂”。由于没有天敌的制约,斑贝的数量便急剧增加,五大湖内的疏水管道几乎全被它们“占领”了。到目前为止,人们为了清理和更换管道已耗资数十亿美元。来自亚洲的天牛和南美的红蚂蚁是另外两种困扰美国人的“入侵者”,前者疯狂破坏芝加哥和纽约的树木,后者则专门叮咬人畜,传播疾病。
    
    “生物入侵者”在给人类造成难以估量的经济损失的同时,也对被入侵地的其他物种以及物种多样性构成极大威胁。二战期间,棕树蛇随一艘军用货船落户关岛,这种栖息在树上的爬行动物专门捕食鸟类,偷袭鸟巢,吞食鸟蛋。从二战至今,关岛本地的11种鸟类中已有9种被棕树蛇赶尽杀绝,仅存的两种鸟类的数量也在与日俱减,随时有绝种的危险。一些生物学家在乘坐由关岛飞往夏威夷岛的飞机上曾先后6次看到棕树蛇的身影。他们警告说,夏威夷岛上没有任何可以扼制棕树蛇繁殖的天敌,一旦棕树蛇在夏威夷安家落户,该岛的鸟类将在劫难逃。许多生物学家和生态学家将“生物入侵者”的增多归罪于日益繁荣的国际贸易,事实上许多“生物入侵者”正是搭乘跨国贸易的“便车”达到“偷渡”的目的的。以目前全球新鲜水果和蔬菜贸易为例,许多昆虫和昆虫的卵附着在这些货物上,其中包括危害极大的害虫,如地中海果蝇等。尽管各国海关动植物检疫中心对这些害虫严加防范,但由于进出口货物数量极大,很难保证没有漏网之“虫”。此外,跨国宠物贸易也为“生物入侵者”提供了方便。近年来,由于引进五彩斑斓的观赏鱼而给某些地区带来霍乱病源的消息时常见诸报端。一些产自他乡的宠物,如蛇、蜥蜴、山猫等,往往会因主人的疏忽或被遗弃而逃出,为害一方。
    
    一些生物学家指出,一旦某种“生物入侵者”在新的环境中站稳脚跟并大规模繁殖,其数量将很难控制。即使在科学技术高度发达的今天,面对那些适应能力和繁殖能力极强的动植物,人们仍将束手无策。
    
    生物学和生态学界的一些学者主张人类不应该过多地干预生物物种的迁移过程,因为失衡是暂时的,一个物种在新的环境中必然遵循物竞天择的法则。“生物入侵者”并不是都能够生存下来,能够生存下来的就是强者,即使生态系统中的强者也同样受到该系统中各种因素的制约,不可能为所欲为,因此,自然界的平衡最终会得以实现。然而更多的学者则持反对意见,他们认为自然调节的过程是非常漫长的,如果听任“生物入侵者”自由发展,许多本土物种将难逃绝种厄运,自然界的物种多样性将受到严重破坏。另外,“生物入侵者”给人类社会造成的经济损失是惊人的。仅在美国,每年由“生物入侵者”造成的经济损失就高达两千多亿美元,面对这样的天文数字,人们岂能无动于衷?
    
    目前,世界上许多国家已开始认识到这一问题的严重性,并采取了相应的措施。例如,美国众议院已于1996年通过了一项议案,要求各有关机构加强对有可能夹带外来物种的交通工具的监控,为此,美国政府正在酝酿一个跨部门的监控计划。
    
\end{normalsize}



\chapter{怀疑与学问}

\begin{normalsize}
    
    \begin{quotation}
    
    学者先要会疑。\\
    
    \hfill ——程颐
    
    \end{quotation}
    
    \begin{quotation}
    
    在可疑而不疑者,不曾学;学则须疑。\\
    
    \hfill ——张载
    
    \end{quotation}
    
    学问的基础是事实和证据。事实和证据的来源有两种:一种是自己亲眼看见的,一种是听别人传说的。譬如在国难危急的时候,各地一定有许多口头的消息,说得如何凶险,那便是别人的传说,不一定可靠。 要知道实际的情形,只有靠自己亲自视察。做学问也是这样,最要紧最可靠的材料是自己亲见的事实证据;但这种证据有时候不能亲自见到,便只能靠别人的传说了。
    
    我们对于传说的话,不论信不信,都应当经过一番思考,不应当随随便便就信了。我们信它,因为它“是”;不信它,因为它“非”。 这一番事前的思索,不肯随便轻信,便是怀疑的精神,做一切学问的基本条件。我们听说古代有三皇五帝\footnote{〔三皇五帝〕出自《周礼·春官·外史》:“外史……掌三皇五帝之书。”},便要问:这是谁说的话?最先见于何书? 书是何时人著的?著者何以知道? 我们又听说“腐草为萤”\footnote{〔腐草为萤〕出自《礼记·月令》。原句为“季夏三月,腐草为萤”。},便要问:死了的植物如何会变飞动的甲虫?有什么科学根据? 我们若能这样追问,一切虚妄的学说便不攻自破了。
    
    我们对于不论哪一本书,哪一种学问,都要先经过怀疑,因怀疑而思索,因思索而辨别是非。经过怀疑、思索、辨别三个步骤以后,那本书才是我的书,那种学问才是我的学问。否则便是盲从,便是迷信。孟子所谓“尽信书不如无书”\footnote{〔尽信书不如无书〕出自《孟子·尽心下》。原句为“尽信《书》,则不如无《书》”,对《尚书》表示怀疑。},也就是教我们有一点怀疑的精神,不要随便盲从或迷信。
    
    怀疑不仅是消极方面辨伪去妄的必须步骤,也是积极方面建设新学说、启迪新发明的基本条件。对于别人的话,都不打折扣地承认,那是思想上的懒惰。 这样的脑筋永远是被动的,永远不能做学问。只有常常怀疑、常常发问的脑筋才有问题,有问题才想求解答。在不断的发问和求解中,一切学问才会发展起来。许多大学问家、大哲学家都是从怀疑中锻炼出来的。清代的一位大学问家——戴震 ,幼时读朱子的《大学·章句》,便问《大学》是何时的书,朱子是何时的人。塾师告诉他《大学》是周代的书,朱子是宋代的大儒,他便问宋代的人如何能知道一千多年前的著者的意思\footnote{〔戴震……〕出自段玉裁《戴东原先生年谱》。戴震,18世纪学者。}。法国的大哲学家笛卡儿也说:“我怀疑,所以我存在。”\footnote{〔……我怀疑,所以我存在〕勒内·笛卡尔,17世纪法国哲学家、数学家,西方现代哲学奠基人之一,唯心主义、理性主义开拓者。为了找出绝对可靠的知识作为思考的起点,笛卡尔提出“我怀疑,所以我存在”,他解释为“我怀疑我的存在,这证明我在思考,因此证明在思考的我必然存在。”}他的哲学就建设在对于万事万物的怀疑和明辨上。一切学问家,不但对于流俗传说,就是对于过去学者的学说也常常要抱怀疑的态度,常常和书中的学说辩论,常常评判书中的学说,常常修正书中的学说。要这样才能有更新更善的学说产生。古往今来,科学上新的发明,哲学上新的理论,美术上新的作风,都是这样来的。 如果后来的学者都墨守前人的旧说,那就没有新问题,没有新发明,一切学术就停滞了,人类的文化也就不会进步了。
    
\end{normalsize}



\chapter{发问的精神}

\begin{normalsize}
    
    我们日常所见所闻所接触的事物里,有很多的道理。大家因为时常见到听到接触到,都觉得那些事物平淡无奇,不足介意。其实这是一种损失。
    
    事物里的道理,不比课本的文字,教师的讲解,看了听了就懂。这种道理犹如封锁在密库石室里的珍奇,我们要用一把钥匙去开启。
    
    这把钥匙就是发问的精神。
    
    发问是思想的初步,研究的动机。一切知识的获得,大都从发问而来;新发明、新创造也常常由发问开端。能发问,勤发问的人,头脑自然会日益丰富,眼光自然会日益敏锐。别人不肯动脑筋的地方,他偏会想出惊人的见解;别人以为平常的事物,他偏会看出不平常的道理。这样的人,古今中外都有的是。
    
    苹果落地,是多么平常的事情,牛顿\footnote{〔牛顿〕艾萨克·牛顿,17世纪英国物理学家、数学家、哲学家。发明了微积分,发现万有引力定律,创立了经典力学三大定律。}看见了,却要问出个所以然来,结果创立了“万有引力”说,支配了人类思想。
    
    壶水滚沸,谁不常常见到?只有瓦特\footnote{〔瓦特〕詹姆斯·瓦特,18世纪英国发明家、机械工程师,发明了可实用的蒸汽机,奠定了工业革命的基础。}把它当作问题研究,因而发明了蒸汽机,使人类至今蒙受其利。
    
    生、老、病、死,都是极普遍的人生现象,释迦牟尼\footnote{〔释迦牟尼〕乔达摩·悉达多,通称释迦牟尼,公元前5世纪南亚思想家,佛教创始人。}偏偏要寻根究底,求个解答。他因此抛弃尊位和家庭,独自去潜修静想,终于创立了佛教。
    
    我们中国的孔子\footnote{〔孔子〕春秋末期哲学家、思想家、政治活动家,儒家思想创始人和代表人物。言行被其弟子门人编纂为《论语》。},也是个好问的人。他到太庙\footnote{〔太庙〕中国古代帝王的宗庙,祭祀帝王的祖先。}里去,看见每样事物都要问。他知道老子\footnote{〔老子〕春秋时期哲学家、思想家,道家思想创始人,留有著作《道德经》。}熟悉典籍,就去向老子问礼。他能够成为万世景仰的圣人,难道真是天生的?
    
    够了,够了,不必多举了。举出这几个人物,无非要证明发问精神的可贵。我们虽然不一定人人能成为大科学家大思想家,但是我们不能不求知识,不能不明道理。要求知识,明道理,处处都会碰到问题。不能勤于发问,是多么可惜啊!
    
    有人也许会说,知识和道理,可以从书上读到,老师也会给我们讲解,只要努力学习,牢牢记住就成了,何必多问?说这话,大概自以为是。可是仔细想想,就会看出漏洞。
    
    第一,书本的记载,老师的讲解,大都是前人思想和研究的结果。可是世界是天天在变动,在进步的。变动和进步又不断地给我们带来许多新道理,新知识,新问题,往往不是前人留下的知识和道理所能包含的,有时甚至彼此冲突。假使墨守旧说,不能发问,那怎样能研究,文明又怎样能进步呢?
    
    其次,知识和道理,是各个人研究出来的。各个人或少数人的见识有限,不免要弄出错误来。而各种知识和道理,经过传播,往往会有歪曲和失实的地方。我们若是一味吸取,不去审问,岂不会把错的、伪的当作了对的、真的,使那些对的、真的反而永不可见?再说,即使我们所吸取的都是真的、对的,经过反复的审问,真的自会益见其真\footnote{〔益〕越,更加。},对的自会益见其对。这样一来,我们可以得到切实和透彻的了解,获得正确的定论。
    
    更进一步说,求知识,明道理,不光是懂得了、记住了就算完事,更要紧的,是把知识和道理贯穿到生活和习惯里去。必须这样,知识和道理才能让我们终身受用,才不会失去学习的价值。怎样贯穿到生活和习惯里去呢?第一步就要把书本上看到的,教师那里听到的,同实际生活里的事物参照比较。在参照比较中,发问是最重要的。发现的问题越多,对于事物一定看得越清楚;同时对于自己的所学也一定更有把握,知道怎样把它安排到生活里去。
    
    你忍心让你的智慧之门永闭吗?你愿意永远盲目地让别人带着你走吗?你愿意永远只做两脚书柜\footnote{〔两脚书柜〕比喻空有知识却不理解、不懂得运用的人。}吗?假如你的答复是否定的,那么,你万不可忘记带那把钥匙,你得能发问,勤发问。
    
\end{normalsize}


\newpage

\textbf{注释}:

\vspace{-1em}

\begin{itemize}
    \setlength\itemsep{-0.2em}
    \item 〔珍奇〕珍贵奇特的物品。
    \item 〔动机〕做某事的理由。
    \item 〔日益〕越来越,每天都比前一天更加……。
    \item 〔万世景仰〕被长久地敬佩仰慕。景:慕。
    \item 〔寻根究底〕寻求和追究事物的原由。
    \item 〔典籍〕记载法令、制度等的重要文献。典:标准、规则。籍:书册。
    \item 〔墨守〕战国时墨子善于守城。后指固执保守,不会变通。
    \item 〔盲目〕没有明确目标,对事物认识不清,做事没有主见没有计划,仿佛突然眼盲了。
    \item 〔贯穿〕从头到尾穿过。贯:古代串钱的绳索。
\end{itemize}

\chapter{七根火柴}

\begin{normalsize}
    
    天亮的时候,雨停了。
    
    草地的气候就是奇怪,明明是月朗星稀的好天气,忽然一阵冷风吹来,浓云像从平地上出来似的,霎时把天遮得严严的,接着,暴雨夹杂着栗子般大的冰雹,不分点地倾泻下来\footnote{〔不分点〕不间断。}。
    
    卢进勇从树丛里探出头来,四下里望了望。整个草地都沉浸在一片迷蒙的雨雾里,看不见人影,听不到人声。
    
    被暴雨冲洗过的荒草,像用梳子梳理过似的,躺倒在烂泥里,连路也给遮没了。天,还是阴沉沉的,偶尔还有几颗冰雹洒落下来,打在那浑浊的绿色水面上,溅起一朵朵浪花。他苦恼地叹了口气。因为小腿伤口发炎,他掉队了。两天来,他日夜赶路,原想在今天赶上大队的,却又碰上了这倒霉的暴雨,耽误了半个晚上。
    
    他咒骂着这鬼天气,从树丛里钻出来,长长地伸了个懒腰。一阵凉风吹得他连打了几个寒颤。他这才发现衣服完全湿透了。
    
    “要是有堆火烤,该多好啊!”他使劲绞着衣服,望着那顺着裤脚流下的水滴想道。他也知道这是妄想——不但是现在,就在他掉队的前一天,他们连里已经因为没有引火的东西而只好吃生干粮了。他下意识地把手插进裤袋里,意外地,手指触到了一点黏黏的东西。他心里一喜,连忙蹲下身,把裤袋翻过来。果然,在裤袋底部粘着一小撮青稞面粉;面粉被雨水一泡,成了稀糊了。他小心地把这些稀糊刮下来,居然有鸡蛋那么大的一团。他吝惜地捏着这块面团,心里不由得暗自庆幸:“幸亏昨天早晨没有发现它们。已经一昼夜没有吃东西了,这会儿看见了可吃的东西,更觉饿得难以忍受。为了不致一口吞下去,他把面团捏成了长条。正要把它送到嘴边,突然听见一声低低的叫声:“同志——”
    
    这声音那么微弱、低沉,就像从地底下发出来的。他略微愣了一下,便一瘸一拐地向着那声音走去。卢进勇蹒跚地跨过两道水沟,来到一棵小树底下,才看清楚那个打招呼的人。他倚着树杈半躺在那里,身子底下是一汪浑浊的污水,看来已经有很长时间没有挪动了。他的脸色更是怕人,被雨打湿了的头发粘贴在前额上,雨水沿着头发、脸颊滴滴地流着。眼眶深深地塌陷下去,眼睛努力地闭着,只有腭下\footnote{〔腭下〕指下巴后方。}的喉结在一上一下地抖动,干裂的嘴唇一张一翕地发出低低的声音:“同志——同志——”
    
    听见卢进勇的脚步声,那个同志吃力地张开眼睛,挣扎了一下,似乎想坐起来,但动不了。
    
    卢进勇看着这情景,眼睛里像揉进了什么\footnote{〔揉进了什么〕指仿佛揉进了沙子而流泪。},一阵酸涩。在掉队的两天里,他这已经是第三次看见战友倒下来了。“一定是饿坏了!”他想,连忙抢上一步,搂住那个同志的肩膀,把那点青稞面递到那同志的嘴边说:“同志,快吃点吧!”
    
    那同志抬起失神的眼睛,呆滞地望了卢进勇一眼,吃力地举起手推开他的胳膊,嘴唇翕动了好几下,齿缝里挤出了几个字:“不,没……没用了。”
    
    卢进勇一时不知怎么好。他望着那张被寒风冷雨冻得乌青的脸,和那脸上挂着的雨滴,痛苦地想:“要是有一堆火,有一杯热水,也许他能活下去!”他抬起头,望望那雾蒙蒙的远处,随即拉住那同志的手腕说:“走,我扶你走吧。”那同志闭着眼睛摇了摇头,没有回答,看来是在积攒着浑身的力量。好大一会儿,他忽然睁开了眼,右手指着自己的左腋窝,急急地说:“这……这里!”
    
    卢进勇惶惑地把手插进那湿漉漉的衣服。他觉得那同志的胸口和衣服一样冰冷了,在左腋窝里,他摸出了一个硬硬的纸包,递到那个同志的手里。
    
    那同志一只手抖抖索索地打开了纸包,那是一个党证,揭开党证,里面并排摆着一小堆火柴,干燥的火柴。
    
    红红的火柴头聚集在一起,正压在那朱红的印章的中心,像一簇火焰在跳。
    
    “同志,你看着……”那同志向卢进勇招招手,等他凑近广使伸开一个僵直的手指,小心翼翼地一根根拨弄着火柴.口里小声数着:“一,二,三,四……”’一共只有七根火柴,他却数了很长时间。数完了,又向卢进勇望了一眼,意思好像说:“看明白了?”
    
    “是,看明白了!”卢进勇高兴地点点头,心想:这下子可好办了!他仿佛看见了一个通红的火堆,他正抱着这个同志偎依在火旁……
    
    就在这一瞬间,他发现那个同志的脸色好像舒展开来,眼睛里那死灰般的颜色忽然不见了,发射出一种喜悦的光。那同志合拢了夹着火柴的党证,双手捧起,像擎着一只贮满水的碗一样,小心地放到卢进勇的手里,紧紧地把它连手握在一起,两眼直直地盯着卢进勇的脸。
    
    “记住,这,这是,大家的!”他蓦地抽回手去,深深地吸了一口气,用尽所有的力气举起手来,直指着正北方向:“好,好同志……你……你把它带给……’,话就在这里停住了。卢进勇觉得自己的臂弯猛然沉了下去!他的眼睛模糊了。远处的树、近处的草、那湿漉漉的衣服、那双紧闭的眼睛……一切都像整个草地一样,雾蒙蒙的;只有那只手是清晰的,它高高地攀着,像一只路标,笔直地指向长征部队前进的方向……
    
    这以后的路,卢进勇走得特别快。天黑的时候,他追上了后卫部队。
    
    在无边的暗夜里,一簇簇的黄火烧起来了。在风雨中、在烂泥里跃滚了几天的战士们,围着这熊熊的野火谈笑着,湿透的衣服上冒起一层雾气,洋瓷碗\footnote{〔洋瓷〕搪瓷,涂烧在金属底坯表面上的无机玻璃瓷釉。}里的野菜“前南”地响着……
    
    卢进勇悄悄走到后卫连指导员的身边。映着那闪闪跳动的火光,他用颤抖的手指打开了那个党证,把剩下的六根火柴一根根递到指导员的手里,何时,以一种异样的声调在数着:“一,二,三,四……”
    
\end{normalsize}


\newpage

\textbf{注释}:

\vspace{-1em}

\begin{itemize}
    \setlength\itemsep{-0.2em}
    \item 〔蹒跚〕走动迟缓、摇晃不稳的样子。
    \item 〔惶惑〕慌乱,惊慌害怕而迷茫。
    \item 〔霎时〕立刻,极短时间内。
    \item 〔抖抖索索〕不停地颤抖。
    \item 〔一张一翕〕一张一合。
    \item 〔翕动〕一张一合地动。
    \item 〔一瘸一拐〕行走时无法保持身体平衡,双腿动作不自然。
    \item 〔脸颊〕脸的两侧。
    \item 〔蓦地〕突然地,没有预兆地。
\end{itemize}

\chapter{最后一次的讲演}

\begin{normalsize}
    
    这几天,大家晓得,在昆明出现了历史上最卑劣,最无耻的事情!李先生\footnote{〔李先生〕指李公朴。李公朴:中国社会教育家,中国民主同盟的发起人之一。}究竟犯了甚么罪?竟遭此毒手,他只不过用笔写写文章,用嘴说说话,而他所写的,所说的,都无非是一个没有失掉良心的中国人的话!大家都有一只笔有一张嘴,有什么理由拿出来讲啊!有事实拿出来说啊!为什么要打要杀,而且又不敢光明正大的来打来杀,而偷偷摸摸的来暗杀!这成什么话?
    
    今天,这里有没有特务!你站出来,是好汉的站出来!你出来讲!凭什么要杀死李先生?杀死了人,又不敢承认,还要诬蔑人,说什么“桃色案件”\footnote{〔“桃色案件”〕云南省警备代理总司令霍揆彰暗杀李公朴后,指示散播谣言。},说什么共产党杀共产党,无耻啊!无耻啊!这是某集团的无耻,恰是李先生的光荣!李先生在昆明被暗杀,是李先生留给昆明的光荣!也是昆明人的光荣!
    
    去年“一二·一”\footnote{〔“一二·一”〕1945年10月12月1日,国民党特务镇压云南大学反战运动,杀死师生四人,称为“一二·一”惨案。}昆明青年学生为了反对内战,遭受屠杀,那算是年青的一代,献出了他们的血,献出了他们最宝贵的生命!现在李先生为了争取民主和平,而遭受了反动派的暗杀,我们骄傲一点说,这算是像我这样大年纪的一代,我们的老战友,献出了最宝贵的生命。这两桩事发生在昆明,这算是昆明无限的光荣!
    
    反动派暗杀李先生的消息传出后,大家听了都摇头。我心里想,这些无耻的东西,不知他们是怎么想法?他们的心理是什么状态?他们的心是怎么长的?其实很简单,他们这样疯狂的来制造恐怖,正是他们自己在慌啊!在害怕啊!所以他们制造恐怖,其实是他们自己在恐怖啊!特务们,你们想想,你们还有几天,你们完了,快完了!你们以为打伤几个,杀死几个,就可以了事,就可以把人民吓倒了吗?其实广大的人民是打不尽的,杀不完的,要是这样可以的话,世界上早没有人了。你们杀死了一个李公朴,会有千百万个李公朴站起来!你们将失去千百万的人民!你们看着我们人少,没有力量。告诉你们,我们的力量大得很!多得很!看今天来的这些人,都是我们的人,都是我们的力量!此外还有广大的市民!我们有这个信心:人民的力量是要胜利的,真理是永远存在的。历史上没有一个反人民的势力不被人民毁灭的!希特勒\footnote{〔希特勒〕阿道夫·希特勒,纳粹德国元首,法西斯主义者,发动了第二次世界大战。},莫索里尼\footnote{〔莫索里尼〕贝尼托·墨索里尼,二战时期意大利的独裁者,法西斯主义者。}不都在人民之前倒下去了吗?翻开历史看看,你还站得住几天!你完了,快完了!我们的光明就要出现了。我们看,光明就在我们的眼前,而现在正是黎明之前那个最黑暗的时候。我们有力量打破这个黑暗,争到光明!我们的光明,就是反动派的末日!
    
    李先生的血,不会白流的。李先生赔上了这条性命,我们要换来一个代价。“一二·一”四烈士倒下了,年青的战士们的血,换来了政治协商会议\footnote{〔政治协商会议〕指按照国共《双十协定》,于1946年1月在重庆召开的会议。}的召开,现在李先生倒下了,他的血要换取政协会议的重开!我们有这个信心!
    
    “一二·一”是昆明的光荣,是云南人民的光荣。云南有光荣的历史,远的如护国\footnote{〔护国〕指护国战争,1915年南方各省反对袁世凯复辟帝制的运动。},这不用说了,近的如“一二·一”,都是属于云南人民的,我们要发扬云南光荣的历史!
    
    反动派挑拨离间,卑鄙无耻,你们看见联大\footnote{〔联大〕指国立西南联合大学,由抗战时期迁移到云南的多所大学联合而成。1946年5月联大宣布结束,师生开始回迁。}走了,学生放暑假了,便以为我们没有力量了吗?特务们!你们错了!你们看看今天到会的一千多青年,又握起手来了,我们昆明的青年决不会让你们这样横干下去的!
    
    正义是杀不完的,因为真理永远存在!
    
    历史赋予昆明的任务是争取民主和平,我们昆明的青年必须完成这任务!
    
    我们不怕死,我们有牺牲的精神,我们随时像李先生一样,前脚跨出大门,后脚就不准备再跨进大门!
    
\end{normalsize}


\newpage

\textbf{注释}:

\vspace{-1em}

\begin{itemize}
    \setlength\itemsep{-0.2em}
    \item 〔挑拨离间〕搬弄是非,制造矛盾,煽动仇恨,破坏团结。
    \item 〔诬蔑〕捏造事实来损害别人的名誉。
    \item 〔赋予〕给予,交给,寄托。
\end{itemize}

\chapter{藤野先生}

\begin{normalsize}
    
    东京\footnote{〔东京〕日本首都。}也无非是这样。上野\footnote{〔上野〕这里指东京台东区的上野公园。}的樱花烂熳\footnote{〔烂熳〕烂漫。}的时节,望去确也象绯红的轻云,但花下也缺不了成群结队的“清国留学生”的速成班\footnote{〔“清国留学生”的速成班〕指清末到日本留学,先在东京弘文书院速成班学习日语的中国学生。},头顶上盘着大辫子,顶得学生制帽的顶上高高耸起,形成一座富士山\footnote{〔富士山〕日本第一高峰,位于本州岛中南部,为活火山,山体圆锥形,高3600米。在日本文化中有重要地位。}。也有解散辫子,盘得平的,除下帽来,油光可鉴\footnote{〔油光可鉴〕指油亮得像镜子一样可以照人。},宛如小姑娘的发髻一般,还要将脖子扭几扭。实在标致极了。
    
    中国留学生会馆\footnote{〔会馆〕古代同乡、同业的人在京城等大都市、大商埠设立的机构,为同乡、同业提供聚会场所和住宿。这里指设在东京供中国留学生活动、居住的场所。}的门房里有几本书买,有时还值得去一转;倘在上午,里面的几间洋房里倒也还可以坐坐的。但到傍晚,有一间的地板便常不免要咚咚咚地响得震天,兼以满房烟尘斗乱\footnote{〔斗乱〕飞舞杂乱。斗:抖。};问问精通时事的人,答道,“那是在学跳舞。”
    
    到别的地方去看看,如何呢?
    
    我就往仙台\footnote{〔仙台〕日本城市名,在本州岛东北,距离东京约400公里。}的医学专门学校去。从东京出发,不久便到一处驿站,写道:日暮里。不知怎地,我到现在还记得这名目。其次却只记得水户了,这是明的遗民朱舜水\footnote{〔朱舜水〕朱之瑜,明清之际的学者和教育家,流亡日本,在水户讲学。}先生客死的地方。仙台是一个市镇,并不大;冬天冷得厉害;还没有中国的学生。
    
    大概是物以希为贵罢。北京的白菜运往浙江,便用红头绳系住菜根,倒挂在水果店头,尊为“胶菜”;福建野生着的芦荟,一到北京就请进温室,且美其名曰“龙舌兰”。我到仙台也颇受了这样的优待,不但学校不收学费,几个职员还为我的食宿操心。我先是住在监狱旁边一个客店里的,初冬已经颇冷,蚊子却还多,后来用被盖了全身,用衣服包了头脸,只留两个鼻孔出气。在这呼吸不息的地方,蚊子竟无从插嘴,居然睡安稳了。饭食也不坏。但一位先生却以为这客店也包办囚人的饭食,我住在那里不相宜,几次三番,几次三番地说。我虽然觉得客店兼办囚人的饭食和我不相干,然而好意难却,也只得别寻相宜的住处了。于是搬到别一家,离监狱也很远,可惜每天总要喝难以下咽的芋梗汤。
    
    从此就看见许多陌生的先生,听到许多新鲜的讲义\footnote{〔照相〕照片、相片。}。解剖学\footnote{〔解剖学〕研究生命体的结构和组织的学科。}是两个教授分任的。最初是骨学。其时进来的是一个黑瘦的先生,八字须,戴着眼镜,挟着一迭\footnote{〔一迭〕一叠。}大大小小的书。一将书放在讲台上,便用了缓慢而很有顿挫的声调,向学生介绍自己道:
    
    “我就是叫作藤野严九郎的……。”
    
    后面有几个人笑起来了。他接着便讲述解剖学在日本发达\footnote{〔发达〕发展。}的历史,那些大大小小的书,便是从最初到现今关于这一门学问的著作。起初有几本是线装的;还有翻刻中国译本的,他们的翻译和研究新的医学,并不比中国早。
    
    那坐在后面发笑的是上学年不及格的留级学生,在校已经一年,掌故\footnote{〔掌故〕历史上的制度、文化沿革以及人物事迹等。这里指学校的旧事。}颇为熟悉的了。他们便给新生讲演每个教授的历史。这藤野先生,据说是穿衣服太模糊\footnote{〔模糊〕这里指穿着打扮随意,不讲究。}了,有时竟会忘记带领结;冬天是一件旧外套,寒颤颤的,有一回上火车去,致使管车的疑心他是扒手,叫车里的客人大家小心些。
    
    他们的话大概是真的,我就亲见他有一次上讲堂没有带领结。
    
    过了一星期,大约是星期六,他使助手来叫我了。到得研究室,见他坐在人骨和许多单独的头骨中间,——他其时正在研究着头骨,后来有一篇论文在本校的杂志上发表出来。
    
    “我的讲义,你能抄下来么?”他问。
    
    “可以抄一点。”
    
    “拿来我看!”
    
    我交出所抄的讲义去,他收下了,第二三天便还我,并且说,此后每一星期要送给他看一回。我拿下来打开看时,很吃了一惊,同时也感到一种不安和感激。原来我的讲义已经从头到末,都用红笔添改过了,不但增加了许多脱漏的地方,连文法的错误,也都一一订正。这样一直继续到教完了他所担任的功课\footnote{〔功课〕指课程。}:骨学、血管学、神经学。
    
    可惜我那时太不用功,有时也很任性。还记得有一回藤野先生将我叫到他的研究室里去,翻出我那讲义上的一个图来,是下臂的血管,指着,向我和蔼的说道:
    
    “你看,你将这条血管移了一点位置了。——自然,这样一移,的确比较的好看些,然而解剖图不是美术,实物是那么样的,我们没法改换它。现在我给你改好了,以后你要全照着黑板上那样的画。”
    
    但是我还不服气,口头答应着,心里却想道:
    
    “图还是我画的不错;至于实在的情形,我心里自然记得的。”
    
    学年试验完毕之后,我便到东京玩了一夏天,秋初再回学校,成绩早已发表了,同学一百余人之中,我在中间,不过是没有落第。这回藤野先生所担任的功课,是解剖实习和局部解剖学。
    
    解剖实习了大概一星期,他又叫我去了,很高兴地,仍用了极有抑扬的声调对我说道:
    
    “我因为听说中国人是很敬重鬼的,所以很担心,怕你不肯解剖尸体。现在总算放心了,没有这回事。”
    
    但他也偶有使我很为难的时候。他听说中国的女人是裹脚的,但不知道详细,所以要问我怎么裹法,足骨变成怎样的畸形,还叹息道,“总要看一看才知道。究竟是怎么一回事呢?”
    
    有一天,本级的学生会干事到我寓里来了,要借我的讲义看。我检出来交给他们,却只翻检了一通,并没有带走。但他们一走,邮差就送到一封很厚的信,拆开看时,第一句是:
    
    “你改悔罢!”
    
    这是《新约》\footnote{〔《新约》〕基督教的圣经。讲述耶稣的故事,和犹太教经典区别,后者称为《旧约》。}上的句子罢,但经托尔斯泰\footnote{〔托尔斯泰〕列夫·托尔斯泰,19世纪俄罗斯作家。主要著作有《战争与和平》《安娜·卡列尼娜》《复活》等。}新近引用过的。其时正值日俄战争\footnote{〔日俄战争〕1904至1905年日本与俄罗斯为争夺在我国东北和朝鲜的殖民利益而进行的战争,主要在我国境内进行。},托老先生便写了一封给俄国和日本的皇帝的信\footnote{〔托老先生……信〕托尔斯泰写给日本和俄罗斯君主的信。刊登在1904年6月27日英国《泰晤士报》上。两个月后翻译成日语登载在日本《平民新闻》上。},开首便是这一句。日本报纸上很斥责他的不逊,爱国青年也愤然,然而暗地里却早受了他的影响了。其次的话,大略是说上年解剖学试验的题目,是藤野先生讲义上做了记号,我预先知道的,所以能有这样的成绩。末尾是匿名。 我这才回忆到前几天的一件事。因为要开同级会,干事便在黑板上写广告,末一句是“请全数到会勿漏为要”,而且在“漏”字旁边加了一个圈。我当时虽然觉到圈得可笑,但是毫不介意,这回才悟出那字也在讥刺我了,犹言我得了教员漏泄出来的题目。
    
    我便将这事告知了藤野先生;有几个和我熟识的同学也很不平,一同去诘责干事托辞检查的无礼,并且要求他们将检查的结果,发表出来。终于这流言消灭了,干事却又竭力运动,要收回那一封匿名信去。结末是我便将这托尔斯泰式的信退还了他们。
    
    中国是弱国,所以中国人当然是低能儿,分数在六十分以上,便不是自己的能力了:也无怪他们疑惑。但我接着便有参观枪毙中国人的命运了。第二年添教霉菌学\footnote{〔霉菌学〕研究真菌的学科。霉菌:真菌。},细菌的形状是全用电影\footnote{〔电影〕这里指幻灯片。}来显示的,一段落已完而还没有到下课的时候,便影\footnote{〔影〕放映。}几片时事的片子,自然都是日本战胜俄国的情形。但偏有中国人夹在里边:给俄国人做侦探,被日本军捕获,要枪毙了,围着看的也是一群中国人;在讲堂里的还有一个我。
    
    “万岁!”他们都拍掌欢呼起来。
    
    这种欢呼,是每看一片都有的,但在我,这一声却特别听得刺耳。此后回到中国来,我看见那些闲看枪毙犯人的人们,他们也何尝不酒醉似的喝彩,——呜呼,无法可想!但在那时那地,我的意见\footnote{〔意见〕想法。}却变化了。
    
    到第二学年的终结\footnote{〔终结〕结束。},我便去寻藤野先生,告诉他我将不学医学,并且离开这仙台。他的脸色仿佛有些悲哀,似乎想说话,但竟没有说。
    
    “我想去学生物学\footnote{〔生物学〕研究生命现象和生命活动规律的学科。},先生教给我的学问,也还有用的。”其实我并没有决意要学生物学,因为看得他有些凄然,便说了一个慰安\footnote{〔慰安〕安慰。}他的谎话。
    
    “为医学而教的解剖学之类,怕于生物学也没有什么大帮助。”他叹息说。
    
    将走的前几天,他叫我到他家里去,交给我一张照相\footnote{〔照相〕照片、相片。},后面写着两个字道:“惜别”,还说希望将我的也送他。但我这时适值没有照相了;他便叮嘱我将来照了寄给他,并且时时通信告诉他此后的状况。
    
    我离开仙台之后,就多年没有照过相,又因为状况也无聊,说起来无非使他失望,便连信也怕敢写了。经过的年月一多,话更无从说起,所以虽然有时想写信,却又难以下笔,这样的一直到现在,竟没有寄过一封信和一张照片。从他那一面看起来,是一去之后,杳无消息了。
    
    但不知怎地,我总还时时记起他,在我所认为我师的之中,他是最使我感激,给我鼓励的一个。有时我常常想:他的对于我的热心的希望,不倦的教诲,小而言之,是为中国,就是希望中国有新的医学;大而言之,是为学术,就是希望新的医学传到中国去。他的性格,在我的眼里和心里是伟大的,虽然他的姓名并不为许多人所知道。
    
    他所改正的讲义,我曾经订成三厚本,收藏着的,将作为永久的纪念。不幸七年前迁居的时候\footnote{〔七年前迁居〕指1919年12月鲁迅从绍兴搬家到北京。},中途毁坏了一口书箱,失去半箱书,恰巧这讲义也遗失在内了\footnote{〔这讲义也遗失〕《解剖学笔记》后来1951年从鲁迅家藏书中找到。现存于鲁迅纪念馆。}。责成运送局去找寻,寂无回信。只有他的照相至今还挂在我北京寓居的东墙上,书桌对面。每当夜间疲倦,正想偷懒时,仰面在灯光中瞥见他黑瘦的面貌,似乎正要说出抑扬顿挫的话来,便使我忽又良心发现,而且增加勇气了,于是点上一枝烟,再继续写些为“正人君子”之流所深恶痛疾的文字。
    
    \hfill 十月十二日
    
\end{normalsize}


\newpage

\textbf{注释}:

\vspace{-1em}

\begin{itemize}
    \setlength\itemsep{-0.2em}
    \item 〔不逊〕不恭敬,没有礼貌。
    \item 〔讥刺〕暗中挖苦,说坏话。讥:旁敲侧击地批评。刺:用尖锐的话指出别人的坏处。
    \item 〔诘责〕质问并责备。诘:追问。
    \item 〔匿名〕隐藏名字,不表露身份。
    \item 〔侦探〕受托探听、调查的人。
    \item 〔适值〕正好处于……的时候。
    \item 〔托辞〕借口。找借口。
    \item 〔畸形〕生物某部分在发育中形成的不正常的形状。
    \item 〔杳无消息〕毫无消息。杳:幽深,高远。
    \item 〔深恶痛疾〕极端厌恶仇恨。痛:深切地、彻底地。疾:恨。
\end{itemize}

\chapter{孔乙己}

\begin{normalsize}
    
    鲁镇的酒店\footnote{〔酒店〕提供酒水、小食的小店,也叫酒吧、酒馆。}的格局,是和别处不同的:都是当街一个曲尺形\footnote{〔曲尺形〕曲尺的形状。曲尺:由张成直角的两个直尺组成的尺子,用于画直角,也叫角尺、拐尺。}的大柜台,柜里面预备着热水,可以随时温酒。做工的人,傍午傍晚散了工,每每花四文\footnote{〔文〕铜钱的单位。一枚铜钱为一文。}铜钱,买一碗酒,——这是二十多年前的事,现在每碗要涨到十文,——靠柜外站着,热热的喝了休息;倘肯多花一文,便可以买一碟盐煮笋,或者茴香豆,做下酒物了,如果出到十几文,那就能买一样荤菜,但这些顾客,多是短衣帮,大抵没有这样阔绰。只有穿长衫的,才踱进店面隔壁的房子里,要酒要菜,慢慢地坐喝。
    
    我从十二岁起,便在镇口的咸亨酒店里当伙计\footnote{〔伙计〕餐厅酒馆里服侍顾客的雇员,也叫小二。},掌柜说,我样子太傻,怕侍候不了长衫主顾,就在外面做点事罢。外面的短衣主顾,虽然容易说话,但唠唠叨叨缠夹不清的也很不少。他们往往要亲眼看着黄酒从坛子里舀出,看过壶子底里有水没有,又亲看将壶子放在热水里,然后放心:在这严重\footnote{〔严重〕严格。}监督下,羼水\footnote{〔羼水〕往酒里加水,也写作掺水。}也很为难。所以过了几天,掌柜又说我干不了这事。幸亏荐头\footnote{〔荐头〕介绍佣工为职业的人,用工中介。}的情面大,辞退不得,便改为专管温酒的一种无聊职务了。
    
    我从此便整天的站在柜台里,专管我的职务。虽然没有什么失职,但总觉得有些单调,有些无聊。掌柜是一副凶脸孔,主顾也没有好声气,教人活泼不得;只有孔乙己到店,才可以笑几声,所以至今还记得。
    
    孔乙己是站着喝酒而穿长衫的唯一的人。他身材很高大;青白脸色,皱纹间时常夹些伤痕;一部乱蓬蓬的花白的胡子。穿的虽然是长衫,可是又脏又破,似乎十多年没有补,也没有洗。他对人说话,总是满口之乎者也,叫人半懂不懂的。因为他姓孔,别人便从描红\footnote{〔描红〕习字手段。用墨笔在红字上描着抄写。}纸上的“上大人孔乙己”\footnote{〔“上大人孔乙己”〕一种古代儿童习字用的启蒙短文的开头六个字。}这半懂不懂的话里,替他取下一个绰号,叫作孔乙己。孔乙己一到店,所有喝酒的人便都看着他笑,有的叫道,“孔乙己,你脸上又添上新伤疤了!”他不回答,对柜里说,“温两碗酒,要一碟茴香豆。”便排出九文大钱\footnote{〔大钱〕清朝咸丰年间铸造的劣质铜铁货币。}。他们又故意的高声嚷道,“你一定又偷了人家的东西了!”孔乙己睁大眼睛说,“你怎么这样凭空污人清白……”“什么清白?我前天亲眼见你偷了何家的书,吊着打。”孔乙己便涨红了脸,额上的青筋条条绽出\footnote{〔绽出〕指激动时血管鼓起像裂纹。},争辩道,“窃书不能算偷……窃书!……读书人的事,能算偷么?”接连便是难懂的话,什么“君子固穷\footnote{〔君子固穷〕出自《论语·卫灵公》。意思是君子不因为贫穷而改变品行操守。}”,什么“者乎”之类,引得众人都哄笑起来:店内外充满了快活的空气。
    
    听人家背地里谈论,孔乙己原来也读过书,但终于没有进学\footnote{〔进学〕清代指通过科举院试,进入官学的人,称为生员或秀才。},又不会营生;于是愈过愈穷,弄到将要讨饭了。幸而写得一笔好字,便替人家抄抄书,换一碗饭吃。可惜他又有一样坏脾气,便是好喝懒做。坐不到几天,便连人和书籍纸张笔砚,一齐失踪。如是几次,叫他抄书的人也没有了。孔乙己没有法,便免不了偶然做些偷窃的事。但他在我们店里,品行却比别人都好,就是从不拖欠;虽然间或没有现钱,暂时记在粉板\footnote{〔粉板〕旧时店铺里用来记事的一种白漆木板。}上,但不出一月,定然还清,从粉板上拭去了孔乙己的名字。
    
    孔乙己喝过半碗酒,涨红的脸色渐渐复了原,旁人便又问道,“孔乙己,你当真认识字么?”孔乙己看着问他的人,显出不屑置辩的神气。他们便接着说道,“你怎的连半个秀才\footnote{〔秀才〕清代科举中,通过院试称为秀才,也叫生员。秀才可免除自身赋税徭役,出入公堂,不被随意惩处用刑。}也捞不到呢?”孔乙己立刻显出颓唐不安模样,脸上笼上了一层灰色,嘴里说些话;这回可是全是之乎者也之类,一些不懂了。在这时候,众人也都哄笑起来:店内外充满了快活的空气。
    
    在这些时候,我可以附和着笑,掌柜是决不责备的。而且掌柜见了孔乙己,也每每这样问他,引人发笑。孔乙己自己知道不能和他们谈天,便只好向孩子说话。有一回对我说道,“你读过书么?”我略略点一点头。他说,“读过书,……我便考你一考。茴香豆的茴字,怎样写的?”我想,讨饭一样的人,也配考我么?便回过脸去,不再理会。孔乙己等了许久,很恳切的说道,“不能写罢?……我教给你,记着!这些字应该记着。将来做掌柜的时候,写账要用。”我暗想我和掌柜的等级还很远呢,而且我们掌柜也从不将茴香豆上账;又好笑,又不耐烦,懒懒的答他道,“谁要你教,不是草头底下一个来回的回字么?”孔乙己显出极高兴的样子,将两个指头的长指甲敲着柜台,点头说,“对呀对呀!……回字有四样写法,你知道么?\footnote{〔回字有四样写法〕“回”字较常见的写法有三种:“回”“囘”“囬”,其他写法都极罕见。}”我愈不耐烦了,努着嘴\footnote{〔努嘴〕翘起嘴唇。}走远。孔乙己刚用指甲蘸了酒,想在柜上写字,见我毫不热心,便又叹一口气,显出极惋惜的样子。
    
    有几回,邻居孩子听得笑声,也赶热闹,围住了孔乙己。他便给他们一人一颗。孩子吃完豆,仍然不散,眼睛都望着碟子。孔乙己着了慌,伸开五指将碟子罩住,弯腰下去说道,“不多了,我已经不多了。”直起身又看一看豆,自己摇头说,“不多不多!多乎哉?不多也。\footnote{〔多乎哉?不多也。〕出自《论语·子罕》。}”于是这一群孩子都在笑声里走散了。
    
    孔乙己是这样的使人快活,可是没有他,别人也便这么过。
    
    有一天,大约是中秋前的两三天,掌柜正在慢慢的结账,取下粉板,忽然说,“孔乙己长久没有来了。还欠十九个钱呢!”我才也觉得他的确长久没有来了。一个喝酒的人说道,“他怎么会来?……他打折了腿了。”掌柜说,“哦!”“他总仍旧是偷。这一回,是自己发昏,竟偷到丁举人\footnote{〔举人〕清代科举中,秀才通过乡试,称为举人(指被荐举的人)。举人有做官的资格,可以免除百人的税赋徭役,可通过出租免税名额获得稳定收入。}家里去了。他家的东西,偷得的吗?”“后来怎么样?”“怎么样?先写服辩\footnote{〔服辩〕认罪书。这里是说孔乙己为了私了而认罪。},后来是打,打了大半夜,再打折了腿。”“后来呢?”“后来打折了腿了。”“打折了怎样呢?”“怎样?……谁晓得?许是死了。”掌柜也不再问,仍然慢慢的算他的账。
    
    中秋过后,秋风是一天凉比一天,看看将近初冬;我整天的靠着火,也须穿上棉袄了。一天的下半天,没有一个顾客,我正合了眼坐着。忽然间听得一个声音,“温一碗酒。”这声音虽然极低,却很耳熟。看时又全没有人。站起来向外一望,那孔乙己便在柜台下对了门槛坐着。他脸上黑而且瘦,已经不成样子;穿一件破夹袄\footnote{〔夹袄〕双层的上衣。},盘着两腿,下面垫一个蒲包,用草绳在肩上挂住;见了我,又说道,“温一碗酒。”掌柜也伸出头去,一面说,“孔乙己么?你还欠十九个钱呢!”孔乙己很颓唐的仰面答道,“这……下回还清罢。这一回是现钱,酒要好。”掌柜仍然同平常一样,笑着对他说,“孔乙己,你又偷了东西了!”但他这回却不十分分辩,单说了一句“不要取笑!”“取笑?要是不偷,怎么会打断腿?”孔乙己低声说道,“跌断,跌,跌……”他的眼色,很像恳求掌柜,不要再提。此时已经聚集了几个人,便和掌柜都笑了。我温了酒,端出去,放在门槛上。他从破衣袋里摸出四文大钱,放在我手里,见他满手是泥,原来他便用这手走来的。不一会,他喝完酒,便又在旁人的说笑声中,坐着用这手慢慢走去了。
    
    自此以后,又长久没有看见孔乙己。到了年关\footnote{〔年关〕农历年底。古代过年时生活困难,所以将过年比作过关。},掌柜取下粉板说,“孔乙己还欠十九个钱呢!”到第二年的端午,又说“孔乙己还欠十九个钱呢!”到中秋可是没有说,再到年关也没有看见他。
    
    我到现在终于没有见——大约孔乙己的确死了。
    
\end{normalsize}


\newpage

\textbf{注释}:

\vspace{-1em}

\begin{itemize}
    \setlength\itemsep{-0.2em}
    \item 〔当街〕就在街边。
    \item 〔荤菜〕荤:指辛辣的香菜,引申指肉食。
    \item 〔缠夹不清〕纠缠,把各种是非杂七杂八搅在一起。
    \item 〔绰号〕外号。
    \item 〔阔绰〕有钱,能花钱。
    \item 〔间或〕偶尔,有时候。
    \item 〔颓唐〕情绪低落,精神苦闷。
    \item 〔附和〕顺着别人的话,表示赞同或理解。
    \item 〔置辩〕值得辩论、辩解。
\end{itemize}

\chapter{鲁提辖拳打镇关西}

\begin{normalsize}
    
    三人上到潘家酒楼上,拣个齐楚阁儿\footnote{〔齐楚阁儿〕干净整洁的包房。}里坐下。鲁提辖\footnote{〔提辖〕宋代一类官职的称呼。多用于称呼负责武备、治安、采买等实际事务的主管。}坐了主位,李忠对席,史进下首坐了。酒保唱了喏,认得是鲁提辖,便道:“提辖官人\footnote{〔官人〕对有官职的人的敬称。},打多少酒?”鲁达道:“先打四角酒来\footnote{〔角〕古代盛酒器的泛称,一角大约0.6至0.8升。}。”一面铺下菜蔬、果品按酒,又问道:“官人,吃甚下饭?”鲁达道:“问甚么?但有,只顾卖来,一发算钱还你。这厮只顾来聒噪。”酒保下去,随即烫酒上来,但是\footnote{〔但是〕但凡是,只要是。}下口肉食,只顾将来\footnote{〔将来〕拿来,多指上菜。},摆一桌子。
    
    三个酒至数杯,正说些闲话,较量些枪法,说得入港\footnote{〔入港〕指谈得投入,谈得高兴。},只听得隔壁阁子里有人哽哽咽咽啼哭。鲁达焦躁,便把碟儿、盏儿,都丢在楼板上。酒保听得,慌忙上来看时,见鲁提辖气愤愤地。酒保抄手\footnote{〔抄手〕双手放到胸前,交互插在衣袖中。也指两臂交叉放在胸前。}道:“官人要甚东西,分付买来。”鲁达道:“洒家\footnote{〔洒家〕我。鲁达自称用语。}要甚么?你也须认的洒家,却恁地教甚么人在间壁\footnote{〔间壁〕隔壁。}吱吱的哭,搅俺弟兄们吃酒。洒家须不曾少了你酒钱!”酒保道:“官人息怒,小人怎敢教人啼哭,打搅官人吃酒。这个哭的,是绰酒座儿唱的父子两人\footnote{〔绰酒座〕驻唱,指在酒肆饭店桌边巡走,应客人要求唱歌,赚取小费收入。}。不知官人们在此吃酒,一时间自苦了啼哭。”鲁提辖道:“可是作怪!你与我唤的他来。”
    
    酒保去叫,不多时,只见两个到来:前面一个十八九岁的妇人,背后一个五六十岁的老儿,手里拿串拍板,都来到面前。那妇人,虽无十分的容貌,也有些动人的颜色,但拭着眼泪,向前来深深的道了三个万福\footnote{〔万福〕古代女子行的敬礼。双手轻轻抱拳在胸前右下侧上下移动,同时做鞠躬的姿势。}。那老儿也都相见了。鲁达问道:“你两个是那里人家\footnote{〔那里〕哪里。古白话中“哪”“那”不分。}?为甚啼哭?”那妇人便道:“官人不知,容奴\footnote{〔奴〕古代女性谦称自己。}告禀:奴家是东京\footnote{〔东京〕指北宋开封府治所汴梁,北宋的首都,又称汴京。}人氏。因同父母来这渭州,投奔亲眷,不想搬移南京去了。母亲在客店里染病身故,子父二人,流落在此生受。此间有个财主,叫做镇关西郑大官人,因见奴家,便使强媒硬保,要奴作妾。谁想写了三千贯文书\footnote{〔贯〕宋代货币单位,一贯是一千文。},虚钱实契\footnote{〔虚钱实契〕指签的契约中说已经付了钱,但其实没有付钱。},要了奴家身体。未及三个月,他家大娘子好生利害\footnote{〔利害〕厉害。},将奴赶打出来,不容完聚。着落\footnote{〔着落〕要求落实某事。}店主人家追要原典身钱三千贯\footnote{〔典〕典当,将物品寄放换钱,以后有钱了再拿钱换回来。这里“典身”是“卖身”的委婉说法。}。父亲懦弱,和他争执不得,他又有钱有势。当初不曾得他一文,如今那讨钱来还他?没计奈何,父亲自小教得奴家些小曲儿,来这里酒楼上赶座子。每日但得些钱来,将大半还他;留些少子父们盘缠\footnote{〔盘缠〕古代出远门时盘起缠在腰上的袋子,装金银等重要物件。比喻出远门路上的花费。}。这两日酒客稀少,违了他钱限,怕他来讨时,受他羞耻。子父们想起这苦楚来,无处告诉,因此啼哭。不想误触犯了官人,望乞恕罪,高抬贵手。”
    
    鲁提辖又问道:“你姓甚么?在那个客店里歇?那个镇关西郑大官人在那里住?”老儿答道:“老汉姓金,排行第二;孩儿小字翠莲;郑大官人便是此间状元桥下卖肉的郑屠,绰号镇关西。老汉父子两个,只在前面东门里鲁家客店安下。”鲁达听了道:“呸!俺只道哪个郑大官人,却原来是杀猪的郑屠。这个腌臜\footnote{〔腌臜〕肮脏,下贱。}泼才,投托着俺小种经略相公\footnote{〔小种经略相公〕指种师中,名将种世衡之孙,种师道的弟弟。种师道被称为“老种”,种师中被称为“小种”。经略,指经略使,北宋末期皇帝选派忠心能干的大臣到边疆掌管一路军政防务,但不管财赋漕运等民事。}门下做个肉铺户,却原来这等欺负人!”回头看着李忠、史进道:“你两个且在这里,等洒家去打死了那厮便来。”史进、李忠抱住劝道:“哥哥息怒,明日却理会。”两个三回五次劝得他住。
    
    鲁达又道:“老儿,你来,洒家与你些盘缠,明日便回东京去如何?”父子两个告道:“若是能够回乡去时,便是重生父母,再长爷娘。只是店主人家如何肯放?郑大官人须着落他要钱。”鲁提辖道:“这个不妨事,俺自有道理。”便去身边摸出五两来银子,放在桌上,看着史进道:“洒家今日不曾多带得些出来,你有银子,借些与俺,洒家明日便送还你。”史进道:“直甚么\footnote{〔直〕值。},要哥哥还。”去包裹里取出一锭十两银子,放在桌上。鲁达看着李忠道:“你也借些出来与洒家。”李忠去身边摸出二两来银子。鲁提辖看了见少,便道:“也是个不爽利的人。”鲁达只把十五两银子与了金老,分付道:“你父子两个将去做盘缠,一面收拾行李,俺明日清早来,发付你两个起身,看那个店主人敢留你!”金老并女儿拜谢去了。
    
    鲁达把这二两银子丢还了李忠。三人再吃了两角酒,下楼来叫道:“主人家,酒钱洒家明日送来还你。”主人家连声应道:“提辖只顾自去,但吃不妨,只怕提辖不来赊。”三个人出了潘家酒肆,到街上分手,史进、李忠各自投客店去了。只说鲁提辖回到经略府前下处\footnote{〔下处〕下脚处,临时的住所。},到房里,晚饭也不吃,气愤愤的睡了。主人家又不敢问他。
    
    再说金老得了这一十五两银子,回到店中,安顿了女儿。先去城外远处觅下一辆车儿,回来收拾了行李,还了房宿钱,算清了柴米钱,只等来日天明。当夜无事。
    
    次早五更起来,子父两个先打火做饭,吃罢,收拾了,天色微明,只见鲁提辖大踏步走入店里来,高声叫道:“店小二,那里是金老歇处?”小二哥道:“金公,提辖在此寻你。”金老开了房门,便道:“提辖官人,里面请坐。”鲁达道:“坐甚么?你去便去,等甚么?”金老引了女儿,挑了担儿,作谢提辖,便待出门,店小二拦住道:“金公,那里去?”鲁达问道:“他少你房钱?”小二道:“小人房钱,昨夜都算还了。须欠郑大官人典身钱,着落在小人身上看管他哩!”鲁提辖道:“郑屠的钱,洒家自还他。你放这老儿还乡去。”那店小二那里肯放。鲁达大怒,揸开五指,去那小二脸上只一掌,打的那店小二口中吐血;再复一拳,打下当门两个牙齿。小二扒将起来,一道烟走向店里去躲了。店主人那里敢出来拦他?金老父子两个,忙忙离了店中,出城自去寻昨日觅下的车儿去了。
    
    且说鲁达寻思:恐怕店小二赶去拦截他,且向店里掇条凳子,坐了两个时辰。约莫\footnote{〔约莫〕估摸,估计。}金公去的远了,方才起身,径到状元桥来。
    
    且说郑屠开着两间门面,两副肉案,悬挂着三五片猪肉。郑屠正在门前柜身内坐定,看那十来个刀手卖肉。鲁达走到面前,叫声:“郑屠!”郑屠看时,见是鲁提辖,慌忙出柜身来唱喏道:“提辖恕罪。”便叫副手:“掇条凳子来,提辖请坐。”鲁达坐下道:“奉着经略相公钧旨\footnote{〔钧旨〕对上司命令的敬称。},要十斤精肉,切做臊子,不要见半点肥的在上头。”郑屠道:“使得,你们快选好的,切十斤去。”鲁提辖道:“不要那等腌臜厮们动手,你自与我切。”郑屠道:“说得是。小人自切便了。”自去肉案上,拣下十斤精肉,细细切做臊子。那店小二把手帕包了头,正来郑屠家报说金老之事,却见鲁提辖坐在肉案门边,不敢拢来,只得远远的立住,在房檐下望。
    
    这郑屠整整的自切了半个时辰,用荷叶包了道:“提辖,教人送去。”鲁达道:“送甚么?且住!再要十斤,都是肥的,不要见些精的在上面,也要切做臊子。”郑屠道:“却才精的,怕府里要裹馄饨,肥的臊子何用?”鲁达睁着眼道:“相公钧旨,分付洒家,谁敢问他?”郑屠道:“是合用的东西,小人切便了。”又选了十斤实膘的肥肉,也细细的切做臊子,把荷叶来包了。整弄了一早晨,却得饭罢时候。那店小二那里敢过来,连那正要买肉的主顾,也不敢拢来。
    
    郑屠道:“着人与提辖拿了,送将府里去。”鲁达道:“再要十斤寸金软骨,也要细细地剁做臊子,不要见些肉在上面。”郑屠笑道:“却不是特地来消遣我!”鲁达听罢,跳起身来,拿着那两包臊子在手里,睁眼看着郑屠道:“洒家特地要消遣你!”把两包臊子,劈面打将去,却似下了一阵的肉雨。
    
    郑屠大怒,两条忿气从脚底下直冲到顶门心头。那一把无明业火\footnote{〔无明业火〕佛教用语,这里指怒火。}焰腾腾的按纳不住,从肉案上抢了一把剔骨尖刀,托地跳将下来。鲁提辖早拔步在当街上。众邻舍并十来个火家\footnote{〔火家〕伙计。},那个敢向前来劝?两边过路的人都立住了脚,和那店小二也惊的呆了。
    
    郑屠右手拿刀,左手便来要揪鲁达,被这鲁提辖就势按住左手,赶将入去,望小腹上只一脚,腾地踢倒在当街上,鲁达再入一步,踏住胸脯,提着那醋钵儿大小拳头\footnote{〔醋钵儿〕盛醋的小钵。钵:类似碗盆的一种平底器皿。},看着这郑屠道:“洒家始投老种经略相公,做到关西五路廉访使,也不枉了叫做镇关西。你是个卖肉的操刀屠户,狗一般的人,也叫做镇关西!你如何强骗了金翠莲?”扑的只一拳,正打在鼻子上,打得鲜血迸流,鼻子歪在半边,却便似开了个油酱铺,咸的、酸的、辣的,一发都滚出来。郑屠挣不起来\footnote{〔挣〕挣扎。},那把尖刀,也丢在一边,口里只叫:“打得好!”鲁达骂道:“直娘贼,还敢应口!”提起拳头来,就眼眶际眉梢只一拳,打得眼棱缝裂,乌珠迸出,也似开了个彩帛铺的,红的、黑的、绛的,都绽将出来。两边看的人,惧怕鲁提辖,谁敢向前来劝。郑屠当不过,讨饶。鲁达喝道:“咄!你是个破落户,若是和俺硬到底,洒家倒饶了你;你如何对俺讨饶,洒家偏不饶你。”又只一拳,太阳上正着,却似做了一个全堂水陆的道场\footnote{〔全堂水陆的道场〕道场,佛教为死人做法事的仪式。全堂水陆,指超度水中陆上一切亡灵。这里指做法事时各种乐器的声音。},磬儿、钹儿、铙儿一齐响。鲁达看时,只见郑屠挺在地下,口里只有出的气,没了入的气,动弹不得。鲁提辖假意道:“你这厮诈死,洒家再打。”只见面皮渐渐的变了。鲁达寻思道:“俺只指望痛打这厮一顿,不想三拳真个打死了他。洒家须吃官司,又没人送饭,不如及早撒开。”拔步便走,回头指着郑屠尸道:“你诈死,洒家和你慢慢理会。”一头骂,一头大踏步去了。街坊邻舍,并郑屠的火家,谁敢向前来拦他?
    
    鲁提辖回到下处,急急卷了些衣服、盘缠、细软、银两,但是旧衣粗重,都弃了。提了一条齐眉短棒,奔出南门,一道烟走了。
    
\end{normalsize}


\newpage

\textbf{注释}:

\vspace{-1em}

\begin{itemize}
    \setlength\itemsep{-0.2em}
    \item 〔聒噪〕烦人的吵闹。
    \item 〔禀〕下对上的报告。
    \item 〔掇〕拾取,用双手拿。
\end{itemize}

\chapter{回忆我的母亲}

\begin{normalsize}
    
    得到母亲去世的消息,我很悲痛。我爱我母亲,特别是她勤劳一生,很多事情是值得我永远回忆的。
    
    我家是佃农。祖籍广东韶关,客籍人\footnote{〔客籍人〕即客家人,汉族的一支民系。主要由五胡乱华至唐宋元时期从中原南迁的汉人组成。主要分布于广东广西北部、江西南部、福建西部以及东南亚地区。},在“湖广填四川”\footnote{〔“湖广填四川”〕清朝初期一次大规模移民。湖广,最初指元代设立的湖广行省。当时主要包括湖南、广西。清军灭明时在四川屠杀过多,导致西南地区人口锐减,因此清朝初期鼓动湖南、湖北、广西、广东、江西等地往西移民,持续近百年。}时迁移四川仪陇县马鞍场。世代为地主耕种,家境是贫苦的,和我们来往的朋友也都是老老实实的贫苦农民。
    
    母亲一共生了十三个儿女。因为家境贫穷,无法全部养活,只留下了八个,以后再生下的被迫溺死了。这在母亲心里是多么惨痛悲哀和无可奈何的事情啊!母亲把八个孩子一手养大成人。可是她的时间大半被家务和耕种占去了,没法多照顾孩子,只好让孩子们在地里爬着。
    
    母亲是个好劳动的人。从我能记忆时起,总是天不亮就起床。全家二十多口人,妇女们轮班煮饭,轮到就煮一年。母亲把饭煮了,还要种田、种菜、喂猪、养蚕、纺棉花。因为她身体高大结实,还能挑水挑粪。
    
    母亲这样地整日劳碌着。我到四五岁时就很自然地在旁边帮她的忙,到八九岁时就不但能挑能背,还会种地了。记得那时我从私塾\footnote{〔私塾〕私人出资设立的学校、课堂。}回家,常见母亲在灶上汗流满面地烧饭,我就悄悄把书一放,挑水或放牛去了。有的季节里,我上午读书,下午种地;一到农忙,便整日在地里跟着母亲劳动。这个时期母亲教给我许多生产知识。
    
    佃户\footnote{〔佃户〕向地主或官府租种土地的农民。}家庭的生活自然是艰苦的,可是由于母亲的聪明能干,也勉强过得下去。我们用桐子\footnote{〔桐子〕桐树的果实种子。桐树产于我国西南部地区,形似梧桐、种子可榨油,因此也叫“油桐”。}榨油来点灯,吃的是豌豆饭、菜饭、红薯饭、杂粮饭,把菜籽榨出的油放在饭里做调料。这类地主富人家看也不看的饭食,母亲却能做得使一家人吃起来有滋味。赶上丰年,才能缝上一些新衣服,衣服也是自己生产出来的。母亲亲手纺出线,请人织成布,染了颜色,我们叫它“家织布”,有铜钱那样厚。一套衣服老大穿过了,老二老三接着穿还穿不烂。
    
    勤劳的家庭是有规律有组织的。我的祖父是一个中国标本式的农民,到八九十岁还非耕田不可,不耕田就会害病,直到临死前不久还在地里劳动。祖母是家庭的组织者,一切生产事务由她管理分派,每年除夕就分派好一年的工作。每天天还没亮,母亲就第一个起身,接着听见祖父起来的声音,接着大家都离开床铺,喂猪的喂猪,砍柴的砍柴,挑水的挑水。母亲在家庭里极能任劳任怨。她性格和蔼,没有打骂过我们,也没有同任何人吵过架。因此,虽然在这样的大家庭里,长幼、伯叔、妯娌\footnote{〔妯娌〕兄弟的妻子的合称。}相处都很和睦。母亲同情贫苦的人——这是朴素的阶级意识,虽然自己不富裕,还周济和照顾比自己更穷的亲戚。她自己是很节省的。父亲有时吸点旱烟,喝点酒;母亲管束着我们,不允许我们染上一点。母亲那种勤劳俭朴的习惯,母亲那种宽厚仁慈的态度,至今还在我心中留有深刻的印象。
    
    但是灾难不因为中国农民的和平就不降临到他们身上。庚子年\footnote{〔庚子年〕即公元1900年。}前后,四川连年旱灾,很多的农民饥饿、破产,不得不成群结队地去“吃大户”。我亲眼见到,六七百穿得破破烂烂的农民和他们的妻子儿女被所谓官兵一阵凶杀毒打,血溅四五十里,哭声动天。在这样的年月里,我家也遭受更多的困难,仅仅吃些小菜叶、高粱,通年\footnote{〔通年〕整年,一年到头。}没吃过白米。特别是乙未\footnote{〔乙未〕即公元1895年。}那一年,地主欺压佃户,要在租种的地上加租子,因为办不到,就趁大年除夕,威胁着我家要退佃,逼着我们搬家。在悲惨的情况下,我们一家人哭泣着连夜分散。从此我家被迫分两处住下。人手少了,又遇天灾,庄稼没收成,这是我家最悲惨的一次遭遇。母亲没有灰心,她对穷苦农民的同情和对为富不仁者的反感却更强烈了。母亲沉痛的三言两语的诉说以及我亲眼见到的许多不平事实,启发了我幼年时期反抗压迫追求光明的思想,使我决心寻找新的生活。
    
    我不久就离开母亲,因为我读书了。我是一个佃农家庭的子弟,本来是没有钱读书的。那时乡间豪绅\footnote{〔豪绅〕地方上仗势欺人的士绅。豪:势大强横。}地主的欺压,衙门差役的横蛮,逼得母亲和父亲决心节衣缩食培养出一个读书人来“支撑门户”。我念过私塾,光绪三十一年\footnote{〔光绪三十一年〕即公元1905年。}考了科举,以后又到更远的顺庆和成都去读书。这个时候的学费都是东挪西借来的,总共用了二百多块钱,直到我后来当护国军\footnote{〔护国军〕指1915年蔡锷在护国运动中组织讨伐袁世凯的军队。}旅长时才还清。
    
    光绪三十四年\footnote{〔光绪三十四年〕即公元1908年。}我从成都回来,在仪陇县办高等小学\footnote{〔高等小学〕20世纪上半叶的教育制度,分为初等小学(四年)和高等小学(三年)。新中国成立后小学采用一贯制,不再区分。},一年回家两三次去看母亲。那时新旧思想冲突得很厉害。我们抱了科学民主的思想,想在家乡做点事情,守旧的豪绅们便出来反对我们。我决心瞒着母亲离开家乡,远走云南,参加新军和同盟会\footnote{〔新军和同盟会〕新军:甲午战争后清末朝廷的“新政”之一,采用西方军队的制度训练,招收知识分子当兵。同盟会:清末由兴中会、华兴会等团体集合组成的革命组织。1905年在日本东京成立,尝试推翻清朝统治,把新军作为笼络活动的对象。}。我到云南后,从家信中知道,我母亲对我这一举动不但不反对,还给我许多慰勉。
    
    从宣统元年\footnote{〔宣统元年〕即公元1909年。}到现在,我再没有回过一次家,只在民国八年\footnote{〔民国八年〕即公元1919年。}我曾经把父亲和母亲接出来。但是他俩劳动惯了,离开土地就不舒服,所以还是回了家。父亲就在回家途中死了。母亲回家继续劳动,一直到最后。
    
    中国革命继续向前发展,我的思想也继续向前发展。当我发现了中国革命的正确道路时,我便加入了中国共产党。大革命\footnote{〔大革命〕指1924年至1927年国民政府推翻帝国主义扶持的北洋军阀统治的革命运动。主要运动为1926年开始的北伐战争。}失败了,我和家庭完全隔绝了。母亲就靠那三十亩地独立支持一家人的生活。抗战以后,我才能和家里通信。母亲知道我所做的事业,她期望着中国民族解放的成功。她知道我们党的困难,依然在家里过着勤苦的农妇生活。七年中间,我曾寄回几百元钱和几张自己的照片给母亲。母亲年老了,但她永远想念着我,如同我永远想念着她一样。去年收到侄儿的来信说:“祖母今年已有八十五岁,精神不如昨年之健康,饮食起居亦不如前,甚望见你一面,聊叙别后情景。”但我献身于民族抗战事业,竟未能报答母亲的希望。
    
    母亲最大的特点是一生不曾脱离过劳动。母亲生我前一分钟还在灶上煮饭。虽到老年,仍然热爱生产。去年另一封外甥的家信中说:“外祖母大人因年老关系,今年不比往年健康,但仍不辍劳作,尤喜纺棉。”
    
    我应该感谢母亲,她教给我与困难作斗争的经验。我在家庭中已经饱尝艰苦,这使我在三十多年的军事生活和革命生活中再没感到过困难,没被困难吓倒。母亲又给我一个强健的身体,一个勤劳的习惯,使我从来没感到过劳累。
    
    我应该感谢母亲,她教给我生产的知识和革命的意志,鼓励我以后走上革命的道路。在这条路上,我一天比一天更加认识:只有这种知识,这种意志,才是世界上最可宝贵的财产。
    
    母亲现在离我而去了,我将永不能再见她一面了,这个哀痛是无法补救的。母亲是一个平凡的人,她只是中国千百万劳动人民中的一员,但是,正是这千百万人创造了和创造着中国的历史。我用什么方法来报答母亲的深恩呢?我将继续尽忠于我们的民族和人民,尽忠于我们的民族和人民的希望——中国共产党,使和母亲同样生活着的人能够过快乐的生活。这是我能做到的,一定能做到的。
    
    愿母亲在地下安息!
    
\end{normalsize}


\newpage

\textbf{注释}:

\vspace{-1em}

\begin{itemize}
    \setlength\itemsep{-0.2em}
    \item 〔溺死〕在水中无法呼吸而死。
    \item 〔劳碌〕做的事情多而辛苦。碌:忙。
    \item 〔和睦〕和平友爱相处,不争斗。
    \item 〔朴素〕没有加工修饰。这里指直接产生的、没有形成成熟理论的思想。朴:木头没有经过细加工。素:布没有染色。
    \item 〔安息〕安静休息。对死者表示哀悼的用语。
    \item 〔节衣缩食〕省吃省穿。
    \item 〔为富不仁〕作为富有的人,不讲仁义。
    \item 〔衙门〕古代官吏办公的地方。比喻官府。
    \item 〔差役〕在官府中办事的底层人员。差:派遣去办事,引申指办的事和办事的人。役:统治者强制看守边疆,引申指强制劳动和被强制劳动的人。
    \item 〔慰勉〕安慰勉励。
\end{itemize}

\chapter{人类的语言}

\begin{normalsize}
    
    语言,也就是说话,好像是极其稀松平常的事儿。可是仔细想想,实在是一件了不起的大事。正是因为说话跟吃饭、走路一样的平常,人们才不去想它究竟是怎么回事儿。其实这三件事儿都是极不平常的,都是使人类不同于别的高等动物的特征。别的动物都吃生的,只有人类会烧熟了吃。别的动物,除了天上飞的和水里游的,走路都是让身体跟地面平行,有几条腿使几条腿,只有人类直起身子来用两条腿走路,把上肢\footnote{〔上肢〕指哺乳类动物靠近头部的一对肢。}解放出来干别的、更重要的活儿。同样,别的动物的嘴只会吃东西,人类的嘴除了吃东西还会说话。
    
    记得在小学里读书的时候,班上有一位“能文”的大师兄,在一篇作文的开头写下这么两句:“鹦鹉能言,不离于禽;猩猩能言,不离于兽。”我们看了都非常佩服。后来知道这两句话是有来历的,只是字句有些出入。又过了若干年,才知道这两句话都有问题。鹦鹉能学人说话,可只是作为现成的公式\footnote{〔公式〕通用的方式方法。}来说,不会加以变化(所以我们管人云亦云的说话叫“鹦鹉学舌”)。只有人们的说话是从具体情况(包括外界情况和本人意图)出发,情况一变,话也跟着变。至于猩猩,根据西方学者拿黑猩猩做试验的结果,它们能学会极其有限的一点符号语言,可是学不会把它变成有声语言。人类语言之所以能够“随机应变”,在于一方面能够把语音分析成若干音素\footnote{〔音素〕语音的最小单位,音素变化会导致语义变化。}(当然是不自觉地),又把这些音素组合成音节\footnote{〔音节〕语音的最小结构单位。汉语中,一个音节基本对应一个字。},再把音节连缀起来,——音素数目有限,各种语言一般都只有几十个音素,可是组成音节就可以成百上千,再组成双音节、三音节,就能有几十万、几百万。另一方面,人们又能分析外界事物及其变化,形成无数的“意念”,——配以语音,然后综合运用,表达各种复杂的意思。一句话,人类语言的特点就在于能用变化无穷的语音,表达变化无穷的意义。这是任何其他动物办不到的。
    
    人类语言采用声音作为手段,而不采用手势或图画,也不是偶然的。人类的视觉最发达,可是语言诉之于听觉\footnote{〔诉之于〕采用、依靠(某事物、某种方法),也写作“诉诸”。诉:控告,求助。}。这是因为一切倚赖视觉的手段,要发挥作用,离不开光线,夜里不成,黑暗的地方或者有障碍物的地方也不成,声音则白天黑夜都可以发挥作用,也不容易受阻碍。手势之类,距离大了看不清,声音的有效距离大得多。打手势或者画画儿要用手,手就不能同时做别的事,说话用嘴,可以一边儿说话,一边儿劳动。论快慢,打手势赶不上说话,画画儿更赶不上。声音唯一不如形象的地方在于缺乏稳定性和持久性,但在原始社会的交际情况下,这方面的要求是次要的,是可以用图形来补充的。总之,正是由于采用了嘴里的声音作为手段,人类语言才得到前程万里的发展。
    
\end{normalsize}


\newpage

\textbf{注释}:

\vspace{-1em}

\begin{itemize}
    \setlength\itemsep{-0.2em}
    \item 〔稀松平常〕平常普通,不难办到。稀松:(土地)松散,容易耕作。
    \item 〔人云亦云〕别人说什么,也跟着说什么。没有主见。云:说。
    \item 〔符号〕传达特定意义的印记、标识。
    \item 〔倚赖〕依赖,依靠。
\end{itemize}

\chapter{食物从何处来}

\begin{normalsize}
    
    一切生物都离不开食物。如何获得食物?这有两种不同的途径和方法。
    
    一种叫自养,绿色植物都属于这一类。他们自己把无机物\footnote{〔无机物〕传统上把生物机体的部分称为有机物,此外的物质称为无机物。现代科学已经发现,生物体内也有传统所说的无机物。现在一般把有碳骨架的分子结构称为有机物,把不含碳的分子结构称为无机物。一些含碳的物质也依照传统称为无机物。}制造成有机的食物,满足生长的需要。
    
    绿色开花的植物有庞大得惊人的根系。每条根的尖端都有很多根毛,每一个根毛就是一个最基层的原料采集站。大量吸收土壤中的水分和无机盐\footnote{〔无机盐〕一类无机化合物,由酸根离子和金属离子组成,是很多矿石的主要成分,所以也叫矿物质。代表是食盐。}等原料,经过运输干线——茎,源源不断送入叶子里。叶子就是一个食品工厂。叶子上面有着许多气孔。在阳光下,这些气孔一面排出氧气和蒸腾水分,一面还吸入大量的二氧化碳\footnote{〔二氧化碳〕一种碳、氧的化合物,常态为气体,是空气的成分。}。有时,一个气孔在一秒钟内能吸进两万五千亿个二氧化碳分子。
    
    二氧化碳和水在合成车间——叶绿体里,发生奇妙的变化。叶绿体是叶绿素\footnote{〔叶绿素〕一种有机物,光合作用的关键。}和蛋白质\footnote{〔蛋白质〕人类食物主要养分之一。瘦肉、蛋白主要为蛋白质。}等组成的小颗粒,一个叶肉细胞里,一般含20至100个。叶子的绿色就是它们的颜色。叶绿素吸收了太阳的光能\footnote{〔光能〕光中的能量。},就把二氧化碳和水合成为含有高能\footnote{〔高能〕很多能量。}的有机物质(主要是碳水化合物),同时放出废气——氧,由气孔排出。这就是赫赫有名的光合作用。看来很简单,实际上是一个非常复杂的过程。
    
    植物合成了这些食物,大部分都用来组成躯体和贮藏在种子或块根、块茎\footnote{〔块根〕块根、块茎,即块状的根茎。}中,小部分经呼吸作用\footnote{〔呼吸作用〕生物体细胞把有机物和氧气转化为水和二氧化碳,并释放能量,支持生命活动的化学过程。}又被分解成水和二氧化碳,同时,放出能量,供给生命活动之用。
    
    另一种叫异养。所有的动物和大部分微生物\footnote{〔微生物〕肉眼看不见的微小生物的总称。}都是这一类。它们自己不能制造食物,靠植物来生活。
    
    例如,野兔靠吃野草来生活。狼以野兔为食物。狼一旦碰到了老虎,也就成了牺牲品。老虎死后,又成了细菌的乐园;不用多久,尸体就分解得精光,变成了二氧化碳、水和无机盐,回到大自然中,又成了植物制造食物的原料。
    
    所以兔、狼、虎、细菌,归根结底都是靠植物来生活。
    
    人,每天除了要吃进一定量的水和盐以外,还要吃淀粉\footnote{〔淀粉〕人类食物的主要养分之一。米面等谷物类食物主要为淀粉。}、蛋白质、脂肪\footnote{〔脂肪〕人类食物主要养分之一。肥肉、油脂主要为脂肪。}。我们皮肤上不会长出叶绿素,当然是属于异养型。吃荤也好,吃素也好,反正都是靠植物而生活。不过人是靠劳动获得食物的,能够用各种方式改造植物,使它更好地为人服务。很久以前,人们就懂得了农业,办起了绿色工厂,让庄稼来把二氧化碳和水变成食物。人们把其中营养最丰富的部分如种子、果实、块根、块茎等拿来做粮食;剩下的秸秆\footnote{〔秸秆〕农作物的茎杆。}、糠麸\footnote{〔糠麸〕谷物的外壳,加工时的副产品。 糠:从稻、麦等谷上脱落的皮。也就是谷的外壳。 麸:小麦磨面过箩后剩下的皮。}也是有机物,就再拿来办加工厂:养猪,养牛,养鸡。那些不好吃的东西经过猪、牛、鸡的消化吸收和转化,就变成了猪肉、牛奶、鸡蛋等高级食物。
    
    所以,世界上除了极个别的细菌能不依赖阳光而靠化学能\footnote{〔化学能〕化学反应中吸收或释放的能量。}来合成食物以外,其他一切生物都靠绿色植物的光合作用来获得食物。全世界的植物,一年中能制造出好几千亿吨有机物,这真是一个无比巨大的合成工厂。
    
\end{normalsize}


\newpage

\textbf{注释}:

\vspace{-1em}

\begin{itemize}
    \setlength\itemsep{-0.2em}
    \item 〔颗粒〕谷物、果实的种子。引申指很小的球体,很小的东西。
    \item 〔根系〕植物的所有根须。系:顶部连合的垂下的散丝,引申为从一点散出的线条。
    \item 〔蒸腾〕让水变成水蒸气升腾到空气中。腾:奔跑,跳跃,上升到空中。
    \item 〔赫赫有名〕形容非常有名。
    \item 〔归根结底〕归结到根基上。
\end{itemize}

\chapter{中国人民寻求救国真理的道路}

\begin{normalsize}
    
    我们党走过二十八年了,大家知道,不是和平地走过的,而是在困难的环境中走过的,我们要和国内外党内外的敌人作战。谢谢马克思、恩格斯、列宁\footnote{〔列宁〕弗拉基米尔·列宁,俄罗斯无产阶级革命家、政治家、理论家、思想家。领导俄罗斯十月革命取得成功,建立苏联。}和斯大林\footnote{〔斯大林〕约瑟夫·斯大林,苏联无产阶级革命家、思想家、政治家、军事家。列宁逝世后成为苏联最高领导人。},他们给了我们以武器。这武器不是机关枪,而是马克思列宁主义。
    
    列宁在1920年在《共产主义运动中的“左派”幼稚病》一书中,描写过俄国人寻找革命理论的经过\footnote{〔列宁在……〕指第二章中:“在将近半个世纪里,大约从上一世纪四十年代至九十年代,俄国进步的思想界在空前野蛮和反动的沙皇制度的压迫之下,曾如饥似渴地寻求正确的革命理论,专心致志地、密切地注视着欧美在这方面的每一种‘最新成就’。俄国在半个世纪里,经受了闻所未闻的痛苦和牺牲,表现了空前未有的革命英雄气概,以难以置信的毅力和舍身忘我的精神去探索、学习和实验,经受了失望,进行了验证,参照了欧洲的经验,真是饱经苦难找到了马克思主义这个唯一正确的革命理论。”}。俄国人曾经在几十个年头内,经历艰难困苦,方才找到了马克思主义。中国有许多事情和十月革命\footnote{〔十月革命〕1917年俄罗斯爆发的社会主义革命,建立了人类历史上首个由共产主义政党领导的无产阶级统治的社会主义国家:苏联。}以前的俄国相同,或者近似。封建主义的压迫,这是相同的。经济和文化落后,这是近似的。两个国家都落后,中国则更落后。先进的人们,为了使国家复兴,不惜艰苦奋斗,寻找革命真理,这是相同的。
    
    自从1840年鸦片战争\footnote{〔鸦片战争〕1840年英国为了倾销鸦片而侵略中国的战争,以中国战败,签订不平等的《南京条约》,割地赔款告终,是中国近代史的屈辱开端。}失败那时起,先进的中国人,经过千辛万苦,向西方国家寻找真理。洪秀全\footnote{〔洪秀全〕清晚期太平天国运动的发起者和领袖。}、康有为\footnote{〔康有为〕清末政治家、思想家。提倡君主立宪。1898年策动戊戌变法失败后逃亡海外,成为保皇派,反对辛亥革命。}、严复\footnote{〔严复〕清末翻译家、思想家。主张引进西学,改革教育,开启民智,提倡民主科学。}和孙中山\footnote{〔孙中山〕清末革命家、政治家、思想家。多次发起推翻清王朝的革命,最终取得成功,建立中华民国。},代表了在中国共产党出世以前向西方寻找真理的一派人物。那时,求进步的中国人,只要是西方的新道理,什么书也看。向日本、英国、美国、法国、德国派遣留学生之多,达到了惊人的程度。国内废科举,兴学校,好像雨后春笋,努力学习西方。我自己在青年时期,学的也是这些东西。这些是西方资产阶级民主主义的文化,即所谓新学,包括那时的社会学说和自然科学,和中国封建主义的文化即所谓旧学是对立的。学了这些新学的人们,在很长的时期内产生了一种信心,认为这些很可以救中国,除了旧学派,新学派自己表示怀疑的很少。要救国,只有维新,要维新,只有学外国。那时的外国只有西方资本主义国家是进步的,它们成功地建设了资产阶级的现代国家。日本人向西方学习有成效,中国人也想向日本人学。在那时的中国人看来,俄国是落后的,很少人想学俄国。这就是十九世纪四十年代至二十世纪初期中国人学习外国的情形。
    
    帝国主义\footnote{〔帝国主义〕资本主义社会形式,依靠资本输出剥削其他国家的劳动者,进行统治。}的侵略打破了中国人学西方的迷梦。很奇怪,为什么先生老是侵略学生呢?中国人向西方学得很不少,但是行不通,理想总是不能实现。多次奋斗,包括辛亥革命\footnote{〔辛亥革命〕1911年中国爆发的推翻清王朝统治的武装革命。}那样全国规模的运动,都失败了。国家的情况一天一天坏,环境迫使人们活不下去。怀疑产生了,增长了,发展了。第一次世界大战震动了全世界。俄国人举行了十月革命,创立了世界上第一个社会主义国家。过去蕴藏在地下为外国人所看不见的伟大的俄国无产阶级和劳动人民的革命精力,在列宁、斯大林领导之下,像火山一样突然爆发出来了,中国人和全人类对俄国人都另眼相看了。这时,也只是在这时,中国人从思想到生活,才出现了一个崭新的时期。中国人找到了马克思列宁主义这个放之四海而皆准的普遍真理,中国的面目就起了变化了。
    
    中国人找到马克思主义,是经过俄国人介绍的。在十月革命以前,中国人不但不知道列宁、斯大林,也不知道马克思、恩格斯。十月革命一声炮响,给我们送来了马克思列宁主义。十月革命帮助了全世界的也帮助了中国的先进分子,用无产阶级的宇宙观作为观察国家命运的工具,重新考虑自己的问题。走俄国人的路——这就是结论。1919年,中国发生了五四运动。1921年,中国共产党成立。孙中山在绝望里,遇到了十月革命和中国共产党。孙中山欢迎十月革命,欢迎俄国人对中国人的帮助,欢迎中国共产党同他合作。孙中山死了,蒋介石\footnote{〔蒋介石〕蒋中正,世称蒋介石,清末政治家,南京国民政府主席,中国国民党总裁,实行独裁统治,依靠美帝国主义,残酷镇压中国共产党与民主进步人士,国共内战失败后退守台湾。}起来。在二十二年的长时间内,蒋介石把中国拖到了绝境。在这个时期中,以苏联为主力军的反法西斯\footnote{〔法西斯〕20世纪诞生的一种政治思想,标志为一捆棍棒中间插着一把斧头,象征团结一致服从一个意志、一个权力。法西斯主义主张独裁集权下的集体主义、民族主义,用军事武力维护民族资产阶级利益。}的第二次世界大战,打倒了三个帝国主义大国,两个帝国主义大国在战争中被削弱了,世界上只剩下一个帝国主义大国即美国没有受损失。而美国的国内危机是很深重的。它要奴役全世界,它用武器帮助蒋介石杀戮了几百万中国人。中国人民在中国共产党领导之下,在驱逐日本帝国主义之后,进行了三年的人民解放战争,取得了基本的胜利。
    
    就是这样,西方资产阶级的文明,资产阶级的民主主义,资产阶级共和国的方案,在中国人民的心目中,一齐破了产。资产阶级的民主主义让位给工人阶级领导的人民民主主义,资产阶级共和国让位给人民共和国。这样就造成了一种可能性:经过人民共和国到达社会主义和共产主义,到达阶级的消灭和世界的大同。康有为写了《大同书》,他没有也不可能找到一条到达大同的路。资产阶级的共和国,外国有过的,中国不能有,因为中国是受帝国主义压迫的国家。唯一的路是经过工人阶级领导的人民共和国。
    
\end{normalsize}


\newpage

\textbf{注释}:

\vspace{-1em}

\begin{itemize}
    \setlength\itemsep{-0.2em}
    \item 〔放之四海而皆准〕无论放在什么地方都不会错。
    \item 〔雨后春笋〕比喻新生事物大量快速涌现。
    \item 〔阶级〕因社会地位和对生产资料关系不同形成的利益集团。
    \item 〔杀戮〕大量杀害。
    \item 〔宇宙观〕即世界观,对世界、社会的根本看法。
    \item 〔大同〕出自《礼记》,是儒家学说对理想社会的描述,指人人友爱互助,家家安居乐业,没有差异,没有战争的社会。
    \item 〔破产〕企业依法宣布无力偿还债务,由司法部门接收其财产抵债。比喻失败、破灭。
\end{itemize}

\chapter{《农村调查》序言}

\begin{normalsize}
    
    现在党的农村政策,不是十年内战时期\footnote{〔十年内战时期〕指1927年8月南昌起义至1937年9月国共合作为止的十年。}那样的土地革命政策,而是抗日民族统一战线的政策。全党应该执1940年7月7日和12月25日的中央指示,应该执行即将到来的七次大会的指示。所以印这个材料,是为了帮助同志们找一个研究问题的方法。现在我们很多同志,还保存着一种粗枝大叶、不求甚解的作风,甚至全然不了解下情,却在那里担负指导工作,这是异常危险的现象。对于中国各个社会阶级的实际情况,没有真正具体的了解,真正好的领导是不会有的。
    
    要了解情况,唯一的方法是向社会作调查,调查社会各阶级的生动\footnote{〔生动〕这里指实际的鲜活的(情况)。}情况。对于担负指导工作的人来说,有计划地抓住几个城市、几个乡村,用马克思主义的基本观点,即阶级分析的方法,作几次周密的调查,乃是了解情况的最基本的方法。只有这样,才能使我们具有对中国社会问题的最基础的知识。
    
    要做这件事,第一是眼睛向下,不要只是昂首望天。没有眼睛向下的兴趣和决心,是一辈子也不会真正懂得中国的事情的。
    
    第二是开调查会。东张西望,道听途说,决然得不到什么完全的知识。我用开调查会的方法得来的材料,湖南的几个,井冈山的几个,都失掉了。这里印的,主要的是一个《兴国调查》\footnote{〔《兴国调查》〕1930年10月毛泽东对江西省赣州市兴国县做的社会调查。},一个《长冈乡调查》\footnote{〔《长冈乡调查》〕1933年11月毛泽东对兴国县长冈乡做的社会调查。}和一个《才溪乡调查》\footnote{〔《才溪乡调查》〕1933年11月毛泽东对福建省龙岩市上杭县才溪乡做的社会调查。}。开调查会,是最简单易行又最忠实可靠的方法,我用这个方法得了很大的益处,这是比较什么大学还要高明的学校。到会的人,应是真正有经验的中级和下级的干部,或老百姓。我在湖南五县调查\footnote{〔湖南五县调查〕1927年1月毛泽东对湖南湘潭、湘乡、衡山、醴陵、长沙五县做的社会调查。}和井冈山两县调查\footnote{〔井冈山两县调查〕1927年底、1928年初毛泽东对江西、湖南省交界的永新、宁冈二县做的社会调查。},找的是各县中级负责干部;寻乌调查\footnote{〔寻乌调查〕1930年5月毛泽东对闽粤赣三省交界的寻乌县做的社会调查。}找的是一部分中级干部,一部分下级干部,一个穷秀才,一个破产了的商会会长,一个在知县衙门管钱粮的已经失了业的小官吏。他们都给了我很多闻所未闻的知识。使我第一次懂得中国监狱全部腐败情形的,是在湖南衡山县作调查时该县的一个小狱吏。兴国调查和长冈、才溪两乡调查,找的是乡级工作同志和普通农民。这些干部、农民、秀才、狱吏、商人和钱粮师爷,就是我的可敬爱的先生,我给他们当学生是必须恭谨勤劳和采取同志态度的,否则他们就不理我,知而不言,言而不尽。开调查会每次人不必多,三五个七八个人即够。必须给予时间,必须有调查纲目,还必须自己口问手写,并同到会人展开讨论。因此,没有满腔的热忱,没有眼睛向下的决心,没有求知的渴望,没有放下臭架子、甘当小学生的精神,是一定不能做,也一定做不好的。必须明白:群众是真正的英雄,而我们自己则往往是幼稚可笑的,不了解这一点,就不能得到起码的知识。
    
    我再度申明:出版这个参考材料的主要目的,在于指出一个如何了解下层情况的方法,而不是要同志们去记那些具体材料及其结论。一般地说,中国幼稚的资产阶级还没有来得及也永远不可能替我们预备关于社会情况的较完备的甚至起码的材料,如同欧美日本的资产阶级那样,所以我们自己非做搜集材料的工作不可。特殊地说,实际工作者须随时去了解变化着的情况,这是任何国家的共产党也不能依靠别人预备的。所以,一切实际工作者必须向下作调查。对于只懂得理论不懂得实际情况的人,这种调查工作尤有必要,否则他们就不能将理论和实际相联系。“没有调查就没有发言权”,这句话,虽然曾经被人讥为“狭隘经验论”的,我却至今不悔\footnote{〔“没有调查就没有发言权”……〕“没有调查就没有发言权”出自1930年5月毛泽东写的《反对本本主义》,是毛泽东在井冈山时期就持有的观点,被王明等教条主义者认为是“狭隘经验主义”。};不但不悔,我仍然坚持没有调查是不可能有发言权的。有许多人,“下车伊始\footnote{〔下车伊始〕新官刚到任所下车。这里讽刺带有教条主义的新上任干部。}”,就哇喇哇喇地发议论,提意见,这也批评,那也指责,其实这种人十个有十个要失败。因为这种议论或批评,没有经过周密调查,不过是无知妄说。我们党吃所谓“钦差大臣”的亏,是不可胜数的。而这种“钦差大臣”则是满天飞,几乎到处都有。斯大林的话说得对:“理论若不和革命实践联系起来,就会变成无对象的理论。”当然又是他的话对:“实践若不以革命理论为指南,就会变成盲目的实践。”除了盲目的、无前途的、无远见的实际家,是不能叫做“狭隘经验论”的。
    
    我现在还痛感有周密研究中国事情和国际事情的必要,这是和我自己对于中国事情和国际事情依然还只是一知半解这种事实相关联的,并非说我是什么都懂得了,只是人家不懂得。和全党同志共同一起向群众学习,继续当一个小学生,这就是我的志愿。
    
\end{normalsize}


\newpage

\textbf{注释}:

\vspace{-1em}

\begin{itemize}
    \setlength\itemsep{-0.2em}
    \item 〔粗枝大叶〕粗鲁,不细心,容易疏漏。
    \item 〔热忱〕真诚的热情。忱:真挚的情意。
    \item 〔高明〕见解独到不凡,技艺高超。
    \item 〔道听途说〕路上听来的消息,指没有根据的传闻。
    \item 〔闻所未闻〕从来没听说过。
    \item 〔痛感〕深深地感觉到、体会到。
    \item 〔周密〕周到严密。
\end{itemize}

\end{document}
