\documentclass[12pt,UTF-8,openany]{ctexbook}
\usepackage{ctex}
\usepackage{titlesec}
\usepackage{xeCJK}
\usepackage{verse}
\usepackage{fontspec,xunicode,xltxtra}
\usepackage{xpinyin}
\usepackage{geometry}
\usepackage{indentfirst}
\usepackage{wrapfig}
\usepackage{caption}
\usepackage{pifont}
\usepackage{enumitem}
\usepackage[perpage,symbol*]{footmisc}
\usepackage[table,dvipsnames]{xcolor}

\geometry{a5paper,left=1.4cm,right=1.4cm,top=2.3cm,bottom=2.3cm}
\renewcommand{\footnotesize}{\fontsize{8.5pt}{10.5pt}\selectfont}
\setmainfont{Mona Sans Light}
\setCJKmainfont[BoldFont=STZhongsong]{汉字之美仿宋GBK 免费}
\xeCJKDeclareCharClass{CJK}{`0 -> `9}
\xeCJKsetup{AllowBreakBetweenPuncts=true}
\DefineFNsymbols{circled}{{\ding{192}}{\ding{193}}{\ding{194}}{\ding{195}}{\ding{196}}{\ding{197}}{\ding{198}}{\ding{199}}{\ding{200}}{\ding{201}}}
\setfnsymbol{circled}
\xpinyinsetup{ratio=0.5,hsep={.6em plus .6em},vsep={1em}}

\setlength{\intextsep}{2pt}%
\titleformat{\chapter}{\zihao{-1}\bfseries}{ }{16pt}{}
\titleformat{\section}{\zihao{-2}\bfseries}{ }{0pt}{}
\title{\zihao{0} \bfseries 使琉球记}
\setlength{\lineskip}{24pt}
\setlength{\parskip}{6pt}
\author{}
\date{}
\begin{document}
% \maketitle
% \tableofcontents
% \newpage

% \chapter{使琉球记}

\begin{center}
    \begin{Huge}
        \textbf{使琉球记}
    \end{Huge}
    
    \vspace{8pt}
    李鼎元\footnote{节选自《使琉球记》。嘉庆四年九月,大清国藩属琉球中山国王尚穆世孙尚温表请袭封。嘉庆五年(公元1800年)春,李鼎元过海奉使赴琉球,遍游琉球国诸岛,顺道考察风俗人情,写成日记体《使琉球记》六卷。这里节选其中部分日记。}
    \vspace{8pt}
\end{center}

\begin{normalsize}
    
    
    \begin{center}
        \textbf{出使琉球}
    \end{center}
    
    初八日己丑,晴。午风大。黎明,有二白鸟绕船而飞。午刻,丁风,仍用辰针,计行四更。申刻,过米糠洋。漩皆圆,波浪密而细,如初筛之米,点点零落;“米糠”字,极有形容。日落,计又行三更;船伙长\footnote{〔伙长〕船上掌管罗盘的人。}云:“鸡笼山、花瓶屿去船远,不应见”。是夜,用乙辰针,行船六更。舟中吐者甚多;余日坐将台\footnote{〔将台〕阅兵点将台,指挥者坐的位置。现在称为舰桥。},全不觉险,饮食如常。
    
    初九日庚寅,晴。卯刻,见彭家山\footnote{〔彭家山〕即彭佳屿,在台湾省基隆港北约30海里。}。山列三峰,东高而西下。计自开洋,行船十六更矣;由山北过船。辰刻,转丁未风,用单乙针,行十更船。申正,见钓鱼台\footnote{〔钓鱼台〕即钓鱼岛,在彭佳屿东约80海里,属基隆。},三峰离立如笔架,皆石骨。惟时水天一色,舟平而驶;有白鸟无数绕船而送,不知所自来。入夜,星影横斜,月色破碎,海面尽作火焰,浮沉出没。
    

    \begin{wrapfigure}{r}{0.42\textwidth} %this figure will be at the right
        \vspace{-20pt}
        \begin{flushright}
            \includegraphics[width=0.4\textwidth]{../src/other/hour12.png}
        \end{flushright}
        \caption*{\textbf{一日十二时辰}}
    \end{wrapfigure}

    初十日辛卯,晴。丁未风,仍用单乙针。东方黑云蔽日,水面白鸟无数。计彭家至此,行船十四更。辰正,见赤尾屿;屿方而赤,东西凸而中凹,凹中又有小峰二。船从山北过。有大鱼二,夹舟行,不见首尾,脊黑而微绿,如十围枯木附于舟侧;舟人举酒相庆。
    
    十一日壬辰,阴。丁未风,仍用单乙针。计赤尾屿至此,行十四更船。午刻,见姑米山。山共八岭,岭各一、二峰,或断或续;舟中人欢声沸海。未刻,大风,暴雨如注,然雨虽暴而风顺。酉刻,舟已近山,计又行五更船。球人以姑米多礁,黑夜不敢进,待明而行。丑刻,有小船来引导;乃放舟由山南行。

    \begin{wrapfigure}{l}{0.42\textwidth} %this figure will be at the right
        \vspace{-10pt}
        \begin{flushleft}
            \includegraphics[width=0.4\textwidth]{../src/other/dir24.png}
        \end{flushleft}
        \caption*{\textbf{罗盘二十四针}}
    \end{wrapfigure}
    
    十二日癸巳,晴。辰刻,过马齿山。山如犬牙相错,四峰离立,若马行空。计又行七更,船再用甲寅针,取那霸港。考历来针路所见,尚有小琉球、鸡笼山、黄麻屿;此行俱未见。问之琉球伙长,年已六十,往来海面八次,云此次最为简捷,而所见亦仅三山,即至姑米。惟纪更以香,殊难为据。据琉球伙长云:“海上行舟,风小固不能驶,风过大亦不能驶;风大则浪大,浪大力能壅船,进尺仍退二寸。惟风七分、浪五分,最宜驾驶;此次是也。从来渡海,未有平稳而驶如此者。”辰刻,进那霸港。午刻,登岸。倾国人士聚观于路,世孙\footnote{〔世孙〕指琉球中山王国第十五代国王尚温。当时琉球第十四代国王尚穆病逝,尚穆的儿子尚哲已死,李鼎元奉旨到琉球册封尚穆的孙子尚温为下一代国王。由于册封还没完成,所以称为世孙。}率百官迎诏如仪。
    
    \begin{center}
        \textbf{琉球见闻}
    \end{center}
    
    十七日戊戌,晴。阅案头食单,有所谓“龙头虾”者。取视之,长尺馀,绦甲朱髯、血睛火鬣,类世所画龙头,见之悚然!取其壳以为灯,可供两日玩;三日而色变矣。
    
    十八日己亥,雨。栽荔枝于使院庭后,南北分列。移自牧荔园,种曰“陈家紫”。
    
    二十一日壬寅,阴。连日食海味,腹渐作泻。令庖人但供时蔬、淡粥。庖人乃以佳苏鱼进;问之,曰:“此非鱼名也,系削黑鳗鱼脊肉,干而为之”。长五、六寸许,形如梭、质如枯木。食法,先以温水浸洗,裹蕉叶煨之,切片如刨花,连五、七片不断,又如兰花;宜清酱\footnote{〔清酱〕即“酱清”,生抽酱油。},颇有异味。清酱甘美,十倍于闽。惟求“佳苏”之义不得,适有长史至,问亦不解;因呼至前细核之,据云“此品在敝国既多且美,自王官以及贫民皆得食”;意殆谓如家常蔬菜,人人得食也。球人字皆对音,殆实为“家蔬”也。
    
    二十二日癸卯,晴。午后,偕介山\footnote{〔介山〕指这次出使琉球的大使赵文楷,字介山。李鼎元是副使。}策骑游波上山。一名石筍崖,以形似名之也;石垣四周,垣后可望海,沿海多浮石,嵌空玲珑;潮水击之,声作钟磬。东北有山,曰雪崎。又东北,有小石山,曰龟山。稍下为护国寺,国王祷雨之所也。龛内有神,黑而裸,手剑立,状甚狞;名曰“不动”,或曰火神。庭中有景泰七年铸钟一,廊下又有乾隆五十七年新铸钟一。寺后多凤尾蕉,一名铁树。西有石,高五、六尺,黑而润,状如骈佛手。因书“仙人掌”三字于上。
    
    初二日癸丑,大暑,阴。从官往游泊村,归以新稻穗见示,云稻已尽收;乃知球阳地气温暖,稻常早熟,种以十一月,收以五、六月。薯则四时皆种,三熟为丰,四熟则为大丰。稻田少、薯田多。国人以薯为命,米则王官始得食。亦有麦、豆,所产不多。薯一名地瓜,闽人土语。午后,微雨。
    
    初五日丙辰,阴。巳后,大雨。长史\footnote{〔长史〕职官名,相当于幕僚长。}送佛桑\footnote{〔佛桑〕朱槿,也叫赤槿、佛桑、红扶桑、大红花,原产于中国南部,广泛分布于亚洲的观赏花卉。}四株。一种千层如榴,有深红、粉红二色。一种单层,花如灯盘,蕊单出如烛,长二寸许,有红、白二色;朝开暮落,落则瓣卷如烛。花而不实;四季有花,深冬叶始凋谢。此地花开四季者甚多,气暖故也。余感长史意,嘱从客酬以酒;意有花再相致耳。
    
    初七日戊午,晴。辰刻,微雨,旋止。长史复以花二盆见贻,标曰“水翁花”;视之,乃马兰花也。中山草木,多与中朝\footnote{〔中朝〕中央王朝,指中国。}异称;盖因国中少书,多不识古来草木之名。如罗汉松,谓之\hbox{\scalebox{0.4}[1]{木}\kern-.2em\scalebox{0.75}[1]{坚}}木;冬青,谓之福木;万寿菊,谓之禅菊:其初以意名之,后遂相沿不改。惜未携《群芳谱》来,一一证辨之耳!
    
    十四日乙丑,阴。荔枝栽近一月,新叶茂发,有生机矣。早起,偶步其侧,见新叶有蚀者;薄视之,有虫黄体而苍文,两角、八足,身方而毛,世所谓毛虫类。附叶为巢,蒙如小蛛网;卵生如蚕子而速,大者二寸以来。命仆捉而坑之,尽扫其巢。
    
    二十九日庚辰,晴。是日初见五彩鱼\footnote{〔五彩鱼〕即花斑连鳍\hbox{\scalebox{0.4}[1]{魚}\kern-.2em\scalebox{0.75}[1]{銜}},俗名七彩麒麟、五彩青蛙。}。有红绿翠黄诸色,绿鳞红章,五彩相间。土人就形色呼之,无定名。又有一石眉巴鱼\footnote{〔石眉巴鱼〕可能是红鳍笛鲷。},色红如金鱼。余俱不敢食,养盎中以为玩品。又有鳐如白鸟,云飞丈馀始入水,疑即燕鱼\footnote{〔燕鱼〕渤海地区对蓝点马鲛的称呼。}也。
    
    初六日丙戌,大风。是日,食品有蕉实,状如手指,不相属。色黄,味甘,瓤如柚,亦名甘露。闻初熟色青,以糠覆之则黄,与中国制柿无异。其花红,一穗数尺,瓣须五六出。岁实为常,实如其须之数。中国亦有蕉,不闻岁结实,亦无有抽其丝作布者;或其性殊欤?
    
    \begin{center}
        \textbf{行成归来}
    \end{center}
    
    二十日己巳,晴。东北风利,促解缆。卯刻,扬帆出那霸港;岸上、舟中送者如云,举手辞谢之。午刻,雨,入暮不止。伙长恐有暴,收马齿山安护浦下碇。山势横袤二十里,犬牙相错,出没海中,若断若续;分东、西二岛,为中山\footnote{〔中山〕指琉球国。琉球国全名为琉球中山王国。}第一外障。泊处青山围绕,无出路。有鹿见于山间,疑亦海鱼所化。雨景大佳。
    
    二十二日辛未,雨,风仍西北。午刻,晴。偕介山驾小舟登岸;沿沙洲行至山麓,有石高丈馀,玲珑可爱。坐石上观渔,皆赤身入水,无寒色。马齿人善泅,习使然也。
    
    二十四日癸酉,晴。北风少平,促伙长出洋;对以“风信未定”。余曰:“风信定,能无变乎?可行,则行!”介山曰:“姑俟之!”遂止。
    
    二十五日甲戌,晴。北风如故,决令开帆,介山亦以为然;遂于巳刻解缆。子丑风,用辛针。酉刻,过姑米山。终日峭帆,舟转驶,微侧而震,有吐者;余仍日坐将台,饮食如故。
    
    二十九日戊寅,辰卯风微,大雾,针如故。巳刻,稍霁;见温州南杞山\footnote{〔南杞山〕现称南麂岛,在浙江温州市平阳县东南海面。},舟人大喜。少顷,见杞山北有船数十只泊焉;舟人皆喜曰:“此必迎护船也!”雾渐消,山渐近;守备\footnote{〔守备〕清朝武官名,正五品,管理军队总务、军饷、军粮。}登后艄以望,惊报曰:“泊者,贼船也!”余曰:“舟巳至此,戒兵无哗!速食,备器械!”余亦饱食。守备又报贼船皆扬帆矣;与介山衣冠出,令吐者、病者悉归舱;登战台,誓众曰:“贼众我寡,尔等未免胆怯。然贼船小、我船大,彼络绎开帆,纵善驾驶,不能并集,犹一与一之势也。且既已遇之,惧亦无益!惟有以死相拼,可望死中求活。此我与汝致命之秋也,生死共之!”众兵勇气顿振,皆曰“惟命!”乃下令曰:“贼船未及三百步,不得放子母炮;未及八十步,不得放枪;未及四十步,不得放箭。如果近,始用长枪相拼。有能毙贼者,重赏;违者,按以军法”。各整暇以俟。
    
    未几,贼船十六只吆喝而来,第一只已入三百步。余举旗麾之,吴得进从舵门放子母炮,立毙四人,击喝者堕海;贼退不及,入百步,枪并发,又毙六人。一只乃退,二只又入三百步,复以炮击之,毙五人;稍进,又击之,复毙四人,乃退去。其时,三只贼船已占上风;暗移子母炮至舵右舷边,连毙贼十二人,焚其头蓬:皆转舵而退。中二船较大,复鼓噪由上风飞至。余曰:“此必贼首也!”密令舵工将船稍横,俟大炮准对贼船,即施放一发,中之。炮响后,烟迷里许;既散,则贼船巳尽退。是役也,王得禄首先士卒,兵丁吴得进、陈成德、林安顺、张大良、王名标、甘耀等枪炮俱无虚发,幸免于危。惟时日将暮,风甚微;恐贼乘夜来袭,默祷于天后\footnote{〔天后〕即妈祖,俗称“海神娘娘”,东南沿海民间崇拜的神灵。}求风。不一时,北风大至,浪飞过船。余倦极,思卧。念前险假遇害,岂复能虑此险!况求风得风,即忧亦无着力处。遂解衣熟睡,付之不见不闻。
    
    十一月朔日己卯,阴。梦中闻舟人哗曰:“到官塘\footnote{〔官塘〕官方的港口。塘:堤岸。}矣!”惊起。介山、从客皆一夜不眠,语余曰:“险至此,服汝能睡。设葬鱼腹,亦为糊涂鬼矣!”余曰:“险奈何”?介山曰:“上则九天\footnote{〔九天〕传说古代天地有九重,“九天”天的最高处,比喻极高处。也作“九重天”、“九霄”。后面“九地”指地的最深处,比喻极低处。},下则九地,声如转水车、锯湿木,时复疟颤;每侧,则篷皆卧水。一浪盖船,则船身入水,惟闻瀑布声垂流不息;其不覆者,幸耳!”余曰:“脱覆,君等能免之乎!余乐拾得一觉,又忘其险,余幸矣!”介山乃大笑。舟人指曰:“前即定海\footnote{〔定海〕指现在浙江省舟山市定海区。清代重要的港口和军事要塞。},可无虑!”申刻,乃得泊。总兵\footnote{〔总兵〕明清武官名。清代总兵统领一地汉军,正二品。}何定江来迎护,余笑谢之。因语以北杞之战,定江惶悚失措;余曰:“馁矣!他事且缓商”。
\end{normalsize}


\newpage

\textbf{译文}:

\vspace{-1em}

\begin{normalsize}
    
\begin{center}
    \textbf{出使琉球}
\end{center}
    
    初八日己丑,晴。中午风很大。黎明,有两只白鸟绕船飞翔。中午吹西南风,仍朝东南航行,总计航行了四更。下午四点,经过米糠洋。(海里的)漩涡都是圆的,波浪密而细,好像初筛的米,点点零落;“米糠”两字,非常形象。日落之后,总计又航行了三更;船伙长说:“鸡笼山、花瓶屿离船很远,应该见不到了。”这晚上,朝东南方向,航行了六更。船上呕吐的人很多;我天天坐在舰桥上,完全不觉得颠簸,饮食和平常一样。
    
    初九日庚寅,晴。早上五六点时,望见彭家山。彭家山并列三座峰,东峰最高,往西则低下。从出海开始算,船已经航行了十六更了;船从山北过。早上七八点时,转为西南风,船向转东,又航行了十更。下午四点正,望见钓鱼台,三座山峰分立,好像笔架,都是石头。当时水天一色,船平行岸边行驶,有无数白鸟绕船相送,也不知从哪里来的。入夜,星星的倒影(随波)横斜,月色(被浪)破成碎片,海面仿佛到处是点点火焰,浮沉出没。
    
    初十日辛卯,晴。吹西南风,仍朝东航行。东方黑云蔽日,水面上有白鸟无数。从彭家山到这里,合计航行了十四更。早上八点,看见赤尾屿;岛屿是方的,红色,东西向凸出而中部凹进,凹进的地方又有两个小山峰。船从山北经过。有两条大鱼,夹着船游着,看不见首尾,(鱼)背脊黑而微绿,好像十围粗的枯木附在船两侧;船上的人举酒相庆。
    
    十一日壬辰,阴。吹西南风,仍朝东航行。从赤尾屿到这里,合计航行了十四更。中午,看见姑米山。(姑米)山一共八个岭,每个岭各有一、二个山峰,有的断开,有的连续;船上的人欢喜的呼声让海都沸腾了。午后一两点的时候,(刮起了)大风,暴雨如注,但是雨下得虽然大,但风并不大。傍晚五六点,船已经靠近(姑米)山,合计又航行了五更。琉球人因为姑米山(附近)暗礁多,黑夜里不敢进港,等到天亮了才接着航行。下半夜一两点的时候,有小船来引导;于是起锚绕着(姑米)山继续往南航行。
    
    十二日癸巳,晴。早上七八点,经过马齿山。山势如同犬牙交错,有四个山峰相离而矗立,好像马在空中跑。合计又航行了七更,船转向东北航行,向那霸港驶去。考察以往的航海记录里所见到的地方,还有小琉球、鸡笼山、黄麻屿;这次航行都没有见到。向琉球伙长询问,(他)已经六十岁了,往来出海八次,说这次航行航路最简单快捷,而我们(途中)也只看见三座山,(然后)就到了姑米山。只是由于用烧香来记录更数,很难用来作为依据。据琉球伙长说:“海上行船,风太小固然不方便航行,风太大也不方便航行;风大则浪大,浪太大,其力量能把船堵在海上,进一尺就要退二寸。只有风七分、浪五分的情况,最适宜驾驶;这次就是这样。向来渡海,没有试过像这次一样平稳航行的。”早上八点,船进入那霸港。中午,(我们)上岸了。(琉球)全国的人都聚集来,在路上观看,世孙率领百官按照礼仪迎接(皇帝的)诏书。
    
    \begin{center}
        \textbf{琉球见闻}
    \end{center}
    
    十七日戊戌,晴。我翻阅桌子上的菜单,发现有所谓的“龙头虾”。取来一看,长一尺有余,甲壳一节一节的,有红色的长须,眼睛像血滴,鬃毛像火一样,就像世人所画的龙头,看了令人毛骨悚然!取它的壳来做灯,可以拿来玩两天;第三天颜色就变了。
    
    十八日己亥,雨。在使馆后庭栽了荔枝,南北各(栽种了)一列。(荔枝)是从牧荔园移植来的,品种叫“陈家紫”。
    
    二十一日壬寅,阴。连续几天吃海鲜,渐渐腹泻了。让厨子只供应时令蔬菜和稀粥。于是厨子做了佳苏鱼;问他,他说:“这不是鱼的名字,是削了黑鳗鱼的脊肉,腌干做成的。”(佳苏鱼)大概五、六寸长,形状像梭子、质地像枯木。使用的方法是先用温水浸洗,用蕉叶裹着煨,然后像刨花一样切片,(能)连切五、七片不断,好像兰花一样;最好沾酱油来吃,颇有异国风味。(琉球的)酱油,比福建甜美十倍。只是(我)想知道“佳苏”之含义而不得,恰好长史过来,问(他),(他)也不知道;于是把(厨子)叫到跟前仔细查核,据(厨子)说,“这个品种在我国又多又好吃,从国王、官员到穷人都能吃到”;意思大概是说它就好像家常蔬菜,人人都能吃到。琉球人用字都只求对上音就好,大概(佳苏鱼)本来应该叫“家蔬(鱼)”。
    
    二十二日癸卯,晴。午后,和介山一起策马到波上山游玩。(波上山)又叫石筍崖,因为形似而得名;(波上宫)四周是石墙,墙后可以望海。沿海很多露出水面的石头,嵌有空洞,玲珑剔透;潮水冲击它,发出钟磬一样的声音。东北边有一座山,叫雪崎山。再往东北,有一座小石山,叫龟山。往下走是护国寺,是国王祈雨的地方。龛内有黑色裸体的神像,手像剑一样竖立,形状很狰狞,名叫“不动”,有人说是火神。庭院中有一个景泰七年铸的钟,廊屋下又有一个乾隆五十七年新铸的钟。(护国)寺后有很多凤尾蕉,又叫铁树。西边有一块高五六尺的岩石,是黑色的,质地温润,形状像连起来的佛手。
    
    初二日癸丑,大暑,阴。下属官员到泊村游玩,回来拿着新稻穗给我看,说稻谷已经收完了;(我)才知道琉球山南边的土地由于气候温暖,稻谷经常早熟,十一月种稻谷,五、六月就收成了。红薯则四季都种,一年三熟就是丰收,一年四熟就是大丰收。稻田少、薯田多。琉球国人把红薯作为救命粮食,大米则是国王和官员才吃的起的。还有小麦和黄豆,产的不多。红薯在闽人的方言里也叫地瓜。午后,微雨。
    
    初五日丙辰,阴。早上十点多,下大雨。长史送来了四株佛桑。一株有千层花瓣,好像石榴,有深红、粉红二种颜色。一株花瓣是单层的,好像灯盘,花蕊单出,好像蜡烛,约二寸长,有红、白二种颜色;早上开花,傍晚就谢了,落下的花瓣卷起来像蜡烛。(佛桑)只开花不结果;四季都有花,到了深冬才落叶。这个地方很多花四季都开,因为气候温暖。我被长史的好意感动,吩咐下属官员用酒作为谢礼;希望(长史)下次有花再送来。
    
    初七日戊午,晴。早上七八点,微雨,很快就停了。长史又带了两盆花来送我,标的名字是“水翁花”;(我)看了看,其实是马兰花。中山国的草木,大多和中国称呼不一样;因为中山国书籍少,大多不认识自古以来草木的名字。比如把罗汉松叫做\hbox{\scalebox{0.4}[1]{木}\kern-.2em\scalebox{0.75}[1]{坚}}木;把冬青叫做福木;把万寿菊叫做禅菊:一开始按自己的想法命名,之后就沿用不改了。可惜(我)没带《群芳谱》来,(没法)一一考证辨识它们!
    
    十四日乙丑,阴。荔枝种下已经接近一个月了,长出了很多新叶子,有生机了。早上起来,偶而走过它旁边,看到新叶子上有虫咬的洞;靠近来看,有一种身体黄色而有黑色纹路的虫子,有两个触角、八只脚,虫身方而有毛,即世人所说的毛虫一类。(它)附在叶子上作巢,像小蜘蛛网一样盖在叶子上;(它)产卵像蚕,但产得更快,大的有两寸多。让仆人抓了埋到坑里,把它的巢全部扫除了。
    
    二十九日庚辰,晴。这一天,第一次见到五彩鱼。(五彩鱼)有红绿翠黄等各种颜色,绿鳞红纹,五彩相间。当地人根据形状和颜色来称呼它们,没有固定的名字。又有一种叫石眉巴鱼的,像金鱼一样,是红色的。我哪个都不敢吃,养在盎里赏玩品味。还有鳐鱼,像白色的鸟一样,可以飞出水面一丈多再入水,(我)怀疑就是燕鱼。
    
    初六日丙戌,大风。这一天,吃的东西里有香蕉果实,形状像手指一样,不相连。(它)是黄色的,吃起来甜,果肉像柚子,也叫甘露。听说刚熟的时候是青色的,用米糠盖在上面就变黄了,和中国处理柿子没有不同。它的花是红的,一穗花有数尺长,花瓣有五六须。一般来说每年都结果,(每穗)果实数量和须一样。中国也有香蕉,但没听过每年都结果的,也没有听过抽丝用来织布的;难道是(水土不同导致的)性状差异吗?
    
    \begin{center}
        \textbf{行成归来}
    \end{center}
    
    二十日己巳,晴。东北风很不错,(我)催促解缆开船。早上五点多,(我们)扬帆驶出那霸港;岸上、船上送的人很多,(我们)挥手辞谢。中午下雨,到了傍晚也没停。伙长怕有风暴,收了帆停靠在马齿山安护浦码头。(马齿山)山势横着有二十里宽,犬牙相错,出没海中,仿佛断开又仿佛连续;(马齿山)分东、西两个岛,是中山国对外的第一屏障。停泊的地方青山围绕,没有出路。有鹿出现在山里,(我)怀疑是海鱼化成的。雨中景色非常好看。
    
    二十二日辛未,雨,仍然吹西北风。中午天气放晴。我和介山驾小船上岸;沿沙洲走到山麓,有一块一丈多高的石,玲珑可爱。坐在石头上看渔民(劳作),都是赤裸着身子入水,不觉得寒冷。马齿人水性好,是从小练出来的。
    
    二十四日癸酉,晴。北风稍微变小了点,(我)催促伙长出海;(伙长)回答说“还没确定风的规律”。我说:“(即便现在)确定了风的规律,就不会再变吗?(现在)可以走,就走!”介山说:“姑且再等等吧!”于是作罢。
    
    二十五日甲戌,晴。仍然吹北风,(我)决定下令开船,介山也同意;于是在九点解缆开船。吹北风,朝西航行。下午五六点,经过姑米山。(船夫)一整天都在操帆,船转来转去地航行,略微倾侧且颠簸,有人呕吐;我仍然每天坐在将台,照常饮食。
    
    二十九日戊寅,早上七点后,吹微弱的东风,大雾弥漫,仍按原定罗盘针位航行。九点以后,天气稍有放晴;看见温州南杞山,船夫们非常高兴。不久后,看见杞山北边有数十只船停泊在那里;船夫们都高兴地说:“这一定是前来迎接和护送我们的船!”雾气逐渐消散,山也渐渐靠近;守备登上船尾眺望,惊慌地报告说:“停泊的船是海盗船!”我说:“船已经到了这里,命令士兵不要喧哗!赶快吃饭,准备武器!”我也饱餐一顿。守备又报告说海盗船都扬起了帆;我和介山整理衣冠后走出船舱,命令呕吐和生病的人都回到舱内;登上战斗平台,向众人发誓说:“敌人众多,我们人少,你们难免会感到胆怯。但敌船小,我们的船大,他们扬帆有先后,即使擅长驾驶,也无法同时齐聚,仍然是以一对一的态势。而且既然已经遇到了敌人,害怕也没有用!只有拼死一搏,才有可能在绝境中求得生存。这是我们拼死搏命的时候,(大家)同生共死!”士兵们的勇气顿时振奋,都说“听从命令!”于是(我们)下令说:“海盗船未靠近到三百步以内,不得发射子母炮;未靠近到八十步以内,不得开枪;未靠近到四十步以内,不得射箭。如果确实靠近,才用长枪与他们拼杀。有能击毙敌人的,重重赏赐;违抗命令的,按照军法惩处。”大家各自整理好武器,等待(敌人靠近)。
    
    没过多久,十六只海盗船吆喝着驶来,第一只已经进入三百步范围。我举起旗子指挥,吴得进从舵门发射子母炮,立即击毙四人,击中吆喝的海盗(使其)坠海;敌人来不及退,驶进百步以内,(我方)枪声齐发,又击毙六人。一只海盗船于是退去,(另外)两只海盗船又进入三百步范围,再次用炮攻击,击毙五人;它稍微靠近后,(我方)又用炮攻击,又击毙四人,(敌人)这才退去。这时,有三只海盗船已经占据了上风口;我们暗中将子母炮移到舵右舷边,连续击毙十二名海盗,烧毁了他们的船帆:海盗船都转舵退去。中间第二条海盗船较大,又鼓噪着从上风方向飞速驶来。我说:“这一定是海盗头目的船!”急忙命令舵工将船稍微横过来,等大炮对准海盗船后,立即发射一炮,击中了目标。炮响后,烟雾弥漫了一里多远;烟雾散去后,海盗船已经全部退去。这场战斗中,王得禄身先士卒,士兵吴得进、陈成德、林安顺、张大良、王名标、甘耀等人枪炮无一虚发,(我们)幸运地挺过了危险。只是当时天色已晚,风力很弱;担心海盗乘夜袭击,我默默向天后祈求来风。不一会儿,北风大作,浪花飞溅过船舷。我极度疲惫,想躺下(休息)。(我)想到之前的险状,要是(当时)遇害了,又哪能再来考虑现在的危险呢?况且之前求风,可不得到了风吗?即使忧虑也无济于事。
    
    十一月朔日己卯,阴。梦中听到船夫大叫:“到官塘了!”我惊醒起身。介山和随从都一夜没睡,对我说:“这么危险的情况,你还能睡着,真服了你。要是葬身鱼腹,也是个糊涂鬼了!”我说:“有多危险?”介山说:“往上好像到了九天之上,往下好像到了九地之下,(隆隆的)声响就像转水车、锯湿木一样,时不时又像得了疟疾一样震颤;每次(船)倾侧,船帆都卧倒在水里。一旦有浪盖在船上,船身都进到水里,只听得像瀑布声垂流不息;没有翻船,靠的是幸运啊!”我说:“倘若翻船了,你们逃得过吗?我乐得睡了一觉,又错过了危险,我太幸运了!”介山于是大笑。船夫指出:“前面就是定海了,不需要再担心了!”下午三四点,(船)才终于靠岸。总兵何定江来迎接护送,我笑着答谢他。于是跟他说起北杞的战斗,定江惊惶失措;我说:“(我)饿了!这些事以后再说吧。”
    
\end{normalsize}

\newpage

\textbf{注解}:

\vspace{-1em}

\begin{itemize}
    \setlength\itemsep{-0.2em}
    \item〔初九日庚寅〕传统记日的方法。“初九”是阴历记日,表示当月第九天;“庚寅”是干支记日,用天干地支顺序计数,每六十天一循环。两种记日方法并用,交叉比对,准确不出差错。
    \item〔卯刻〕传统记时方法。古人把一天分为十二个时辰,用十二地支记录。每个时辰又分为八刻。初刻、正刻,各一小时。“卯刻”也就是“卯时”,指早上五到七点,“申正”指“申时正刻”,即下午四点正。
    \item〔转丁未风,用单乙针〕传统罗盘方位记法,分八方二十四针,一方三针。又有单针、双针之分。单针指最近一针,双针指介于相邻双针之间。“丁”是南偏西南(南偏西15度),“未”是西南偏南(南偏西30度),“丁未”说明方向介于“丁”、“未”之间。“单乙针”表示以“乙”针(东偏南15度)为准航行。
    \item〔计自开洋,行船十六更矣〕“更”是明清时代发展出来的计量海路里程的方法。最初是把一夜分为五更,每更大约2.4小时。后来把一更航行的里程作为单位,每更大约50里(25公里)。李鼎元此次出海由于出发仓促,没有准备沙漏,靠焚香记时,因此有较大误差。实际路程远比十六更少,下同。
    \item〔鳐如白鸟,云飞丈馀始入水〕文鰩出海南,大者長尺許,有翅,與尾齊,一名飛魚。羣飛水上,海人候之,當有大風。
    \item〔其花红,一穗数尺,瓣须五六出。岁实为常,实如其须之数。〕香蕉的花与一般的花不太一样,并没有典型的花瓣,而是像笋壳一样的苞片。雌花的子房初时细长像须一样,逐渐发育为香蕉。一般来说,香蕉树结果之后第二年会枯萎死掉,然后重新从根部长出新的香蕉树,因此不会每年都结果。
\end{itemize}

\end{document}
