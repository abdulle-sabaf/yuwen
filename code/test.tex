\documentclass[12pt,UTF-8,openany]{ctexbook}
\usepackage{ctex}
\usepackage{titlesec}
\usepackage{xeCJK}
\usepackage{fontspec,xunicode,xltxtra}
\usepackage{xpinyin}
\usepackage{geometry}
\usepackage{indentfirst}

\geometry{a5paper,left=1.4cm,right=1.4cm,top=2.4cm,bottom=2.4cm}
\setmainfont{Arial}
\setCJKmainfont[BoldFont=STZhongsong]{汉字之美仿宋GBK 免费}
\xeCJKDeclareCharClass{CJK}{`0 -> `9}
\xeCJKsetup{AllowBreakBetweenPuncts=true}
\xpinyinsetup{ratio=0.5,hsep={.7em plus .7em},vsep={.9em}}
\titleformat{\chapter}{\zihao{-1}\bfseries}{ }{16pt}{}
\titleformat{\section}{\zihao{-2}\bfseries}{ }{0pt}{}
\title{\zihao{0} \bfseries 小学语文古诗集}
\setlength{\lineskip}{24pt}
\setlength{\parskip}{6pt}
\author{}
\date{}
\begin{document}
\maketitle
\tableofcontents
\newpage

\chapter{第一层}

\section{\centerline{江南}}

\begin{center}
    \vspace{10pt}
    
    \begin{Large}
        
        〔无名氏〕
        
    \end{Large}
    
    \vspace{8pt}
    
    \begin{Large}
        
        江南可采莲,
        
        莲叶何田田,
        
        鱼戏莲叶间。
        
        鱼戏莲叶东,
        
        鱼戏莲叶西,
        
        鱼戏莲叶南,
        
        鱼戏莲叶北。
        
    \end{Large}
    
\end{center}

\vspace{8pt}


\chapter{第二层}

\section{\centerline{画}}

\begin{center}
    \vspace{10pt}
    
    \begin{Large}
        
        〔无名氏〕
        
    \end{Large}
    
    \vspace{8pt}
    
    \begin{Large}
        
        远看山有色,近听水无声。
        
        春去花还在,人来鸟不惊。
        
    \end{Large}
    
\end{center}

\vspace{8pt}


\section{\centerline{春晓}}

\begin{center}
    \vspace{10pt}
    
    \begin{Large}
        
        〔孟浩然〕
        
    \end{Large}
    
    \vspace{8pt}
    
    \begin{Large}
        
        春眠不觉晓,处处闻啼鸟。
        
        夜来风雨声,花落知多少。
        
    \end{Large}
    
\end{center}

\vspace{8pt}


\section{\centerline{静夜思}}

\begin{center}
    \vspace{10pt}
    
    \begin{Large}
        
        〔李白〕
        
    \end{Large}
    
    \vspace{8pt}
    
    \begin{Large}
        
        床前明月光,疑是地上霜。
        
        举头望明月,低头思故乡。
        
    \end{Large}
    
\end{center}

\vspace{8pt}


\section{\centerline{登鹳雀楼}}

\begin{center}
    \vspace{10pt}
    
    \begin{Large}
        
        〔王之涣〕
        
    \end{Large}
    
    \vspace{8pt}
    
    \begin{Large}
        
        白日依山尽,黄河入海流。
        
        欲穷千里目,更上一层楼。
        
    \end{Large}
    
\end{center}

\vspace{8pt}


\chapter{第三层}

\section{\centerline{草}}

\begin{center}
    \vspace{10pt}
    
    \begin{Large}
        
        〔白居易〕
        
    \end{Large}
    
    \vspace{8pt}
    
    \begin{Large}
        
        离离原上草,一岁一枯荣。
        
        野火烧不尽,春风吹又生。
        
    \end{Large}
    
\end{center}

\vspace{8pt}


\section{\centerline{早发白帝城}}

\begin{center}
    \vspace{10pt}
    
    \begin{Large}
        
        〔李白〕
        
    \end{Large}
    
    \vspace{8pt}
    
    \begin{Large}
        
        朝辞白帝彩云间,千里江陵一日还。
        
        两岸猿声啼不住,轻舟已过万重山。
        
    \end{Large}
    
\end{center}

\vspace{8pt}


\section{\centerline{山行}}

\begin{center}
    \vspace{10pt}
    
    \begin{Large}
        
        〔杜牧〕
        
    \end{Large}
    
    \vspace{8pt}
    
    \begin{Large}
        
        远上寒山石径斜,白云深处有人家。
        
        停车坐爱枫林晚,霜叶红于二月花。
        
    \end{Large}
    
\end{center}

\vspace{8pt}


\section{\centerline{咏柳}}

\begin{center}
    \vspace{10pt}
    
    \begin{Large}
        
        〔贺知章〕
        
    \end{Large}
    
    \vspace{8pt}
    
    \begin{Large}
        
        碧玉妆成一树高,万条垂下绿丝绦。
        
        不知细叶谁裁出?二月春风似剪刀。
        
    \end{Large}
    
\end{center}

\vspace{8pt}


\section{\centerline{大林寺桃花}}

\begin{center}
    \vspace{10pt}
    
    \begin{Large}
        
        〔白居易〕
        
    \end{Large}
    
    \vspace{8pt}
    
    \begin{Large}
        
        人间四月芳菲尽,山寺桃花始盛开。
        
        长恨春归无觅处,不知转入此中来。
        
    \end{Large}
    
\end{center}

\vspace{8pt}


\section{\centerline{鸟鸣涧}}

\begin{center}
    \vspace{10pt}
    
    \begin{Large}
        
        〔王维〕
        
    \end{Large}
    
    \vspace{8pt}
    
    \begin{Large}
        
        人闲桂花落,夜静春山空。
        
        月出惊山鸟,时鸣春涧中。
        
    \end{Large}
    
\end{center}

\vspace{8pt}


\section{\centerline{咏华山}}

\begin{center}
    \vspace{10pt}
    
    \begin{Large}
        
        〔寇准〕
        
    \end{Large}
    
    \vspace{8pt}
    
    \begin{Large}
        
        只有天在上,更无山与齐。
        
        举头红日近,回首白云低。
        
    \end{Large}
    
\end{center}

\vspace{8pt}


\section{\centerline{望天门山}}

\begin{center}
    \vspace{10pt}
    
    \begin{Large}
        
        〔李白〕
        
    \end{Large}
    
    \vspace{8pt}
    
    \begin{Large}
        
        天门中断楚江开,碧水东流至此回。
        
        两岸青山相对出,孤帆一片日边来。
        
    \end{Large}
    
\end{center}

\vspace{8pt}


\section{\centerline{鹿柴}}

\begin{center}
    \vspace{10pt}
    
    \begin{Large}
        
        〔王维〕
        
    \end{Large}
    
    \vspace{8pt}
    
    \begin{Large}
        
        空山不见人,但闻人语响。
        
        返景入深林,复照青苔上。
        
    \end{Large}
    
\end{center}

\vspace{8pt}


\section{\centerline{江雪}}

\begin{center}
    \vspace{10pt}
    
    \begin{Large}
        
        〔柳宗元〕
        
    \end{Large}
    
    \vspace{8pt}
    
    \begin{Large}
        
        千山鸟飞绝,万径人踪灭。
        
        孤舟蓑笠翁,独钓寒江雪。
        
    \end{Large}
    
\end{center}

\vspace{8pt}


\section{\centerline{寻隐者不遇}}

\begin{center}
    \vspace{10pt}
    
    \begin{Large}
        
        〔贾岛〕
        
    \end{Large}
    
    \vspace{8pt}
    
    \begin{Large}
        
        松下问童子,言师采药去。
        
        只在此山中,云深不知处。
        
    \end{Large}
    
\end{center}

\vspace{8pt}


\section{\centerline{题西林壁}}

\begin{center}
    \vspace{10pt}
    
    \begin{Large}
        
        〔苏轼〕
        
    \end{Large}
    
    \vspace{8pt}
    
    \begin{Large}
        
        横看成岭侧成峰,远近高低各不同。
        
        不识庐山真面目,只缘身在此山中。
        
    \end{Large}
    
\end{center}

\vspace{8pt}


\section{\centerline{暮江吟}}

\begin{center}
    \vspace{10pt}
    
    \begin{Large}
        
        〔白居易〕
        
    \end{Large}
    
    \vspace{8pt}
    
    \begin{Large}
        
        一道残阳铺水中,半江瑟瑟半江红。
        
        可怜九月初三夜,露似真珠月似弓。
        
    \end{Large}
    
\end{center}

\vspace{8pt}


\section{\centerline{悯农(一)}}

\begin{center}
    \vspace{10pt}
    
    \begin{Large}
        
        〔李绅〕
        
    \end{Large}
    
    \vspace{8pt}
    
    \begin{Large}
        
        春种一粒粟,秋收万颗子。
        
        四海无闲田,农夫犹饿死。
        
    \end{Large}
    
\end{center}

\vspace{8pt}


\section{\centerline{悯农(二)}}

\begin{center}
    \vspace{10pt}
    
    \begin{Large}
        
        〔李绅〕
        
    \end{Large}
    
    \vspace{8pt}
    
    \begin{Large}
        
        锄禾日当午,汗滴禾下土。
        
        谁知盘中餐,粒粒皆辛苦。
        
    \end{Large}
    
\end{center}

\vspace{8pt}


\section{\centerline{舟夜书所见}}

\begin{center}
    \vspace{10pt}
    
    \begin{Large}
        
        〔查慎行〕
        
    \end{Large}
    
    \vspace{8pt}
    
    \begin{Large}
        
        月黑见渔灯,孤光一点萤。
        
        微微风簇浪,散作满河星。
        
    \end{Large}
    
\end{center}

\vspace{8pt}


\section{\centerline{江上渔者}}

\begin{center}
    \vspace{10pt}
    
    \begin{Large}
        
        〔范仲淹〕
        
    \end{Large}
    
    \vspace{8pt}
    
    \begin{Large}
        
        江上往来人,但爱鲈鱼美。
        
        君看一叶舟,出没风波里。
        
    \end{Large}
    
\end{center}

\vspace{8pt}


\section{\centerline{蚕妇}}

\begin{center}
    \vspace{10pt}
    
    \begin{Large}
        
        〔杜荀鹤〕
        
    \end{Large}
    
    \vspace{8pt}
    
    \begin{Large}
        
        昨日入城市,归来泪满巾。
        
        遍身罗绮者,不是养蚕人。
        
    \end{Large}
    
\end{center}

\vspace{8pt}


\section{\centerline{送元二使安西}}

\begin{center}
    \vspace{10pt}
    
    \begin{Large}
        
        〔王维〕
        
    \end{Large}
    
    \vspace{8pt}
    
    \begin{Large}
        
        渭城朝雨浥轻尘,客舍青青柳色新。
        
        劝君更尽一杯酒,西出阳关无故人。
        
    \end{Large}
    
\end{center}

\vspace{8pt}


\section{\centerline{送孟浩然之广陵}}

\begin{center}
    \vspace{10pt}
    
    \begin{Large}
        
        〔李白〕
        
    \end{Large}
    
    \vspace{8pt}
    
    \begin{Large}
        
        故人西辞黄鹤楼,烟花三月下扬州。
        
        孤帆远影碧空尽,唯见长江天际流。
        
    \end{Large}
    
\end{center}

\vspace{8pt}


\chapter{第四层}

\section{\centerline{望庐山瀑布}}

\begin{center}
    \vspace{10pt}
    
    \begin{Large}
        
        〔李白〕
        
    \end{Large}
    
    \vspace{8pt}
    
    \begin{Large}
        
        日照香炉生紫烟,遥看瀑布挂前川。
        
        飞流直下三千尺,疑是银河落九天。
        
    \end{Large}
    
\end{center}

\vspace{8pt}


\section{\centerline{绝句}}

\begin{center}
    \vspace{10pt}
    
    \begin{Large}
        
        〔杜甫〕
        
    \end{Large}
    
    \vspace{8pt}
    
    \begin{Large}
        
        两个黄鹂鸣翠柳,一行白鹭上青天
        
        窗含西岭千秋雪,门泊东吴万里船
        
    \end{Large}
    
\end{center}

\vspace{8pt}


\section{\centerline{滁州西涧}}

\begin{center}
    \vspace{10pt}
    
    \begin{Large}
        
        〔韦应物〕
        
    \end{Large}
    
    \vspace{8pt}
    
    \begin{Large}
        
        独怜幽草涧边生,上有黄鹂深树鸣。
        
        春潮带雨晚来急,野渡无人舟自横。
        
    \end{Large}
    
\end{center}

\vspace{8pt}


\section{\centerline{宿新市徐公店}}

\begin{center}
    \vspace{10pt}
    
    \begin{Large}
        
        〔杨万里〕
        
    \end{Large}
    
    \vspace{8pt}
    
    \begin{Large}
        
        篱落疏疏一径深,树头新绿未成阴。
        
        儿童急走追黄蝶,飞入菜花无处寻。
        
    \end{Large}
    
\end{center}

\vspace{8pt}


\section{\centerline{小儿垂钓}}

\begin{center}
    \vspace{10pt}
    
    \begin{Large}
        
        〔胡令能〕
        
    \end{Large}
    
    \vspace{8pt}
    
    \begin{Large}
        
        蓬头稚子学垂纶,侧坐莓苔草映身。
        
        路人借问遥招手,怕得鱼惊不应人。
        
    \end{Large}
    
\end{center}

\vspace{8pt}


\section{\centerline{夜宿山寺}}

\begin{center}
    \vspace{10pt}
    
    \begin{Large}
        
        〔李白〕
        
    \end{Large}
    
    \vspace{8pt}
    
    \begin{Large}
        
        危楼高百尺,手可摘星辰。
        
        不敢高声语,恐惊天上人。
        
    \end{Large}
    
\end{center}

\vspace{8pt}


\section{\centerline{宿建德江}}

\begin{center}
    \vspace{10pt}
    
    \begin{Large}
        
        〔孟浩然〕
        
    \end{Large}
    
    \vspace{8pt}
    
    \begin{Large}
        
        移舟泊烟渚,日暮客愁新。
        
        野旷天低树,江清月近人。
        
    \end{Large}
    
\end{center}

\vspace{8pt}


\section{\centerline{游子吟}}

\begin{center}
    \vspace{10pt}
    
    \begin{Large}
        
        〔孟郊〕
        
    \end{Large}
    
    \vspace{8pt}
    
    \begin{Large}
        
        慈母手中线,游子身上衣。
        
        临行密密缝,意恐迟迟归。
        
        谁言寸草心,报得三春晖。
        
    \end{Large}
    
\end{center}

\vspace{8pt}


\section{\centerline{梅花}}

\begin{center}
    \vspace{10pt}
    
    \begin{Large}
        
        〔王安石〕
        
    \end{Large}
    
    \vspace{8pt}
    
    \begin{Large}
        
        墙角数枝梅,凌寒独自开。
        
        遥知不是雪,为有暗香来。
        
    \end{Large}
    
\end{center}

\vspace{8pt}


\section{\centerline{芙蓉楼送辛渐}}

\begin{center}
    \vspace{10pt}
    
    \begin{Large}
        
        〔王昌龄 〕
        
    \end{Large}
    
    \vspace{8pt}
    
    \begin{Large}
        
        寒雨连江夜入吴,平明送客楚山孤。
        
        洛阳亲友如相问,一片冰心在玉壶。
        
    \end{Large}
    
\end{center}

\vspace{8pt}


\chapter{第五层}

\section{\centerline{墨梅}}

\begin{center}
    \vspace{10pt}
    
    \begin{Large}
        
        〔王冕〕
        
    \end{Large}
    
    \vspace{8pt}
    
    \begin{Large}
        
        我家洗砚池头树,朵朵花开淡墨痕。
        
        不要人夸好颜色,只留清气满乾坤。
        
    \end{Large}
    
\end{center}

\vspace{8pt}


\section{\centerline{早春呈水部张十八员外}}

\begin{center}
    \vspace{10pt}
    
    \begin{Large}
        
        〔韩愈〕
        
    \end{Large}
    
    \vspace{8pt}
    
    \begin{Large}
        
        天街小雨润如酥,草色遥看近却无。 
        
        最是一年春好处,绝胜烟柳满皇都。
        
    \end{Large}
    
\end{center}

\vspace{8pt}


\section{\centerline{江畔独步寻花}}

\begin{center}
    \vspace{10pt}
    
    \begin{Large}
        
        〔杜甫〕
        
    \end{Large}
    
    \vspace{8pt}
    
    \begin{Large}
        
        黄四娘家花满蹊,千朵万朵压枝低。
        
        流连戏蝶时时舞,自在娇莺恰恰啼。
        
    \end{Large}
    
\end{center}

\vspace{8pt}


\section{\centerline{游园不值}}

\begin{center}
    \vspace{10pt}
    
    \begin{Large}
        
        〔叶绍翁〕
        
    \end{Large}
    
    \vspace{8pt}
    
    \begin{Large}
        
        应怜屐齿印苍苔,小扣柴扉久不开。
        
        春色满园关不住,一枝红杏出墙来。
        
    \end{Large}
    
\end{center}

\vspace{8pt}


\section{\centerline{书湖阴先生壁}}

\begin{center}
    \vspace{10pt}
    
    \begin{Large}
        
        〔王安石〕
        
    \end{Large}
    
    \vspace{8pt}
    
    \begin{Large}
        
        茅檐长扫净无苔,花木成畦手自栽。
        
        一水护田将绿绕,两山排闼送青来。
        
    \end{Large}
    
\end{center}

\vspace{8pt}


\chapter{第六层}

\section{\centerline{关山月}}

\begin{center}
    \vspace{10pt}
    
    \begin{Large}
        
        〔李白〕
        
    \end{Large}
    
    \vspace{8pt}
    
    \begin{Large}
        
        明月出天山,苍茫云海间。
        
        长风几万里,吹度玉门关。
        
        汉下白登道,胡窥青海湾。
        
        由来征战地,不见有人还。
        
        戍客望边色,思归多苦颜。
        
        高楼当此夜,叹息未应闲。
        
    \end{Large}
    
\end{center}

\vspace{8pt}


\section{\centerline{十一月四日风雨大作}}

\begin{center}
    \vspace{10pt}
    
    \begin{Large}
        
        〔陆游〕
        
    \end{Large}
    
    \vspace{8pt}
    
    \begin{Large}
        
        僵卧孤村不自哀,尚思为国戍轮台。
        
        夜阑卧听风吹雨,铁马冰河入梦来。
        
    \end{Large}
    
\end{center}

\vspace{8pt}


\section{\centerline{泊船瓜洲}}

\begin{center}
    \vspace{10pt}
    
    \begin{Large}
        
        〔王安石〕
        
    \end{Large}
    
    \vspace{8pt}
    
    \begin{Large}
        
        京口瓜洲一水间,钟山只隔数重山。
        
        春风又绿江南岸,明月何时照我还。
        
    \end{Large}
    
\end{center}

\vspace{8pt}


\section{\centerline{秋思}}

\begin{center}
    \vspace{10pt}
    
    \begin{Large}
        
        〔张籍〕
        
    \end{Large}
    
    \vspace{8pt}
    
    \begin{Large}
        
        洛阳城里见秋风,欲作家书意万重。
        
        复恐匆匆说不尽,行人临发又开封。
        
    \end{Large}
    
\end{center}

\vspace{8pt}


\section{\centerline{八阵图}}

\begin{center}
    \vspace{10pt}
    
    \begin{Large}
        
        〔杜甫〕
        
    \end{Large}
    
    \vspace{8pt}
    
    \begin{Large}
        
        功盖三分国,名成八阵图。
        
        江流石不转,遣恨失吞吴。
        
    \end{Large}
    
\end{center}

\vspace{8pt}


\section{\centerline{渭城曲}}

\begin{center}
    \vspace{10pt}
    
    \begin{Large}
        
        〔王维〕
        
    \end{Large}
    
    \vspace{8pt}
    
    \begin{Large}
        
        渭城朝雨浥轻尘,客舍青青柳色新。
        
        劝君更尽一杯酒,西出阳关无故人。
        
    \end{Large}
    
\end{center}

\vspace{8pt}


\section{\centerline{清明}}

\begin{center}
    \vspace{10pt}
    
    \begin{Large}
        
        〔杜牧〕
        
    \end{Large}
    
    \vspace{8pt}
    
    \begin{Large}
        
        清明时节雨纷纷,路上行人欲断魂。
        
        借问酒家何处有,牧童遥指杏花村。
        
    \end{Large}
    
\end{center}

\vspace{8pt}


\chapter{第七层}

\section{\centerline{塞下曲}}

\begin{center}
    \vspace{10pt}
    
    \begin{Large}
        
        〔卢纶〕
        
    \end{Large}
    
    \vspace{8pt}
    
    \begin{Large}
        
        月黑雁飞高,单于夜遁逃。
        
        欲将轻骑逐,大雪满弓刀。
        
    \end{Large}
    
\end{center}

\vspace{8pt}


\section{\centerline{江村即事}}

\begin{center}
    \vspace{10pt}
    
    \begin{Large}
        
        〔司空曙〕
        
    \end{Large}
    
    \vspace{8pt}
    
    \begin{Large}
        
        钓罢归来不系船,江村月落正堪眠。
        
        纵然一夜风吹去,只在芦花浅水边。
        
    \end{Large}
    
\end{center}

\vspace{8pt}


\chapter{第八层}

\section{\centerline{出塞}}

\begin{center}
    \vspace{10pt}
    
    \begin{Large}
        
        〔王昌龄〕
        
    \end{Large}
    
    \vspace{8pt}
    
    \begin{Large}
        
        秦时明月汉时关,万里长征人未还。
        
        但使龙城飞将在,不教胡马渡阴山。
        
    \end{Large}
    
\end{center}

\vspace{8pt}


\section{\centerline{春日偶成}}

\begin{center}
    \vspace{10pt}
    
    \begin{Large}
        
        〔程颢〕
        
    \end{Large}
    
    \vspace{8pt}
    
    \begin{Large}
        
        云淡风轻近午天,傍花随柳过前川。
        
        时人不识余心乐,将谓偷闲学少年。
        
    \end{Large}
    
\end{center}

\vspace{8pt}


\section{\centerline{九月九日忆山东兄弟}}

\begin{center}
    \vspace{10pt}
    
    \begin{Large}
        
        〔王维〕
        
    \end{Large}
    
    \vspace{8pt}
    
    \begin{Large}
        
        独在异乡为异客,每逢佳节倍思亲。
        
        遥知兄弟登高处,遍插茱萸少一人。
        
    \end{Large}
    
\end{center}

\vspace{8pt}


\section{\centerline{观书有感}}

\begin{center}
    \vspace{10pt}
    
    \begin{Large}
        
        〔朱熹〕
        
    \end{Large}
    
    \vspace{8pt}
    
    \begin{Large}
        
        半亩方塘一鉴开,天光云影共徘徊。
        
        问渠哪得清如许?为有源头活水来。
        
    \end{Large}
    
\end{center}

\vspace{8pt}


\chapter{第九层}

\section{\centerline{凉州词}}

\begin{center}
    \vspace{10pt}
    
    \begin{Large}
        
        〔王翰〕
        
    \end{Large}
    
    \vspace{8pt}
    
    \begin{Large}
        
        葡萄美酒夜光杯,欲饮琵琶马上催。
        
        醉卧沙场君莫笑,古来征战几人回。
        
    \end{Large}
    
\end{center}

\vspace{8pt}


\section{\centerline{赠汪伦}}

\begin{center}
    \vspace{10pt}
    
    \begin{Large}
        
        〔李白〕
        
    \end{Large}
    
    \vspace{8pt}
    
    \begin{Large}
        
        李白乘舟将欲行,忽闻岸上踏歌声。
        
        桃花潭水深千尺,不及汪伦送我情。
        
    \end{Large}
    
\end{center}

\vspace{8pt}


\chapter{第十层}

\section{\centerline{枫桥夜泊}}

\begin{center}
    \vspace{10pt}
    
    \begin{Large}
        
        〔张继〕
        
    \end{Large}
    
    \vspace{8pt}
    
    \begin{Large}
        
        月落乌啼霜满天,江枫渔火对愁眠。
        
        姑苏城外寒山寺,夜半钟声到客船。
        
    \end{Large}
    
\end{center}

\vspace{8pt}


\section{\centerline{石灰吟}}

\begin{center}
    \vspace{10pt}
    
    \begin{Large}
        
        〔于谦〕
        
    \end{Large}
    
    \vspace{8pt}
    
    \begin{Large}
        
        千锤万凿出深山,烈火焚烧若等闲。
        
        粉身碎骨浑不怕,要留清白在人间。
        
    \end{Large}
    
\end{center}

\vspace{8pt}


\section{\centerline{乌衣巷}}

\begin{center}
    \vspace{10pt}
    
    \begin{Large}
        
        〔刘禹锡〕
        
    \end{Large}
    
    \vspace{8pt}
    
    \begin{Large}
        
        朱雀桥边野草花,乌衣巷口夕阳斜。
        
        旧时王谢堂前燕,飞入寻常百姓家。
        
    \end{Large}
    
\end{center}

\vspace{8pt}


\section{\centerline{江南春}}

\begin{center}
    \vspace{10pt}
    
    \begin{Large}
        
        〔杜牧〕
        
    \end{Large}
    
    \vspace{8pt}
    
    \begin{Large}
        
        千里莺啼绿映红,水村山郭酒旗风。
        
        南朝四百八十寺,多少楼台烟雨中。
        
    \end{Large}
    
\end{center}

\vspace{8pt}


\section{\centerline{示儿}}

\begin{center}
    \vspace{10pt}
    
    \begin{Large}
        
        〔陆游〕
        
    \end{Large}
    
    \vspace{8pt}
    
    \begin{Large}
        
        死去元知万事空,但悲不见九州同。
        
        王师北定中原日,家祭无忘告乃翁。
        
    \end{Large}
    
\end{center}

\vspace{8pt}


\section{\centerline{闻官军收河南河北}}

\begin{center}
    \vspace{10pt}
    
    \begin{Large}
        
        〔杜甫〕
        
    \end{Large}
    
    \vspace{8pt}
    
    \begin{Large}
        
        剑外忽传收蓟北,忽闻涕泪满衣裳。
        
        去看妻子愁何在,漫卷诗书喜欲狂。
        
        白日放歌须纵酒,青春作伴好还乡。
        
        即从巴峡穿巫峡,便下襄阳向洛阳。
        
    \end{Large}
    
\end{center}

\vspace{8pt}


\section{\centerline{寒食}}

\begin{center}
    \vspace{10pt}
    
    \begin{Large}
        
        〔韩翃〕
        
    \end{Large}
    
    \vspace{8pt}
    
    \begin{Large}
        
        春城无处不飞花,寒食东风御柳斜。
        
        日暮汉宫传蜡烛,轻烟散入五侯家。
        
    \end{Large}
    
\end{center}

\vspace{8pt}


\chapter{其他}

\section{\centerline{秋夜将晓出篱门迎凉有感}}

\begin{center}
    \vspace{10pt}
    
    \begin{Large}
        
        〔陆游〕
        
    \end{Large}
    
    \vspace{8pt}
    
    \begin{Large}
        
        三万里河东入海,五千仞岳上摩天。
        
        遗民泪尽胡尘里,南望王师又一年。
        
    \end{Large}
    
\end{center}

\vspace{8pt}


\section{\centerline{春夜喜雨}}

\begin{center}
    \vspace{10pt}
    
    \begin{Large}
        
        〔杜甫 〕
        
    \end{Large}
    
    \vspace{8pt}
    
    \begin{Large}
        
        好雨知时节,当春乃发生。随风潜入夜,润物细无声。
        
        野径云俱黑,江船火独明。晓看红湿处,花重锦官城。
        
    \end{Large}
    
\end{center}

\vspace{8pt}


\section{\centerline{望月怀远}}

\begin{center}
    \vspace{10pt}
    
    \begin{Large}
        
        〔张九龄〕
        
    \end{Large}
    
    \vspace{8pt}
    
    \begin{Large}
        
        海上生明月,天涯共此时。情人怨遥夜,竟夕起相思。
        
        灭烛怜光满,披衣觉露滋。不堪盈手赠,还寝梦佳期。
        
    \end{Large}
    
\end{center}

\vspace{8pt}


\section{\centerline{迢迢牵牛星}}

\begin{center}
    \vspace{10pt}
    
    \begin{Large}
        
        〔无名氏〕
        
    \end{Large}
    
    \vspace{8pt}
    
    \begin{Large}
        
        迢迢牵牛星,皎皎河汉女。
        
        纤纤擢素手,札札弄机杼。
        
        终日不成章,泣涕零如雨。
        
        河汉清且浅,相去复几许。
        
        盈盈一水间,脉脉不得语。
        
    \end{Large}
    
\end{center}

\vspace{8pt}


\section{\centerline{送杜少府之任蜀州}}

\begin{center}
    \vspace{10pt}
    
    \begin{Large}
        
        〔王勃〕
        
    \end{Large}
    
    \vspace{8pt}
    
    \begin{Large}
        
        城阙辅三秦,风烟望五津。与君离别意,同是宦游人。
        
        海内存知己,天涯若比邻。无为在岐路,儿女共沾巾。
        
    \end{Large}
    
\end{center}

\vspace{8pt}


\section{\centerline{春望}}

\begin{center}
    \vspace{10pt}
    
    \begin{Large}
        
        〔杜甫〕
        
    \end{Large}
    
    \vspace{8pt}
    
    \begin{Large}
        
        国破山河在,城春草木深。感时花溅泪,恨别鸟惊心。
        
        烽火连三月,家书抵万金。白头搔更短,浑欲不胜簪。
        
    \end{Large}
    
\end{center}

\vspace{8pt}


\section{\centerline{登岳阳楼}}

\begin{center}
    \vspace{10pt}
    
    \begin{Large}
        
        〔杜甫〕
        
    \end{Large}
    
    \vspace{8pt}
    
    \begin{Large}
        
        昔闻洞庭水,今上岳阳楼。吴楚东南坼,乾坤日夜浮。
        
        亲朋无一字,老病有孤舟。戎马关山北,凭轩涕泗流。
        
    \end{Large}
    
\end{center}

\vspace{8pt}


\section{\centerline{黄鹤楼}}

\begin{center}
    \vspace{10pt}
    
    \begin{Large}
        
        〔崔颢〕
        
    \end{Large}
    
    \vspace{8pt}
    
    \begin{Large}
        
        昔人已乘黄鹤去,此地空余黄鹤楼。黄鹤一去不复返,白云千载空悠悠。
        
        晴川历历汉阳树,芳草萋萋鹦鹉洲。日暮乡关何处是,烟波江上使人愁。
        
    \end{Large}
    
\end{center}

\vspace{8pt}


\section{\centerline{题都城南庄}}

\begin{center}
    \vspace{10pt}
    
    \begin{Large}
        
        〔崔护〕
        
    \end{Large}
    
    \vspace{8pt}
    
    \begin{Large}
        
        去年今日此门中,人面桃花相映红。
        
        人面不知何处去,桃花依旧笑春风。
        
    \end{Large}
    
\end{center}

\vspace{8pt}


\section{\centerline{七律·长征}}

\begin{center}
    \vspace{10pt}
    
    \begin{Large}
        
        〔毛泽东〕
        
    \end{Large}
    
    \vspace{8pt}
    
    \begin{Large}
        
        红军不怕远征难,万水千山只等闲。
        
        五岭逶迤腾细浪,乌蒙磅礴走泥丸。
        
        金沙水拍云崖暖,大渡桥横铁索寒。
        
        更喜岷山千里雪,三军过后尽开颜。
        
    \end{Large}
    
\end{center}

\vspace{8pt}


\end{document}
