\documentclass[12pt,UTF-8,openany]{ctexbook}
\usepackage{ctex}
\usepackage{titlesec}
\usepackage{xeCJK}
\usepackage{verse}
\usepackage{fontspec,xunicode,xltxtra}
\usepackage{xpinyin}
\usepackage{geometry}
\usepackage{indentfirst}
\usepackage{pifont}
\usepackage[perpage,symbol*]{footmisc}

\geometry{a5paper,left=1.4cm,right=1.4cm,top=2.3cm,bottom=2.3cm}
\renewcommand{\footnotesize}{\fontsize{8.5pt}{10.5pt}\selectfont}
\setmainfont{Mona Sans Light}
\setCJKmainfont[BoldFont=STZhongsong]{汉字之美仿宋GBK 免费}
\xeCJKDeclareCharClass{CJK}{`0 -> `9}
\xeCJKsetup{AllowBreakBetweenPuncts=true}
\DefineFNsymbols{circled}{{\ding{192}}{\ding{193}}{\ding{194}}{\ding{195}}{\ding{196}}{\ding{197}}{\ding{198}}{\ding{199}}{\ding{200}}{\ding{201}}}
\setfnsymbol{circled}
\xpinyinsetup{ratio=0.44,hsep={.6em plus .6em},vsep={1em}}
\titleformat{\chapter}{\zihao{-1}\bfseries}{ }{16pt}{}
\titleformat{\section}{\zihao{-2}\bfseries}{ }{0pt}{}
\title{\zihao{0} \bfseries 小学语文课文集萃}
\setlength{\lineskip}{24pt}
\setlength{\parskip}{6pt}
\author{}
\date{}
\begin{document}
\maketitle
\tableofcontents
\newpage

\chapter{北京}

\begin{large}
    
    北京是我国的首都,是一座美丽的城市。
    
    天安门在北京城的中央,红墙、黄瓦,又庄严,又美丽。天安门前面是宽阔的广场。广场中间竖立着人民英雄纪念碑。
    
    北京有许多又宽又长的大马路。道路两旁,绿树成荫,鲜花盛开。北京还有许多立交桥。立交桥的四围有绿油油的草坪和拼成图案的花坛。各种车辆在桥上桥下来来往往。
    
    北京有许多名胜古迹和风景优美的公园,还有许多高楼大厦。站在高处一看,全城到处是绿树,到处是大楼。
    
    北京真美啊!我们爱北京,我们爱祖国的首都!
    
\end{large}



\chapter{虫}

\begin{large}
    
    \begin{verse}[0.5\linewidth]
        蜻蜓草上展翅飞,蝴蝶花间捉迷藏。 \\
        蚯蚓土里造宫殿,蚂蚁树下运食粮。 \\
        蝌蚪池中游得欢,蜘蛛房前结网忙。
    \end{verse}
    
\end{large}



\chapter{狼和小羊}

\begin{large}
    
    狼来到小溪边,看见一只小羊在那儿喝水。
    
    狼想吃小羊,但又怕他逃了,就故意找碴儿,说:“你把我喝的水弄脏了!你安的什么心?”
    
    小羊吃了一惊,细声细气地说:“怎么会呢?您在上游,我在下游。水从您那儿流到我这儿来,不会从我这儿流到您那儿去。”
    
    狼转了转眼珠,又气冲冲地说:“就算这样吧,你总是个坏家伙。我听说,去年你在背地里说我的坏话。”
    
    小羊喊道:“那是不可能的!您一定是认错了。去年我还没有出生呢!”
    
    狼又转了转眼珠,说:“我可不信。我听说新生的小羊肚子上没有毛。给我看看你的肚子。”
    
    小羊躺下来,向狼露出肚子。狼走近两步,咧嘴一笑,扑上去,咬断了小羊的脖子。
    
\end{large}



\chapter{翠鸟}

\begin{large}
    
    翠鸟喜欢停在水边的苇秆上。它头顶的羽毛像橄榄色的头巾,布满亮青色的斑纹。一双透亮灵活的眼睛下面,长着一张又尖又长的嘴。背上的羽毛像翠色的袍子。肚子上的羽毛像橘色的衬衣。一双红色的小爪子紧紧抓着苇秆。
    
    翠鸟鸣声清脆,爱贴着水面疾飞。一眨眼儿,又轻轻地停在苇秆上了。它一动不动,注视着泛着微波的水面,等待游近水面的小鱼。
    
    小鱼悄悄地游上来,摆摆尾巴,嘴一张一张地,就想下去。尽管它这样机灵,还是逃不过翠鸟锐利的眼睛。翠鸟蹬开苇秆,像利箭一样,贴着水面飞过去,叼起小鱼,便远远飞开。只有苇秆还在摇晃,水波还在荡漾。
    
    我们真想捉一只翠鸟来养。渔翁伯伯跟我们说:“孩子们,你们知道翠鸟的家在哪里吗?沿着小溪上去,在那陡峭的石壁上,翠鸟就爱找那又窄又深的洞隙安家。捉起来可难了。”我们便打消了这个想法,只在翠鸟飞来的时候,远远望着它那艳丽的羽毛,希望它多停留一会儿。
    
\end{large}



\chapter{揠苗助长}

\begin{large}
    
    古时候宋国有个农夫,担心田里的禾苗迟迟不长,就跑到田里,把禾苗往上拔。他从早上忙到中午,筋疲力尽,回到家里,对家里人说:“今天可把我累坏了!田里的禾苗总不长,我帮它们长高了。”他的儿子连忙到田里去看。只见禾苗都枯死了。
    
\end{large}



\chapter{守株待兔}

\begin{large}
    
    古时候宋国有个农夫,他家的田里有个树桩。一天,他在田里干活。忽然,有只兔子跑到田里,撞到树桩,折断了脖子,死了。农夫白白得了一只兔子,开心极了。他想,要是天天有兔子吃,何必要耕田呢?从此他丢下了锄头,天天守在树桩旁,希望再有兔子跑来撞死。然而,一天天过去了,田里长满了杂草,却再也没有兔子跑进田里。
    
\end{large}



\chapter{初冬}

\begin{large}
    
    早上,白茫茫的一片大雾。
    
    远处的塔、小山,都望不见了。近处的田野、树林,像隔着一层纱,模模糊糊看不清。
    
    太阳像个红球,慢慢地升起来,发出淡淡的光,一点儿也不刺眼。
    
    地里的庄稼都收完了,人们正在园子里忙着收白菜。
    
    雾慢慢地散了,太阳射出光芒来。
    
    远处的塔、小山,都望得见了。近处的田野、树林,也看得清了。
    
    柿子树上挂了许多大柿子,像一个一个的红灯笼。
    
    树林里落了厚厚的一层黄叶。只有松树、柏树不怕冷,还是那么绿。
    
\end{large}



\chapter{秋天}

\begin{large}
    
    天那么高,那么蓝。高高的蓝天上,飘着几朵白云。
    
    蓝天下是一眼望不到边的稻田。稻子熟了,黄澄澄的,像铺了一地金子。
    
    稻田旁有个池塘。池塘边上有棵梧桐树。一片一片的黄叶,从树上落下来。有的落在水里,鱼儿游过去,藏在底下,把它当作伞。有的落在地上,蚂蚁爬上去,来回跑着,把它当作运动场。
    
    稻田上飞来两只燕子,看见树叶往下落,一边飞一边叫,好像在说:电报来了,催我们赶快到南方去呢!
    
\end{large}



\chapter{坐井观天}

\begin{large}
    
    青蛙坐在井里。燕子飞累了,在井沿上落脚。
    
    青蛙问燕子:“你从哪儿来啊?”
    
    燕子回答说:“我从北国来,穿过大海,穿过群山,每天在天上飞几十里,到温暖的南国去。”
    
    青蛙说:“朋友,别说大话了!天不过井口那么大,用得着飞那么远吗?”
    
    燕子说:“你错了。天无边无际,大得很哪!”
    
    青蛙笑了,说:“不可能。我天天在井里,一抬头就看见天。我不会弄错的。”
    
    燕子也笑了,说:“朋友,你错了。不信,你跳出井口来看一看吧。”
    
\end{large}



\chapter{骆驼和羊}

\begin{large}
    
    骆驼长得高,羊长得矮。骆驼说:“长得高好。”羊说:“不对,长得矮才好呢。”骆驼说:“我可以做一件事,证明高比矮好”。羊说:“我也可以做一件事,证明矮比高好。”
    
    他们走到一个园子旁边。园子四面有围墙,里面种了很多树,茂盛的枝叶伸出墙外来。骆驼一抬头就吃到了树叶。羊抬起前腿,扒在墙上,脖子伸得老长,还是吃不着。骆驼说:“你看,这可以证明了吧,高比矮好。”羊摇了摇头,不肯认输。
    
    他们俩又走了几步,看见围墙有个又窄又矮的门。羊大模大样地走进门去吃园子里的草。骆驼跪下前腿,低下头,往门里钻,怎么也钻不进去。羊说:“你看,这可以证明了吧,矮比高好。”骆驼摇了摇头,也不肯认输。
    
    他们去找老牛评理。老牛说:“你们俩都只看到自己的长处,看不到自己的短处。这是不对的。”
    
\end{large}



\chapter{狐狸和乌鸦}

\begin{large}
    
    一天早上,狐狸饿着肚子,出来找东西吃。他看到树上停着一只乌鸦,嘴里叼着一块肉。
    
    “这块肉应当是我的!”狐狸这么想着,蹭了蹭树脚,用崇拜的眼神望着乌鸦,喊道:“美丽的鸟儿,您好吗?”
    
    乌鸦看了看狐狸,没有回答。
    
    狐狸摇摇尾巴,又说:“您的羽毛真漂亮,五彩缤纷,比其它鸟都漂亮。您如此美貌,声音一定也很好听。可以唱首歌给我听吗?”
    
    乌鸦听了非常得意,就唱了起来。可是刚一张嘴,肉就掉了。狐狸叼着肉,一溜烟地跑了。
    
\end{large}



\chapter{曹冲称象}

\begin{large}
    
    曹冲是曹操的小儿子,五六岁的时候,已经很聪明了。孙权曾经送曹操一头大象。曹操想知道大象有多重,但手下的官员都不知道怎么称出大象的重量。曹冲知道了,说:“首先把大象放到船上,记下船吃水的位置。再把大象换下来,找一些石头,称好重量,逐个放到船上,直到船吃水和大象一样深为止。石头的重量加起来,就是大象的重量了  。”
    
\end{large}



\chapter{乌鸦喝水}

\begin{large}
    
    这是干燥的季节,乌鸦到处找水喝。它发现一个酒瓶,里面有半瓶水。瓶子很高,瓶颈很长。乌鸦把嘴伸进瓶口,却够不到瓶里的水。这可把乌鸦急的直跳。
    
    乌鸦想了个办法。它从别处叼来一些小石子,把石子投进瓶里。瓶里的石子越来越多,水位也越来越高。很快,水就离瓶口很近了。乌鸦把嘴伸进瓶口,喝到了水。
    
\end{large}



\chapter{狐狸和公鸡}

\begin{large}
    
    公鸡在树下找虫子吃。一只狐狸悄悄走近,公鸡连忙跳到树上。
    
    狐狸看着树上的公鸡,笑着说:“亲爱的公鸡,您不知道吗?昨天农场里发生了一件大好事!”
    
    “是吗?什么好事?”公鸡抬着头,四处张望。
    
    “我和农场约定好了,以后所有的动物都是好朋友,要和平相处,不会互相伤害了。”狐狸开心地咧开嘴,“来吧!请从树上下来,我们一起庆祝吧!我已经等不及和您一起跳舞了!”
    
    “真是个好消息。”公鸡漫不经心地回应着,转过头去,望着远处一个方向。
    
    “您怎么还不下来呢?”狐狸叫道,“您在看什么?”
    
    “喔,我看见农场里的几条大猎狗过来了。看来它们也听说了这个消息,想来——”
    
    “什么?我得走了!”狐狸扭头就跑。
    
    “怎么了?”公鸡问道,“它们不也是您的好朋友吗?”
    
    “他们大概还没听说这个消息。我突然有点急事,下次再庆祝吧!”狐狸已经跑远了。
    
    公鸡看着狐狸远去的背影,跳下树走了。
    
\end{large}



\chapter{老狼分饼}

\begin{large}
    
    小白兔和小灰兔得到了一张大饼。两只兔子很高兴,想把饼分了,各得一半。
    
    小白兔说:“我来把饼切成两半吧。”小灰兔不同意,说:“我怕你给我切少了。”两只兔子就争吵起来。
    
    这时来了一匹狼。狼看见两只兔子争吵,就问:“你们在争什么?”
    
    小灰兔说:“我们想把饼分成两半,可是找不到公平的办法。”
    
    老狼眼珠一转,说:“把饼给我,我一定让你们都满意。”说着,老狼把饼抓过来,撕成两块。
    
    小灰兔一看,说:“这两块饼一大一小,不公平!”
    
    老狼朝大的那块咬了一口,说:“现在这块和那块一样大了吧?”
    
    小白兔说:“现在这块反而比那块小了!”
    
    老狼又朝另一块饼咬了一口,说:“这样总行了吧?”
    
    小灰兔说:“不对,现在这块又太大了。”
    
    老狼抓着两块饼,这咬一口,那咬一口。很快,两块饼都只剩一点点了。老狼看了看手里的饼,对两只兔子说:“就剩这么点了,你们就别争了。”
    
    老狼说完,丢下剩下的饼,走了。两只兔子看着只剩一点点的饼,后悔极了。
    
\end{large}



\chapter{叶公好龙}

\begin{large}
    
    叶公子高很喜欢龙,衣带钩、酒勺子上都刻着龙,家住的屋子里雕刻装饰的也是龙。天上的真龙知道他这样爱龙,便从天上降到他家里,龙头搭在窗台上探望,龙尾伸到了厅堂里。叶公一看见真龙,吓得像失了魂似的,惊恐万状,转身就跑,直到见不到龙了才停下来。由此看来,叶公并不是真的喜欢龙,他喜欢的只不过是那些像龙却不是龙的东西罢了。
    
\end{large}



\chapter{十二月花名歌}

\begin{large}
    
    \begin{verse}[0.5\linewidth]
        正月山茶三五色, \\
        二月迎春花嫩黄。 \\
        三月桃花红十里, \\
        四月牡丹国色香。 \\
        五月石榴花似火, \\
        六月荷花遍池塘。 \\
        七月兰花生幽谷, \\
        八月桂花满枝芳。 \\
        九月菊花寒秋露, \\
        十月芙蓉正上妆。 \\
        冬月水仙清如玉, \\
        腊月寒梅斗雪霜。
    \end{verse}
    
\end{large}



\chapter{画蛇添足}

\begin{large}
    
    古时候楚国有个家族礼拜宗祠。仪式结束之后,酒还有剩下的,就送给来帮忙的人。几个帮忙的人说:“我们这里有好几个人,这酒不够分。但要是给一个人喝,又太多了。不如这样吧,大家各自在地上画一条蛇。谁先画好,就先喝酒。”
    
    其中一个人画得最快。他画好蛇的时候,别人还在画。于是他把酒拿过来,说:“哈哈!我画得最快,我先喝了。”他用左手拿着酒壶,右手继续在地上画,笑着说:“我还能给蛇画上脚呢!我把蛇的脚画完,你们也画不好蛇。”
    
    他还没画完蛇的脚,另一个人也画好了,一把抢过酒壶,说:“蛇没有脚,你怎么能给蛇画上脚呢?你画的不是蛇!”。大家纷纷说是。
    
    于是这个人再没法分到酒了。
    
\end{large}



\chapter{茅以升立志造桥}

\begin{large}
    
    茅以升是我国建造桥梁的专家。他小时候住在南京。每年端午节,南京的秦淮河上都举行龙舟比赛。每到这一天,秦淮河两岸人山人海,非常热闹。夏天一到,茅以升就和所有的小伙伴一样,盼望着到河上的文德桥看龙舟了。
    
    有一年端午节,茅以升病倒了。小伙伴们都去看龙舟了,茅以升一个人躺在床上,只好等小伙伴回来,给自己讲今天的比赛有多热闹。谁知到了傍晚,小伙伴跑来对他说:“不好了!秦淮河上出事了!”
    
    茅以升吃了一惊,连忙问:“出了什么事?”
    
    “文德桥塌了!好多人掉到河里了!”
    
    那一年,到桥上看龙舟的人太多,文德桥塌了,数百人丧命。
    
    病好了,茅以升一个人跑到秦淮河边,看着断桥发呆。平时好好的桥,却会突然倒塌,害死许多人,这是为什么呢?好桥坏桥,差别到底在哪里?他不禁想:我长大了也要学造桥,造永远不会塌的好桥!
    
    怀着这样的理想,十五岁时,茅以升考入了唐山路矿学堂,学习桥梁专业。之后,他又赴美国求学,取得博士学位。1937年9月,他主持建造的钱塘江大桥通车。这是一座双层铁路、公路两用桥,是中国人自己设计、建造的第一座现代化大桥。经过长期的努力,茅以升终于实现了自己的理想。
    
\end{large}



\chapter{美丽的小兴安岭}

\begin{large}
    
    小兴安岭在我国东北。那里有数不清的红松、白桦、栎树……几百里连成一片,就像树的海洋。
    
    春天,树木抽出新的枝条,长出嫩绿的叶子。山上的积雪融化了,雪水汇成小溪,淙淙流淌。溪边,小鹿俯下身子喝水,相互梳理毛发。水面上,一根根原木随着溪流漂去,像一支舰队在前进。
    
    夏天,树木长得葱葱茏茏,密密层层的枝叶,遮住了蓝蓝的天空。早晨,雾从山谷里升起来,把整片森林浸在乳白色中。太阳出来了,金色的阳光消散了雾气,透过树梢,照在工人宿舍门前的草地上。草地上满是各种各样的野花,红的,白的,黄的,紫的。草叶间,花瓣上,还有晶莹的露珠,反射着晨光,仿佛宝石一样。
    
    秋天,白桦和栎树的叶子变黄了,松树和柏树显得更苍翠了。凉风吹来,落叶在林间飞舞。这时的森林仿佛一座宝库,有酸甜可口的山葡萄和都柿,又香又脆的榛子和松子,还有各种各样的蘑菇、木耳等美味。
    
    冬天,下雪了,树上积满了白雪。地上的雪厚厚的,又松又软,常常没过膝盖。西北风呼呼地刮过树梢。紫貂和黑熊不得不躲进各自的洞里。紫貂捕到一只野兔当作美餐,黑熊只好睡大觉。松鼠上蹦下跳,靠秋天收藏在树洞里的松子过日子,有时还到枝头散散步,看看春天是不是快要来临。
    
    小兴安岭一年四季美景无穷,物产丰富,是大自然给我们的珍贵礼物。
    
\end{large}



\chapter{大海的歌}

\begin{large}
    
    早晨,我们一起床就得到通知,今天有船出海。我们马上向码头走去。眼前是广阔的天空,碧绿的大海,正从东方升起的朝阳。
    
    我们登上一艘海轮。海轮在港湾里静静地随着海波荡漾。船长邀请我们到驾驶室瞭望。只见海港两岸,钢铁巨人一般的装卸吊车有如密林,数不尽的巨臂上下挥动。飘着各色旗帜的海轮,就像待检阅的卫队,层层排列在码头两边。随着嘹亮的汽笛声,我们的海轮出发了,驶出了港口,进入大海。
    
    太阳升高了,波浪尖上闪烁着点点金光。我走向船头,迎着猛烈的海风,望着无边无际的大海。船头飞溅的浪花,奏着欣悦的序曲。
    
    船在大海中航行。海的颜色由绿变蓝,由碧蓝变成深蓝。这时,有人走近我身边,指着前方叫我看。顺着他指的方向,我极目远眺。海平线上,朦朦胧胧的雾气中,透出一片暗色的巨影,像是有一座城堡耸立在海天之间。他告诉我,那是咱们自己的石油钻探船。
    
    咱们自己的石油钻探船!啊!我仿佛能听见,大海在唱一曲新歌。
    
\end{large}



\chapter{让我们荡起双桨}

\begin{large}
    
    \begin{verse}[0.5\linewidth]
        让我们荡起双桨, \\
        小船儿推开波浪, \\
        海面倒映着美丽的白塔, \\
        四周环绕着绿树红墙。 \\
        小船儿轻轻飘荡在水中, \\
        迎面吹来了凉爽的风。
    \end{verse}
    
    
    \begin{verse}[0.5\linewidth]
        红领巾迎着太阳, \\
        阳光洒在海面上, \\
        水中的鱼儿望着我们, \\
        悄悄地听我们愉快歌唱。 \\
        小船儿轻轻飘荡在水中, \\
        迎面吹来了凉爽的风。
    \end{verse}
    
    
    \begin{verse}[0.5\linewidth]
        做完了一天的功课, \\
        我们来尽情欢乐, \\
        我问你亲爱的伙伴, \\
        谁给我们安排下幸福生活? \\
        小船儿轻轻飘荡在水中, \\
        迎面吹来了凉爽的风。
    \end{verse}
    
\end{large}



\chapter{小马过河}

\begin{large}
    
    小马独自在马棚里,望着远方的磨坊。马棚里的马常常到磨坊那儿干活,可他还小,从没去过。
    
    农夫走进马棚,看见小马,就说:“今天只有你在。你把这袋麦子送到磨坊那儿去吧。”
    
    小马高兴极了,驮起口袋,就往磨坊奔去。跑着跑着,一条小河挡住了去路。河水哗哗地流着。小马为难了,心想,我能不能过去呢?要是来之前打听清楚,该多好啊!可是他离家很远了。
    
    小马向四周望了望,看见一头老牛在河边吃草。他跑过去,问道:“牛伯伯,请您告诉我,这条河我能趟过去吗?”
    
    老牛说:“水很浅,我过河时,河水还没碰到我的肚子。不用担心,能趟过去。”
    
    小马听了老牛的话,立刻跑到河边,准备趟过去。突然,从树上跳下一只松鼠,拦住他大叫:“小马!别过河,别过河!你会淹死的!”
    
    小马吃惊地问:“河水很深吗?”
    
    松鼠认真地回答:“深得很呢!昨天,我的一个伙伴掉到河里,被河水冲走,淹死了。”
    
    小马连忙收住脚步,不知道怎么办才好。他叹了口气,说:“唉!还是问问妈妈吧!可妈妈还在干活呢!”
    
    一只白鹅在河面上梳理羽毛。听到小马叹气,不禁说道:“小马呀小马,你要用自己的脑袋好好想啊。”
    
    小马说:“大家的说法都不一样,我该听谁的呢?”
    
    白鹅说:“光听别人说,自己不动脑筋,不去试试,是不行的。”
    
    小马想了想说:“我的背比牛伯伯的肚子高。牛伯伯说水还没碰到肚子,我只要抬起头,应该就能趟过河。”
    
    小马跑到河边,刚要踏入水里,松鼠又大叫起来:“怎么,你不要命了!”
    
    小马说:“让我试试吧。”他下了河,小心地趟到了对岸。原来河水既不像老牛说的那样浅,也不像松鼠说的那样深。
    
\end{large}



\chapter{刻舟求剑}

\begin{large}
    
    古时候,有个楚国人坐船过河。船到河中的时候,他一不小心,把剑掉到河里了。于是,他在船上刻了个记号,说:“剑是从这里掉下去的。”
    
    有人问他:“不下河把剑捞上来,却在船上做记号,有什么用呢?”
    
    那人回答:“不急,等船靠岸了,我从这里跳下去,就能把剑捞上来了。”
    
\end{large}



\chapter{八角楼上}

\begin{large}
    
    在井冈山艰苦斗争的年代,毛主席住在茅坪村的八角楼。每当夜幕降临的时候,八角楼上的灯就亮了。
    
    这是个寒冬腊月的深夜,毛主席穿着单军衣,披着薄毯子,坐在竹椅上写文章。他右手握着笔,左手轻轻地拨了拨灯芯,灯光更加明亮了。凝视着这星星之火,毛主席在沉思,连毯子滑落下来也没觉察到。
    
    就在这盏清油灯下,毛主席写下了许多光辉著作,指明了中国革命胜利的道路。
    
\end{large}



\chapter{赵州桥}

\begin{large}
    
    河北省赵县的洨河上,有一座世界闻名的石拱桥,叫安济桥,又叫赵州桥。它是隋朝的石匠李春设计和参加建造的,已经有一千四百多年了。
    
    赵州桥非常雄伟。桥长五十多米,有九米多宽,中间行车马,两旁走人。这么长的桥,全部用石头砌成,下面没有桥墩,只有一个拱形的大桥洞,横跨在三十七米多宽的河面上。大桥洞顶上的左右两边,还各有两个拱形的小桥洞。平时,河水从大桥洞流过,发大水的时候,河水还可以从四个小桥洞流过。这种设计,在建桥史上是一个创举,既减轻了流水对桥身的冲击力,使桥不容易被大水冲毁,又减轻了桥身的重量,节省了石料。
    
    这座桥不但坚固,而且美观。桥面两侧有石栏,栏板上雕刻着精美的图案:有的刻着两条相互缠绕的龙,嘴里吐出美丽的水花;有的刻着两条飞龙,前爪相互抵着,各自回首遥望;还有的刻着双龙戏珠。所有的龙似乎都在游动,真像活了一样。
    
    赵州桥体现了古代劳动人民的智慧和才干,是我国宝贵的历史文化遗产。
    
\end{large}



\chapter{南京长江大桥}

\begin{large}
    
    早上,我参观了南京长江大桥。今天的天气格外好,万里碧空飘着朵朵白云。大桥在明媚的阳光下,显得十分壮丽。波浪滚滚的江水中,9个巨大的桥墩稳稳地托住桥身。正桥连接着22孔引桥,仿佛一条钢铁巨龙卧在大江上面。大桥分两层,底下一层是火车道,铺着双轨,上面一层是公路,公路两边是人行道。宽阔的公路上,行人车辆穿梭似的来来往往。
    
    我沿着人行道,走近正桥。两座雄伟的工农兵塑像左右挺立。塑像后面,耸立着两个高大的桥头堡,顶端的一面面红旗,映着阳光,十分艳丽。正桥笔直的公路上,一对对玉兰花灯柱,像等候检阅的队伍,站得整整齐齐。我手扶着桥栏杆,站在大桥上,远望江面。江上的轮船像一叶叶扁舟,随着波浪时起时伏。侧耳倾听,一列列火车鸣着汽笛,从脚下呼啸而过。
    
    滔滔的江水浩浩荡荡,奔向大海。自古称作天堑的长江,被我们征服了。一桥飞架南北,天堑变通途。
    
\end{large}



\chapter{雨}

\begin{large}
    
    星期天的下午,我坐在窗前做作业。屋里特别闷热。忽然,天色暗了下来,刮起一阵狂风,要下雨了。我赶快关上窗户。
    
    一会儿,粗大的雨点儿落下来了,打在玻璃窗上叭叭直响。雨越下越大。我透过玻璃窗向外望去。天地间像挂着无比宽大的珠帘,迷蒙蒙的一片。雨落在对面屋顶的瓦片上,溅起一朵朵水花,像一层薄烟笼罩在屋顶上。雨水顺着房檐流下来,开始像断了线的珠子,渐渐地连成了一条线。地上的水越来越多,汇合成一条条小溪。
    
    真是一场及时雨啊!大田里的玉米苗一定会咕咯咕咚喝个痛快。我仿佛看到雨水流进地里,流进果园里,流进人们的心窝里。
    
    云散了,雨住了,太阳照亮了大地。我推开窗户,一股泥土的清香迎面扑来。空气像滤过似的,格外清新。窗外的杨树、柳树,经过雨水的冲洗,舒枝展叶,绿得发亮,美丽极了。
    
\end{large}



\chapter{放风筝}

\begin{large}
    
    星期天的早晨,天气晴朗,阳光明媚。我和哥哥拿着叔叔帮我们做的风筝,高高兴兴地来到体育场。
    
    到体育场来放风筝的人可真不少。他们三个一群,两个一伙,有的已经把风筝放上了天空,有的举着风筝正要放。风筝花花绿绿,各式各样,有“老鹰”,有“鹦鹉”,有“仙鹤”,有“蜈蚣”……就是没有“大蜻蜓”。我跟哥哥说:“快,快点让咱们的‘大蜻蜓’飞上天吧。”
    
    哥哥让我端端正正地举着“大蜻蜓”,他拿着线轴,飞快地向前跑,边跑边放线。等他喊一声“放”,我赶紧松开手。哥哥拽着风筝又跑了一阵才收住脚,我们的“大蜻蜓”已经稳稳当当地飞上了天空。它那两对大翅膀微微地呼扇着,两只眼睛骨碌碌直转。
    
    这时候,有一架飞机从西边飞过来。啊,我们的“大蜻蜓”仿佛比飞机飞得还高呢。我高兴得一边拍手一边嚷:“蜻蜓赛过飞机啦!蜻蜓赛过飞机啦!”
    
    一会儿,飞来几只小鸟,它们围着“大蜻蜓”叽叽喳喳地叫,好像在奇怪地说:“你是从哪儿飞来的呀?好漂亮啊!”我正看得入神,西边又飞来一只美丽的“大蝴蝶”,橘红色的身子布满墨绿的斑纹,呼扇着翅膀缓缓上升。
    
    天空中的风筝越来越多,热闹极了。那金黄色的“小蜜蜂”,翘着两只绿色的翅膀,好像在百花丛中飞来飞去。那鲜红色的“大金鱼”,尾巴一摆一摆的,好像在水里游。还有那精致的“小卫星”,闪着金光,仿佛在宇宙中飞行……
    
    五颜六色的风筝随风飘荡,衬着瓦蓝瓦蓝的天空,显得更加鲜艳,更加美丽。
    
\end{large}



\chapter{荷花}

\begin{large}
    
    清晨,我到公园去玩,一进门就闻到一阵清香,我赶紧往荷花\footnote{〔荷花〕莲的花,又叫莲花。}池边跑去。
    
    荷花已经开了不少了。荷叶挨挨挤挤的,像一个个碧绿的大圆盘。白荷花在这些大圆盘之间冒出来。有的才展开两三片花瓣儿。有的花瓣儿全都展开了,露出嫩黄色的小莲蓬\footnote{〔莲蓬〕莲的花托,内有莲子。}。有的还是花骨朵儿\footnote{〔花骨朵〕快要开的花。},看起来饱胀得马上要破裂似的。
    
    这么多的白荷花,一朵有一朵的姿势。看看这一朵,很美;看看那一朵,也很美。如果把眼前的这一池荷花看做一大幅活的画,那画家的本领可真了不起。
    
    我忽然觉得自己仿佛就是一朵荷花,穿着雪白的衣裳,站在阳光里。一阵微风吹来,我就翩翩起舞,雪白的衣裳随风飘动。不光是我一朵,一池的荷花都在舞蹈。风过了,我停止舞蹈,静静地站在那儿。蜻蜓飞过来,告诉我清早飞行的快乐。鱼儿在脚下游过,告诉我昨夜做的好梦……
    
    过了好一会儿,我才记起我不是荷花,我是在看荷花呢。
    
\end{large}


\newpage

\textbf{注释}:

\vspace{-1em}

\begin{itemize}
    \setlength\itemsep{-0.2em}
    \item 〔翩翩起舞〕轻盈优雅地跳舞。
    \item 〔挨挨挤挤〕很多个紧挨着,不留空。
\end{itemize}

\chapter{掩耳盗铃}

\begin{large}
    
    晋国的贵族范氏败落之后,有个贼到范家偷东西,看到一口大钟,想把它偷走。大钟太重,贼背不动,就打算用锤子敲碎大钟。锤子敲在钟上,咣咣作响。贼怕声音太大,让人听见,就把自己的耳朵捂住,以为只要自己听不见,别人就听不见了。这难道不荒谬吗?
    
\end{large}



\chapter{自相矛盾}

\begin{large}
    
    从前楚国有一个商人,在市场卖矛和盾。他这样夸自己的盾:“我卖的盾可坚固了,没有什么东西能穿透它。”又这样夸自己的矛:“我卖的矛锋利极了,什么东西都能刺穿。”有人问:“用你的矛,刺你的盾,结果会怎么样呢?”商人回答不上来。
    
\end{large}



\chapter{滥竽充数}

\begin{large}
    
    齐宣王喜好排场,听人吹竽,总要找三百人一齐吹。但懂得吹竽的人不多,常常人手不足。
    
    有个住在南郭的处士,对齐宣王说:请让我来帮您解决吹竽的问题。齐宣王听了很高兴,给他数百人份的俸禄。
    
    齐宣王死后,他的儿子齐湣王也喜欢听吹竽。可是他不要听许多人一齐吹,而要一个一个地听。南郭处士就逃走了。
    
\end{large}



\chapter{惊弓之鸟}

\begin{large}
    
    更羸是魏国有名的射手。一天,更羸和魏王登高,看见天上的飞鸟,就对魏王说:“我不需要射中飞鸟,也能让它落下来。”
    
    魏王疑惑地问:“你的射术竟能达到这么神奇的地步吗?”
    
    更羸自信地说:“能!”
    
    过了一会儿,一只大雁从东方飞来。更羸张弓射箭。箭没有射中大雁,但大雁还是掉下地来。魏王十分惊讶,说:“竟然真有这样的射术!”
    
    更羸说:“这只大雁不仅飞得慢,而且鸣声悲切。飞得慢,是因为受过伤还没好,伤口还痛。鸣声悲切,是因为离开雁群很久了,失去同伴,内心慌乱。因此,听到弓弦的响声,心里害怕,就连忙往高处飞。结果旧伤复发,就掉下来了。”
    
\end{large}



\chapter{绿色的办公室}

\begin{large}
    
    1917年十月革命以前,有好几个月,列宁化装成割草工人,隐蔽在圣彼得堡西北的拉兹里夫湖畔。
    
    湖边的树林是列宁的“绿色的办公室”。屋顶是蔚蓝的天空,地板是碧绿的草地,一截树桩是列宁办公的椅子。树桩后面有个“人”字形草棚,草盖得厚厚的,只容得下一个人躺在里面。那是列宁的卧室。草棚的一头堆着高高的草垛。那是列宁,这位“割草工人”的劳动成果。对着草棚,两根树杈支着的横木吊着一口用旧了的锅,旁边放着一把黑铁水壶。那是列宁的厨房。
    
    天亮了,“绿色的办公室”沐浴在柔和的晨光中,列宁已经坐在了树桩上开始了一天的工作。他埋着头,双膝托着文件夹,笔尖在稿纸上沙沙地画着。身旁的草地上放着几页已经写好的稿子。不远的地方,篝火还在燃烧,锅里的早餐散发出一阵阵香气。列宁全神贯注地工作,忘记了周围的一切。
    
    在这个最简陋的“办公室”里,列宁写出了伟大的著作《国家与革命》,拟订了许多重要文件,有力地指导着俄国革命。
    
\end{large}



\chapter{黄继光}

\begin{large}
    
    1952年10月,上甘岭战役打响了。这是朝鲜战场上最激烈的一次阵地战。
    
    黄继光所在的营已经持续战斗了四天四夜;第五天夜晚接到上级的命令,要在黎明之前夺下敌人的597.9高地。
    
    进攻开始了,大炮在轰鸣。战士们占领了一个又一个山头,就要到达597.9高地的主峰了。突然,敌人一个火力点凶猛地射击起来。战士们屡次突击,都被比雨点还密的枪弹压了回来。
    
    东方升起了启明星,指导员看看表,已经四点多了。如果不快摧毁这个火力点,在黎明前就攻不下597.9高地的主峰,已经夺得的那些山头就会全部丢失。
    
    黄继光愤怒地注视着敌人的火力点,他转过身来坚定地对指导员说:“指导员,请把这个任务交给我吧!”
    
    指导员紧握着黄继光的手,说:“好,我相信你一定能完成这个光荣而艰巨的任务。”
    
    黄继光带上两个战士,拿了手雷,喊了一声“让祖国人民听我们胜利的消息吧”,向敌人的火力点爬去。
    
    敌人发现他们了。无数照明弹升上天空,黑夜变成了白天。炮弹在他们周围爆炸。他们冒着浓烟,冒着烈火,匍匐前进。一个战士牺牲了,另一个战士也负伤了。摧毁火力点的重任落在了黄继光一个人的肩上。
    
    火力点里的敌人把机枪对准黄继光,子弹像冰雹一样射过来。黄继光肩上腿上都负了伤。他用尽全身的力气,更加顽强地向前爬,还有二十米,十米……近了,更近了。
    
    啊!黄继光突然站起来了!在暴风雨一样的子弹中站起来了!他举起右臂,手雷在探照灯的光亮中闪闪发光。
    
    轰!敌人的火力点塌了半边,黄继光晕倒了。战士们赶紧冲上去,不料才冲到半路,敌人的机枪又叫起来,战士们被压在山坡上。
    
    天快亮了,规定的时间马上到了。指导员正在着急,只见黄继光又站起来了!他张开双臂,向喷射着火舌的火力点猛扑上去,用自己的胸膛堵住了敌人的枪口。
    
    “冲啊!为黄继光报仇!”喊声惊天动地。战士们像海涛一样向上冲,占领了597.9高地,消灭了阵地上的全部敌人!
    
\end{large}



\chapter{颐和园}

\begin{large}
    
    北京的颐和园是个美丽的大公园。
    
    进了颐和园的大门,绕过大殿,就来到有名的长廊。绿漆的柱子,红漆的栏杆,一眼望不到头。这条长廊有七百多米长,分成两百七十三间。每一间的横槛上都有五彩的画,画着人物、花草、风景,几千幅画没有哪两幅是相同的。长廊两旁栽满了花木,这一种花还没谢,那一种花又开了。微风从左边的昆明湖上吹来,使人神清气爽。
    
    走完长廊,就来到了万寿山脚下。抬头一看,一座八角宝塔形的三层建筑耸立在半山腰上,黄色的琉璃瓦闪闪发光。那就是佛香阁。下面的一排排金碧辉煌的宫殿,就是排云殿。
    
    登上万寿山,站在佛香阁的前面向下望,颐和园的景色大半收在眼底。葱郁的树丛,掩映着黄的绿的琉璃瓦屋顶和朱红的宫墙。正前面,昆明湖静的像一面镜子,绿的像一块碧玉。游船、画舫在湖面慢慢地滑过,几乎不留一点儿痕迹。向东远眺,隐隐约约可以望见几座古老的城楼和城里的白塔。
    
    从万寿山下来!就是昆明湖。昆明湖围着长长的堤岸,堤上有好几座式样不同的石桥,两岸栽着数不清的垂柳。湖中心有个小岛,远远望去,岛上一片葱绿,树丛中露出宫殿的一角。游人走过长长的石桥,就可以去小岛上玩。这座石桥有十七个桥洞,叫十七孔桥;桥栏杆上有上百根石柱,柱子上都雕刻着小狮子。这么多的狮子,姿态不一,没有哪两只是相同的。
    
    颐和园到处有美丽的景色,说也说不尽,希望你有机会去细细游赏。
    
\end{large}



\chapter{五彩池}

\begin{large}
    
    我小时候听奶奶讲,西方有座昆仑山,山上有个瑶池,那是天上神仙住的地方;池里的水好看极了,有五种颜色,红的,黄的,绿的,蓝的,紫的。奶奶是哄着我玩儿,我却当作了真的,真想有一天能遇上神仙,跟着他腾云驾雾,飞到那五彩的池边去看看。没想到今年夏天去四川松潘旅游,在藏龙山上,我真的看到了像瑶池那样神奇的五彩池。
    
    那是个晴朗的日子,我乘汽车来到藏龙山,只见漫山遍野都是大大小小的水池。无数的水池在灿烂的阳光下,闪耀着各种不同的颜色的光辉,好像是铺展着的巨幅地毯上的宝石。水池大的面积不足一亩,水深不过一丈;小的像个菜碟,水很浅,用小拇指就能触到池底。池边是金黄色的石粉凝成的,像一圈圈彩带,把大大小小的水池围成各种不同的形状,有像葫芦的,有像镰刀的,有像盘子的,有像莲花的……
    
    更使我惊奇的是,所有的池水来自同一条溪流,溪水流到各个水池里,颜色却不同了。有些水池的水还不止一种颜色,上层是咖啡色的,下层却成了柠檬黄;左半边是天蓝色的,右半边却成了橄榄绿。可是把水舀起来看,又跟普通的清水一个样,什么颜色也没有。
    
    明明是清水,为什么在水池里会显出不同的颜色来呢?原来池底长着许多石笋,有的像起伏的丘陵,有的像险峻的山峰,有的像矗立的宝塔,有的像成簇的珊瑚。石笋表面凝结着一层细腻的透明的石粉。阳光透过池水射到池底,石笋就像高低不平的折光镜,把阳光折射成各种不同的色彩。水池周围的树木花草长得很茂盛,五光十色的倒影使池水更加瑰丽。
    
    原来五彩的瑶池就在人间,不在天上。
    
\end{large}



\chapter{青蛙的眼睛}

\begin{large}
    
    青蛙喜欢吃昆虫。苍蝇,蚊子,白蛉,蚱蜢,它都爱吃。它鼓着一双大眼睛,蹲在池塘边上,只要有虫子飞过,它噌地跳起来,舌头一伸,就把虫子卷进嘴里去了。
    
    有人把青蛙养在笼子里,拿许多死苍蝇放在笼子里来喂它。可是奇怪,青蛙一只也不吃,竟活活饿死了。是不是因为苍蝇是死的,青蛙不爱吃呢?不是。只要把死苍蝇拴在线上,在青蛙眼前掠过,青蛙跳起来就把它吞了,跟吃活的苍蝇一个样。
    
    青蛙的眼睛非常特殊,看动的东西很敏锐,看静的东西却很迟钝。只要虫子在飞,飞得多快,往哪个方向飞,它都能分辨清楚,还能判断什么时候跳起来准能把虫子逮住。可是虫子如果停住不飞,它就看不见了。所以拿死苍蝇来喂青蛙,青蛙不知道眼前放着可吃的东西,只好活活饿死。
    
\end{large}



\chapter{爬山虎的脚}

\begin{large}
    
    学校操场北边墙上满是爬山虎。我家也有爬山虎,从小院的西墙爬上去,在房顶上占了一大片地方。
    
    爬山虎刚长出来的叶子是嫩红的,不几天叶子长大,就变成嫩绿的。爬山虎的嫩叶,不大引人注意,引人注意的是长大了的叶子。那些叶子绿得那么新鲜,看着非常舒服。叶尖一顺儿朝下,在墙上铺得那么均匀,没有重叠起来的,也不留一点儿空隙。一阵风拂过,一墙的叶子就漾起波纹,好看得很。
    
    以前,我只知道这种植物叫爬山虎,可不知道它怎么能爬。今年,我注意了,原来爬山虎是有脚的。爬山虎的脚长在茎上。茎上长叶柄的地方,反面伸出枝状的六七根细丝,每根细丝像蜗牛的触角。细丝跟新叶子一样,也是嫩红的。这就是爬山虎的脚。
    
    爬山虎的脚触着墙的时候,六七根细丝的头上就变成小圆片,巴住墙。细丝原先是直的,现在弯曲了,把爬山虎的嫩茎拉一把,使它紧贴在墙上。爬山虎就是这样一脚一脚地往上爬。如果你仔细看那些细小的脚,你会想起图画上蛟龙的爪子。
    
    爬山虎的脚要是没触着墙,不几天就萎了,后来连痕迹也没有了。触着墙的,细丝和小圆片逐渐变成灰色。不要瞧不起那些灰色的脚,那些脚巴在墙上相当牢固,要是你的手指不费一点儿劲,休想拉下爬山虎的一根茎。
    
\end{large}


\newpage

\textbf{注释}:

\vspace{-1em}

\begin{itemize}
    \setlength\itemsep{-0.2em}
    \item 〔重叠〕盖住同一处。
    \item 〔空隙〕细小的空处。
    \item 〔漾〕水面起伏。引申为受扰稍动。
    \item 〔巴〕粘附。
    \item 〔蛟〕传说生物。水族龙属,似蛇而四足。
    \item 〔萎〕
    \item 〔痕迹〕
\end{itemize}

\chapter{课间十分钟}

\begin{large}
    
    下课铃响了,同学们快步走出教室,到操场上参加自己喜爱的课间活动。校园里顿时沸腾起来。
    
    校园的东墙边,有一张乒乓球台。球台的四周围满了同学,不时传来喝彩声和欢笑声。乒兵小将们打得多认真啊!他们你推我挡,一个球常常打了十几个回合还不分胜负。
    
    球台右边的大槐树下,也围着一些同学。他们在爬竿。一个大同学刚从竿上滑下来,一个小同学纵身一跃,用力抓住竹竿,像敏捷的猴子,迅速地爬了上去。不ー会儿,他就爬到了竿顶。多高兴啊,他笑着向下张望。
    
    “丢沙包”是同学们十分喜爱的活动。操场的西墙边,这一组,那一组,玩得多带劲儿!两头丢包的同学密切合作,向中间的同学发动猛攻。中间躲包的同学非常沉着,眼睛盯着沙包飞来的方向,左躲右闪,蹦来跳去。沙包飞来了,只见这个同学轻巧地一抬腿,沙包嗖地从裤腿边飞了过去。沙包又从背后飞来了,她猛地一转身来个海底捞月,抓住了沙包。他们蹦啊跳啊,心里多么欢畅。
    
    课间活动真是丰富多彩。看,操场中间,有的跳皮筋儿,有的跳绳,有的踢毽子。最有趣的是一年级的小同学,他们由老师带着,在做老鹰捉小鸡的游戏呢。
    
\end{large}



\chapter{日出}

\begin{large}
    
    太阳还没有出来。东边灰暗的天变成暗红色的了。天上的云像放牧的羊群,像飞奔的马,像摔跤的大力士,像起舞的仙女,在不断变化着。东边的云渐渐地红了,亮了。
    
    太阳出来了。先是像红色的线;不一会儿,像烧红的镰刀,像半只橘子;一转眼,太阳像红色的大气球,慢慢地从东方升起来了。
    
    阳光躲过云彩,穿过树丛,透过晨雾,斜斜地、慢慢地、密密地,洒满了大地。拖拉机突突地下地了。农民骑着自行车、摩托车,三五成群地下地了。
    
    田野的早晨是令人神往的。这令人神往的美景是大自然给予的,也是人们用劳动创造的。
    
\end{large}



\chapter{捞铁牛}

\begin{large}
    
    宋朝的时候,有一回黄河发大水,冲走了河中府的一座浮桥。河两岸拴住浮桥的八只大铁牛,也被大水冲走了,陷在河底的淤泥里。
    
    洪水退了,马上要重修浮桥,可是大铁牛还在河底,怎么办呢?官府也没办法,到处寻找能把铁牛捞出来的人。
    
    一个和尚听说了,说:“我来试试看。铁牛是被水冲走的,我还叫水把它们送回来。”
    
    和尚让人准备两只大船,船上装满泥沙,行驶到铁牛沉没的地方。船停稳了,他叫人把两只船并排拴得紧紧的,用结实的木料搭个架子,横跨在两只船上。又请熟悉水性的人带了很粗的绳子潜到水底,把绳子的一头牢牢地拴住铁牛,绳子的另一头绑在船之间的架子上。
    
    准备工作做好了,和尚让船工把船上的泥沙铲到河里去。船里的泥沙渐渐地少了,船身慢慢地向上浮,拴住铁牛的绳子越绷越紧,铁牛一点儿一点儿地从淤泥里向上拔。船上的泥沙铲空了,铁牛也离开了河底。和尚让水手们使劲划桨,两只大船终于把水里的铁牛拖回岸边。
    
    用同样的办法,一只一只大铁牛都拖了回来。
    
\end{large}



\chapter{纸上谈兵}

\begin{large}
    
    赵括是赵国名将赵奢的儿子。他从小就读了很多兵书,提起用兵作战、调兵遣将的方法,有理有据,滔滔不绝,赵奢也辩论不过他。赵括自认为用兵天下无敌,他的父亲却不同意,对他的母亲说:“用兵事关生死,他却把打仗当成下棋一样。要是让他当将军,赵国的军队必定要打败仗。”
    
    赵奢死后,有一年,赵国与秦国在长平打仗。当时赵国的将军是廉颇。秦军数次引诱赵军出城交战,廉颇坚持不出战。赵王觉得廉颇不会打仗,想起赵奢来,就想让赵括代替廉颇做将军。赵括的母亲求见赵王,说:“不能让赵括当将军。”赵王问她原因,赵括的母亲说:“我丈夫做将军的时候,家里养了数十个参谋,还有数百个交好的。大王赏赐我家的财宝,都给了士兵。赵括率兵的时候,把财宝拿来给自己买田买房。检阅军队时,士兵都不敢看他。他和他父亲不一样,愿大王莫要让他当将军。”赵王仍然坚持让赵括做将军。赵括的母亲说:“如果赵括败了,请大王只处罚他,不要连累我赵家其他人。”赵王答应了。
    
    赵括做了将军后,马上换了军官,改了军法,士兵都不满意。一天,秦军又来挑战。赵括率军出城。秦军假装不敌,引诱赵括率军来追。赵括远离城市,又中了秦军埋伏,断了后路,被困在野外。他想了很多精妙的办法突围,但士兵都不信他。四十多天后,赵军又饿又累,赵括只好自己出战,结果被秦军射杀。赵军投降,秦军大获全胜。赵王想起赵括母亲的话,后悔极了。
    
\end{large}



\chapter{趵突泉}

\begin{large}
    
    千佛山、大明湖和趵突泉,是济南的三大名胜。现在单讲趵突泉。
    
    出了济南的西门,在西门外的桥上,便看见一溪活水,清浅,鲜洁,由南向北的流着。这就是由趵突泉流出来的。倘若没有这泉,济南定会失去它一半的美。
    
    沿着小溪往南走,就来到趵突泉公园。一个开阔的池子,差不多是见方的,占了大半个公园。池里的水清极了,游鱼水藻,都看得清清楚楚。泉池中央偏西,有三个大泉眼。水从泉眼里往上涌,冒出水面半米来高,像煮沸了似的。一年四季,昼夜不停,不断地翻滚。永远那么纯洁,永远那么活泼,永远那么鲜明,冒,冒,冒,永不疲乏,永不退缩。大自然才有这般的力量!冬天更好,泉上起了一片热气,白而轻软,在深绿绵长的水藻上飘荡着,使你不由得陷入一种神秘的境界。
    
    池边还有小泉呢:有的象大鱼吐水,极轻快地上来一串小泡;有的象一串明珠,走到中途又歪下去,真象一串珍珠在水里斜放着;有的半天才上来一个泡,大,扁一点,慢慢地,有姿态地,摇动上来,碰着水面,碎了;看,又来了一个!有的好几串小碎珠一齐挤上来,象一朵攒整齐的珠花,雪白……这比那大泉还更有趣。
    
\end{large}



\chapter{鸟的天堂}

\begin{large}
    
    我们吃过晚饭,暑气已经退了。太阳落下了山坡,只留下一段灿烂的红霞在天边。
    
    我们走过一段石子路,很快就到了河边。在河边大树下,我们发现了几只小船。
    
    我们陆续跳上一只船。一个朋友解开了绳,拿起竹竿一拨,船缓缓地动了,向河中心移去。
    
    河面很宽,白茫茫的水上没有一点波浪。船平静地在水面移动。三支桨有规律地在水里划,那声音就像一支乐曲。
    
    在一个地方,河面变窄了。一簇簇树叶伸到水面上。树叶真绿得可爱。那是许多株茂盛的榕树,看不出主干在什么地方。
    
    当我说许多株榕树的时候,朋友们马上纠正我的错误。一个朋友说那里只有一株榕树,另一个朋友说是两株。我见过不少榕树,这样大的还是第一次看见。
    
    我们的船渐渐逼近榕树了。我有机会看清它的真面目,真是一株大树,枝干的数目不可计数。枝上又生根,有许多根直垂到地上,伸进泥土里。一部分树枝垂到水面,从远处看,就像一株大树卧在水面上。
    
    榕树正在茂盛的时期,好像把它的全部生命力展示给我们看。那么多的绿叶,一簇堆在另一簇上面,不留一点儿缝隙。那翠绿的颜色,明亮地照耀着我们的眼睛,似乎每一片绿叶上都有一个新的生命在颤动。这美丽的南国的树!
    
    船在树下泊了片刻。岸上很湿,我们没有上去。朋友说这里是“鸟的天堂”,有许多鸟在这树上做巢,农民不许人去捉它们。我仿佛听见几只鸟扑翅的声音,等我注意去看,却不见一只鸟的影儿。只有无数的树根立在地上,像许多根木桩。土地是湿的,大概涨潮的时候河水会冲上岸去。“鸟的天堂”里没有一只鸟,我不禁这样想。于是船开了,一个朋友拨着桨,船缓缓地移向河中心。
    
    第二天,我们划着船到一个朋友的家乡去。那是个有山有塔的地方。从学校出发,我们又经过那“鸟的天堂”。
    
    这一次是在早晨。阳光照耀在水面,在树梢,一切都显得更加光明了。我们又把船在树下泊了片刻。
    
    起初周围是静寂的。后来忽然起了一声鸟叫。我们把手一拍,便看见一只大鸟飞了起来。接着又看见第二只,第三只。我们继续拍掌,树上就变得热闹了,到处都是鸟声,到处都是鸟影。大的,小的,花的,黑的,有的站在树枝上叫,有的飞起来,有的在扑翅膀。
    
    我注意地看着,眼睛应接不暇,看清楚了这只,又错过了那只,看见了那只,另一只又飞起来了。一只画眉鸟飞了出来,被我们的掌声一吓,又飞进了叶丛,站在一根小枝上兴奋地叫着,那歌声真好听。
    
    当小船向着高塔下面的乡村划去的时候,我回头看那被抛在后面的茂盛的榕树。我感到一点儿留恋。昨天是我的眼睛骗了我,“鸟的天堂”的确是鸟的天堂啊!
    
\end{large}



\chapter{桂林山水}

\begin{large}
    
    人们都说:“桂林山水甲天下。”我们乘着木船,荡漾在漓江上,来观赏桂林的山水。
    
    我看见过波澜壮阔的大海,玩赏过水平如镜的西湖,却从没看见过漓江这样的水。漓江的水真静啊,静得让你感觉不到它在流动;漓江的水真清啊,清得可以看见江底的沙石;漓江的水真绿啊,绿得仿佛那是一块无瑕的翡翠。船桨激起的微波扩散出一道道水纹,才让你感觉到船在前进,岸在后移。
    
    我攀登过峰峦雄伟的泰山,游览过红叶似火的香山,却从没看见过桂林这一带的山,桂林的山真奇啊,一座座拔地而起,各不相连,像老人,像巨象,像骆驼,奇峰罗列,形态万千;桂林的山真秀啊,像翠绿的屏障,像新生的竹笋,色彩明丽,倒映水中;桂林的山真险啊,危峰兀立,怪石嶙峋,好像一不小心就会栽倒下来。
    
    这样的山围绕着这样的水,这样的水倒映着这样的山,再加上空中云雾迷蒙,山间绿树红花,江上竹筏小舟,让你感到像是走进了连绵不断的画卷,真是“舟行碧波上,人在画中游”。
    
\end{large}



\chapter{天安门广场}

\begin{large}
    
    天安门广场位于首都北京的中心,它是世界上最宽广、最壮观的城市广场。
    
    广场北端是天安门。天安门红墙黄瓦,雕梁画栋,显得雄伟壮丽。天安门前是金水河,河上横跨着五座汉白玉石桥,这就是金水桥。金水桥两旁有一对汉白玉华表,上面雕刻着蟠龙花纹,在蓝天白云的映衬下,显得格外挺拔。
    
    登上天安门城楼向南眺望,整个广场尽收眼底。广场中央矗立着高大的人民英雄纪念碑,碑身正面是毛泽东主席题写的“人民英雄永垂不朽”八个金光闪闪的大字。广场南端是毛主席纪念堂。东西两侧,中国国家博物馆与巍峨壮丽的人民大会堂遥遥相对。
    
    天安门是新中国的象征。1949年10月1日,北京30万人在天安门广场举行了开国大典。人民领袖毛泽东在天安门城楼上向全世界庄严宣告:中华人民共和国中央人民政府成立了!从此,天安门广场成了全国各族人民无比向往的地方。
    
    清晨,东方露出了淡淡的曙光,天安门城楼在晨曦中显现出它的雄姿。庄严的升旗仪式就在这时开始,五星红旗与旭日一同升起。
    
    每当节日到来,天安门广场更是花团锦簇,姹紫嫣红。无数盆鲜花组成一个个大花坛,把广场装点得犹如五彩缤纷的大花园。入夜,华灯齐放,礼花飞舞,天安门广场一片辉煌。来自祖国各地的人们翩翩起舞,纵情歌唱。鲜花与彩灯辉映,礼花伴歌声齐飞,天安门广场沸腾起来了。
    
\end{large}



\chapter{火烧云}

\begin{large}
    
    晚饭过后,火烧云上来了,霞光照得小孩子的脸红红的。大白狗变成红的了。红公鸡变成金的了。黑母鸡变成紫檀色的了。喂猪的老头儿在墙根靠着,笑盈盈地看着他的两头小白猪变成小金猪了。他刚想说:“你们也变了……”旁边走来个乘凉的人对他说:“您老人家必要高寿,您老是金胡子了。”
    
    天上的云从西边一直烧到东边,红彤彤的,好像是天空着了火。
    
    这地方的火烧云变化极多,一会儿红彤彤的,一会儿金灿灿的,一会儿半紫半黄,一会儿半灰半百合色。葡萄灰、梨黄、茄子紫,这些颜色天空都有。还有些说也说不出来、见也没见过的颜色。
    
    一会儿,天空出现一匹马,马头向南,马尾向西。马是跪着的,像等人骑上它的背,它才站起来似的。过了两三秒钟,那匹马大起来了,腿伸开了,脖子也长了,尾巴可不见了。看的人正在寻找马尾巴,那匹马变模糊了。
    
    忽然又来了一条大狗。那条狗十分凶猛,在向前跑,后边似乎还跟着好几条小狗。跑着跑着,小狗不知跑到哪里去了,大狗也不见了。
    
    接着又来了一头大狮子,跟庙门前的石头狮子一模一样,也那么大,也那样蹲着,很威武很镇静地蹲着。可是一转眼就变了。再也找不着了。
    
    一时恍恍惚惚的,天空里又像这个又像那个,其实什么也不像,什么也看不清了,必须低下头,揉一揉眼睛,沉静一会儿再看。可是天空偏偏不等待那些爱好它的孩子。一会儿工夫,火烧云下去了。
    
\end{large}



\chapter{卢沟桥的狮子}

\begin{large}
    
    北京有句歇后语:“卢沟桥的狮子——数不清。”这座狮子多得数不清的桥,建于公元1189年。它是一座联拱石桥,总长约266米,有281根望柱,每根柱子上都雕着狮子。要不仔细数,真是数不清呢!
    
    这些狮子真有意思。它们有大有小。大的有几十厘米高,小的只有几厘米,甚至连鼻子眼睛都看不清。它们的形状各不相同:有的蹲坐在石柱上,好像朝着远方长吼;有的低着头,好像专心听桥下的流水声;有的小狮子偎依在母狮子的怀里,好像正在熟睡;有的小狮子藏在大狮子的身后,好像在做有趣的游戏;还有的小狮子大概太淘气了,被大狮子用爪子按在地上……
    
    卢沟桥的狮子大小不一,形态各异,真是很难数清楚。但是文物工作者已经数清了,总共是501只。
    
    卢沟桥是七七事变的发生地。1937年7月7日,侵华日军向中国军队挑衅,驻扎宛平的中国军队奋起反击,抗日战争全面爆发。正是由于这件事,卢沟桥成了我国人民永远难忘的一处具有历史意义的建筑。
    
\end{large}



\chapter{海上日出}

\begin{large}
    
    为了看日出,我常常早起。那时天还没有大亮,周围非常清静,船上只有机器的响声。
    
    天空还是一片浅蓝,颜色很浅。转眼间天边出现了一道红霞,慢慢地在扩大它的范围,加强它的亮光。我知道太阳要从天边升起来了,便不转睛地望着那里。
    
    果然,过了一会儿,在那个地方出现了太阳的小半边脸,红是真红,却没有亮光。太阳好像负着重荷似的,一步一步,慢慢地努力上升,到了最后,终于冲破了云霞,完全跳出了海面,颜色红得非常可爱。一刹那间,这个深红的圆东西,忽然发出了夺目的亮光,射得人眼睛发痛,它旁边的云片也突然有了光彩。
    
    有时太阳走进了云堆中,它的光线却从云里射下来,直射到水面上。这时候要分辨出哪里是水,哪里是天,倒也不容易,因为只能看见一片灿烂的亮光。
    
    有时天边有乌云,而且很厚,太阳出来了,但藏在云里,还看不见。然而,太阳的光芒,透过重围,替乌云镶了一道发光的金边。后来,太阳才慢慢地冲出重围,出现在天空,甚至把乌云也染成了紫色或者红色。这时候发亮的不仅是太阳、云和海水,连我自己也成了光亮的了。
    
    这不是很伟大的奇观么?
    
\end{large}



\chapter{董存瑞舍身炸碉堡}

\begin{large}
    
    1948年5月25日,攻城的大炮震撼着整个隆化城,解放隆化的战斗打响了。战士们像潮水一般冲向敌军司令部所在地———隆化中学。嗒嗒嗒……从一座桥上,突然喷出六条火舌,封锁了我军前进的道路。那座桥架在隆化中学墙外的一条旱河上。狡猾的敌人在桥的两侧筑了墙,顶上加了盖,构成了一座碉堡。冲锋的部队被压在一个小土坡下面,抬不起头。冲在最前面的一个战士倒下了。
    
    董存瑞瞪着敌人的碉堡,两眼迸射出仇恨的火花。他跑到连长身边,坚决地说:“连长,我去炸掉它!”
    
    “连长,我掩护!”战友郅顺义也恳切地说。
    
    连长和指导员商量了一下,同意了他们的请求。
    
    “同志们,扔手榴弹!”连长给战士们下了命令。
    
    董存瑞抱起炸药包,郅顺义背起两兜手榴弹,同时跃出战壕,冲了上去。他们互相配合,郅顺义扔一阵手榴弹,董存瑞就向前跃进几步;郅顺义再扔一阵,董存瑞再跃进几步。跟在后面的战友把一捆捆手榴弹送到郅顺义手里。
    
    敌人的机枪更疯狂了,子弹扑哧扑哧打在董存瑞身边,地上冒起了点点尘土和白烟。董存瑞夹紧炸药包一会儿忽左忽右地匍匐前进,一会儿又向前滚了好几米。突然,他身子一震,左腿中了一枪。他用手一摸,全是血。敌人的机枪一齐向董存瑞扫射,在他面前交织成一道火网。董存瑞离碉堡只有几十米了。他隐蔽在一小块凹地里。郅顺义接二连三地扔手榴弹掩护。董存瑞趁着腾起的黑烟,猛冲到桥下。
    
    董存瑞看看四周,这座桥有一人多高,两边是光滑的斜坡。炸药包放在哪儿呢?他想把炸药包放到河沿上,试了两次,都滑了下来。要是把炸药包放在河床上,又炸不毁碉堡。就在这时候,嘹亮的冲锋号响了,惊天动地的喊杀声由远而近。在这万分紧急的关头,董存瑞昂首挺胸,站在桥底中央,左手托起炸药包,顶住桥底,右手猛地一拉导火索。导火索“哧哧”地冒着白烟,闪着火花。火光照亮了他那钢铸一般的脸。一秒钟、两秒钟……他像巨人一样挺立着,两眼放射着坚毅的光芒。他抬头眺望远方,用尽力气高喊着:“同志们,为了新中国,冲啊!”
    
    巨大的喊声震得地动山摇,前进的道路炸开了。战士们冲过烟雾,沿着董存瑞开辟的道路杀向敌军司令部,消灭了全部敌人。胜利的红旗在隆化中学上空迎风飘扬。
    
\end{large}



\chapter{十里长街送总理}

\begin{large}
    
    天灰蒙蒙的,又阴又冷。长安街两旁的人行道上挤满了男女老少。路那样长,人那样多,向东望不见头,向西望不见尾。人们臂上都缠着黑纱,胸前都佩着白花,眼睛都望着周总理的灵车将要开来的方向。一位满头银发的老奶奶拄着拐杖,背靠着一棵洋槐树,焦急而又耐心地等待着。一对青年夫妇,丈夫抱着小女儿,妻子领着六七岁的儿子,他们挤下了人行道,探着身子张望。一群泪痕满面的红领巾,相互扶着肩,踮着脚望着,望着……
    
    夜幕开始降下来。几辆前导车过去以后,总理的灵车缓缓地开来了。灵车四周挂着黑色和黄色的挽幛,上面装饰着白花,庄严,肃穆。人们心情沉痛,目光随着灵车移动。好像有谁在无声地指挥。老人、青年、小孩,都不约而同地站直了身体,摘下帽子,静静地望着灵车,哭泣着,顾不得擦去腮边的泪水。
    
    就在这十里长街上,我们的周总理迎送过多少位来自五洲四海的国际友人,陪着毛主席检阅过多少次人民群众。人们常常幸福地看到周总理,看到他矫健的身躯,慈祥的面庞。然而今天,他静静地躺在灵车里,渐渐远去,和我们永别了!
    
    灵车缓缓地前进,牵动着千万人的心。许多人在人行道上追着灵车奔跑。人们多么希望车子能停下来,希望时间能停下来!可是灵车渐渐地远去了,最后消失在苍茫的夜色中了。人们还是面向灵车开去的方向,静静地站着,站着,好像在等待周总理回来。
    
\end{large}



\chapter{狐狸和山羊}

\begin{large}
    
    狐狸掉进了一口井里。虽然井不算深,可狐狸腿短,出不来。一头山羊来到井边,问狐狸井里有没有水。
    
    狐狸在井里喊道:“当然!井水清凉又甘甜,好喝极了,足够我们喝个饱。您也下来一起喝吧!”
    
    山羊很高兴,跳到井里,低下头来找水喝。狐狸就跳上山羊的背,踩着山羊的角,跳出了井。
    
    山羊发现被骗了,急忙让狐狸帮自己从井里出来。狐狸笑着说:“您的智慧要是有您的胡须那样多,跳进去之前就该想好怎么出来!”说完就跑了。
    
\end{large}



\chapter{燕子}

\begin{large}
    
    一身乌黑光亮的羽毛,一对俊俏轻快的翅膀,加上剪刀似的尾巴,凑成了活泼机灵的小燕子。
    
    才下过几阵蒙蒙的细雨。微风吹拂着千万条才展开带黄色的嫩叶的柳丝。青的草,绿的叶,各色鲜艳的花,都像赶集似的聚拢过来,形成了光彩夺目的春天。小燕子从南方赶来,为春光增添了许多生机。
    
    在微风中,在阳光中,燕子斜着身子在天空中掠过,唧唧地叫着,有的由这边的稻田上,一转眼飞到了那边的柳树下边;有的横掠过湖面,尾尖偶尔沾了一下水面,就看到波纹一圈一圈地荡漾开去。
    
    几对燕子飞倦了,落在电线上。蓝蓝的天空,电杆之间连着几痕细线,多么像五线谱啊,停着的燕子成了音符,谱出一支正待演奏的春天的赞歌。
    
\end{large}


\newpage

\textbf{注释}:

\vspace{-1em}

\begin{itemize}
    \setlength\itemsep{-0.2em}
    \item 〔俊俏〕俊美秀丽。
    \item 〔蒙蒙〕形容雨水细密,看起来灰暗不清。
    \item 〔拂〕轻扫。
    \item 〔掠过〕从上方飞过。
    \item 〔赶集〕地广人少的乡村,约定时间地点贸易,叫做集市。到时买卖的人从四方赶来,叫做赶集。
    \item 〔五线谱〕用五条横线之间的圆点标记音乐的方式。谱:记录音乐、棋局的图形。
    \item 〔荡漾〕水波起伏摇动。
    \item 〔倦〕累,失去兴致,不想继续做。
    \item 〔痕〕留下的印子。
\end{itemize}

\chapter{晏子使楚}

\begin{large}
    
    齐国和楚国都是大国。
    
    有一回,齐王派大夫晏子去访问楚国。楚王仗着自己国势强盛,想乘机侮辱晏子,显显楚国的威风。
    
    楚王知道晏子身材矮小,就叫人在城门旁边开了一个五尺来高的洞。晏子来到楚国,楚王叫人把城门关了,让晏子从这个洞进去。晏子看了看,对接待的人说:“这是个狗洞,不是城门。只有访问‘狗国’,才从狗洞进去。我在这儿等一会儿。你们先去问个明白,楚国到底是个什么样的国家?”接待的人立刻把晏子的话传给了楚王。楚王只好吩咐大开城门,迎接晏子。
    
    晏子见了楚王。楚王瞅了他一眼,冷笑一声,说:“难道齐国没有人了吗?”晏子严肃地回答:“这是什么话?我国首都临淄住满了人。大伙儿把袖子举起来,就是一片云;大伙儿甩一把汗,就是一阵雨;街上的行人肩膀擦着肩膀,脚尖碰着脚跟。大王怎么说齐国没有人呢?”楚王说:“既然有这么多人,为什么打发你来呢?”晏子装着很为难的样子,说:“您这一问,我实在不好回答。撒谎吧,怕犯了欺骗大王的罪;说实话吧,又怕大王生气。”楚王说:“实话实说,我不生气。”晏子拱了拱手,说:“敝国有个规矩:访问上等的国家,就派上等人去;访问下等的国家,就派下等人去。我最不中用,所以派到这儿来了。”说着他故意笑了笑,楚王只好陪着笑。
    
    楚王安排酒席招待晏子。正当他们吃得高兴的时候,有两个武士押着一个囚犯,从堂下走过。楚王看见了,问他们:“那个囚犯犯的什么罪?他是哪里人?”武士回答说:“犯了盗窃罪,是齐国人。”楚王笑嘻嘻地对晏子说:“齐国人怎么这样没出息,干这种事?”楚国的大臣们听了,都得意扬扬地笑起来,以为这一下可让晏子丢尽了脸了。哪知晏子面不改色,站起来,说:“大王怎么不知道哇?淮南的柑橘,又大又甜。可是橘树一种到淮北,就只能结又小又苦的枳,还不是因为水土不同吗?同样道理,齐国人在齐国安居乐业,好好地劳动,一到楚国,就做起盗贼来了,也许是两国的水土不同吧。”楚王听了,只好赔不是,说:“我原来想取笑大夫,没想到反让大夫取笑了。”
    
    从这以后,楚王不敢不尊重晏子了。
    
\end{large}



\chapter{狼牙山五壮士}

\begin{large}
    
    1941年秋,日寇集中兵力,向我晋察冀根据地大举进犯。当时,七连奉命在狼牙山一带坚持游击战争。经过一个多月英勇奋战,七连决定向龙王庙转移,把掩护群众和连队转移的任务交给了六班。
    
    为了拖住敌人,七连六班的五个战士一边痛击追上来的敌人,一边有计划地把大批敌人引上了狼牙山。他们利用险要的地形,把冲上来的敌人一次又一次地打了下去。班长马宝玉沉着地指挥战斗,让敌人走近了,才命令狠狠地打。副班长葛振林打一枪就大吼一声,好像那个细小的枪口喷不完他的满腔怒火。战士宋学义扔手榴弹总要把胳膊抡个一圈,好使出浑身的力气。胡德林和胡福才这两个小战士把脸绷得紧紧的,全神贯注地瞄准敌人射击。战斗进行了很久,敌人始终不能前进一步。在崎岖的山路上,横七竖八地躺着许多敌人的尸体。
    
    五位战士胜利地完成了掩护任务,准备转移。面前有两条路:一条通往主力转移的方向,走这条路可以很快追上连队,可是敌人紧跟在身后;另一条通向狼牙山的顶峰的棋盘陀,那里三面都是悬崖绝壁。走哪条路呢?为了不让敌人发现人民群众和连队主力,班长马宝玉斩钉截铁地说了一声:“走!”他带头向棋盘陀走去。战士们热血沸腾,紧跟在班长后面。他们知道班长要把敌人引上绝路。
    
    五位壮士一面向顶峰攀登,一面依托大树和岩石向敌人射击。山路上又留下了许多具敌人的尸体。到了狼牙山峰顶五位壮士居高临下,继续向紧跟在身后的敌人射击。不少敌人坠落山涧,粉身碎骨。班长马宝玉负伤了,子弹都打完了,只有胡福才手里还剩下一颗手榴弹。他刚要拧开盖子,马宝玉抢前一步,夺过手榴弹插在腰间,猛地举起一块大石头,大声喊道:“同志们!用石头砸!”顿时,石头像雹子一样,带着五位壮士的决心,带着中国人民的仇恨,向敌人头上砸去。山坡上传来一阵叽里呱啦的叫声,敌人纷纷滚落深谷。
    
    又一群敌人扑上来了。马宝玉嗖的一声拔出手榴弹,拧开盖子,用尽全身气力扔向敌人。随着一声巨响,手榴弹在敌群中开了花。
    
    五位壮士屹立在狼牙山顶峰,眺望着群众和部队主力远去的方向。他们回头望望还在向上爬的敌人,脸上露出了胜利的喜悦。班长马宝玉激动的说:“同志们,我们的任务胜利完成了!”说罢,他把那支从敌人手里夺来的枪砸碎了,然后走到悬崖边上,像每次发起冲锋一样,第一个纵身跳下深谷,战士们也昂首挺胸,相继从悬崖往下跳。狼牙山上响起了他们壮烈豪迈的口号声:
    
    “打倒日本帝国主义!”
    
    “中国共产党万岁!”
    
    这是英雄的中国人民坚强不屈的声音!这声音惊天动地,气壮山河!
    
\end{large}



\chapter{我的战友邱少云}

\begin{large}
    
    敌人控制的“391”高地,像一颗毒牙,揳入我们志愿军的阵地。我们准备在黄昏时分发动突然袭击,拔掉这颗毒牙,把战线往南推移。
    
    那一天,天还没有亮,我们悄悄摸进“391”高地下面的山坳,潜伏在一条比较隐蔽的山沟里。太阳渐渐爬上山头。我发现前面六十多米的地方就是敌人的前沿阵地,不但可以看见铁丝网和胸墙,还可以看见地堡和火力点,甚至连敌人说话都听得见。敌人居高临下,当然很容易发现我们。我们趴在地上必须纹丝不动,咳嗽一声或者蜷一下腿,都可能被敌人发觉。我看了一下前面,班长和几个战士伏在枯黄的茅草丛里。他们身上披着厚厚的茅草作伪装,猛一看去,很难发现他们。我又看了看伏在我身边不远的邱少云。他也全身伪装,隐蔽得更好,相隔这么近,我几乎找不到他。
    
    我们的炮兵不断地轰击敌人的阵地,山顶上腾起一团一团的青烟。敌人前沿的地堡一个接一个被掀翻了。看着这种情景,我只盼望天快点黑,好痛痛快快地打一仗。
    
    到了中午,敌人突然打起炮来,炮弹一排又一排,在我们附近爆炸。显然,敌人已经感觉到他们的前沿阵地不太安全了,可是没有胆量冒着我军的炮火出来搜查,只好把看家的本领“火力警戒”拿出来了。
    
    排炮过后,敌人竟使用了燃烧弹,我们附近的荒草着火了。火苗子呼呼地蔓延,烧枯黄的茅草毕毕剥剥的响。我忽然闻到一股浓重的棉布焦味,扭转头一看,哎呀!火烧到邱少云身上了!他的棉衣已经烧着,火苗趁风势乱窜,一团烈火把他整个身子包住了。
    
    这时候,邱少云只要从火里跳出来,就地打几个滚,就可以把身上的火扑灭。我趴在他的附近,只要跳过去扯掉他的棉衣,也能救出自己的战友。但是这样一来,我们就会被山头的敌人发现,我们整个班,我们身后的整个潜伏部队,都会受到重大的损失,这一次的作战计划就全部落空了。
    
    我的心绷的紧紧的。这怎么忍受得呢?我担心这个年轻的战士会突然跳起来,或者突然叫起来。我不敢朝他那儿看,不忍眼巴巴地看着我的战友被活活烧死。但是我忍不住看,我盼望出现什么奇迹——火突然间熄灭。我的心像刀绞一般,泪水模糊了我的眼睛。
    
    为了整个班,为了整个潜伏部队,为了这次战斗的胜利,邱少云像千斤巨石一般,趴在火堆里一动也不动。烈火在他身上烧了半个多钟头才渐渐地熄灭。这位伟大的战士,直到最后一息,也没动一寸地方,没发出一声呻吟。
    
    黄昏时候,漫山遍野响起了激动人心的口号:“为邱少云同志报仇!”我们怀着满腔怒火,勇猛地冲上“391”高地。敌人全部被我们歼灭了。看看时间,从发起冲锋到战斗结束,才二十分钟。
    
    我永远忘不了那一天——1952年10月12日。
    
\end{large}



\chapter{草原}

\begin{large}
    
    这次,我看到了草原。那里的天比别处的更可爱,空气是那么清鲜,天空是那么明朗,使我总想高歌一曲,表示我满心的愉快。在天底下,一碧千里,而并不茫茫。四面都有小丘,平地是绿的,小丘也是绿的。羊群一会儿上了小丘,一会儿又下来,走在哪里都像给无边的绿毯绣上了白色的大花。那些小丘的线条是那么柔美,就像只用绿色渲染,不用墨线勾勒的中国画那样,到处翠色流淌,轻轻沁入云际。这种境界,既使人惊叹,又叫人舒服,既愿久立四望,又想坐下低吟一首奇丽的小诗。在这境界里,连骏马和大牛都有时候静立不动,好像回味着草原的无限乐趣。
    
    我们访问的是陈巴尔虎旗\footnote{〔陈巴尔虎旗〕内蒙古东北部呼伦贝尔市下的一个旗。旗:内蒙古的县级行政区。}。汽车走了一百五十里,才到达目的地。一百五十里全是草原。再走一百五十里,也还是草原。草原上行车十分洒脱,只要方向不错,怎么走都可以。初入草原,听不见一点儿声音,也看不见什么东西,除了一些忽飞忽落的小鸟。走了许久,远远地望见了一条迂回的明如玻璃的带子——河!牛羊多起来,也看到了马群,隐隐有鞭子的轻响。快了,快到了。忽然,像被一阵风吹来似的,远处的小丘上出现了一群马,马上的男女老少穿着各色的衣裳,群马疾驰,襟飘带舞,像一条彩虹向我们飞过来。这是主人来到几十里外欢迎远客。见到我们,主人们立刻拨转马头,欢呼着,飞驰着,在汽车左右与前面引路。静寂的草原热闹起来:欢呼声,车声,马蹄声,响成一片。车跟着马飞过小丘,看见了几座蒙古包。
    
    蒙古包外,许多匹马,许多辆车。人很多,都是从几十里外乘马或坐车来看我们的。主人们下了马,我们下了车。也不知道是谁的手,总是热乎乎地握着,握住不放。大家的语言不同,心可是一样。你说你的,我说我的,总的意思是民族团结互助。
    
    也不知怎的,就进了蒙古包\footnote{〔蒙古包〕中亚牧民居住的圆形帐篷,用木架和毛毡搭建。}。奶茶倒上了,奶豆腐\footnote{〔奶豆腐〕蒙古族牧民家中常见的奶制食品,形如豆腐。}摆上了,主客都盘腿坐下,谁都有礼貌,谁都又那么亲热,一点儿不拘束。不大一会儿,好客的主人端进来大盘的手抓羊肉。干部向我们敬酒,七十岁的老翁向我们敬酒。我们回敬,主人再举杯,我们再回敬。这时候,鄂温克族\footnote{〔鄂温克族〕东北亚的世居民族,主要居住于俄罗斯西伯利亚以及中国内蒙古、黑龙江。}姑娘们戴着尖尖的帽子,既大方,又稍有点儿羞涩,来给客人们唱民歌。我们同行的歌手也赶紧唱起来,歌声似乎比什么语言都更响亮,都更感人,不管唱的是什么,听者总会露出会心的微笑。
    
    饭后,小伙子们表演套马、摔跤,姑娘们表演了民族舞蹈。客人们也舞的舞,唱的唱,还要骑一骑蒙古马。太阳已经偏西,谁也不肯走。是啊!蒙汉情深何忍别,天涯碧草话斜阳!
    
\end{large}


\newpage

\textbf{注释}:

\vspace{-1em}

\begin{itemize}
    \setlength\itemsep{-0.2em}
    \item 〔茫茫〕广阔而看不清楚。
    \item 〔渲染〕中国画里,用水墨淡色涂抹画面,大块着色的方法。
    \item 〔勾勒〕用线条描画出物体的边线。
    \item 〔沁〕像水一样渐渐渗、透到某个地方或某种东西里。
    \item 〔洒脱〕自由自在地做事,不担心有约束。
    \item 〔拘束〕不自在,不敢敞开心怀。
\end{itemize}

\chapter{马踏飞燕}

\begin{large}
    
    “马踏飞燕”是东汉时期的艺术珍品\footnote{〔马踏飞燕〕东汉时期的青铜器,1969年10月出土于甘肃省武威市雷台汉墓,现藏于甘肃省博物馆,是我国国宝。}。它1969年出土以后,很快名闻天下,受到人们的赞美。
    
    看,这匹铜制的骏马,膘肥身健,体形匀称,鬃毛整齐,四蹄坚韧有力。它头微微后仰而稍向左歪,尾巴向后扬起。它张开大口,人们仿佛听到了它高亢的嘶鸣声。
    
    这是一匹奔跑中的骏马,怎样表现它的速度之快呢?艺术家构思奇妙,让马的右前腿大步前跨,左后腿向后平伸,以表现它正在快速奔跑。快到什么程度呢?艺术家匠心独运,让马的右后蹄踏在一只飞燕上。这样就把“快”化为具象了:连飞燕都来不及躲闪,真跑得快啊!马蹄踏在飞燕上,飞燕竟安然无恙,可见这匹马几乎是四蹄离地,风驰电掣般地飞奔。
    
    铜奔马全身的重量都靠踏燕的右后蹄支撑,怎样才能保持平衡?这是一个难题。艺术家有意使马的头和颈往后收缩,让重心\footnote{〔重心〕物体各部分重量的中心,相当于物体的重量集中在这一点。}尽量后移;同时使右后蹄尽量前伸,让支撑点和重心正好在一条竖直线上。还有那向前后伸出的两条腿和扬起的尾巴,不仅使马在整体上保持平衡,而且使马的造型更加优美。
    
    从这匹踏燕的奔马身上,我们看到了古代劳动人民具有丰富的艺术想象力和先进的科学知识,看到了他们卓越的创造才能。
    
\end{large}


\newpage

\textbf{注释}:

\vspace{-1em}

\begin{itemize}
    \setlength\itemsep{-0.2em}
    \item 〔鬃毛〕马、猪颈背上的硬长毛。
    \item 〔具象〕具有形象的,可观可闻可感受的。
    \item 〔匠心独运〕运用了独特巧妙的心思。
    \item 〔安然无恙〕原本指平安没有疾病或忧患。现泛指平安无事,没有遭受损害。
    \item 〔风驰电掣〕像刮风、闪电一样。形容非常快。
\end{itemize}

\chapter{牛郎织女的故事}

\begin{large}
    
    晚上,天气好的时候,能看到夜空中有一条暗淡的银河,那就是传说中的天河。天河的东边,有位织衣的女子,她是天帝的女儿,人们叫她织女。
    
    织女年年在织机前劳作,织出云霞一般美丽的衣裳,忙得连整理妆容的时间都没有。天帝可怜她孤单一人,就将她许给了天河西边的牵牛郎。
    
    织女嫁给了牵牛郎,生下一双儿女,一家人幸福美满。牛郎每天出门耕作,织女在家照顾孩子,织的衣裳渐渐就少了。天帝知道了,大发雷霆,命令织女回到河东,只许她每年与牛郎相见一次。
    
    每年的七月初七,喜鹊头顶就会掉毛。这是因为孩子想娘亲了,牛郎就牵着儿女到天河边上,叫喜鹊搭成桥梁。织女踩着鹊桥过河见孩子,把喜鹊头顶的羽毛弄掉了。
    
\end{large}



\chapter{搭船的鸟}

\begin{large}
    
    我和母亲坐着小船,到乡下外祖父\footnote{〔外祖父〕母亲的父亲。}家里去。天下着大雨,雨点打在船篷上,沙啦沙啦地响。我们坐在船舱里。船夫披着蓑衣在船后用力地摇着橹\footnote{〔橹〕行船的工具。中部定在船尾,摇动上柄,下片拨水使船行进。}。
    
    雨停了,我看见一只彩色的小鸟立在船头。多么美丽啊!它的羽毛是翠绿的,翅膀带有一抹蓝色,还有一张红色的长嘴,比鹦鹉还漂亮。
    
    它是什么时候飞来的呢?它静悄悄地停在船头,不知有多久了。它站在那里做什么呢?难道它要和我们一起坐船到外祖父家去吗?
    
    我正想着,它一下子冲进水里,不见了。没一会儿,它又飞出水来,红色的长嘴衔\footnote{〔衔〕咬住,叼。}着一条小鱼。它立上船头,一口把小鱼吞了下去。
    
    母亲告诉我,这是一只翠鸟。哦,这只翠鸟搭了我们的船,在捕鱼吃呢。
    
\end{large}


\newpage

\textbf{注释}:

\vspace{-1em}

\begin{itemize}
    \setlength\itemsep{-0.2em}
    \item 〔船舱〕船内载客、载货的空间。
    \item 〔船篷〕盖在小船顶上,用来遮挡阳光、风雨的东西。
    \item 〔蓑衣〕用蓑草做的雨衣。
\end{itemize}

\chapter{狐假虎威}

\begin{large}
    
    老虎寻找各种动物来吃。他捉到一只狐狸。狐狸对老虎说:“你不该吃我。天帝派我做百兽的首领,如果你吃掉我,就违背了天帝的命令。”
    
    老虎不信,说:“你这么弱小,天帝怎么会叫你做百兽的首领?”
    
    狐狸说:“百兽都怕我。你如果不相信我说的话,我在前面走,你跟在我的后面,看看大家见了我,有哪一个敢不逃跑?”
    
    老虎就和狐狸一起走。狐狸走在前面,老虎跟在后面。各种动物见了老虎,很害怕,纷纷逃跑。老虎也吓了一跳,对狐狸说:“原来你说的是真的,看来我不应该吃你。”于是让狐狸走了。
    
\end{large}



\chapter{塞翁失马}

\begin{large}
    
    从前,在边塞地方,有一个很会推算的老翁。一天,他的儿子发现,家里的一匹马走丢了,跑到塞外胡人的地方去了。大伙儿知道了,都来安慰他。老翁却并不在乎,说:“丢了马确实是件坏事,但怎么就不能是一件好事呢?”
    
    过了几个月,那匹马带着几匹好马,从胡人的地方回来了。大家都来祝贺老翁。老翁又说:“有了好马确实是好事,但怎么就不能是一件坏事呢?”
    
    老翁家里的好马多起来了。他的儿子喜欢骑马,结果从马上掉下来,摔断了大腿,干不了活了。大伙儿听说了,都来安慰老翁。老翁说:“摔断了腿,确实是件坏事。但怎么就不能是一件好事呢?”
    
    过了一年,胡人大举攻破边塞,侵入汉地。官府征召所有壮丁入伍,拿起弓箭去作战。边塞地方的人,十个里死了九个。老翁的儿子因为腿断了,无法去作战,留在家里和父亲一起守家,反而活下来了。
    
\end{large}



\chapter{买椟还珠}

\begin{large}
    
    古时候楚国有一个商人,想要到郑国卖掉一颗夜明珠。他怕人不知道这珠子有多珍贵,就做了一个漂亮的匣子来装这珠子。
    
    匣子是用上好的兰木做的,用月桂、山椒等香料熏过,匣子上用珍珠和玉石装饰,有火红的琉璃珠,有鲜绿的翡翠玉。
    
    到了郑国的集市上,一个有钱的人看到了这个匣子。他说:“这匣子实在太美了!阳光照在翡翠和琉璃上,闪闪发光!我想要这个匣子!”
    
    楚国商人说:“我是来卖夜明珠的。您看!”说着打开匣子,现出里面的珠子来。
    
    有钱人说:“这珠子会不会发光,要等到晚上才知道。我看这匣子很好,我只要这个匣子。”于是他买下了这个匣子,把珠子还给了楚国商人。
    
\end{large}



\chapter{我和企鹅}

\begin{large}
    
    1986年1月8日,我们从首都北京出发,乘飞机从东半球飞到西半球,从北半球飞到南半球,1月14日才到达目的地——南极大陆。我们高兴得又蹦又跳。
    
    尽管南半球正是夏天,可是南极大陆上仍然覆盖着厚厚的白雪,仅仅在我国南极考察队建立的长城站附近,才露出一些地面。我看到长城站旁边有几只黑白相间的小动物。啊,原来是企鹅。早就听考察队的高伯伯说过,长城站对面有个企鹅岛,上面的企鹅成千上万,可壮观了。我真想早一点儿到企鹅岛上去看看!
    
    高伯伯好像看出了我的心思,当天下午就带我们登上了企鹅岛。我以前从电视里看到,企鹅差不多有小孩子那么高。可现在一看,这个岛上的成年企鹅只有一尺来高,未成年的就更小了,只有小猫那么大。
    
    岛上一共有三种企鹅。最漂亮的是金企鹅,嘴是金红色的,头部有两块白毛,又叫花脸企鹅。还有一种企鹅颈部有一圈黑毛,好像系帽子的带儿,叫帽带儿企鹅。它们彬彬有礼,站在远处向我们点头,像欢迎我们似的。最凶猛的是阿德雷鹅,我刚迈进它们的“领地”,一只企鹅就尖叫着把我驱逐“出境”了。它们的叫声很像毛驴,所以又叫驴企鹅。
    
    企鹅的毛不同于别的鸟,小企鹅浑身是绒毛,成年的长着鱼鳞状的毛。它们的躯体呈流线型,背部黑色,腹部白色,对比鲜明:翅膀退化成鳍状,走路来一摇一摆,十分有趣。我抱起一只全身灰色的小企鹅,跟它合影留念。高伯伯告诉我们这只毛茸茸的小企鹅是刚孵化出来的,还不能下海捕食,只能吃母企鹅嘴里的食物。果然,我看到有的小企鹅追着母企鹅,用嘴挠它的脖子。挠了几次以后,母企鹅只得张开嘴,让小企鹅把嘴伸到它嘴里,吃它从胃里呕出来的食物。
    
    高伯伯还告诉我们,过几个月,小企鹅就要换毛了,换毛以后才能自己下海捕食。再过几个月,小企鹅第二次换毛,再长出来的毛就成了鱼鳞状。这时侯,它就是成年企鹅了。我仔细观察,果然看到有的小企鹅肚皮上的毛一块块脱落了。看来,那些全身光滑油亮的就是成年的企鹅了。
    
\end{large}



\chapter{白求恩大夫}

\begin{large}
    
    离前线五里地的温家屯村边,有一座小庙,庙门前面是一片空地。星光下,可以看见空地上整齐地摆着一排担架。担架上躺着刚从前线抬下来的伤员。低沉的,痛苦的叫喊声,浮荡在空地上。护理伤员的人,轻轻走在担架旁,低声安抚每个伤员,告诉他们,快轮到谁动手术了。担架一直排到庙门口,里面一个做好手术的伤员抬出来,外面就立马抬一个进去。
    
    小庙里四周都挂着白布。当中垂下来一盏煤气油灯。酒精味混杂着血腥味,从里面飘溢出来。白求恩大夫穿着白大褂,面前挂着红橡皮围裙,头上戴着一盏小电灯,身上背着电池,紧张地动着手术。
    
    猛地传来轰的一声,一颗炮弹落在手术室的后面,爆炸开来,震得地面都动了。庙顶上的瓦片咯咯地响。呯——有一片落在地上,碎了。
    
    童翻译来到白求恩大夫身边,说:“白大夫,这边炮火很激烈,最好移动一下。”
    
    白求恩大夫连考虑也没考虑,摇摇头说:“打仗就是这样。前面有队伍,不要紧。这不算什么。我在西班牙的时候,比这更厉害,飞机大炮更多呢。”
    
    “可不可以转移一下?后面五里地就有个村子,里面也能做手术。”野战师卫生部的金部长,这个小矮个子,提出这样的意见。
    
    “军医离火线越近越好。往后移五里地,伤员送来的时间就长了。伤员越早救护越好。早一个小时救护,能救活,晚一个小时,就会死的。”
    
    “这儿太危险了,白大夫。我看还是……”
    
    “危险?”白求恩大夫抬起头来,望了金部长一眼,仿佛他说了什么奇怪的话,“火线下才危险呢,随时都要死人。战士们在火线下都不怕,我怕什么?”。金部长碰了个钉子,没话好说了。可他是师卫生部的负责人,要是白大夫出了什么事,他怎么向师里解释呢?金部长急了。想来想去,最后他给自己想了个法子:时时刻刻都紧靠在白大夫身边。这样一来,如果有什么意外,也要一起倒下。
    
    ……
    
    战斗持续了三天三夜,终于胜利结束了。白求恩大夫走出小庙,空地上已经没有担架了。童翻译正在数空地上的尸体。那是敌人的尸体。
    
    几个战士围上来。“听说你们好几天没睡觉。白大夫,你们太辛苦了。”
    
    “辛苦吗?不。你们比我辛苦多了。你们的成绩是这几百具尸体。我们的成绩是从担架转到床上,舒服地躺着的伤员。”白求恩大夫开心地笑了。
    
\end{large}


\newpage

\textbf{注释}:

\vspace{-1em}

\begin{itemize}
    \setlength\itemsep{-0.2em}
    \item 〔庙〕供奉祖先或神佛的地方。
    \item 〔褂〕罩在外面的长衣。白大褂:医生常用的外衣。
\end{itemize}

\chapter{我的弟弟“小萝卜头”}

\begin{large}
    
    1941年,我的爸爸妈妈和只有八个月的弟弟,被国民党反动派秘密逮捕了。弟弟跟着妈妈住在女牢房里。牢房阴暗潮湿,终年不见阳光。弟弟穿的是妈妈改小的囚衣,吃的和大人一样,是发霉发臭的牢饭。牢门上的小窗口,是唯一能见到天的地方,小弟弟总想凑在这里看看外面的世界。难友们见他人虽小,却十分倔强、勇敢,就给他起了个“小萝卜头”的名字。
    
    弟弟六岁了,爸爸向特务提出,应当让弟弟去上学。特务怕弟弟把监狱的内幕泄露出去,硬是不让。爸爸再三交涉,特务才答应由狱中关押的几个政治犯做弟弟的老师,教他读书。第一天上课时,罗世文伯伯教他语文,“小萝卜头”跟着罗伯伯大声地念:“我爱中国共产党!”
    
    从此以后,小弟弟每天由特务押着到老师那儿上课、下课。时间一长,特务就放松了对他的看押,上课、下课由着他自己来去。难友们就利用这个条件,让“小萝卜头”传送消息。
    
    即使在最艰苦的情况下,“小萝卜头”依然坚持学习。一年多时间,他进步很大,不但学会了乘法口诀,罗伯伯还教会他背叶挺将军的《囚歌》。
    
    一天,小弟弟走进罗伯伯的牢房,却不见了他的人影。他呆呆地站在那里,一位叔叔说:“昨天晚上,罗伯伯被特务杀害了!”弟弟流着泪,把仇恨埋在心里!
    
    罗世文牺牲后,黄显声将军成了小弟弟的老师,他除了教语文和算术,还教“小萝卜头”一些俄语。遇到特务监视时,他们就用俄语讲话,特务急得干瞪眼。
    
    在“小萝卜头”生日那天,黄伯伯把自己仅有的一支铅笔送给他。“小萝卜头”珍惜得像宝贝似的,只有在上课时才拿出来用。他懂得自己的学习机会是斗争得来的,所以学习十分刻苦认真。
    
    难友们只要听到“小萝卜头”喊:“王八又打败仗了!”就知道解放军打了胜仗,心中充满喜悦。淮海战役胜利,小弟弟高兴地把好消息从楼下传到楼上,大家都夸他是个好“通讯员”。
    
\end{large}



\chapter{帐篷}

\begin{large}
    
    \begin{verse}[0.5\linewidth]
        哪儿需要我们, \\
        就在哪儿住下, \\
        一个个帐篷, \\
        是我们流动的家;
    \end{verse}
    
    
    \begin{verse}[0.5\linewidth]
        荒原最早的住户, \\
        野地最早的人家, \\
        我们到了哪儿, \\
        哪儿就起了喧哗;
    \end{verse}
    
    
    \begin{verse}[0.5\linewidth]
        探索大地的秘密, \\
        要把宝藏开发, \\
        架大桥,修铁路, \\
        盖起高楼大厦;
    \end{verse}
    
    
    \begin{verse}[0.5\linewidth]
        换一次工地, \\
        就搬一次家, \\
        带走的是荒凉, \\
        留下的是繁华。
    \end{verse}
    
    
    \begin{verse}[0.5\linewidth]
        今天你建我的, \\
        明天我建你的。 \\
        一起走向繁荣富强, \\
        祖国是我们共同的家。
    \end{verse}
    
\end{large}



\chapter{参观人民大会堂}

\begin{large}
    
    王叔叔带我去参观人民大会堂。我高兴极了,我早就盼望着这一天。
    
    人民大会堂真是雄伟壮丽!
    
    老远,我就望见正门上那个庄严的国徽,红底镶金,闪闪发光,十二根大理石的淡青色柱子,有六七层高,要四个人才合抱得过来。
    
    进了大门就是中央大厅,大厅的天花板上挂着五盏水晶玻璃大吊灯;大理石的地面晶光闪耀,能映出人的倒影来。
    
    穿过大厅,走进了大礼堂。啊!礼堂真是大极了,楼上楼下这么多座位!我正想数一数这样大的礼堂到底能坐多少人,王叔叔说:“不用数啦,三层一共有九千六百多个座位,加上主席台上的三百多个,全场能坐一万人。”
    
    王叔叔叫我看顶上。我抬头一看,只见天花板上有无数盏电灯,像满天繁星,中央是一盏红五角星大灯,放出灿烂的光芒。王叔叔告诉我,这里是全国人民代表大会开会的地方,党和国家的领导人就在这里和各族人民代表商量国家大事。
    
    从大礼堂出来,我们又上楼,参观北面的宴会厅。我简直不知往哪儿看才好,柱子上,天花板上,都描着金花彩画,看得我眼花缭乱。在这宽敞的大厅里,整整齐齐地放着一排一排的大圆桌和皮椅子。听王叔叔说,这里可以举行五千人的大宴会。
    
    我们又参观了几个会议厅。每个厅都很宽敞,装饰和摆设各不相同,王叔叔说:“这种会议厅是用包括台湾省在内的省、市自治区的名称命名的。各个厅都有自己的地方特点。”王叔叔还告诉我,中央领导同志经常在这里会见外国朋友。走出大门的时候,王叔叔告诉我说:“这样壮丽的人民大会堂,不到一年工夫就建成了!”我情不自禁地说:“盖得可真快,工人叔叔真伟大!”
    
\end{large}



\chapter{海底世界}

\begin{large}
    
    当你站在海边,望着浩瀚无边的大海,你可知道,大海的深处到底是怎样的呢?
    
    海底是一个幽静的世界。海面上波涛澎湃的时候,海底也是平静的。海面上最大的风浪也只能影响到几十米深,更深处的海水,仍然按海底的规律徐徐流动。海面上阳光灿烂的时候,海底也是幽暗的。阳光照入海水,水越深,光线就越弱,从水下五百米开始,大海就被黑暗统治了。可是在一片漆黑中,还有点点光芒闪烁。那是具有发光器官的深水鱼。它们的身体能分泌发光的物质,是海底世界唯一的光明使者。
    
    海底是否静得一点儿声音也没有呢?不是的。深海居民们经常切切私语,你用水中听音器就可以听见:那像蜜蜂嗡嗡的声音是小鲇鱼;那像小鸟叽叽的声音是大群青鱼;有的鱼发出的声音像打鼾;有的鱼竟会模仿狗叫……他们吃食的时候发出一种声音,游动的时候又是另一种,遇到危险还会发出警报。
    
    深海的动物,已知的就有数万种,海洋学家估计还有数千万种尚未发现。他们游动的方式形形色色,非常有趣。“海底蜗牛”——海参依靠肌肉收缩爬行,每小时只前进四米。身体像梭子一样的旗魚,在攻击其它动物的时候,每小时可游九十公里,比狮子还快。乌贼和章鱼可以把水从身体中迅速地喷出来,利用反推力前进。还有些贝类更妙,他们吸附在轮船底下,作免费长途旅行。
    
    海洋的平均深度是四千米。海底大体上是平坦的,但也有峡谷和深渊,其中最深的是马利亚纳海沟,深达一万一千米。海底的山脉也不少,在太平洋甚至还有海底火山。海底的森林和草地比陆地上的更绚烂多彩,有褐色的,有淡紫色的,还有红色的。最小的连肉眼都看不见,而有些大海藻却长达两、三百米,是地球上最长的生物。
    
    海底蕴藏着大量矿产资源。光是太平洋海底,就有至少万亿吨的金属矿,是陆地上的数十倍。此外,海底还有大量的煤、石油、天然气等等。汪洋大海真是地球上矿藏最丰富的后备军!但是,如何开采海底资源,又不破坏海底的生态环境,也是一个难题。二十世纪以来,人类的活动已经对海底世界产生了很大影响。惊动海底世界,又会对人类产生什么样的影响呢?这是需要我们认真研究的课题。
    
\end{large}



\chapter{故乡的杨梅}

\begin{large}
    
    我的故乡在江南。江南有一种水果叫杨梅。
    
    杨梅在初夏成熟。端午节过后,杨梅树上就挂满了杨梅。
    
    杨梅圆圆的,比大拇指头更大些,遍身生着小刺。等杨梅渐渐长熟,刺也渐渐软了,平了。摘一个放进嘴里,舌头触及,反而有种蓬松的感觉。
    
    杨梅先是淡红的,随后变成深红、深紫,最后几乎变成黑的了。它不是真的变黑,因为太红了,所以像黑的。你轻轻咬开它,就可以看见新鲜红嫩的果肉,嘴唇上舌头上染满了鲜红的汁水。
    
    没有熟透的杨梅又酸又甜,熟透了就甜津津的,叫人越吃越爱吃。我小时候,有一次吃杨梅,吃得太多了,发觉牙齿又酸又软,连豆腐也咬不动了。我才知道杨梅虽然熟透了,酸味还是有的,因为它太甜,吃起来就不觉得酸了。吃饱了杨梅再吃别的东西,才感觉到牙齿被它酸倒了。
    
\end{large}



\chapter{杏儿熟了}

\begin{large}
    
    我们家院子里有一棵大杏树。每年到了麦收时节,树上就结满了黄澄澄的杏儿。从我家门前路过的人,总要望望那棵杏树,羡慕地说道:“呵,好杏儿呀!”
    
    杏树是奶奶亲手栽的。听奶奶说,栽杏树的时候还没有我呢。有一回,我偎依在奶奶的怀里问她:“奶奶,咱们家树上的杏儿有多少个?”
    
    “多得数不清啊。要不,你数数看。”
    
    我仰着头数起来,“一、二、三……”数呀,数呀,数到后来就糊涂了。奶奶忍不住笑了。我不知道奶奶是笑我傻, 还是笑杏儿多。
    
    这一年杏子又熟了。有一天,奶奶正在做饭,忽然听见有孩子在哭。奶奶急忙走出去,原来是邻居家的小淘淘偷摘杏儿,不小心从树上摔下来了。一块儿来的小伙伴见了奶奶都低下了头,不敢吱声。我没好气地说:“你们这些馋猴儿,偷吃人家的东西,摔了活该!”
    
    我心里想:“看我奶奶怎么收拾你们!”奶奶走过去扶起淘淘,给他揉揉腿,看他没伤着,就站起身往屋里走,又回过头来对孩子们说:“你们先别走。”
    
    过了一会儿,奶奶拿了一根长竹竿从屋里出来了。她走到树下,挑熟了的杏子往下打。她脚底下站不大稳,身子颤颤巍巍的。
    
    杏儿一个接一个落在地上。我连忙弯腰去捡,不一会儿就捡了一衣兜。奶奶把小淘淘和他的伙伴都叫了过来,一人分给五六个,剩下的几个给了我。看他们吃得那样香甜,奶奶的嘴角上露出了微笑。我有点儿不高兴,奶奶却笑着说:“果子大家吃才香甜。要记住,杏儿熟了,让乡亲们都尝尝鲜。''
    
    听了奶奶的话,我点了点头。以后,我每年都照奶奶的吩咐,把熟透了的杏儿分给小伙伴们吃,也送给邻居的叔叔婶子们尝鲜。
    
    今年的杏儿又熟了。望着黄澄澄的杏儿挂满了枝头,我眼前又出现了奶奶颤颤巍巍地打杏儿的情景。于是,我挑熟透了的杏儿打下一些来,给乡亲们送去——给他们送去香甜,也给他们送去喜悦。
    
\end{large}



\chapter{春蚕}

\begin{large}
    
    春天是养蚕的季节。每到这时候,我常常想起母亲来。解放前,我们家很穷,母亲就靠养蚕换点儿钱,给我们姐弟俩交学费。
    
    我们家门口有几株桑树。春天一到,桑树刚发出新芽,母亲就照例拿出几张蚕种来。每张蚕种不过一尺见方,上面布满了比芝麻还小的褐色的蚕卵。等桑叶长到榆钱大小的时候,蚕种上便有许多极小极小的蚕在蠕动。蚕的生命就是这样开始的。
    
    母亲微笑着,把这些小生命抖落在小匾里。匾里已经撒了一层剪成细丝的嫩桑叶。这是母亲带着我从桑树上摘来的,擦得干干净净,剪得又细又匀。
    
    蚕一天天地大起来,桑叶也一天天地剪得粗起来。等蚕长到半寸来长的时候,小匾换成大匾,就开始喂整片的桑叶了。每天清晨,姐姐把桑叶采回来,母亲吩咐我洗了手,用毛巾把一片片桑叶擦干,再轻轻地均匀地撒在匾里。
    
    蚕越来越大了,呆在一个匾里太挤了,就分成两个匾,再分成三个匾……匾一个又一个地增加着。母亲带着我和姐姐把隔壁的一间屋子打扫得干干净净,当作养蚕室,把七八个匾都搬了进去。每天深夜,母亲总要起来添桑叶。我一觉醒来,常常看见母亲拿着烛台去喂蚕。闪闪的烛光照着她那带着皱纹的慈祥的脸。
    
    推开养蚕室的门,立刻传来一片沙沙的声音,像下雨似的。那是蚕在大口大口地吃桑叶。那些日子,采桑叶的担子就落在父亲的肩上。父亲用很大的桑剪把桑叶连枝剪下来,成捆地背回来。
    
    蚕快“上山”了,母亲一夜要起来两次,累得她腰酸背痛。我和姐姐也常常起来帮忙。母亲把蚕沙大捧大捧地从匾里清出来,姐姐把桑叶大捧大捧地放进匾里。我用簸箕接蚕沙的时候,总看见母亲的额角上渗着汗。
    
    蚕“上山”了。它们被捉到用一束束麦秆扎成的“山”上。几天以后,“山”上结满了白的黄的茧子。母亲一面摘茧子,一面轻轻地对我和姐姐说:“孩子,上学得用功啊!这学费可来得不易呀……”
    
    我抬起头,看见母亲的两鬓又添了一些银丝。
    
\end{large}



\chapter{李时珍}

\begin{large}
    
    明朝有一位伟大的医药学家,叫做李时珍。
    
    李时珍家世代行医,父亲是太医院的名医。受父亲影响,他从小热爱医学。民间医生地位低下,生活艰苦,父亲不希望李时珍学医,想让他考科举。李时珍十四岁就考中了秀才,但他并不热心科举,三次乡试不中,对医学的兴趣却不减反增。父亲明白他心思,终于教他学医。三十岁时,李时珍已经是当地的名医了。三十八岁时,他被朝廷招入太医院。
    
    李时珍一面行医,一面博览医书、药书,研究药物。他发现旧的药书有不少缺点:同一种药物,往往有多个名称;不同的药物,却会用同一个名称。更有许多有用的药物没有记载,或者只有名称,性状、功效都有错漏。各地的大夫,都照着药书开药方,一个错误,不知要害死多少病人。于是,他决定重新编写一部完善的药书。
    
    为了写这部药书,李时珍不但在给人治病的时候注意积累经验,还不辞劳苦,远涉深山旷野,观察和采集药物,甚至冒着生命危险亲自试药,判断药性和药效。除了他家乡所在的湖广地区,江西、江苏、安徽的崇山峻岭,大多都留下了他的足迹。他不但到各地采药、试药,还拜访各地的名医,请教渔夫、樵夫、猎人、农民、车夫等各行各业的人,向他们学到了书上没有的知识。
    
    李时珍用了整整二十七年,终于编写成了一部新的药书,就是著名的《本草纲目》。《本草纲目》不仅记载了各种药物药方,还将各种有用的动植物、矿物分类,既是宝贵的药学经典,也是博物学的巨著。
    
\end{large}



\chapter{画杨桃}

\begin{large}
    
    我读小学四年级的时候,父亲开始教我画画。他对我要求很严,经常叮嘱我:“你看见一件东西,是什么样的,就画成什么样,不要想当然,画走了样。”
    
    有一次学校上图画课,老师把两个杨桃摆在讲桌上,要同学们画。我的座位在前排靠边的地方。讲桌上那两个杨桃的一端正对着我。我看到的杨桃根本不像平时看到的那样,而像是五个角的什么东西。我认认真真地看,老老实实地画,自己觉得画得很准确。
    
    当我把这幅画交出去的时候,有几个同学看见了,却哈哈大笑起来。
    
    “杨桃是这个样子的吗?”
    
    “倒不如说是五角星吧!”
    
    老师看了看这幅画,到我的座位坐下来,审视了一下讲桌上的杨桃,然后回到讲台前,举起我的画问大家:“这幅画画得像不像?”
    
    “不像!”
    
    “它像什么?”
    
    “像五角星!”
    
    老师的神情变得严肃了。半晌,他又问道:“画杨桃画成了五角星,好笑么?”
    
    “好——笑!”有几个同学抢着答道,同时发出嘻嘻的笑声。
    
    于是,老师请这几个同学轮流坐到我的座位上。他对第一个坐下的同学说:“现在你看看那杨桃,像你平时看到的杨桃吗?”
    
    “不……像。”
    
    “那么,像什么呢?”
    
    “像……五……五角星。”
    
    “好,下一个。”
    
    老师让这几个同学回到自己的座位上,然后和颜悦色地说:“提起杨桃,大家都很熟悉。但是,看的角度不同,杨桃的样子也就不一样,有时候看起来真像个五角星。因此,当我们看见别的人把杨桃画成五角星的时候,不要忙着发笑,要看看人家是从什么角度看的。我们应该相信自己的眼睛,看到是什么样的就画成什么样。”
    
    这位老师的话同我父亲讲的是那么相似。他们的教诲使我一生受用。
    
\end{large}



\chapter{珍贵的教科书}

\begin{large}
    
    1947年春天,我们延安小学转移到一个小山村里。在那炮火连天的战争环境中,我们仍然坚持学习。没有桌椅,就坐在地上,把小板凳当桌子;没有黑板,就用锅烟灰在墙上刷一块;没有粉笔,就拿黄土块代替。最困难的是没有书,我们只能抄一课学一课。我们多么渴望每人都能有一本教科书啊!
    
    一天下午,老师张指导员兴高采烈地对我们说:“告诉大家一个好消息,咱们有书啦!”真是个振奋人心的消息,我们都高兴得跳起来。指导员接着说:“同学们知道书是怎么来的吗?是在毛主席的关怀下印出来的!印书用的纸,是党中央从印文件用的纸里节省出来的!”在同学们的欢呼声中,我亮开嗓门喊起来:“快把书发给我们吧!”
    
    “书还在印刷所呢!”指导员微笑着说,“因为情况紧急,印刷所准备转移,所以今天必须有一个人和我一块儿把书取回来。
    
    “我去!”“我去!”同学们争先恐后地喊。最后决定让我跟指导员去印刷所取书。
    
    书领到了。我和指导员每人背上一捆,高兴地跨着大步往回走,恨不得一步赶回村子,把书发给同学们。
    
    正在这个时候,三架敌机从东北方向飞来,在村子上空盘旋着,嘶叫着。突然一架敌机呼啸着向我们这边飞来,一个俯冲,投下了一颗炸弹。
    
    “快卧倒……”指导员刚喊出口,轰隆一声,炸弹在我们身边爆炸了。我两耳一阵轰鸣,就什么也不知道了……等我醒来,才发觉自己头部受了伤。指导员趴在离我不远的地方,一动也不动。那捆书完整无缺地压在他的身子下面,被鲜血染红了。
    
    我扑到指导员身上大声喊:“指导员,指导员……”喊了好半天,指导员才微微睁开眼睛,嘴里叨念着:“书……书……”我扶他坐起来,激动地说:“指导员,书都在这儿。走,我背你回村。”他轻轻地摇了摇头,两眼望着那捆书,用微弱的声音说:“你们要……好好学习……将来……”
    
    多少年来,那捆生命换来的教科书和指导员没有说完的话,一直激励着我前进。
    
\end{large}



\chapter{爸爸和书}

\begin{large}
    
    我和姐姐有个书架,书架上整齐地排列着一百来本书,有童话、有历史故事、还有关于作文的。其中有一本薄薄的童话集,叫《皇帝的悲哀》。
    
    小伙伴常常来我们这儿借书,从来没有谁翻过这本《皇帝的悲哀》,因为它太破旧了。但是对我来说,这本书却比任何一本都要珍贵。只要一翻开这本书,当年买书的情景就清晰地浮现在我眼前。
    
    那时候我还小,爸爸所在的那家公司倒闭了。爸爸只好去做临时工,妈妈也要出去干零活。爸爸天天带着我,他干活,我就在一边玩。
    
    有一天下班回家,经过一家旧书店,爸爸突然停住了脚步,对我说:“给你买本书吧!”我听了高兴极了。
    
    “不过,给你买一本书,今天就不能乘汽车回家了。你愿意走路,就给你买。”
    
    “好,咱们走回去!”我说。
    
    爸爸走进旧书店,从一堆书里给我挑了这本《皇帝的悲哀》。尽管这本书很薄很薄,我捧着它还是像捧着一件珍宝似的。
    
    我跟着爸爸往回走。路真远啊!我有些走不动了,爸爸把我背了起来。过荒山大桥时,寒风呼啸,我冷得直发抖。爸爸问我:“怎么样,乘汽车比买书强吧?”
    
    我紧紧地伏在爸爸的背上,忍受着刺骨的寒冷,大声说:“不,买书比坐汽车强!”
    
    爸爸点了点头,用他的上衣把我裹得严严实实的。
    
    “走这么长的路是很累的。但是,不这样做的话,爸爸就没法给你买这本书了!”
    
    我仿佛觉得爸爸是含着热泪在说这番话的。
    
    那时我还没上学,老缠着爸爸,让他一遍又一遍地给我念这本薄薄的童话集。大概念了好几十遍吧,听着爸爸念书,我渐渐地懂得了读书的乐趣。
    
    后来,爸爸病了,我们的生活更苦了。可是,爸爸的精神一直很好,只要有点儿钱,就一定会给姐姐和我买书看。每一回买了书回来,他总要对我们说:“从明天起,咱们又得节衣缩食啦!”说着笑了起来。
    
    “节衣缩食,省吃俭用!”我们异口同声地说,和爸爸一同开心地笑起来。
    
    我和姐姐的一百来本书,全是爸爸这样省吃俭用买来的。他一心希望我们成为热爱学习的人。”
    
\end{large}



\chapter{劳动最有滋味}

\begin{large}
    
    劳动是最有滋味的事。肯劳动,连过新年都更有滋味,更多乐趣。
    
    记得当初我还是个孩子的时候,家里很穷,所以母亲在一入冬季就必须积极劳动,给人家浆洗大堆大堆的衣服,或代人赶做新大衫等,以便挣到一些钱,作过年之用。
    
    姐姐和我也不能闲着。她帮助母亲洗、做;我在一旁打下手儿——递烙铁、添火,送热水与凉水等等。我也兼管喂狗、扫地,和给灶王爷上香。我必须这么做,以便母亲和姐姐多赶出点活计来,增加收入,好在除夕与元旦吃得上包饺子!
    
    快到年底,活计都交出去,我们就忙着筹备过年。我们的收入有限,当然不能过个肥年。可是,我们也有非办不可的事:灶王龛上总得贴上新对联,屋子总得大扫除一次,破桌子上已经不齐全的铜活总得擦亮,猪肉与白菜什么的也总得多少买一些。由大户人家看来,我们的这点筹办工作的确简单得可怜。我们自己却非常兴奋。
    
    我们当然兴奋。首先是我们过年的那一点费用是用我们自己的劳动换来的,来得硬正。每逢我向母亲报告:当铺刘家宰了两口大猪,或放债的孙家请来三堂供佛的、像些小塔似的头号“ 密供” ,母亲总会说:咱们的饺子里菜多肉少,可是最好吃!当时,我不大明白为什么菜多肉少的饺子反倒最好吃。在今天想起来,才体会到母亲的话里确有很高的思想性。是呀,第一我们的饺子不是由开当铺或放高利贷得来的,第二我们的饺子是亲手包的,亲手煮的,怎能不最好吃呢?刘家和孙家的饺子必是油多肉满,非常可口,但是我们的饺子会使我们的胃里和心里一齐舒服。
    
    劳动使我们穷人骨头硬,有自信心。回忆起来,在那黑暗的岁月里,我们一家子怎么闯过了一关又一关,终于挣扎过来,得到解放,实在不能不感谢共产党,也不能不提到母亲的热爱劳动。她不懂得革命,可是她使儿女们相信:只要手脚不闲着,便不会走到绝路,而且会走得噔噔的响。
    
    虽然母亲也迷信,天天给灶王上三炷香,可是赶到实在没钱请香的时节,她会告诉灶王:对不起,今天饿一顿,明天我挣来钱再补上吧!是的,她自信能够挣来钱,使神仙不至于长期挨饿。我看哪,神佛似乎倒应当向她致谢、致敬!
    
    我也体会到:劳动会使我们心思细腻。任何工作都不是马马虎虎就能做好的。马马虎虎,必须另做一回,倒不如一下手就仔仔细细,做得妥妥贴贴。劳动与取巧是结合不到一处的。要不,怎么说劳动能改变人的气质呢。
    
\end{large}



\chapter{花生花}

\begin{large}
    
    七月的一天,我和妈妈路过一块花生地,看见绿叶丛中疏密有致地开着点点黄花,犹如绿毯上镶着金灿灿的宝石。我不由自主地走上前,顺手就要摘花生花。
    
    “快别摘。”妈妈制止了我,“花生的花,没有一朵是‘空花’。你摘一朵,就要少长一颗花生。”妈妈指了指不远处的果园子,又说:“花生开花不起眼,不像那些桃花、梨花爱张扬,花谢之后,把果实埋在地下,不愿意让人们知道结果有自己的一份功劳。”
    
    妈妈的话引起我的深思:多么可爱的花生花,就是这千千万万朵小黄花,默默地开放,默默地凋谢,默默地贡献千千万万颗花生。
    
\end{large}



\chapter{种子}

\begin{large}
    
    讲桌上堆放着洋槐树籽,有浅黄色的,有豆绿色的,还有紫红色的……籽粒里掺杂着荚皮和角柄。虽然这些树籽不太起眼,但毕竟是孩子们用一双双小手捡来的,是他们的劳动成果,更何况他们每个人采的树种还挺多呢!我扫视了全班同学一眼,准备说点儿什么……
    
    这时,走上来一个小女孩儿。她穿着一身素雅的秋装,显得落落大方,又略带羞涩。小女孩儿走到我跟前,冲我抿嘴一笑,低下头,把手伸进裤兜里。
    
    “怎么,没采到?”我问。
    
    “不,可是没有他们那么多。”她的脸刷地红了,撩起上眼皮看了我一眼,惭愧地站在那儿。
    
    “那,你采的呢?”我又问。
    
    她从兜里掏出一个小葫芦,又从兜里掏出一张纸,在桌子上展平,然后凝望着那小葫芦的嘴儿,小心翼翼地往外抖。一颗,两颗,三颗……我看着她倒出来的树种,不由得心里一动。那种子一般大小,有如饱满的黑豆,每一颗都闪着乌亮的光泽。
    
    我想她一定是用那双小手挑了又挑,选了又选,树种才能如此一般大小,闪闪发光!我被一颗虔诚的童心感染了,心里充满温暖。望着她那俊秀的脸颊、专注的神情,我仿佛看见在茫茫的山川原野上,一棵棵洋槐树正在茁壮成长,为辽阔的大地撑起一柄柄绿色的大伞。
    
    “就这么一点儿。”她摇晃一下小葫芦,抬起头来。我们的目光正好相遇。我笑着点点头。她害羞地一笑,轻轻掠了一下额前乌黑的短发,拿着小葫芦回到自己的座位上。
    
    我扫视了一下全班同学,发现几十双眼睛都在注视着那白纸上不多的洋槐树籽。我小心翼翼地把这些树籽包起来,唯恐丢失一颗。
    
\end{large}



\chapter{观潮}

\begin{large}
    
    钱塘江大潮,自古以来被称为天下奇观。
    
    农历八月十八是一年一度的观潮日。这一天早上,我们来到了海宁市的盐官镇,据说这里是观潮最好的地方。潮水到了这儿,场面最为壮观。我们随着观潮的人群,登上了海塘大堤。宽阔的钱塘江横卧在眼前。平静的江面,沐浴在雨后的阳光下,波光粼粼。江面越往东越宽,极远的地方水天相连,一层清蒙蒙的薄雾,笼罩江面。远处,几座小山在云雾中若隐若现。近处,镇海古塔、中山亭和观潮台屹立在江边。江潮还没有来,海塘大堤上却已经满是人了。没有多少喧闹,人们仿佛有着默契一般,时不时昂首东望,等着,盼着。
    
    午后一点左右,从远处传来隆隆的响声,好像闷雷滚动。顿时人声鼎沸,潮来了!我们踮着脚往东望去,江面还是风平浪静,看不出有什么变化。过了一会儿,响声越来越大,只见东边一条白线,把水天分开,人群又沸腾起来。
    
    那条白线很快地向我们移来,逐渐拉长,变粗,横贯江面。再近些,只见白浪翻滚,形成一堵两丈多高的水墙。浪潮越来越近,犹如千万匹白色战马齐头并进,浩浩荡荡地飞奔而来;那声音如同山崩地裂,大地仿佛都被震得颤动起来。
    
    霎时,潮头奔腾西去,可是余波还在漫天卷地般涌来,江面上依旧风号浪吼。过了好久,钱塘江才恢复了平静。看看堤下,江水已经涨了两丈来高了。
    
\end{large}



\chapter{高大的皂荚树}

\begin{large}
    
    在我们学校前面,有一块大约三十步见方的空地,这块空地后来成了我们的小操场。在这个操场东边,有一棵很高大的皂荚树。
    
    好大的皂荚树啊,我们六个小同学手拉着手,才能把它抱住。
    
    好茂盛的皂荚树啊,它向四面伸展的枝叶,差不多可以阴盖住我们整个小操场。
    
    皂荚树的叶片是小小的,有点像槐树的叶子。小小的叶子一串串,一层层,长得密密麻麻,结成了一项巨大的绿色的帐篷。
    
    春天,下小雨啦。皂荚树为我们遮挡着,雨滴就不会很快掉下来。我们就能够像平常一样,在操场上做体操,做游戏。
    
    夏天,暴烈的太阳当头照。有了皂荚树的遮挡,烈日就只能投下星星点点的光斑。我们活动在操场上,觉得格外凉爽。
    
    秋天,皂荚树上许许多多的皂荚儿成熟了,那样子,就像常见的大扁豆。高年级的同学爬上树去,用带钩子的小竹竿把皂荚儿构下来。小同学呢,把它们捡进筐子里,交给老师。
    
    每天,老师用皂荚熬了水,盛在脸盆里。上完课,我们的手上沾了些墨水,用皂荚水一洗,就又白白净净了。劳动过后,我们的手上、胳膊上满是土,满是泥,用皂荚水一洗,就又清清爽爽了。
    
    冬天,皂英树落叶了。枯黄的小叶子,打着旋儿,不断地飘落,在地上铺了一层又一层。这时候,我们就把树叶扫到一起,堆放在墙脚下。
    
    记得有一天,天气很冷,同学们欢叫着点燃了一堆树叶,轻烟袅袅。
    
    褐红色的火苗升了起来。飘舞的轻烟和跳动着的火苗,映在我们的笑眼里,引起了我的沉思:“皂荚树啊皂荚树,你曾经自己淋着,给我们挡雨;你曾经自己晒着,给我们遮阳;现在你又燃烧着自己,给我们温暖。皂荚树啊,你给了我们多少快乐,多少启迪。”想着想着,我的心里,好像有一颗种子在生根、发芽……
    
\end{large}



\chapter{海滨小城}

\begin{large}
    
    这是一座海滨小城。人们走到街道尽头,就可以看见浩瀚的大海。天是蓝的,海也是蓝的。海天交界的水平线上,有棕色的机帆船和银白色的军舰来来往往。天空飞翔着白色的、灰色的海鸥,还飘着跟海鸥一样颜色的云朵。
    
    早晨,机帆船、军舰、海鸥、云朵,都被朝阳镀上了一层金黄色。帆船上的渔民,军舰上的战士,他们的脸和胳臂也镀上了一层金黄色。
    
    海边是一片沙滩,沙滩上遍地是各种颜色、各种花纹的贝壳。这里的孩子见得多了,都不去理睬这些贝壳,贝壳只好寂寞地躺在那里。远处响起了汽笛声,那是出海捕鱼的船队回来了。船上满载着银光闪闪的鱼,还有青色的虾和蟹,金黄色的海螺。船队一靠岸,海滩上就喧闹起来。
    
    小城里每一个庭院都栽了很多树。有桉树、椰子树、橄榄树、凤凰树,还有别的许多亚热带树木。初夏,桉树叶子散发出来的香味,飘得满街满院都是。凤凰树开了花,开得那么热闹,小城好像笼罩在一片片红云中。
    
    小城的公园更美。这里栽着许许多多榕树。一棵棵榕树就像一顶顶撑开的绿绒大伞,树叶密不透风,可以遮太阳,挡风雨。树下摆着石凳,每逢休息的日子,石凳上总是坐满了人。
    
    小城的街道也美。除了沥青的大路,都是用细沙铺成的,踩上去咯吱咯吱地响,好像踩在沙滩上一样。人们把街道打扫得十分干净,甚至连一片落叶都没有。
    
    海滨小城美如画,连时间都想停留下来。
    
\end{large}


\newpage

\textbf{注释}:

\vspace{-1em}

\begin{itemize}
    \setlength\itemsep{-0.2em}
    \item 〔海滨〕海边。
    \item 〔浩瀚〕(水)广大,无边无际。
    \item 〔舰〕战船。
    \item 〔棕色〕褐色。
    \item 〔镀〕在金属表面附上另一种金属薄层的工艺。引申为在表面附上薄薄一层。
    \item 〔理睬〕注意到,并给出反应。
    \item 〔寂寞〕因无人陪伴、得不到他人回应而难受。
    \item 〔满载〕装满。
    \item 〔喧闹〕声多且大而杂乱。指环境热闹或令人心烦。
    \item 〔沥青〕石油的一种,又称柏油。色褐黑而粘稠,常用来铺路。
    \item 〔绒〕柔软细小的毛。
\end{itemize}

\chapter{蝙蝠和雷达}

\begin{large}
    
    晴朗的夜空出现两个亮点,越来越近,才看清楚是一红一绿的两盏灯。接着传来了隆隆声,一架飞机要降落了。在漆黑的夜里,飞机怎么能安全飞行呢?原来是人们从蝙蝠身上得到了启示。
    
    蝙蝠是在夜里飞行的,还能捕捉飞蛾和蚊子;而且无论怎么飞,从来没见过它跟什么东西相撞,即使一根极细的电线,它也能灵巧地避开。难道它的眼睛特别敏锐,能在漆黑的夜里看清楚所有的东西吗?
    
    为了弄清楚这个问题,科学家做了一次试验。在一间屋子里横七竖八地拉了许多绳子,绳子上系着许多铃铛。他们把蝙蝠的眼睛蒙上,让它在屋子里飞。蝙蝠飞了几个钟头,铃铛一个也没响,那么多的绳子,它一根也没碰着。
    
    科学家又做了两次试验:一次把蝙蝠的耳朵塞上,一次把蝙蝠的嘴封住,让它在屋子里飞。蝙蝠就像没头苍蝇似的到处乱撞,挂在绳子上的铃铛响个不停。三次不同的试验证明,蝙蝠夜里飞行,靠的不是眼睛,它是用嘴和耳朵配合起来探路的。
    
    科学家经过反复研究,终于揭开了蝙蝠能在夜里飞行的秘密。它一边飞,一边从嘴里发出一种声音。这种声音叫做超声波,人的耳朵是听不见的,蝙蝠的耳朵却能听见。超声波像波浪一样向前推进,遇到障碍物就反射回来,传到蝙蝠的耳朵里,蝙蝠就立刻改变飞行的方向。
    
    科学家模仿蝙蝠探路的方法,给飞机装上了雷达。雷达通过天线发出无线电波,无线电波遇到障碍物就反射回来,显示在荧光屏上。驾驶员从雷达的荧光屏上,能够看清楚前方有没有障碍物,所以飞机在夜里飞行也十分安全。
    
\end{large}



\chapter{各种各样的玻璃}

\begin{large}
    
    夜深了,突然,一座博物馆里传出急促的报警声。警察马上赶来,抓住了一个划破玻璃、企图盗窃展品的犯罪嫌疑人。你也许不会相信,报警的不是值夜班的看守,而是被划破的玻璃!
    
    这是一种特殊的玻璃,里面有一层极细的金属丝网。金属丝网接通电源,跟自动报警器相连。犯罪嫌疑人划破玻璃,碰着了金属丝网,警报就响起来了。这种玻璃叫“夹丝网防盗玻璃”,博物馆可以采用,银行可以采用,珠宝店可以采用,存放重要图纸、文件的建筑物也可以采用。
    
    另一种“夹层玻璃”不是用来防盗的。它非常坚硬,受到猛击仍安然无恙,即使被打碎了,碎片仍然藕断丝连地粘在一起,不会伤人。高层建筑必须采用这种安全可靠的玻璃。
    
    还有一种“变色玻璃”,能够对阳光起反射作用。建筑物装上这种玻璃,从室内看外面很清楚,从外面看室内却什么也瞧不见。变色玻璃还会随着阳光的强弱而改变颜色的深浅,调节室内的光线。所以人们又把这种玻璃叫做“自动窗帘”。
    
    你可能会想,窗子上的玻璃要是能使房间里冬暖夏凉,那该多好!这样的玻璃早就问世了,它就是“吸热玻璃”。在炎热的夏天,它能阻挡强烈的阳光,使室内比室外凉爽;在严寒的冬季,它把冷空气挡在室外,使室内保持温暖。
    
    噪音像一个来无影去无踪的“隐身人”,不像烟尘和废水那样可以集中起来处理。尽管这位“隐身人”难以对付,人们还是想出了许多制服它的办法。“消音玻璃”就是消除噪音的能手。临街的窗子上如果装上这种玻璃,街上的声音传到房间里就几乎听不到了。
    
    在现代化的建筑中,新型玻璃正在起着重要作用。随着科技不断进步,人们将会创造出更多的奇迹。
    
\end{large}



\chapter{糖画}

\begin{large}
    
    今天,老师带我们来到糖艺手工作坊。作坊在离流花湖不远的一条街上。进门是一个不大的院子,院子里种了一棵洋紫荆树。树荫下就是工作间的大门,里面是宽敞的空间,放着两条长长的桌子。糖艺师傅已经在等着我们了。
    
    我们首先观看了介绍糖艺的纪录片。糖艺是我国的传统手工艺,已经有数百年的历史。不同地方的糖艺各有不同。作坊里可以体验的糖艺有糖画和糖塑两种。两种糖艺用的原料都差不多,是用麦芽糖熬煮得到的糖稀。
    
    我和几个同学决定学画糖画。只见师傅用一个平平的铜勺子,把调好的糖稀从保温的锅里舀出来,手微微一侧,细细的糖稀就从勺边淌下。师傅握着勺子的手有节奏地晃动,白板上就留下了一道美丽的图案。这是要画什么呢?只过了一小会儿,我们就喊出来了:“这是马!”师傅笑了笑,勺子轻晃,又画上了鬃毛。马的轮廓画完,用更细的线反复填充,再附上沾了糖稀的木签,最后用铲子小心地铲起来,一匹昂首飞奔的糖马就完成了。
    
    轮到我们了。或许是有点紧张,我握着勺子的手总有些发抖。师傅看到了,笑着说:不要怕,多试几次,熟能生巧嘛。果然,过了几分钟,我们已经可以按自己的想法,在板上留下粗细不同的线条了。接下来是对着纸上描好的兔子,把糖稀浇在轮廓上。看着简单的图案,我心想,这也并不难嘛。谁知道,说起来容易,做起来难,我心里想的均匀流畅的线条,到了手下,变得忽左忽右,时粗时细,歪歪扭扭的。对比师傅画的,差的不是一点儿。师傅见我有点气馁,就说:“世上无难事,只怕有心人。只要多练,总能进步的。”在师傅的鼓励下,我反复练习起来。渐渐地,我找到了一点感觉。手移动的快慢,勺子的高低,会对线条造成什么影响,心里大致有数了。
    
    一个小时后,看着串在木签上,略显臃肿,有点毛糙的兔子,我竟然有些舍不得了。每一门技艺,都是勤学苦练的结果啊。
    
\end{large}



\chapter{西门豹}

\begin{large}
    
    战国时期,魏王让西门豹做邺县的县令\footnote{〔县令〕由国王任命,管理一县的官。}。邺县在漳河边上。西门豹到了这个地方,看到良田荒芜,人烟稀少,就找了一位老大爷来,问他是怎么回事。
    
    老大爷说:“这都是河伯娶媳妇给闹的。这河伯是漳河的神,每年都要娶一个年轻姑娘。要是不给他送去的话,漳河就必定要发大水,把田地全淹了。”
    
    西门豹问:“这话是谁说的?”
    
    老大爷说:“这是三老说的\footnote{〔三老〕每乡掌管教化民众、举荐人才的职位。由五十岁以上的人担任。}。三老和乡里的长老们,还有县里的大人,每年出面给河伯办喜事,逼着老百姓们出钱出力。每年闹一次,要收几百万钱\footnote{〔钱〕古代国家发行的通货,用铜、铁等金属制作,大小、重量、成分、形状统一。}。办喜事只花二三十万,多下来的都跟巫婆分了。”
    
    西门豹问:“那新娘是从哪儿来的呢?”
    
    老大爷说:“哪家有年轻的女孩儿,巫婆就到哪家去选。有钱的人家花点儿钱就糊弄过去了,没钱的人家,只好眼睁睁地看着自家女儿被他们拉走。到了河伯娶媳妇那天,他们会在漳河边做好丝绸帷帐,床铺枕席,让女孩儿坐在上面,顺着水流漂去。漂个几十里,女孩儿就沉到河里去了。有女孩儿的人家差不多都逃到外地去了,所以人越来越少,这地方也越来越穷。” 
    
    西门豹说:“这样说,河伯还真是灵啊。下一回他娶媳妇的时候,告诉我一声,我也去送送新娘子。”
    
    到了河伯娶媳妇的这天,漳河边上站满了百姓。西门豹带着士兵和廷掾\footnote{〔廷掾〕县令的下属,管理各乡税收。由本县人担任。},真的来了。三老和巫婆急忙去迎接。那巫婆看着已经七十多岁了,背后跟着几个弟子。
    
    西门豹说:“去把新娘领过来给我瞧瞧。”巫婆就叫弟子把那个打扮好的姑娘带过来。
    
    西门豹一瞧,那女孩儿满脸都是泪水。他回头对巫婆说:“不行,这个姑娘不够年轻漂亮,河伯是不会满意的。麻烦你先去跟河伯通告一声,说我再要选一个漂亮的,过几天再送过去。”说完,他叫士兵架起巫婆,把她投进了漳河里。
    
    巫婆在河里扑腾了几下,然后就沉下去了。过了一会儿,西门豹皱着眉头说:“这巫婆怎么还不回来呀,麻烦弟子去催催吧。”说着,叫士兵把一个弟子丢进了河里。过了许久,河里没有动静。西门豹就叫士兵继续把巫婆的弟子丢进河里。如此反复,丢了好几个弟子,仍然没有人回来。西门豹说:“这巫婆教不好弟子,传个话都做不好。麻烦三老去跟河伯说说,让她们回来。”话说完,又叫士兵把三老也丢进了河里。
    
    西门豹面对着漳河站了许久。廷掾和长老们提心吊胆,连大气都不敢出。西门豹回头,看着他们说道:“这三老也是没用!请你们一起去催催吧。”说着又要叫士兵把他们都投下漳河去。
    
    廷掾和长老们面如死灰,跪在地上磕头求饶,把头都磕破了,一直在淌血。西门豹说:“好吧,我再等一会儿。”
    
    过了好一会儿,他才说:“都起来吧。看样子,河伯把他们都留下了。你们先回去吧。”从此,乡里再也不敢提给河伯娶媳妇的事。西门豹立刻发动老百姓,开挖了十二条渠道,把漳河的水引到百姓的农田里。庄稼得到了灌溉,年年都有好收成。
    
\end{large}



\chapter{中国石}

\begin{large}
    
    驻守在戈壁滩上,举目是无边的沙海,脚下是温滩的碎石。战士们休息的时候,常常随手拣几块可心的石头,积聚多了,还要举行“赛石会”。经过几次比赛,我也成了石头迷。
    
    一个初夏的早晨,姗姗来迟的春雨洗润了戈壁滩。这正是拣石头的好机会,我背起挎包出了营房。
    
    雨后大漠,风清气爽,哨所前面那一排排白杨显得更加挺拔、繁茂。瑰丽的朝霞倾泻在戈壁滩上,裸露在黄沙上的石头闪着珠光玉彩。我俯身打量这些散落满地的小精灵,把可心的一颗颗拣起来。
    
    “丁冬”的驼铃声伴来了拉水的勒勒车,赶车的是个小姑娘。她看见我背着满挎包石头,就邀我去珍珠泉,说经那里泉水洗过的石头会特别清亮。
    
    我们来到珍珠泉,小姑娘帮我舀起泉水清洗拣来的石头。忽然,她惊叫起来:“雄鸡!这块石头像雄鸡!”我接过来一看,真是块像雄鸡的石头。我注视了一会儿,忽然又觉得它很像祖国版图的形状,不禁喊出来:“它像中国,应该叫‘中国石’!”
    
    “中国石?”小姑娘吃惊地看着我。我说:“老师常讲,我们祖国的版图像一只傲然挺立的雄鸡,这块石头不正是这样的吗?”
    
    这块“中国石”有拳头大小,黄白色,不仅整体酷似祖国版图,连表面皱痕的起伏也与大陆地貌相似。小姑娘在它上面找到了天山、祁连山、吐鲁番盆地,我找到了长江、黄河、大兴安岭。我们俩都很高兴,又挑选了一些石粒,让“台湾”、“海南”等岛屿依偎着“祖国”。
    
    回到哨所,大伙争相观看“中国石”。为了保存好这块石头,连长拿出了自己装军功章的盒子,文书在上面工工整整地写了“中国石”三个金字。他们嘱咐我一定要精心保管,千万别摔坏了,弄丢了。
    
    在以后的赛石会上,“中国石”屡屡得魁,赢得了“最佳宝石”的盛誉。大家都把“中国石”奉为至宝,寂寞的时候,捧着它看看,心里感到温暖;劳累的时候,捧着它看看,就忘记了劳累。每逢假日,连里让我把它放在俱乐部里展览,战友们都说戈壁滩虽然艰苦,可苦中有乐,因为我们在“祖国”身边,祖国在我们心中。
    
\end{large}



\chapter{古井}

\begin{large}
    
    我们村的东头有一口古井。井里的水清凉可口。村里的人都到这儿取水。
    
    古井离我们家才十来米远。每天从晨光熹微到暮色降临,取水的人络绎不绝地从我家门前走过,桶儿叮叮当当,扁担吱悠吱悠,像一支支快乐的乡间小曲。门前的路面湿漉漉的,老是像刚下过一场春雨似的。
    
    我们家的邻居是一对年过六旬的老夫妇。丈夫是个老党员,在抗日战争时期腿负过伤,走路一瘸一拐的;妻子又矮又瘦,身子很单薄,简直像一阵风能把她吹倒似的。老两口只有一个女儿,在外地教书。
    
    乡亲们见两位老人用水有困难,这个帮着挑一担,那个帮着提一桶,老人家里的水缸总是满满的。两位老人多次表示,要给帮他们挑水的人一些报酬,可是谁也不肯接受。
    
    “日子长着哩,俺们不能总让大家白出力气啊。”老两口带着歉意说。
    
    “那口古井给人们出了多少力气?可它从来没跟人们要过报酬。”乡亲们总是这样劝说两位老人。
    
    多好的古井啊,它不仅为乡亲们提供生命的泉水,还陶冶着乡亲们的品格,使他们懂得应该怎样做人。
    
\end{large}



\chapter{峨眉道上}

\begin{large}
    
    前面没有平坦的路了。所谓路,就是用一块块两尺见方的石板接连起来的台阶。我们一步一步向上爬,非常吃力。停步仰望,只见石阶像一架通天的梯子,竖在前面树木葱茏的陡坡上。我们的旅程,就是要攀登这架长长的天梯。
    
    在途中,我们遇到了十几个背竹篓的人。他们把竹篓靠在路旁岩石上,站在那儿歇息。走近了才发现,每个竹篓里都装着一块大石板。背着石板攀登天梯,可真了不起。
    
    我问他们,往山上背石板做什么。一位长者指了指脚下的石阶,操着浓重的四川口音回答说:“干这个!”
    
    “铺路?”
    
    他点了点头,告诉我说:去洪椿坪的那段路被山洪冲毁了。他们在十多里外开山取石,凿成了石板,背上山去重铺冲毁的路。
    
    他们是给峨眉山铺路的人啊!
    
    峨眉山光是游览路线就有两百多里。铺这么长的路要多少块石板呢?几万,几十万,还是几百万?这数不清的石板,不都是这样一块一块背上山来的吗?山路这样狭窄,不能用机械,只有靠人工。想到这里,我内心感到一阵愧疚。我们走在别人铺的道路上还嫌吃力,而铺路的人,把一块块石板背上山,铺成路。这样日复一日,年复一年,他们默默地吃了多少苦,流了多少汗,全是为了别人的方便。如果没有他们的辛劳,没有他们的牺牲,就没有这用石板砌成的阶梯,就没有脚下的路,也就没有游人的欢乐。
    
    默默付出辛劳的铺路人啊,你们是真正的英雄!
    
\end{large}



\chapter{太阳}

\begin{large}
    
    传说古时候天上有十个太阳,晒得地面寸草不生,人们热得受不了。一个箭法很好的人射掉九个太阳,只留下一个。其实,太阳离我们有一亿五千万公里远。到太阳上去,如果步行,日夜不停地走,差不多要走三千五百年;就是坐飞机,也要飞二十几年。这么远,箭哪能射得到呢?
    
    我们看到太阳,觉得它并不大,实际上它大得很,约一百三十万个地球的体积才能抵得上一个太阳。因为太阳离地球太远了,所以看上去只有一个盘子那么大。
    
    太阳会发光,会发热,是个大火球。太阳的温度很高,表面温度有五千多摄氏度,就是钢铁碰到它,也会变成气体。
    
    太阳虽然离我们很远很远,但是它和我们的关系非常密切。有了太阳,地球上的庄稼和树木才能发芽、长叶、开花、结果;鸟、兽、虫、鱼才能生存、繁殖。如果没有太阳,地球上就不会有植物,也不会有动物。我们吃的粮食、蔬菜、水果、肉类,穿的棉、麻、毛、丝,都和太阳有密切的关系。埋在地下的煤炭,看起来好像跟太阳没有关系,其实离开太阳也不能形成。因为煤炭是由远古时代的植物埋在地层底下变成的。
    
    地面上的水被太阳晒着的时候,吸收了热,变成了水蒸气。空气上升时,温度下降,其中的水蒸气凝成了无数的小水滴,飘浮在空中,变成云。云层里的小水滴越聚越多,就变成雨或雪落下来。
    
    太阳晒着地面,有些地区吸收的热量多,那里的空气就比较热;有些地区吸收的热量少,那里的空气就比较冷。空气有冷有热,才能流动,形成风。
    
    太阳光有杀菌的作用,我们可以利用它来预防和治疗疾病。
    
    地球上的光明和温暖都是太阳送来的。如果没有太阳,地球上将到处是黑暗,到处是寒冷,没有风、雪、雨、露,没有草、木、鸟、兽,自然也不会有人。一句话,没有太阳,就没有我们这个美丽可爱的世界。
    
\end{large}



\chapter{绿叶}

\begin{large}
    
    我的童年,是在大自然里度过的。高粱秆儿剥开来,做成马车、灯笼;河边抓把泥,捏成碾磨、盆碗;柳条儿、苇叶儿做笛子,葫芦瓢做船,荷叶当伞……一双小手创造了多少可爱的玩具!然而,最使我着迷的还是绿叶。
    
    我永远感激我的启蒙老师。除了教我们读书以外,老师还教我们搜集标本,采来各种绿叶汇拢在一起,并讲述它们的知识和趣闻。那真是令人愉快的活动。
    
    星期天,我们跑遍密密的树林、杂草丛生的河边、广阔的田野和一道道土岗。爬大树,钻丛林,嬉笑,打闹,欢乐的笑声惊飞了觅食的小鸟。篮子里装满了各种绿叶。我们用元宝树叶串成项链,用金黄色的菟丝子草做成戒指和手镯,豆角花挂在耳朵上,野菊花插满了小辫儿……在旷野的课堂里,绿叶和野花谱写成我们生活的乐章。
    
    我们每人都想找到一种新奇少见的叶子,因此少不了争强和探险。我们常常有意外的发现,但也少不了刺伤手脚。毛栗子、酸枣树的尖刺儿几乎在每人身上都留下过伤痕。
    
    秋天,树叶在风中飘落,像一群群蝴蝶飞向我们。我们背着筐,扛着筢,奔跑着,欢呼着,搂树叶,堆成垛,躺在上面打滚,翻跟头,坐下来挑选出那些漂亮的叶子。鲜红的,金黄的,串成长长的彩色叶链挂在教室里,满屋洋溢着丰收的欢乐。那许许多多留做标本的叶子,成了我们的珍宝:圆形的、条状的、桃形的、针状的、蛋形的、元宝状的、叶面带茸毛的……一一陈列开来。
    
    老师让我们观察这些叶子,讲述关于它们的故事。羽状的山扁豆叶子可当茶,祛痰止渴;细长的垂柳叶子可解酒毒,治皮癣;桑叶清热明目,治手脚麻木;薄荷叶医感冒头疼……祖辈传下来的民间药方,丰富有趣的生活知识,也随着一串串叶片,留在我的记忆里。
    
\end{large}



\chapter{九寨沟}

\begin{large}
    
    在四川北部南坪、平武、松潘三县交界的群山之中,有几条神奇的山沟。因为周围散布着九个村寨,所以人们称它九寨沟。
    
    从南坪往西四十公里,就来到九寨沟。一进入景区,就像到了一个童话世界。
    
    一座座雪峰插入云霄,峰顶银光闪闪。大大小小的湖泊,像颗颗宝石镶嵌在彩带般的沟谷中。湖水清澈见底,湖底石块色彩斑斓。从河谷至山坡,都是森林。天气晴朗时,蓝天、白云、雪峰,森林,都倒映在湖水中,构成一幅幅五彩缤纷的图画。难怪人们把这些湖泊叫做“五花海”、“五彩池”呢。由于河谷高低不平,湖泊与湖泊之间恰似一级级的台阶。由此形成的一道道高低错落的瀑布,宛如白练腾空,银花四溅,蔚为壮观。
    
    继续前进,林深叶茂,游人逐渐稀少。注意,这时你已经来到珍稀动物经常出没的地区。也许,就在不远处,有一只金丝猴,正攀吊在一棵大树上,眨巴着一对机灵的小眼睛向你窥视。也许,会有一群羚羊突然窜出来,还没等你看清它们,又消失在前方的丛林中。也许,你的运气好,会在远处密密的竹丛中,发现一只憨态可掬的大熊猫,若无其事地坐着,咀嚼鲜嫩的竹子。也许,你还会看见一只机敏灵活的小熊猫,从山坡跑下谷底,对着湖面美滋滋地照镜子。
    
    雪峰插云,古木参天,平湖飞瀑,异兽珍禽……九寨沟真是个神奇的地方啊!
    
\end{large}



\chapter{兵马俑}

\begin{large}
    
    1974年3月29日,陕西省临潼县西杨村的农民在打井时发现了一些陶俑\footnote{〔陶俑〕用陶做的人像,代替真人陪葬。俑:用于陪葬的人。}的碎片。陕西省的考古队闻讯来调查时,以为一周就能结束工作。谁知道,直到今天,这里的考古工作仍在继续。这就是举世闻名的秦始皇兵马俑。
    
    兵马俑规模宏大。已发掘的三个俑坑,总面积近两万平方米,差不多有五十个篮球场那么大,坑内有兵马俑近八千个。三个俑坑中,一号坑最大,东西长230米,南北宽62米,总面积有14260平方米;坑里的兵马俑也最多,有六千多个。一号坑上面,现在已经盖起了一座巨大的拱形大厅。走进大厅,站在高处鸟瞰,坑里的兵马俑一行行,一列列,十分整齐,排成了一个巨大的长方形军阵,真像是秦始皇当年统率的一支南征北战、所向披靡\footnote{〔鹖冠〕古代武将戴的头冠,两侧有上竖的鹖羽。鹖:一种类似野鸡的鸟。}的大军。
    
    兵马俑不仅规模宏大,而且类型众多,个性鲜明。
    
    将军俑身材魁梧\footnote{〔颔首低眉〕颔首:低头,低眉:向下看。},头戴鹖冠\footnote{〔乘〕战车的量词。战车千乘:战车的数量上千。},身披铠甲,手握宝剑,昂首挺胸。那神态自若的样子,一看就知道是久经沙场,重任在肩。
    
    武士俑平均身高约1.8米,体格健壮,体形匀称。它们身穿战袍,披挂铠甲,脚蹬翘头战靴,手持兵器,整装待发。
    
    骑兵俑上身着短甲,下身着紧口裤,脚蹬长靴,右手执缰绳,左手持弓箭,好像随时准备上马冲杀。
    
    马俑与真马一般大小,一匹匹形体健壮,肌肉丰满。那跃跃欲试的样子,好像一声令下,就会撒开四蹄,腾空而起,踏上征程。
    
    每个兵马俑都是极为精美的艺术珍品。仔细端详,神态各异:有的颔首低眉\footnote{5},若有所思,好像在考虑如何相互配合,战胜敌手;有的目光炯炯,神态庄重,好像在暗下决心,誓为秦国统一天下而殊死搏斗\footnote{6};有的紧握双拳,好像在听候号角,待命出征;有的凝视远方,好像在思念家乡的亲人……走近它们的身旁,似乎能感受到轻微的呼吸声。
    
    秦兵马俑,在古今中外的雕塑史上是绝无仅有的。它惟妙惟肖地模拟军阵的排列,生动地再现了秦军雄兵百万、战车千乘\footnote{7}的宏伟气势,是古代劳动人民智慧与汗水的结晶。
    
\end{large}


\newpage

\textbf{注释}:

\vspace{-1em}

\begin{itemize}
    \setlength\itemsep{-0.2em}
    \item 〔魁梧〕指人强壮高大。
    \item 〔披靡〕(草木)随风散乱地倒下。所向披靡:所到之处,敌军就溃散。
    \item 〔殊死〕拼命,抱着必死的心。
\end{itemize}

\chapter{冬眠}

\begin{large}
    
    九月的一个傍晚,我带着猎狗在镇外的葡萄园散步。猎狗把鼻尖凑在地面上嗅来嗅去,突然汪汪地叫起来。我过去一看,一只受惊的刺猬在那里紧缩成一团,像个刺儿球。我决定把它带回家去,看看它怎样冬眠。
    
    我脱下毛衣,抛在刺猬旁边用靴尖轻轻地把这一团小东西拨进毛衣里,包好了带回家。当天晚上,我正睡着,忽然听到从厨房里传来骨碌碌的滚动声。我轻轻地下了楼,推开厨房门,扭亮电灯一看,真好玩,原来刺猬偶然碰倒了酒瓶,它喜欢那骨碌碌的声音,正把酒瓶当球玩呢。第二天早晨,我发现放在它面前的牛奶碗空了,牛肉也吃光了。
    
    我在一本书中读过,冬眠不是睡眠,和一年四季也扯不上关系。一般地说,它是动物在漫长的严冬,在不容易找到食物的季节中减少体力消耗的一种自然现象。如果你在七月把动物放进冰箱里,它也是要冬眠的。因此,低温是促成冬眠的主要原因。
    
    动物进入冬眠,体温下降,以适应周围的气温,其他的生理机能也一同减弱。举例来说,一只清醒的刺猬每分钟呼吸约五十次,在冬眠的时候至多呼吸八次,有时只呼吸一次,甚至一连几分钟都不呼吸。一只清醒的刺猬每分钟心跳二百次,冬眠的时候减少到二十次。
    
    这小东西跟我还不熟,我不能把耳朵贴在它胸前计算它的心跳;恐怕它也不肯让我把温度计插进它的鼻孔去测量体温。我只好用别的方法测验这个满身是刺的朋友怎样冬眠。刺猬爱在户外地下找个洞穴,或者钻进大堆树叶下面隐藏起来冬眠。我把它放进地窖,把一碗牛奶和一盘牛肉放在盛满细\footnote{〔刨花〕刨木料时刨下来的薄片,多为卷状。}的竹篮旁边。我本来以为,刺猬在十五摄氏度以下会失去胃口,钻进细刨花里去沉沉入睡。结果完全不是这样,它似乎对饮食比睡眠更感兴趣,气温降到十二摄氏度了,它还是不肯进竹篮去睡,用它的刺把刨花拨的满地都是。一会儿工夫,实验用的牛奶和牛肉都不见了。原来它感到冷,需要多吃东西来保持体温。它是在竭力驱走冬眠。
    
    直到有一天早晨,气温降到七摄氏度,我到地窖去作例行观察,才听不到刺猬摆动身体发出的嘶嘶声了,它终于进入梦乡。半个月后,刺猬圈成一团,睡的真甜,小鼻尖从刺丛中露出来。我拿走细刨花,它没有反应。我相信,即使把它拎起来,它也不会醒,至少不会立刻醒。第三周开始,我检查竹篮,一切状况如故,刚转身离开,电筒的光突然照见地上有一小滩水。
    
    地窖里一向干燥得很,水究竟是从哪里来的? 走近一看,我发现从那小滩水到竹篮之间的水泥地上有它的足迹。我断定这是它撒的尿。我立刻端来牛奶和牛肉,放在竹篮附近. 第二天早晨,发现牛奶和牛肉都不见了。以后每隔两三个星期,刺猬会醒过来一次。假如地上有一小滩水,我就知道该给它预备饮食了。它会吃些牛奶和牛肉——有时多,有时少,然后继续大睡。整个冬季,地窖里的气温一直很稳定,总在七摄氏度左右。
    
    到了第二年三月,阳光灿烂,我把刺猬连竹篮从地窖里拿上来,放到比较温暖的车房里,好让阳光照射到它身上。一天晚上,我又听到酒瓶滚动的声音——酒瓶是我放在旁边作信号用的,刺猬显得很健壮。我能使它安然度过冬天,心里感到特别高兴。
    
    我向这位小朋友问早安,没想到碰了个大钉子,它凶狠地向我嘶叫。以前那段交情荡然无存,它又成为野性难训的畜生了。我于是又用旧毛衣把它包好,把这一团刺儿球送回当初发现它的地方。
    
\end{large}



\chapter{七月的天山}

\begin{large}
    
    七月的新疆,最理想的是骑马上天山\footnote{〔天山〕亚洲中部山脉,横贯新疆中部。}。
    
    进入天山,戈壁滩\footnote{〔戈壁滩〕中亚的广大荒漠地区,也叫瀚海沙漠。}上的炎暑就被远远地抛在后边,迎面送来的雪山寒气,会使你感到像秋天似的凉爽。蓝天衬着高耸的巨大的雪峰,太阳下,雪峰间的云影就像白缎上绣了几朵银灰色的花。融化的雪水,从高悬的山涧、从峭壁断崖上飞泻下来,像千百条闪耀的银链,在山脚下汇成冲激的溪流,浪花往上抛,形成千万朵盛开的白莲。每到水势缓慢的洄水涡,都有鱼儿在欢快地跳跃。这个时候,饮马溪边,你骑在马上,可以俯视阳光透射到的清澈的水底,在五彩斑斓的溪水和石子之间,鱼群闪闪的鳞光映着雪水清流,给寂静的天山增添了无限的生机。
    
    再往里走,天山显得越来越美。沿着白皑皑群峰的雪线\footnote{〔雪线〕冰川、雪山冰雪累积和融化平衡之处,雪线以上终年积雪。}以下,是蜿蜒无尽的翠绿的森林,密密的塔松像撑开的巨伞,重重叠叠的枝桠,漏下斑斑点点细碎的日影。骑马穿行林中,只听见马蹄溅起漫流在岩石上的水声,使密林显得更加的幽静。
    
    走进天山深处,山色逐渐变得柔嫩,山形也逐渐变得柔美。这里溪流缓慢,萦绕着每一个山脚。在轻轻荡漾着的溪流的两岸,满是高过马头的野花,五彩缤纷,像织不完的锦缎那么绵延,像天边的霞光那么耀眼,像高空的彩虹那么绚烂。马走在花海中,显得格外矫健;人浮在花海上,显得格外精神。在马上你用不着离鞍,只要稍一伸手就可以捧到满怀心爱的鲜花。
    
    虽然天山这时并不是春天,但是,哪一个春天的花园,能比得过这时的天山呢?
    
\end{large}


\newpage

\textbf{注释}:

\vspace{-1em}

\begin{itemize}
    \setlength\itemsep{-0.2em}
    \item 〔皑皑〕形容洁白的样子。常用来形容雪和被雪覆盖的巨大事物,比如山脉、山峰等。
    \item 〔缤纷〕繁多而杂乱。
    \item 〔荡漾〕水波起伏摇动。
    \item 〔蜿蜒〕蛇类曲折爬行的样子。引申指曲折延伸的动作和样子。
    \item 〔萦绕〕环绕、缠绕。萦:柔软的细丝或藤蔓回卷环绕物体的样子。
    \item 〔绚烂〕光彩耀眼,色泽浓烈。绚:色彩华丽。烂:光彩明亮。
    \item 〔绵延〕接连不断。
    \item 〔矫健〕强壮有力。
    \item 〔桠〕成叉状的树枝。
    \item 〔洄水涡〕流水回旋形成的漩涡。洄:水回旋而流。
    \item 〔斑斓〕色彩错杂灿烂。
    \item 〔漫流〕漫出而流过。漫:水过满而外流出本来的水道,四处流淌。
\end{itemize}

\chapter{小英雄雨来}

\begin{large}
    
    一
    
    晋察冀边区的北部有一条还乡河,河里长着很多芦苇。河边有个小村庄。芦花开的时候,远远望去,黄绿的芦苇上好像盖了一层厚厚的白雪。风一吹,鹅毛般的苇絮就飘飘悠悠地飞起来,把这几十家小房屋都罩在柔软的芦花里。因此,这村就叫芦花村。十二岁的雨来就是这村的。
    
    雨来最喜欢这条紧靠着村边的还乡河。每到夏天,雨来和铁头、三钻儿,还有许多小朋友,好像一群鱼,在河里钻上钻下,藏猫猫,狗刨,立浮,仰浮。雨来仰浮的本领最高,能够脸朝天在水里躺着,不但不沉底,还要把小肚皮露在水面上。
    
    妈妈不让雨来耍水,怕出危险。有一天,妈妈见雨来从外面进来,光着身子,浑身被太阳晒得黝黑发亮。妈妈知道他又去耍水了,把脸一沉,叫他过来,扭身就到炕上抓笤帚。雨来一看要挨打了,撒腿就往外跑。
    
    妈妈紧跟着追出来。雨来一边跑一边回头看。糟了!眼看要追上了。往哪儿跑呢?铁头正赶着牛从河沿回来,远远地向雨来喊:“往河沿跑!往河沿跑!”雨来听出了话里的意思,转身就朝河沿跑。妈妈还是死命追着不放,到底追上了,可是雨来浑身光溜溜的像条小泥鳅,怎么也抓不住。只听见扑通一声,雨来扎进河里不见了。妈妈立在河沿上,望着渐渐扩大的水圈直发愣。
    
    忽然,远远的水面上露出个小脑袋来。雨来像小鸭子一样抖着头上的水,用手抹一下眼睛和鼻子,嘴里吹着气,望着妈妈笑。
    
    二
    
    秋天。
    
    爸爸从集上卖苇席回来,同妈妈商量:“看见了区上的工作同志,说是孩子们不上学念书不行,起码要上夜校。叫雨来上夜校吧。要不,将来闹个睁眼瞎。”
    
    夜校就在三钻儿家的豆腐房里。房子很破。教夜课的是东庄学堂里的女老师,穿着青布裤褂,胖胖的,剪着短发。女老师走到黑板前面,屋里嗡嗡嗡嗡说话声音立刻停止了,只听见哗啦哗啦翻课本的声音。雨来从口袋里掏出课本,这是用土纸油印的,软鼓囊囊的。雨来怕揉坏了,向妈妈要了一块红布,包了个书皮,上面用铅笔歪歪斜斜地写了“雨来”两个字。雨来把书放在腿上,翻开书。
    
    女老师斜着身子,用手指点着黑板上的字,念着:
    
    “我们是中国人,
    
    我们爱自己的祖国。”
    
    大家就随着女老师的手指,齐声轻轻地念起来:
    
    “我们——是——中国人,
    
    我们——爱——自己的——祖国。”
    
    三
    
    有一天,雨来从夜校回到家,躺在炕上,背诵当天晚上学会的课文。可是背不到一半,他就睡着了。
    
    不知什么时候,门吱扭响了一声。雨来睁开眼,看见闪进一个黑影。妈妈划了根火柴,点着灯,一看,原来是爸爸出外卖席子回来了。他肩上披着子弹袋,腰里插着手榴弹,背上还背着一根长长的步枪。爸爸怎么忽然这样打扮起来了呢?
    
    爸爸对妈妈说:“鬼子又‘扫荡’了,民兵都到区上集合,要一两个月才能回来。”雨来问爸爸说:“爸爸,远不远?”爸爸把手伸进被里,摸着雨来光溜溜的脊背,说:“这哪儿有准呢?说远就远,说近就近。”爸爸又转过脸对妈妈说:“明天你到东庄他姥姥家去一趟,告诉他舅舅,就说区上说的,叫他赶快把村里的民兵带到区上去集合。”妈妈问:“区上在哪儿?”爸爸装了一袋烟,吧嗒吧嗒抽着,说:“叫他们在河北一带村里打听。”
    
    雨来还想说什么,可是门哐啷响了一下,就听见爸爸走出去的脚步声。不大一会儿,什么也听不见了,只从街上传来一两声狗叫。
    
    第二天,吃过早饭,妈妈就到东庄去,临走说晚上才能回来。过了晌午,雨来吃了点剩饭,因为看家,不能到外面去,就趴在炕上念他那红布包着的识字课本。
    
    忽然听见街上咕咚咕咚有人跑,把屋子震得好像摇晃起来,窗户纸哗啦哗啦响。
    
    雨来一骨碌下了炕,把书塞在怀里就往外跑,刚要迈门槛,进来一个人,雨来正撞在这个人的怀里。他抬头一看,是李大叔。李大叔是区上的交通员,常在雨来家落脚。
    
    随后听见日本鬼子唔哩哇啦地叫。李大叔忙把墙角那盛着一半糠皮的缸搬开。雨来两眼楞住了,“咦!这是什么时候挖的洞呢?”李大叔跳进洞里,说:“把缸搬回原地方。你就快到别的院里去,对谁也不许说。”
    
    十二岁的雨来使尽气力,才把缸挪回到原地。
    
    雨来刚到堂屋,见十几把雪亮的刺刀从前门进来。他撒腿就往后院跑,背后咔啦一声枪栓响,有人大声叫道:“站住!”雨来没理他,脚下像踩着风,一直朝后院跑去。只听见子弹向他头上嗖嗖地飞来。可是后院没有门,把雨来急出一身冷汗。靠墙有一棵桃树,雨来抱着树就往上爬。鬼子已经追到树底下,伸手抓住雨来的脚,往下一拉,雨来就摔在地下。鬼子把他两只胳膊向背后一拧,捆绑起来,推推搡搡回到屋里。
    
    四
    
    鬼子把前后院都翻遍了。
    
    屋子里遭了劫难,连枕头都给刺刀挑破了。炕沿上坐着个鬼子军官,两眼红红的,
    
    用中国话问雨来说:“小孩,问你话,不许撒谎!”他突然望着雨来的胸脯,张着嘴,眼睛睁得圆圆的。
    
    雨来低头一看,原来刚才一阵子挣扎,识字课本从怀里露出来了。鬼子一把抓在手里,翻着看了看,问他:“谁给你的?”雨来说:“捡来的!”
    
    鬼子露出满口金牙,做了个鬼脸,温和地对雨来说:“不要害怕!小孩,皇军是爱护的!”说着,就叫人给他松绑。
    
    雨来把手放下来,觉得胳膊发麻发痛,扁鼻子军官用手摸着雨来的脑袋,说:“这本书谁给你的,没有关系,我不问了。别的话要统统告诉我!刚才有个人跑进来,看见没有?”雨来用手背抹了一下鼻子,嘟嘟囔囔地说:“我在屋里,什么也没看见。”
    
    扁鼻子军官把书扔在地上,伸手望皮包里掏。雨来心里想:“掏什么呢?找刀子?鬼子生了气要挖小孩眼睛的!”只见他掏出来的却是一把雪白的糖块。
    
    扁鼻子军官把糖往雨来手里一塞,说:“吃!你吃!你得说出来,他在什么地方?”他又伸出那个戴金戒指的手指,说:“这个,金的,也给你!”
    
    雨来没有接他的糖,也没有回答他。
    
    旁边一个鬼子嗖地抽出刀来,瞪着眼睛要向雨来头上劈。扁鼻子军官摇摇头。两个人唧唧咕咕说了一阵。那鬼子向雨来横着脖子翻白眼,使劲把刀放回鞘里。
    
    扁鼻子军官压住肚里的火气,用手轻轻地拍着雨来的肩膀,说:“我最喜欢小孩。那个人,你看见没有?说呀!”
    
    雨来摇摇头,说:“我在屋里,什么也没看见。”
    
    扁鼻子军官的眼光立刻变得凶恶可怕,他向前弓着身子,伸出两只大手。啊!那双手就像鹰的爪子,扭着雨来的两只耳朵,向两边拉。雨来疼得直咧嘴。鬼子又抽出一只手来,在雨来的脸上打了两巴掌,又把他脸上的肉揪起一块,咬着牙拧。雨来的脸立刻变成白一块,青一块,紫一块。鬼子又向他胸脯上打了一拳。雨来打个趔趄,后退几步,后脑勺正碰在柜板上,但立刻又被抓过来,肚子撞在炕沿上。
    
    雨来半天才喘过气来,脑袋里像有一窝蜂,嗡嗡地叫。他两眼直冒金花,鼻子流着血。一滴一滴的血滴下来,溅在课本那几行字上:
    
    “我们是中国人,
    
    我们爱自己的祖国。”
    
    鬼子打得累了,雨来还是咬着牙,说:“没看见!”
    
    扁鼻子军官气得暴跳起来,嗷嗷地叫:“枪毙,枪毙!拉出去,拉出去!”
    
    五
    
    太阳已经落下去。蓝蓝的天上飘着的浮云像一块一块红绸子,映在还乡河上,像开了一大朵一大朵鸡冠花。苇塘的芦花被风吹起来,在上面飘飘悠悠地飞着。
    
    芦花村里的人听到河沿上响了几枪。老人们含着泪,说:
    
    “雨来是个好孩子!死得可惜!”
    
    “有志不在年高。”
    
    芦花村的孩子们,雨来的好朋友铁头和三钻儿几个人,听到枪声都呜呜地哭了。
    
    六
    
    交通员李大叔在地洞里等了好久,不见雨来来搬缸,就往另一个出口走。他试探着推开洞口的石板,扒开苇叶,院子里空空的,一个人影也没有,四处也不见动静。忽然听见街上有人吆喝:“豆腐啦!卖豆腐啦!”这是芦花村的暗号,李大叔知道敌人已经走远了。
    
    可是雨来怎么还不见呢?他跑到街上,看见许多人往河沿跑,一打听,才知道雨来被鬼子打死在河里了!
    
    李大叔脑袋轰的一声,眼泪就流下来了。他一股劲儿地跟着人们向河沿跑。
    
    到了河沿,别说尸首,连一滴血也没看见。
    
    大家呆呆地在河沿上立着。还乡河静静的,河水打着漩涡哗哗地向下流去。虫子在草窝里叫着。不知谁说:“也许鬼子把雨来扔在河里,冲走了!”大家就顺着河岸向下找。突然铁头叫起来:“啊!雨来!雨来!”
    
    在芦苇丛里,水面上露出个小脑袋来。雨来还是像小鸭子一样抖着头上的水,用手抹一下眼睛和鼻子,扒着芦苇,向岸上的人问道:“鬼子走了?”
    
    “啊!”大家都高兴得叫起来,“雨来没有死!雨来没有死!”
    
    原来枪响以前,雨来就趁鬼子不防备,一头扎到河里去。鬼子慌忙向水里打枪,可是我们的小英雄雨来已经从水底游到远处去了。
    
\end{large}



\chapter{参观刘家峡水电站}

\begin{large}
    
    去年暑假,我去甘肃永靖县看爸爸。到那儿的第二天,爸爸就带我去参观刘家峡水电站。
    
    汽车沿着黄河岸边的公路行驶。我坐在车上,远远地望见一座银灰色的大坝,镶嵌在狭窄陡峭的山壁中间。爸爸说:“那就是水电站的拦河大坝。这座大坝有四十层楼房那么高,奔腾的黄河水进入峡谷,就被它拦腰截住了。”
    
    下了汽车,我们登上大坝。坝顶宽阔平坦,可以并排行驶四辆卡车。站在坝顶,展现在眼前的是一个巨大的人工湖。碧绿的湖水映着蓝天白云,更显得清澈。爸爸指着坝底告诉我:湖水从大坝的进水口直冲下来,流入电机房底部,推动水轮机。水轮机不断运转,发电机就产生了强大的电流。电流通过高压输电线,输送到各地去。
    
    我正望着一条条伸向远方的高压输电线出神,爸爸说:“快看,泄洪道开闸了!”顿时,湖水如万马奔腾,倾泻直下,发出一阵阵轰鸣,掀起一团团水雾。我问爸爸:“泄洪道的闸门每天都开吗?水库里的水是要用来发电的,这样白白放掉,不可惜吗?”“水库有一定的容量,超过容量就要把多余的水放掉;要不,拦河坝就有被冲垮的危险。近来雨水多,所以常常要开闸放水。等到枯水期,就要闭闸蓄水了。”
    
    我们从坝顶乘电梯下了大坝,钻进水电站的心脏——电机房。电机房里灯火辉煌,五台绿色的大型发电机组,整齐地排列着。爸爸说:“这五台发电机给每年发的电,比解放前全国一年发的电还多。甘肃、青海、陕西等省广大城乡用的电,都是从这儿输送去的。”
    
    我们走出电机房,依依不舍地离开了刘家峡水电站。
    
\end{large}



\chapter{小站}

\begin{large}
    
    这是一个铁路线上的小站,只有慢车才停两三分钟。快车疾驰而过,旅客们甚至连站名还来不及看清楚。
    
    就在这一刹那,你也许看到一间红瓦灰墙的小屋,一排漆成白色的小栅栏,或者还有三五个人影。而这一切又立即消失了,火车两旁依然是逼人而来的山崖和巨石。
    
    这是一个在北方山区常见的小站。小屋左面有一张红榜,上面用大字标明了“241天安全无事故”的记录,贴着竞赛优胜者的照片。红榜旁边是一块小黑板,上面用白粉写着早晨广播的新闻和首都报纸摘要。出站口的旁边贴着一张讲卫生的宣传画。月台上,有两三个挑着箩筐的农民。几步以外,站上的两位工作人员正在商量着什么。
    
    月台中间有一个小小的喷水池,显然是经过精心设计的。喷水池中间堆起一座小小的假山,假山上栽着一棵尺把高的小树。喷泉从小树下面的石孔喷出来,水珠四射,把假山上的小宝塔洗得一尘不染。
    
    月台的两头种了几株杏树,花开得正艳,引来一群蜜蜂。蜜蜂嗡嗡地边歌边舞,点缀着这个宁静的小站。
    
    小站上没有钟,也没有电铃。站长吹一长声哨子,刚到站的火车跟着长啸一声,缓缓的离开小站,继续走自己的征途。
    
    这个小站坐落在山坳里。站在月台上向四周望去,只看到光秃秃的石头山,没有什么秀丽的景色。可是就在这儿,就在这个小站上,却出现了一股活泼的喷泉,几树灿烂的杏花。
    
    这喷泉,这杏花,给旅客们带来了温暖的春意。
    
\end{large}



\chapter{挑山工}

\begin{large}
    
    在泰山上,随处都可以碰到挑山工。他们肩上搭一根光溜溜的扁担,扁担两头的绳子挂着沉甸甸的货物。登山的时候,他们一只胳膊搭在扁担上,另一只胳膊随着步子有节奏地一甩一甩,使身体保持平衡。他们的路线是折尺形的,从台阶的左侧起步,斜行向上,登上七八级,到了台阶右侧,就转过身子,反方向斜行,到了左侧再转回来,每一次转身,扁担换一次肩。他们这样曲折向上登,才能使挂在扁担前头的东西不碰在台阶上,还可以省些力气。担了重物,如果照一般登山的人那样直上直下,膝盖会受不住的。但是路线曲折,就会使路线加长。挑山工登一次山,走的路程大约比游人多一倍。
    
    奇怪的是挑山工花的时间并不比游人多。你轻快地从他们身边越过,以为把他们甩在后边很远了。你在什么地方饱览壮丽的山色,或者在道边诵读凿在石壁上的古人的题句,或者在喧闹的溪流边洗脸洗脚,他们就会不声不响地从你身旁走过,悄悄地走到你的前头去了。等你发现,你会大吃一惊,以为他们是像仙人那样,是腾云驾雾赶上来的。
    
    有一次,我同几个画友去泰山写生,就遇到过这种情况。我们在山下买登山用的青竹杖,遇到一个挑山工,矮个子,脸儿黑黝黝的,眉毛很浓,大约四十来岁,敞开的白土布褂子中间露出鲜红的背心。他扁担一头拴着几张木凳子,另一头捆着五六个青皮西瓜。我们很快就越过了他。到了回马岭那条陡直的山道前,我们累了,舒开身子躺在一块被风吹得干干净净的大石头上歇歇脚。我们发现那个挑山工就坐在对面的草地上抽烟。随后,我们跟他差不多同时起程,很快就把他甩在后边了,直到看不见他。我们爬上半山的五松亭,看见在那株姿态奇特的古松下整理挑儿的正是他,褂子脱掉了,光穿着红背心,现出健美的黑黝黝的肌肉。我很惊异,走过去跟他攀谈起来,这个山民倒不拘束,挺爱说话。他告诉我,他家住在山脚下,天天挑货上山,干了近二十年,一年四季,一天一个来回。他说:“你看我个子小吗?干挑山工的,给扁担压得长不高,都是又矮又粗的。像您这样的高个儿干不了这种活儿,走起路晃悠!”他浓眉一抬,咧开嘴笑了,露出洁白的牙齿。山民们喝泉水,牙齿都很白。
    
    谈话更随便些了,我把心中那个不解之谜说了出来:“我看你们走得很慢,怎么反而常常跑到我们前头去了呢?你们有什么近道吗?” 他听了,黑生生的脸上显出一丝得意的神色。他想了想说:“我们哪里有近道,还不和你们是一条道?你们走得快,可是你们在路上东看西看,玩玩闹闹,总停下来呗!我们跟你们不一样。不像你们那么随便,高兴怎么就怎么。一步踩不实不行,停停住住更不行。那样,两天也到不了山顶。就得一个劲儿往前走。别看我们慢,走长了就跑到你们前边去了。你看,是不是这个理?”
    
    我心悦诚服地点着头,感到这山民的几句朴素的话,似乎包蕴着意味深长的哲理。我还没来得及细细体味,他就担起挑儿起程了。在前边的山道上,我们又几次超过了他;但是总在我们留连山色的时候,他又悄悄地超过了我们。在极顶的小卖部门前,我们又碰见了他,他已经在那里交货了。他憨厚地对我们点头一笑,好像在说:“瞧,我可又跑到你们前头来了!”
    
    从泰山回来,我画了一幅画——在陡直的似乎没有尽头的山道上,一个穿红背心的挑山工给肩头的重物压弯了腰,他一步一步地向上登攀。这幅画一直挂在我的书桌前,因为我需要它。
    
\end{large}



\chapter{可爱的草塘}

\begin{large}
    
    初到北大荒,我感到一切都不习惯。带去的几本书看完了,时间一长,觉得没意思。小丽好像看出了我的心思,笑嘻嘻地问我:“姐夫,待腻了吧?我带你去散散心好吗?”
    
    “上哪儿去?”
    
    “到野地里去。不过你得紧跟着我走,俺这儿狼可多啦!”
    
    我说:“去就去,你不怕,我还能怕?”
    
    说走就走。小丽挎着个篮子蹦蹦跳跳地在前边引路,不多时就来到草塘边上。这么大这么美的草塘,我还是第一次看到,走了进去就像置身于大海中一样,浪花翠绿翠绿的,绿得发光,绿得鲜亮,欢笑着,翻滚着,一层赶着一层涌向远方。仔细瞧那浪花,近处的呈现鲜绿色,远一点儿的呈翠绿色,再远的呈墨绿色,一层又一层,最后连成一片,茫茫的跟蓝天相接。
    
    我情不自禁地说:“这草塘真美啊!”
    
    ”那当然!‘棒打狍子瓢舀鱼,野鸡飞到饭锅里’,你听说过吧。可惜你来的不是时候。要是春天,小草刚发芽,河水刚开化,藏了一冬的鱼都从水底游上来了。开河的鱼,下蛋的鸡,肉最香不过了!今年春天给你们邮的鱼干,一点儿不掺假,都是我用瓢舀的。”
    
    看着小丽那自豪的模样,我故意逗她:“别光说美的,若是冬天呢,天天刮大风,冻得人出不去屋……”
    
    “冬天?冬天更好玩啦!穿得像个棉花包似的,戴上皮帽子、皮手套,提着根棍子到草塘里去逮野鸡,追狍子。天越冷越好,冻得野鸡连眼睛都睁不开。它冷极了就把头往雪里扎,你走到它跟前,像拔萝卜似的,一下就把它拔出来了。别看狍子跑得快,在雪地就不行了,腿陷在雪坑里再也拔不出来,眼睁睁地让人逮!”
    
    “哦,你这么一说,北大荒好得哪儿也比不上啦?”
    
    “就是哪儿也比不上!”
    
    “那你说说,现在怎么个好法?”
    
    “你自己看嘛!给你一说就没意思了。”小丽知道我又逗她,故意关上了话匣子。
    
    往前没走多远,就听到小丽喊:“快来呀,姐夫。”我跑到眼前,扒开草丛一看,是个不大的水泡子,水面上波光粼粼,仔细一看,挤挤挨挨的都是鱼。我不禁惊叫起来:“啊,这么多鱼!”连忙脱掉鞋袜,跳进没膝盖深的水里逮起来。筷子长的鲇鱼,手掌宽的鲫鱼,一条又一条不住地往岸上抛。小丽不住地往篮子里拾。我逮着逮着,忽然哗啦啦一阵水点儿落在我的脸上和身上。下雨了吗?我抬头一看,是小丽捣的鬼!她淘气地笑着:“你真是贪心不足哇,篮子都满了,再往哪儿装呀?”
    
    我恋恋不舍地上了岸。小丽问我:“你知道这鱼是哪儿来的吗?”
    
    “那还用问,有水就有鱼嘛!”
    
    “我是问你这里有河没有?”
    
    我举目四望,茫茫的一片草塘,哪里有什么河呀?小丽紧走几步,拨开眼前的芦苇。啊,一条清澈的小河奇迹般地出现在我的眼前,芦苇和蒲草倒映在清凌凌的河水里,显得更绿了;天空倒映在清凌凌的河水里,显得更蓝了;云朵倒映在清凌凌的河水里,显得更白了。
    
    我朝前紧走几步,想捧起这清凉的河水痛痛快快地洗一洗脸。但是我犹豫了,生怕弄坏了这一幅美好的画卷。
    
\end{large}



\chapter{雪猴}

\begin{large}
    
    我冒着霏霏细雨,乘独木舟过江。上岸,走进大峡谷,天已经黑了。吃罢饭,边防军连长有些神秘地对我说:“不久会有可爱的小朋友给你送礼物。”送礼物,谁呢?我想一定是哪个孩子。
    
    边防哨所从来没有上锁插门的习惯。敞着门,我睡得甜极了。第二天醒来,蒙眬中听到一点响动,睁眼一看,见桌子上多了一小把香蕉。忙起身追出去,见是一只毛色鲜亮的猴子,回头看了我一眼,还点点头,转过屋角,闪进丛林,蹿到树上消失了。
    
    这就是雪猴,与普通猴子的最大区别是躯体较为高大,鼻孔高傲地翻向天空。它们是边防军人的好朋友,见了穿军装的人就显得非常高兴。如果来了新客人,就会热情地送上一点森林中的礼物。
    
    我的窗前是一片茂密的森林,这就是雪猴的乐园。几天下来我发现,雪猴和边防军人相处得是那么友好。
    
    清早,嘹亮的号音伴来晨光,军人出操了。树林里,雪猴也按号音起身,攀枝欢跳,飞身跃林,开始做“早操”。军人到哨所旁边溪水畔洗脸刷牙,猴儿也一齐拥到水边,用爪子捧水抹抹脸,把脚趾伸到嘴里掏掏,那神态像是很认真。军人在场上操练,猴儿就蹲在枝头观赏,有时还咧嘴龇牙地狂叫,仿佛为军人鼓劲叫好。兴致来了,它们还会跳到地上,在一旁摹仿。一次,一个新战士从单杠上摔下来,首先奔上去的竟是猴儿。它们把那个战士围得严严实实,又蹦又跳,用它们特有的方式表示关切和同情。战士们上课,它们也会不远不近地学着战士们席地而坐,凝神听讲,只有身上不舒服时才用爪子搔一搔。
    
    一次,友人陪我到山下寨子里去采访。归来时,我们在一片青草地上休息,聊天。突然,猴王带领它的部下把我们团团围住。它们有的把头垂得很低,有的用爪子紧捂朝天鼻,有的甚至把鼻孔抵在肚子上;它们左蹦右跳,扯着嗓子乱叫。我正大惑不解时,友人忙拉我快走。他说这里气候变化莫测,常常一个时辰就可以出现阴晴雨雪、冰雹风霜多种天气。雪猴对这里的气候最敏感,刚才它们是向我们预报有大雨,催我们快走。果然,我们刚回到哨所,就下起暴雨,雨中还夹着冰雹。
    
    离开大峡谷哨所时,我才更加明白,为什么有的战士服役期满,不只与部队难以割舍,还舍不得那些雪猴。
    
\end{large}



\chapter{蟋蟀的住宅}

\begin{large}
    
    居住在草地上的蟋蟀,差不多和蝉一样有名。它有名不仅由于它的歌声,还由于它的住宅。
    
    别的昆虫大多在临时的隐蔽所藏身。它们的隐蔽所得来不费功夫,弃去毫不可惜。蟋蟀和它们不同,不肯随遇而安。它常常慎重地选择住址,一定要排水优良,并且有温和的阳光。它不利用现成的洞穴,它的舒适的住宅是自己一点一点挖掘的,从大厅一直到卧室。蟋蟀怎么会有建筑住宅的才能呢?它有特别好的工具吗?没有。蟋蟀并不是挖掘技术的专家。它的工具是那样柔弱,所以人们对它的劳动成果感到惊奇。
    
    在儿童时代,我到草地上去捉蟋蟀,把它们养在笼子里,用菜叶喂它们,是为了玩。现在为了研究蟋蟀,我又搜索起它们的巢穴来。
    
    在朝着阳光的堤岸上,青草丛中隐藏着一条倾斜的隧道,即使有骤雨,这里也立刻就会干的。隧道顺着地势弯弯曲曲,最多不过九寸深,一指宽,这便是蟋蟀的住宅。出口的地方总有一丛草半掩着,就像一扇门。蟋蟀出来吃周围的嫩草,决不去碰这一丛草。那微斜的门口,经过仔细耙扫,收拾得很平坦。这就是蟋蟀的平台。当四周很安静的时候,蟋蟀就在这平台上拨动它的琴弦。
    
    屋子的内部没什么布置,但是墙壁很光滑。 主人有的是时间,把粗糙的地方修理平整。大体上讲,住所是很简朴的,清洁、干燥,很卫生。假使我们想到蟋蟀用来挖掘的工具是那样简单,这座住宅真可以算是伟大的工程了。
    
    蟋蟀盖房子大多是在十月末,初寒袭人的时候。它用前足扒土,还用钳子搬掉较大的土块。它用强有力的后足踏地,后足上有两排锯齿。它倒退着将泥土耙到后面,铺成斜面。这就是整个工序了。
    
    一开始,工作进展得很快,蟋蟀轻易就钻到了土底下。如果感到疲劳,它就在未完工的家门口休息一会儿,头朝着外面,触须轻微地摆动。不大一会儿,它又进去,用钳子和“耙子”继续工作。我看了很久,蟋蟀休息的时间越来越长,我也看得有些不耐烦了。
    
    住宅的重要部分快完成了。洞已经挖了有两寸深,够宽敞的了。余下的是长时间的整修,今天做一点,明天做一点。这个洞可以随天气的变冷和它身体的增长而加深加阔。即使在冬天,只要气候温和,太阳晒到它住宅的门口,还可以看见蟋蟀从里面不断地抛出泥土来。
    
\end{large}


\newpage

\textbf{注释}:

\vspace{-1em}

\begin{itemize}
    \setlength\itemsep{-0.2em}
    \item 〔随遇而安〕在任何处境里都能安乐满足。
    \item 〔骤雨〕忽然发生、为时不长的降雨,也叫阵雨。
    \item 〔耙〕用于碎土、平地的农具。
    \item 〔隧道〕山中或地下凿出的通道。
\end{itemize}

\chapter{鲸}

\begin{large}
    
    不少人看过大象,都说大象是很大的动物。其实还有比象大得多的动物,那就是鲸。
    
    已知最大的鲸约有十六万公斤重,最小的也有两千公斤。我国捕获过一头四万公斤重的鲸,约十七米长,一条舌头就有十几头大肥猪那么重。它要是张开嘴,人站在它嘴里,举起手来还摸不到它的上腭,四个人围着桌子坐在它的嘴里看书,还显得很宽敞。
    
    鲸生活在海洋里,因为体形像鱼,许多人管它叫鲸鱼。其实它不属于鱼类,而是哺乳动物\footnote{〔哺乳动物〕胎生后代且用乳汁哺育幼崽的动物。人、牛、羊、马等都是哺乳动物。}。在很远的古代,鲸的祖先跟牛羊的祖先一样,都生活在陆地上。后来环境发生了变化,鲸的祖先生活在靠近陆地的浅海里。又经过了很长很长的年代,它们的前肢和尾巴渐渐变成了鳍\footnote{〔鳍〕鱼类等水生动物的运动器官,由薄膜和硬刺组成,起着推进、平衡及导向的作用。},后肢完全退化了,整个身子成了鱼的样子,适应了海洋的生活。
    
    鲸的种类很多,总的来说可以分为两大类:一类是须鲸,没有牙齿;一类是齿鲸,有锋利的牙齿。
    
    鲸的身子这么大,它们吃什么呢?须鲸主要吃虾和小鱼。它们在海洋里游的时候,张着大嘴,把许多小鱼小虾连同海水一齐吸进嘴里,然后闭上嘴,把海水从须板中间滤出来,把小鱼小虾吞进肚子里,一顿就可以吃两千多公斤。齿鲸主要吃大鱼和海兽。它们遇到大鱼和海兽,就凶猛地扑上去,用锋利的牙齿咬住,很快就能吃掉。有一种号称“海中之虎”的虎鲸,常常好几十头结成一群,围住一头三十多吨重的长须鲸,几个小时就能把它吃光。
    
    鲸用肺呼吸,也说明它不属于鱼类。鲸的鼻孔长在脑袋顶上,呼气的时候浮出海面,废气从鼻孔喷出来,形成一股水柱,就像花园里的喷泉一样;它在海面上吸足了气,再潜入水中。每隔一定的时间呼吸一次,也就是“喷潮”一次。不同种类的鲸,“喷潮”的水柱也不一样。须鲸的水柱是垂直的,又细又高;齿鲸的水柱是倾斜的,又粗又矮。有经验的渔民根据水柱就可以判断鲸的种类和大小。
    
    鲸每天都要睡觉。鲸睡觉的时候,总是找一个比较安全的地方,几头聚在一起,头朝里,尾巴向外,围成一圈,静静地浮在海面上。如果听到什么声响,它们立即四散游开。
    
    鲸是胎生\footnote{〔胎生〕幼体在母体内发育成熟后出生、离开母体的繁殖方式。胎:与母体形似,有一定能力的幼体。}的,长须鲸刚生下来就有十多米长,七千公斤重。它靠吃母鲸的奶,每天能长三十公斤到五十公斤,两三年就可以长成大鲸。鲸的寿命很长,一般可以活几十年到一百年。
    
\end{large}



\chapter{圆明园的毁灭}

\begin{large}
    
    圆明园的毁灭是中国文化史上不可估量的损失,也是世界文化史上不可估量的损失!
    
    圆明园在北京西北郊,是一座举世闻名的皇家园林。它由圆明园、万春园和长春园组成,所以也叫圆明三园。此外,还有许多小园,分布在圆明园东、西、南三面。众星拱月般环绕在圆明园周围。
    
    圆明园中,有金碧辉煌的殿堂,也有玲珑剔透的亭台楼阁;有象征着热闹街市的“买卖街”,也有象征着田园风光的山乡村野。园中许多景物都是仿照各地名胜建造的,如海宁的安澜园\footnote{〔安澜园〕浙江省海宁市盐官镇的园林,本来是南宋安化郡王的私园,明代万历年间重建。清代乾隆皇帝南巡时作为行宫,赐名安澜园。},苏州的狮子林\footnote{〔狮子林〕江苏省苏州市的园林,由元末僧人修建,明初大书画家倪云林曾参与修造并题诗作画,使狮子林成为佛门讲经说法、文人赋诗作画的胜地。},杭州西湖的平湖秋月\footnote{〔西湖〕浙江省杭州市西边的湖,中国历史文化胜地。“平湖秋月”是西湖的代表景观。};还有很多景物是根据古代文人的诗情画意建造的,如蓬岛瑶台\footnote{〔蓬岛瑶台〕圆明园景观。蓬岛:指蓬莱,传说中东海的仙山。瑶台:美玉砌成的楼台,传说中神仙居住的地方。}、武陵春色\footnote{〔先秦时代〕秦始皇统一中国之前的时代。}。园中不仅有民族建筑,还有西洋景观。漫步园内,有如漫游在天南海北,饱览着中外风景名胜;流连其间,仿佛置身在幻想的境界里。
    
    圆明园不但建筑宏伟,还收藏着最珍贵的历史文物:上自先秦时代的青铜礼器,下至唐、宋、元、明、清历代的名人书画和各种奇珍异宝。所以,它又是当时世界上最大的博物馆、艺术馆。此外,圆明园还藏有超过一万卷图书,包括历史、科技、哲学及艺术的稀世精美的著作。
    
    1860年10月6日,英法联军侵入北京\footnote{〔礼器〕古时祭祀行礼用的各种器物,比如钟、鼎、簋、琮等,不作实用。},闯进圆明园。他们把园内凡是能拿得动的东西,统统掠走;拿不动的,就用大车或牲口搬运;实在运不走的,就任意破坏、毁掉。为了销毁罪证,10月18日和19日,三千多名侵略者奉命在园内放火。大火连烧三天,烟云笼罩了整个北京城,园內三百多人被活活烧死。我国这一园林艺术的瑰宝、建筑艺术的精华,就这样化为一片废墟。
    
\end{large}


\newpage

\textbf{注释}:

\vspace{-1em}

\begin{itemize}
    \setlength\itemsep{-0.2em}
    \item 〔举世闻名〕全世界都知道名字,表示很有名。
    \item 〔不可估量〕无法估计衡量,形容重大。
    \item 〔名胜〕因景色优美出名的地方。胜:(景色)优美。
    \item 〔天南海北〕距离遥远的不同地方。也写作“天南地北”。
    \item 〔西洋〕大西洋两岸,泛指欧美国家。
    \item 〔哲学〕研究世间根本问题的学科。
    \item 〔稀世〕世间少有。形容珍贵。
    \item 〔任意〕按照自己的心意,不受约束。
    \item 〔瑰宝〕珍奇贵重而美丽的珠宝。泛指珍贵美好的东西。
\end{itemize}

\chapter{喂药}

\begin{large}
    
    汤姆转移了注意力,不再为心中的秘密所苦恼,他现在感兴趣的是另一件更重要的事情:贝基·撒切尔不来上学了。汤姆想了结这桩心事,可经过几天的内心斗争,回过神来,他发现自己正一个人伤心地围着她家转悠。
    
    她原来是生病了,可万一要是死了呢!想到这,他就心若死灰。什么打仗啦,当海盗呀,全都没了兴趣。美好的生活一去不复返,留下的尽是些烦恼。他收起铁环,球拍也被放到了一边,这些东西已经没用了,不再能带来快乐。
    
    最担心他的是他姨妈。她马上试着想用各种药来治疗他。波莉姨妈最爱研究各种养生疗法,什么骨相学\footnote{〔骨相学〕一种迷信。认为人的骨头形状和位置决定其运势和命运。},风水学\footnote{〔风水学〕一种迷信。认为人居住的房屋、家具的造型和位置影响其运势和命运。},饮食穿衣的禁忌,这一切胡话都被她当作至理名言。可是她自己从不生病,所以逮着谁生病了,就拿谁开刀。有趣的是,尽管养生杂志上的内容前后两期说得驴唇不对马嘴,她却从来没有发现过。她这人头脑简单,心思单纯,所以极容易上当受骗。于是,她带上胡话连篇的杂志和骗人的药,用比喻的说法,就是带上镰刀,骑上灰马,身后跟着魔鬼出发了。可在她眼里,带的是灵丹妙药,要治病救人的。
    
    时下,水疗法是个新玩意儿,正巧汤姆精神也不怎么样,这下可得了她的劲。早晨天一亮,她就把汤姆叫到外边,让他在木棚里站好,然后没头没脸地给他浇上一阵凉水。她还用毛巾使劲给汤姆擦身,让他清醒过来。接着她用湿床单包起汤姆,再盖上毯子,直捂得他大汗淋漓,洗净灵魂。
    
    用汤姆的话来说,就是“要让污泥秽水从每根毛管中流出来”。
    
    经过这番“好心”的折腾,孩子却更加忧郁、更加苍白、没精打采。于是乎,她又动用了热水浴、坐浴、淋浴,直至全身水浴法,但都无济于事。那孩子仍然看上去像口棺材,死气沉沉,于是她又往水里加了一点燕麦和治水泡的药膏。她还仔细合计着用药的量,每天拿些所谓的灵丹妙药,给他灌上一通。
    
    此时此刻的汤姆对这种等同“迫害”的治疗已经麻木了,老太太对此惊恐万状。她要不惜一切代价治好这麻木不醒的孩子。她头一回听说“发汗散”这个名词,现在就派上了用场。她马上买了一些,尝后觉得这下有救了。用这种药简直等于拿火烧人。她丢下水疗法和别的,一心把希望寄托在这药上。她给汤姆服了一汤匙药,然后万分焦虑地等着结果。果然见效了,汤姆不再麻木不仁了,她心头的大石终于落下来了。瞧那孩子,仿佛突然从梦中惊醒,劲头十足,就算老太太真地把他放在火上,也比不上他这阵子的劲头。
    
    汤姆觉得他该醒了。尽管姨妈的折腾让他觉得很有浪漫情调,但是缺少理智,花样多得让人眼花缭乱。他绞尽脑汁,终于想出一个解脱的计划:假称喜欢吃发汗散。于是他时不时地找姨妈要药吃,结果弄得她烦起来,最后她干脆让汤姆自己动手,爱拿多少就拿多少,不要再来烦她就行。要是换成希德,她完全可以放心,可这是汤姆,所以,她暗中注意药瓶的情况。她发现药瓶里的药越来越少,但她可想不到,汤姆正在客厅里用药补地板的裂缝。
    
    有一天,汤姆正在给裂缝“喂药”,这时,他姨妈养的那只黄猫彼得喵喵地叫着走过来,眼睛贪婪地盯着汤匙,好像是要尝一口。汤姆说:
    
    “彼得,你若不是真想要,就别要了。”可是彼得点点头,表示它确实想要。
    
    “你最好别弄错了。”彼得伸了伸爪子,拿定了主意要。
    
    “这可是你自找的。就这么一点儿,可别说我小气。你要是吃了觉得不对劲,别怨别人,只能怪你自己。”
    
    彼得并无异议。因此汤姆撬开猫的嘴,把发汗散灌了下去。彼得窜出两三码\footnote{〔码〕英制长度单位。一码约为0.91米。}远,狂叫着在屋里转来转去。它砰的一声撞在柜子上,碰翻了花瓶,弄得一塌糊涂。接着它昂起头,后腿着地,欢快地跳来跳去,按捺不住发出高兴的声音。随后,它又在屋里狂奔乱跑,所到之处,不是碰倒这个就是毁了那个。波莉姨妈进来时正好看见它在连翻筋斗。最后,它哇地大叫一声,从敞开的窗户一跃而出,把余下的花瓶也带了下去。老太太惊呆了,站在那儿,眼睛从镜框上方往外瞪着;而汤姆却躺在地板上,笑得喘不过气来。
    
    “汤姆,那猫到底出了什么毛病?”
    
    “我不知道,姨妈。”他喘着气说。
    
    “我还没见过这样的事情,它到底怎么了?”
    
    “我真的不知道,波莉姨妈。猫快活的时候不就是那个样子嘛。”
    
    “是那个样子的吗?”语气有点令汤姆生畏。
    
    “嗯,我想,就是那样的。”
    
    “你真的这么想?”
    
    “嗯,真的。”
    
    老太太弯下腰,汤姆焦虑万分地关注着。当他看出老太太的用意时,为时已晚,因为说明问题的那把汤匙已暴露在床下。老太太捡起汤匙,汤姆害怕了,垂下了眼皮。波莉姨妈一把揪住他的耳杂把他拽起来,还用顶针狠狠地敲他头,敲得砰砰响。
    
    “我的小祖宗,你干吗要这样对待那个可怜的家伙,它又不会说话?”
    
    “我是可怜它才给它吃药的。你瞧,它又没有什么姨妈。”
    
    “什么它没有姨妈!你在说什么,傻瓜!这有什么关系?”
    
    “关系多着呢。它要是有姨妈,那肯定会不考虑它的感受,给它灌药,非烧坏它的五脏六腑不可!”
    
    听到这,波莉姨妈突然感到一阵难受,后悔不已。汤姆的说法让她开了窍。猫受不了,那孩子不也同样受不了吗?她软下来,心里感到内疚。她眼睛有点湿润,手放在汤姆头上,亲切地说:
    
    “汤姆,我是为了你好。再说,汤姆,那药确实对你有好处。”
    
    汤姆抬起头,严肃地看着姨妈的脸,眨着眼睛说:“我的好姨妈,你是好意,这我晓得。我对彼得也是好意呀。那药对它也有好处。自我给它灌药以后,我连它的影子都看不到了。”
    
    “哦,去你的,汤姆。别再气我了。你就不能做个听话的孩子吗?哪怕是一次也行。这样的话,就不需要再用药了。”
    
\end{large}



\chapter{阁楼}

\begin{large}
    
    洛蒂是个小不点儿,撒拉的变化让她迷惑不解。她听到流言,说撒拉遇到了意外,但她无法理解她为什么变了样子,为什么换上了老旧的黑裙子,说话也少了。
    
    “撒拉,你现在很穷吗?”这天早上的法语课上,洛蒂悄悄问道,“你变成乞丐了吗?”
    
    她看上去要哭出来,撒拉连忙安慰她:“乞丐没有住的地方,我还有地方住。”
    
    “你住在什么地方?”洛蒂追问道,“那个新来的女孩睡在你的房间里。你睡在哪里?”
    
    “我睡在另外的房间。”撒拉说。
    
    “是个好房间吗?我要看看。”
    
    “别说话了,专心上课。”撒拉不想再说了。
    
    洛蒂是个有决心的小家伙,她一定要看看撒拉的新房间。撒拉不告诉她,她就自己想办法。她从大女孩的闲谈中知道了位置。有天傍晚,她四处寻找,爬上从前不知道的楼梯,来到了阁楼。她看到一扇紧闭的门,打开一看,她亲爱的撒拉站在一张旧桌子上,望着窗外。她惊呆了。
    
    “撒拉!”她喊道。这阁楼的房间空荡荡的,这样丑陋,跟楼下的世界格格不入。
    
    撒拉听见她的喊声,转过身来,也惊呆了。她跳下桌子,奔向洛蒂。
    
    “不要哭,洛蒂!”她恳求着,“否则我会挨骂的。这房间并不算太坏,洛蒂。”
    
    “不坏吗?”小女孩咬着嘴唇环顾四周,“为什么不算太坏?”声音轻得像耳语。
    
    撒拉紧抱着她,这娇小的身体里,传出让人欣慰的温暖。
    
    “你能看到许多楼下看不到的东西。”
    
    “什么样的东西?”洛蒂好奇地问道。
    
    “楼顶呀,有好多烟囱,冒出的烟像花环,又像云雾,一直升上天空去。楼顶上的麻雀,跳来跳去,还会吵架呢。其他的阁楼,也有窗子,随时都可能冒出人头来,让你猜不透,楼里发生了什么事。楼顶呀,是另一个世界。”
    
    “啊,让我看看!”洛蒂喊道,“把我抱上去!”
    
    洛蒂抱着她站到桌子上,扒着屋顶天窗的边缘往外望。
    
    这是一个不同的世界,未曾见过的人是无法想象的。两边屋顶的石板瓦斜斜往下展开,直到房檐的排水槽。麻雀在那儿做窝,啁啾着,蹦跳着,无所畏惧。其中两只栖在最近的烟囱边上,狠狠地争吵着,直到一只把另一只啄跑。相邻那扇阁楼窗子紧闭着,里面没有人住。
    
    “我希望有人住在那儿。”撒拉说,“如果有个小姑娘住在那儿,我们就可以把头探出窗子交谈。还可以爬过去和她见面——如果不怕摔下去的话。”
    
    比从街道上看去,天空似乎近得多。无数蓝色的屋顶仿佛海浪。从烟囱之间望下去,小小的行人和马车仿佛玩具,显得不真实。
    
    “这阁楼这样小,又高高在上。”撒拉说,“几乎像树上的鸟窝。这倾斜的天花板多有趣啊。瞧,这一端低得你几乎站不直。晨光来临,天窗里是粉红色的天空。下雨的时候,雨点打在窗玻璃上,好像在讲故事。要是有星星出现,可以数数窗子里有几颗,它能容纳的可不少呢。还有墙角那只生锈的小壁炉,要是把它擦亮,再点上火,你想想看该有多美。这儿的地板上,可以铺一块又厚又软的印度地毯\footnote{〔印度地毯〕当时印度为英国殖民地,是手工地毯的主要产地。}。那边的墙角,可以放一张柔软的小沙发。沙发上有圆圆的靠垫。沙发前面可以有一个放满书的小书架,伸手就能够着。壁炉前可以放一块\footnote{〔裘皮〕加工后的动物毛皮制品。}小地毯,墙上挂上帷幕和图画。边上可以有一盏灯,玫瑰色的灯罩。正中央放一张小桌子,上面摆着茶具。一把圆墩墩的小铜壶在壁炉上滋滋地响。床铺上铺着花边的丝绸床罩。麻雀看了也会啄啄窗子,要进来做客。”
    
    “撒拉啊!我喜欢阁楼!”洛蒂开心地笑了,“我喜欢这里,这里比楼下好!”
    
    撒拉说服她下楼,把她打发走了。回到阁楼,环顾四周,想象力的魔法已经消失了。床铺是硬木板,盖着破旧的被子。墙上的白灰掉了,地板是冰冷的,没有地毯。壁炉的格子生锈,断了。仅有的一张凳子,凳脚坏了,歪向一边。她坐了一会儿,低头伏在双手上。
    
    “这是个孤寂的地方,”她说,“这是世界上最孤寂的地方。”
    
\end{large}



\chapter{冀中的地道战}

\begin{large}
    
    1942到1944那几年,日本侵略军在冀中平原\footnote{〔冀中平原〕河北省中部平原,西起平汉路,东至津浦路,北临京津,南至仓石路。}上“大扫荡”,还修筑了封锁沟和封锁墙,十里一碉,八里一堡,想搞垮我们的人民武装。
    
    为了粉碎敌人的“扫荡”,冀中人民在中国共产党的领导下,创造了新的斗争方式,这就是地道战。
    
    说起地道战,简直是个奇迹。在广阔平原的地底下,挖了不计其数的地道,横的,竖的,直的,弯的,家家相连,村村相通。敌人来了,我们就钻到地道里去,让他们扑个空;敌人走了,我们就从地道里出来,照常种地过日子,有时候还要打击敌人。靠着地道这种坚强的堡垒,冀中平原上的人民坚持了敌后游击战争。
    
    地道的式样有一百多种。就拿任丘\footnote{〔任丘〕河北中部县级市,位于沧州市西北。}的地道来说吧,村里的地道挖在街道下面,跟别村相通的地道挖在庄稼地下面。地道有四尺多高,个高的人弯着腰可以通过;地道的顶离地面三尺,不妨碍上面种庄稼。地道里每隔一段距离就有一个大洞,洞顶用木料撑住,很牢靠。大洞四壁又挖了许多小洞,有的住人,有的拴牲口,有的搁东西,有的做厕所。一个大洞容得下一百来人,最大的能容二百多人。洞里经常准备着开水、干粮、被子、灯火,在里面住上个三五天,不成问题。洞里有通到地面的气孔,从气孔里还能漏下光线来。气孔的口子都开在隐蔽的地方,敌人很难发现。人藏在洞里,既不气闷,又不嫌暗。有的老太太把纺车也搬进来,还嗡嗡嗡地纺线呢。
    
    地道的出口也开在隐蔽的地方,外面堆满荆棘。有的还在旁边挖一个陷坑,坑里插上尖刀或者埋上地雷,上面用木板虚盖着,板上铺些草,敌人一踏上去就翻下坑里送了命。在地道里,离出口不远的地方挖几个特别坚固的洞,民兵拿着武器在洞里警戒;拐弯的地方挖一些岔道,叫“迷惑洞”,敌人万一进来了,分不清哪条是死道,哪条是活道。进了死道,就有地雷埋在那儿等着他们;就算进了活道,他们也过不了关口。原来地道里每隔一段就有个很窄的“孑口”,只能容一个人爬过去。只要一个人拿一根木棒,就可以把“孑口”守住,真是“一夫当关,万夫莫开”。
    
    敌人尝到了地道的厉害,想方设法来破坏,什么火攻啊,水攻啊,毒气攻啊,都用遍了。大家又想出了许多妙法来防备。洞口准备着土和沙,可以用来灭火。“孑口”上装着吊板,如果敌人放毒气,就把吊板放下来挡住,不让毒气往里透。对付水攻的法子更妙了,把地道跟枯井暗沟连接起来,敌人放水的时候,水从洞口进来,就流到枯井暗沟里去了。任敌人想出什么毒辣的法子也不怕,因为各个村子的地道是相通的,大不了转移到旁的村子去。
    
    人在地道里怎么能了解地面上的情况呢?民兵的指挥部派出一些人分布在各处,发现了敌情就吆喝起来,一个接一个,一直传到指挥部里。老百姓管这种吆喝叫“无线电”。地道里面可就用“有线电”了,一根铁丝牵住一个小铜铃,这儿一拉,那儿就响,拉几下表示什么意思是早就约好了的。为了打击敌人,什么办法都想出来,人民的智慧是无穷无尽的。
    
    有了地道战这个斗争方式,敌人毒辣透顶的“扫荡”被粉碎了。冀中平原上的人民不但坚持了生产,还有力地打击了敌人,在我国抗日战争史上留下了惊人的奇迹。
    
\end{large}



\chapter{田忌赛马}

\begin{large}
    
    齐国的大将田忌很喜欢赛马。有一回他和齐威王约定,进行一次比赛。
    
    他们把各自的马分成上、中、下三等。比赛的时候,上等马对上等马,中等马对中等马,下等马对下等马。由于齐威王每个等级的马都比田忌的强,三场比赛下来,田忌都失败了。田忌觉得很扫兴,垂头丧气地准备离开赛马场。
    
    这时,田忌发现,他的好朋友孙膑也在人群里。孙膑招呼田忌过来,拍着他的肩膀,说:“从刚才的情形看,齐威王的马比你的快不了多少呀……”孙膑还没有说完,田忌瞪了他一眼,说:“想不到你也来挖苦我!”
    
    孙膑说:“我不是挖苦你,你再同他赛一次,我有办法让你取胜。”
    
    田忌疑惑地看着孙膑:“你是说另换几匹马?”
    
    孙膑摇摇头,说:“一匹也不用换。”
    
    田忌没信心地说:“那还不是照样输!”
    
    孙膑胸有成竹地说:“你就照我的主意办吧。”
    
    齐威王正在得意扬扬地夸耀自己的马,看见田忌和孙膑过来了,便讥讽田忌:“怎么,难道你还不服气?”
    
    田忌说:“当然不服气,咱们再赛一次!”
    
    齐威王轻蔑地说:“那就来吧!”
    
    一声锣响,赛马又开始了。
    
    孙膑让田忌先用下等马对齐威王的上等马,第一场输了。
    
    接着进行第二场比赛。孙膑让田忌拿上等马对齐威王的中等马,胜了第二场。齐威王有点心慌了。
    
    第三场,田忌拿中等马对齐威王的下等马,又胜了一场。这下,齐威王说不出话来了。
    
    比赛结果,田忌胜两场输一场,赢了齐威王。
    
    还是原来的马,只掉换了一下出场顺序,就可以转败为胜。
    
\end{large}



\chapter{记金华的双龙洞}

\begin{large}
    
    4月14日,我在浙江金华,游北山的双龙洞。
    
    出金华城大约五公里到罗店,过了罗店就渐渐入山。公路盘曲而上。山上开满了映山红,无论花朵还是叶子,都比盆栽的杜鹃显得有精神。油桐也正开花,这儿一丛,那儿一簇,很不少。山上沙土呈粉红色,在别处似乎没有见过。粉红色的山,各色的映山红,再加上或浓或淡的新绿,眼前一片明艳。
    
    一路迎着溪流。随着山势,溪流时而宽,时而窄,时而缓,时而急,溪声也时时变换调子。入山大约五公里就来到双龙洞口,那溪流就是从洞里出来的。
    
    在洞口抬头望,山相当高,突兀森郁,很有气势。洞口像桥洞似的,很宽。走进去,仿佛到了个大会堂,周围是石壁,头上是高高的石顶,在那里聚集一千或是八百人开个会,一定不觉得拥挤。泉水靠着洞口的右边往外流。这是外洞。
    
    在外洞找泉水的来路,原来从靠左边的石壁下方的孔隙流出。虽说是孔隙,可也容得下一只小船进出。怎样小的小船呢?两个人并排仰卧,刚合适,再没法容第三个人,是这样小的小船。船两头都系着绳子,管理处的工人先进内洞,在里边拉绳子,船就进去,在外洞的工人拉另一头的绳子,船就出来。我怀着好奇的心情独个儿仰卧在小船里,自以为从后脑到肩背,到臀部,到脚跟,没有一处不贴着船底了,才说一声“行了”,船就慢慢移动。眼前昏暗了,可是还能感觉左右和上方的山石似乎都在朝我挤压过来。我又感觉要是把头稍微抬起一点儿,准会撞破额角,擦伤鼻子。大约行了两三丈的水程吧,就登陆了,这就到了内洞。
    
    内洞一团漆黑,什么都看不见。工人提着汽油灯,也只能照见小小的一块地方,余外全是昏暗,不知道有多么宽广。工人高高举起汽油灯,逐一指点洞内的景物。先看到的是蜿蜒在洞顶的双龙,一条黄龙,一条青龙。我顺着他的指点看,有点儿像。其他那些石钟乳和石笋,这是什么,那是什么,大都依据形状想象成神仙、动物以及宫室、器用,名目有四十多。这些石钟乳和石笋,形状变化多端,再加上颜色各异,即使不比作什么,也很值得观赏。
    
    在洞里走了一转,觉得内洞比外洞大得多,大概有十来进房子那么大。泉水靠着右边缓缓地流,声音轻轻的。上源在深黑的石洞里。
    
    我排队等候,又仰卧在小船里,出了洞。
    
\end{large}


\newpage

\textbf{注释}:

\vspace{-1em}

\begin{itemize}
    \setlength\itemsep{-0.2em}
    \item 〔突兀〕高高耸起。
    \item 〔森郁〕草木繁盛茂密。比喻幽暗阴冷的感觉。
\end{itemize}

\chapter{丰碑}

\begin{large}
    
    一支长长的红军队伍,在云中山的冰天雪地里,顶着混沌迷蒙的飞雪前进。严寒把云中山冻成了一个大冰坨,狂风像狼似的嗥叫着,要征服这支装备很差的队伍。
    
    将军的马,早已让给了伤号骑。将军和战士们一道踏着冰雪行军。他不时被寒风呛得咳嗽着。他要率领这支队伍向前挺进,为后续部队开辟一条通道。等待他们的将是十分恶劣的环境和十分残酷的战斗,可能三天两头吃不上饭,可能要睡雪窝,可能一天要走一百几十里路,可能……哦,可能太多了,这支队伍的素质怎么样呢?能不能经受住严峻的考验?
    
    将军思索着……
    
    前面的队伍忽然放慢了行军的速度,有许多人围在一起,不知干什么。
    
    将军边走边喊:不要停下来,快速前进!
    
    将军的警卫员回来告诉他:“……前面……冻死了一个人……”
    
    将军愣了愣,什么话也没说,朝那边走去。风雪太大了。他步履有些踉跄,眼睛有点迷离。
    
    一个冻僵的老战士,倚靠一棵光秃秃的树干坐着,一动也不动,好似一尊塑像。他浑身都落满了雪,可以看出镇定、自然的神情,却一时无法辨认面目,半截带卷的旱烟还夹在右手的中指和食指间,烟火已被风雪打熄。他微微向前伸出手来,好像要向战友借火……“怎么?他的衣服这么单薄、破旧?像树叶,像箔片一样薄薄地贴在身上……他的御寒衣物呢?为什么没有发下来?
    
    将军的脸上顿时阴云密布,嘴角边的肌肉明显地抽动了一下,蓦然转过头向身边的人吼道:“叫军需处长来,老子要……”一阵风雪吞没了他的话。他红着眼睛,像一头发怒的豹子,样子十分可怕。
    
    没有人回答他,也没有人走开……
    
    “听见没有?警卫员!快叫军需处长跑步上来!”将军两腮的肌肉大幅度地抖动着,不知是由于冷,还是由于愤怒。
    
    终于,有什么人对将军小声地说了一声:“这就是军需处长……”
    
    将军就要发火的手势突然停住了。他怔怔地伫立了足有一分钟。雪花无声地落在他的脸上,溶化成闪烁的泪珠……他深深地呼出了一口气,缓缓地举起了右手,举至齐眉处,向那位与云中山化为一体的牺牲者敬了一个庄严的军礼……
    
    雪更大了,风更狂了。大雪很快覆盖了军需处长的身体,他变成了一座晶莹的碑……
    
    将军什么话也没说,大步地钻进了弥天的风雪之中,他听见无数沉重而又坚定的脚步声在说:“如果胜利不属于这样的队伍,还会属于谁呢?”
    
\end{large}


\newpage

\textbf{注释}:

\vspace{-1em}

\begin{itemize}
    \setlength\itemsep{-0.2em}
    \item 〔嗥叫〕(豺狼熊虎等野兽)高声吼叫。
    \item 〔踉跄〕形容走路不稳,跌跌撞撞。
    \item 〔迷离〕视线模糊,看不清楚。
    \item 〔伫立〕久立,一直站着不动。
    \item 〔箔片〕通常为纯金属的薄片,通常和纸一样薄。
    \item 〔弥天〕漫天、满天。弥:布满。
\end{itemize}

\chapter{镜泊湖奇观}

\begin{large}
    
    相传很久以前,牡丹江畔住着一个美丽善良的红罗女。她有一面宝镜。哪里的人们有苦难,她只要用宝镜一照,便可以消灾弭祸。这件事传到了天庭,引起了王母娘娘的忌妒,她派天神盗走了宝镜。红罗女上天索取,发生了争执,宝镜从天上掉了下来,就变成了镜泊湖。这当然是神话。虽说镜泊湖不是神仙宝物,也不能为人消灾袪病,不过,镜泊湖夏季凉爽少风,湖面平滑如镜,倒是事实。
    
    镜泊湖位于黑龙江省宁安县境内。约在一万年前,这里火山喷发,炽热的岩流阻塞了牡丹江的河道,于是水面被抬高,形成了湖泊。
    
    镜泊湖景色的最大特点,是自然朴实而又绮丽多变。除了镜泊山庄有一些精致的别墅外,这里没有多少人工的点缀,只有峭拔的山岩,清澈的湖水,缤纷的花树,一望无际的林海。然而它并不单调:四周峰峦叠起,湖心石岛耸峙,湖中倒影奇幻,真是美不胜收。
    
    吊水楼瀑布是镜泊湖的著名景点,位于湖水泻入牡丹江的地方。瀑布宽四十三米,高二十五米。底部岩石由于上万年激流的冲击,被蚀成了几十米的深潭。本来清澈的湖水静静地淌着,一到陡崖,突然下跌,顿时抛撒万斛珍珠,溅起千朵银花,水雾弥漫,势如千军万马,声闻数里,同幽静的镜泊湖形成鲜明的对照。这里地处北国,冬季气温低达零下三四十摄氏度,但瀑布却从不结冰断流。
    
    镜泊湖附近有一处地下森林。所谓地下森林,实际上是长在火山口里的森林。这里有七个火山口,其中最大的一个直径约五百米,深约一百米,壁陡底平,景色壮丽。由于火山长期没有喷发,火山岩逐渐风化,同火山灰及沉积的浮尘积聚混合,形成了富含钾、磷等元素的肥沃土壤,加上这里降水较多,火山口的东南方向有缺口,阳光可以射入,所以长起了郁郁葱葱的森林。林中有红松、白桦、水曲柳、胡桃楸等树木,还有许多名贵的药材。东北虎、熊、青羊、马鹿等野生动物,也常到火山口活动。游客们爬上火山口的顶部俯视,只见足下峭壁如屏,黝黑的火山口似乎要吞没一切,令人心惊。可是底下的林木却不在乎这谷底的阴暗潮湿,它们欣欣向荣,充满了活力。
    
\end{large}


\newpage

\textbf{注释}:

\vspace{-1em}

\begin{itemize}
    \setlength\itemsep{-0.2em}
    \item 〔弭〕平息,止。
    \item 〔袪〕驱逐,去除。
    \item 〔峭拔〕(山)高而陡。
    \item 〔峙〕直立,耸立。
    \item 〔斛〕容量单位。一斛十斗。
\end{itemize}

\chapter{伟大的友谊}

\begin{large}
    
    马克思和恩格斯是好朋友。他们共同研究学问,共同领导国际工人运动,共同办报、编杂志,共同起草文件。著名的《共产党宣言》就是他们共同起草的。
    
    马克思是共产主义理念的奠基人。他受反动政府的迫害,长期流亡在外,生活很穷苦。他常常跑当铺,把衣服当了钱买面包。由于到期付不出赊购贷物的欠款,他常常受杂货店老板的责备。有时候他为了寄一篇文章到报馆去,竟要借钱买邮票。生活这样窘困,马克思毫不在意,还是坚持进行他的研究工作和革命活动。那时候,恩格斯竭尽全力,在生活上给他很大的帮助。
    
    恩格斯曾经在曼彻斯特\footnote{〔曼彻斯特〕英国中部城市。}一家工厂里做过事。有一个时期,为了维持马克思的生活,他宁愿经营自己十分厌恶的商业,把挣来的钱分给马克思,十镑\footnote{〔镑〕英国货币单位,也称英镑。},一百镑,连续不断地给马克思汇去。
    
    在生活上,恩格斯热忱地帮助马克思,更重要的是在共产主义事业上,他们互相关怀,互相帮助,亲密地合作。
    
    他们同住在伦敦\footnote{〔伦敦〕英国首都,位于中南部。}的时候,每天下午,恩格斯总要到马克思家里去。他们讨论各种政治事件和科学问题,一连谈上好几个钟头,各抒己见,滔滔不绝,有时候还进行激烈的争论。天气晴朗的日子,他们就一起到郊外去散步。
    
    后来,马克思住在伦敦,恩格斯住在曼彻斯特\footnote{〔曼彻斯特〕英国北部城市。}。他们几乎每天通信,彼此交换对政治事件的意见和研究工作的成果。这些书信直到现在还保存着。
    
    马克思和恩格斯的互相关怀是无微不至的。他们时时刻刻设法给对方以帮助,都为对方在事业上的成就感到骄傲。马克思答应给一家英文报纸写通讯的时候,还没精通英文,恩格斯就帮他翻译,必要时甚至代他写。恩格斯从事著述的时候,马克思也往往放下自己的工作,编写其中的某些部分。马克思逝世的时候,他的伟大著作《资本论》还没最后完成。恩格斯毅然放下自己的研究工作,竭尽全力从事《资本论》最后两卷的出版工作。
    
    马克思和恩格斯合作了四十年,共同创造了伟大的马克思主义。在四十年里头,在向着共同目标的奋斗中,他们建立了伟大的友谊。
    
\end{large}


\newpage

\textbf{注释}:

\vspace{-1em}

\begin{itemize}
    \setlength\itemsep{-0.2em}
    \item 〔奠基〕打好基础。奠:安置,定。
    \item 〔赊购〕买东西时先不付钱,以后再付。
    \item 〔窘困〕穷困,困难。
    \item 〔各抒己见〕各自尽情表达自己的意见。抒:表达,发泄。
    \item 〔竭尽全力〕用尽所有的力气,形容做出最大努力。竭:尽。
    \item 〔热忱〕热烈真诚。
    \item 〔无微不至〕没有一处细微的地方不照顾到。形容关怀、照顾得非常细心周到。
    \item 〔毅然〕不犹豫地,坚决果断地。
\end{itemize}

\chapter{伏尔加河上的纤夫}

\begin{large}
    
    在辽阔的伏尔加河\footnote{〔伏尔加河〕俄罗斯西南部河流,全长3692公里,流入里海。}上,有一艘货船,因为是逆风行驶,所以帆没有张起来。河面上映着倒影。一群穿着破烂的纤夫\footnote{〔纤夫〕为他人拉船的劳动者。},迈着沉重的步子,踏着黄沙,沿着河岸一步一步向前走。他们大多身子向前倾,可见都在使劲,可见船上载着很重的货物。
    
    领头的纤夫是个肩膀宽阔的老头儿。他包着头巾,衣服上打着补丁,眼睛漠然地望着前方,路还长着哩!老头儿的右边是一个头发胡须都很浓密的中年人。他身体强壮,显得很有力气。这两个人走在最前头。紧跟在他们后面的是个高个子,还保留着农民的打扮。他直着身子,没精打采地衔着烟斗,好像已经厌倦了拉纤的生活。高个子旁边是个肌肉结实的小伙子。他使着蛮劲向前拉,往上凝视的目光充满了诅咒和抗议。在这有群人中有个穿着红上衣的少年,从年龄和肤色都可以看出,他拉纤的日子还不久,还不习惯这种沉重的劳动。他拉了一下把他的肩膀勒得发疼的纤绳,好像要摆脱这种与他的年龄很不相称的重荷。少年右面的老头儿好像有病。他那微微张开的嘴唇和没精打采的目光,显得又虚弱又疲惫。他正在用袖子拭额上的汗珠。一个皮肤黝黑的汉子在少年后面,只露出半边脸。他用同情的目光看着前面白皙的少年和秃顶的老头儿。这个老头儿显然已经习惯于这种工作了,他一边往前走,一边装他的烟袋。后面是个退伍不久的士兵,还穿着官家发给他的皮靴。士兵后面是个高个子,他转过脸去,愤怒地朝货船上望,一定是货船的老板在咒骂他们,驱赶他们。走在最后的是一个神态沮丧的老头儿。他低着头,无可奈何地拖着沉重的步子,拼着命拉着纤绳往前迈步。
    
    这幅画是19世纪70年代俄国\footnote{〔俄国〕指俄罗斯,当时由沙皇统治,也称沙皇俄国。}画家列宾\footnote{〔列宾〕伊利亚·列宾,19世纪俄罗斯现实主义画家。}的作品。当时,俄国的劳动人民处在沙皇\footnote{〔沙皇〕对俄罗斯帝国君主的称呼。}的黑暗统治和资本家\footnote{〔资本家〕以资产为本,雇佣劳动者,赚取利润的人。}的残酷剥削之下,过着非常贫穷非常痛苦的生活。这幅画上的纤夫,为了挣得一块面包,不得不贱价出卖劳动力,终年拉着沉重的货船,在伏尔加河上来来去去。
    
\end{large}


\newpage

\textbf{注释}:

\vspace{-1em}

\begin{itemize}
    \setlength\itemsep{-0.2em}
    \item 〔辽阔〕非常广阔、宽广。
    \item 〔前倾〕向前偏斜。
    \item 〔衔〕咬住,叼。
    \item 〔漠然〕不关心;没有反应。
    \item 〔补丁〕补在破损衣物上,用来遮掩的片块。
    \item 〔没精打采〕情绪低落,提不起精神。
    \item 〔疲惫〕非常累。
    \item 〔诅咒〕祈求鬼神加祸给自己恨的人。现在一般指咒骂,希望对方没有好下场。
    \item 〔贱价〕不合理的、损害自己利益的低价。
    \item 〔重荷〕沉重的负荷。荷:负荷,承受的压力,以及产生压力的重物。
    \item 〔退伍〕军人退出军队。
    \item 〔官家〕指统治机构。
    \item 〔黝黑〕皮肤晒成的深色。
    \item 〔沮丧〕灰心、失望,情绪低落。
    \item 〔无可奈何〕没有办法,无能为力。
    \item 〔挣〕通过出力而获得。
    \item 〔残酷〕残忍冷酷。
    \item 〔剥削〕搜刮夺占。这里指使用劳动力而不给予公平的报酬。
\end{itemize}

\chapter{詹天佑}

\begin{large}
    
    詹天佑是我国杰出的爱国工程师。从北京到张家口这一段铁路,最早是在他的主持下修筑成功的。这是第一条完全由我国的工程技术人员设计施工的铁路干线。
    
    从北京到张家口的铁路长两百公里,是联结华北和西北的交通要道。当时,清政府刚提出修筑的计划,一些帝国主义国家就出来阻挠,他们都要争夺这条铁路的修筑权,想进一步控制我国的北部。帝国主义者谁也不肯让谁,事情争持了好久得不到解决。他们最后提出一个条件:清政府如果用本国的工程师来修筑铁路,他们就不再过问。他们以为这样一要挟,铁路就没法子动工,最后还得求助于他们。帝国主义者完全想错了,中国那时候已经有了自己的工程师,詹天佑就是其中一位。
    
    1905年,清政府任命詹天佑为总工程师,修筑从北京到张家口的铁路。消息一传出来,全国轰动,大家说这一回咱们可争了一口气。帝国主义者却认为这是个笑话。有一家外国报纸轻蔑地说:“能在南口以北修筑铁路的中国工程师还没有出世呢。”原来,从南口往北过居庸关到八达岭,一路都是高山深涧、悬崖峭壁。他们认为,这样艰巨的工程,外国著名的工程师也不敢轻易尝试,至于中国人,是无论如何也完成不了的。
    
    詹天佑不怕困难,也不怕嘲笑,毅然接受了任务,马上开始勘测线路。哪里要开山,哪里要架桥,哪里要把陡坡铲平,哪里要把弯度改小,都要经过勘测,进行周密计算。詹天佑经常勉励工作人员,说:“我们的工作首先要精密,不能有一点儿马虎。‘大概’‘差不多’这类说法不应该出自工程人员之口。”他亲自带着学生和工人,扛着标杆,背着经纬仪,在峭壁上定点,测绘。塞外常常狂风怒号,黄沙满天,一不小心还有坠入深谷的危险。不管条件怎样恶劣,詹天佑始终坚持在野外工作。白天,他攀山越岭,勘测线路;晚上,他就在油灯下绘图、计算。为了寻找一条合适的线路,他常常请教当地的农民。遇到困难,他总是想:这是中国人自己修筑的第一条铁路,一定要把它修好;否则,不但惹外国人讥笑,还会使中国的工程师失掉信心。
    
    铁路要经过很多高山,不得不开凿隧道,其中数居庸关和八达岭两条隧道的工程最艰巨。居庸关山势高,岩层厚,詹天佑决定采用从两端同时向中间凿进的办法。山顶的泉水往下渗,隧道里满是泥浆。工地上没有抽水机,詹天佑就带头挑着水桶去排水。他常常跟工人们同吃同住,不离开工地。八达岭隧道长一千一百多米,有居庸关隧道的三倍长。他跟老工人一起商量,决定采用中部凿井法,先从山顶往下打一口竖井,再分别向两头开凿,外面两端也同时施工,把工期缩短了一半。
    
    铁路经过青龙桥附近,坡度特别大。火车怎样才能爬上这样的陡坡呢?詹天佑顺着山势,设计了一种“人”字形线路。北上的列车到了南口就用两个火车头,一个在前边拉,一个在后边推。过青龙桥,列车向东北前进,过了“人”字形线路的岔道口就倒过来,原先推的火车头拉,原先拉的火车头推,使列车折向西北前进。这样一来,火车上山就容易得多了。
    
    京张铁路不满四年就全线竣工了,比计划提早两年。这件事给了藐视中国的帝国主义者一个有力的回击。今天,我们乘火车去八达岭,过青龙桥车站,可以看到一座铜像,那就是詹天佑的塑像。
    
\end{large}



\chapter{东郭先生与狼}

\begin{large}
    
    东郭先生牵着毛驴在路上走。毛驴驮着个口袋,口袋里装着书。
    
    忽然从后面跑来一只狼,慌慌张张地对他说:“先生,救救我吧!有猎人要杀我。他们带着老鹰和猎犬,骑着马,拿着弓箭,就快追上我了。让我在你的口袋里躲一躲吧。躲过了这场灾难,我永远忘不了你的恩情。”
    
    东郭先生犹豫了一下,他想到师父对自己说过,要像爱自己一样爱别人,看着狼可怜的样子,就答应了狼的要求。他倒出口袋里的书,把狼往口袋里装。可是口袋毕竟不大,狼的身子很长,装来装去,怎么也装不下。
    
    猎人越来越近了,已经听到马蹄声了。狼很着急,它说:“先生,求求你快一点儿!猎人一到,我就完了。”说着就躺在地上,并拢四条腿,把身子紧紧蜷成一团,头贴着尾巴,叫东郭先生用绳子把它捆住。东郭先生把狼捆好,塞进口袋,又装上了书,扎紧了袋口。他把口袋放到驴背上,继续往前走。
    
    猎人追上来了,是晋国的大夫赵简子,带着一大群随从。他找不着狼,就问东郭先生:“你看见一只狼没有?它往哪里跑了?”东郭先生犹豫了一下,说:“我没看见狼。这儿岔道\footnote{〔岔道〕一条路分成的多条路。}多,它也许从岔道上逃走了。”
    
    赵简子怒了,说:“狼明明是往这里来的,你怎么会看不见?你可别骗我。把狼放跑了,我饶不了你!”东郭先生说:“我也知道狼不是好东西,还会吃人,怎么会帮助狼呢?我自己也迷路了,实在不清楚狼在哪里。”
    
    赵简子走了,越走越远,听不到马蹄声了。狼在口袋里说:“先生,我可以出去了。”东郭先生就把它放了出来。狼伸伸腰,舔舔嘴,对东郭先生说:“我现在饿得很,先生,如果找不到东西吃,我一定会饿死的。先生既然救了我,就把好事做到底,让我吃了你吧!”
    
    东郭先生大吃一惊,只得绕着毛驴躲避,嘴里不住地骂着:“你这没良心的东西!我帮了你,你怎么能害我?”狼说:“这有什么不对的?要不,我们找人来评评理。”
    
    东郭先生看到路边有一棵老树,就说:“树啊树!你也见到了,我救了这狼,他却要吃我。请你给我评评理。”
    
    老树开口说道:“我是一棵杏树。当初主人种下我的时候,我只是一颗果核。我长到今天已经二十年了。年年路过的人都吃我的果实。主人还把我的果实拿到集市上卖钱。可现在我老了,主人却要把我砍下来做木材烧了。你给狼的东西有多少,要狼不吃你?”
    
    东郭先生逃到路旁,看到田里有一头老牛,就说:“牛啊牛!我救了这狼,他却要吃我。请你给我评评理。”
    
    老牛开口说道:“我从小就被农夫买来,要我拉犁耕田。田里收获多了,农夫吃得饱,穿得好。我给他卖了十年的力气,现在他见我老了,却想把我宰了吃肉。你有什么功劳,要狼不吃你?”
    
    正在危急的时候,有个老农扛着锄头走过来。东郭先生急忙上前拉住老农,把事情的经过告诉了他,说:“我救了这狼,他却要吃我。请你给我评评理。”
    
    老农说:“我听说,一滴水的恩情,要用一口泉来报答。东郭先生救了狼的命,狼怎么能恩将仇报呢?”
    
    狼急了,说:“他刚才捆住我的腿,把我装进口袋,还压上了好多书,把袋口扎得紧紧的。这哪里是救我,分明是想闷死我。这样的坏人,我不该吃吗?”
    
    老农想了想,说:“你们的话,我一点儿也不信。口袋那么小,装得下一只狼吗?我得看一看,狼是怎样装进去的。”
    
    狼同意了。它又躺下来蜷成一团,并拢四条腿,头贴着尾巴。东郭先生正准备再往口袋里装书,老农立即抢过去,把袋口扎得紧紧的。他对东郭先生说:“对狼讲兼爱,你真是太糊涂了,应该记住这个教训。”说着,他抡起锄头,把狼打死了。
    
\end{large}


\newpage

\textbf{注释}:

\vspace{-1em}

\begin{itemize}
    \setlength\itemsep{-0.2em}
    \item 〔驮〕背。
    \item 〔蜷〕身体弯屈。
    \item 〔恩情〕好处。
    \item 〔犹豫〕拿不了主意,下不了决定,不知道该怎么办。
    \item 〔捆〕用绳子绕圈扎起来。
    \item 〔猎犬〕打猎用的狗。
    \item 〔恩将仇报〕用仇恨回报恩情。
    \item 〔评理〕讲道理,评判是非对错。
    \item 〔糊涂〕认识不清楚,没想明白。
    \item 〔抡〕用力旋动手臂。
    \item 〔兼爱〕像爱自己一样爱其他人。
\end{itemize}

\chapter{瑞雪图}

\begin{large}
    
    从辽远的西伯利亚\footnote{〔西伯利亚〕亚洲北部地区,冬季形成寒流南下,进入我国。}袭来的寒流,侵入了胶东半岛\footnote{〔胶东半岛〕山东省东部临海地区,包括青岛市、烟台市、威海市等。}。
    
    连日来,暖和得如同三月阳春的天气,骤然变得冷起来了。一清早,天空布满了铅色的阴云。中午,凛冽的寒风刮起来了。寒风呼呼地刮了整整一个下午。黄昏时分,风停了,那鹅毛般的大雪,纷纷扬扬地从半空中降落下来了。
    
    这是入冬以来胶东半岛上的第一场雪。这雪下得很大,也很稳。开始的时候,还伴着一阵小雨。不久,雨住了,风停了,就只有那大片大片的雪花,从彤云密布的天空中,簌簌落落地飘将下来。一会儿,地面上就发白了。夜里,冬天的山村,万簌俱寂,只听到那大雪不断降落的沙沙声和树木的枯枝被积雪压断了的咯吱声。
    
    大雪整整下了一夜。第二天早晨,天放晴了,太阳出来了。推开门一看,嗬!好大的雪啊!那山川、河流、树木、房屋,都笼罩上一层白茫茫的厚雪。极目远眺,万里江山变成了一个粉妆玉砌的世界。看近处,那些落光了叶子的树木上,挂满了毛茸茸、亮晶晶的银条儿,那些冬夏常青的松树和柏树上,挂满了蓬松松、沉甸甸的雪球儿。一阵风吹来,树木轻轻地摇晃着,那美丽的银条儿和雪球儿簌簌落落地抖落下来。玉屑似的雪末儿随风飘扬,在清晨的阳光下,幻映出一道道五光十色的彩虹。
    
    大街上,积雪足有一尺深。人在雪地上走着,脚下就发出咯吱咯吱的响声。一群群孩子,在雪地里堆雪人,掷雪球。那欢乐的叫喊声、嘻闹声,把树枝上的雪都震落下来了。
    
    啊!好一幅北国瑞雪图!
    
\end{large}


\newpage

\textbf{注释}:

\vspace{-1em}

\begin{itemize}
    \setlength\itemsep{-0.2em}
    \item 〔寒流〕寒冷的气流。
    \item 〔骤然〕突然。
    \item 〔凛冽〕寒冷刺骨。
    \item 〔掷〕投,扔。
    \item 〔彤云密布〕布满乌云。彤:红色。
    \item 〔瑞雪〕民间有“瑞雪兆丰年”的说法。瑞:好预兆。
\end{itemize}

\chapter{彭德怀速写}

\begin{large}
    
    “一到战场上,我们便只有一个信心,几十个人的精神注在他一个人身上,谁也不敢乱动;就是刚上火线的,也因为有他在而不害怕。只要他一声命令‘去死!’我们就迎着看不见的死,勇猛地冲上去!找不到一个人不高兴的。我们是怕他的,但我们更爱他!”
    
    这是一个二十四岁的青年政治委员\footnote{〔政治委员〕中国共产党在红军各级组织中的政治代表。}告诉我的。当他讲这一段话的时候,发红的脸上隐藏不住他的兴奋。他说的是谁呢?就是现在我所要粗粗画几笔的彭德怀同志,他现在正在前方担任红军\footnote{〔红军〕指中国工农红军。中国土地革命战争时期,中国共产党领导的人民军队。}的前敌副总指挥\footnote{〔前敌副总指挥〕红军设有前敌总指挥部,后改为八路军总指挥部。朱德为总指挥,彭德怀为副总指挥。}。
    
    穿的是最普通的红军装束,在灰色布的表面上,薄薄浮着一层黄的泥灰和黑色的油,显得很旧,而且不大合身,不过他似乎从来都没有感觉到。脸色是看不清的,因为常常有许多被寒风吹裂的小口布满着,但在这不算漂亮的脸上有两个黑的、活泼的眼珠转动,有在成人的脸上找不到的天真和顽皮。还有一张颇大的嘴,充分表示着顽强,这是属于革命\footnote{〔革命〕从根本上改变社会结构和制度。革:改变,取消。}的无产阶级\footnote{〔无产阶级〕不拥有生产资料,依靠他人的生产资料劳动生产的集体。}的顽强的神情。每一遇到一些青年干部或是什么下级同志的时候,看得出那些昂奋的心,都在他那种最自然诚恳的握手里显得温柔起来。他有时也同这些人开玩笑,说着一些粗鲁无伤的笑话,但更多的时候,是耐烦地向他们解释许多政治上工作上的问题,恳切地显着对一个同志的勉励。这些听着的人便望着他,心在沉静了,然而同时又更奋起了。但一当他不说话沉思着什么的时候,周围便安静了,谁也惟恐惊扰了他。有些时候他的确使人怕,因为他对工作是严格的,虽说在生活上是马马虎虎;不过这些受了严厉批评的同志却更爱他了。
    
    拥着一些老百姓的背,揉着它们,听老百姓讲家里事,举着大拇指在那些朴素的脸上摇晃着说:“呱呱叫,你老乡好得很……”那些嘴上长得有长胡的也会拍着他,或是将烟杆送到他的嘴边,哪怕他总是笑着推着拒绝了。后来他走了,但他的印象却永远留在那些简单的纯洁的脑子中。
    
    \hfill 一九三六年十二月
    
\end{large}


\newpage

\textbf{注释}:

\vspace{-1em}

\begin{itemize}
    \setlength\itemsep{-0.2em}
    \item 〔火线〕战斗交火的前线。
    \item 〔顽强〕强硬不屈服。
    \item 〔昂奋〕精神高涨振奋。昂:高,仰。
    \item 〔勉励〕鼓励、鼓舞。
    \item 〔恳切〕诚恳亲切。
    \item 〔老百姓〕对普通民众的通称。
    \item 〔惟恐〕就怕、只怕。
    \item 〔干部〕组织的主干成员。干:事物的主体、重要部分。
\end{itemize}

\chapter{神笔马良}

\begin{large}
    
    从前有个孩子叫马良,他从小喜欢画画。马良父母死的早,靠自己打柴过日子。他到山上打柴时,就折一根树枝在地上画画;到河边割草时,就用草根蘸了水在河边画画;回到家里,就拿一块木炭在门前画画。
    
    一天,马良路过一个学馆的门口,看见学馆里的先生拿着笔写字画画。他不禁走近去问先生:“我想学画画,可以给我一支笔吗?”先生瞪了他一眼,骂道:“呸!哪来的穷娃子,也想学画!”说完就把他撵出去了。
    
    马良是个有志气的孩子。他说:“我偏不相信,没有笔就没法学画画了!”他仍旧每天坚持画画,在山里就画鸟兽树木,在河边就画花草鱼虫。晚上在家,就把白天看过的画过的细细回想,反复来画。
    
    但马良仍旧没有自己的笔。他不甘心,总想着:“要是我有一支笔,就能画得更好了!哪里能找到一支好笔呢?”
    
    马良想着想着,不觉睡着了。不知道什么时候,四周突然闪起五彩的光芒。一个白胡子老头出现在他身前。老头把一支笔交给马良,对他说:“你总说有了笔就能画得多好。我就给你一支好笔,看你能画出什么来!”马良看着这笔,笔管是枣红色的,笔斗是墨黑的,笔毫根白尖灰,细密整齐。他抬起头来刚想说话,白胡子老头已经不见了。
    
    马良一惊,就醒过来,起身一看,原来是做了一个梦。可他看看手里,真的有一支笔!笔管是枣红色的,笔斗是墨黑的,笔毫根白尖灰,和梦里见到的一模一样。
    
    马良高兴极了。要不是太阳还没有出来,他恨不得马上挨家挨户去敲门,告诉大家:“我有一支笔了!”
    
    马良拿起笔,在墙上画了一只鸟。鸟儿眨眨眼睛,扑扑翅膀,飞走了。他又到河边,在泥上画了一条鱼。鱼儿弯弯尾巴,“噗通”一声跳进河里了。马良惊呆了:“这是一支神笔啊!”
    
    马良有了这支笔,天天给乡亲们画画。谁家没了犁耙,就给他画个犁耙。谁家没了耕牛,就给他画头耕牛。谁家没有水车,就给他画个水车。谁家没有石磨,就给他画个石磨。……
    
    世上没有不透风的墙,消息很快传到了村里大财主的耳朵里。有了马良,佃户们\footnote{〔佃户〕向地主租地的农户。}再不用向财主租犁耙租耕牛租水车租石磨了。财主一打听,原来是一个叫马良的小子,有了一支神笔,能画出犁耙、耕牛、水车、石磨来。财主马上叫家丁去把马良抓来。
    
    财主看到马良,就问:“听说你有一支神笔,画什么就有什么。你帮我画几个金元宝看看。你若是画出来了,我给你吃糖。”
    
    马良说:“我要吃糖,自己画就是了,何必要你给?”
    
    财主气得要命,把神笔抢过来自己画。可也许是财主画得不像,画出来的金元宝变不成真的。财主没办法,只好把笔交给马良,让他来画。可马良就是不画。财主就吩咐家丁,把马良关在牛棚里,画不出来就不给他饭吃。
    
    到了晚上,下起大雪来,财主心想,莫要让马良饿死了。他到牛棚门口一看,里头透出红红的火光,还有诱人的香味。他不禁扒开牛棚的门往里看。只见马良烧起了一堆大火,一边烤火,一边吃烤鸡腿呢!财主知道,这火堆和鸡腿,一定是马良画出来的。这小子过得居然比我还好,成何体统!他气呼呼地把家丁叫来,就要把马良打杀,夺走他的笔。
    
    十几个家丁气势汹汹地跑进牛棚,却不见马良。只见东面墙壁上靠着一架梯子。原来马良攀上梯子,翻墙走了。财主急忙爬上梯子去追,可刚爬到一半,梯子不见了。财主一屁股掉在地上,摔得眼冒金星。原来梯子是马良画的。
    
    马良出了村子,知道在村里没法过了。他摇摇手说:“乡亲们,再见了!”就在路上画了一匹骏马,跳上马背,朝路上奔去。
    
    没走出多远,后面传来一阵呼喊声。原来是财主纠集了家丁,手执钢刀,骑着马追上来了。
    
    马良也不害怕,用笔画了一张弓,一支箭。箭一上弦,“嗖”的一声,正射中财主的喉咙。财主翻身跌下马去,家丁连忙去救。混乱之中,马良已经跳上马走远了。
    
    马良走了几天几夜,走到了京城。他在城边上住下来,做了个修补匠。什么桌子缺脚,茶壶破嘴了,他就画个新的给人送回去,只当作修补好了。只要不画活物,就没人发觉他的神笔的妙处。
    
    一天,马良在城郊见了一只五彩的雉鸡\footnote{〔雉鸡〕野鸡。},它的翅膀断了。马良可怜它,就给它画了翅膀。一时兴起,又给它多画了七彩的长尾巴。雉鸡飞到了京城里,城里的人以为是凤凰,纷纷称奇。这事惊动了京城里的大官。大官托人四处查访打听,知道了马良的神笔,就告诉了皇帝。皇帝连忙发下圣旨,叫马良来皇宫里觐见\footnote{〔觐见〕拜见君主、皇帝。}。
    
    皇帝见了马良,问:“听闻你有一支神笔,下笔成真,可有此事?”马良不愿说谎,就把神笔的事情说了。皇帝听了很高兴,又问:“朕\footnote{〔朕〕皇帝的自称。}从没见过龙,你画一条龙给朕看看!”马良在地上画了几笔,一只大壁虎蹦了出来。
    
    皇帝说:“这不是龙。你再给朕画只凤凰来看看。”马良又画了几笔,地上跳出一只大乌鸦。皇帝生气了,说:“这不是凤凰!”马良说:“我没见过龙,也没见过凤凰,画不出来。”
    
    皇帝生气了,说:“你要是画不出来,朕就砍你的脑袋。朕是天子\footnote{〔天子〕古时候认为君主的权力是天授予的,所以称皇帝为天子。},朕要杀你,你走遍天下也逃不脱。”
    
    马良想了想,说:“我没见过龙,但我的笔是神仙给的。神仙带我去了东海的仙山,仙山上有一棵摇钱树,还有长生不老的仙丹。”
    
    皇帝听了,急忙问道:“摇钱树在哪里?仙丹又在哪里?快快给朕画出来!”
    
    马良走到墙边,提笔一挥,画了广阔的大海。皇帝看得皱眉头,问:“摇钱树在哪里?”马良说:“不急,我先画出东海,再画仙山。仙山上才有摇钱树。”
    
    马良又画了几笔,大海中多了一个小岛,小岛上有棵大树,树上满是金银。
    
    皇帝看得眼都直了,问马良:“这摇钱树隔着大海,如何能到手?”马良就画了一艘大船,船上有层层楼阁,有桌椅家具,还有桅杆风帆。皇帝连忙带上文武百官,殿前卫士,要开船到海上找摇钱树去。
    
    可是船太笨重了,一直开不起来。皇帝叫道:“没风!没风!”于是马良又加了几笔,海水掀起密密的波浪,风起来了。
    
    船开进了海里,皇帝又叫道:“风不够大!”于是马良再加了几笔,浪越来越大了,船开得快了。可风浪越来越大,浪头打来,船也摇晃起来。
    
    皇帝叫道:“风够了,莫要再画了!”可马良并不停手。海上的波浪越来越大,船要翻了。
    
    皇帝喊道:“快掉头回去!船要沉了!不要再画了!”可马良仍然不住手,他在天上画了厚厚的乌云,画了倾盆似的暴雨。海浪像一堵堵高墙,向大船打去。
    
    终于,船翻了,皇帝和百官都沉到海里去了。马良把画擦掉,看了看空空的皇宫,离开了。
    
    从此以后,“神笔马良”的故事就传开了。但是,马良后来到哪里去了呢?
    
    有人说,他回到家乡,继续为乡亲们谋幸福去了。
    
    有人说,他到处流浪,专门给天下的穷苦人画画。
    
\end{large}


\newpage

\textbf{注释}:

\vspace{-1em}

\begin{itemize}
    \setlength\itemsep{-0.2em}
    \item 〔蘸〕在水里沾一点就拿出来。
    \item 〔撵〕驱赶,赶走。
    \item 〔纠集〕召集。
    \item 〔志气〕积极上进,把事情做成的决心和勇气。
    \item 〔成何体统〕算什么样子。指违反社会家庭伦常,没有规矩。
    \item 〔眼冒金星〕指脑袋受到撞击或过于疲劳晕眩时眼前的感觉。
    \item 〔谋〕设法寻求。
\end{itemize}

\chapter{长城}

\begin{large}
    
    长城,它西起嘉峪关,东至山海关,像一条长龙,飞过浩瀚的戈壁,穿越茫茫的草原,在崇山峻岭之间蜿蜒盘旋,最后,屹立在渤海岸边。全长一万三千多里。
    
    长城,是中国古代伟大的建筑工程。从春秋战国时起,各国就有修造长城抵御外敌的做法。秦始皇统一天下后,把各国修造的长城连起来,抵御北方匈奴的侵扰。经过各朝各代不断修筑经营,万里长城成为了保护中原人民的强大屏障。
    
    长城并不只是一道单独的城墙,而是由城墙、敌楼、关城、墩堡、营城、卫所、烽火台等组成,代表了一个完整的防御体系,由中央朝廷下属的边防军队层层指挥,节节控制。比如,明代的长城防线上,设有辽东、蓟、宣府、大同、山西、榆林、宁夏、固原、甘肃等九个边防区域,称为“九边重镇”,每个重镇设有总兵,管理边防事务。
    
    长城的主体,是高大坚固,延绵不绝的城墙。它建于高山峻岭或平原险阻之处,采取“因地就险,就地取材”的思路。在平原就用砖石,在沙漠就用柳条沙土,在最为陡峻的地方,就利用悬崖绝壁做墙,制造敌人无法翻越的要塞。比如修筑在八达岭上的城墙,用巨大的条石和城砖筑成。城墙顶上铺着方砖,平整如宽阔的马路,五六匹马可以并行。城墙外沿设有两米高的垛子,一座连着一座。垛子上有方形的瞭望口和射口,供瞭望和射击用。城墙顶上,每隔三百多米,就有一座方形的城台,是屯兵用的墩堡。打仗的时候,城台之间可以遥相呼应,互相支援。
    
    关城是长城防线上兵力最集中的据点,建在易守难攻的地方,有“一夫当关,万夫莫开”的说法。明代的长城就有近千处关城,除了嘉峪关和山海关,著名的还有雁门关、居庸关、萧关等等。
    
    长城这么长,各个地方之间怎么传递消息呢?人们在长城上建造了烽火台。要是敌人来了,白天烧烟,晚上举火,下一个烽火台看到了,就照样操作。不用多久,情报就能传递数千里,从边关传到首都。
    
    登上长城,眺望远方,群岭苍翠,长城犹如一条玉色的裙带飘绕其间。这样宏伟的工程,是一代代爱好和平的劳动人民努力奋斗的结果。无数的肩膀,无数双手,为了保护自己的父老乡亲,想尽办法,把巨大的砖石堆砌到山林之中。绝大的毅力和智慧,才凝结成这前不见头,后不见尾的万里长城。
    
\end{large}



\chapter{森林的主人}

\begin{large}
    
    雨下了整整一个星期。灰色的云层低低地压在大森林上面,潮湿的风缓缓地吹着。吸饱雨水的树枝垂下来。河水涨到齐了岸。我和猎人划着小船顺流而下。到了河身狭窄的地方,小船突然撞在水面下的树桩上,翻了。食物和打来的野味全给冲走了,我们只好带着猎枪上了岸。
    
    这里离住所还很远。我们俩浑身是水,又累又饿。我冷得发抖,呆呆地望着猎人,希望他有个办法。猎人不声不响,只顾拧他的衣服。“应该生一堆火呀!”我提议,可是从口袋里摸出火柴盒一看,里面竟流出水来。
    
    猎人还是不声不响。他在一棵枞树的窟窿里找到了一些干的苔藓,又拿出一颗子弹,拔下弹头,把苔藓塞进弹壳,塞得紧紧的。他吩咐我:“你去找些干的树枝和树皮来。”
    
    我找来了。他把那颗拔掉弹头的子弹装进枪膛,对着地面开了一枪。从枪口喷出来的苔藓烧着了。他小心地把火吹旺,把树枝和树皮一点儿一点儿加上去,不一会儿,篝火熊熊,烧得很旺。
    
    “你照看火堆。我去打些野味来。”猎人说着,转到树背后就不见了。只听见树林里响了几枪。我还没捡到多少干柴,他已经回来了。几只松鸡挂在他腰上,摇摇晃晃的。
    
    “我们做晚饭吧。”他说。他把火堆移到一边,用刀子在刚才烧火的地上挖了个洞。我把松鸡拔了毛,掏了内脏。猎人又找来几片大树叶,把松鸡裹好,放进洞里,盖上薄薄的一层土,然后在上面又烧起一堆火。
    
    等我们把衣服烘干,松鸡也烧好了,扒开洞,就闻到一股香味。我们俩大吃起来,我觉得从来没吃过这么鲜美的东西。
    
    天黑了,风刮过树顶,呼呼地响。
    
    “睡吧。”猎人打了个呵欠说。
    
    我的眼也快要合上了。可是这潮湿冰冷的地面,怎么能睡呢?
    
    猎人带着我折来许多枞树枝。他把两个火堆移开,在烤热的地面上铺上枞树枝,铺了厚厚的一层。热气透上来,暖烘烘的,我们睡得很舒服,跟睡在炕上一个样。
    
    天亮了,我对猎人说:“你真有办法。要不是你,我一定要吃苦头了。”
    
    猎人微笑着说:“大森林里,你不能像个客人,得像个主人。只要肯动脑筋,一切东西都可以拿来用。
    
\end{large}



\chapter{鱼游到了纸上}

\begin{large}
    
    西湖有很多地方可以观鱼。我喜欢花港,更喜欢“泉白如玉”的玉泉。
    
    玉泉的池水清澈见底。坐在池边的茶室里,泡上一壶茶,靠着栏杆看鱼儿自由自在地游来游去,真是赏心悦目。茶室的后院还有十几缸鱼呢,那儿也聚集着许多爱鱼的人:有老人,有孩子,也有青年。
    
    就在金鱼缸边,我认识了一位举止特别的青年。他高高的身材,长得很秀气,一对大眼睛明亮得就像玉泉的水。
    
    说“认识”,其实我并不了解他,只是碰到过几次罢了。说他“特别”,因为他爱鱼爱到了忘我的境界。他老是一个人呆呆地站在金鱼缸边,静静地看着金鱼在水里游动,而且从来不说一句话。
    
    一个星期天,我到玉泉比平时晚了一些。金鱼缸边早已挤满了人,多数是天真活泼的孩子。这些孩子穿着鲜艳的衣裳,好像和金鱼比美似的。
    
    “哟,金鱼游到了他的纸上来啦!”一个女孩惊奇地叫起来。
    
    我挤过去一看,原来是那位青年在静静地画画。他有时工笔细描,把金鱼的每个部位一丝不苟地画下来,像姑娘绣花那样细致;有时又挥笔速写,很快地画出金鱼的动态,仿佛金鱼在纸上游动。
    
    围观的人越来越多,大家赞叹着,议论着,唯一没有任何反应的是他自己。他好像和游鱼已经融为一体了。
    
    我仍旧去茶室喝茶,等到太阳快下山了才起身往回走,路过后院,看到那位青年还在金鱼缸边画画。他似乎忘记了时间,也忘记了自己。
    
    “你真专心哪!”我忍不住轻声问他。没想到他头也不抬,理也不理我。
    
    好骄傲的年轻人。我正想着,眼睛落到他胸前的厂徽上,心不由得咯噔一跳!“福利工厂”,原来他是个聋哑人!
    
    我们开始用笔在纸上交谈。他告诉我,他学画才一年多,为了画好金鱼,每个星期天都到玉泉来,一看就是一整天,常常忘了吃饭,忘了回家。
    
    我把那个女孩说的话写给他:“鱼游到了你的纸上来啦!”
    
    他笑了,笑得那么甜。他接过笔在纸上又加了一句:“先游到了我的心里。”
    
\end{large}



\chapter{高梁情}

\begin{large}
    
    高梁在我的心里有着神圣的地位。一想到高梁,眼前就浮现出它正直的秆子,硕大而血红的穗头,紧紧抓住土地的根,想到它那令人敬佩的品性。
    
    我的家乡在雁门关脚下,土地灰茫茫的,十分贫瘠,能够种麦子的地极少,只有耐得住大自然折磨的强悍的高梁好种。千百年来,土地、人和高梁紧紧地结合在一起。我是吃高粱米张大的。在我离开故乡之前,一年四季,顿顿饭离不开高粱。它塑造了我的躯体和生命。
    
    高粱的全身没有一处不具有鲜明的个性。它那挺拔而粗壮的秆子,给人以自信和力量。尤其使我惊异的是高粱的根,它不但在看不见的底下扎得很深很深,而且在高粱秆的下端离地尺把高的关节处,向下长着许多气根,有点像榕树的根,用手摸摸,是那么坚韧,像鹰爪一样,它们强有力地抓住土地,仿佛擒拿住一个庞大的活物。我问父亲,为什么高粱下边张了这么多“爪子”?父亲告诉我,不要以为草木、庄稼都不如人,都没有知觉。其实庄稼很聪明,她们对大自然的感觉,甚至比人类还要敏锐。她们有的爬蔓,紧紧贴着大地,有的像树一样站着,都是为了生存,为了争取阳光和空间。对于高粱,气根是不可缺少的。夏天暴风雨来临前,它就迅速地生出起根,深深地扎进土里。风暴无法撼动它,就像一个摔跤手,脚跟稳稳地定在地上,等着对手向他扑来。
    
    父亲说:“高梁的根最苦,所有的虫子都不敢咬它,根是它的命。”说着,父亲掐了一小截,让我用舌头舔舔,啊呀,那个苦劲儿到现在我还记得。
    
    麦子、豆秧能用手连根拔起,但是再有力气的庄稼人也很难拔动高梁。我小时练摔跤时,教我的伯伯说:“站得像高梁一样,要有它那抓地的根,要练到根从脚脖子上生出来。”他还说:“摔跤时脚定在地上,觉得自个儿不是有两只脚,而是有几十只!”高梁就有几十只脚,而且每一只脚都深深地扎到地底下。
    
    我虽然练不出高梁的鹰爪般的脚,但它那坚韧不拔的品格却始终激励我顽强地生活着,跋涉着。
    
\end{large}



\chapter{只有一个地球}

\begin{large}
    
    据有幸飞上太空的宇航员介绍,他们在天际遨游时遥望地球,映入眼帘的是一个晶莹的球体,上面蓝色和白色的纹痕相互交错,周围裹着一层薄薄的水蓝色“纱衣”。地球,这位人类的母亲,这个生命的摇篮,是那样的美丽壮观,和蔼可亲。
    
    但是,同茫茫宇宙相比,地球是渺小的。它只是一个半径只有六千三百多公里的星球。在璀璨的星海中,就像一叶扁舟。它只有这么大,不会再长大了。
    
    地球所拥有的自然资源也是有限的。拿矿物资源来说,它不是神仙的恩赐,而是经过几百万年,甚至几亿年的地质变化才形成的。地球是无私的,它向人类慷慨地提供矿产资源。但是,如果不加节制地开采,这些矿产资源也许很快就会枯竭。
    
    人类生活所需要的水资源、森林资源、生物资源、大气资源,本来是可以不断再生,长期给人类作贡献的。但是,人类随意毁坏自然资源,不顾后果地滥用化学制品,不但使它们难以再生,还造成了一系列生态灾难,给自己的生存带来了严重的威胁。
    
    有人会说,宇宙空间不是大得很吗?宇宙里有数不清的星球,在地球资源枯竭的时候,我们不能移居到别的星球上去吗?
    
    从目前来看,至少在以地球为中心的四十万亿公里的范围内,没有适合人类居住的第二个星球。人类不能指望在破坏了地球以后再移居到别的星球上去。
    
    不错,科学家们提出了许多设想。例如,在火星或者月球上建造移民基地。但是,即使这些设想能实现,也是遥远的事情。再说,又有多少人能够去居住呢?
    
    “我们这个地球太可爱了,同时又太脆弱了!”这是宇航员遨游太空目睹地球时发出的感叹。
    
    只有一个地球,如果它被破坏了,我们别无去处。如果地球上的各种资源都枯竭了,我们很难从别的地方得到补充。我们要精心地保护地球,保护地球的生态环境。让地球更好地造福于我们的子孙后代吧!
    
\end{large}



\chapter{一个苹果}

\begin{large}
    
    “连长,给!一个苹果。”
    
    这是五连支援我们战斗的一个火线运输员,年龄顶多不过20岁,身体矮瘦矮瘦的,刚刚跨进防炮洞,一卸完身上背着的弹药,就递给了我一个苹果。
    
    防炮洞只有三米长,两米宽。黄昏时分借着洞口闪进的亮光,我看到这个年轻的运输员,满身尘土,下身的单裤经过一路在敌炮火下爬行、打滚,已经撕了好几条口子了,脚脖上也划破了好几处,浸着血迹。我注视着他那满脸汗水的瘦长的脸孔,简直有点不敢相信似的问:
    
    “哪里来的苹果呀?”
    
    “我在半路上拾的。连长,你的嗓子哑了,吃了润润喉咙吧!”
    
    这是事实:自从24日我连出击开始,除前天晚上营长给了我一块二寸长的萝卜外,7天以来,我们没喝过一口水。我的喉咙早就干得冒火,烟熏火燎般的难受。不用说,战士们更干渴得厉害。
    
    “你们运输辛苦,还是你吃了它吧。”我对运输员同志说。我想到他这些天来和我们一样过着艰苦的生活,也够苦了。
    
    “不,我在路上可以喝凉水。”他对我羞怯地笑着,推让着,固执地说什么也不肯吃。
    
    谁都知道,通往后方的三里路内是找不到一滴水的,这个运输员因为爱护我们而撒了谎。
    
    我怀着感激的心情,望着这个显见是用心擦得很干净的苹果:它青里透红,发出诱人的香味。这会儿,不用说一个,就是一二十个,我一个人也能吃完。
    
    “给谁吃呢?”我拿在手中颠来倒去地想。这时,步话机员李新民正在我的身旁,向上级报告战斗情况。他沙哑的声音,使我突然注意到:这个平时爱说爱唱的步话机员,这些天来,在日日夜夜的战斗中,一直就没很好地休息过,他的嗓子已经全哑了,嘴唇干得裂开好几道血口子,血痂还凝在嘴唇上,满脸的灰尘,深陷在黑色眼眶里的两只眼睛,像害热病似的布满了血丝,红得可怕。
    
    “李新民,你们几个人分着吃了这个苹果,润润喉咙,好继续工作。”我把苹果给了他。
    
    李新民出神地看着我。他知道我的脾气:话出口就不收回。他回头看了看另外几个人,又看了看睡在洞子里面的伤员蓝发保,把苹果接了过去,却并不吃,转手给了蓝发保。
    
    从来没离开过我的通信员蓝发保,在一次执行通信任务中被炮弹打断了右腿,现在睡在那里,很少听到他的呻吟。他的脸黑黄黑黄的,嘴唇干得发紫。他拿起苹果正准备吃,突然向周围望了望,又闭住嘴,把苹果放下了。原来他才发现一共只有一个苹果。
    
    “连长,你几天没喝水了,你吃吧,吃了好指挥咱们打仗。”不管别人怎样劝说,蓝发保说什么也不吃,还是把苹果递给了我。
    
    于是,我又只好把苹果递给了司号员,司号员立刻转手递给了身旁的卫生员。卫生员又把它交给了自己日夜照顾着的伤员蓝发保。最后,苹果转了个圈儿,还是原样落到我的手中。
    
    再传下去是没有用的。我知道:越在最艰苦的时候,战士们就越特别关心自己的首长。我不吃,他们决不肯吃。于是,我决定由我们8个人共同来分吃这个来之不易的苹果。
    
    吃苹果也要做一番动员。我用沙哑的声音说:“同志们,我们能够夺回阵地,赶走了敌人,难道我们就不能吃掉这个苹果吗?”
    
    看看谁也不吭气,我紧接着再逼近一步:“来,一人分吃一口,谁不吃谁就是对胜利不关心!”说罢,我先咬了一口,就转给李新民。李新民放到嘴边,只轻轻咬了一小口,就交给身旁的胡景才。然后一个挨一个地传下去。这回是转了一圈,苹果还剩大半个。
    
    “是谁没有吃?”我问。
    
    仍然没有人吭声。
    
    我真有点不满意了,刚想说几句责备的话,命令大家认真地把苹果分吃了,可是,我忽然觉得周围的空气格外沉静。在昏暗中,我看见一向表现乐观的步话机员李新民,面颊上闪动着晶莹的泪珠;借着洞外射进来的微弱的光线,我看见每个同志都在用手擦着眼睛。一种强大的感情立刻向我冲来,一瞬间,我像给什么东西塞住了喉咙似的。在这战火纷飞的夜晚,我被这种发自阶级友爱的战友间的关怀深深激动着,迸出了幸福的、骄傲的泪花。
    
\end{large}



\chapter{出海}

\begin{large}
    
    船出港了。偶尔,某只船上会传来说话声。但大多数的船都是安静的,只有荡桨的声音。他们一出港口就分开来,每个人向着他认为能找到鱼的那片海直奔而去。老头儿知道他越走越远了。他已经把陆地的气息抛在后面,驶进了黎明时分大海的清新气息里。他在海水里看到了马尾藻\footnote{〔马尾藻〕一种随海流漂移的海藻,有些能发磷光。}的磷光,那是被渔民叫做“大井”的地方,那里有个突然下沉七百英寻\footnote{〔英寻〕英制长度单位。一英寻约为1.83米。}深的海沟。海流碰在海底的峭壁上形成的漩涡,让各种鱼儿都聚集在那儿。深深的海水好似洞穴,洞穴里有大量小鱼小虾,有时候还有成群的乌贼。成簇的小鱼虾在夜里被水流带着,升上海面来,成为大鱼的食物。
    
    黑暗里,老头儿能察觉到,天要亮了。摇桨的时候,他能听见飞鱼\footnote{〔飞鱼〕生活在近海面的鱼类,体型不超过40厘米,受惊时会短暂飞出水面。}出水时的颤声,听见它们在暗里凌空而去时,绷紧的翅膀上发出的咝咝声。他很喜欢飞鱼。飞鱼是他出海时主要的朋友。他觉得海上的鸟儿很可怜,特别是那些娇小的黑色的燕鸥。它们永远在盘旋,永远在张望,但几乎总找不到吃的。他想:“海上的鸟儿过得比我们还难,除了那些特别强壮的,还有抢别的鸟的。为什么造物主让大海如此残酷,而又要把海上的鸟儿造得这样娇弱,这样纤细呢?她是如此温柔而美丽。但她也可以霎那间变得如此残酷。海上的鸟儿啊!带着微弱而悲戚的叫声,盘旋着,搜寻着,轻点水面。对大海来说,它们太娇弱了。”
    
    提起大海,他总会用“她”。喜爱大海的人有时会说大海的坏话,但他们的口气里总把大海看作女子。年轻的渔民,那些用浮标当钓竿浮子,卖了点鲨鱼肝赚了钱弄了艘小汽艇的,他们提到大海,总会用“他”。他们把大海当作竞争者,当作竞技场,乃至敌人。但老头儿总把大海当作女子。她的脾气捉摸不定,不知何时才会回报你的追求。忍不住时,便要任性使坏。大概是受月亮的影响——他是这么想的。
    
    他不慌不忙地划着船。只是保持一定的速度,也不需要使多大的劲儿。海面很平静,海流偶尔打个旋儿。他让海流替他出了三分之一的力气。天开始发白了,他发现,他已经划得比预想中更远了。
    
\end{large}



\chapter{草船借箭}

\begin{large}
    
    鲁肃见到周瑜,把诸葛亮识破蒋干盗书的事说了。周瑜大惊,说:“此人不可留!”鲁肃劝道:“斩了诸葛孔明,只会让曹操笑话。”周瑜说:“我自有办法,让大家心服口服。”
    
    第二天,周瑜召集众人商议军事,请诸葛亮参加。周瑜问诸葛亮:“我们就要跟曹军交战了。水上交战,用什么兵器最好?”诸葛亮说:“大江之上,用弓箭最好。”周瑜说:“对,先生跟我想的一样。现在军中缺箭,想请先生负责赶造十万支。这是公事,希望先生不要推却。”诸葛亮说:“都督委托,当然照办。不知道这十万支箭什么时候用?”周瑜问:“十天造得好吗?”诸葛亮说:“既然就要交战,十天造好,必然误了大事。”周瑜问:“先生预计几天可以造好?”诸葛亮说:“只要三天。”周瑜说:“军中无戏言。”诸葛亮说:“怎么敢戏耍都督?我愿意立下军令状,三天造不好,甘受惩罚。”周瑜很高兴,叫诸葛亮当面立下军令状,又摆好了酒席招待他。诸葛亮说:“今天来不及了。从明天起,到第三天,请派五百个军士到江边来搬箭。”诸葛亮喝了几杯酒就走了。
    
    鲁肃对周瑜说:“十万支箭,三天怎么造得成呢?孔明说的是假话吧?”周瑜说:“是他自己说的,我可没逼他。我得吩咐军匠们,叫他们故意延迟,造箭用的材料,不给他准备齐全。到时候造不成,定他的罪,他就没话可说了。你去探听探听,看他怎么打算,回来报告我。”
    
    鲁肃见了诸葛亮。诸葛亮说:“我叫你莫要对都督说起,你看,现在都督要害我了。三天之内要造十万支箭,得请你救救我。”鲁肃说:“都是你自己找的,我怎么救得了你?”诸葛亮说:“你借给我二十条船,每条船上要三十名军士。船用青布幔子遮起来,还要一千多个草把子,排在船的两边。我自有妙用。第三天管保有十万支箭。不过不能让都督知道。他要是知道了,我的计划就完了。”
    
    鲁肃答应了。他不知道诸葛亮借了船有什么用,回来报告周瑜,果然不提借船的事,只说诸葛亮不用竹子、翎毛、胶漆这些材料。周瑜疑惑起来,说:“到了第三天,看他怎么办!”
    
    鲁肃私自拨了二十条快船,每条船上配三十名军士,照诸葛亮说的,布置好青布幔子和草把子,等诸葛亮调度。第一天,不见诸葛亮有什么动静;第二天,仍然不见诸葛亮有什么动静;直到第三天四更时候,诸葛亮秘密地把鲁肃请到船里。鲁肃问他:“你叫我来做什么?”诸葛亮说:“请你一起去取箭。”鲁肃问:“哪里去取?”诸葛亮说:“不用问,去了就知道。”诸葛亮吩咐把二十条船用绳索连接起来,朝北岸开去。
    
    这时候大雾漫天,长江上连面对面都看不清。天还没亮,船已经靠近曹军的水寨。诸葛亮下令把船头朝西,船尾朝东,一字儿摆开,又叫船上的军士一边擂鼓,一边大声呐喊。鲁肃吃惊地说:“如果曹兵出来,怎么办?”诸葛亮笑着说:“雾这样大,曹操一定不敢派兵出来。我们只管饮酒取乐,天亮了就回去。”
    
    曹操听到鼓声和呐喊声,就下令说:“江上雾很大,敌人忽然来攻,我们看不清虚实,不要轻易出动。只叫弓箭手朝他们射箭,不让他们近前。”他派人去旱寨调来六千名弓箭手,到江边支援水军。一万多名弓箭手一齐朝江中放箭,箭好像下雨一样。诸葛亮又下令把船掉过来,船头朝东,船尾朝西,仍旧擂鼓呐喊,逼近曹军水寨去受箭。
    
    天渐渐亮了,雾还没有散。这时候,船两边的草把子上都插满了箭。诸葛亮吩咐军士齐声高喊“谢谢曹丞相的箭!”接着叫二十条船驶回南岸。曹操知道上了当,可是这边的船顺风顺水,已经飞一样地驶出二十多里,要追也来不及了。
    
    二十多条船靠岸的时候,周瑜派来的五百个军士正好来到江边,诸葛亮让他们上船取箭。每条船大约有五六千支箭,二十条船总共有十万多支。鲁肃见了周瑜,告诉他借箭的经过。周瑜长叹一声,说:“诸葛亮神机妙算,我比不上他!”
    
    有道是:
    
    一天浓雾满长江,远近难分水渺茫。
    
    骤雨飞蝗来上舰,孔明今日服周郎。
    
\end{large}



\chapter{养花}

\begin{large}
    
    我爱花,所以也爱养花。我可还没成为养花专家,因为没有工夫去研究和试验。我只把养花当做生活中的一种乐趣,花开得大小好坏都不计较,只要开花,我就高兴。在我的小院子里,一到夏天满是花草,小猫只好上房去玩,地上没有它们的运动场。
    
    花虽然多,但是没有奇花异草。珍贵的花草不易养活,看着一棵好花生病要死,是件难过的事。北京的气候,对养花来说不算很好,冬天冷,春天多风,夏天不是干旱就是大雨倾盆,秋天最好,可是会忽然闹霜冻。在这种气候里,想把南方的好花养活,我还没有那么大的本事。因此,我只养些好种易活、自己会奋斗的花草。
    
    不过,尽管花草自己会奋斗,我若是置之不理,任其自生自灭,大半还是会死的。我得天天照管它们,像好朋友似的关心它们。一来二去,我摸着一些门道:有的喜阴,就别放在太阳地里;有的喜干,就别多浇水。摸着门道,花草养活了,而且三年五载老活着,开花,多么有意思呀!不是乱吹,这就是知识呀!多得些知识决不是坏事。
    
    我不是有腿病吗,不但不利于行,也不利于久坐。我不知道花草们受我的照顾,感谢我不感谢;我可得感谢它们。我工作的时候,总是写一会儿就到院子里去看看,浇浇这棵,搬搬那盆,然后回到屋里再写一会儿,然后再出去。如此循环,把脑力劳动和体力劳动得到适当的调节,有益身心,胜于吃药。要是赶上狂风暴雨或者天气突变,就得全家动员,抢救花草,十分紧张。几百盆花,都要很快地抢到屋里去,使人腰痠腿疼,热汗直流。第二天,天气好了,又得把花都搬出去,就又一次腰痠腿疼,热汗直流。可是,这多么有意思呀!不劳动,连棵花也养不活,这难道不是真理吗?
    
    送牛奶的同志进门就夸“好香”,这使我们全家都感到骄傲。赶到昙花开放的时候,约几位朋友来看看,更有秉烛夜游的味道——昙花总在夜里开放。花分根了,一棵分为几棵,就赠给朋友们一些。看着友人拿走自己的劳动果实,心里自然特别欢喜。
    
    当然,也有伤心的时候,今年夏天就有这么一回。三百棵菊秧还在地上,下了暴雨,邻家的墙倒了,菊秧被砸死三十多种,一百多棵。全家人几天都没有笑容。
    
    有喜有忧,有笑有泪,有花有果,有香有色,既须劳动,又长见识,这就是养花的乐趣。
    
\end{large}


\newpage

\textbf{注释}:

\vspace{-1em}

\begin{itemize}
    \setlength\itemsep{-0.2em}
    \item 〔工夫〕可以花费的时间和精力。
    \item 〔置之不理〕放着它不去关照、处理。之:他、它。
    \item 〔任其〕任由它。其:他、它。
    \item 〔门道〕复杂的建筑里,穿过墙壁、屋阁的有门的过道。比喻有效的办法、窍门。
    \item 〔脑力劳动〕以思考、记忆为主,使用智力、知识的劳动。区别于以行动、操作为主,使用肢体力量的“体力劳动”。脑力劳动和体力劳动并不互相排斥,也没有明显分界。
    \item 〔秉烛夜游〕手持蜡烛在夜晚游玩。秉:手持、拿着。
\end{itemize}

\chapter{蛇与庄稼}

\begin{large}
    
    几十年前,广东沿海发生了一次海啸\footnote{〔海啸〕由风暴或海底地震引起海水剧烈波动,冲击、淹没沿海陆地的灾害。},是台风\footnote{〔台风〕北太平洋海域的强热带气旋导致的风暴,主要波及北太平洋西岸地区。}引起的,许多田地和村庄被海水淹没了。洪水过后,那儿的庄稼总得不到好收成,接连几年都是这样,即使风调雨顺,也不见起色。后来,老农们想出了一个办法,他们托人去外地买了一批蛇回来,把蛇放到田里。说也奇怪,那一年庄稼就获得了丰收。大家不明白这是什么道理。老农把秘密说破了。
    
    原来,那场洪水把深藏在洞里的蛇都给淹死了,田鼠却游到树上和山坡上,保住了性命。后来洪水退了,田鼠又回到田里糟蹋庄稼。没有蛇来捕捉它们,它们繁殖得特别快,庄稼都让田鼠给糟蹋了。现在田里又有了蛇,大量的田鼠让蛇给吃掉了,因而庄稼又得到了好收成。
    
    生物学家达尔文\footnote{〔达尔文〕查尔斯·达尔文,19世纪英国博物学家、生物学家,演化论的奠基人,著有《物种起源》。}说过猫和苜蓿\footnote{〔苜蓿〕多年生草本植物,也叫三叶草。苜蓿常用作奶牛的饲料。}的故事。他说在英国的好些地方,看村子喂的猫多还是少,就可以知道那里的苜蓿长得怎么样。苜蓿靠土蜂传播花粉,地里田鼠太多,土蜂的活动就受到影响。村子里喂的猫多,田鼠就不能那么放肆了。因此猫的多少,竟和苜蓿的收获有密切的关系。
    
    天上下雨,地面就湿;太阳出来,东西就容易晒干;火会烧毁东西,水可以灭火。这些事儿都是显而易见的,大家都知道。但是世界上的事物除了这样简单的联系之外,还有不少复杂的联系,不是一下子能看清楚的。一个人不讲卫生,在马路上吐一口痰,也许会使许多人得病,甚至染上肺结核\footnote{〔肺结核〕一种主要影响肺部的严重传染病,也叫肺痨。}。肺结核病人躺在床上呻吟的时候,他怎么会想到,使他生病的就是马路上随便吐痰的人呢?事物之间的相互联系是非常复杂的,咱们必须不怕麻烦,研究它们的规律。这样,才能把事儿做得更合咱们的意愿。
    
\end{large}


\newpage

\textbf{注释}:

\vspace{-1em}

\begin{itemize}
    \setlength\itemsep{-0.2em}
    \item 〔风调雨顺〕风雨及时适宜,利于农业生产。
    \item 〔糟蹋〕不爱惜,随意丢弃或毁坏。
    \item 〔事物〕事件和物体,泛指世间存在、发生的一切。
    \item 〔呻吟〕因痛苦而发出类似叹息的叫声。
    \item 〔痰〕呼吸道分泌,而由口、鼻腔排出的粘液。
\end{itemize}

\chapter{彩色的翅膀}

\begin{large}
    
    一场暴雨刚刚过去,碧空如洗,海面上波涛起伏。船有节奏地前后晃荡着。陪我同船前往宝石岛的,是个矮墩墩的战士,宝石岛观察通讯站的信号兵,姓高,刚从黑龙江回来。
    
    小高在码头上有说有笑,这时候不吭声了,紧闭着嘴唇,两眼直发愣。他把他的大提包扔在一边,怀里紧紧地抱着一只纸箱子。
    
    为了调节一下沉闷的气氛,我有意地同他开玩笑:“我猜这只纸箱里,一定装着好吃的东西。从家乡带来的吧?”
    
    小高淡淡地一笑:“不,不能吃。”
    
    “我才不信呢!”我一副认真的样子,“快公开吧,让我也尝尝。“
    
    小高有点急了:“真不能吃。里面装的是一些小昆虫,蝴蝶呀什么的,一打开就飞跑了。”
    
    从没听说过战士探亲回来带这种东西的。我正想问个水落石出,可是小高的嘴唇又闭紧了,脸色比先前还难看。我知道晕船是什么滋味,便打住了。
    
    傍晚,船把我们送到宝石岛。当岛顶的灯塔放射出雪亮的光芒的时候,观察通讯站站长拉着我,说:“走,参加我们的晚会去。”
    
    这真是个特别的晚会。黑板上用仿宋体写着“尝瓜会”三个大字,小讲台上的白瓷盘里放着一个大西瓜。站长右手托起那个大西瓜,笑呵呵地说:“同志们,这是我们岛上结的第一个西瓜。今晚,我们开个尝瓜会表示庆祝,大家来分享自己的劳动果实。”
    
    在一片欢笑声中,我了解到这个西瓜不平常的来历。
    
    两年前,战士们来到宝石岛上,建立起这个新的阵地。他们在岩石下、小路旁,垒出一块块“海岛田”,把从家乡带来的蔬菜种子,连同自己建岛爱岛的深情一起播种下去。去年,站长和战士们撒下了几颗西瓜子。瓜苗出土了,瓜秧拖蔓了,还开了一朵朵小黄花。可是到了收获季节,竟连一个小瓜也没结。有些战士灰心了,撅着嘴巴,说:“西瓜嫌我们的岛艰苦,不愿在这里安家。”
    
    为什么瓜秧开了花不结瓜?是水浇得不够,是肥施得不足,还是土壤根本不行?一位雷达兵懂一些农业知识,他找到了答案:西瓜开了花要授粉。小岛远离大陆,没有蜜蜂,也没有别的昆虫。西瓜花没授粉,当然结不了瓜。经他一说,大家才明白了。今年瓜秧开了花,他们仔仔细细地给每一朵雌花都进行了人工授粉。小瓜果然结了不少,水灵灵的,真惹人喜爱。谁料一阵暴雨过后,巨浪扑上了小岛,把小瓜一个个打掉了,后来一检查,只有大石头后边的一根瓜秧上,还残存着一个小瓜。他们像抚养婴儿似的照看着这个小瓜,浇水,施肥,一点也不敢马虎。奇迹终于出现了,这个岛上成熟了第一个西瓜。
    
    站长把大西瓜切成薄薄的小片,盛在白瓷盘里,送到每一个战士眼前。战士们都笑着,用两个指头捏起一小片来,细细地端详着,轻轻地闻着,慢慢地咬着,不住发出啧啧的赞叹声。好像有一股甘泉,流进了每个战士的心田。
    
    我推推坐在身旁的小高,笑着说:“你那个纸箱的秘密,现在该公开了。”
    
    小高说:“你早就明白了嘛。”
    
    “这么说,你真想让那些蝴蝶呀什么的在这里安家?”
    
    小高点点头告诉我,晚饭以前,他已经把纸箱里的小昆虫全放了。他笑嘻嘻地说:“我就不相信,这些小精灵会不爱我们祖国的海岛,会不愿在这里安居乐业。”
    
    第二天我醒来时,天已经放亮了。我忽然发现窗玻璃上停着一只蝴蝶,正对着朝阳,扇动着它那对彩色的翅膀。
    
\end{large}



\chapter{飞夺泸定桥}

\begin{large}
    
    1935年5月,北上抗日的红军向天险大渡河挺进。大渡河水流湍急,两岸都是高山峻岭,只有一座铁索桥可以通过。这座铁索桥,就是红军北上必须夺取的泸定桥。
    
    国民党反动派早就派了两个团防守泸定桥,阻拦红军北上;后来又调了两个旅赶去增援,妄想把我红军消灭在桥头上。我军早就看穿了敌人的诡计。28日早上,红四团接到上级命令:“29日早晨夺下泸定桥!”时间只剩下二十多个小时了,红四团离泸定桥还有二百四十里。敌人的两个旅援兵正在对岸向泸定桥行进。抢在敌人前头,是我军战胜敌人的关键。
    
    红四团翻山越岭,沿路击溃了好几股阻击的敌人,到晚上七点钟,离泸定桥还有一百一十里。战士们一整天没顾得上吃饭。天又下起雨来,把他们都淋透了。战胜敌人的决心使他们忘记了饥饿和疲劳。在漆黑的夜里,他们冒着雨,踩着泥水继续前进。
    
    忽然对岸出现了无数火把,像一条长蛇向泸定桥的方向奔去,分明是敌人的增援部队。红四团的战士索性也点起火把,照亮了道路跟对岸的敌人赛跑。敌人看到了这边的火把,扯着嗓子喊:“你们是哪个部分的?”我们的战士高声答话:“是碰上红军撤下来的。”对岸的敌人并不疑心。两支军队像两条火龙,隔着大渡河走了二三十里。雨越下越猛,像瓢泼一样,把两岸的火把都浇灭了。对岸的敌人不能再走,只好停下来宿营。红四团仍旧摸黑冒雨前进,终于在29日清晨赶到了泸定桥,把增援的两个旅的敌人抛在后面了。
    
    泸定桥离水面有好几丈高,是由十三根铁链组成的:两边各有两根,算是桥栏;底下并排九根,铺上木板,就是桥面。人走在桥上摇摇晃晃,就像荡秋千似的。现在连木板也被敌人抽掉了,只剩下铁链。向桥下一看,真叫人心惊胆寒,红褐色的河水像瀑布一样,从上游的山峡里直泻下来,撞击在岩石上,溅起一丈多高的浪花,涛声震耳欲聋。桥对岸的泸定桥背靠着山,西门正对着桥头。守城的敌人根本没想到红军这么快就来了,大多在城里花天酒地,只有几十人仓促应战。他们凭着天险,疯狂地向红军喊叫:“来吧,看你们飞过来吧!”
    
    红四团马上发起总攻。团长和政委亲自站在桥头上指挥战斗。号手们吹起冲锋号,所有武器一齐开火,枪炮声,喊杀声,霎时间震动山谷。二连担任突击队,二十二位英雄拿着短枪,背着马刀,带着手榴弹,冒着敌人的枪弹,攀着铁链向对岸前进。跟在他们后面的是三连,战士们除了武器,每人带一块木板,一边前进一边铺桥。
    
    越靠近对岸,枪弹越加密集。最前头的战士中弹牺牲了,第二位也中弹了,但其余的战士前进得更快了。快冲到对岸,敌人就放起火来,桥头立刻被大火包围了,铁索变得滚烫。在这千钧一发的时刻,英雄们仿佛忘记了疼痛,奋不顾身,箭一般地穿过熊熊大火,冲进城去,和城里的敌人展开了激烈的搏斗。激战了两个小时,守城的敌人被消灭了大半,其余的都狼狈地逃跑了。
    
    红四团英勇地夺下了泸定桥,取得了长征中的又一次决定性的胜利。红军的主力渡过了天险大渡河,浩浩荡荡地奔赴抗日的最前线。
    
\end{large}



\chapter{囚歌}

\begin{large}
    
    \begin{verse}[0.5\linewidth]
        为人进出的门紧锁着, \\
        为狗爬出的洞敞开着, \\
        一个声音高叫着: \\
        ——爬出来吧,给你自由! \\
        我渴望自由, \\
        但我深深地知道—— \\
        人的身躯怎能从狗洞里爬出! \\
        我希望有一天, \\
        地下的烈火, \\
        将我连这活棺材一齐烧掉, \\
        我应该在烈火与热血中得到永生!
    \end{verse}
    
\end{large}



\chapter{我的“自白”书}

\begin{large}
    
    \begin{verse}[0.5\linewidth]
        任脚下响着沉重的铁镣, \\
        任你把皮鞭举得高高, \\
        我不需要什么自白, \\
        哪怕胸口对着带血的刺刀!
    \end{verse}
    
    
    \begin{verse}[0.5\linewidth]
        人,不能低下高贵的头, \\
        只有怕死鬼才乞求“自由”; \\
        毒刑拷打算得了什么? \\
        死亡也无法叫我开口!
    \end{verse}
    
    
    \begin{verse}[0.5\linewidth]
        对着死亡我放声大笑, \\
        魔鬼的宫殿在笑声中动摇; \\
        这就是我——一个共产党员的自白, \\
        高唱凯歌埋葬蒋家王朝。
    \end{verse}
    
\end{large}



\chapter{给颜黎民的信}

\begin{large}
    
    \noindent 颜黎民君:
    
    \vspace{24pt}
    
    昨天收到十日来信,知道那些书已经收到,我也放了心。你说专爱看我的书,那也许是我常论时事的缘故。不过只看一个人的著作,结果是不大好的:你就得不到多方面的优点。必须如蜜蜂一样,采过许多花,这才能酿出蜜来,倘若叮在一处,所得就非常有限,枯燥了。
    
    专看文学书,也不好的。先前的文学青年,往往厌恶数学,理化,史地,生物学,以为这些都无足重轻,后来变成连常识也没有,研究文学固然不明白,自己做起文章来也胡涂,所以我希望你们不要放开科学,一味钻进文学里。譬如说罢,古人看见月缺花残,黯然泪下,是可恕的,他那时自然科学还不发达,当然不明白这是自然现象。但如果现在的人还要下泪,那他就是胡涂虫。不过我向来没有留心儿童读物,所以现在说不出哪些书合适,开明书店出版的通俗科学书里,也许有几种,让调查一下再说罢。
    
    其次是可以看看世界旅行记,借此就知道各处的人情风俗和物产。我不知道你们看不看电影;我是看的,但不看什么“获美”“得宝”之类,是看关于非洲和南北极之类的片子,因为我想自己将来未必到非洲或南北极去,只好在影片上得到一点见识了。
    
    说起桃花来,我在上海也看见了。我不知道你到过上海没有?北京的房屋是平铺的,院子大,上海的房屋却是直叠的,连泥土也不容易看见。我的门外却有四尺见方的一块泥土,去年种了一株套话,不料今年竟也开起来,虽然少得很,但总算已经看过了罢。至于看桃花的名所,是龙华,也有屠场,我有好几个青年朋友就死在那里面,所以我是不去的。
    
    我的信如果要发表,且有发表的地方,我可以同意。我们不是没有说什么不能告人的话吗?如果有,既然说了,就不怕发表。
    
    临了,我要通知你一件你疏忽了的地方。你把自己的名字涂改了,会写错自己名字的人,是很少的,所以这是告诉了我所署的是假名。还有,我看你是看了《妇女生活》里的一篇《关于小孩子》的,是不是?
    
    就这样的结束罢。祝
    
    你们好
    
    \vspace{36pt}
    
    \begin{flushright}
        鲁迅
        
        四月十五日
        
    \end{flushright}
    
    
    
\end{large}



\chapter{落花生}

\begin{large}
    
    我们家的后园有半亩空地。母亲说:“让它荒着怪可惜的,你们那么爱吃花生,就开辟出来种花生吧。”我们姐弟几个都很高兴,买种,翻地,播种,浇水,没过几个月,居然收获了。
    
    母亲说:“今晚我们过一个收获节,请你们的父亲也来尝尝我们的新花生,好不好?”母亲把花生做成了好几样食品,还咐附就在后园的茅亭里过这个节。
    
    那晚的天色不大好,可是父亲也来了,实在很难得。
    
    父亲说:“你们爱吃花生吗?”
    
    我们争着回答:“爱!”
    
    “谁能把花生的好处说出来?”
    
    姐姐说:“花生的味道很美。”
    
    哥哥说:“花生可以榨油。”
    
    我说:“花生的价钱便宜,谁都可以买来吃,都喜欢吃。这就是它的好处。”
    
    父亲说:“花生的好处很多,有一样最可贵。它的果实埋在地里,不像桃子、石榴、苹果那样,把鲜红嫩绿的果实高高地挂在枝头上,使人一见就生爱慕之心。你们看它矮矮地长在地上,等到成熟了,也不能立刻分辨出来它有没有果实,必须挖起来才知道。”
    
    我们都说是,母亲也点点头。
    
    父亲接下去说:“所以你们要像花生,它虽然不好看,可是很有用。”
    
    我说:“那么,人要做有用的人,不要做只讲体面,而对别人没有好处的人。”
    
    父亲说:“对。这是我对你们的希望。”
    
    我们谈到深夜才散。花生做的食品都吃完了,父亲的话却深深地印在我的心上。
    
\end{large}



\chapter{难忘的一课}

\begin{large}
    
    抗日战争胜利以后,我在台湾一家航业公司的轮船上工作。
    
    有一次,我们的船停泊在高雄港\footnote{〔高雄港〕台湾西南部高雄市的港口。}。我上了岸,穿过市区,向郊外走去。不记得走了多远,看到前面有一所乡村小学,白色的围墙,门外栽着一排树。
    
    校园里很静,我走近一间教室,站在窗外,见一位年轻的台湾教师正在教孩子们学习祖国的文字。他用粉笔在黑板上一笔一画地写着:
    
    “我是中国人,我爱中国。”
    
    他写得很认真,也很吃力。台湾“光复”不久,不少教师也是重新学习祖国文字的。
    
    接着,他先用闽南语\footnote{〔闽南语〕指福建南部、广东潮汕地区、台湾等地的方言。闽:福建省的简称。},然后又用还不太熟练的国语,带着孩子们一遍一遍地读。老师和孩子们都显得那么严肃认真,又那么富有感情。好像每个字音,都发自他们火热而真挚的心。
    
    我被这动人的情景吸引住了。怀着崇高的敬意,我悄悄地从后门走进教室,在最后一排空位上坐下,和孩子们一起,跟着那位教师,大声地、整齐地、一遍又一遍地朗读着:
    
    “我是中国人,我爱中国。”
    
    老师和孩子们发现了我,但是,好像谁也没有感到意外。从那一双双眼睛里,可以看出大家对我是欢迎的。课在继续,大家读得更起劲了。
    
    下课了,孩子们把我围了起来。
    
    老师也走了过来。他热情地和我握了握手,说;“我的国语讲得不好,是初学的。你知道,在日本统治时期,我们上的都是日本人办的学校,讲国语是不准许的。”
    
    “我觉得,你今天这一课上得好极了!你教得很有感情,孩子们学得也很有感情。”
    
    接着,这位老师一定要领我去看一看他们的小礼堂。
    
    说是礼堂,不过是一间比较宽敞的屋子。
    
    他指着礼堂两面墙上新画的几幅中国历代伟人像,说:“这里原来画的都是日本人,现在‘光复’了,画上了我们中国自己的伟人。”我看到上面有孔子,有诸葛亮,有郑成功\footnote{〔郑成功〕明末清初的抗清名将,从荷兰殖民者手里收复了台湾岛,作为反抗满清的基地。},还有孙中山。看着看着,我的眼睛不觉湿润了。这是多么强烈的民族精神,多么深厚的爱国情意啊!
    
    我紧紧地握着这位年轻的台湾教师的手,激动地重复着他刚才教给孩子们的那句话:“我是中国人,我爱中国。”还有什么别的话比这句最简单的话,更能表达我此时的全部感情呢?
    
\end{large}



\chapter{毛主席在花山}

\begin{large}
    
    1948年的春夏之交,毛主席转移到了花山村。在临时借用的农家房舍里,他夜以继日地为解放全中国的事业操劳着。
    
    一天早晨,毛主席正在看地图,忽然抬起头,问警卫员:“昨天这个时候,门口花椒树下的碾子有碾米声,现在又到了碾米的时候,怎么没动静了呢?”
    
    警卫员说:“报告主席,为了不影响您工作,我和村长商量了,要他请乡亲们到别处碾去了。”毛主席皱了皱眉,把拿起来的香烟又放下了。“这怎么行?”他严肃地说,“这会影响群众吃饭的,不能因为我们在这里工作,就影响群众的生活。昨天傍晚,我们一起散步,你也看见了,这个村只有两台石碾,让乡亲们集中到一个碾子上碾米,就会耽误一半人的正常吃饭。”
    
    警卫员解释道:“这碾子一转,对您工作干扰太大。”
    
    毛主席递给他一支烟,自己也点燃了一支,说:“这怎么会呢?多年的战争生活,使我习惯了在各种环境中工作。这样吧,我交给你一个任务,尽快把乡亲们请到这里来碾米。”
    
    “是!”警卫员拔腿就走。
    
    “注意,抽着烟和群众说话是不礼貌的。说话态度要诚恳。”主席说。
    
    警卫员回头一笑:“知道了,请主席放心。”他走出小院,碰上村长正和一个端簸箕的大娘说话。警卫员迎上去,问:“村长,这位大娘是要去花椒树下推碾子吧?”
    
    大娘用手拢了拢搭在耳下的头发:“不,俺去西头。”说着转身就要走。警卫员忙对村长说:“村长,是首长让我请乡亲们来花椒树下碾米。”村长沉思了一下,说:“这碾子一响,就得转到天黑,怕误首长的事呢。”警卫员再三解释,村长才答应了,对那位大娘说:“那你就去花椒树下碾吧。”
    
    警卫员帮大娘端着盛玉米的簸箕回到了花椒树下的碾台。一会儿,陆续又来了几个碾米的老乡,碾台又吱吱扭扭地转了起来。警卫员刚回到院里,毛主席就叫他。他走进去,毛主席把笔放下,说:“任务完成得不错。还有一件事等着你办说。”说着,毛主席从桌上拿起一筒茶叶,说:“你把这筒茶叶交给炊事员,让他每天这个时候沏一桶茶水,你负责给碾米的群众送去。”
    
    警卫员知道,这筒茶叶是在南方工作的同志送的,转了几道手才送到毛主席这里,他一直没舍得喝。他站在那里,表示为难。主席说:“你想过没有?我们如果没有老百姓的支持,能有今天这个局面吗?我们吃的穿的,哪一样能离开群众的支持?全国的老百姓就是我们胜利的可靠保证。反过来讲,我们进行的斗争,也正是为了全国的老百姓。这些道理你不是不明白。依我看,你是把我摆在特殊位置上了。”警卫员只好接过茶叶筒,端端正正地向毛主席席敬了个礼。毛主席笑着说:“快去吧,炊事员还等着你呢。”
    
    茶沏好了,警卫员拎着清香的茶水来到碾台旁,用粗瓷碗一一晾在石板上,跟碾米的人说:“乡亲们,来喝茶吧。”开始,乡亲们还不好意思,经他一动员也就不拘束了,你一碗我一碗地喝了起来。茶水对这山旮旯的群众来说,确实新鲜。一位上了年纪的大叔端着一碗水,凑到警卫员跟前,说:“我说同志,这水一不甜二不辣的,喝它顶什么用?”警卫员乐呵呵地说:“您老就慢慢地喝吧,一会儿就喝出味道来了。”
    
    这时候,毛主席来了,喝茶水的乡亲们纷纷跟毛主席打招呼。毛主席笑着向大家点头,说:“要说喝茶的好处,确实不少嘛,喝了它浑身有精神,还能让人多吃饭……”毛主席给乡亲们说起喝茶的好处,正在推碾子的大娘和小姑娘越推越慢,转到毛主席身边,便停了下来。毛主席舀了两碗茶水送到她们母女手里,说:“你们俩歇会儿吧!”然后对警卫员说:“来,咱俩试试,半年多不推这玩意儿了。”毛主席推碾子还挺在行,一边推,一边用笤帚往碾盘里扫碾出来的的玉米碎粒。一位老人细细端详着毛主席,说:“这位首长,好像在哪儿见过。在哪儿呢?”
    
\end{large}



\chapter{开国大典}

\begin{large}
    
    1949年10月1日,中华人民共和国中央人民政府成立,在首都北京举行典礼。参加开国大典的,有中华人民共和国中央人民政府主席、副主席、各位委员,有中国人民政治协商会议全体代表,有工人、农民、学校师生、机关工作人员、城防部队,总数达三十万人。观礼台上还有外宾。
    
    会场在天安门广场。广场呈丁字形。丁字形一横的北面是一道河,河上并排架着五座白石桥;再北面是城墙,城墙中央高高耸起天安门的城楼。丁字形的一竖向南直伸到中华门。在一横一竖的交点的南面,场中挺立着一根电动旗杆。
    
    主席台设在天安门城楼上。城楼檐下,八盏大红宫灯分挂两边。靠着城楼左右两边的石栏,八面红旗迎风招展。
    
    丁字形的广场汇集了从四面八方来的群众队伍。早上六点钟起,就有群众的队伍入场了。人们有的擎着红旗,有的提着红灯。进入会场后,按照预定的地点排列。工人队伍中,有从老远的长辛店、丰台、通县来的铁路工人,他们清早到了北京车站,一下火车就直奔会场。郊区的农民是五更天摸着黑起床,步行四五十里路赶来的。到了正午,天安门广场已经成了人的海洋,红旗翻动,像海上的波浪。
    
    下午三点整,会场上爆发出一阵排山倒海的掌声,中华人民共和国中央人民政府主席毛泽东出现在主席台上,跟群众见面了。三十万人的目光一齐投向主席台。
    
    中央人民政府秘书长林伯渠宣布典礼开始。中央人民政府主席、副主席、各位委员就位。乐队奏起了《义勇军进行曲》。正是这战斗的声音,曾经鼓舞中国人民为新中国的诞生而奋斗。接着,毛泽东主席宣布:“中华人民共和国中央人民政府今天成立了!”
    
    这庄严的宣告,这雄伟的声音,使全场三十万人一齐欢呼起来。这庄严的宣告,这雄伟的声音,经过无线电广播,传到长城内外,传到大江南北,使全中国人民的心一齐欢跃起来。
    
    接着,升国旗。毛主席亲自按动连通电动旗杆的电钮,新中国的国旗——五星红旗在雄壮的《义勇军进行曲》中徐徐上升。三十万人一齐脱帽肃立,一齐抬起头,瞻仰这鲜红的国旗。五星红旗升起来了,表明中国人民从此站起来了。
    
    升旗的时候,礼炮响起来。每一响都是五十四门大炮齐发,一共二十八响。起初是全场肃静,只听见炮声和乐曲声,只听见国旗和许多旗帜飘拂的声音,到后来,每一声炮响后,全场就响起一阵雷鸣般的掌声。
    
    接着,毛主席在群众一阵又一阵的掌声中宣读中央人民政府的公告。他用强有力的语调向全世界发出新中国的声音。他读到“选举了毛泽东为中央人民政府主席”这一句的时候,广场上的人们热爱领袖的心情融成一阵热烈的欢呼。观礼台上同时响起一阵掌声。
    
    毛主席宣读公告完毕,阅兵式开始。中国人民解放军朱德总司令任检阅司令员,聂荣臻将军任阅兵总指挥,朱总司令和聂将军同乘汽车,先检阅部队,然后朱总司令回到主席台,宣读中国人民解放军总部的命令。受检阅的部队就由聂将军率领,在《中国人民解放军进行曲》的乐曲声中,由东往西,缓缓进场。
    
    开头是海军两个排,雪白的帽子,跟海洋一个颜色的蓝制服。接着是步兵一个师,以连为单位,列成方阵,齐步行进。接着是炮兵一个师,野炮、榴弹炮等各式各样的炮,都排成一字形的横列前进。接着是一个战车师,各种装甲车和坦克车两辆或三辆一排,整整齐齐地前进;战士们挺着胸膛站在战车上,像钢铁巨人一样。接着是一个骑兵师,“红马连”一色红马、“白马连”一色白马,六马并行,马腿的动作完全一致。以上这些部队,全部以相等的距离和相同的速度经过主席台前。当战车部队经过的时候,人民空军的飞机也一队队排成人字形,飞过天空。毛主席首先向空中招手。群众看见了,都把头上的帽子、手里的报纸和别的东西抛上天去,欢呼声盖过了飞机的隆隆声。
    
    两个半钟头的检阅,广场上不断地欢呼,不断地鼓掌,一个高潮接着一个高潮。群众差不多把嗓子都喊哑了,把手掌都拍麻了,还觉得不能够表达自己心里的欢喜和激动。
    
    阅兵式完毕,已经是傍晚的时候。天安门广场上的灯笼火把全都点起来,一万支礼花陆续射入天空。天上五颜六色的火花结成彩,地上千千万万的灯火一片红。群众游行就在这时候开始。游行队伍分东西两个方向出发,他们擎着灯,舞着火把,高呼“中国共产党万岁!”“中华人民共和国万岁!”“中央人民政府万岁!”他们一队一队按照次序走,走过正对天安门的白石桥前,就举起灯笼火把,高声欢呼“毛主席万岁!”“毛主席万岁!”毛主席在城楼上主席台前边,向前探着身子,不断地向群众挥手,不断地高呼“人民万岁!”“同志们万岁!”
    
    晚上九点半,游行队伍才完全走出会场。两股“红流”分头向东城、西城的街道流去,光明充满了整个北京城。
    
\end{large}



\chapter{狱中联欢}

\begin{large}
    
    期待的日子一转眼就来了。
    
    元旦那天早上,天还未亮,女室一带头,每一间牢房同时响应,像一阵闪电,爆发了洪亮的歌声。人们纵情高歌,唱完一支又一支。
    
    新年大联欢开始了。
    
    唱歌是第一个节目。第二个节目是交换礼品。每间牢房,每个人都准备了礼物,送给认识的或者不认识的战友,作为联欢的纪念品。最多的礼物是“贺年片”,那是用小块的草纸做的,上面用红药水画上鲜红的五角星或者镰刀锤子,写上几句互相鼓励的话。楼七室经过昼夜赶工,刻出了一百多颗红的、黄的、晶亮的五角星,分送给各个牢房的同志。女室送给各室的是一幅幅绣了字的锦旗,那些彩色的线,是从他们的袜子上拆下来的……
    
    接着第三个节目开始了。每间牢房的人都在门口贴春联。所有的春联都是用草纸接连起来做成的。所有的春联都不是一个人写的,同一个字,有老年人苍劲的笔法,也有“孩儿体”的弯弯曲曲的笔记。女室里,江姐捏着“监狱之花”的小手,也写了几笔。所有的对联,都洋溢着革命的乐观精神……
    
    女牢的对联写的是:
    
    洞中才数月
    
    世上已千年
    
    大家心里明白:几千年的封建王朝正在崩溃,人民当家作主的时代就要到来,“世上已千年”还形容不了翻天复地的革命形势的迅速发展咧!
    
    她们还在牢门上贴了一张横额:扭转乾坤。
    
    猩猩也许看不懂,也许看懂了又不敢承认,居然评论道:“这对联倒有些修仙炼道的味道了。”
    
    楼一室的对联写得更妙:
    
    歌乐山下悟道
    
    渣滓洞中参禅
    
    横额是:极乐世界。
    
    大家心里明白:这里悟的是革命之道,参的是马克思列宁主义之禅。“极乐世界”,正是写的人们掌握了革命真理的心境……
    
    猩猩挑起了眉梢,玩味了一会儿,只好说:“真有点仙风道骨!”
    
    楼二室的对联写得十分优美:
    
    看洞中依然旧景
    
    望窗外已是新春
    
    横额是:苦尽甘来。
    
    楼三室的对联借用了古人的诗句:
    
    满园春色关不住
    
    一枝红杏出墙来
    
    横额是:大地回春。
    
    一幅幅春联,全洋溢着乐观、诙谐的情趣。
    
    猩猩来到楼七室门前站定,慢吞吞地读着:“两个天窗——出气;一扇风门——伸头。”挑剔的眼光,在横额“乐在其中”四个大字上凝固起来。不待他说话,余新江便问道:
    
    “喂,这像不像渣滓洞的生活?”
    
    “生活?生活当然……”猩猩犹豫着,“不过,‘乐在其中’,那个‘乐’字总有点刺眼。”
    
    “嘿,改成‘苦’字,‘苦在其中’,你看要得不?”丁长发笑着追问道。
    
    猩猩装做没有听见,溜走了。
    
    表演节目的时间快到了,大家一拥而出,享受这自由而愉快的时刻。这个时刻,正是党的胜利,人民解放军的节节前进,给他们赢来的。
    
    几个戴着脚镣的同志,在往常放风的地坝中间扭起秧歌。沉重的铁镣,撞击得叮当作响,成了节奏强烈的伴奏。欢乐的歌舞里,充满了对黑暗势力的轻蔑。看啊,还有什么节目比得上这种顽强而鲜明的高歌漫舞!
    
    狂热的掌声送走了一间牢房的节目,又迎来了另一间牢房的表演。人潮卷来卷去,地坝变成了绝妙的露天舞台。
    
    楼下四室的“报幕员”宣布:“我们的节目是歌舞表演。表演开始!”只见铁门哗拉一开,一连串的人影,打着空心筋斗翻了出来,博得同志们的齐声喝彩。接着,几个人聚拢来,站成一个圆圈,又有几个人爬上去站在他们肩上,又有人爬上去……一层、二层、三层……他们在叠罗汉。最上边站着一个人,满脸兴奋的微笑,站得比集中营的高墙、电网更高,手里拿着一面红纸做的鲜艳的红旗,遥望着远处的云山。歌声在周围渐渐升起:一杆红旗哗啦啦地飘。一心要把革命闹;盒子枪、土枪,卡啦啦地响,打倒那劣绅和土豪!……
    
    这正是黑牢外面的游击队员最爱唱的歌。
    
    女同志们绕场一周之后,跳起了秧歌舞。彩色的舞衣飞舞着,十分耀眼。在一片叫好声和朗朗的笑声中,夹杂着一声嘲笑:“在国民党统治区里,敢跳秧歌舞?谨防上黑名单,抓走!”
    
    又一个人忍不住笑道:“那倒不一定。国民党统治区也有‘自由’的地方,不是吗?集中营里,可以自由自在地大跳秧歌舞!”
    
    “哈哈哈……”
    
    一边扭着秧歌,女同志们又齐声唱起歌来:正月里来是新春,赶着猪羊出了门,猪呀,羊呀,送到哪里去?——送给那英勇的解放军……
    
    那些想看女共产党员表演节目的特务,这时吓得脸色铁青,立刻在高墙电网上面移动机枪,枪口瞄准着欢乐的人群。
    
    有人笑道:“瞄准有什么用,蒋介石忙着喊停战,没功夫下命令开枪。”
    
    又有人笑道:“真有意思,这里又可以唱歌,又可以跳舞。开联欢会,还有人架上机枪,保卫我们的安全咧!”
    
    “哈哈哈哈!”人们朗声大笑,迎接着1949年胜利的春天。
    
\end{large}



\chapter{凡卡}

\begin{large}
    
    九岁的凡卡·茹科夫,三个月前给送到鞋匠阿里亚希涅那儿做学徒。圣诞节前夜,他没躺下睡觉。他等老板、老板娘和几个伙计到教堂做礼拜去了,就从老板的立柜里拿出一小瓶墨水,一支笔尖生了锈的钢笔,摩平一张揉皱了的白纸,写起信来。
    
    在写第一个字母以前,他担心地朝门口和窗户看了几眼,又斜着眼看了一下那个昏暗的神像,神像两边是两排架子,架子上摆满了楦头\footnote{〔楦头〕制鞋、制帽时所用的模型,大多是木头做的。}。他叹了一口气,跪在作台前边,把那张纸铺在作台上。
    
    “亲爱的爷爷康司坦丁·玛卡里奇,”他写道,“我在给您写信。祝您过一个快乐的圣诞节,求上帝保佑您。我没爹没娘,只有您一个亲人了。”
    
    凡卡朝黑糊糊的窗户看看,玻璃窗上映出蜡烛的模糊的影子;他想象着他爷爷康司坦丁·玛卡里奇,好像爷爷就在眼前。爷爷是日发略维夫老爷家里的守夜人。他是个非常有趣的瘦小的老头儿,六十五岁,老是笑咪咪地眨着眼睛。白天,他总是在大厨房里睡觉。到晚上,他就穿上宽大的羊皮袄,敲着梆子,在别墅的周围走来走去。老母狗卡希旦卡和公狗泥鳅低着头跟在他后头。泥鳅是一条非常听话非常讨人喜欢的狗。它身子是黑的,像黄鼠狼那样长长的,所以叫它泥鳅。
    
    现在,爷爷一定站在大门口,眯缝着眼睛看那乡村教堂的红亮的窗户。他一定在跺着穿着高筒毡靴的脚,他的梆子挂在腰带上,他冻得缩成一团,耸着肩膀……
    
    天气真好,晴朗,一丝风也没有,干冷干冷的。那是没有月亮的夜晚,可是整个村子——白房顶啦,烟囱里冒出来的一缕缕的烟啦,披着浓霜一身银白的树木啦,雪堆啦,全看得见。天空撒满了快活地眨着眼睛的星星,天河显得很清楚,仿佛为了过节,有人拿雪把它擦亮了似的……
    
    凡卡叹了口气,蘸了蘸笔尖,接着写下去。
    
    “昨天晚上我挨了一顿毒打,因为我给他们的小崽子摇摇篮的时候,不知不觉睡着了。老板揪着我的头发,把我拖到院子里,拿皮带揍了我一顿。这个礼拜,老板娘叫我收拾一条青鱼,我从尾巴上弄起,她就捞起那条青鱼,拿鱼嘴直戳我的脸。伙计们捉弄我,他们打发我上酒店去打酒。吃的呢,简直没有。早晨吃一点儿面包,午饭是稀粥,晚上又是一点儿面包;至于菜啦,茶啦,只有老板自己才大吃大喝。他们叫我睡在过道里,他们的小崽子一哭,我就别想睡觉,只好摇那个摇篮。亲爱的爷爷,发发慈悲吧,带我离开这儿回家,回到我们村子里去吧!我再也受不住了!……我给您跪下了,我会永远为您祷告上帝。带我离开这儿吧,要不,我就要死了!……”
    
    凡卡撇撇嘴,拿脏手背揉揉眼睛,抽噎了一下。
    
    “我会替您搓烟叶,”他继续写道,“我会为您祷告上帝。要是我做错了事,您就结结实实地打我一顿好了。要是您怕我找不着活儿,我可以去求那位管家的,看在上帝面上,让我擦皮鞋;要不,我去求菲吉卡答应我帮他放羊。亲爱的爷爷,我再也受不住了,只有死路一条了!……我原想跑回我们村子去,可是我没有鞋,又怕冷。等我长大了,我会照顾您,谁也不敢来欺负您。
    
    “讲到莫斯科,这是个大城市,房子全是老爷们的,有很多马,没有羊,狗一点儿也不凶。圣诞节,这里的小孩子并不举着星星灯走来走去,教堂里的唱诗台不准人随便上去唱诗。有一回,我在一家铺子的橱窗里看见跟钓竿钓丝一块出卖的钓钩,能钓各种各样的鱼,很贵。有一种甚至钓得起一普特\footnote{〔普特〕俄国重量单位,一普特等于16.38千克。}重的大鲇鱼呢。我还看见有些铺子卖各种枪,有一种跟我们老板的枪一样,我想一杆枪要卖一百个卢布吧。肉店里有山鹬啊,鹧鸪啊,野兔啊……可是那些东西哪儿打来的,店里的伙计不肯说。
    
    “亲爱的爷爷,老爷在圣诞树上挂上糖果的时候,请您摘一颗金胡桃,藏在我的绿匣子里头。”
    
    凡卡伤心地叹口气,又呆呆地望着窗口。他想起到树林里去砍圣诞树的总是爷爷,爷爷总是带着他去。多么快乐的日子啊!冻了的山林喳喳地响,爷爷冷得吭吭地咳,他也跟着吭吭地咳……要砍圣诞树了,爷爷先抽一斗烟,再吸一阵子鼻烟,还跟冻僵的小凡卡逗笑一会儿……许多小枞树披着浓霜,一动不动地站在那儿,等着看哪一棵该死。忽然不知从什么地方跳出一只野兔来,箭一样的窜过雪堆。爷爷不由得叫起来,“逮住它,逮住它,逮住它!嘿,短尾巴鬼!”
    
    爷爷把砍下来的树拖回老爷家里,大家就动手打扮那棵树。
    
    “快来吧,亲爱的爷爷,”凡卡接着写道,“我求您看在基督的面上,带我离开这儿。可怜可怜我这个不幸的孤儿吧。这儿的人都打我。我饿得要命,又孤零零的,难受得没法说。我老是哭。有一天,老板那楦头打我的脑袋,我昏倒了,好容易才醒过来。我的生活没有指望了,连狗都不如!……我问候阿辽娜,问候独眼的艾果尔,问候马车夫。别让旁人拿我的小风琴。您的孙子伊凡·茹科夫。亲爱的爷爷,来吧!”
    
    凡卡把那张写满字的纸折成四折,装进一个信封里,那个信封是前一天晚上花了一个戈比买的。他想了一想,蘸一蘸墨水,写上地址。
    
    “乡下 爷爷收”
    
    然后他抓抓脑袋,再想一想,添上几个字。
    
    “康司坦丁·玛卡里奇”
    
    他很满意没人打搅他写信,就戴上帽子,连破皮袄都没披,只穿着衬衫,跑到街上去了……前一天晚上他问过肉店的伙计,伙计告诉他,信应该丢在邮筒里,从那儿用邮车分送到各地去。邮车上还套着三匹马,响着铃铛,坐着醉醺醺的邮差。凡卡跑到第一个邮筒那儿,把他那宝贵的信塞了进去。
    
    过了一个钟头,他怀着甜蜜的希望睡熟了。他在梦里看见一铺暖炕,炕上坐着他的爷爷,耷拉着两条腿,正在念他的信……泥鳅在炕边走来走去,摇着尾巴……
    
\end{large}



\chapter{金色的鱼钩}

\begin{large}
    
    1935年秋天,红四方面军\footnote{〔红四方面军〕红军主力部队之一,1931年成立,1935年3月开始长征,1936年10月与红一方面军胜利会师。1937年抗日战争全面爆发后,改编为国民革命军第八路军第129师。}进入草地,许多同志因此得了肠胃病。我和两个小同志病得实在赶不上队伍了,指导员派炊事班长照顾我们,让我们走在后面。
    
    炊事班长快四十岁了,个子挺高,背有点儿驼,四方脸,高颧骨,脸上布满了皱纹。全连数他岁数大,对大家又特别亲,大伙都叫他“老班长”。
    
    三个病号走不快,一天只走二十来里路。一路上,老班长带我们走一阵歇一阵。到了宿营地,他就到处去找野菜,和着青稞面\footnote{〔青稞面〕青稞磨成的面,是藏区的主食。青稞:一种大麦,生长在高原寒冷地区。}给我们做饭。不到半个月,两袋青稞面吃完了。饥饿威胁着我们。老班长到处找野菜,挖草根,可是光吃这些东西怎么行呢!老班长看我们一天天瘦下去,他整夜整夜地合不拢眼,其实,这些天他比我们瘦得还厉害呢!
    
    一天,他在一个水塘边给我们洗衣裳,忽然看见一条鱼跳出水面。他喜出望外地跑回来,取出一根缝衣针,烧红了,弯成个钓鱼钩。这天夜里,我们就吃到了新鲜的鱼汤。尽管没有调料,可我们觉得没有比这鱼汤更鲜美的了,端起碗来吃了个精光。
    
    从那以后,老班长尽可能找有水塘的地方宿营,把我们安顿好,就带着鱼钩出去了。第二天,他总能端着热气腾腾的鲜鱼野菜汤给我们吃。我们虽然还是一天一天衰弱下去,但比起光吃草根野菜来毕竟好多啦。可是老班长自己呢,我从来没见他吃过一点儿鱼。
    
    有一次,我禁不住问他:“老班长,你怎么不吃鱼啊?”
    
    他摸了摸嘴,好像回味似的说:“吃过了。我一起锅就吃,比你们还先吃呢。”
    
    我不信,等他收拾完碗筷走了,就悄悄地跟着他。走近前一看,啊!我不由得呆住了。他坐在那里捧着搪瓷碗\footnote{〔搪瓷〕在金属表面涂上玻璃质材料后高温烧制而成的器物。比陶瓷耐摔,不易生锈。},嚼着几根草根和我们吃剩下的鱼骨头,嚼了一会儿,就皱紧眉头硬咽下去。我觉得好像有万根钢针扎着喉管,失声喊起来:“老班长,你怎么……”
    
    老班长猛抬起头,看见我目不转睛地看着他手里的搪瓷碗,就支吾着说:“我,我早就吃过了。看到碗里还没吃干净,扔了怪可惜的……”
    
    “不,我全知道了。”我打断了他的话。
    
    老班长转身朝两个小同志睡觉的地方看了一眼,一把把我搂到身边,轻声说:“小声点儿,小梁!咱们俩是党员,你既然知道了,可不要再告诉别人。”
    
    “可是,你也要爱惜自己啊!”
    
    “不要紧,我身体还硬实!”他抬起头,望着夜色弥漫的草地,好久,才用低沉的声音说,“指导员把你们三个人交给我,他临走的时候说:‘他们年轻。一路上,你是上级,是保姆,是勤务员啊,无论多么艰苦,也要把他们带出草地。’小梁,你看这草地,无边无涯,没个尽头。我估计,还要二十天才能走出去。熬过这二十天不简单啊!眼看你们的身子一天比一天衰弱,只要哪一天吃不上东西,说不定就会起不来,真有个三长两短,我怎么去向党报告呢?难道我能说,指导员,我把同志们留在草地上,我自己克服了困难出来啦!”
    
    “可是,你总该跟我们一起吃一点儿啊!”
    
    “不行,太少啦。”他轻轻地摇摇头,“小梁,说真的,弄点儿吃的不容易啊!有时候等了半夜,也不见鱼上钩。为了弄一点儿鱼饵,我翻了多少草皮也找不到一条蚯蚓……还有,我的眼睛坏了,天色一暗,找野菜就得一棵一棵地摸……”
    
    我再也忍不住了,抢着说:“老班长,以后我帮你一起找,我看得见。”
    
    “不,咱们不是早就分好工了吗?再说,你的病也不轻,不好好休息会撑不住的。”
    
    我还坚持我的意见。老班长忽然严厉地说:“小梁同志,共产党员要服从党的分配。你的任务是坚持走路,安定两个小同志的情绪,增强他们的信心!”
    
    望着他那十分严峻的脸,我一句话也说不上来,竟扑倒在他怀里哭了。
    
    第二天,老班长端来的鱼汤特别少,每个搪瓷碗里只有小半条猫鱼,上面漂着一丁点儿野菜。他笑着说:“吃吧,就是少了点儿。唉!一条好大的鱼已经上了钩,又跑啦!”
    
    我端起搪瓷碗,觉得这个碗有千斤重,怎么也送不到嘴边。两个小同志不知道为什么,也端着碗不往嘴边送。老班长看到这情况,收敛了笑容,眉头拧成了疙瘩。他说:“怎么了,吃不下?要是不吃,咱们就走不出这草地。同志们,为了革命,你们必须吃下去。小梁,你不要太脆弱!”最后这句话是严厉的,意思只有我知道。
    
    我把碗端到嘴边,泪珠大颗大颗地落在热气腾腾的鱼汤里。我悄悄背转身,擦擦眼睛,大口大口地咽着鱼汤。老班长看着我们吃完,脸上的皱纹舒展开了,嘴边露出了一丝笑意。可是我的心里好像塞了铅块似的,沉重极了。
    
    挨了一天又一天,渐渐接近草地的边了,我们的病却越来越重。我还能勉强挺着走路,那两个小同志连直起腰来的力气也没有了。老班长虽然瘦得只剩皮包骨头,眼睛深深地陷了下去,但是还一直用饱满的情绪鼓励着我们。我们就这样扶一段,搀一段,终于走到草地边上。远处,重重叠叠的山峰已经看得见了。
    
    这天上午,老班长快活地说:“同志们,咱们在这儿停一下,好好弄点儿吃的,鼓一鼓劲,一口气走出草地去。”说罢,他就拿起鱼钩找水塘去了。
    
    我们的精神特别好,四处去找野菜,拾干草,好像过节似的。但是过了好久,还不见老班长回来。我们四面寻找,最后在一个水塘旁边找到了他,他已经昏迷不醒了。
    
    我们都着慌了。过雪山的时候有过不少这样的例子,战士用惊人的毅力支持着自己的生命,但是一倒下去就再也起不来了。要挽救老班长,最好的办法是让他赶快吃些东西。我们立即分了工,我去钓鱼,剩下的一个人照料老班长,一个人生火。
    
    我蹲在水边,心里不停地念叨:“鱼啊!快些来吧!这是挽救一个革命战士的生命啊!”可是,等了好久,还不见鱼上钩。好容易看到漂在水面的芦秆动了一下,赶紧扯起钓竿,还是空的……
    
    当我俯下身子,把鱼汤送到老班长嘴边的时候,老班长已经奄奄一息了。他微微地睁开眼睛,看见我端着的鱼汤,头一句话就说:“小梁,别浪费东西了。我……我不行啦。你们吃吧!还有二十多里路,吃完了,一定要走出去!”
    
    “老班长,你吃啊!我们抬也要把你抬出去!”我几乎要哭出来了。
    
    “不,你们吃吧。你们一定要走出草地去!见着指导员,告诉他,我没完成党交给我的任务,没把你们照顾好。看,你们都瘦得……”
    
    老班长用粗糙的手无力地抚摸着我的头。突然间,他的手垂了下去。
    
    “老班长!老班长!”我们失声地叫着。但老班长,还是慢慢地闭上了眼睛。
    
    我们扑在老班长身上,抽噎着,很久很久。
    
    擦干了眼泪,我把老班长留下的鱼钩小心地包起来,放在贴身的衣兜里。我想,等革命胜利以后,一定要把这个闪烁着金色光芒的鱼钩送到革命烈士纪念馆去,让我们的子孙都来瞻仰它。
    
\end{large}


\newpage

\textbf{注释}:

\vspace{-1em}

\begin{itemize}
    \setlength\itemsep{-0.2em}
    \item 〔抽噎〕一吸一顿地哭泣,抽泣。
    \item 〔瞻仰〕抬头看。比喻恭敬地观看。
    \item 〔支吾〕用含糊的话语躲闪应付。
    \item 〔喜出望外〕结果超过了期望,因而特别高兴。
    \item 〔奄奄一息〕只剩下微弱的气息。形容呼吸微弱,就要死亡。
\end{itemize}

\chapter{一夜的工作}

\begin{large}
    
    周总理在“中国文学艺术工作者第一次代表大会”上作了报告。《人民文学》杂志要发表这个报告,由我作了整理,把稿子送给他审阅。
    
    一天,总理办公室通知我到中南海政务院去。我走进他的办公室。这是高大的宫殿式的房子,室内陈设极其简单,一个不大的写字台,两张小转椅,一盏台灯,如此而已。总理见了我,指着写字台上厚厚的一叠文件说:“我今晚上要批这些文件。你们送来的稿子,我放在最后。你到隔壁值班室去睡一觉,到时候叫你。”
    
    我就到值班室去睡了。不知到了什么时候,值班室的同志把我叫醒。他对我:“总理叫你去。”
    
    我立刻起来,揉揉矇眬的睡眼,走进总理的办公室。总理招呼我坐在他的写字台对面,要我陪他审阅我整理的稿子,这是以备咨询的意思。他一句一句地审阅,看完一句就用铅笔在那一句后面画上一个小圆圈。他不是普通的浏览,而是一边看一边在思索,有时停笔想一想,有时还问我一两句。夜很静,经过相当长时间他才审阅完,把稿子交给了我。
    
    这时候,值班室的同志送来两杯热腾腾的绿茶,一小碟花生米,放在写字台上。总理让我跟他一起喝茶,吃花生米。花生米并不多,可以数得清颗数,好像并没有因为今夜多了一个人而增加了分量。喝了一会儿茶,就听见公鸡喔喔喔地叫明了。总理站了起来对我说:“我要去休息了。上午睡一觉,下午还要参加活动。你也回去睡觉吧。”
    
    我也站了起来,站起来的时候把小转椅的上面部分带歪了。他过来把我的转椅扶正,然后就走进后面去了。
    
    在回来的路上,我不断地想着,并且对自己说:“这就是我们新中国的总理。我看见了他一夜的工作。他是多么劳苦,多么简朴!”
    
    我这样对自己说了几遍,我又想高声对全世界说,好像我的声音全世界都能听见似的:“看啊,这就是我们中华人民共和国的总理。我看见了他一夜的工作。他每个夜晚都是这样工作的。你们看见过这样的总理吗?”
    
\end{large}



\chapter{林海}

\begin{large}
    
    我总以为大兴安岭\footnote{〔大兴安岭〕兴安岭是我国黑龙江省的山脉。黑龙江南岸、嫩江以西称大兴安岭,嫩江以东称小兴安岭。}奇峰怪石,高不可攀。这回有机会看到它,并且走进原始森林,脚落在积得几尺厚的松针上,手摸到那些古木,才证实了这个悦耳的名字是那种亲切与舒服。
    
    大兴安岭这个“岭”字,跟秦岭\footnote{〔秦岭〕中国重要的山脉之一,横亘于中国中部,是中国南北地理的分界线。}的“岭”字可大不一样。这里岭的确很多,横着的,顺着的高点儿的,矮点儿的,长点儿的,短儿点的,可是没有一条使人想起“云横秦岭”那种险境。多少条岭啊,在疾驰的火车上看了几个钟头,既看不完,也看不厌。每条岭都是那么温柔,自山脚至岭顶长满了珍贵的树木,谁也不孤峰突起,盛气凌人。
    
    目之所及,哪里都是绿的。的确是林海。群岭起伏是林海的波浪。多少种绿颜色呀:深的,浅的,明的,暗的,绿得难以形容。恐怕只有画家才能够描绘出这么多的绿色来呢!
    
    兴安岭上千般宝,第一应夸落叶松。是的,这是落叶松的海洋。看,海边上不是还泛着白色的浪花吗?那是些俏丽的白桦,树干是银白色的。在阳光下,大片青松的边沿闪动着白桦的银裙,不是像海边上的浪花吗?
    
    两山之间往往流动着清可见底的小河。河岸上有多少野花呀。我是爱花的人,到这里我却叫不出那些花的名儿来。兴安岭多么会打扮自己呀:青松作衫,白桦为裙,还穿着绣花鞋。连树与树之间的空隙也不缺乏色彩:松影下开着各种的小花,招来各色的小蝴蝶——它们很亲热地落在客人的身上。花丛里还隐藏着珊瑚珠似的小红豆,兴安岭中酒厂所造的红豆酒就是用这些小野果酿成的,味道很好。
    
    看到那数不尽的青松白桦,谁能不向四面八方望一望呢?有多少省市用过这里的木材呀!大至矿井、铁路,小至椽柱\footnote{〔椽柱〕泛指造房屋的木料。椽:支撑屋顶的横木。}、桌椅。千山一碧,万古常青,恰好与广厦\footnote{〔广厦〕高大的房屋。}、良材联系在一起。所以,兴安岭越看越可爱!它的美丽就与建设结为一体,美的并不空洞,叫人心中感到亲切、舒服。
    
    及至看到了林场,这种亲切之感便更加深厚了。我们伐木取材,也造林护苗,一手砍,一手栽。我们不仅取宝,也作科学研究,使林海不但能够万古长青,而且可以综合利用。山林中已经有不少的市镇,给兴安岭添上了新的景色,添上了愉快的劳动歌声。人与山的关系日益密切,怎能不使我们感到亲切、舒服呢?我不晓得当初为什么管它叫做兴安岭,由今天看来,它的确含有兴国安邦的意义。
    
\end{large}


\newpage

\textbf{注释}:

\vspace{-1em}

\begin{itemize}
    \setlength\itemsep{-0.2em}
    \item 〔目之所及〕眼睛能看到的地方。
    \item 〔盛气凌人〕气势盛大,压迫着别人。形容骄横傲慢。
    \item 〔空洞〕不实在,缺少内容。
    \item 〔万古长青〕永远像春天的草木一样青翠茂盛,形容精神、事业等永远存在。
    \item 〔兴国安邦〕使国家兴盛平安。
\end{itemize}

\chapter{第一场雪}

\begin{large}
    
    前天,气象台发布了西伯利亚\footnote{〔西伯利亚〕亚洲北部地区,冬季形成寒流南下,进入我国。}来的寒流即将南下的消息;昨天,强大的冷空气就侵入了胶东半岛。
    
    前些天还暖和得如同阳春三月,昨天清早,天气骤然变冷,空中布满了铅色的阴云,中午,凛冽的寒风刮起来了,呼呼地刮了整整一个下午。黄昏时分风停了,就下起雪来。这是入冬以来的第一场雪。
    
    开始下雪时还伴着小雨,不久就只见鹅毛般的雪花,从彤云密布的天空中飘落下来,地上一会儿就白了。冬天的山村到了夜里格外寂静,只听见雪花簌簌地不断下落。偶尔咯吱一声响。树木的枯枝被积雪压断了。
    
    大雪整整下了一夜。早晨,天放晴了,太阳出来了。推开门一看,嗬!好大的雪啊!山川、树木、房屋,全部罩上了一层厚厚的雪,万里江山变成了粉妆玉砌的世界。落光叶子的柳树上,挂满了毛茸茸、亮晶晶的银条儿;冬夏常青的松树和柏树,堆满了蓬松松、沉甸甸的雪球。一阵风吹来,树枝轻轻地摇晃,银条儿和雪球儿簌簌地落下来,玉屑似的雪末儿随风飘扬,映着清晨的阳光,显出一道道五光十色的彩虹。
    
    大街上的积雪有一尺多深,脚踩上去发出咯吱咯吱的响声。一群群孩子在雪地里堆雪人,掷雪球。
    
    俗话说,“瑞雪兆丰年”。这并不是迷信,有着充分的科学根据。寒冬大雪可以冻死一部分越冬的害虫;雪水渗进土层深处,又能供应支离破庄稼生长的需要。这场大雪十分及时,一定会给明年的小麦带来丰收。有经验的老农把雪比作是“麦子的棉被”,冬天“棉被”盖得越厚,春天麦子就长得越好。所以又有这样一句谚语:“冬天麦盖三层被,来年枕着馒头睡。”
    
    我想:这就是人们为什么把及时的大雪称为“瑞雪”的道理吧。
    
\end{large}



\chapter{灯光}

\begin{large}
    
    我爱到天安门广场走走,尤其是晚上。广场上千万盏灯静静地照耀着天安门广场周围的宏伟建筑,使人心头感到光明,感到温暖。
    
    清明节前的一个晚上,我又漫步在广场上,忽然背后传来一声赞叹:“多好啊!”我心头微微一震,是什么时候听到过这句话来着?噢,对了,那是很久以前了。于是,我沉入了深深的回忆。
    
    1947年的初秋,当时我是战地记者。挺进豫皖苏平原\footnote{〔豫皖苏平原〕河南省、安徽省、江苏省接壤的大片土地,华北平原的南部。}的我军部队,把国民党军五十七师紧紧地包围在一个叫沙土集的村子里。激烈的围歼战就要开始了。天黑的时候,我摸进一片茂密的沙柳林,在匆匆挖成的交通沟\footnote{〔交通沟〕用于转移、转运的壕沟。}里找到了突击连,来到了郝副营长的身边。
    
    郝副营长是一位著名的战斗英雄,虽然只有22岁,已经打过不少仗了。今晚就由他带领突击连去攻破守敌的围墙,为全军打通歼灭敌军的道路。大约一切准备工作都完成了,这会儿,他正倚着交通沟的胸墙\footnote{〔胸墙〕壕沟两侧的土堆,可挡子弹。}坐着,一手拿着火柴盒,夹着自制的烟卷,一手轻轻地划着火柴。他并没有点烟,却借着微弱的亮光看摆在双膝上的一本破旧的书。书上有一幅插图,画的是一盏吊着的电灯,一个孩子正在灯下聚精会神地读书。他注视着那幅图,默默地沉思着。
    
    “多好啊!”他在自言自语。突然,他凑到我的耳朵边轻轻地问:“记者,你见过电灯吗?”
    
    我不由得一楞,摇了摇头,说:“没见过。”我说的是真话。我从小生活在农村,真的没见过电灯。
    
    “听说一按电钮,那玩意儿就亮了,很亮很亮……”他又划着一根火柴,点燃了烟,又望了一眼图画,深情地说,“赶明儿胜利了,咱们也能用上电灯,让孩子们都在那样亮的灯光底下学习,该多好啊!”他把头靠在墙上,望着漆黑的夜空,完全陷入了对未来的憧憬里。
    
    半个小时以后,我刚回到团指挥所,战斗就打响了。三发绿色的信号弹升上天空,接着就是震天动地的炸药包爆炸声。守敌的围墙被炸开一个缺口,突击连马上冲了进去。没想到后续部队遭到敌人炮火猛烈的阻击,在黑暗里找不到突破口,和突击连失去了联系。
    
    整个团指挥所的人都焦急地钻出了地堡,望着黑魆魆的围墙。突然,黑暗里出现一星火光,一闪,又一闪。这火光虽然微弱,对于寻找突破口的部队来说已经够亮了。战士们靠着这微弱的火光冲进了围墙,响起了一片喊杀声。
    
    后来才知道,在这千钧一发的时刻,是郝副营长划着了火柴,点燃了那本书,举得高高的,为后续部队照亮了前进的路。可是,火光暴露了他自己,他被敌人的机枪打中了。
    
    这一仗,我们消灭了敌人的一个整编师。战斗结束后,我们把郝副营长埋在茂密的沙柳丛里。这位年轻的战友不惜自己的性命,为了让孩子们能够在电灯底下学习,他自己却没有来得及见一见电灯。
    
    事情已经过去很长时间了。在天安门前璀璨的华灯下,我又想起这位亲爱的战友来。
    
\end{large}


\newpage

\textbf{注释}:

\vspace{-1em}

\begin{itemize}
    \setlength\itemsep{-0.2em}
    \item 〔憧憬〕期待,向往。
    \item 〔黑魆魆〕黑暗无光,黑漆漆。魆:昏暗。
    \item 〔璀璨〕光彩鲜明绚丽。
    \item 〔千钧一发〕千钧的重物吊在一根头发上。比喻极其危急。钧:重量单位,一钧三十斤。
\end{itemize}

\chapter{卖火柴的小女孩}

\begin{large}
    
    天冷极了,下着雪,又快黑了。这是一年的最后一天——大年夜。在这又冷又黑的晚上,一个乖巧的小女孩,赤着脚在街上走着。她从家里出来的时候还穿着一双拖鞋,但是有什么用呢?那是一双很大的拖鞋——那么大,一向是她妈妈穿的。她穿过马路的时候,两辆马车飞快地冲过来,吓得她把鞋都跑掉了。一只怎么也找不着,另一只叫一个男孩捡起来拿着跑了。他说,将来他有了孩子可以用它当摇篮。
    
    小女孩只好赤着脚走,一双小脚冻得红一块青一块的。她的旧围裙里兜着许多火柴,手里还拿着一把。这一整天,谁也没买过她一根火柴,谁也没给过她一个硬币。
    
    可怜的小女孩!她又冷又饿,哆哆嗦嗦地向前走。雪花落在她金黄的长头发上,那头发打成卷披在肩上,看上去很美丽,不过她没注意这些。每个窗子里都透出灯光来,街上飘着一股烤鹅的香味,因为这是大年夜——她可忘不了这个。
    
    她在一座房子的墙角里坐下来,蜷着腿缩成一团。她觉得更冷了。她不敢回家,因为她没卖掉一根火柴,没挣到一个钱,爸爸一定会打她的。再说,家里跟街上一样冷。他们头上只有个房顶,虽然最大的裂缝已经用草和破布堵住了,风还是可以灌进来。
    
    她的一双小手几乎冻僵了。啊,哪怕一根小小的火柴,对她也是有好处的!她敢从成把的火柴里抽出一根,在墙上擦燃了,来暖和暖和自己的小手吗?她终于抽出了一根。哧!火柴燃起来了,冒出火焰来了!她把小手拢在火焰上。多么温暖多么明亮的火焰啊,简直像一支小小的蜡烛。这是一道奇异的火光!小女孩觉得自己好像坐在一个大火炉前面,火炉装着闪亮的铜脚和铜把手,烧得旺旺的,暖烘烘的,多么舒服啊!唉,这是怎么回事呢?她刚把脚伸出去,想让脚也暖和一下,火柴灭了,火炉不见了。她坐在那儿,手里只有一根烧过了的火柴梗。
    
    她又擦了一根。火柴燃起来了,发出亮光来了。亮光落在墙上,那儿忽然变得像薄纱那么透明,她可以一直看到屋里。桌上铺着雪白的台布,摆着精致的盘子和碗,肚子里填满了苹果和梅子的烤鹅正冒着香气。更妙的是这只鹅从盘子里跳下来,背上插着刀和叉,摇摇摆摆地在地板上走着,一直向这个穷苦的小女孩走来。这时候,火柴灭了,她面前只有一堵又厚又冷的墙。
    
    她又擦着了一根火柴。这一回,她坐在美丽的圣诞树下。这棵圣诞树,比她去年圣诞节透过富商家的玻璃门看到的还要大,还要美。翠绿的树枝上点着几千支明晃晃的蜡烛,许多幅美丽的彩色画片,跟挂在商店橱窗里的一个样,在向她眨眼睛。小女孩向画片伸出手去。这时候,火柴又灭了。只见圣诞树上的烛光越升越高,最后成了在天空中闪烁的星星。有一颗星星落了下来,在天空中划出了一道细长的红光。“有一个什么人快要死了。”小女孩说。唯一疼她的奶奶活着的时候告诉过她:一颗星星落下来,就有一个人要离去了。
    
    她在墙上又擦着了一根火柴。这一回,火柴把周围全照亮了。奶奶出现在亮光里,是那么温和,那么慈爱。“奶奶!”小女孩叫起来,“啊!请把我带走吧!我知道,火柴一灭,您就会不见的,像那暖和的火炉,喷香的烤鹅,美丽的圣诞树一个样,就会不见的!”
    
    她赶紧擦着了一大把火柴,要把奶奶留住。一大把火柴发出强烈的光,照得跟白天一样明亮。奶奶从来没有像现在这样高大,这样美丽。奶奶把小女孩抱起来,搂在怀里。她俩在光明和快乐中飞走了,越飞越高,飞到那没有寒冷,没有饥饿,也没有痛苦的地方去了。
    
    第二天清晨,这个小女孩坐在墙角,两腮通红,嘴上带着微笑。她死了,在旧年的大年夜冻死了。新年的太阳升起来了,照在她小小的尸体上。小女孩坐在那儿,手里还捏着一把烧过了的火柴梗。
    
    “她想给自己暖和一下……”人们说。谁也不知道她曾经看到过多么美丽的东西,她曾经多么幸福,跟着她奶奶一起走向新年的幸福中去。
    
\end{large}



\chapter{将相和}

\begin{large}
    
    战国时,秦国很强大,常常进攻别的国家。
    
    有一回,赵王得了一件稀世之宝,叫和氏璧\footnote{〔和氏璧〕古代名玉,相传是楚国人卞和发现,因此叫和氏璧。}。秦王知道了,就写了一封信给赵王,说是愿意拿十五座城换这块璧。
    
    赵王接到信后非常着急,立即召集大臣来商议。大家说秦王不过想把和氏璧骗到手罢了,不能上他的当;可要是不答应,又怕他派兵来进攻。
    
    正在为难的时候,有人说有个叫蔺相如的人,勇敢机智,也许他能解决这个难题。
    
    赵王把蔺相如找来,问他该怎么办。
    
    蔺相如想了一会儿,说:“如果秦国提出用城换璧,我国却不答应,那理亏的是我们。如果我们把和氏璧给了秦国,秦国却不给我们十五座城,那理亏的就是他们。我愿意带着和氏璧到秦国去。如果秦王真的拿十五座城来换,我就把璧交给他;如果他不肯交出十五座城,我一定把璧完好无缺地送回来。”于是赵王就派蔺相如带着和氏璧去了秦国。
    
    蔺相如到了秦国,进宫见了秦王,献上和氏璧。秦王双手捧住璧,一边看一边称赞,绝口不提十五座城的事。蔺相如看这情形,知道秦王没有拿城换璧的诚意,就上前一步,说:“这块璧有点儿小毛病,让我指给您看。”秦王听他这么一说,就把和氏璧交给蔺相如。蔺相如捧着璧,往后退了几步,靠着柱子站定。他怒发冲冠,说:“我看您并不想交付十五座城。现在璧在我手里,您要是强逼我,我的脑袋就和璧一起撞碎在这柱子上!”说着,他举起和氏璧就要向柱子上撞。秦王怕他真的把璧撞碎了,连忙说一切都好商量,就叫人拿出地图,把允诺划归赵国的十五座城指给他看。蔺相如说和氏璧是无价之宝,要举行个隆重的典礼,他才肯交出来。秦王只好跟他约定了举行典礼的日期。
    
    蔺相如反复思量,觉得秦王还是不会信守承诺的,一到客舍,就叫手下人化了装,带着和氏璧抄小路先回赵国去了。到了举行典礼那一天,蔺相如进宫见了秦王,说:“秦国的国君历来不守信用,我怕有负赵王所托,已经让人把和氏璧送回赵国了。如果您有诚意,先把十五座城交给我国,我国马上派人把璧送来。我们怎么敢为了一块璧而得罪强大的秦国呢?我知道欺骗了您是死罪,您可以杀了我,但请好好考虑我的话。”秦王没有办法,只得客客气气地把蔺相如送回了赵国。
    
    蔺相如立了大功,赵王封他做上大夫。
    
    过了几年,秦王约赵王在渑池\footnote{〔渑池〕古城名,在今河南省西部。}会面。赵王胆怯,不敢去。但蔺相如和大将军廉颇认为对秦王不能示弱,还是去的好,赵王才决定动身,让蔺相如随行。廉颇带着军队送他们到边境上,作好了抵御秦军的准备。
    
    赵王到渑池与秦王会面。秦王要赵王鼓瑟。赵王不好推辞,鼓了一段。秦王就叫人记录下来,说在渑池会上,秦王令赵王鼓瑟。
    
    蔺相如看秦王存心侮辱赵王,便向前走了几步,说:“赵王听说秦王擅长秦国的音乐,希望您能击缶助兴。”秦王很生气,拒绝了。蔺相如再次上前要求,秦王还是拒绝。蔺相如说:“您现在离我只有五步远。如果您不答应,我就跟您同归于尽!”秦王左右的卫士想杀了蔺相如,但蔺相如怒目圆睁,厉声呵斥,卫士竟不敢上前。秦王被逼得没办法,只好敲了一下缶。蔺相如也叫人记录下来,说在渑池会上,秦王为赵王击缶。
    
    秦国的大臣不甘心,继续发难,但蔺相如毫不示弱,直到会面结束,秦王也没占到便宜。秦王知道廉颇已经在边境上作好了准备,不敢拿赵王怎么样,只好让赵王回去。
    
    蔺相如在渑池会上又立了功。赵王封蔺相如为上卿,职位比廉颇还高。
    
    廉颇很不服气,他对别人说:“我廉颇立下了那么多战功,他蔺相如就靠一张嘴,反而爬到我头上去了。要是我碰见他,一定要让他下不来台!”蔺相如听说了,就请病假不上朝,免得跟廉颇见面。
    
    有一天,蔺相如坐车出去,远远看见廉颇过来了,他赶紧叫车夫把车往回赶。蔺相如的门客们可的睦不顺眼了,对蔺相如说:“您见了廉颇像老鼠见了猫似的,为什么要怕他呢?”蔺相如说:“诸位请想一想,廉将军和秦王比,谁厉害?”门客们说:“当然是秦王厉害!”蔺相如说:“秦王我都不怕,还会怕廉将军吗?秦王之所以不敢进攻我们赵国,就是因为武有我们两个人在。如果我们俩闹不和,就会削弱赵国的力量,秦国必然乘机来攻打我们。我之所以避着廉将军,为的是我们赵国啊!”
    
    蔺相如的话传到了廉颇的耳朵里。廉颇静下心来想了想,觉得自己为了争一口气,就不顾国家利益,真不应该。于是,他脱下战袍,背上绑着荆条,到蔺相如门上请罪。蔺相如见廉颇来请罪,连忙出来迎接。从此以后,他们俩成了好朋友,同心协力保卫赵国。
    
\end{large}


\newpage

\textbf{注释}:

\vspace{-1em}

\begin{itemize}
    \setlength\itemsep{-0.2em}
    \item 〔稀世〕世上少有。形容珍贵。
    \item 〔理亏〕不占理,按道理来说有错。
    \item 〔璧〕中间有孔的玉盘。
    \item 〔瑟〕一种拨弦乐器。鼓瑟:弹奏瑟。
    \item 〔缶〕盛酒水的瓦器,也作敲击乐器。
\end{itemize}

\chapter{梅花魂}

\begin{large}
    
    故乡的梅花又开了。一年一度,那朵朵冷艳、缕缕幽芳的梅花,总让我想起飘泊他乡、葬身异国的外祖父。
    
    我出生在东南亚的星岛,从小和外祖父生活在一起。外祖父年轻时读了不少经、史、诗、词,又能书善画,在星岛文坛颇负盛名。我很小的时候,外祖父常常抱着我,坐在梨花木大交椅上,一遍又一遍、不厌其烦地教我读唐诗宋词。每当读到“独在异乡为异客,每逢佳节倍思亲”“春草明年绿,王孙归不归”“自在飞花轻似梦,无边丝雨细如愁”之类的句子,常会有一颗两颗冰凉的泪珠落在我的腮边、手背。这时候,我会拍着手笑起来:“外公哭了!外公哭了!”老人总是摇摇头,长长地叹一口气,说:“莺儿,你还小呢,不懂!”
    
    外祖父家中有不少古玩,我偶尔摆弄,老人也不甚在意。唯独书房里那一幅老干虬枝的墨梅图,他分外爱惜,家人碰也碰不得。我五岁那年,有一回到书房玩耍,不小心在上面留了个脏手印,外祖父顿时拉下脸来。有生以来,我第一次听到他训斥我母亲:“孩子要管教好,这清白的梅花,是能玷污的吗?”训罢,便用刀片轻轻刮去污迹,又用细绸子慢慢抹净。看见慈祥的外祖父大发脾气,我心里又害怕又奇怪:一幅画而已,有什么稀罕的呢?
    
    有一天,母亲忽然跟我说:“莺儿,我们要回中国去!”
    
    “干吗要回去呢?”
    
    “那儿才是我们的祖国啊!”
    
    哦!祖国,就是那拥有长江、黄河、万里长城的地方吗?我欢呼起来,小小的心充满了欢乐。
    
    可是,我马上想起外祖父,我亲爱的外祖父。我问母亲:“外公走吗?”
    
    “外公年纪太大了……”
    
    我跑进外祖父的书房,老人正躺在藤椅上。我说:“外公,您也回祖国去吧!”
    
    想不到外祖父竟像小孩子一样,呜呜呜地哭了起来……
    
    离别的前一天早上,外祖父早早地起了床,把我叫到书房里,郑重地递给我一卷白杭绸包着的东西。我打开一看,原来是那幅墨梅图,就说:“外公,这不是您最宝贵的画吗?”
    
    “是啊,莺儿,你要好好保存!这梅花,是我们中国最有名的花。旁的花,大抵是春暖才开花。她却不一样,愈是寒冷,愈是风欺雪压,花开得愈精神,愈秀气。她是最有品格、最有灵魂、最有骨气的!几千年来,我们中华民族出了许多有气节的人物,他们不管历经多少磨难,不管受到怎样的欺凌,从来都是顶天立地,不肯低头折节。他们就像这梅花一样。一个中国人,无论在怎样的境遇里,总要有梅花的秉性才好!”
    
    回国的那一天正是元旦,虽然热带是无所谓隆冬的,但腊月天气,毕竟也凉飕飕的。外祖父把我们送到码头。风撩乱了老人平日梳理得整整齐齐的银发,我觉得外祖父一下子衰老了许多。
    
    船快开了,母亲只好狠下心来,拉着我登上大客轮。想不到眼含泪水的外祖父也随着上了船,递给我一块手绢——雪白的细亚麻布上绣着血色的梅花。
    
    当年的我,还过于稚嫩,并不懂得,我带走的,岂止是我慈爱的外祖父珍藏的一幅丹青、几朵血梅?我带走的,是身在异国的华侨老人一颗眷恋祖国的赤子心啊!
    
\end{large}



\chapter{为人民服务}

\begin{large}
    
    我们的共产党和共产党所领导的八路军、新四军,是革命的队伍。我们这个队伍完全是为着解放人民的,是彻底地为人民的利益工作的。张思德同志就是我们这个队伍中的一个同志。
    
    人总是要死的,但死的意义有不同。中国古时候有个文学家叫做司马迁的说过:人固有一死,或重于泰山,或轻于鸿毛。为人民利益而死,就比泰山还重;替法西斯\footnote{〔法西斯〕20世纪诞生的一种政治思想,主张独裁集权下的集体主义、民族主义,用军事武力维护民族资产阶级利益。}卖力,替剥削人民和压迫人民的人去死,就比鸿毛还轻。张思德同志是为人民利益而死的,他的死是比泰山还要重的。
    
    因为我们是为人民服务的,所以,我们如果有缺点,就不怕别人批评指出。不管是什么人,谁向我们指出都行。只要你说得对,我们就改正。你说的办法对人民有好处,我们就照你的办。“精兵简政”这一条意见,就是党外人士李鼎铭先生\footnote{〔李鼎铭〕清末民国的进步人士。辛亥革命后,极力拥护孙中山的革命主张,兴办新学。抗日战争爆发后,积极拥护中国共产党抗日统一战线,当选陕甘宁边区政府副主席。1947年病逝。}提出来的;他提得好,对人民有好处,我们就采用了。只要我们为人民的利益坚持好的,为人民的利益改正错的,我们这个队伍就一定会兴旺起来。
    
    我们都是来自五湖四海,为了一个共同的革命目标,走到一起来了。我们还要和全国大多数人民走这一条路。我们今天已经领导着有九千一百万人口的根据地,但是还不够,还要更大些,才能取得全民族的解放。我们的同志在困难的时候,要看到成绩,要看到光明,要提高我们的勇气。中国人民正在受难,我们有责任解救他们,我们要努力奋斗。要奋斗就会有牺牲,死人的事是经常发生的。但是我们想到人民的利益,想到大多数人民的痛苦,我们为人民而死,就是死得其所。不过,我们应当尽量地减少那些不必要的牺牲。我们的干部要关心每一个战士,一切革命队伍的人都要互相关心,互相爱护,互相帮助。
    
    今后我们的队伍里,不管死了谁,不管是炊事员,是战士,只要他是做过一些有益的工作的,我们都要给他送葬,开追悼会。这要成为一个制度。这个方法也要介绍到老百姓那里去。村上的人死了,开个追悼会。用这样的方法,寄托我们的哀思,使整个人民团结起来。
    
\end{large}



\chapter{难忘的启蒙}

\begin{large}
    
    我时常怀着深深的感激之情,思念着我的启蒙老师们。是他们,在我童稚的心灵里播下美好的种子,教导我:要爱祖国,要勤勉,要做一个正直的、诚实的人。几十年过去了,老师们的话仿佛还在我的耳边回响。
    
    我的启蒙学堂叫竺西小学,它坐落在一个江南小镇的街外。我还依稀记得那狭窄的天井,晦暗的教室和没有座位、只有一个石砌的小“舞台”的礼堂……记得在这个礼堂里我们有过的永生难忘的集会。
    
    那大概是1942年,沦陷时期的艰难岁月,我上小学三年级的时候,老师们曾组织全校的学生在这里举行过多次抗日讲演比赛。我也登过台,讲演稿是级任老师冯先生帮我写的。
    
    学校离日本兵的炮楼很近,只二百来米。讲演比赛时,专门有人在校门口放哨,见到日本兵或翻译官经过,就跑进来报告,讲演随即暂时停止,大家一起唱歌。
    
    那时我还小,不大懂得这件事可能带来的后果。后来,当我知道日本侵略者是怎样残忍地虐杀中国的爱国者的时候,我对老师们的勇敢,不能不从心底里感到无限的敬佩。
    
    我虚岁六岁就上学了,年纪小,上课时总是很规矩地坐在前排,老师们都很喜欢我。他们说过不少表扬我的话,这些我已经谈忘了;可我还是免不了受到批评乃至惩戒,这方面的情景我倒是至今未曾忘却。
    
    在班上,作文和写字算是我的“强项”了,然而就是在这两门课上,我也受到过申斥。有一次作文,题目是“记秋游”。在文章的开头,我说:“星期天的早晨,我和几个同学在西街外的草场上玩,忽然闻到一阵桂花香,我们就一起到棠下村摘桂花去了。”陈先生阅后在末尾批了“嗅觉特长”四个字。我不明白这批语的含义,就去问。先生板着面孔对我说:“棠下离这儿有三里路,那里的桂花香你们也闻得见,难道鼻子有这么长吗?”这话有点儿刺伤我,不过我还是感到羞愧,因为我确实没有闻到桂花香,开头那几句是凭想象编造出来的。有一次上大字课,老师在发本子时把我叫到讲台前,严肃地对我说:“你这次的成绩是丙,丙就是及格了,可对你来说,这是不及格,因为你本该得甲的。以后再这样,就要打手心了。”当着全班同学的面这样说我,我感到有点儿难堪,不过我心里还是服气的,因为那节大字课的前半堂我净和同座的同学说悄悄话了,字确实写得很不用心。
    
    在我的印象里,只有对一门课,老师们的态度特别宽容,那就是翻译官上的日语。即使逃课,老师们也是不管的。从这种宽严之间,我们这些小学生也领悟到了老师们没有明说的某些道理。
    
    解放后的第二年,我到北京参加了工作。从那以后,漫长的岁月过去了,经历的事情许多也已淡忘,而少年时代生活的情景,启蒙老师们的音容笑貌,还不时地在我的记忆中浮现,引起我的思念和遐想。
    
    从五十年代后期开始,我也走上了我的启蒙老师们走过的路,成了一名教师。当我站在讲坛上向年轻人宣讲自己所崇奉的信念的时候,我会想起我的启蒙老师们。我由此想到,老师们在平凡的教学岗位上付出的一切不会是徒劳的。既然我的老师们播下的种子在他们学生的身上开花结果了,我们播下的种子有什么理由不在自己学生的身上开花结果呢?
    
\end{large}



\chapter{珊瑚}

\begin{large}
    
    大海退潮了,海面上露出了美丽的珊瑚,有红的,有白的,还有花的。它们有的像鹿角,有的像扇面,有的像菊花,有的像树枝。
    
    人们看到珊瑚的色彩这样美丽,形状这样奇怪,以为它们是长在海底的植物。其实它们不是植物,是珊瑚虫分泌出来的石灰质。
    
    珊瑚虫是浅海里的一种小动物。它们生活在海底洁净的岩石上,比米粒还小。它们长着花瓣一样的触手,触手中间有一株很小的嘴,猎取比它更小的生物当食物。它们不断地分泌石灰质,这些石灰质连在一起,就形成了各种各样美丽的珊瑚。
    
    珊瑚虫一代又一代地在岩石上生长,繁殖,死亡。经过几万年,它们遗留下来的石灰质就成了珊瑚礁。再经过几万年,有的珊瑚礁露出海面,就成了珊瑚岛。我国的西沙群岛,就是由许多珊瑚岛组成的。
    
\end{large}



\chapter{三亚落日}

\begin{large}
    
    在三亚看落日真有诗意。夕阳滑落的景象美妙绝伦,一点儿也不比日出逊色。
    
    三亚在海南岛的最南端,被蓝透了的海水围着,洋溢着浓浓的热带风情。蓝蓝的天与蓝蓝的海融为一体,低翔的白鸥掠过蓝蓝的海面,真让人担心洁白的翅尖会被海水蘸蓝了。挺拔俊秀的椰子树,不时在海风中摇曳着碧玉般的树冠。海滩上玉屑银末般的细沙,金灿灿亮闪闪的,软软地暖暖地搔着人们的脚板,谁都想捏一捏,团一团,将它揉成韧韧的面。
    
    活跃了一天的太阳,依旧像一个快乐的孩童。它歪着红扑扑的脸蛋,毫无倦态,潇潇洒洒地从身上抖落下赤朱丹彤,在大海上溅出无数夺目的亮点。于是,天和海都被它的笑颜感染了,金红一色,热烈一片。
    
    时光悄悄地溜走,暑气跟着阵阵海风徐徐地远离。夕阳也渐渐收敛了光芒,变得温和起来,像一只光焰柔和的大红灯笼,悬在海与天的边缘。兴许是悬得太久的缘故,只见它慢慢地下沉,刚一挨到海面,又平稳地停住了。它似乎借助了大海的支撑,再一次任性地在这张硕大无朋的床面上顽皮地蹦跳。大海失去了原色,像饱饮了玫瑰酒似的,醉醺醺地涨溢出光与彩。人们惊讶得不敢眨眼,生怕眨眼的一瞬间,那盏红灯笼会被一只巨手提走。我瞪大双眼正在欣赏着,突然那落日颤动了两下,最后像跳水员那样,轻快一跳,以一个悄然无声、水波不惊的优美姿势入了水,向人们道了“再见”。
    
    哦,这就是三亚的落日!
    
\end{large}


\newpage

\textbf{注释}:

\vspace{-1em}

\begin{itemize}
    \setlength\itemsep{-0.2em}
    \item 〔诗意〕像诗歌一样优美的意境。
    \item 〔美妙绝伦〕美妙而独特。绝伦:独一无二,没有类似的。
    \item 〔逊色〕比不上,更差。
    \item 〔蘸〕在粉、液中沾一点就拿出来。
    \item 〔韧〕容易变形但不容易断裂。
    \item 〔醉醺醺〕喝醉酒的样子
    \item 〔赤朱丹彤〕从鲜艳到浓烈的红色。指各种各样的红色。
    \item 〔热带〕地球南北回归线之间的地带,四季炎热。三亚处于热带。
    \item 〔水波不惊〕不引起水花波浪。惊:引动。
\end{itemize}

\chapter{富饶的西沙群岛}

\begin{large}
    
    西沙群岛位于南海的西北部,是我国海南省三沙市的一部分。那里风景优美,物产丰富,是个可爱的地方。
    
    西沙群岛一带海水五光十色,瑰丽无比:有深蓝的,淡青的,浅绿的,杏黄的。一块块,一条条,相互交错着。因为海底高低不平,有山崖,有峡谷,海水有深有浅,从海面看,色彩就不同了。
    
    海底的岩石上长着各种各样的珊瑚,有的像绽开的花朵,有的像分枝的鹿角。海参到处都是,在海底懒洋洋地蠕动。大龙虾全身披甲,划过来,划过去,样子挺威武。
    
    鱼成群结队地在珊瑚丛中穿来穿去,好看极了。有的全身布满彩色的条纹;有的头上长着一簇红缨;有的周身像插着好些扇子,游动的时候飘飘摇摇;有的眼睛圆溜溜的,身上长满了刺,鼓起气来像皮球一样圆。各种各样的鱼多得数不清。正像人们说的那样,西沙群岛的海里一半是水,一半是鱼。
    
    西沙群岛也是鸟的天下。岛上有一片片茂密的树林,树林里栖息着各种海鸟。遍地都是鸟蛋。树下堆积着一层厚厚的鸟粪,这是非常宝贵的肥料。
    
    富饶的西沙群岛,是我们祖祖辈辈生活的地方。随着祖国建设事业的发展,可爱的西沙群岛,必将变得更加美丽,更加富饶。
    
\end{large}


\newpage

\textbf{注释}:

\vspace{-1em}

\begin{itemize}
    \setlength\itemsep{-0.2em}
    \item 〔缨〕系在枪脖子上的须带。像缨的形状。
    \item 〔富饶〕丰富充足。
    \item 〔五光十色〕发出多种颜色的光芒,形容色彩鲜亮纷多,光彩照人。
    \item 〔瑰丽〕珍稀、美丽、奇特。瑰:似玉的奇石。
    \item 〔蠕动〕虫类爬行的样子。
\end{itemize}

\chapter{带刺的朋友}

\begin{large}
    
    秋天,枣树上挂满了红枣,风儿一吹,轻轻摆动,如同无数颗飘香的玛瑙\footnote{〔玛瑙〕一种细纹宝石,有绿、红、黄、褐、白多种颜色,常加工成蛋状或果实状。}晃来晃去,看着就让人眼馋。
    
    一天晚上,新月斜挂,朦胧的月光透过树枝,斑斑驳驳地洒在地上。我刚走到后院的枣树旁边,忽然看见一个圆乎乎的东西,正慢慢地往树上爬……
    
    我非常惊讶,赶忙贴到墙根\footnote{〔墙根〕墙的下段靠近地面的部分,墙脚。},注视着它的一举一动。
    
    “是猫,还是别的什么?”我暗暗猜想。
    
    那个东西一定没有发现我在监视它,仍旧鬼祟地爬向老树杈,又爬向伸出的枝条……
    
    挂满红枣的枝杈慢慢弯下来。
    
    后来,那个东西停住了脚,兴许\footnote{〔兴许〕也许,或许。}是在用力摇晃吧,树枝哗哗作响,红枣噼里啪啦地落了一地。
    
    我还没弄清楚是怎么回事,树上那个家伙就噗的一声掉了下来。听得出,摔得还挺重呢!
    
    我恍然大悟:这不是刺猬吗?
    
    很快,它又慢慢地活动起来了,看样子,劲头比上树的时候足多了。它匆匆地爬来爬去,把散落的红枣逐个归拢到一起,然后就地打了一个滚儿。你猜怎么着,归拢的那堆红枣,全都扎在它的背上了。立刻,它的身子“长”大了一圈。也许是怕被人发现吧,它驮着满背的红枣,向着墙角的水沟眼儿,急火火地跑去了……
    
    我暗暗钦佩:聪明的小东西,偷枣的本事可真高明啊!
    
    可是,它住在什么地方呢?离这儿远不远?窝里还有没有伙伴?好奇心驱使我蹑手蹑脚地追到水沟眼儿,弯腰望去,水沟眼儿里黑洞洞的,小刺猬已经没了踪影。
    
\end{large}


\newpage

\textbf{注释}:

\vspace{-1em}

\begin{itemize}
    \setlength\itemsep{-0.2em}
    \item 〔斑斑驳驳〕色彩、光影相杂,好像许多斑点。
    \item 〔朦胧〕月光不明,看不清。
    \item 〔监视〕一直关注,要知道动向。
    \item 〔鬼祟〕不想让人发现,偷偷摸摸。
    \item 〔急火火〕匆忙,着急而欠考虑。
    \item 〔枝杈〕树枝。杈:分枝,泛指树枝。
    \item 〔钦佩〕尊敬佩服。
    \item 〔蹑手蹑脚〕不想让人发现而悄悄走动的样子。
\end{itemize}

\chapter{科学家竺可桢}

\begin{large}
    
    春天,在北海公园\footnote{〔北海公园〕北京市中心区的公园。},常常能看见一位清瘦的老人。他早晨从北门进园,南门出去,晚上从南门进来,北门出去。这位老人就是科学家竺可桢。竺可桢每天上下班,正好经过北海公园。他本来可以坐汽车去上班,但是他宁愿步行穿过公园。从建国以来,走了十几个年头。
    
    竺可桢走北海公园,可不是抱着游人的心情来观赏景物,而是以一个物候学家的身份来观察物候。冰融花开,絮飞燕到,在他看来,都是物候学的信号。物候学和气象学可以说是姐妹学科。不同在于,气象学是观测和记录一个地方的冷暖晴雨,风云变幻,而求其原因和趋向;物候学则是记录一年中植物的生长荣枯,动物的来往生育,了解气候变化和它对动植物的影响。物候学研究的目的是认识自然季节现象变化的规律,以服务于农业生产和科学研究。竺可桢是研究我国物候学的倡导者。
    
    在北海岸边,竺可桢细心观察:哪天桃花开了,哪天柳絮\footnote{〔柳絮〕成熟的柳树的种子,上面有白色绒毛。}飞了,哪天布谷鸟叫了。这些自然现象的变化,他都作了翔实的记录。遇到工作紧张或者外出,就让他爱人帮着留心燕子什么时候飞来,也让他女儿帮着观察北海的冰什么时候初融,还让邻居的孩子向他报告哪天杏花开了第一朵……
    
    每天早晨一起来,他就把那支放在白铜套子里的钢笔式的温度表拿到院子里放好,然后做早操。做完早操,又把温度表拿进屋里,记录量得的气温。这支温度表,他经常插在外衣左边的小口袋里。长久地插来插去,小口袋的盖布总是先磨坏了。这样,做衣服的时候,他爱人就请成衣工人多做一片小口袋盖布,留着拆换。
    
    经过多年的观察,他积累了丰富的物候记录数据,绘制了北京春季物候现象变化的曲线图\footnote{〔曲线图〕在平面上用曲线表示数量关系的图。}。这幅图表明了1950至1972年的23年中北京春季物候变化的迟早顺序,为编制自然历提供了科学的依据。在他的著作中,我们可以看到这幅有意义的曲线图。
    
    为了使科学研究更好地服务于生产,竺可桢并不限于在北海观察。他在七十多岁的时候,还换上耐磨的网球鞋,到野外去工作。去时总带着那钢笔式的温度表,还带着罗盘\footnote{〔罗盘〕在地面上确定方向的一种仪器,又叫指南针。}、高度表\footnote{〔高度表〕确定到指定水平面的垂直距离的一种仪器。}和照相机。这是他的随身四宝。每到一处,总是先拿出罗盘定方向,接着用高度表测量海拔,用温度表测量气温,再用照相机把一些景观照下来,作为科学研究的资料。经过深入的调查研究和实践,他写出了《关于我国气候若干特点与粮食作物生产的关系》这篇学术论文,综合地分析了光、温度和降水对作物生长的影响,判断我国农业生产还有很大潜力,指出发挥这些潜力应采取的若干途径,给气候工作和农业生产开辟了崭新的前景。
    
\end{large}


\newpage

\textbf{注释}:

\vspace{-1em}

\begin{itemize}
    \setlength\itemsep{-0.2em}
    \item 〔清瘦〕瘦。
    \item 〔崭新〕新。
    \item 〔综合〕把不同的事物、情况、部分结合在一起。
    \item 〔翔实〕详实,详细而确实。
    \item 〔潜力〕潜在的能力、力量。潜在:如同潜在水里,存在事物中,尚未被发觉、尚未表现出来的。
\end{itemize}

\chapter{白鹭}

\begin{large}
    
    白鹭是一首精巧的诗。
    
    色素的配合,身段的大小,一切都很适宜。
    
    白鹤太大而嫌生硬,即如粉红的朱鹭或灰色的苍鹭,也觉得大了一些,而且太不寻常了。
    
    然而白鹭却因为它的常见,而被人忘却了它的美。
    
    那雪白的蓑毛\footnote{〔蓑毛〕像蓑衣草一样的羽毛,为白鹭特有。},那利流的身形,那铁色的长喙,那青色的脚,增之一分又嫌长,减之一分则嫌短,素之一忽又嫌白,黛之一忽则嫌黑。
    
    在清水田里时有一只两只站着钓鱼,整个的田便成了一幅嵌在玻璃框里的画面。田的大小好像有心人为白鹭设计的镜匣\footnote{〔镜匣〕一种梳妆用的盒子,里面装有可以支起来的镜子。匣:开盖的小盒子。}。
    
    晴天的清晨,每每看见它孤独地站立在小树的绝顶,看来像是不安稳,而它却很悠然。这是别的鸟很难表现的一种嗜好。人们说它是在望哨,可它真是在望哨吗?
    
    黄昏的空中偶见白鹭的低飞,更是乡居生活中的一种恩惠。那是“清澄”二字的具现,而且具有了生命了。
    
    或许有人会感到美中的不足,白鹭不会唱歌。但是白鹭的本身不就是一首很优美的歌吗?——不,歌未免太铿锵了。
    
    白鹭实在是一首诗,一首韵在骨子里的散文的诗。
    
\end{large}


\newpage

\textbf{注释}:

\vspace{-1em}

\begin{itemize}
    \setlength\itemsep{-0.2em}
    \item 〔喙〕鸟嘴。
    \item 〔色素〕作为元素的颜色。
    \item 〔嗜好〕特别爱好;爱好的东西。
    \item 〔一忽〕一点点,一分。
    \item 〔嫌〕过分而不好。
    \item 〔绝顶〕最高处。
    \item 〔利流〕在流动的风或水中阻力很小的(形状)。
    \item 〔望哨〕警戒观望,放哨。
    \item 〔悠然〕安闲的样子。
    \item 〔铿锵〕形容乐器声音响亮节奏分明,也用来形容诗词文曲声调响亮,节奏明快。
    \item 〔清澄〕清澈透明纯洁。
    \item 〔具现〕具体呈现。
\end{itemize}

\chapter{向祖国致敬}

\begin{large}
    
    今天我们重新踏上祖国的土地,觉得无限的愉快和兴奋。过去四五年来,因为美国政府无理的羁留\footnote{〔美国政府……〕美国发起冷战、入侵朝鲜后,一方面禁止中国留美学生回国,一方面又视其为敌人而软禁、监视。严重妨碍留美学生学习、工作。},归国无期,天天在焦虑和气愤中过活。现在靠了政府在外交上严正有力的支持,和全世界爱好和平的人民在舆论上的援助,我们才能安然返国,我们向政府和所有帮助我们的人民致谢。
    
    回想解放以前,人民生活困苦,国际地位低落。再看见现在的祖国,充满着生气和希望,处处在大量的建设,人人都快乐地奋进,短短的几年中有这样的成就,简直是一个奇迹。我们深知这奇迹是国内父老兄弟姊妹们在中国共产党正确的领导下,用血汗争取得来的。但在那最艰苦的解放建国初期中,我们身在海外,无法来尽我们应尽的责任。今天却回来分享这做一个新中国人民的光荣,实在非常惭愧。
    
    从旧社会里出来,又多年生活在一个资本主义\footnote{〔资本主义〕一种以生产资料私有制和批量生产技术为基础、以私有资本雇佣劳动作为基本生产关系、以市场作为基本资源配置方式的社会经济制度。}的国家里,一旦回到人民民主主义\footnote{〔人民民主主义〕指对人民实行民主的思想。}的新中国来,思想上一定会落后,不自觉地仍有余毒。我们要抱着决心,处处去向人民学习,同时我们要全心全力在英明的政府的领导下,来参加建国工作,向社会主义\footnote{〔社会主义〕以生产资料公有制为基础,共同占有生产成果、按劳分配的社会经济制度。}的光明前途迈进。
    
    \hfill 克利夫兰轮十月八日由美抵港全体归国同学
    
    \hfill 钱学森、蒋英、王祖耆、何国柱、沈学均、李整武、洪用林、胡聿贤、陈炳兆、孙湘、陆孝颐、许国志、许顺生、张士铎、张发慧、冯启德、疏松桂、蒋丽金、刘豫麒、刘尔雄、刘骊生、戴月棣、萧伦、萧蓉春。
    
\end{large}


\newpage

\textbf{注释}:

\vspace{-1em}

\begin{itemize}
    \setlength\itemsep{-0.2em}
    \item 〔羁留〕拘留、扣留。限制尚未判决的人的行动自由。
    \item 〔姊妹〕姐妹。
    \item 〔惭愧〕因有缺点、有错误、未能尽责任等原因而感到不安或羞耻。
    \item 〔海外〕国外,通常指距离较远或隔海的外国。
    \item 〔英明〕明智而有远见。
\end{itemize}

\chapter{王二小}

\begin{large}
    
    \begin{verse}[0.5\linewidth]
        牛儿还在山坡吃草, \\
        放牛的却不知哪儿去了。 \\
        不是他贪玩耍丢了牛, \\
        那放牛的孩子王二小。
    \end{verse}
    
    
    \begin{verse}[0.5\linewidth]
        九月十六那天早上, \\
        敌人向一条山沟扫荡。 \\
        山沟里掩护着后方机关, \\
        掩护着几千老乡。
    \end{verse}
    
    
    \begin{verse}[0.5\linewidth]
        正在那十分危急的时候, \\
        敌人快要走到山口。 \\
        昏头昏脑地迷失了方向, \\
        抓住了二小要他带路。
    \end{verse}
    
    
    \begin{verse}[0.5\linewidth]
        二小他顺从地走在前面, \\
        把敌人带进我们的埋伏圈。 \\
        四下里乒乒乓乓响起了枪炮, \\
        敌人才知道受了骗。
    \end{verse}
    
    
    \begin{verse}[0.5\linewidth]
        敌人把二小挑在枪尖, \\
        摔死在大石头的上面。 \\
        我们那十三岁的王二小, \\
        英勇的牺牲在山间。
    \end{verse}
    
    
    \begin{verse}[0.5\linewidth]
        干部和老乡得到了安全, \\
        他却睡在冰冷的山间。 \\
        他的脸上含着微笑, \\
        他的血染红蓝蓝的天。
    \end{verse}
    
    
    \begin{verse}[0.5\linewidth]
        秋风吹遍了这个村庄, \\
        它把这动人的故事传扬。 \\
        每一个老乡都含着眼泪, \\
        歌唱着二小放牛郎。
    \end{verse}
    
\end{large}



\chapter{啊,姑娘再见!}

\begin{large}
    
    \begin{verse}[0.5\linewidth]
        这一天清晨,从梦中醒来。 \\
        啊,姑娘再见了, \\
        再见吧,再见吧! \\
        这一天清晨,从梦中醒来, \\
        侵略者闯进了我家。
    \end{verse}
    
    
    \begin{verse}[0.5\linewidth]
        啊游击队啊,快带我走吧。 \\
        啊,姑娘再见了, \\
        再见吧,再见吧! \\
        啊游击队啊,快带我走吧, \\
        我实在不能再忍受。
    \end{verse}
    
    
    \begin{verse}[0.5\linewidth]
        如果我不幸牺牲在战场。 \\
        啊,姑娘再见了, \\
        再见吧,再见吧! \\
        如果我不幸牺牲在战场, \\
        你一定要把我埋葬。
    \end{verse}
    
    
    \begin{verse}[0.5\linewidth]
        请把我葬在高高的山岗。 \\
        啊,姑娘再见了, \\
        再见吧,再见吧! \\
        请把我葬在高高的山岗, \\
        再插上朵美丽的花。
    \end{verse}
    
    
    \begin{verse}[0.5\linewidth]
        每当后人们从这里走过。 \\
        啊,姑娘再见了, \\
        再见吧,再见吧! \\
        每当后人们从这里走过, \\
        都说:“多么美丽的花!”
    \end{verse}
    
    
    \begin{verse}[0.5\linewidth]
        这朵花属于游击队战士。 \\
        啊,姑娘再见了, \\
        再见吧,再见吧! \\
        这朵花属于游击队战士, \\
        他为自由献出生命!
    \end{verse}
    
\end{large}



\chapter{种金子}

\begin{large}
    
    蓝蓝的天上没有一丝云,阳光照在市集的小路上。巴依老爷拿着账本,又来催账了。
    
    巴依老爷敲着账本,对艾尔肯大爷说:“你去年向我借了两个鸡蛋,如今你要还我五十个银币。”
    
    “天呐!五十个银币,够买一头羊了。两个鸡蛋,哪里值这么多钱?”大爷委屈地嚷嚷起来。
    
    “你不懂!你自个儿好好算算。这两个鸡蛋要不是给你吃了,早就变成公鸡和母鸡了。母鸡又能下蛋,蛋又能孵出鸡来。你算算,你拿了我的蛋,让我少了几只鸡?”巴依老爷说着,眼泪都要掉下来了,仿佛自己才是那个受了委屈的人。
    
    巴依老爷又对着古丽小姑娘说:“还有你!你之前向我借了一口铁锅,如今你要还我一百个银币。”
    
    “那口锅还好着呢,我把锅还给你不就成了。一百个银币,够买好几口铁锅了。”小姑娘也不高兴了。
    
    巴依老爷转了转眼珠,说:“你不懂!我这口锅是母锅。要是不借给你,不知能生几口小锅呢!你要是还我的锅,就把小锅也还上。”
    
    “天呐!我简直一点儿也不明白!”巴依老爷走了以后,大爷感叹了起来。
    
    路过的阿凡提听到了,就过来问:“大爷,您遇到什么事了?”
    
    “巴依老爷财迷心窍,想钱想疯了!”大爷就把巴依老爷的说法告诉了阿凡提。阿凡提听得皱起了眉头。
    
    过了两天,阿凡提也到巴依老爷家做客。巴依老爷的锅里正在煮肉。巴依老爷指着锅里的肉,问阿凡提:“这肉可真香啊,你也闻到了吧?”
    
    阿凡提说:“当然,这肉可真香啊,恐怕连真主\footnote{〔真主〕伊斯兰教崇拜的神。}也闻到了。”
    
    “既然闻到了,就该付钱了吧?”巴依老爷得意地伸出了手掌。
    
    “闻个香味也要付钱吗?”阿凡提问。
    
    “那当然!香味也是肉的一部分嘛!品尝了肉香,就是享用了我的肉。”巴依老爷眨着眼睛说。
    
    “那好吧,如您所愿。”阿凡提把钱袋子拿出来,在巴依老爷面前晃了晃,里面的银币哐当作响。
    
    “听到了吗?”阿凡提问。
    
    “听到了,听到了!是银币的响声!”巴依老爷听得心欢喜,伸手就要拿钱袋,阿凡提又把钱袋收回去了。
    
    “怎么不给钱?”
    
    “既然您听过钱的响声了,那就够了。”
    
    “怎么能这么说!听个响也算付钱吗?”
    
    “那当然!响声也是钱的一部分嘛!我让您欣赏了钱的响声,就算给肉香付钱了。”
    
    巴依老爷没占到便宜。他看着阿凡提的钱袋,突然想到:这个阿凡提,怎么突然有钱了?怕不是偷来的,还是骗来的。我要好好查查这事,不能让这些刁民占了便宜。
    
    太阳落山了,巴依老爷偷偷跟着阿凡提回家。他躲在墙缝里偷看,只看到阿凡提在后院的一棵歪脖子树下挖了个坑,把钱袋子里的银币倒出来,埋进挖好的坑里。阿凡提一边埋银币,一边口里念念有词。
    
    巴依老爷记下了埋银币的地方。每天晚上,他都去查看。第三天的晚上,阿凡提又到了后院,从树下挖出了昨晚埋下的银币,居然装满了两个钱袋子!
    
    巴依老爷惊呆了。他想:“好家伙,这个阿凡提,不知道从哪里学来了这等法术,种下一袋银币,居然能收获两袋银币!”他又想:“要是我家的金银财宝也能这么种,那该有多好!”他从藏身的地方走出来,对阿凡提说:“阿凡提,你从哪里学来这种钱的法子?”
    
    阿凡提吓了一跳,说:“哪有什么种钱的法子?这是我昨天埋下的。”
    
    巴依老爷说:“别骗我了,我都看见了。种下十枚银币,收获二十枚银币。你得教我这种钱的法子。我给你一箱金子,你给我种出两箱金子来。”
    
    阿凡提说:“这可不行。”
    
    巴依老爷生气了:“你要是不教我,我就告到阿訇\footnote{〔阿訇〕伊斯兰教对主持宗教事务人员的称呼。}那里去,说你不守教规,偷窃真主的财富!”
    
    阿凡提吓住了,只好说:“那好吧,如您所愿。您把金子给我,三天之后收成。这法子见不得光,见不得人。您切不可对别人说起,也不能在白天里挖开来看,否则这法子就不灵了。”
    
    巴依老爷高兴极了,连忙从家里运来一大箱金子,看着阿凡提在树下挖了一个大坑,把金子都埋进去了。
    
    想到三天之后就能收获两箱金子,巴依老爷就高兴得睡不着觉,每天都想着树底下埋着的金子。到了第三天早上,他实在不放心,就到阿凡提家里来。阿凡提不在家,巴依老爷跑到后院,悄悄把土翻开来看:金子不见了!
    
    巴依老爷的心好像被锤子砸碎了。他失魂落魄地找到阿凡提,问:“金子怎么没了?你还我的金子!”
    
    阿凡提说:“我的好老爷!我早就告诉过您,可不能在白天里挖开来看呀!您这么一弄,金子都死了!”
    
    巴依老爷气得吹胡子:“胡说!金子怎么会死呢?”
    
    阿凡提说:“您既然相信金子能种,怎么就不相信金子会死呢?”
    
    巴依老爷的金子没了,就把阿凡提抓到官府里问罪。官府宣判,阿凡提骗了巴依老爷的金子,把阿凡提砍了脑袋。可也有人说,阿凡提并没有死,因为不久之后的一个早上,城里每户穷人家的门前都出现了一小块金子。人们都说,那是阿凡提种出来的金子。
    
\end{large}


\newpage

\textbf{注释}:

\vspace{-1em}

\begin{itemize}
    \setlength\itemsep{-0.2em}
    \item 〔财迷心窍〕被钱财迷惑了心智。指人因为贪财而失去了常性,不讲道理。
    \item 〔刁民〕奸恶狡猾、难以管教的人民。统治阶级对百姓的贬称。
    \item 〔念念有词〕小声念咒语或祷告。引申指低声自语或含糊不清地说个不停,仿佛在念什么内容。
    \item 〔失魂落魄〕丢失了魂魄。形容心神不安、惊慌失措的样子。
    \item 〔问罪〕声讨或严厉指责对方的罪状。
\end{itemize}

\chapter{武松打虎}

\begin{large}
    
    武松走了几天,到了阳谷县地界。这里离县府还很远。中午的时候,武松走得肚子饿了,正好望见前面有一家酒店\footnote{〔酒店〕古代提供餐饮酒食的饭店,有些也提供住宿。}。酒店门前的招旗\footnote{〔招旗〕店铺招徕客人的旗子。}上写着五个字:“三碗不过冈”。武松走进店里坐下,把梢棒\footnote{〔梢棒〕防身用的棍棒,也写作“哨棒”。}倚在一边,喊道:“店家,给我上点酒菜来。”
    
    店家很快上了一碟热菜、一双筷子,又在武松面前排开三个酒碗,满满筛了一碗酒。武松拿起碗,一饮而尽,叫道:“这酒真够劲!老板,来点下酒的好菜。”店家道:“只有熟牛肉。”武松道:“切两三斤好肉,给我下酒。”店家就到后面切了两斤熟牛肉,满满一大盘端上来,又筛了一碗酒。武松又把酒喝了,叫道:“好酒!”店家又再筛了一碗酒给武松喝,就再也不添酒了。
    
    武松敲着桌子叫道:“老板,怎么还不来盛酒?”店家道:“客官要添肉吗?”武松道:“肉也添一些,酒也要添。”店家道:“客官要添肉,这就切来,酒却不能再添了。”武松道:“这就怪了,怎么不肯卖酒了?”店家道:“客官您看,我的店门前写着呢,‘三碗不过冈’。”
    
    武松问:“什么叫做‘三碗不过冈’?”店家道:“我这酒虽然只是自酿的,但比老酒还要来劲。但凡客人来我店里喝了三碗以上,就要醉了,过不了前面的山冈。因此叫做‘三碗不过冈’。要是熟客来我这里,只喝三碗酒就走了,也不须多问。”武松笑道:“原来是这么回事。我已经喝了三碗酒,怎么没醉呢?”店家道:“我这酒有个名字叫作‘透瓶香’,又叫‘出门倒’,刚入口的时候醇香好喝,过一会儿就醉倒了。”武松叫道:“休要胡说。我又不是不给钱,你再给我上三碗酒。”
    
    店家见武松不肯走,只好又筛了三碗酒。武松把酒喝了,叫道:“真是好酒!老板,我喝一碗,就给你一碗的酒钱。你只管倒酒。”店家道:“客官别只顾着喝酒,这酒喝多了真要醉倒的,没药医。”武松叫道:“休得胡说!哪怕你下了蒙汗药\footnote{〔蒙汗药〕旧戏曲小说中指能使人暂时失去知觉的麻药。},我也能闻出来。”店家拗不过他,只好又筛了三碗酒。武松道:“再切两斤肉来吃。”酒家又切了两斤熟牛肉,再筛了三碗酒。
    
    武松越喝越上瘾,只想再喝,从身上掏出几块碎银子,叫道:“老板你看,我这些钱够吗?”店家道:“有多,还得找钱给你。”武松道:“不用你找钱,继续上酒便是。”店家道:“多出来的钱,还能喝五六碗酒呢。只怕你喝不了。”武松道:“五六碗不打紧,你尽管上。”店家道:“你这大块头,要是醉倒了,我怎么扶得住?”武松道:“要你扶就算不得好汉。”店家哪里肯过来筛酒。
    
    武松心中焦躁,道:“我又不白吃你的!休要惹得老爷发脾气,将你屋里砸得粉碎!把你这破店子倒翻转来!”店家道:“这厮醉了,莫要惹他。”再筛了六碗酒给武松喝了。武松前后喝了十八碗酒,拎起梢棒,起身道:“我可还没醉。”走出店门,回头笑道:“还说什么‘三碗不过冈’呢!不过如此!”提着梢棒就要走。
    
    店家赶出来叫道:“客官哪里去?”武松停下脚步问道:“叫我干什么?我又不欠你酒钱。”店家叫道:“我是好意。你且回来我这,看看官府贴的榜文。”武松问道:“什么榜文?”店家道:“如今前面的景阳冈上,有只吊睛白额大老虎,夜里出来伤人,已经害了二三十条大汉的性命。官府现在已经责令打猎的人,限期捉住这老虎。这景阳冈下,路口两边的店铺人家里,都张贴了榜文,让往来的客人,结伙成队,每天于巳、午、未三个时辰\footnote{〔时辰〕古代计时单位,一个时辰等于两小时。时辰名字按十二地支排列,巳、午、未对应上午9点到下午3点,卯、辰对应早上5点到9点,申、酉、戌、亥对应下午3点到晚上11点。}过冈,其余卯、辰、申、酉、戌、亥六个时辰,不许过冈。此外,单身的客人,白天也不许过冈,一定要等人多了,结伙而过。现在正是未末申初时分,天将晚了,我见你到离店了都没问过这事,怕你枉送了自家性命。不如到我这店里歇一晚上,等明天慢慢凑个二三十人,一齐好过冈子。”
    
    武松听了笑道:“我是清河县人氏,在这条景阳冈上少说也走过了一二十遭,几时听说有老虎了!你休说这话来吓我!即便有虎,我也不怕。”店家道:“我是好意救你。你若是不信,进来看看官府的榜文。”武松道:“便真个有虎,老爷也不怕。你留我在店里过夜,莫不是想半夜三更谋我财,害我性命,才拿老虎来吓我?”店家道:“你看看!我是一片好心,反被当作恶意,倒落得让你这般数落。你若是不信我,请尊便自行。”
    
    店主人摇着头,回到店里去了。武松提了梢棒,大步往景阳冈走去。大约走了四五里路,来到冈子下,只见一棵大树,刮去了皮,一片白,上面写了两行字。武松也颇识得几个字,抬头一看,只见上面写道:“近因景阳冈有虎伤人,但凡过往客商,可于巳、午、未三个时辰,结伙成队过冈。请勿自误。”武松看了,笑道:“这是店家唬弄,惊吓往来客人,想让人去他家住店罢了。我可不怕!”横拖着梢棒,便上山了。
    
    这时已是下午三点。这轮红日,慢慢往山下落去了。武松乘着酒兴,只管走上冈子来。走不到半里路,见到一个败落的山神庙。来到庙前,就见这庙门上贴着一张印信\footnote{〔印信〕政府机关盖的印章,盖印为信,因此叫印信。}榜文。武松停下脚步来读这榜文,上面写道:
    
    \begin{quotation}
    
    阳谷县示:为这景阳冈上新有一只老虎,近来伤害人命。现责令各乡里正\footnote{〔里正〕古代管治乡里的基层小吏。}及猎户人等限期捕捉,尚未捕获。如有过往客商人等,可于巳、午、未三个时辰,结伴过冈。其余时分,及单身客人,不许过冈,恐被老虎伤害,性命不保。各宜知悉。
    
    \end{quotation}
    
    武松读了印信榜文,方知真的有虎。他想要掉头再回酒店去,又寻思道:“我要是回去,一定叫他耻笑,不是好汉,这可难办了。”他慢慢想了一会儿,说道:“怕甚么!且只顾上去,看怎地!”武松走着走着,酒劲涌上来,就把毡笠儿背在脊梁上,将梢棒系在肋下,一步步上那冈子去。回头看这太阳,渐渐地坠下去了。
    
    此时已是十月,日短夜长,天黑得快。武松自言自语道:“哪里有什么老虎!人人都怕了传言,不敢上山罢了。”武松走着走着,酒力发作,燥热起来,一只手提着梢棒,一只手把胸膛前袒开,踉踉跄跄,直奔入乱树林里去。看见一块光溜溜的大青石,就把那梢棒倚在一边,躺平身子,正要入睡,却只见一阵狂风袭来。
    
    古人说得好:云从龙,风从虎。那一阵风过处,只听得乱树丛后扑地一声响,跳出一只吊睛白额的老虎来。武松见了,叫声“哎呀!”从青石上翻将下来,便将那条梢棒拿在手里,闪在青石边。那老虎又饥又渴,把两只爪子在地下略按一按,和身往上一扑,从半空里蹿将下来。武松被那虎一惊,出了一身冷汗,酒也醒了。说时迟,那时快,武松见老虎扑来,只一闪,闪在老虎背后。那老虎背后看人最难,便把前爪搭在地下,把腰胯一掀,掀将起来。武松只一躲,躲在一边。大虫见掀他不着,大吼一声,却似半天里起了个霹雳,震得那山冈也动。把这铁棒似的虎尾倒竖起来,只一剪。武松却又闪在一边。
    
    原来那老虎拿人,就是一扑、一掀、一剪,要是三招抓不着人,气势就少了一半。那老虎剪不着,再吼了一声,一兜兜将回来。武松见那老虎翻身回来,双手轮起梢棒,尽平生气力,只一棒,从半空劈将下来。只听得一声响,簌簌地连枝带叶掉了下来。定睛看时,原来心里慌张,这一棒没劈到老虎,却打在一棵枯树上,把枝叶打下来不少,那条梢棒却也折做两截,只拿得一半在手里。那老虎咆哮,发起性来,翻身又只一扑,扑将来。武松又只一跳,退了十步远。那老虎正好把两只前爪搭在武松面前。武松将半截棒丢在一边,两只手就势把老虎头顶花皮揪住,一按按将下来。那只老虎急着要挣扎,早没了气力。武松用尽力气按定了老虎,哪里肯放松半点儿。武松把只脚朝老虎的面门上、眼睛里只顾乱踢。那老虎咆哮起来,在身底下扒出了两堆黄泥,扒了一个土坑。武松把那老虎的嘴直按到黄泥坑里去。那老虎被武松降伏住,气力渐渐少了。武松用左手紧紧揪住老虎头顶花皮,空出右手来,提起铁锤般大小的拳头,尽平生之力,只顾打。打了五六十拳,那老虎眼里、口里、鼻子、耳朵里都迸出鲜血来。那武松尽了平生神威,仗了胸中武艺,半歇儿把大虫打得好似一个躺平的锦布袋。
    
    武松放了手,到枯树边寻回那打折的棒子,拿在手里。只怕老虎不死,用棒子又打了一回。那老虎气都没了。武松再寻思道:“我就这么拖这死老虎下冈子去。”就到血泊里用双手来提,可哪里提得动?原来打老虎时使尽了气力,现在手脚都酥软了。
    
    武松又在青石上坐了半歇,寻思道:“天要黑了,要是再跳出一只老虎来,我怎么斗得它过?且挣扎下冈子去,明早再来理会。”就到石头边找回了毡笠儿,转过乱树林边,一步步捱下冈子来。
    
    走不到半里路,只见枯草丛中,又钻出两只老虎来。武松道:“哎呀!这回要完了!”却见那两只老虎直立起来。武松定睛一看,原来是两个人,用虎皮缝做衣裳,紧紧绷在身上。那两人手里各拿着一条五股叉,见了武松,吃惊道:“你这人吃了熊心豹子胆了。天要黑了,如何敢独自一人,又没器械,走过冈子来!不知你是人是鬼?”武松道:“你两个是什么人?”那个人道:“我们是这里的猎户。”武松道:“你们上岭来做什么?”两个猎户吃惊道:“你竟然不知哩!如今景阳冈上,有一只极大的老虎,夜夜出来伤人。单单我们猎户,就折了七八个;过往客人,不计其数,都被这畜生吃了。本县知县\footnote{〔知县〕古代管治一县的主官。知:掌管、治理。}责令各乡里正和我们猎户来捕捉。那业畜\footnote{〔业畜〕作恶的畜生。}势大难近,谁敢向前!为了它,我们不知吃了多少苦头,只捉它不到!今夜又该我们两个捕猎,和十几个乡夫在此,上上下下,放了窝弓药箭等它。正在这里埋伏,却见你大剌剌地从冈子上走将下来,吓了我两个一跳。我们才想问哩,你是什么人?见到老虎了么?”
    
    武松道:“我是清河县人氏,姓武,排行第二。刚才在冈子上乱树林边,正撞见那老虎,被我一顿拳脚打死了。”两个猎户惊呆了,说道:“怕没这话?”武松道:“你要是不信,就看我身上,还有血迹。”两个道:“怎么打的?”武松就把打老虎的经过说了一遍。
    
    两个猎户听了,又惊又喜,叫拢那些乡夫来。只见这些乡夫都拿着钢叉踏弩、刀枪棍棒。武松问道:“他们为何不随着你两个上山?”猎户道:“那畜生厉害,他们如何敢上来?”两个猎户把武松打杀老虎的事向众人说了一遍,众人都不肯信。武松道:“你们不信,就随我去看看。”
    
    众人身边都有火刀、火石,就打起五六个火把,都跟着武松,一同再上冈子来。只见那老虎瘫软了,死在那里。众人见了大喜,先叫一个人去报知县里。又叫五六个乡夫,把死老虎绑了,抬下冈子来。
    
\end{large}



\chapter{菩萨蛮·大柏地}

\begin{large}
    
    \begin{verse}[0.5\linewidth]
        赤橙黄绿青蓝紫, \\
        谁持彩练当空舞? \\
        雨后复斜阳,关山阵阵苍。
    \end{verse}
    
    
    \begin{verse}[0.5\linewidth]
        当年鏖战急,弹洞前村壁 \\
        装点此关山,今朝更好看。
    \end{verse}
    
\end{large}


\newpage

\textbf{注释}:

\vspace{-1em}

\begin{itemize}
    \setlength\itemsep{-0.2em}
    \item 〔练〕绢带。
    \item 〔关山〕关口和山川。
\end{itemize}

\chapter{穷人}

\begin{large}
    
    渔夫的妻子桑娜坐在火炉旁补一张破帆。屋外寒风呼啸,汹涌澎湃的海浪拍击着海岸,溅起一阵阵浪花。海上正起着风暴,外面又黑又冷,这间渔家的小屋里却温暖而舒适。地扫得干干净净,炉子里的火还没有熄,食具在搁板上闪闪发亮。挂着白色帐子的床上,五个孩子正在海风呼啸声中安静地睡着。丈夫清早驾着小船出海,这时候还没有回来。桑娜听着波涛的轰鸣和狂风的怒吼,感到心惊肉跳。
    
    古老的钟发哑地敲了十下,十一下……始终不见丈夫回来。桑娜沉思:丈夫不顾惜身体,冒着寒冷和风暴出去打鱼,她自己也从早到晚地干活,还只能勉强填饱肚子。孩子们没有鞋穿,不论冬夏都光着脚跑来跑去;吃的是黑面包,菜只有鱼。不过,孩子们都还健康,没什么可抱怨的。桑娜倾听着风暴的声音。“他现在在哪儿?老天啊,保佑他,救救他,开开恩吧!”她自言自语着。
    
    睡觉还早。桑娜站起身来,把一块很厚的围巾包在头上,提着马灯走出门去。她想看看灯塔上的灯是不是亮着,丈夫的小船能不能望见。海面上什么也看不见。风掀起她的围巾,卷着被刮断的什么东西敲打着邻居小屋的门。桑娜想起了傍晚就想去探望的那个生病的女邻居。“没有一个人照顾她啊!”桑娜一边想一边敲了敲门。她侧着耳朵听,没有人答应。
    
    “寡妇的日子真困难啊!”桑娜站在门口想,“孩子虽然不算多——只有两个,可是全靠她一个人张罗,如今又加上病。唉,寡妇的日子真难过啊!进去看看吧!”
    
    桑娜一次又一次地敲门,仍旧没有人答应。
    
    “喂,西蒙!”桑娜喊了一声,心想,莫不是出什么事了?她猛地推开门。
    
    屋子里没有生炉子,又潮湿又阴冷。桑娜举起马灯,想看看病人在什么地方。首先投入眼帘的是对着门的一张床,床上仰面躺着她的女邻居。她一动不动。桑娜把马灯举得更近一些,不错,是西蒙。她头往后仰着,冰冷发青的脸上显出死的宁静,一只苍白僵硬的手像要抓住什么似的,从稻草铺上垂下来。就在这死去的母亲旁边,睡着两个很小的孩子,都是卷头发、圆脸蛋,身上盖着旧衣服,蜷缩着身子,两个浅黄头发的小脑袋紧紧地靠在一起。显然,母亲在临死的时候,拿自己的衣服盖在他们身上,还用旧头巾包住他们的小脚。孩子的呼吸均匀而平静,睡得正香甜。
    
    桑娜用头巾裹住睡着的孩子,把他们抱回家里。她的心跳得很厉害,自己也不知道为什么要这样做,但是觉得非这样做不可。她把这两个熟睡的孩子放在床上,让他们同自己的孩子睡在一起,又连忙把帐子拉好。
    
    桑娜脸色苍白,神情激动。她忐忑不安地想:“他会说什么呢?这是闹着玩的吗?自己的五个孩子已经够他受的了……是他来啦?……不,还没来!……为什么把他们抱过来啊?……他会揍我的!那也活该,我自作自受……嗯,揍我一顿也好!”
    
    门吱嘎一声,仿佛有人进来了。桑娜一惊,从椅子上站起来。
    
    “不,没有人!天啊,我为什么要这样做?……如今叫我怎么对他说呢?”……桑娜沉思着,久久地坐在床前。
    
    门突然开了,一股清新的海风冲进屋子。魁梧黧黑的渔夫拖着湿淋淋的被撕破了的鱼网,一边走进来,一边说:“嘿,我回来啦,桑娜!”
    
    “哦,是你!”桑娜站起来,不敢抬起眼睛看他。
    
    “瞧,这样的夜晚!真可怕!”
    
    “是啊,是啊,天气坏透了!哦,鱼打得怎么样?”
    
    “糟糕,真糟糕!什么也没有打到,还把网给撕破了。倒霉,倒霉!天气可真厉害!我简直记不起几时有过这样的夜晚了,还谈得上什么打鱼!还好,总算活着回来啦。……我不在,你在家里做些什么呢?”
    
    渔夫说着,把网拖进屋里,坐在炉子旁边。
    
    “我?”桑娜脸色发白,说,“我嘛……缝缝补补……风吼得这么凶,真叫人害怕。我可替你担心呢!”
    
    “是啊,是啊,”丈夫喃喃地说,“这天气真是活见鬼!可是有什么办法呢!”
    
    两个人沉默了一阵。
    
    “你知道吗?”桑娜说,“咱们的邻居西蒙死了。”
    
    “哦?什么时候?”
    
    “我也不知道,大概是昨天。唉!她死得好惨啊!两个孩子都在她身边,睡着了。他们那么小……一个还不会说话,另一个刚会爬……”桑娜沉默了。
    
    渔夫皱起眉,他的脸变得严肃、忧虑。“嗯,是个问题!”他搔搔后脑勺说,“嗯,你看怎么办?得把他们抱来,同死人待在一起怎么行!哦,我们,我们总能熬过去的!快去!别等他们醒来。”
    
    但桑娜坐着一动不动。
    
    “你怎么啦?不愿意吗?你怎么啦,桑娜?”
    
    “你瞧,他们在这里啦。”桑娜拉开了帐子。
    
\end{large}


\newpage

\textbf{注释}:

\vspace{-1em}

\begin{itemize}
    \setlength\itemsep{-0.2em}
    \item 〔黧黑〕脸色黑。黧:黑中带黄。
    \item 〔寡妇〕死了丈夫的已婚女子。
    \item 〔蜷缩〕缩成一团。蜷:肢体屈曲。
\end{itemize}

\chapter{琥珀}

\begin{large}
    
    这个故事发生在很久很久以前,约莫算来,总有一万年了。
    
    一个夏天,太阳暖暖地照着,海在很远的地方奔腾怒吼,绿叶在树上飒飒地响。
    
    一只小苍蝇展开柔嫩的绿翅膀,在太阳光里快乐地飞舞。后来,她嗡嗡地穿过草地,飞进树林。那里长着许多高大的松树,太阳照得火热,可以闻到一股松脂\footnote{〔松脂〕松树树干分泌的油脂,也称松香、松膏。}的香味。
    
    那只小苍蝇停在一棵大松树上。她伸起腿来掸掸翅膀,刷那长着一对红眼睛的圆脑袋。她飞了大半天,身上已经沾满了灰尘。
    
    忽然,有个蜘蛛划着长长的腿慢慢地爬过来,想把小苍蝇当做一顿美餐。她小心地划动长腿,沿着树干爬下来,离小苍蝇越来越近了。
    
    “哎呀!”他想,“这位小姑娘身子并不大,除去一双绿翅膀,一对触须,剩下的就很少了。不过少虽少,总还是一顿美餐。要是我不小心,被她的大眼睛看见了,她马上飞开,我的美餐就要落空,说不定会饿上一天呢!”
    
    小苍蝇不住地掸绿翅膀,刷圆脑袋,一点儿也不知道蜘蛛越来越近了。
    
    晌午的太阳光热辣辣地照射着整个树林。许多老松树渗出厚厚的松脂,在太阳光里闪闪地发出金黄的光彩。
    
    蜘蛛刚扑过去,忽然发生了一件可怕的事情。一大滴松脂从树上滴下来,刚好落在树干上,把苍蝇和蜘蛛一齐包在里头。
    
    小苍蝇不能掸翅膀了,蜘蛛也不再想什么美餐了。两只小虫都淹没在老松树的黄色泪珠里。她们前俯后仰地挣扎了一番,终于不动了。
    
    松脂继续滴下来,盖住了原来的,最后积成一个松脂球,把两只小虫重重包裹在里面。
    
    几十年,几百年,几千年,时间一转眼就过去了。成千上万绿翅膀的苍蝇和八只脚的蜘蛛来了又去了,谁也不会想到很久很久以前,有两只小虫被埋在一个松脂球里,挂在一棵老松树上。
    
    后来又发生了变化。陆地渐渐沉下去,海水渐渐漫上来,逼近那片古老的森林。有一天,水把森林淹没了。波浪不断地向树干冲刷,甚至把树连根拔起。树就断绝了生机,慢慢地腐烂了。剩下的只有那些松脂球,淹没在海沙下面。
    
    又是几千年过去了,那些松脂球成了化石\footnote{〔化石〕存留在古代地层中的古生物遗体、遗物或遗迹,硬如石头。}。
    
    海风猛烈地吹,澎湃的波涛把海里的泥沙卷到岸边。
    
    有个渔民带着他的儿子经过海滩。那孩子赤着脚,踏着了沙土里一块硬硬的东西,就把它挖了出来。
    
    “爸爸,您看!”他快活得叫起来,“这是什么?”
    
    他爸爸接过来,仔细看了看。
    
    “这是琥珀\footnote{〔琥珀〕一种很硬的、黄褐色半透明的树脂化石,常用作装饰品。},孩子。”他高兴地说,“有两个小东西关在里面呢,一个苍蝇,一个蜘蛛。这是很少见的。”
    
    在那块透明的琥珀里面,两个小东西还是好好地躺着。我们可以看见苍蝇的翅膀和蜘蛛的长腿,可以看见他当时在黏稠的松脂里怎样挣扎,因为他们的腿的四周显出好几圈黑色的圆环。从那块琥珀,我们可以推测发生在一万年以前的故事的详细情形,并且可以知道,在远古时代,世界上早已有那样的苍蝇和蜘蛛了。
    
\end{large}


\newpage

\textbf{注释}:

\vspace{-1em}

\begin{itemize}
    \setlength\itemsep{-0.2em}
    \item 〔澎湃〕波涛猛烈产生的巨大响声,引申指声势、气势等浩大雄伟。
    \item 〔掸〕弹击,使振动。
    \item 〔生机〕生命的活力。
    \item 〔前俯后仰〕身体前后晃动。
\end{itemize}

\chapter{十六年前的回忆}

\begin{large}
    
    1927年4月28日,我永远忘不了那一天。那是父亲的被难日,离现在已经十六年了。
    
    那年春天,父亲每天夜里回来得很晚。每天早晨,不知道什么时候他又出去了。有时候他留在家里,埋头整理书籍和文件。我蹲在旁边,看他把书和有字的纸片投到火炉里去。
    
    我奇怪地问他:“爹,为什么要烧掉呢?怪可惜的。”
    
    待了一会儿,父亲才回答:“不要了就烧掉。你小孩子家知道什么!”
    
    父亲一向是慈祥的,从没有骂过我们,更没有打过我们。我总爱向父亲问许多幼稚可笑的问题。他不论多忙,对我的问题总是很感兴趣,总是耐心地讲给我听。这一次不知道为什么,父亲竟这样含糊地回答我。
    
    后来听母亲说,军阀张作霖要派人来检查。为了避免党组织被破坏,父亲只好把一些书籍和文件烧掉。才过了两天,果然出事了。工友阎振三一早上街买东西,直到夜里还不见回来。第二天,父亲才知道他被抓到警察厅里去了。我们心里都很不安,为这位工友着急。
    
    局势越来越严峻,父亲的工作也越来越紧张。他的朋友劝他离开北京,母亲也几次劝他。父亲坚决地对母亲说:“不是常对你说吗?我是不能轻易离开北京的。你要知道现在是什么时候,这里的工作多么重要。我哪能离开呢?”母亲只好不再说什么了。
    
    可怕的一天果然来了。4月6日的早晨,妹妹换上了新夹衣,母亲带她到儿童娱乐场去散步了。父亲在里间屋里写字,我坐在外间的长木椅上看报。短短的一段新闻还没看完,就听见啪,啪……几声尖锐的枪声,接着是一阵纷乱的喊叫。
    
    “什么?爹!”我瞪着眼睛问父亲。
    
    “没有什么,不要怕。星儿,跟我到外面看看去。”
    
    父亲不慌不忙地向外走去。我紧跟在他身后,走出院子,暂时躲在一间僻静的小屋里。
    
    一会儿,外面传来一阵沉重的皮鞋声。我的心剧烈地跳动起来,用恐怖的眼光瞅了瞅父亲。
    
    “不要放走一个!”窗外响起粗暴的吼声。穿灰制服和长筒皮靴的宪兵,穿便衣的侦探,穿黑制服的警察,一拥而入,挤满了这间小屋。他们像一群魔鬼似的,把我们包围起来。他们每人拿着一支手枪,枪口对着父亲和我。在军警中间,我发现了前几天被捕的工友阎振三。他的胳膊上拴着绳子,被一个肥胖的便衣侦探拉着。
    
    那个满脸横肉的便衣侦探指着父亲问阎振三:“你认识他吗?”
    
    阎振三摇了摇头。他那披散的长头发中间露出一张苍白的脸,显然是受过苦刑了。
    
    “哼!你不认识?我可认识他。”侦探冷笑着,又吩咐他手下的那一伙,“看好,别让他自杀!”
    
    他们仔细地把父亲全身搜了一遍。父亲保持着他那惯有的严峻态度,没有向他们讲任何道理。因为他明白,对他们是没有道理可讲的。
    
    残暴的匪徒把父亲绑起来,拖走了。我也被他们带走了。在高高的砖墙围起来的警察厅的院子里,我看见母亲和妹妹也都被带来了。我们被关在女拘留所里。
    
    十几天过去了,我们始终没看见父亲。有一天,我们正在啃手里的窝窝头,听见警察喊我们母女的名字,说是提审。
    
    在法庭上,我们跟父亲见了面。父亲仍旧穿着他那件灰布旧棉袍,可是没戴眼镜。我看到了他那乱蓬蓬的长头发下面的平静而慈祥的脸。
    
    “爹!”我忍不住喊出声来。母亲哭了,妹妹也跟着哭起来了。
    
    “不许乱喊!”法官拿起惊堂木\footnote{〔惊堂木〕法庭用具,可以发出巨大响声,震慑人群,维护法官权威。}重重地在桌子上拍了一下。
    
    父亲瞅了瞅我们,没有说一句话。他的神情非常安定,非常沉着。他的心被一种伟大的力量占据着。这个力量就是他平日对我们讲的——他对于革命事业的信心。
    
    “这是我的妻子。”他指着母亲说。接着他又指了一下我和妹妹,“这是我的两个女孩子。”
    
    “她是你最大的孩子吗?”法官指着我问父亲。
    
    “是的,我是最大的。”我怕父亲说出哥哥来,就这样抢着说了。我不知道当时哪里来的机智和勇敢。
    
    “不要多嘴!”法官怒气冲冲的,又拿起他面前那块木板狠狠地拍了几下。
    
    父亲立刻就会意了,接着说:“她是我最大的孩子。我的妻子是个乡下人,我的孩子年纪都还小,她们什么也不懂,一切都跟她们没有关系。”父亲说完了这段话,又望了望我们。
    
    法官命令把我们押下去。我们就这样跟父亲见了一面,匆匆分别了。想不到这竟是我们最后一次见面。
    
    28日黄昏,警察叫我们收拾行李出拘留所。
    
    我们回到家里,天已经全黑了。第二天,舅姥爷到街上去买报。他是哭着从街上回来的,手里无力地握着一份报。我看到报上用头号字登着“李大钊等昨已执行绞刑”,立刻感到眼前蒙了一团云雾,昏倒在床上了。母亲伤心过度,昏过去三次,每次都是刚刚叫醒又昏过去了。
    
    过了好半天,母亲醒过来了,她低声问我:“昨天是几号?记住,昨天是你爹被害的日子。”
    
    我又哭了,从地上捡起那张报纸,咬紧牙,又勉强看了一遍,低声对母亲说:“妈,昨天是4月28日。”
    
\end{large}



\chapter{夜莺的歌声}

\begin{large}
    
    战斗刚刚结束,一小队德国兵进了村庄。大道两旁全是黑色的碎瓦。空旷的花园里,烧焦的树垂头丧气地弯着腰。
    
    夜莺的歌声打破了夏日的沉寂。这歌声停了一会儿,接着又用一股新的劲头唱起来。
    
    士兵们和军官注意听着,开始注视周围的灌木丛和挂在道旁的白桦树枝。他们发现就在很近很近的地方,有个孩子坐在河岸边上,耷拉着两条腿。他光着头,穿一件颜色跟树叶差不多的绿上衣,手里拿着一块木头,不知在削什么。
    
    “喂。你来!”军官叫那个孩子。
    
    孩子赶紧把小刀放到衣袋里,抖了抖衣服上的木屑,走到军官跟前。
    
    “呶,让我看看!”军官说。
    
    孩子从嘴里掏出一个小玩意儿,递给他,用快活的蓝眼睛望着他。
    
    那是个白桦树皮做的口哨。
    
    “挺巧!小孩子,你做得挺巧哇。”军官点了点头。转眼间,他那阴沉的脸上闪出一种冷笑的光,“谁教你这样吹哨子的?”
    
    “我自己学的。我还会学杜鹃叫呢。”
    
    孩子学了几声杜鹃叫。接着又把哨子塞到嘴里吹起来。
    
    “村子里就剩下你一个了吗?”军官继续盘问他。
    
    “怎么会就剩下我一个?这里有麻雀、乌鸦、猫头鹰,多着呢。夜莺倒是只有我一个!”
    
    “你这个坏家伙!”军官打断孩子的话,“我是问你这里有没有人。”
    
    “人哪?战争一开始这里就没有人了。”小孩不慌不忙地回答,“刚刚一开火,村子就着火了,大家都喊:‘野兽来了,野兽来了’——就都跑了。”
    
    “蠢东西!”军官想着,轻蔑地微笑了一下。
    
    “呶,你认识往苏蒙塔斯村去的路吗?那个村子大概是叫这个名字吧?”
    
    “怎么会不认识!”孩子很有信心地回答,“我和叔叔常到磨坊那儿的堤坝上去钓鱼。那儿的狗鱼可凶呢,能吃小鹅!”
    
    “好啦,好啦,带我们去吧。要是你带路带得对,我就把这个小东西送给你。”军官说着,指了指他的打火机,“要是你把我们带到别处去,我就把你的脑袋拧下来。听懂了吗?”
    
    队伍出发了,行军灶打头,跟着就是小孩和军官,俩人并排着走。小孩有时候学夜莺唱,有时候学杜鹃叫,胳膊一甩一甩地打着路旁的树枝,有时候弯下腰去拾球果,还用脚把球果踢起来。他好像把身边的军官完全忘了。
    
    森林越来越密。弯弯曲曲的小路穿过密密的白桦树林,穿过杂草丛生的空地,又爬上了长满古松的小山。
    
    “你们这里有游击队吗?”军官突然问。
    
    “你说的是一种蘑菇吗?没有,我们这里没有这种蘑菇。这里只有红蘑菇、白蘑菇,还有洋蘑菇。”孩子回答。
    
    军官觉得从孩子嘴里什么也问不出来,就不再问了。
    
    树林深处,有几个游击队员埋伏在那里,树旁架着冲锋枪。他们从树枝缝里往外望,能够看见曲折的小路。他们不时说几句简单的话,小心地拨开树枝,聚精会神地盯着远方。
    
    “你们听见了吗?”一个游击队员突然说。他伸直了腰,好像有什么鸟的叫声,透过树叶的沙沙声,模模糊糊地传来。他侧着头,往叫声那边仔细听,“夜莺!”
    
    “没听错吗?”另一个游击队员说。他紧张起来,仔细听,可又什么也听不见了。他从大树桩下边掏出四个手榴弹,放在跟前以防万一。
    
    “这回你听见了没有?”
    
    夜莺的歌声越来越响了。
    
    那个最先听见夜莺叫的凝神地站着,好像钉在那里似的。他注意数着一声一声的鸟叫:“一,二,三,四……”一边数一边用手打着拍子。
    
    夜莺的叫声停止了。“三十二个鬼子……”那个人说。只有游击队员才知道这鸟叫的意思。接着传来两声杜鹃叫。“两挺机关枪。”他又补充说。
    
    “对付得了!”一个满脸胡子的汉子端着冲锋枪说。他理了理挂在腰间的子弹袋。
    
    “应该对付得了!”听鸟叫的那个人回答,“我和斯切潘叔叔把他们放过去,等你们开了火,我们在后边加油。如果我们出了什么事,你们可不要忘了小夜莺……”
    
    过了几分钟,德国兵在松树林后边出现了。夜莺还是兴致勃勃地唱着,但是对藏在寂静森林里的人们来说,那歌声已经没有什么新鲜的意思了。
    
    德国兵走到林中空地上的时候,突然从松树林里发出一声口哨响,像回声一样回答了孩子。孩子突然站住,转了个身,钻到树林里不见了。枪声打破了林中的寂静。军官还没来得及抓起手枪,就滚到了路边的尘埃里。被冲锋枪打伤的德国兵一个跟一个地倒下。呻吟声、叫喊声、断断续续的口令声充满了树林。
    
    第二天,在被烧毁的村子的围墙旁边,在那小路分岔的地方,孩子又穿着那件绿上衣,坐在原来那河岸边削什么东西,并且不时回过头去,望望那通向村子的几条道路,好像在等谁似的。
    
    从孩子的嘴里飞出宛转的夜莺的歌声。这歌声,即使是听惯了鸟叫的人也觉察不出跟真夜莺的有什么两样。
    
\end{large}



\chapter{花潮}

\begin{large}
    
    昆明有个圆通寺。寺后就是圆通山。从前是一座荒山,现在是一个公园,就叫圆通公园。
    
    公园在山上。有亭,有台,有池,有榭,有花,有树,有鸟,有兽。
    
    后山沿路,有一大片海棠,平时枯枝瘦叶,并不惹人注意,一到三四月间,真是花团锦簇,变成一个花世界。
    
    这几天天气特别好,花开得也正好,看花的人也就最多。办公室里,餐厅里,晚会上,道路上,经常听到有人问答:“你去看海棠没有?”“我去过了。”或者说:“我正想去。”到了星期天,道路相逢,多争说圆通山海棠消息。一时之间,几乎形成一种空气,甚至是一种压力,一种诱惑,如果谁没有到圆通山看花,就好像是一大憾事,不得不挤点时间,去凑个热闹。
    
    星期天,我们也去看花。一路同去看花的人可多着哩。进了公园门,步步登山,接踵摩肩,人就更多了。向高处看,隔着密密层层的绿荫,只见一片红云,望不到边际。这时候,什么苍松啊,翠柏啊,碧梧啊,修竹啊……都挽不住游人。大家都一口气地攀到最高峰,淹没在海棠花的红海里。后山一条大路,两旁,四周,都是海棠。人们坐在花下,走在路上,既望不见花外的青天,也看不见花外还有别的世界。花开得正盛,每棵树都炫耀自己的鼎盛时代,每一朵花都在微风中枝头上颤抖着,说出自己的喜悦。一条花街,上天下地都是花,可谓花天花地。但是,这些说法还不够,还是“花潮”好。你看那一望无际的花,有风,花在动,无风,花也潮水一般地动,在阳光照射下,每一个花瓣都有它自己的阴影,就仿佛多少波浪在大海上翻腾,你越看得出神,你就越感到这一片花潮正在向天空向四面八方伸张,好像有一种生命力在不断扩展。而且,你仿佛可以听到潮水的声音。谁知道呢,也许是花下的人语声,也许是花丛中蜜蜂嗡嗡声,也许什么地方有黄莺的歌声,还有什么地方送来看花人的琴声,歌声,笑声……这一切交织在一起,再加上风声,就如同海上午夜的潮声。大家都是来看花的,可是,这个花到底怎么看法?有人走累了,拣个最好的地方坐下来看,不一会,又感到这里不够好,也许别个地方更好吧,于是站起来,既依依不舍,又满怀向往,慢步移向别处去。多数人都在花下走来走去,这棵树下看看,好,那棵树下看看,也好,伫立在另一棵树下仔细端详一番,更好,看看,想想,再看看,再想想。有人很大方,只是驻足观赏,有人贪心重,伸手牵过一枝花来摇摇,或者干脆翘起鼻子一嗅,再嗅,甚至三嗅。面对这绮丽的风光,真个是手足无措了。
    
    老头儿们看花,一面看,一面自言自语,或者嘴里低吟着什么。老妈妈看花,扶着拐杖,牵着孙孙,很珍惜地折下一朵,簪在自己的发髻上。青年们穿得整整齐齐,干干净净,好像参加什么盛会,不少人已经穿上雪白的衬衫,有的甚至是绸衬衫,有的甚至已是短袖衬衫,好像夏天已经来到他们身上,东张张,西望望,既看花,又看人,洋气得很。青年妇女们,也都打扮得利利落落,很多人都穿着花衣花裙,好像要与花争妍,也有人擦了点胭脂,抹了点口红,显得很突出,可是,在这花世界里,又叫人感到无所谓了。画家们正调好了颜色对花写生,看花的人又围住了画花的,出神地看画家画花。喜欢照相的人,抱着相机跑来跑去,不知是照花,还是照人,是怕人遮了花,还是怕花遮了人,还是要选一个最好的镜头,使如花的人永远伴着最美的花。有人在花下喝茶.有人在花下弹琴有人在花下下象棋,有人在花下打桥牌。昆明四季如春,四季有花,可是不管山茶也罢,报春也罢,梅花也罢,杜鹃也罢,都没有海棠这样幸运,有这么多人。这样热热闹闹地来访它,来赏它,这样兴致勃勃地来赶这个开花的季节。还有桃花什么的,目前也还开着,在这附近,就有几树碧桃正开,显得冷冷落落地呆在一旁,并没有谁去理睬。在这圆通山头,可以看西山和滇池,可以看平林和原野,可是这时候,大家都在看花,什么也顾不得了。
    
    回家的路上,还是听到很多人纷纷议论。
    
    有人说:“今年的花,比去年好,去年,比前年好,解放以前谈不到。”
    
    有人说:“最好早晨来看花,迎风带露的花,会更娇更美。”
    
    有人说:“雨天来看花更好,‘海棠著雨胭脂透’\footnote{〔海棠著雨胭脂透〕出自宋代王雱的词《倦寻芳慢·露晞向晚》。},当然不是大雨滂沱,而是斜风细雨。”
    
    有人说:“也许月下来看花更好,将是花气氤氲。”
    
    有人说:“下星期再来看花,再不来就完了。”
    
    有人说:“不怕花落去,明年花更好。”
    
    好一个“明年花更好”。我一面走着,一面听人家说着,自己也默念着这样两句话:
    
    春光似海,盛世如花。但愿春常在,花入千万家。
    
\end{large}


\newpage

\textbf{注释}:

\vspace{-1em}

\begin{itemize}
    \setlength\itemsep{-0.2em}
    \item 〔写生〕直接以实物或风景为对象绘画。
    \item 〔滂沱〕形容雨下得很大。
    \item 〔发髻〕盘在脑后的头发。
    \item 〔簪〕针形的发饰,别在头发上,用来插定发髻。
    \item 〔伫立〕长时间站立不动。
    \item 〔端详〕仔细地看。
    \item 〔接踵摩肩〕肩碰着肩,脚碰着脚,形容人多拥挤。也写作“摩肩接踵”。
    \item 〔氤氲〕烟气、烟云弥漫的样子;气或光混合动荡的样子。
\end{itemize}

\chapter{少年闰土}

\begin{large}
    
    深蓝的天空中挂着一轮金黄的圆月,下面是海边的沙地,都种着一望无际的碧绿的西瓜,其间有一个十一二岁的少年,项戴银圈,手捏一柄钢叉,向一匹猹\footnote{〔猹〕作者1929年5月4日给舒新城的信中说:“‘猹’字是我据乡下人所说的声音,生造出来的,……现在想起来,也许是獾罢。”}尽力刺去。那猹却将身一扭,反从他的胯下逃走了。
    
    这少年便是闰土。我认识他时,也不过十多岁,离现在将有三十年了;那时我的父亲还在世,家景也好,我正是一个少爷。那一年,我家是一件大祭祀的值年。这祭祀,说是三十多年才能轮到一回,所以很郑重;正月里供祖像,供品很多,祭器很讲究,拜的人也很多,祭器也很要防偷去。我家只有一个忙月(我们这里给人做工的分三种:整年给一定人家做工的叫长年;按日给人做工的叫短工;自己也种地,只在过年过节以及收租时候来给一定的人家做工的称忙月),忙不过来,他便对父亲说,可以叫他的儿子闰土来管祭器的。
    
    我的父亲允许了;我也很高兴,因为我早听到闰土这名字,而且知道他和我仿佛年纪,闰月生的,五行缺土,所以他的父亲叫他闰土。他是能装弶捉小鸟雀的。
    
    我于是日日盼望新年,新年到,闰土也就到了。好容易到了年末,有一日,母亲告诉我,闰土来了,我便飞跑地去看。他正在厨房里,紫色的圆脸,头戴一顶小毡帽,颈上套一个明晃晃的银项圈,这可见他的父亲十分爱他,怕他死去,所以在神佛面前许下愿心,用圈子将他套住了。他见人很怕羞,只是不怕我,没有旁人的时候,便和我说话,于是不到半日,我们便熟识了。
    
    我们那时候不知道谈些什么,只记得闰土很高兴,说是上城之后,见了许多没有见过的东西。
    
    第二日,我便要他捕鸟。他说:
    
    “这不能。须大雪下了才好。我们沙地上,下了雪,我扫出一块空地来,用短棒支起一个大竹匾,撒下秕谷,看鸟雀来吃时,我远远地将缚在棒上的绳子只一拉,那鸟雀就罩在竹匾下了。什么都有:稻鸡,角鸡,鹁鸪,蓝背……”
    
    我于是又很盼望下雪。
    
    闰土又对我说:
    
    “现在太冷,你夏天到我们这里来。我们日里到海边拣贝壳去,红的绿的都有,鬼见怕也有,观音手也有。晚上我和爹管西瓜去,你也去。”
    
    “管贼么?”
    
    “不是。走路的人口渴了摘一个瓜吃,我们这里是不算偷的。要管的是獾猪,刺猬,猹。月亮地下,你听,啦啦的响了,猹在咬瓜了。你便捏了胡叉,轻轻地走去……”
    
    我那时并不知道这所谓猹的是怎么一件东西——便是现在也没有知道——只是无端地觉得状如小狗而很凶猛。
    
    “它不咬人吗?”
    
    “有胡叉呢。走到了,看见猹了,你便刺。这畜生很伶俐,倒向你奔来,反从胯下窜了。它的皮毛是油一般的滑……”
    
    我素不知道天下有这许多新鲜事:海边有如许五色的贝壳;西瓜有这样危险的经历,我先前单知道他在水果店里出卖罢了。
    
    “我们沙地里,潮汛要来的时候,就有许多跳鱼儿只是跳,都有青蛙似的两个脚……”
    
    啊!闰土的心里有无穷无尽的稀奇的事,都是我往常的朋友所不知道的。他们不知道一些事,闰土在海边时,他们都和我一样只看见院子里高墙上的四角的天空。
    
    可惜正月过去了,闰土须回家里去,我急得大哭,他也躲到厨房里,哭着不肯出门,但终于被他父亲带走了。他后来还托他的父亲带给我一包贝壳和几支很好看的鸟毛,我也曾送他一两次东西,但从此没有再见面。
    
\end{large}



\chapter{青山}

\begin{large}
    
    窗外是参天的杨柳。院子在山沟里,山上青翠一片,全是树。我们盘腿坐在土炕上,就像坐在船上,四周全是绿色的波浪,风一吹,树梢卷过涛声,叶间闪着粼粼的波光。
    
    说出来让人难以相信,这是中国的晋西北,是西伯利亚\footnote{〔西伯利亚〕亚洲北部地区,冬季形成寒流南下,进入我国。}大风常来肆虐的地方,是干旱、霜冻、沙尘暴等与生命作对的怪物盘踞之地。过去,这里风吹沙起,能一直埋到城头。当地县志记载:“风大作时,能逆吹牛马使倒行,或擎之高二三丈而坠。”就在如此险恶的地方,我对面这个手端一杆旱烟袋的瘦小老头,竟创造了这处青山绿海。
    
    院子里,一排三间房,就剩下老者一人。老人每天早晨抓把柴煮饭,带上干粮扛上铁锹进沟上山;晚上回来,吃过饭,抽袋烟睡觉。六十五岁那年,他组织了七位老汉,开始治理这条沟,现在已有五人离世。他可敬的老伴,与他风雨同舟一生。一天他栽树回来时,发现她已静静地躺在炕上过世了。他已经八十一岁,知道终有一天自已也会爬不起来。他唯一的女儿三番五次地从城里回来,接他去享清福,他不走。他觉得种树是自己生命的意义,屋后的青山就是生命的归宿。
    
    他敲着旱烟锅不紧不慢地说着,村干部在旁边恭敬地补充着……十五年啊,绿化了八条沟,造了七条防风林带,三千七百亩林网,这是多么了不起的奇迹。去年冬天,他用林业收入资助每户村民买了一台电视机——他还有宏伟设想,还要栽树,直到自己爬不起来为止。
    
    在屋里说完话,老人陪我们到沟里去看树。杨树、柳树,如臂如股,劲挺在山洼、山腰。看不见它们的根,山洪涌下的泥埋住了树的下半截,树却勇敢地顶住了山洪的凶猛。这山已失去了原来的坡形,依着一层层的树形成一层层的梯。老人说:“这树下的淤泥有两米厚,都是好土啊!”是的,保住了这片土,我们才有这绿树;有了这绿树,我们才守住了这片土。
    
    看完树,我们在村口道别。看着屋后的青山,我不禁想,再过十五年,三十年,这里会怎么样呢?老人完成了自己的志愿,也许十五年后,三十年后,他已经不在了,但他的精神就和这青山一样,一定会留下来,也一定要留下来。
    
\end{large}


\newpage

\textbf{注释}:

\vspace{-1em}

\begin{itemize}
    \setlength\itemsep{-0.2em}
    \item 〔粼粼〕水流清澈、闪亮的样子。
    \item 〔肆虐〕不顾一切地任意残害、凌虐。肆:放纵,任意行事,不顾及其它。
    \item 〔盘踞〕用蛮力占据。比喻自然灾害长期在某地肆虐。
    \item 〔县志〕专门记载一个县的历史、地理、风俗、人物、文教、物产等的历史资料,一般由当地政府编修。
    \item 〔擎〕向上托,举起。
    \item 〔淤泥〕水底沉积的泥沙。
\end{itemize}

\chapter{草方格}

\begin{large}
    
    在中国西北的腾格里沙漠\footnote{〔腾格里沙漠〕内蒙古自治区阿拉善地区东南部的沙漠。},有一项人类治沙历史上的奇迹:这里的人们,从沙漠手里夺回了土地。这一切,要从一个看似简单的想法说起:草方格。
    
    天还没亮,我坐上小面包车,准备体验治沙人的一天。路上,我们顺道接了几个今天的工友。他们有的已经戴好了头巾和面罩,有的只戴着鸭舌帽,露出古铜色的脸,沉默不语。一时间气氛有些尴尬,我试着打破沉默:“听说这儿以前都是沙……”
    
    “我们治沙已经几十年了。”旁边的大姐立马接上了话,后座的大哥也说道:“以前一下雨,过了这桥,那边全是沙子,到处都是呢。”
    
    “变化可真大啊!”
    
    “那可不……”
    
    我和工友们聊了起来。他们其实很健谈,一下子,我们就成了朋友。欢声笑语中,车子已经开出了几十公里。迎着朝阳,我们来到了治沙第一线。沙土路的两边是无边的沙丘。很多沙丘上仿佛披上了一层渔网。那网格就是我们今天工作的范本:草方格。
    
    前面没有路了,我下了车,戴上一顶草帽,拿起一把铲子。靴子踩在沙面上,微微陷下去,走起来略微吃力。工友们走得很快,我急忙追上去,一边感慨:没人告诉我,这里如此偏远,路途如此艰难。举目都是黄沙,就连蓝天也仿佛被黄沙侵蚀了。半小时后,我们到了工地。队长宣布,我们今天的任务是二十五亩,也就是17000平方米。惊讶中,我被分配给一位姓马的女士,组成两人小组。
    
    马女士是我认识的第一个治沙人。她戴着白色的鸭舌帽,花布面罩,裹着花头巾,穿着浅灰尼龙风衣\footnote{〔尼龙〕一种人工合成纤维,可用来制作衣服。}和蓝色牛仔裤\footnote{〔牛仔裤〕斜纹布或斜纹粗棉布做的裤子,坚固耐磨。}。这种混乱的打扮似乎是这里常见的风格。相比之下,戴着草帽的我倒像是个来拍照的杂志模特儿。我戴上面罩,开始工作。
    
    在我看来,我们只需要把干草铺在地上,然后用铲子把它扎进沙地里。干草两头会竖起来,像矮墙一样挡风。这样排列的干草墙,组成一个个边长一米左右的方格,就能让沙子不被风吹走。看起来挺简单的,我心想。
    
    干了不一会儿,马女士就指着我的干草墙说:“你这样没扎下去,风一吹就跑了。”说着,她把铲子插进缝里,使劲踩了几脚。队长走过来,看了看,说:“你这边的草方格不行。”他把我的铲子拿过来,横放在地上,“你看,这个铲子有一米长。你这两行的间距,比铲子长多了。”据他所说,我的草方格太大,定不住沙子。作为补救,要在里面再扎一行,这意味着翻倍的工作量。他又指着我的干草墙说:“你看,这里有个缺口。这样风一吹,沙子就过界了。”看着其他人的工作成果,我承认,我弄砸了,只好按照他们说的重做。
    
    一个小时后,我已经精疲力尽了。看看旁边这些四十多岁的人,他们甚至还没流汗。队长见我累了,拉着我坐在干草堆上休息。我们聊了起来。他说,从二十世纪五十年代起,他们就在这里治沙了。以前是用麦草平铺在沙上,一刮风,草就被刮跑了。一天,几个工人吃了饭,用铲子把草扎进沙里写字玩。第二天来的时候,发现草没被风吹走。专家听说了,就过来研究。经过几年的试验,他们发现,一米见方的草方格,固沙\footnote{〔固沙〕指让沙尘不再飘扬移动,逐渐成为土壤。}效果好,成本也低。这个方法一直沿用到今天。
    
    歇了一会儿,我又拿起铲子,和马女士一起劳动。渐渐地,我掌握了诀窍,干起活来也不怎么累了。马女士说,以前,他们工作的地方就在家附近。那时候一刮风,院子里,窗台上,全是沙子。现在,一点沙子也没有了!我能理解她话里的自豪感。我本以为,我绝对无法忍受这样一整天扎草的工作。但事实证明,看着脚下不断产生的方格,成就感能化解心灵和肉体的疲累。
    
    晚上,我参观了附近的中国科学院沙坡头沙漠试验研究站。这个1955年成立的研究站里,钢化玻璃\footnote{〔钢化玻璃〕经过特别处理的玻璃,比普通玻璃坚固得多。}围成两米高的透明罩子,整齐地排列着,里面是同样整齐排列着的小花盆。盆里栽培着各种植物。研究所里的专家通过大量对比试验,培育出固沙效果最好的植物品种,再将它们移植到草方格里,把荒凉的沙丘转化为肥沃的土壤。
    
    第二天,我跟着他们再次来到沙漠里,看着纤弱的草籽被小心地埋进沙里。这些中国人,也许一辈子都在制造这些草方格。日复一日,年复一年。相比广袤的沙漠,每个草方格是这样渺小。但经过六十多年的努力,我们能通过卫星图像\footnote{〔卫星图像〕使用人造卫星在高空拍摄的图像。},看到他们劳动的成果。荒凉的沙漠变成了树林、农田、果园。我们参观了一个薰衣草园\footnote{〔薰衣草〕一种产自地中海沿岸的灌木,6月开紫色花,可作庭园观赏和香薰料使用。},旁边是清澈的小溪、翠绿的草地和干净的农舍,一切和普罗旺斯\footnote{〔普罗旺斯〕法国东南部地区,以种植业著名。}没什么分别。这里已经看不到草方格了,但这个世界会记得,它们曾经存在过。
    
\end{large}


\newpage

\textbf{注释}:

\vspace{-1em}

\begin{itemize}
    \setlength\itemsep{-0.2em}
    \item 〔模特儿〕展示衣饰等时尚设计的职业。
    \item 〔诀窍〕关键的好办法。
    \item 〔健谈〕善于谈话,可以谈很多很久。
    \item 〔尴尬〕处境困难或让人难为情。
    \item 〔亩〕面积单位。十五亩等于一万平方米,一亩约为667平方米。
    \item 〔沿用〕继续用,不改变。沿:顺流而下,引申指一直顺着进行下去,不改变不中止。
    \item 〔广袤〕广阔。
    \item 〔渺小〕指非常小。
\end{itemize}

\chapter{北京的春节}

\begin{large}
    
    按照北京的老规矩,春节差不多在腊月的初旬就开始了。“腊七腊八,冻死寒鸦”,这是一年里最冷的时候。可是,到了严冬,不久便是春天,所以人们并不因为寒冷而减少过年与迎春的热情。在腊八那天,家家都熬腊八粥。这种特制的粥是祭祖祭神的,可是细一想,它倒是农业社会一种自傲的表现——这种粥是用各种米,各种豆,与各种干果(杏仁、核桃仁、瓜子、荔枝肉、桂圆肉、莲子、花生米、葡萄干、菱角米……)熬成的。这不是粥,而是小型的农业产品展览会。
    
    腊八这天还要泡腊八蒜。把蒜瓣放到高醋里,封起来,为过年吃饺子用。到年底,蒜泡得色如翡翠,醋也有了些辣味,色味双美,使人忍不住要多吃几个饺子。在北京,过年时,家家吃饺子。
    
    从腊八起,铺户就加紧上年货,街上增加了货摊子——卖春联的、卖年画的、卖蜜供的、卖水仙花的等等,他们都是只在这个季节才会出现的。这些摊子都让孩子们的心跳得更快一些。在胡同里,吆喝的声音也比平时更多更复杂,其中也有仅在腊月才出现的,像卖松枝的、薏仁米的、年糕的等等。
    
    孩子们准备过年,第一件事是买杂拌儿。这是用各种干果(花生、胶枣、榛子、栗子等)与蜜饯掺和成的,普通的带皮,高级的没有皮——例如普通的用带皮的榛子,高级的就用榛仁。孩子们喜欢吃这些零七八碎儿,即使没有饺子吃,也必须买杂拌儿。他们的第二件事是买爆竹,特别是男孩子们。恐怕第三件事才是买玩意儿——风筝、空竹、口琴等,和年画。
    
    孩子们忙乱,大人们也紧张。他们必须预备过年吃的喝的用的一切,也必须赶快给孩子做新鞋新衣,好在新年时显出万象更新的气象。
    
    腊月二十三过小年,差不多就是过春节的“彩排”。在老年间,这天晚上家家祭灶王,从一擦黑儿,鞭炮就响起来,人们随着鞭炮声把灶王的纸像焚化,美其名曰送灶王上天。在前几天,街上就有好多卖麦芽糖与江米糖的,糖形或为长方块或为大小瓜形。按旧日的说法,用糖粘住灶王的嘴,他到了天上就不会向玉帝报告家中的坏事了。现在,还有卖糖的,但是只由大家享用,并不再粘灶王的嘴了。
    
    过了二十三,大家就更忙了,春节眨眼就到了啊。在除夕以前,家家必须把春联贴好,必须大扫除一次,名曰扫房。必须把肉、鸡、鱼、青菜、年糕什么的都预备充足,至少足够吃用一个星期的——按老习惯,铺户多数关五天门,到正月初六才开张。假若不预备下几天的吃食,临时不容易补充。
    
    除夕真热闹。家家赶做年菜,到处是酒肉的香味。老少男女都穿起新衣,门外贴好红红的对联,屋里贴好各色的年画,哪一家都灯火通宵,不许间断,鞭炮声日夜不绝。在外边做事的人,除非万不得已,必定赶回家来,吃团圆饭,祭祖。这一夜,除了很小的孩子,没有什么人睡觉,都要守岁。
    
    初一的光景与除夕截然不同:除夕,街上挤满了人;初一,铺户都上着板子,门前堆着昨夜燃放的爆竹纸皮,全城都在休息。
    
    男人们在午前就出动,到亲戚家、朋友家去拜年。女人们在家中接待客人。城内城外有许多寺院开放,任人游览,小贩们在庙外摆摊儿,卖茶、食品和各种玩具。北城外的大钟寺、西城外的白云观、南城的火神庙(厂甸)是最有名的。可是,开庙最初的两三天,并不十分热闹,因为人们正忙着彼此贺年,无暇顾及。到了初五初六,庙会开始风光起来。孩子们特别热心去逛,为的是到城外看看野景,可以骑毛驴,还能买到那些新年特有的玩具。白云观外的广场上有赛轿车赛马的,在老年间,据说还有赛骆驼的。这些比赛并不为争谁第一谁第二,而是在观众面前表演骡马与骑者的美好姿态和娴熟技能。
    
    多数铺户在初六开张,又放鞭炮,从黎明到清早,全城鞭炮声不绝。虽然开了张,可是除了吃食与其他重要日用品的铺子,大家并不很忙,铺中的伙计们还可以轮流去逛庙会、逛天桥和听戏。
    
    元宵(汤圆)上市,春节的又一个高潮到了。除夕是热闹的,可是没有月光;元宵节呢,恰好是明月当空。大年初一是体面的,家家门前贴着鲜红的春联,人们穿着新衣裳,可是它还不够美;元宵节,处处悬灯结彩,整条大街像是办喜事,火炽而美丽。有名的老铺都要挂出几百盏灯来:有的一律是玻璃的,有的清一色是牛角的,有的都是纱灯;有的通通彩绘《红楼梦》或《水浒传》故事,有的图案各式各样。这在当年,也就是一种广告。灯一悬起,任何人都可以进到铺中参观,晚间灯中都点上蜡烛,观者就更多。这广告可不庸俗。干果店在灯节还要做一批杂拌儿生意,所以每每独出心裁,制成各样的冰灯,或用麦苗做成一两条碧绿的长龙,把顾客招来。
    
    孩子们买各种花炮燃放,即使不跑到街上去淘气,在家中也照样能有声有光地玩耍。家中也有灯:走马灯、宫灯、各形各色的纸灯,还有纱灯,里面有小铃,到时候就叮叮地响。大家还必须吃元宵啊。这的确是美好快乐的日子。
    
    一眨眼,到了残灯末庙,学生该去上学,大人又去照常做事,春节在正月十九结束了。腊月和正月,在农村正是大家最闲在的时候。过了灯节,天气转暖,大家就又去忙着干活儿了。北京虽是城市,可是它也跟着农村一齐过年,而且过得分外热闹。
    
\end{large}



\chapter{他们那时候多有趣啊}

\begin{large}
    
    那天晚上,玛琪甚至把这件事记在自己的日记里了。2155年5月17日这一天,她打开自拍,说道:“今天,托米发现了一本真正的书!”
    
    这是一本很旧的书。玛琪的爷爷有一次告诉过她,当他还是一个小孩子的时候,他的爷爷曾经对他讲,曾经有那么一个时候,所有的故事都是印在纸上的。
    
    玛琪和托米翻着这本书,书页已经发黄,皱皱巴巴的。里面只有黑色的,静止的文字。没有背景动画,也没有旁白音。他们的视线追着文字,不断往下,很快就到了页底:需要翻页了。这真是有趣极了。读到后面,再翻回来看前面的一页时,刚刚读过的那些字仍然停留在原地。
    
    玛琪问:“你在哪儿找到这本书的?”
    
    “在我们家。”托米指了一下,可并没有抬起头,因为他正在全神贯注地看书。“在顶楼上。”他又说。
    
    “书里写的什么?”
    
    “学校。”
    
    玛琪脸上露出鄙夷不屑的神情:“学校?学校有什么好写的?我讨厌学校。”玛琪一向讨厌学校,可现在她比以往任何时候都更憎恶它。那个机器老师一次又一次地给她做地理测验,她一次比一次答得糟,最后她的妈妈发愁地摇了摇头,把教学视察员找了来。
    
    视察员把机器老师调好以后,拍拍她的脑袋,笑着对她妈妈说:“这不是小姑娘的错,琼斯太太。我认为是软件的问题,地理部分的进度调得太快了一些,这种事是常有的。我把它调慢了,已经适合十岁左右孩子们的水平了。说实在的,她总的学习情况够令人满意了。”说着,他又拍了拍玛琪的脑袋。
    
    玛琪失望极了,她本来希望他把机器老师停掉,他们有一次就把托米的老师停了将近一个月之久,因为历史那部分完全错乱了,净说胡话,需要重新部署。
    
    所以她对托米说:“怎么会有人写学校呢?”
    
    托米非常高傲地瞧了她一眼:“因为那不是我们这种类型的学校,傻瓜。那是几百年前的那种老式学校。”接着他一字一顿地说:“几世纪前。”
    
    玛琪很难过。“嗯,我不知道古时候他们有什么样的学校。”她从他肩膀后面看了一会儿那本书,开口说:“不管怎么说,他们得有一个老师吧?”
    
    “当然,他们有个老师,可不是我们这样的老师。是一个真人!”
    
    “一个真人?真人怎么会是老师呢?”
    
    “是这样的,他只不过给孩子们讲讲课,留些作业,提提问题。”
    
    “真人可没那么聪明。”
    
    “当然聪明啦。我爸爸就和我的机器老师知道得一样多。”
    
    玛琪不打算争吵下去,便说:“我可不想让一个陌生人到我家里来教我功课。”
    
    托米尖声大笑。“你不知道的事太多了,玛琪。那些老师才不到你家里来上课呢。他们有一个专门的地方,所有的孩子都到那儿去上学。”
    
    “所有的孩子都学一样的功课吗?”
    
    “那当然,如果他们的年龄一样的话。”
    
    “可我妈妈说,老师是需要调整的,好适合他所教的每个孩子的智力。另外,对每个孩子的教法都应该是不同的。”
    
    “他们那时候恰恰不是那么做的。如果你不喜欢书里说的这些事,你就干脆别读这本书。”
    
    “我没说我不喜欢。”玛琪急忙说。她很想知道学校是怎么回事。
    
    他们还没看完一半,玛琪的妈妈就喊了起来:“玛琪!该上课了!”
    
    玛琪叹了口气,去上课了。她脑子里还在想着当她爷爷的爷爷是个小孩子的时候,他们上的那种老式学校。附近所有的孩子都到一处去上学,他们在校园里笑啊、喊啊,他们一起坐在课堂里上课;上完一天的课,就一块儿回家。他们学的功课都一样,这样,在做作业的时候他们就可以互相帮助,有问题还可以互相讨论。
    
    而且他们的老师是真人……
    
    机器老师的合成头像正在屏幕上深情地说道:“让我们把这两个分数加在一起──”
    
    玛琪想,在过去的日子里,那些孩子一定非常热爱他们的学校。她想,他们那时候多有趣啊!
    
\end{large}



\chapter{故宫博物院}

\begin{large}
    
    在北京城的中心,有一座城中之城,这就是紫禁城。现在人们叫它故宫,也叫故宫博物院。这是明清两代的皇宫,是我国现存的最大最完整的古代宫殿建筑群,有近六百年历史了。
    
    紫禁城城墙十米多高,有四座城门:南面午门,北面神武门,东西两边分别是东华门、西华门。宫城呈长方形,占地七十二万平方米,有大小宫殿七十多座、房屋九千多间。城墙外是五十多米宽的护城河。城墙的四角,各有一座玲珑奇巧的角楼。故宫建筑群规模宏大,建筑精美,布局统一,集中体现了我国古代建筑艺术的独特风格。
    
    从天安门往里走,沿着一条笔直的大道穿过端门,就到了午门的前面。午门俗称“五凤楼”,是紫禁城的正门。走进午门,是一个宽阔的广场,弯弯的内金水河像一条玉带横贯东西,河上是五座精美的汉白玉石桥。桥的北面是太和门,一对威武的铜狮守卫在门的两侧。
    
    进了太和门,就来到紫禁城的中心——三大殿:太和殿、中和殿、保和殿。三座大殿矗立在七米多高的白石台基上。台基有三层,每层的边缘都有汉白玉栏杆围绕着,栏杆上面刻着龙凤流云,四角和望柱下面伸出一千多个圆雕鳌头,鳌头嘴里都有一个小圆洞,是台基的排水管道。
    
    太和殿俗称金銮殿,高二十八米,面积两千三百八十多平方米,是故宫最大的殿堂。在湛蓝的天空下,那金黄色的琉璃瓦重檐屋顶,显得格外辉煌。殿檐斗拱、额枋、梁柱,装饰着青蓝点金和贴金彩画。正面是十二根红色大圆柱,金琐窗,朱漆门,同台基上的白色栏杆相互衬映,色彩鲜明,雄伟壮丽。
    
    大殿正中是一个约两米高的朱漆方台,上面安放着金漆雕龙宝座,背后是雕龙屏。方台两旁有六根高大的蟠龙金柱,每根大柱上都盘绕着矫健的金龙。仰望殿顶,中央藻井有一条巨大的雕金蟠龙。从龙口里垂下一颗银白色大圆珠,周围环绕着六颗小珠,龙头、宝珠正对着下面的宝座。梁枋间彩画绚丽,有双龙戏珠、单龙翔舞,有行龙、升龙、降龙,多态多姿,龙身周围还衬托着流云火焰。
    
    三大殿建筑在紫禁城的中轴线上。这条线也是北京城的中轴线,向南从午门到天安门延伸到正阳门、永定门,往北从神武门到地安门、鼓楼、钟楼,全长约八公里。
    
    太和殿是举行重大典礼的地方。皇帝即位、生日、婚礼和元旦等,都在这里接受朝贺。每逢大典,皇帝端坐在宝座上,殿外的白石台基上下,文武百官分列左右,中间御道两边排列着仪仗,大殿廊下,鸣钟击磬,乐声悠扬。台基上的香炉和铜龟、铜鹤里点起檀香或松柏枝,烟雾缭绕。
    
    太和殿后面是中和殿。这是一个亭子形大殿,殿顶把四道垂脊攒在一起,正中安放着一个大圆镏金宝顶,轮廓非常优美。举行大典时,皇帝先在这里休息。
    
    中和殿后面是保和殿。雍正以后,这里是举行最高一级考试——殿试的地方。
    
    从保和殿出来,下了石级是一个长方形小广场。广场西起隆宗门,东到景运门。它把紫禁城分为前后两大部分。广场以南,主要建筑是三大殿和东西两侧的文华殿、武英殿,叫“前朝”。广场北面乾清门以内叫“内廷”,是皇帝和后妃们起居生活的地方,主要建筑有乾清宫、交泰殿、坤宁宫和东六宫、西六宫。
    
    乾清宫是皇帝处理日常政务、批阅各种奏章的地方,后来还在这里接见外国使节。
    
    乾清宫后面是交泰殿,交泰殿后面是坤宁宫。坤宁宫是皇后宫,也是皇帝结婚的地方。
    
    乾清宫、交泰殿、坤宁宫合称“后三宫”。布局和前三殿基本一样,但庄严肃穆的气氛减少了,彩画图案也有明显的变化。前三殿的图案以龙为主,后三宫凤凰逐渐增加,出现了双凤朝阳、龙凤呈祥的彩画,还有飞凤、舞凤、凤凰牡丹等图案。
    
    后三宫往北是御花园。御花园面积不是很大,有大小建筑二十多座,但毫无拥挤和重复的感觉。这里的建筑布局、环境气氛,和前几部分迥然不同。亭台楼阁、池馆水榭,掩映在青松翠柏之中;假山怪石、花坛盆景、藤萝翠竹,点缀其间。来到这里,仿佛进入苏州园林。
    
    从御花园出顺贞门,就到了紫禁城的北门——神武门,对面就是景山。据说景山是明代修建紫禁城的时候,用护城河中挖出的泥土堆起来的,现在成了风景优美的景山公园。站在景山的高处望故宫,重重殿宇,层层楼阁,道道宫墙,错综相连而又井然有序。这样宏伟的建筑群,这样和谐统一的布局,不能不令人惊叹。
    
\end{large}



\chapter{海的颜色}

\begin{large}
    
    海是什么颜色的?
    
    对这个问题,估计多数人会回答:蓝的。
    
    什么蓝?怎样的蓝?一定是蓝色吗?
    
    例如在渤海湾\footnote{〔渤海湾〕我国东部海湾,邻接渤海。大连、北戴河、烟台都是渤海湾沿岸城市。},我就没有获得过蓝海的感受。不论在大连、北戴河还是烟台,我看到的海基本上是草绿色的。阴雨天,海是灰蒙蒙的,天与海的色彩最为接近,很难分清哪是天哪是海。浅海上常见黄褐色,可能是因为那里的沙滩是金黄色的缘故。浅海处因为涨潮退潮,因为风浪,因为游泳的人跑来跑去,把沙翻上来,便黄了,而遇到大风浪,便成了黄褐色。风浪特别大的时候,表面是白色的浪花,往下是黄褐色的海,颜色非常分明。
    
    渤海的颜色令人觉得温暖,亲切,随和,让人愿意接近。
    
    我到过西沙群岛,那里的海完全不同。那是深深的湛蓝色,阳光下映出一片金色的光辉。飞鱼在海面上飞行,军舰在海面上行驶,浪花庄严无声。海的颜色神秘而又伟大。人们说这种颜色是由于海非常深。这里的海确实非常深。不能见底。这深深的蓝色令人肃然起敬。
    
    有一年,我去意大利西西里岛\footnote{〔西西里岛〕意大利南部、地中海北部的群岛。},有机会几次下海游泳。海滩的沙子全是白色的,海水则是天蓝色的,晶莹而明亮。在这样的水里游泳,可以隔着海水看到海底的白沙的一切形状,似乎比不隔水(即通过空气)还看得清楚。只是游到深处的时候,往下一看,一片漆黑,漆黑中似乎有几根乱草在水中浮动,不由得让人汗毛倒竖。
    
    还有一年,我到法国参加电影节,顺便看了看摩纳哥\footnote{〔摩纳哥〕法国东南边缘沿地中海的小国。}这个小国的风光,那儿的海也是天蓝色的,但似乎比西西里岛附近的海颜色深一些。
    
    不管海是什么颜色,用手捧起来海水来,却都是无色透明的,似乎这个海那个海与湖泊与江河并无区别,都是水嘛。浪花又都这么白,白得叫人心醉。
    
\end{large}



\chapter{有的人}

\begin{large}
    
    \begin{verse}[0.5\linewidth]
        有的人活着,他已经死了; \\
        有的人死了,他还活着。 \\
        有的人骑在人民头上:“呵,我多伟大!” \\
        有的人俯下身子给人民当牛马。 \\
        有的人把名字刻入石头,想“不朽”; \\
        有的人情愿作野草,等着地下的火烧。 \\
        有的人,他活着别人就不能活; \\
        有的人,他活着为了多数人更好地活。
    \end{verse}
    
    
    \begin{verse}[0.5\linewidth]
        骑在人民头上的,人民把他摔垮; \\
        给人民作牛马的,人民永远记住他! \\
        把名字刻入石头的,名字比尸首烂得更早; \\
        只要春风吹到的地方,到处是青青的野草。
    \end{verse}
    
    
    \begin{verse}[0.5\linewidth]
        他活着别人就不能活的人, \\
        他的下场可以看到; \\
        他活着为了多数人更好地活的人, \\
        群众把他抬举得很高,很高。
    \end{verse}
    
\end{large}



\chapter{撤离班加西}

\begin{large}
    
    2011年2月19日,我在国内休完假,回到班加西\footnote{〔班加西〕利比亚北部的地中海港口城市,全国第二大城市,经济文化中心。}。刚下飞机,就看到机场多了几架战斗机和不少荷枪实弹的士兵。接机的同事也是面色紧张。从机场回营地,一路上,车比往日少了很多。同事告诉我,15日起,市里爆发了游行,形势已经十分紧张。
    
    第二天,我们听说昨晚市里局势突变,发生了交火,死伤十几人,就在我们公司的办公室附近。听同事讲,他们关了灯,摸黑蜷缩在窗户下面躲子弹。流弹时不时飞过,把窗玻璃都打碎了。
    
    城市已经陷入混乱。市政府已经瘫痪了,街上都是武装分子,已经没有警察维持秩序了,监狱里的犯人也逃出来了。听说三公里外的一个武器库被哄抢了。公司下通知提醒,要加强防御措施。我们开始行动起来,给围墙加了铁丝网,把大门用石块堵死。我把钱藏在鞋底,把护照放在贴身衣服里,手机也保持充满电。我们轮流爬上屋顶侦查,看旁边的公路上有没有人聚集。
    
    天刚黑下来,我就听到大门那边传来一声清脆的枪响。所有人都一愣,领导大喊:“到院子里来!紧急集合!”我们排成四列纵队,撤到工地里的另一个营地,没人掉队。当晚,我们在会议室里度过。外头火光冲天,人声嘈杂,大家精神紧张到了极点,一晚上没合眼。
    
    到了早上,我们回营地收拾东西。院子里一片狼藉:没来得及带走的现金、电脑等贵重物品都被抢光了,门窗都被砸破,车也被砸烂了。我们赶忙把粮食和水运到新营地。虽说一夜没睡,大家都不困,还有说有笑的,其实心里非常焦虑。
    
    下午,利比亚分公司\footnote{〔利比亚分公司〕指中国建筑集团第八工程局有限公司利比亚分公司,下面称为中建八局利比亚分公司。}的领导冒险来营地里慰问我们,跟大家说国家和公司都在行动,一定尽快把大家带回国,鼓励我们行动起来,保护自己。
    
    于是我们每人准备了一根削尖的钢管,重新加固了营地的围墙和大门,安排人24小时不间断地巡逻。为了防备歹徒开车撞击,我们在围墙外挖了一条壕沟,进出只留下大门一条通道。
    
    有几个人带着猎枪到营地,想把我们的皮卡车开走。我们一声怒吼,提着钢管围了上去。他们见势不妙,夺路而逃。
    
    这次胜利深深地鼓舞了我们的士气。但由于粮食和水有限,我们开始担心什么时候才能摆脱困境。机场、港口、边境都关闭了,和国内的通讯也中断了。我们联系不上家人。回国恐怕希望渺茫,就连撤离也难。绝望中,我脑海里总闪过两幅画面:老家的院子,和北京的大马路。当天深夜,我爬上屋顶试了好久,终于打通了中国驻利比亚大使馆的电话。我把营地的情况简单做了说明,对方说情况已经知晓,国家正在尽力协调,一定尽快接我们回家。听到这话,我心里终于踏实了。
    
    国家的动作很快。我没想到,次日我们就接到了通知,国家租用的第一艘希腊游轮,已经赶赴班加西港口。剩下的问题是:怎么到港口去。我们的翻译冒着生命危险找了不少卡车,但找不到司机。但很快,当地人知道了这事。不少老百姓自发开车过来帮助我们,还有人给我们带来苹果、鸡蛋、矿泉水、毛毯等物资。到了港口,见到了船,我们终于放心了。公司安排妇女、伤员和最早来的工人第一批撤离,我们普通员工是第二批撤离的。整个撤离过程井然有序。23日,我们就撤到了克里特岛\footnote{〔克里特岛〕希腊第一大岛,位于地中海东部中间,著名旅游度假景点。}。
    
    希腊的华侨华人,不管是留学生,还是做生意的,都放下了手头的事情,组织起来做志愿者,帮我们安排住宿,办理各种手续。刚下船到了酒店,就有同胞自掏腰包买了电话卡,让我们赶紧给家里打电话报平安。中国人遇到灾难时迸发出来的力量和团结,是不可估量的。
    
    事后我得知,仅用三天时间,中建八局利比亚分公司就把九千二百多名本公司员工,九百五十多名孟加拉国、越南的外籍员工和三千七百多名兄弟公司的中国员工安全撤出。整个利比亚一共三万五千八百六十名中国人,在国家的帮助下,都平安撤出。有的人来不了港口,只能走陆路到突尼斯和埃及去。路上要穿过反对派和政府军的地盘。幸好,他们凭着国旗和护照,都安全通过了,没有被为难。当时边境已经关闭,国家派了外交官协调,打开边境让中国人通过。丢了护照的人,只要会唱国歌,就放行了。
    
    我永远记得,登船撤离的时候,当地的好朋友来送我们。他蹲在地上抱头痛哭,对我们说:“我的国家完了。你们可以回国,但我们的家园毁了,要去哪里呢?”我默默地想,总有一天,我们会回去的。美丽的班加西,一定会再次迎来和平。
    
\end{large}


\newpage

\textbf{注释}:

\vspace{-1em}

\begin{itemize}
    \setlength\itemsep{-0.2em}
    \item 〔荷枪实弹〕带着子弹上了膛的枪,随时可以射击。形容高度戒备,随时准备战斗的情况。
    \item 〔瘫痪〕身体部位丧失了感觉或无法运动了。比喻机构散乱,无法正常运作了。
    \item 〔井然有序〕有秩序有规矩,不混乱的样子。井然:整齐不乱的样子。
    \item 〔士气〕士兵的积极战斗意志。比喻群体维持意志的积极主动性、凝聚力和信心。
    \item 〔壕沟〕狭长的凹沟。军事上指把挖出的土堆在前方的狭长凹沟,用来阻止对方前进。
    \item 〔迸发〕由内而外地突然发出。
    \item 〔流弹〕无端飞来的子弹。
    \item 〔蜷缩〕缩成一团。蜷:肢体屈曲。
\end{itemize}

\chapter{北约轰炸大使馆}

\begin{large}
    
    5月7日晚上,北约\footnote{〔北约〕北大西洋公约组织。由美国等一些欧美国家组成的军事霸权组织。}再次摧毁了南斯拉夫\footnote{〔南斯拉夫〕1918年成立于南欧巴尔干半岛上的国家。2003年解体。}的供电系统,贝尔格莱德\footnote{〔贝尔格莱德〕南斯拉夫首都。}一片漆黑。大使馆的工作人员只能通过无线电关注势态发展。大家坐在院子里,一边看北约飞机轰炸和南联盟\footnote{〔南联盟〕1992年后南斯拉夫的简称。}防空炮火的还击,一边讨论形势。我跟大家说,希望写一篇有关他们的文章。一位年轻的外交官建议说,最好是每人拍一张头像的照片,然后每人自己写一段话。我觉得这是个非常好的主意,大家打算明天就开始动手写。我可以赶在下周一发回《环球时报》,做一个整版。可是,这个计划永远不会实现了。
    
    11点半,潘占林大使见天色已晚,就劝大家早点休息。于是大家返回了楼上宿舍。没想到大使的这句话救了我们十几个人的命。
    
    我和爱人小赵刚刚上楼不到一分钟,就听到一声巨响。当时屋里漆黑一片,还没来得及点蜡烛,小赵在卫生间洗手,我在门外和她说话。随着那声巨响,就见到天花板直塌下来,钢筋水泥的碎块就在眼前落下。紧接着又是一声巨响。这是爆炸的声音。整个大使馆一片刺眼的白光。我才意识到,大使馆被击中了。
    
    来不及多想,我们本能地抓起照相机、摄影包和海事卫星电话\footnote{〔海事卫星电话〕海上通信用的电话。通过卫星转接,可以随时全球通信。}就向门口冲去。同一楼道的三个人也出来了两个。大家手拉手,互相搀扶着迈过废墟。浓烟滚滚,气味刺鼻,呛得我们睁不开眼,喘不过气。楼梯已经炸没了,我们拉着从房顶掉下来的、被炮火烧得烫手的钢筋一步一步往下挪,终于来到院子里。
    
    院子里燃着熊熊大火,时不时响起爆炸声。不少人还困在楼里。一些人用着床单和窗帘绳往下爬。五楼的人守着国家财产,没法下楼。大家都想着等同事救出来再走,但情况太糟,只好在浓烟中摸着栅栏绕过弹坑,翻出院墙,呼叫救护人员。
    
    南斯拉夫方面的救援人员也赶到了。我们在院子内外来回跑。院子里的伤员越来越多,使馆一秘\footnote{〔一秘〕一等秘书,外交官职衔。}曹荣飞和另一名外交官郑海峰满面鲜血,意识模糊。办公室主任刘锦荣受了重伤,一只胳膊折了,头部也受了伤。虽然他伤势很严重,却依然守在现场问其他人怎么样了,直到大家把他抬到救护车上。
    
    大家镇定下来,开始清点人数,发现少了四个人。一个是新华社记者邵云环,一个是光明日报记者许杏虎,他爱人朱颖,还有使馆武官任宝凯。大家非常焦急,向救援人员指出他们的住处。可轰炸又开始了。我们感觉整个世界都在颤抖,四处都在爆炸,大家不知道该逃向哪儿。很多人只好原地卧倒。
    
    邵云环是第一个出来的。救援人员在二楼找到了她的尸体,把她绑在担架上,从二楼慢慢运了下来。她的双脚光着,头发散落在脸上,一只胳膊显然是断了,在空中荡来荡去。她应该是第一次轰炸时就死了。看到她的尸体,我再也控制不住情绪,放声大哭。
    
    北约第一次轰炸后,总会在短时间内再次轰炸目标。为了安全起见,救援人员都撤出了大楼,但不少同志心里焦急,想自己冲进去救人,一个叫布什科的雇员带着氧气罐到许杏虎的屋子里摸了一圈,但没找到人。大使、参赞和一些同志一直坚守着,不愿撤离。几个小时后,北约的轰炸远去了,救援人员又跑回来。凌晨三点多的时候,终于找到了许杏虎。能看出,小许死得很痛苦,衣服都破了,双手保持着猛烈挣扎的样子。小许的爱人朱颖也被找到了。她死得更惨,从二楼炸到了地下室。爆炸前十五分钟,我们还和她一起谈笑。她说他们这次回国休假,打算生个孩子。
    
    武官任宝凯是最后一个找到的。大家在各个地方找了半天,最后不甘心,又找来南斯拉夫的救援人员,反复说了好久,才同意再次进楼寻找。上午8时15分,任武官终于被找到了。他被抬出来时已经失去了知觉,但还有呼吸,头部受了伤,脸上满是泥土和鲜血。
    
    北约的这次轰炸完全不是误击。五枚导弹\footnote{〔导弹〕制导炸弹。在控制下飞行并导向目标的炸弹。}从不同方向击来,其中两枚是从使馆的两个角切入的,一枚从邵云环家那边打进来,还有一枚直接从五楼打进地下室,也许他们知道使馆的人平时总在地下室里躲藏。还有一枚是打向大使官邸的,现在官邸已被炸毁,好在大使幸免于难。昨天南斯拉夫外长约万诺维奇等官员来看望我们。约万诺维奇外长说,今晚,北约对另一个国家开战了。的确,从国际法来讲,大使馆是中华人民共和国的领土。我们的同胞在我们的领土上被炸死了。
    
    5月9日,中国驻南斯拉夫大使馆的废墟上,国旗依旧在飘扬。在蓝天、烈火与浓烟的衬托下,五星红旗是那么的醒目,那么的悲壮。
    
\end{large}


\newpage

\textbf{注释}:

\vspace{-1em}

\begin{itemize}
    \setlength\itemsep{-0.2em}
    \item 〔官邸〕分配给重要官员的居所。
    \item 〔熊熊〕形容火势旺盛而猛烈。
\end{itemize}

\chapter{在巴黎传递奥运火炬}

\begin{large}
    
    2008年4月6日晚上8点,我们带着奥运火炬从伦敦\footnote{〔伦敦〕英国首都。}飞抵巴黎\footnote{〔巴黎〕法国首都。}。驻法使馆派了不少人来迎接我们。到了使馆,我们终于放心了。大多数组员倒头就睡——实在太辛苦了。明天又是一场恶战。
    
    奥运会火炬传递\footnote{〔奥运会火炬传递〕奥运会仪式。把在希腊奥林匹亚的古代奥运会会场采集的火种经由各国接力,传递至奥林匹克运动会主场馆的接力活动。}面临的形势极为严峻。欧洲不少反华组织没能赶往伦敦,都憋足了劲儿,要在巴黎捣乱。法国媒体几个月来持续蛊惑宣传,煽动仇恨,反华组织数年来精心引导、设计抹黑叙事,就是要破坏中国形象。他们设计的主要议题是藏独\footnote{〔藏独〕西藏独立运动。流亡西藏旧贵族势力在北约扶持下成立的分裂组织。}和“人权”。巴黎市议会几天前通过议案,接收达赖\footnote{〔达赖〕十四世达赖喇嘛。原名丹增嘉措,藏独组织核心人物。}为巴黎荣誉市民。反华组织制作了“五环手铐黑旗”,到处张贴,刊登广告。反华政客们口出狂言:“肯定要发生很多事情,而且绝不会平静地发生。”使馆和火炬传递小组领导成员连夜开会,研判形势,商讨次日议程。
    
    4月7日12时30分,北京奥运会火炬巴黎接力总指挥蒋效愚在埃菲尔铁塔二层接过火炬,现场展示后交给巴黎市政府代表,再转交第一棒火炬手、法国四百米跨栏世界冠军迪亚加纳。迪亚加纳手持火炬跑下埃菲尔铁塔。北京奥运火炬接力巴黎站传递正式开始。
    
    在藏独等反华组织的煽动和收买之下,数千人一早就在铁塔前的人权广场集结,他们身穿“手铐五环”黑背心、挥舞“手铐五环”黑旗和达赖集团“雪山狮子”旗\footnote{〔“雪山狮子”旗〕藏独组织认定的“国旗”。},高举横幅,狂吹喇叭,狂呼口号。他们从铁塔下穿过,走向铁塔后面的“自由广场”。这些人要一直尾随奥运火炬,与沿途的抗议者汇合,走完全程。他们租用大巴横穿巴黎,租用游艇在塞纳河上穿行,不达目的誓不罢休。
    
    从火炬进入铁塔那一刻起,就不断有人试图抢夺火炬。手持火炬的迪亚加纳刚迈步,巴黎绿党议员加雷尔便张开双臂扑了上来,被安保人员挡了回去。火炬沿着塞纳河前往巴黎市区。许多人高喊“解放中国”、“解放西藏”,挥舞着小旗,蜂拥而上,冲破警方防线,抢占道路,多次试图抢夺火炬。
    
    第二棒是法国著名篮球运动员里加尔多。由于现场秩序十分混乱,超出警方维持秩序的能力,我们决定把火炬转入车内前进。藏独分子大失所望,尾随车队狂奔。
    
    第三棒是中国残疾运动员金晶。十余人疯狂地推开警察,冲向坐在轮椅上的小姑娘。金晶本能地把火炬抱在胸前,愤怒地弯下腰,拼尽全力保护火炬。她的脸颊和下巴被抓破,我方两位护跑手和翻译被打伤。
    
    经过电视大楼时,数十人拦住警察,数十人扑向火炬手。警方发放催泪瓦斯\footnote{〔催泪瓦斯〕一种警用武器。可刺激人的眼睛、面部皮肤、呼吸道,造成强烈不适。}驱赶,一名警察被打得头破血流。传递组再次决定火炬转入车内前进。由于在车内时间较长,我们决定暂时熄灭火炬,下车后再用火炬灯里的母火点燃。有人当即宣称胜利:火炬已被扑灭。警方辟谣说是主动熄灭的,符合奥委会规定,母火安全,不受影响。
    
    火炬传递至香榭丽舍大道。“人权斗士”和藏独分子在栏杆后大呼小叫。火炬绕凯旋门一周,再顺香榭丽舍大道通过协和广场,再次经大巴传递。巴黎市政府大门两边分别出现了“手铐五环”黑旗和达赖集团的“雪山狮子”旗,大批人聚集在市政府广场如野兽般嚎叫。火炬从巴黎圣母院前时,又有人从楼顶放下一面巨幅“手铐五环”黑旗。
    
    有黑暗就有光明,在法国的留学生和爱国华侨向这些恶徒做了针锋相对的斗争。大家高举五星红旗和写有“北京奥运加油”、“同一个世界,同一个梦想”的中法文横幅。火炬传递沿途,到处都有飘扬的五星红旗和两种语言的横幅,加油助威。这是中华民族受到挑衅的一天,也是《义勇军进行曲》、《歌唱祖国》、《团结就是力量》响彻巴黎上空的一天。快到终点时,一个老华侨在地铁站门口对我们说:“你们做的是对的。他们给钱我们去做坏事,我们没有人理他的。”
    
    17时30分,北京奥运会火炬进入巴黎传递终点夏勒蒂体育场。一些人仍然追随而来,进行最后的破坏活动。巴黎警察局发言人在电视直播上总结一天的过程:虽然逮捕了数十人,警方和抗议者各有受伤,但奥运火炬跑完全程,没有被抢走或扑灭。然而电视台记者立刻补了一句:可是,北京奥运会火炬也没有照亮巴黎。
    
    奥运火炬从西方来,到中国去,一路上既照出了真善美,也照出了假丑恶。
    
\end{large}



\chapter{美猴王当弼马温}

\begin{large}
    
    太白金星\footnote{〔太白金星〕道教神话中掌管金星的星君。玉皇大帝的特使,负责传达命令。}与美猴王,同出了花果山洞天,一齐驾云而起。悟空的筋斗云与众不同,转眼就把金星撇在脑后,先到了南天门\footnote{〔南天门〕天庭共有东、南、西、北四大门,南天门是天庭正门。}外。增长天王\footnote{〔增长天王〕镇守南天门的天将,四大天王之一。}见了,急忙领着天兵天将,枪刀剑戟,挡住天门。猴王道:“这个金星老儿,乃奸诈之徒!既请老孙,怎么又教人动刀动枪,阻塞门路?”正嚷嚷间,这边金星也到了。悟空狠狠说道:“你这老儿,怎么哄我?你说奉玉帝\footnote{〔玉帝〕玉皇大帝,又称昊天上帝。道教神话中众神仙的君主。}旨意来请我,怎么又教这些人阻我进门?”
    
    金星笑道:“大王息怒。你从未来过天庭,众天将与你素不相识,怎肯放你进入?等见了天尊\footnote{〔天尊〕对玉皇大帝的尊称。},授了仙箓\footnote{〔箓〕道教神话中天帝赐予的任命文书,受命于天的凭据。},注了官名,随你出入,再无阻拦。”
    
    悟空道:“这般麻烦,也罢,我不进去了。”
    
    金星急忙扯住,向天门高叫道:“那天门天将,且放开路。此乃下界仙人,我奉天尊圣旨,宣他来也。”这增长天王与众天兵才收起刀枪,让出道路。猴王这才安心,信步入门观看。真个是:
    
    初登上界,乍入天庭。瑞气千条喷紫雾,金光万道滚红霓。只见那南天门,碧沉沉,琉璃造就;明晃晃,宝玉妆成。三十三座天宫,一宫宫脊吞金吻兽;七十二重宝殿,一殿殿柱列玉麒麟。金钉攒玉户,彩凤舞朱门。复道回廊,处处玲珑剔透;三檐四出,层层龙凤翱翔。猴王有份来天境,不堕人间免污泥。
    
    太白金星领着美猴王,到了凌霄殿\footnote{〔凌霄殿〕玉皇大帝接见众神仙的大殿。}外。不等宣召,直至御前,朝上礼拜。悟空在旁站直了,也不拜伏参见,侧着耳朵,听金星启奏。
    
    金星奏道:“臣领圣旨,已宣妖仙到了。”
    
    玉帝问道:“妖仙在哪里?”
    
    悟空这才躬身答应道:“老孙便是。”
    
    众仙官都大惊失色道:“这个野猴!怎么不拜伏参见,竟敢这般回答,却该死了!该死了!”
    
    玉帝传旨道:“孙悟空乃下界妖仙,初得人身,不知朝礼\footnote{〔朝礼〕朝见帝王时的礼仪。},姑且恕罪。”
    
    众仙官连称:“谢恩!”猴王也躬身唱喏\footnote{〔唱喏〕古代下属对上级、晚辈对长辈的应答礼节。拱手作揖并口称颂词。}。
    
    玉帝又问各位文武仙官,看有哪处官职空缺,可让孙悟空就任。武曲星君\footnote{〔武曲星君〕道教神话中掌管北斗七星中的开阳星的星君。}启奏道:“天宫里各宫各殿,各方各处,都不少官,只是御马监缺个正堂管事。”
    
    玉帝传旨道:“那就让他做个‘弼马温’罢。”众臣又叫谢恩,孙悟空又躬身唱个喏。玉帝就派木德星君\footnote{〔木德星君〕道教神话中掌管木星的星君。}送他去御马监到任。
    
    美猴王欢欢喜喜,随木德星君到了御马监,领了仙箓,注了官名。事毕,木德星君自回宫去了,而悟空新官上任,把监丞、监副、典簿、力士,大小吏员人等叫来一起,查明本监事务。
    
    御马监共有天马千匹。典簿管征备草料;力士管刷洗马匹、扎草、饮水、煮料;监丞、监副辅佐催办。这猴王查看了文簿,点明了马数。自从当了弼马温,昼夜不睡,保养马匹。日间辛苦操练,夜间殷勤看管。但凡马睡了,赶起来吃草;走脱的,捉回来靠槽。那些天马给他养得肉膘肥满,见了他,抿耳攒蹄,好不亲昵。
    
    不知不觉,过了半月有余。一天早上,闲来无事,众监吏安排酒席,一则给他接风\footnote{〔接风〕接待远来或新来的人。},一则向他贺喜。正在欢饮之间,猴王忽然停杯问道:“我这‘弼马温’是个什么官衔?”
    
    众人答道:“弼马温就是弼马温。”
    
    猴王又问:“这弼马温是几品官\footnote{〔品〕古代官职的等级。}?”
    
    众人道:“没有品。”
    
    猴王道:“没品,肯定是官太大了。”
    
    众人道:“不大,不大,只叫‘未入流’。”
    
    猴王道:“怎么叫做‘未入流’?”
    
    众人道:“弼马温这官儿,最低最小,只不过是个看马的。您到任之后,如此尽责,把马喂得膘肥体壮,也就得个‘好’字;若是马养得瘦了,还要怪责;要是有个损伤,可就得问罪处罚了。”
    
    猴王听了,心头火起,咬牙大怒道:“这般藐视老孙!老孙在花果山称王,逍遥自在。这玉帝老儿,却哄我来替他养马?不做了!不做了!我去也!”呼喇一声,推倒案台,从耳中取出宝贝\footnote{〔宝贝〕指如意金箍棒。},晃一晃,涨到腕口粗细,一路打出御马监,来到南天门。守门的天兵知他受了官职,不敢阻挡,让他打出天门去了。
    
\end{large}


\newpage

\textbf{注释}:

\vspace{-1em}

\begin{itemize}
    \setlength\itemsep{-0.2em}
    \item 〔戟〕长柄兵器。
    \item 〔素不相识〕向来互不认识。
    \item 〔授〕给予,任命。和“受”相对。
    \item 〔旨〕帝王的命令。也称“圣旨”。
    \item 〔宣〕传达帝王的旨意。也称“传”、“召”。
    \item 〔霓〕色彩排列相反的虹,红色在内,紫色在外。
    \item 〔攒〕积聚,拼凑。
    \item 〔玲珑剔透〕形容做工精美、结构奇巧、薄可透光的工艺品。玲珑:小巧精致。剔透:将不透明的材料镂空剔薄至透光。
    \item 〔翱翔〕在高空回旋飞行。
    \item 〔吻兽〕屋脊两端的装饰,常为兽形,又称吞脊兽。
    \item 〔麒麟〕神话传说中的瑞兽。
    \item 〔三檐四出〕指庑殿顶。庑殿顶是皇家建筑中最高级的宫殿屋顶的样式,以“三重檐”、“四出水”为特征。
    \item 〔弼〕辅佐,辅助。
    \item 〔信步〕随意走。
    \item 〔启奏〕向帝王陈述意见或说明事情。启:陈述。奏:呈现,说明。
    \item 〔膘〕肥肉。
    \item 〔昼夜〕白天和夜晚,表示全天。昼:白天。
    \item 〔抿耳攒蹄〕耳朵低垂,两蹄并拢,指乖巧亲昵的样子。
    \item 〔藐视〕轻视,看不起,认为没有价值或令人厌恶。
    \item 〔逍遥〕自由自在,不受拘束。
\end{itemize}

\chapter{海滨仲夏夜}

\begin{large}
    
    夜,来临了。
    
    夕阳落山不久,西方的天空,还燃烧着一片橘红色的晚霞。大海,也被这霞光染成了红色,而且比天空的景色更要壮观。因为它是活动的,每当一排排波浪涌起的时候,那映照在浪峰上的霞光,又红又亮,就像一片片霍霍燃烧的火焰,闪烁着,消失了。而后面的一排,又闪烁着,滚动着,涌了过来。
    
    天空的霞光渐渐地淡下去了,深红的颜色变成了绯红,绯红又变为浅红。最后,当这一切红光都消失了的时候,那突然显得高而远了的天空,呈现出一片肃穆,最早出现的启明星\footnote{〔启明星〕早晨出现于天空东方的金星。},在这深蓝色的天幕上闪烁起来了。它是那么大,那么亮,整个广漠的天幕上只有它在那里放射着令人注目的光辉,活像一盏悬挂在高空的明灯。
    
    夜色加浓,苍空中的“明灯”越来越多了。而城市各处的真的灯火也次第亮了起来,尤其是围绕在海港周围山坡上的那一片灯光,从半空倒映在乌蓝的海面上,随着波浪,晃动着,闪烁着,像一串流动着的珍珠,和那一片片密布在苍穹里的星斗互相辉映,煞是好看。
    
    在这幽美的夜色中,我踏着软绵绵的沙滩,沿着海边,慢慢地向前走去。海水,轻轻地抚摸着细软的沙滩,发出温柔的刷刷声。晚来的海风,清新而又凉爽。我的心里,有着说不出的兴奋和愉快。
    
    我到过不少的海滨城市,那些地方,都令人喜爱,然而,我最喜爱的却还是这儿——威海\footnote{〔威海〕山东东部沿海城市。}。
    
    这是一座非常幽美、安静的城市。不,与其说它是城市,还不如说它是渔村更为合适。它那独特的美也就在这一点上。它没有一般城市那样的喧嚣嘈杂,也没有一般城市那样灰尘弥漫。它有的只是安静和清洁,幽美与和谐。瞧,它的空气是多么清新,简直像用甚么过滤过似的,一星灰尘都没有,不信,你用鼻子嗅一嗅,香喷喷的,甜丝丝的,既清新而又凉爽。还有,那明净的天空,碧蓝碧蓝的,像用水抹洗过似的,一尘不染。即使在下雨的时候,也使人感到清爽舒畅。而不像江南的黄梅天\footnote{〔黄梅天〕长江中下游地区每年初夏时节的阴雨天气,潮湿闷热。}那样使人烦闷。
    
    夜风轻飘飘地吹拂着,空气中飘荡着一种大海和田禾相混合的香味,柔软的沙滩上还残留着白天太阳炙晒的余温。那些在各个工作岗位上劳动了一天的人们,三三两两地来到了这软绵绵的沙滩上,他们浴着凉爽的海风,望着那缀满了星星的夜空,尽情地说笑,尽情地休憩。愉快的笑声,不时地从这儿那儿飞扬开来,像平静的海面上不断地从这儿那儿涌起的波浪。
    
    我漫步沙滩,徘徊在我的乡亲朋友们中间。
    
    我看到,在那边,在一只底儿朝上反扣在沙滩上的木船旁边,是一群刚从田里收割麦子归来的人们,他们在谈论着今年的收成。今春,雨水足,麦苗长得旺,收成比去年好。眼下,又下了一场透雨,秋后的丰收局面,也大体可以确定下来了。人们为这大好年景所鼓舞着,谈话中也充满了愉快欢乐的笑声。
    
    月亮上来了。
    
    是一轮灿烂的满月。它像一面光辉四射的银盘似的,从那平静的大海里涌了出来。大海里,闪烁着一片鱼鳞似的银波。沙滩上,也突然明亮了起来,一片片坐着、卧着、走着的人影,看得清清楚楚了。啊!海滩上,居然有这么多的人在乘凉。说话声、欢笑声、唱歌声、嬉闹声,响遍了整个的海滩。
    
    月亮升得很高了。它是那么皎洁,那么明亮。
    
    夜已经深了。
    
    沙滩上的人,有的躺在那软绵绵的沙滩上睡着了,有的还在谈笑。凉爽的风轻轻地吹拂着,皎洁的月光照耀着。让这些英雄的人们,在这自由的天幕下,干净的沙滩上,海阔天空地尽情谈笑吧,酣畅地休憩吧。
    
\end{large}


\newpage

\textbf{注释}:

\vspace{-1em}

\begin{itemize}
    \setlength\itemsep{-0.2em}
    \item 〔霍霍〕光芒闪亮。
    \item 〔绯红〕鲜红。绯:红色。
    \item 〔广漠〕广大空旷。
    \item 〔次第〕一个挨一个。
    \item 〔苍穹〕苍天,天空。
    \item 〔煞〕很,非常。
    \item 〔皎洁〕(月亮)明亮洁白。
    \item 〔弥漫〕布满。
    \item 〔休憩〕休息。
\end{itemize}

\chapter{延安的秋天}

\begin{large}
    
    那是1942年,为了响应毛主席“自己动手,丰衣足食”的号召,全延安,全陕甘宁边区,热火朝天地开展了大生产运动。我们中央党校,在校务部莫部长的领导下,上上下下没有一个闲人。按照校务部的规定,我们除了参加学校一部分农业生产以外,还要种半亩蔬菜,完成规定的纺棉线的任务。
    
    莫部长是全校最忙的一个人。可是,傍晚生产时间里,他也扛着纺车,到院里和我们一起学纺线。他学得很认真,指导员讲怎样卷棉卷、上锭子、摇把、上穗子之后,他还要提几个问题,叫指导员看着他做几遍。
    
    一天夜里,我从校务部平房前走过,听到了单调而柔和的纺线声,“嗡嗡嗡……”我心里奇怪,这是谁在夜里纺线呀?我走向前一看,原来是莫部长在向他老伴儿学纺线!在黄昏的油灯下,莫部长的老伴胡大姐坐在窗口纺车前面,右手熟练的摇着车把,左手灵巧的握着棉卷儿向后抽。一根又硬又细的白线,越抽越长。她边纺边说:“呶,就这么轻轻的抽,不紧不慢,两只手配合好,你看……”说着,她把叶轮加速地转了几个圈,白线旋紧了,左手往高一抬,向前一送,右手轻轻捷地摇了半转叶轮,那根又匀又细的白线,像变魔术似的,一下子卷到了穗子上。“就这样上线,你来试试!”
    
    莫部长小心的坐到小板凳上去,按着老伴儿指点,谨慎的纺着,可他就是纺不好,不是断线,就是粗细不匀,一股节一股节扯断的线,被扔在车锭子一旁。大姐气虎虎地说:
    
    “你呀,怎么这样笨哇?……像你这样纺,一斤不糟蹋半斤才怪呢!”
    
    莫部长对大姐笑着,说:“你这是啥教员?一点耐心也没有。”说着他从口袋里掏出一块粗布手帕,在发红的两眼上拭了拭,说:
    
    “鬼眼睛嘛,……眼花了,不顶用啰!”
    
    大姐生气道:“你呀,不要无理强辩三分。你说你不会,不就行了?……我不是说大话,月亮底下纺线我也不会断一根。”
    
    我忍俊不禁,笑出声来:好厉害的大姐呀,你要求也太高了,谁学也有个过程啊。我轻轻地推开门,笑嘻嘻地说:
    
    “莫部长,你真积极呀!”
    
    莫部长把脸一沉,埋怨道:“积极?人家只嫌我笨哩!……笨就笨吧,笨鸟就得先飞哇。”
    
    延安秋日的黄昏,是一个美好而漫长的时刻。蔚蓝的天空里,飘着一缕缕白云。在夕阳的辉映下,现出灿烂的光彩。延河边上,三五成伙的青年男女,有的坐在大青石上,濯足于延水之中;有的一边洗衣裳、一边说笑。碧清的流水,从他们脚面上、小腿旁缓缓向东流去。一曲曲优美、动人的陕北民歌,随风荡漾。
    
    延河川里广阔的田野上,人们分头在生产田劳动着。有的给大白菜浇水、施肥,有的摘西红柿,有的拉秧整地……大家学习、工作了一天,现在正是让大脑休息,活动胳膊腿儿的好时光。这种业余劳动的乐趣,成为一种最美好的享受了。
    
    西方天边的霞光照耀在校务部门前的草坪上,把这里的一切都抹上了一层金色的光。秋风迎着新凉,送来花圃里波斯菊的馨香。空气是这么新鲜,温度是这么适宜,草坪里绿绒绒的,山坡上的野花娇艳多姿,真叫人甜蜜欲醉,心旷神怡。
    
    不必鸣钟,不必催促。人们搁下饭碗就扛着纺车来了。女同志来了,男同志来了,勤务员小鬼们来了,莫部长也来了!刹那间,从草坪一头那棵高大的白杨起,纺车一个挨着一个,一行挨着一行地排起来。一个临时的纺线场就这样形成了。
    
    “嗡嗡嗡……”车轮飞速的转动起来,“嗡嗡嗡……”美妙的合唱渐渐进入高潮。
    
    在这合唱中,也汇合了莫部长纺车的有节奏的鸣响。你看,他摆开了架势,一手摇把,一手抽线。一根又细又匀又白的棉线,飞也似的往线穗子上窜。他纺的那么熟稔,使人不禁感到,他是个稀有的老手,纺线已经到了得心应手的地步。
    
    “嗡嗡嗡……”,“嗡嗡嗡……”,纺车不停地歌唱着。歌声越来越嘹亮,越来越雄壮。这豪迈的自力更生大合唱,是我们这个时代的强音。这种新奇的合唱,给延安生活带来了特有的生气,把延安人的心连接在一起,使我们前进的步伐迈得更大,更快。世界上还有什么音乐比它更动人吗?
    
\end{large}



\chapter{绿宝团}

\begin{large}
    
    夏天的傍晚,我正坐在天山牧场上的一条小河边,和几个哈萨克族\footnote{〔哈萨克族〕我国少数民族,主要分布在新疆地区。}的牧民闲谈,忽然,附近河湾的野树丛里传来马蹄击溅起浪花的响声。我们回头一看,一个长得很俊的姑娘,正骑着马闪出树丛,趟着水流斜过河来。
    
    “绿宝团!”坐在我身边的牧民们突然喊着,跳起来,都飞奔到河边迎接。
    
    姑娘还牵着一匹马。那马驮着用帆布紧紧包裹着的驮子\footnote{〔驮子〕牲口背上负载的货物。},好像很重。看样子,她已经赶了一天路,马疲乏得厉害,那上了驮子的马,累得上不了岸。
    
    有牧民跳进河里去。岸上拉,水里推,这才把马弄上来。
    
    那姑娘下了马,就被牧民们高高兴兴地簇拥着,往搭满了帐篷的聚居区走去了。
    
    我和一个替她牵着马的年轻牧民一块走在后面。姑娘笑声清亮。我从人缝里打量着她的背影。她比那几个强壮的牧民几乎低一头,两条辫子垂到腰,裤腿塞进泥糊糊的马靴里,后脑勺上扣着一顶小花帽。
    
    “她是谁?”我悄悄地问牵马的小伙子。
    
    “我们的绿宝团!”他亲切地说,晚霞照着他得意的脸孔。
    
    我正想再问,忽然帐篷那边有人在喊。整个聚居区都骚动起来了。多少孩子光着脚丫欢欢乱蹦地奔来,多少姑娘妇女飘曳着长裙跑出帐蓬来。到处都是欢呼,这刚日落的草原简直成了翻腾的海洋。
    
    一个戴着白头巾的老婆婆蹒跚地挤过来,抱着姑娘直亲脸,欢喜的泪珠在霞光中闪动。
    
    “快去告诉山那边的人,说绿宝团来了!”一个连鬓白胡子的老人,高声吩咐着他跟前的半大孩子。那孩子就骑上一匹膘壮的马,飞也似地去了。
    
    家家妇女都想把这个叫绿宝团的姑娘拉到自己的帐篷里吃晚饭。但她笑着摆脱了每一只手。趁着天还没黑,她忙着选好一片最平坦的草地,架起了一部机器。
    
    我仔细一看,那是一部小型的电影放映机。原来,这姑娘是一个电影放映员。
    
    牧民们争着竖起杆子,替绿宝团挂幕布。洁白的幕布挂到杆子上了,在笼罩着草原的淡青的暮色中格外显眼,仿佛海上的一面白帆。
    
    暮色中,草原上响起了嘈杂的马蹄声。山那边的牧民,男男女女,老老少少,都骑着马赶来看电影了。
    
    草原的天很快就黑下来。绿宝团连一碗马奶也来不及喝,就开动了小发电机。轰鸣声把草原震得微微发颤。电影放映机旁边的电灯亮起来了。灿烂的灯光照得大伙儿直眯眼睛,四周响起了快乐的笑声。
    
    绿宝团站在灯光下上片子,无数对眼睛都望着她。她嘴里含笑,眉眼间流露出自豪。
    
    电灯灭了,银幕上出现了雄伟的天安门和狂欢的人流,草原的夜旋即被牧民的欢呼声震动了。新闻片放过后就是故事片。大家都被银幕上的动人情节吸引住了,草原上一片寂静,只剩下发电机转动的声音。
    
    突然,放映机发生了故障,电灯又亮起来了,只看到绿宝团在灯光下排除故障。牧民们安静地等待着,可她还是急得手忙脚乱,饱满的额头上冒出一颗颗闪亮的汗珠。
    
    草原的夏夜虫子多,一见灯光都飞来了。密密麻麻的虫子落到她身上,叮她的手,咬她的脸。她忍着疼痒,只集中精神排障。
    
    一个姑娘和一个老牧民,各自手拿一把马尾拂,在她左右给她赶虫子。一个老婆婆也跑过去,用绣花巾拭去她脸上的汗。
    
    放映机终于修好了,电影继续放映。
    
    电影放完了,牧民们散开来,但谁也舍不得走。很多人从小河边砍回来许多枯树枝,烧起了一堆堆篝火,把夜空和草原都照红了。几个青年在火光中使劲地擂起了羊皮鼓。
    
    绿宝团在火边吃羊排,听见羊皮鼓一响,就跳到篝火圈圈当中去,快活地跳起舞来。她的哈萨克舞跳得很好,舞姿灵活、刚健、优美,两根乌黑的长辫子飞一般地旋动。姑娘们羡慕她,青年们喜欢他,老牧民们为她点头捋胡子,老婆婆们看着她慈祥地笑。
    
    大家的脚慢慢地移动了,不论是大胡子的老汉,还是衫角上坠着银饰的姑娘,都走到篝火中间,一起跳起舞来。几百人起舞,把绿宝团围在中心了。一直到篝火都烧成了灰烬,住在山那边的牧民才骑马回去。
    
    夜已深了,草原恢复了宁静。
    
    绿宝团和我在一顶帐篷里过夜。这是牧业生产合作社党支部书记的家。
    
    我躺下了。为了贪图草原上夏夜的凉爽,连被子也不盖。萤火\footnote{〔萤火〕萤火虫发的光。}似的酥油灯\footnote{〔酥油灯〕用牛羊奶提取的酥油做灯油的灯。}下,绿宝团的影子闪到我跟前来。
    
    “同志,我们这牧区天亮前很冷的!”
    
    一条柔软的毛毯盖到了我身上。
    
    我非常惊讶,不是因为她态度大方,而是因为她的汉话说得这么好。
    
    “绿宝团同志,你的家在自治州里吗?”我问。
    
    她低声笑了。
    
    “你以为我是哈萨克人吗?我家离这万把里!”
    
    我惊得掀开毯子坐了起来:“你不是这里人?”
    
    她轻巧地在我旁边的地铺上坐下来:“老家在广西,我是僮族\footnote{〔僮族〕即壮族,我国少数民族,主要分布在广西壮族自治区。}人!”
    
    我简直不敢相信,她是一个僮族姑娘!那她怎么会从老远的广西跑到这牧区来工作呢?
    
    她说她1950年就参军到了这边疆。开头是文工队员,后来就调到地方上,在乌鲁木齐\footnote{〔乌鲁木齐〕新疆省会和最大城市。}上了一期电影训练班。在训练班一毕业,她就自愿请求分配到这牧区来工作。
    
    “你就这么一个人在牧区跑来跑去?”我有点替她担心。
    
    “不!我是组长,还有两个组员呢。因为这一带山上牧群多,跑不过来,我就叫他们另外带了一部放映机分开跑。”
    
    显然她没有理解我的意思,我就直接地说了:“牧区这么大,你一个人跑不害怕?”
    
    她在酥油灯下很奇怪地看了我一眼:“不害怕,牧民们哪一个不认得我!”
    
    “山上不是有狼吗?听说还有黑熊!”
    
    她笑了,笑声虽然很低,但很豪迈:“我有时还在半道上打猎呢,可谁也看不出我有枪!”
    
    她坦白地告诉我,她从小就是个顽皮姑娘。在故乡洛满\footnote{〔洛满〕广西柳州市柳南区西北部的镇。}那个小小的山城里,每个夏天她都要在小河里捉鱼,悄悄地跑到河边的果林里偷摘柠檬吃。
    
    我问她为什么要从出产柠檬的炎热的南方,来到这群峰终年积雪的天山上。她说得很天真,是因为当时有个同学画了一群骆驼,她听说骆驼出在西北边疆的大沙漠上,就要求参军到这里来了。
    
    “骆驼自然很多,可是没有柠檬树!”我笑着说。
    
    “这地方虽然没有我爱吃的柠檬,可离这不远的果子沟里就有野苹果,和柠檬一样酸甜!”她机灵地回答。
    
    “南方多花,这里多雪。”
    
    “我也喜欢花,我也喜欢雪。”
    
    她又告诉我,她特别喜欢在雪原上奔驰。冬天太冷,牧区还没有建立电影院,外面不能放电影,她就从这个牧区跑到那个牧区,给牧民们在冬窝子\footnote{〔冬窝子〕游牧民冬天为避寒防风选择居住的山窝。}里上课。
    
    “你懂得哈萨克文?”
    
    “一学就会了。”
    
    “在雪天里跑,不冷吗?”
    
    “有时不碰巧,跑到半道上天就黑了。我就只好裹紧皮大衣在雪里睡觉。”
    
    “那不冻坏了!”我惊讶地说。
    
    “简直冻得不见了!”在朦胧的灯光里,她闪着洁白的牙齿笑道,“天亮时,雪把我全埋住了。可我从雪窝里爬出来,浑身还是好好地!”
    
    对这样一个早早就远离家乡的姑娘,我不由得问起她的母亲。
    
    “我妈在柳州\footnote{〔柳州〕广西第二大城市,位于广西中北部,工业发达。}的纺织厂里做工。”
    
    “你妈不想你?”
    
    “她来信说,我离家那时候栽在院子里的柚子树,今年已经结柚子了!”
    
    我第一次发现她的眼里流露出一种少女的柔情。
    
    “你还没有爱人吧?”
    
    她并不觉得我问得太突然,只是有点含羞:“有。”
    
    “他也在这边疆吗?”
    
    “不,在海上。”她低头轻柔地说。
    
    “怎么是在海上?”我很诧异。
    
    她把头骄傲地一抬:“他是个海员!”
    
    “你们怎么会认识的?”我越觉得奇怪了。
    
    “他就是我那个画骆驼的同学。”
    
    我沉默下来,心想:一个在祖国极东的大海里,一个在祖国极西的高山上……
    
    她立刻猜到了我沉默的缘由,就微笑着说:“我跟他都还年轻。正该趁这个时候,给国家多干点工作。”
    
    夜已经很深了,快点完的酥油灯,焰光扑闪扑闪的。我和绿宝团各自躺下了。我闭上眼睛,但这个可敬的姑娘的身影还浮现在我的脑海里。
    
    我也不知道是什么时候睡着的,刚天亮,我就被羊群的叫声吵醒了。
    
    绿宝团已经不在帐篷里了。我走出去,党支部书记的爱人戴着桃红花边白头巾,正在帐篷边一棵结满了红果子的小树下挤牛奶。
    
    “绿宝团呢?”我问。
    
    她把手往东边远远一指,一座被朝霞染红了的大雪山。看不见的山脚下,绿宝团已经在路上了,她要把欢乐带给更多的人。
    
\end{large}


\newpage

\textbf{注释}:

\vspace{-1em}

\begin{itemize}
    \setlength\itemsep{-0.2em}
    \item 〔篝火〕在野外燃起的一堆柴火。
    \item 〔捋〕用手握着条状、须状物,顺着移动、抚摩。
    \item 〔擂〕敲打,捶打。
    \item 〔灰烬〕物品燃烧后剩余的东西。
    \item 〔刚健〕强健有力,硬朗康健。
    \item 〔蹒跚〕走动迟缓、摇晃不稳的样子。
    \item 〔膘壮〕指牲畜肥壮结实。
    \item 〔扑闪〕形容眨眼睛或翅膀扇动。引申为灯光明暗不定。
    \item 〔手忙脚乱〕指忙乱地行动,顾及不了其他事。
\end{itemize}

\chapter{我为少男少女们歌唱}

\begin{large}
    
    \begin{verse}[0.5\linewidth]
        我为少男少女们歌唱。 \\
        我歌唱早晨, \\
        我歌唱希望, \\
        我歌唱那些属于未来的事物, \\
        我歌唱那些正在生长的力量。
    \end{verse}
    
    
    \begin{verse}[0.5\linewidth]
        我的歌呵, \\
        你飞吧, \\
        飞到年轻人的心中, \\
        去找你停留的地方。
    \end{verse}
    
    
    \begin{verse}[0.5\linewidth]
        所有使我像草一样颤抖过的 \\
        快乐或者好的思想, \\
        都变成声音, \\
        飞到四方八面去吧, \\
        不管它像一阵微风 \\
        或者一片阳光。
    \end{verse}
    
    
    \begin{verse}[0.5\linewidth]
        轻轻地从我琴弦上, \\
        失掉了成年的忧伤, \\
        我重新变得年轻了, \\
        我的血流得很快, \\
        对于生活我又充满了梦想,充满了渴望。
    \end{verse}
    
\end{large}



\end{document}
