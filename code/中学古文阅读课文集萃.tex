\documentclass[12pt,UTF-8,openany]{ctexbook}
\usepackage{ctex}
\usepackage{titlesec}
\usepackage{xeCJK}
\usepackage{verse}
\usepackage{fontspec,xunicode,xltxtra}
\usepackage{xpinyin}
\usepackage{geometry}
\usepackage{indentfirst}
\usepackage{pifont}
\usepackage{enumitem}
\usepackage[perpage,symbol*]{footmisc}
\usepackage[table,dvipsnames]{xcolor}

\geometry{a5paper,left=1.4cm,right=1.4cm,top=2.3cm,bottom=2.3cm}
\renewcommand{\footnotesize}{\fontsize{8.5pt}{10.5pt}\selectfont}
\setmainfont{Mona Sans Light}
\setCJKmainfont[BoldFont=STZhongsong]{汉字之美仿宋GBK 免费}
\xeCJKDeclareCharClass{CJK}{`0 -> `9}
\xeCJKsetup{AllowBreakBetweenPuncts=true}
\DefineFNsymbols{circled}{{\ding{192}}{\ding{193}}{\ding{194}}{\ding{195}}{\ding{196}}{\ding{197}}{\ding{198}}{\ding{199}}{\ding{200}}{\ding{201}}}
\setfnsymbol{circled}
\xpinyinsetup{ratio=0.5,hsep={.6em plus .6em},vsep={1em}}

\titleformat{\chapter}{\zihao{-1}\bfseries}{ }{16pt}{}
\titleformat{\section}{\zihao{-2}\bfseries}{ }{0pt}{}
\title{\zihao{0} \bfseries 中学古文课文集萃}
\setlength{\lineskip}{24pt}
\setlength{\parskip}{6pt}
\author{}
\date{}
\begin{document}
\maketitle
\tableofcontents
\newpage

\chapter{《论语》十则}

\begin{normalsize}
    
    子曰:“学而时习之,不亦说乎?有朋自远方来,不亦乐乎?人不知而不愠,不亦君子\footnote{〔君子〕君王的儿子,泛指道德高尚的人。}乎?”
    
    子曰:“我非生而知之者,好古,敏以求之者也。”
    
    曾子\footnote{〔曾子〕曾参,字子舆,春秋时鲁国人,孔子的弟子,孔子学说的主要继承人,参与编制了《论语》。}曰:“吾日三省吾身:为人谋而不忠乎?与朋友交而不信乎?传不习乎?”
    
    子曰:“温故而知新,可以为师矣。”
    
    子曰:“学而不思则罔,思而不学则殆。”
    
    子曰:“由\footnote{〔由〕仲由,字子路,春秋时鲁国人,孔子的弟子。忠厚正直,力大勇武。},诲汝知之乎!知之为知之,不知为不知,是知也。”
    
    子曰:“见贤思齐焉,见不贤而内自省也。”
    
    子曰:“三人行,必有我师焉。择其善者而从之,其不善者而改之。”
    
    子贡\footnote{〔子贡〕端木赐,字子贡,春秋时卫国人,孔子的弟子。贤能善辩,曾任鲁国、卫国的相国,还善于经商。}问曰:“有一言而可以终身行之者乎?”子曰:“其恕乎!己所不欲,勿施于人。”
    
    子在川上曰:“逝者如斯夫,不舍昼夜。”
\end{normalsize}


\newpage

\textbf{注解}:

\vspace{-1em}

\begin{itemize}
    \setlength\itemsep{-0.2em}
    \item〔不亦君子乎〕君王的儿子,表示将来要做君王的人。儒是向统治者解释经书的人。因此儒家讲述道理的对象和理想典范都叫做“君子”,因为儒家认为只有道德高尚,符合儒家学说道德要求和规范的人,才是将来可以做君王的人。
    \item〔吾日三省吾身〕古人以三为约数,表示大概的“好几次”、“不止一两次”,但又明显少于十次的次数。后面“三人行”也是如此。
    \item〔知之为知之,不知为不知,是知也〕什么叫“知道”?不仅要知道自己知道什么,还要知道自己不知道什么。这样才叫知道了。明明知道,却以为自己不知道;或者明明自己不知道,却以为自己知道,都不能叫做真的知道。《论语注疏》:“此章明知也。孔子以子路性刚,好以不知为知,故此抑之。”《荀子·儒效篇》:“知之曰知之,不知曰不知,内不以自诬,外不以自欺,以是尊贤畏法而不敢怠慢,是雅儒者也。”
    \item〔择其善者而从之,其不善者而改之〕找出别人做得好的,就跟着去做,发现别人做得不好的,如果自己也有这个毛病,就(自己)改正它。通过观察他人的长处和短处,就能学到东西,自我改进,这就是“有我师”,而不是真的说在几个人里面找到一个人做我的老师。
    \item〔其恕乎!己所不欲,勿施于人〕这是后世“忠恕”的来源之一。忠,谓尽心为人;恕,谓推己及人。“恕”者“如”也,像对待自己一样对待他人就是“恕”。“己所不欲,勿施于人”就是典型的“恕”的表现。“恕”的本义不是现代汉语中的“宽恕”、“饶恕”的意思。因此“忠恕”也不是说要无条件地宽恕别人,做滥好人。
\end{itemize}

\chapter{《使琉球记》}

\begin{normalsize}
    
    出使琉球
    
    初八日己丑,晴。午风大。黎明,有二白鸟绕船而飞。午刻,丁风,仍用辰针,计行四更。申刻,过米糠洋。漩皆圆,波浪密而细,如初筛之米,点点零落;“米糠”字,极有形容。日落,计又行三更;船伙长\footnote{〔伙长〕船上掌管罗盘的人。}云:“鸡笼山、花瓶屿去船远,不应见”。是夜,用乙辰针,行船六更。舟中吐者甚多;余日坐将台\footnote{〔将台〕阅兵点将台,指挥者坐的位置。现在称为舰桥。},全不觉险,饮食如常。
    
    初九日庚寅,晴。卯刻,见彭家山\footnote{〔彭家山〕即彭佳屿,在台湾省基隆港北约30海里。}。山列三峰,东高而西下。计自开洋,行船十六更矣;由山北过船。辰刻,转丁未风,用单乙针,行十更船。申正,见钓鱼台\footnote{〔钓鱼台〕即钓鱼岛,在彭佳屿东约80海里,属基隆。},三峰离立如笔架,皆石骨。惟时水天一色,舟平而驶;有白鸟无数绕船而送,不知所自来。入夜,星影横斜,月色破碎,海面尽作火焰,浮沉出没。
    
    初十日辛卯,晴。丁未风,仍用单乙针。东方黑云蔽日,水面白鸟无数。计彭家至此,行船十四更。辰正,见赤尾屿;屿方而赤,东西凸而中凹,凹中又有小峰二。船从山北过。有大鱼二,夹舟行,不见首尾,脊黑而微绿,如十围枯木附于舟侧;舟人举酒相庆。
    
    十一日壬辰,阴。丁未风,仍用单乙针。计赤尾屿至此,行十四更船。午刻,见姑米山。山共八岭,岭各一、二峰,或断或续;舟中人欢声沸海。未刻,大风,暴雨如注,然雨虽暴而风顺。酉刻,舟已近山,计又行五更船。球人以姑米多礁,黑夜不敢进,待明而行。丑刻,有小船来引导;乃放舟由山南行。
    
    十二日癸巳,晴。辰刻,过马齿山。山如犬牙相错,四峰离立,若马行空。计又行七更,船再用甲寅针,取那霸港。考历来针路所见,尚有小琉球、鸡笼山、黄麻屿;此行俱未见。问之琉球伙长,年已六十,往来海面八次,云此次最为简捷,而所见亦仅三山,即至姑米。惟纪更以香,殊难为据。据琉球伙长云:“海上行舟,风小固不能驶,风过大亦不能驶;风大则浪大,浪大力能壅船,进尺仍退二寸。惟风七分、浪五分,最宜驾驶;此次是也。从来渡海,未有平稳而驶如此者。”辰刻,进那霸港。午刻,登岸。倾国人士聚观于路,世孙\footnote{〔世孙〕指琉球中山王国第十五代国王尚温。当时琉球第十四代国王尚穆病逝,尚穆的儿子尚哲已死,李鼎元奉旨到琉球册封尚穆的孙子尚温为下一代国王。由于册封还没完成,所以称为世孙。}率百官迎诏如仪。
    
    琉球见闻
    
    十七日戊戌,晴。阅案头食单,有所谓“龙头虾”者。取视之,长尺馀,绦甲朱髯、血睛火鬣,类世所画龙头,见之悚然!取其壳以为灯,可供两日玩;三日而色变矣。
    
    十八日己亥,雨。栽荔枝于使院庭后,南北分列。移自牧荔园,种曰“陈家紫”。
    
    二十一日壬寅,阴。连日食海味,腹渐作泻。令庖人但供时蔬、淡粥。庖人乃以佳苏鱼进;问之,曰:“此非鱼名也,系削黑鳗鱼脊肉,干而为之”。长五、六寸许,形如梭、质如枯木。食法,先以温水浸洗,裹蕉叶煨之,切片如刨花,连五、七片不断,又如兰花;宜清酱\footnote{〔清酱〕即“酱清”,生抽酱油。},颇有异味。清酱甘美,十倍于闽。惟求“佳苏”之义不得,适有长史至,问亦不解;因呼至前细核之,据云“此品在敝国既多且美,自王官以及贫民皆得食”;意殆谓如家常蔬菜,人人得食也。球人字皆对音,殆实为“家蔬”也。
    
    二十二日癸卯,晴。午后,偕介山\footnote{〔介山〕指这次出使琉球的大使赵文楷,字介山。李鼎元是副使。}策骑游波上山。一名石筍崖,以形似名之也;石垣四周,垣后可望海,沿海多浮石,嵌空玲珑;潮水击之,声作钟磬。东北有山,曰雪崎。又东北,有小石山,曰龟山。稍下为护国寺,国王祷雨之所也。龛内有神,黑而裸,手剑立,状甚狞;名曰“不动”,或曰火神。庭中有景泰七年铸钟一,廊下又有乾隆五十七年新铸钟一。寺后多凤尾蕉,一名铁树。西有石,高五、六尺,黑而润,状如骈佛手。因书“仙人掌”三字于上。
    
    初二日癸丑,大暑,阴。从官往游泊村,归以新稻穗见示,云稻已尽收;乃知球阳地气温暖,稻常早熟,种以十一月,收以五、六月。薯则四时皆种,三熟为丰,四熟则为大丰。稻田少、薯田多。国人以薯为命,米则王官始得食。亦有麦、豆,所产不多。薯一名地瓜,闽人土语。午后,微雨。
    
    初五日丙辰,阴。巳后,大雨。长史\footnote{〔长史〕职官名,相当于幕僚长。}送佛桑\footnote{〔佛桑〕朱槿,也叫赤槿、佛桑、红扶桑、大红花,原产于中国南部,广泛分布于亚洲的观赏花卉。}四株。一种千层如榴,有深红、粉红二色。一种单层,花如灯盘,蕊单出如烛,长二寸许,有红、白二色;朝开暮落,落则瓣卷如烛。花而不实;四季有花,深冬叶始凋谢。此地花开四季者甚多,气暖故也。余感长史意,嘱从客酬以酒;意有花再相致耳。
    
    初七日戊午,晴。辰刻,微雨,旋止。长史复以花二盆见贻,标曰“水翁花”;视之,乃马兰花也。中山草木,多与中朝\footnote{〔中朝〕中央王朝,指中国内地。}异称;盖因国中少书,多不识古来草木之名。如罗汉松,谓之㭴木;冬青,谓之福木;万寿菊,谓之禅菊:其初以意名之,后遂相沿不改。惜未携《群芳谱》来,一一证辨之耳!
    
    十四日乙丑,阴。荔枝栽近一月,新叶茂发,有生机矣。早起,偶步其侧,见新叶有蚀者;薄视之,有虫黄体而苍文,两角、八足,身方而毛,世所谓毛虫类。附叶为巢,蒙如小蛛网;卵生如蚕子而速,大者二寸以来。命仆捉而坑之,尽扫其巢。
    
    二十九日庚辰,晴。是日初见五彩鱼\footnote{〔五彩鱼〕即花斑连鳍䲗,俗名七彩麒麟、五彩青蛙。}。有红绿翠黄诸色,绿鳞红章,五彩相间。土人就形色呼之,无定名。又有一石眉巴鱼\footnote{〔石眉巴鱼〕可能是红鳍笛鲷。},色红如金鱼。余俱不敢食,养盎中以为玩品。又有鳐如白鸟,云飞丈馀始入水,疑即燕鱼\footnote{〔燕鱼〕渤海地区对蓝点马鲛的称呼。}也。
    
    初六日丙戌,大风。是日,食品有蕉实,状如手指,不相属。色黄,味甘,瓤如柚,亦名甘露。闻初熟色青,以糠覆之则黄,与中国制柿无异。其花红,一穗数尺,瓣须五六出。岁实为常,实如其须之数。中国亦有蕉,不闻岁结实,亦无有抽其丝作布者;或其性殊欤?
    
    行成归来
    
    二十日己巳,晴。东北风利,促解缆。卯刻,扬帆出那霸港;岸上、舟中送者如云,举手辞谢之。午刻,雨,入暮不止。伙长恐有暴,收马齿山安护浦下碇。山势横袤二十里,犬牙相错,出没海中,若断若续;分东、西二岛,为中山\footnote{〔中山〕指琉球国。琉球国全名为琉球中山王国。}第一外障。泊处青山围绕,无出路。有鹿见于山间,疑亦海鱼所化。雨景大佳。
    
    二十二日辛未,雨,风仍西北。午刻,晴。偕介山驾小舟登岸;沿沙洲行至山麓,有石高丈馀,玲珑可爱。坐石上观渔,皆赤身入水,无寒色。马齿人善泅,习使然也。
    
    二十四日癸酉,晴。北风少平,促伙长出洋;对以“风信未定”。余曰:“风信定,能无变乎?可行,则行!”介山曰:“姑俟之!”遂止。
    
    二十五日甲戌,晴。北风如故,决令开帆,介山亦以为然;遂于巳刻解缆。子丑风,用辛针。酉刻,过姑米山。终日峭帆,舟转驶,微侧而震,有吐者;余仍日坐将台,饮食如故。
    
    二十九日戊寅,辰卯风微,大雾,针如故。巳刻,稍霁;见温州南杞山\footnote{〔南杞山〕现称南麂岛,在浙江温州市平阳县东南海面。},舟人大喜。少顷,见杞山北有船数十只泊焉;舟人皆喜曰:“此必迎护船也!”雾渐消,山渐近;守备\footnote{〔守备〕清朝武官名,正五品,管理军队总务、军饷、军粮。}登后艄以望,惊报曰:“泊者,贼船也!”余曰:“舟巳至此,戒兵无哗!速食,备器械!”余亦饱食。守备又报贼船皆扬帆矣;与介山衣冠出,令吐者、病者悉归舱;登战台,誓众曰:“贼众我寡,尔等未免胆怯。然贼船小、我船大,彼络绎开帆,纵善驾驶,不能并集,犹一与一之势也。且既已遇之,惧亦无益!惟有以死相拼,可望死中求活。此我与汝致命之秋也,生死共之!”众兵勇气顿振,皆曰“惟命!”乃下令曰:“贼船未及三百步,不得放子母炮;未及八十步,不得放枪;未及四十步,不得放箭。如果近,始用长枪相拼。有能毙贼者,重赏;违者,按以军法”。各整暇以俟。
    
    未几,贼船十六只吆喝而来,第一只已入三百步。余举旗麾之,吴得进从舵门放子母炮,立毙四人,击喝者堕海;贼退不及,入百步,枪并发,又毙六人。一只乃退,二只又入三百步,复以炮击之,毙五人;稍进,又击之,复毙四人,乃退去。其时,三只贼船已占上风;暗移子母炮至舵右舷边,连毙贼十二人,焚其头蓬:皆转舵而退。中二船较大,复鼓噪由上风飞至。余曰:“此必贼首也!”密令舵工将船稍横,俟大炮准对贼船,即施放一发,中之。炮响后,烟迷里许;既散,则贼船巳尽退。是役也,王得禄首先士卒,兵丁吴得进、陈成德、林安顺、张大良、王名标、甘耀等枪炮俱无虚发,幸免于危。惟时日将暮,风甚微;恐贼乘夜来袭,默祷于天后\footnote{〔天后〕即妈祖,俗称“海神娘娘”,东南沿海民间崇拜的神灵。}求风。不一时,北风大至,浪飞过船。余倦极,思卧。念前险假遇害,岂复能虑此险!况求风得风,即忧亦无着力处。遂解衣熟睡,付之不见不闻。
    
    十一月朔日己卯,阴。梦中闻舟人哗曰:“到官塘\footnote{〔官塘〕官方的港口。塘:堤岸。}矣!”惊起。介山、从客皆一夜不眠,语余曰:“险至此,服汝能睡。设葬鱼腹,亦为糊涂鬼矣!”余曰:“险奈何”?介山曰:“上则九天\footnote{〔九天〕传说古代天地有九重,“九天”天的最高处,比喻极高处。也作“九重天”、“九霄”。后面“九地”指地的最深处,比喻极低处。},下则九地,声如转水车、锯湿木,时复疟颤;每侧,则篷皆卧水。一浪盖船,则船身入水,惟闻瀑布声垂流不息;其不覆者,幸耳!”余曰:“脱覆,君等能免之乎!余乐拾得一觉,又忘其险,余幸矣!”介山乃大笑。舟人指曰:“前即定海\footnote{〔定海〕指现在浙江省舟山市定海区。清代重要的港口和军事要塞。},可无虑!”申刻,乃得泊。总兵\footnote{〔总兵〕明清武官名。清代总兵统领一地汉军,正二品。}何定江来迎护,余笑谢之。因语以北杞之战,定江惶悚失措;余曰:“馁矣!他事且缓商”。
\end{normalsize}


\newpage

\textbf{注解}:

\vspace{-1em}

\begin{itemize}
    \setlength\itemsep{-0.2em}
    \item〔初九日庚寅〕传统记日的方法。“初九”是阴历记日,表示当月第九天;“庚寅”是干支记日,用天干地支顺序计数,每六十天一循环。两种记日方法并用,交叉比对,准确不出差错。
    \item〔卯刻〕传统记时方法。古人把一天分为十二个时辰,用十二地支记录。每个时辰又分为八刻。初刻、正刻,各一小时。“卯刻”也就是“卯时”,指早上五到七点,“申正”指“申时正刻”,即下午四点正。
    \item〔转丁未风,用单乙针〕传统罗盘方位记法,分八方二十四针,一方三针。又有单针、双针之分。单针指最近一针,双针指介于相邻双针之间。“丁”是南偏西南(南偏西15度),“未”是西南偏南(南偏西30度),“丁未”说明方向介于“丁”、“未”之间。“单乙针”表示以“乙”针(东偏南15度)为准航行。
    \item〔计自开洋,行船十六更矣〕“更”是明清时代发展出来的计量海路里程的方法。最初是把一夜分为五更,每更大约2.4小时。后来把一更航行的里程作为单位,每更大约50里(25公里)。李鼎元此次出海由于出发仓促,没有准备沙漏,靠焚香记时,因此有较大误差。实际路程远比十六更少,下同。
    \item〔鳐如白鸟,云飞丈馀始入水〕文鰩出海南,大者長尺許,有翅,與尾齊,一名飛魚。羣飛水上,海人候之,當有大風。
    \item〔其花红,一穗数尺,瓣须五六出。岁实为常,实如其须之数。〕香蕉的花与一般的花不太一样,并没有典型的花瓣,而是像笋壳一样的苞片。雌花的子房初时细长像须一样,逐渐发育为香蕉。一般来说,香蕉树结果之后第二年会枯萎死掉,然后重新从根部长出新的香蕉树,因此不会每年都结果。
\end{itemize}

\chapter{《世说新语》两则}

\begin{normalsize}
    
    咏雪
    
    谢太傅寒雪日内集,与儿女讲论文义。俄而雪骤,公欣然曰:“白雪纷纷何所似?”兄子胡儿曰:“撒盐空中差可拟。”兄女曰:“未若柳絮因风起。”公大笑乐。即公大兄无奕女,左将军王凝之妻也。
    
    陈太丘与友期
    
    陈太丘与友期行,期日中,过中不至,太丘舍去,去后乃至。元方时年七岁,门外戏。客问元方:“尊君在不?”答曰:“待君久不至,已去。”友人便怒:“非人哉!与人期行,相委而去。”元方曰:“君与家君期日中,日中不至,则是无信;对子骂父,则是无礼。”友人惭,下车引之,元方入门不顾。
\end{normalsize}



\chapter{天文地理}

\begin{normalsize}
    
    混沌初开,乾坤始奠。
    
    气之清轻者上浮为天,气之浊重者下凝为地。
    
    日月五星,谓之七曜;天地与人,谓之三才。
    
    北斗为帝车之象,北辰是太一所居。
    
    叁星为白虎之体,大火是苍龙之心。
    
    二十八宿,是日月之旅舍。黄道白道,载日月之行迹。
    
    叁商二星,其出没不相见。牛女两宿,惟七夕一相逢。
    
    后羿射日,女娲补天。羲和置闰,授民以时。
    
    黄帝画野,始分都邑。夏禹治水,初奠山川。
    
    宇宙之江山不改,古今之称谓各殊。
    
    北京原属幽燕,金台是其异号;南京原为建业,金陵为其别名。
    
    浙江是武林之区,原为越国;江西是豫章之地,昔从九江。
    
    福建省属闽中,湖广地名三楚。
    
    东鲁西鲁,即山东山西之分;东粤西粤,乃广东广西之域。
    
    河南在华夏之中,故曰中州;陕西即长安之地,原为秦境。
    
    四川为西蜀,云南为古滇。
    
    贵州省近蛮方,自古名为黔地。
    
    东岳泰山,西岳华山,南岳衡山,北岳恒山,中岳嵩山,此为天下之五岳。
    
    饶州之鄱阳,岳州之青草,润州之丹阳,鄂州之洞庭,苏州之太湖,此为天下之五湖。
    
    沧海桑田,谓世事之多变。河清海晏,兆天下之升平。
    
    道不拾遗,由在上有善政。途通天堑,知中国有圣人。
\end{normalsize}


\newpage

\textbf{注解}:

\vspace{-1em}

\begin{itemize}
    \setlength\itemsep{-0.2em}
    \item〔叁星为白虎之体,大火是苍龙之心〕《史记·天官书》:“叁为白虎。”《晋书·天文志》:“叁,白兽之体。”《史记·天官书》:“ 叁为白虎。叁星直者,是为衡石。下有叁星,兑,曰罚,为斩艾事。”大火指心宿二,也叫商星。心宿是东方苍龙七宿之一,位于中心。
    \item〔叁商二星,其出没不相见〕《左传·昭公》记载:“昔高辛氏有二子,伯曰阏伯,季曰实沈,居于旷林,不相能也。日寻干戈,以相征讨。后帝不臧,迁阏伯于商丘,主辰。商人是因,故辰为商星。迁实沈于大夏,主参。唐人是因,以服事夏商。”
    \item〔二十八宿,是日月之旅舍〕《论衡·谈天》:“二十八宿为日月舍,犹地有邮亭,为长吏廨矣。邮亭著地,亦如星舍著天也。”
    \item〔日月五星,谓之七曜〕日月和水星、金星、火星、木星、土星合称“七曜”。《春秋谷梁传注疏》:“七曜者,日月五星皆照天下,故谓之‘七曜’。五星者,即东方岁星,南方荧惑,西方太白,北方辰星,中央镇星是也。”
    \item〔帝车之象〕《史记·天官书》:“斗为帝车,运于中央,临制四乡。”。
    \item〔北辰是太一所居〕北辰就是北极星。《尔雅·释天》:“北极谓之北辰。” 太一是天帝的名称。《易纬·乾凿度》所说:“太一者,北辰之神名也,居其所曰太帝。”
    \item〔羲和置闰,授民以时〕《尚书·尧典》:“乃命羲和,钦若昊天,历象日月星辰,敬授人时。……帝曰:‘咨!汝羲暨和。期三百有六旬有六日,以闰月定四时成岁。’”
    \item〔黄帝画野,始分都邑。夏禹治水,初奠山川〕《汉书·地理志》:“昔在黄帝,作舟车以济不通,旁行天下,方制万里,画野分州,得百里之国万区。”《尚书·吕刑》:“禹平水土,主名山川。”
    \item〔福建省属闽中〕秦代设闽中郡,包括福建,浙江宁海县及灵江、瓯江、飞云江流域。后来福建也称闽中。
    \item〔湖广地名三楚〕元代设置湖广等处行中书省,包括湖南湖北,明代湖广行省,按春秋时楚国名号称为“楚”。《淮南子》:“楚人地南卷沅湘,北绕颍泗,西包巴蜀,东裹郯邳,颍汝以为洫,江汉以为池,垣之以邓林。”秦汉时有西楚、东楚、南楚的分野,《史记·货殖列传》:“夫自淮北沛、陈、汝南、南郡,此西楚也。其俗剽轻,易发怒,地薄,寡于积聚。彭城以东,东海、吴、广陵,此东楚也。衡山、九江、江南、豫章、长沙,是南楚也,其俗大类西楚。”即今天湖北省长江以北、河南中东部以及江苏安徽北部一带总称为西楚,江苏中南部与浙江北部等长三角一带为东楚,长江以南的江西与湖南等地为南楚。总称三楚。“三楚”的分野在汉代初年可谓社会共识,一直成为区别各地民风、物产等社会环境的代称,足以证明楚之影响巨大。东汉末年,黄巾之乱带来的人口迁徙让长江中下游的吴越地区逐渐富庶起来。孙氏家族在江东地区,即今天的江苏南部、浙江和江西一带,建立了东吴政权,开启了六朝时期,逐渐以三吴都会自居,最早脱离楚地的范畴。至隋唐以后,楚地逐渐收缩为湖南湖北为核心的荆湖地区。
    \item〔黄帝画野,始分都邑。夏禹治水,初奠山川〕《汉书·地理志》:“昔在黄帝,作舟车以济不通,旁行天下,方制万里,画野分州,得百里之国万区。”《尚书·吕刑》:“禹平水土,主名山川。”
    \item〔黄帝画野,始分都邑。夏禹治水,初奠山川〕《汉书·地理志》:“昔在黄帝,作舟车以济不通,旁行天下,方制万里,画野分州,得百里之国万区。”《尚书·吕刑》:“禹平水土,主名山川。”
    \item〔黄帝画野,始分都邑。夏禹治水,初奠山川〕《汉书·地理志》:“昔在黄帝,作舟车以济不通,旁行天下,方制万里,画野分州,得百里之国万区。”《尚书·吕刑》:“禹平水土,主名山川。”
    \item〔贵州省近蛮方〕蛮:指南方未开化的部落。方:区域,上古称部落邦国,如土方、鬼方等。
    \item〔饶州之鄱阳〕隋朝设饶州,后为鄱阳郡,今有江西上饶市。鄱阳湖古称彭蠡泽、彭泽,是江西北部的大湖。
    \item〔岳州之青草〕隋朝设岳州,历史上也叫巴陵、巴州,今有湖南岳阳市。青草湖是湖南古代的大湖,位于岳阳市西南,洞庭湖的南部,并与之相连。南北朝时期已连为一体。清末之后逐渐淤积枯萎,现已不存。
    \item〔润州之丹阳〕隋唐设润州,在今江苏西南部和安徽交界。丹阳湖是江南古代的大泽,因秦置丹阳县得名,也叫丹湖、南湖,位于润州西南,今南京南面,高淳、溧水、当涂一带。因泥沙淤积和围垦,逐渐消亡,如今剩余部分有石臼湖。
    \item〔鄂州之洞庭〕隋朝设鄂州,包括今湖北地区,历史上也叫江夏、武昌,治所在今湖北武汉市。这里的洞庭指唐宋时湖南北部、长江以南的大湖。晋唐以来,由于地势升降原因形成的青草湖(洞庭湖的前身)水域面积不断扩大,成为除长江之外楚地南北分界的自然地理标识。至北宋时期,随着水域扩展,使原来在汉晋时期彼此支离的洞庭、青草、赤沙3个湖泊在高水位时得以连成汪洋一片。也正是在北宋时期,以洞庭湖为界线的荆湖北路与荆湖南路的行政区划分离也随之出现。不过,由于宋代的路一级区划并非实际具有独立管辖权的高级别行政单位,所以洞庭南北的荆湖楚地在本质上联结依旧紧密。曾经作为荆襄首府的江陵城(湖北荆州)在隋唐以后逐渐没落,宋以后鄂州(明代武昌,今湖北武汉市)逐渐成为长江中游的首府。元代设湖广行省,将鄂州行省并入湖广行省,最后迁治所于鄂州。当时湖广行省的管辖范围为今湖北、湖南、广西和广东、贵州的一部分,是当时最大的行省。明朝设湖广布政使司,简称“楚”。《大明清类天文分野之书》:“自房、陵、白帝而东,尽汉之南郡、江夏,东达庐江南部,滨彭蠡之西,得长沙、武陵,又逾南纪,尽郁林、合浦之地,自沅湘上流,西达黔安之左,皆全楚之分。”
    \item〔苏州之太湖〕隋朝设苏州,包括今江苏、浙江地区,历史上也叫吴州,治所在今江苏苏州市。太湖是江苏南部大湖,古称震泽。
\end{itemize}

\chapter{童趣}

\begin{normalsize}
    
    余忆童稚时,能张目对日,明察秋毫,见藐小之物必细察其纹理,故时有物外之趣。
    
    夏蚊成雷,私拟作群鹤舞于空中,心之所向,则或千或百,果然鹤也;昂首观之,项为之强。又留蚊于素帐中,徐喷以烟,使之冲烟而飞鸣,作青云白鹤观,果如鹤唳云端,为之怡然称快。
    
    余常于土墙凹凸处,花台小草丛杂处,蹲其身,使与台齐;定神细视,以丛草为林,以虫蚁为兽,以土砾凸者为丘,凹者为壑,神游其中,怡然自得。
    
    一日,见二虫斗草间,观之,兴正浓,忽有庞然大物,拔山倒树而来,盖一癞虾蟆,舌一吐而二虫尽为所吞。余年幼,方出神,不觉呀然一惊。神定,捉虾蟆,鞭数十,驱之别院。
\end{normalsize}



\chapter{智子疑邻}

\begin{normalsize}
    
    宋有富人。天雨墙坏,其子曰:“不筑,必将有盗。”其邻人之父亦云。暮而果大亡其财。其家甚智其子,而疑邻人之父。
\end{normalsize}



\chapter{口技}

\begin{normalsize}
    
    京中有善口技者。会宾客大宴,于厅事之东北角,施八尺屏障,口技人坐屏障中,一桌、一椅、一扇、一抚尺而已。众宾团坐。少顷,但闻屏障中抚尺\footnote{〔抚尺〕曲艺演员表演时用以拍桌子以引起听众注意的木块。}一下,满坐寂然,无敢哗者。
    
    遥遥闻深巷中犬吠,便有妇人惊觉欠伸,其夫呓语。既而儿醒,大啼。夫亦醒。妇抚儿乳,儿含乳啼,妇拍而呜之。又一大儿醒,絮絮不止。当是时,妇手拍儿声,口中呜声,儿含乳啼声,大儿初醒声,夫叱大儿声,一齐凑发,众妙毕备。满坐宾客无不伸颈侧目,微笑默叹,以为妙绝。
    
    未几,夫齁声起,妇拍儿亦渐拍渐止。微闻有鼠作作索索,盆器倾侧,妇梦中咳嗽。宾客意少舒,稍稍正坐。
    
    忽一人大呼:“火起”,夫起大呼,妇亦起大呼。两儿齐哭。俄而百千人大呼,百千儿哭,百千犬吠。中间力拉崩倒之声,火爆声,呼呼风声,百千齐作;又夹百千求救声,曳屋许许声,抢夺声,泼水声。凡所应有,无所不有。虽人有百手,手有百指,不能指其一端;人有百口,口有百舌,不能名其一处也。于是宾客无不变色离席,奋袖出臂,两股战战,几欲先走。
    
    而忽然抚尺一下,群响毕绝。撤屏视之,一人、一桌、一椅、一扇、一抚尺而已。
\end{normalsize}



\chapter{伤仲永}

\begin{normalsize}
    
    金溪\footnote{〔金溪〕现在江西抚州市金溪县。}民方仲永,世隶耕。仲永生五年,未尝识书具,忽啼求之。父异焉,借旁近与之,即书诗四句,并自为其名。其诗以养父母、收族为意,传一乡秀才观之。自是指物作诗立就,其文理皆有可观者。邑人奇之,稍稍宾客其父。或以钱币乞之。父利其然也,日扳仲永环谒于邑人,不使学。
    
    余闻之也久。明道\footnote{〔明道〕宋仁宗(赵祯)年号(公元1032至1033年)。}中,从先人还家,于舅家见之,十二三矣。令作诗,不能称前时之闻。又七年,还自扬州,复到舅家问焉。曰:“泯然众人矣。”
    
    王子曰:仲永之通悟,受之天也。其受之天也,贤于材人远矣。卒之为众人,则其受于人者不至也。彼其受之天也,如此其贤也,不受之人,且为众人;今夫不受之天,固众人,又不受之人,得为众人而已耶?
\end{normalsize}



\chapter{孙权劝学}

\begin{normalsize}
    
    初,权谓吕蒙曰:“卿今当涂掌事,不可不学!”蒙辞以军中多务。权曰:“孤岂欲卿治经为博士邪?但当涉猎,见往事耳。卿言多务,孰若孤?孤常读书,自以为大有所益。”蒙乃始就学。及鲁肃过寻阳,与蒙论议,大惊曰:“卿今者才略,非复吴下阿蒙!”蒙曰:“士别三日,即更刮目相待,大兄何见事之晚乎!”肃遂拜蒙母,结友而别。
\end{normalsize}



\chapter{爱莲说}

\begin{normalsize}
    
    水陆草木之花,可爱者甚蕃。晋陶渊明\footnote{〔陶渊明〕名潜,字元亮,东晋末杰出的诗人、辞赋家、散文家。}独爱菊。自李唐\footnote{〔李唐〕指唐代。唐朝皇帝姓李,故称李唐。}来,世人盛爱牡丹。予独爱莲之出淤泥而不然,濯清涟而不妖,中通外直,不蔓不枝,香远益清,亭亭净植,可远观而不可亵玩焉。
    
    予谓菊,花之隐逸者也;牡丹,花之富贵者也;莲,花之君子者也。噫!菊之爱,陶后鲜有闻。莲之爱,同予者何人?牡丹之爱,宜乎众矣。
\end{normalsize}



\chapter{记承天寺夜游}

\begin{normalsize}
    
    元丰六年\footnote{〔元丰〕宋神宗(赵顼)年号(公元1078年至1085年)。元丰六年是公元1083年,是苏轼因乌台诗案被贬黄州第四年。}十月十二日夜,解衣欲睡,月色入户,欣然起行。念无与为乐者,遂至承天寺\footnote{〔承天寺〕在今湖北黄冈县城南。}寻张怀民\footnote{〔张怀民〕苏轼的朋友。}。怀民亦未寝,相与步于中庭。庭下如积水空明,水中藻荇交横,盖竹柏影也。何夜无月?何处无竹柏?但少闲人如吾两人者耳。
\end{normalsize}



\chapter{大道之行也}

\begin{normalsize}
    
    大道之行也,天下为公。选贤与能,讲信修睦。故人不独亲其亲,不独子其子,使老有所终,壮有所用,幼有所长,鳏、寡、孤、独、废疾者皆有所养。男有分,女有归。货恶其弃于地也,不必藏于己;力恶其不出于身也,不必为己。是故谋闭而不兴,盗窃乱贼而不作,故外户而不闭。是谓大同。
\end{normalsize}


\newpage

\textbf{注解}:

\vspace{-1em}

\begin{itemize}
    \setlength\itemsep{-0.2em}
    \item〔货恶其弃于地也,不必藏于己;力恶其不出于身也,不必为己〕《礼记正义》:“货,谓财货也。既天下共之,不独藏府库,但若人不收录,弃掷山林,则物坏世穷,无所资用,故各收宝而藏之。是恶弃地耳,非是藏之为巳,有乏者便与也。”行大道的人认为把财货丢弃在地上是不好的,所以收在自己家里,但不会去怀疑这是为了藏起来给自己用的借口。财货即便放在自己家里,也是共有的。放在自己家里是为了保存财货,当人有需要时再施与,而不是为了独吞。《礼记正义》:“力,谓为事用力。言凡所事,不惮劬劳,而各竭筋力者,正是恶于相欺,惜力不出于身耳。非是欲自营赡。”行大道的人认为,做事情的时候要尽力而为,不要惜身不出力,但不会去怀疑这是让别人出力而自己不出力、让别人供养自己的借口。所有人都要出力,不假装已经尽力而藏私欺瞒,事情才能办好。郑玄注:“劳事不惮,施无吝心,仁厚之教也。”出力的时候不会忌惮,施与的时候不会不舍得,这是仁厚带来的好处。
    \item〔男有分,女有归〕《礼记正义》:“分,职也。无才者耕,有能者仕,各当其职,无失分也。”有才能的人就做官,没才能的人就耕地,表示按才华能力分配合适的工作,称为“有分”。《礼记正义》:“女谓嫁为归。君上有道,不为失时,故有归也。”古代女性没有独立的能力,因此将合适的妇女嫁到合适的家庭,称为“有归”。
\end{itemize}

\chapter{观潮}

\begin{normalsize}
    
    浙江\footnote{〔浙江〕就是钱塘江。}之潮,天下之伟观也。自既望\footnote{〔既望〕农历十六日。}以至十八日为最盛。方其远出海门,仅如银线;既而渐近,则玉城雪岭际天而来,大声如雷霆,震撼激射,吞天沃日,势极雄豪。杨诚斋诗云“海涌银为郭,江横玉系腰”\footnote{〔杨诚斋〕杨万里,字廷秀,号诚斋,南宋文学家。“海涌银为郭,江横玉系腰”出自《浙江观潮》。}者是也。
    
    每岁京尹\footnote{〔京尹〕京都临安府,现在浙江杭州市。}出浙江亭教阅水军,艨艟数百,分列两岸;既而尽奔腾分合五阵\footnote{〔五阵〕指两、伍、专、参、偏五种阵法。}之势,并有乘骑弄旗标枪舞刀于水面者,如履平地。倏尔黄烟四起,人物略不相睹,水爆轰震,声如崩山。烟消波静,则一舸无迹,仅有敌船为火所焚,随波而逝。
    
    吴儿善泅者数百,皆披发文身,手持十幅大彩旗,争先鼓勇,溯迎而上,出没于鲸波万仞中,腾身百变,而旗尾略不沾湿,以此夸能。而豪民贵宦,争赏银彩。
    
    江干上下十余里间,珠翠罗绮溢目。车马塞途,饮食百物皆倍穹常时。而僦赁看幕,虽席地不容间也。
\end{normalsize}



\chapter{核舟记}

\begin{normalsize}
    
    明有奇巧人曰王叔远,能以径寸之木为宫室、器皿、人物,以至鸟兽、木石,罔不因势象形,各具情态。尝贻余核舟一,盖大苏泛赤壁云。
    
    舟首尾长约八分有奇,高可二黍许。中轩敞者为舱,箬篷覆之。旁开小窗,左右各四,共八扇。启窗而观,雕栏相望焉。闭之,则右刻“山高月小,水落石出”,左刻“清风徐来,水波不兴”,石青糁之。
    
    船头坐三人,中峨冠而多髯者为东坡,佛印居右,鲁直居左。苏、黄共阅一手卷。东坡右手执卷端,左手抚鲁直背。鲁直左手执卷末,右手指卷,如有所语。东坡现右足,鲁直现左足,各微侧,其两膝相比者,各隐卷底衣褶中。佛印绝类弥勒,袒胸露乳,矫首昂视,神情与苏黄不属。卧右膝,诎右臂支船,而竖其左膝,左臂挂念珠倚之,珠可历历数也。
    
    舟尾横卧一楫。楫左右舟子各一人。居右者椎髻仰面,左手倚一衡木,右手攀右趾,若啸呼状。居左者右手执蒲葵扇,左手抚炉,炉上有壶,其人视端容寂,若听茶声然。
    
    其船背稍夷,则题名其上,文曰“天启壬戌秋日,虞山王毅叔远甫刻”,细若蚊足,钩画了了,其色墨。又用篆章一,文曰“初平山人”,其色丹。
    
    通计一舟,为人五,为窗八,为箬篷,为楫,为炉,为壶,为手卷,为念珠各一;对联、题名并篆文,为字共三十有四。而计其长,曾不盈寸。盖简桃核修狭者为之。
    
    魏子详瞩既毕,诧曰:嘻,技亦灵怪矣哉!《庄》《列》所载,称惊犹鬼神者良多,然谁有游削于不寸之质,而须麋瞭然者?假有人焉,举我言以复于我,亦必疑其诳。乃今亲睹之。由斯以观,棘刺之端,未必不可为母猴也。嘻,技亦灵怪矣哉!
\end{normalsize}



\chapter{湖心亭看雪}

\begin{normalsize}
    
    崇祯五年\footnote{〔崇祯〕明思宗(朱由检)的年号(公元1628至1644年)。}十二月,余住西湖\footnote{〔西湖〕杭州市西的浅水湖,中国著名文化景观。}。大雪三日,湖中人鸟声俱绝。是日更定矣,余拏一小舟,拥毳衣炉火,独往湖心亭看雪。雾凇沆砀,天与云、与山、与水,上下一白。湖上影子,惟长堤一痕、湖心亭一点、与余舟一芥、舟中人两三粒而已。
    
    到亭上,有两人铺毡对坐,一童子烧酒,炉正沸。见余大喜,曰:“湖中焉得更有此人?”拉余同饮。余强饮三大白而别。问其姓氏,是金陵人\footnote{〔金陵〕现江苏南京市。},客此。及下船,舟子喃喃曰:“莫说相公痴,更有痴似相公者。”
\end{normalsize}



\chapter{陋室铭}

\begin{normalsize}
    
    山不在高,有仙则名。水不在深,有龙则灵。斯是陋室,惟吾德馨。台痕上阶绿,草色入帘青。谈笑有鸿儒,往来无白丁。可以调素琴,阅金经。无丝竹之乱耳,无案牍之劳形。南阳诸葛庐\footnote{〔南阳诸葛庐〕南阳:今河南省南阳市一带。诸葛亮在出山之前,曾在南阳卧龙岗中隐居躬耕。诸葛亮,字孔明,三国时蜀汉丞相,著名的政治家和军事家。},西蜀子云亭\footnote{〔西蜀子云亭〕扬雄:字子云,西汉时文学家,蜀郡成都人。}。孔子云:何陋之有\footnote{〔孔子云……〕出自《论语·子罕第九》:“君子居之,何陋之有?”}?
\end{normalsize}



\chapter{三峡}

\begin{normalsize}
    
    自三峡\footnote{〔三峡〕指长江上游重庆、湖北两省间的瞿塘峡、巫峡和西陵峡。}七百里中,两岸连山,略无阙处。重岩叠嶂,隐天蔽日。自非亭午夜分,不见曦月。
    
    至于夏水襄陵,沿溯阻绝。或王命急宣,有时朝发白帝\footnote{〔白帝〕白帝城,在重庆奉节市东。},暮到江陵\footnote{〔江陵〕今湖北省江陵县。},其间千二百里,虽乘奔御风,不似疾也。
    
    春冬之时,则素湍绿潭,回清倒影。绝巘多生怪柏,悬泉瀑布,飞漱其间,清荣峻茂,良多趣味。
    
    每至晴初霜旦,林寒涧肃,常有高猿长啸,属引凄异,空谷传响,哀转久绝。故渔者歌曰:“巴东三峡巫峡长,猿鸣三声泪沾裳。”
\end{normalsize}



\chapter{桃花源记}

\begin{normalsize}
    
    晋太元\footnote{〔太元〕东晋孝武帝的年号(公元376至396年)。}中,武陵\footnote{〔武陵〕武陵郡,现在湖南常德市一带。}人捕鱼为业。缘溪行,忘路之远近。忽逢桃花林,夹岸数百步,中无杂树,芳草鲜美,落英缤纷。渔人甚异之。复前行,欲穷其林。
    
    林尽水源,便得一山,山有小口,仿佛若有光。便舍船,从口入。初极狭,才通人。复行数十步,豁然开朗。土地平旷,屋舍俨然,有良田美池桑竹之属。阡陌交通,鸡犬相闻。其中往来种作,男女衣着,悉如外人。黄发垂髫,并怡然自乐。
    
    见渔人,乃大惊,问所从来。具答之。便要还家,设酒杀鸡作食。村中闻有此人,咸来问讯。自云先世避秦时乱,率妻子邑人来此绝境,不复出焉,遂与外人间隔。问今是何世,乃不知有汉,无论魏晋。此人一一为具言所闻,皆叹惋。余人各复延至其家,皆出酒食。停数日,辞去。此中人语云:“不足为外人道也。”
    
    既出,得其船,便扶向路,处处志之。及郡下,诣太守\footnote{〔太守〕一郡之长。},说如此。太守即遣人随其往,寻向所志,遂迷,不复得路。
    
    南阳\footnote{〔南阳〕今河南省南阳市一带。}刘子骥,高尚士也,闻之,欣然规往,未果,寻病终。后遂无问津者。
\end{normalsize}



\chapter{满井游记}

\begin{normalsize}
    
    燕地寒,花朝节\footnote{〔花朝节〕旧时以农历二月十二日为花朝节。说这一天是百花生日。}后,余寒犹厉。冻风时作,作则飞沙走砾。局促一室之内,欲出不得。每冒风驰行,未百步辄返。
    
    廿二日\footnote{〔廿二日〕农历二月二十二日。}天稍和,偕数友出东直\footnote{〔东直〕北京东直门,在旧城东北角。满井在东直门北三四里。},至满井。高柳夹堤,土膏微润,一望空阔,若脱笼之鹄。于时冰皮始解,波色乍明,鳞浪层层,清澈见底,晶晶然如镜之新开而冷光之乍出于匣也。山峦为晴雪所洗,娟然如拭,鲜妍明媚,如倩女之靧面而髻鬟之始掠也。柳条将舒未舒,柔梢披风,麦田浅鬣寸许。游人虽未盛,泉而茗者,罍而歌者,红装而蹇者,亦时时有。风力虽尚劲,然徒步则汗出浃背。凡曝沙之鸟,呷浪之鳞,悠然自得,毛羽鳞鬣之间皆有喜气。始知郊田之外未始无春,而城居者未之知也。
    
    夫不能以游堕事而潇然于山石草木之间者,惟此官也。而此地适与余近,余之游将自此始,恶能无纪?己亥\footnote{〔己亥〕明朝万历二十七年(公元1599年)。}之二月也。
\end{normalsize}



\chapter{马说}

\begin{normalsize}
    
    世有伯乐\footnote{〔伯乐〕孙阳。春秋时人,擅长相马,得到秦穆公信赖,被封为“伯乐将军”。},然后有千里马。千里马常有,而伯乐不常有。故虽有名马,祇辱于奴隶人之手,骈死于槽枥之间,不以千里称也。
    
    马之千里者,一食或尽粟一石\footnote{〔石〕容量单位,十斗为一石。}。食马者,不知其能千里而食也。是马也,虽有千里之能,食不饱,力不足,才美不外见,且欲与常马等不可得,安求其能千里也?
    
    策之不以其道,食之不能尽其材,鸣之而不能通其意,执策而临之,曰:“天下无马!”呜呼,其真无马邪?其真不知马也!
\end{normalsize}



\chapter{五柳先生传}

\begin{normalsize}
    
    先生不知何许人也,亦不详其姓字。宅边有五柳树,因以为号焉。闲静少言,不慕荣利。好读书,不求甚解;每有会意,便欣然忘食。性嗜酒,家贫不能常得。亲旧知其如此,或置酒而招之;造饮辄尽,期在必醉。既醉而退,曾不吝情去留。环堵萧然,不蔽风日;短褐穿结,箪瓢屡空,晏如也。常著文章自娱,颇示己志。忘怀得失,以此自终。
    
    赞曰:黔娄\footnote{〔黔娄〕战国时齐国的隐士。}之妻有言:“不戚戚于贫贱,不汲汲于富贵。”其言兹若人之俦乎?衔觞赋诗,以乐其志。无怀氏\footnote{〔无怀氏〕传说中的上古氏族。}之民欤,葛天氏\footnote{〔葛天氏〕传说中的上古氏族。}之民欤?
\end{normalsize}



\chapter{小石潭记}

\begin{normalsize}
    
    从小丘\footnote{〔小丘〕即钴鉧潭西小丘,见前一篇《钴鉧潭西小丘记》。}西行百二十步,隔篁竹,闻水声,如鸣佩环,心乐之。伐竹取道,下见小潭,水尤清冽。全石以为底,近岸,卷石底以出,为坻,为屿,为嵁,为岩。青树翠蔓,蒙络摇缀,参差披拂。
    
    潭中鱼可百许头,皆若空游无所依,日光下澈,影布石上。佁然不动,俶尔远逝,往来翕忽。似与游者相乐。
    
    潭西南而望,斗折蛇行,明灭可见。其岸势犬牙差互,不可知其源。
    
    坐潭上,四面竹树环合,寂寥无人,凄神寒骨,悄怆幽邃。以其境过清,不可久居,乃记之而去。
    
    同游者:吴武陵\footnote{〔吴武陵〕信州(今重庆奉节一带)人,唐宪宗元和初进士,因罪贬官永州,与作者友善。},龚古\footnote{〔龚古〕作者朋友。},余弟宗玄\footnote{〔宗玄〕作者的堂弟。}。隶而从者,崔氏二小生:曰恕己,曰奉壹。
\end{normalsize}



\chapter{岳阳楼记}

\begin{normalsize}
    
    庆历四年\footnote{〔庆历四年〕庆历,宋仁宗赵祯的年号(公元1041至1048年)。庆历四年是公元1044年。}春,滕子京\footnote{〔滕子京〕滕宗谅,字子京,范仲淹的朋友。任泾州知州抵御西夏有功,经范仲淹举荐,擢天章阁待制。范仲淹发起庆历新政后遭攻击,因“泾州公款案”被贬,庆历四年任岳州太守。}谪守巴陵郡\footnote{〔巴陵郡〕岳州,治所在今湖南省岳阳市。}。越明年,政通人和,百废具兴。乃重修岳阳楼,增其旧制,刻唐贤今人诗赋于其上。属予作文以记之。
    
    予观夫巴陵胜状,在洞庭一湖。衔远山,吞长江,浩浩汤汤,横无际涯;朝晖夕阴,气象万千。此则岳阳楼之大观也,前人之述备矣。然则北通巫峡\footnote{〔巫峡〕长江三峡之一。},南极潇湘\footnote{〔潇湘〕潇水是湘水的支流。湘水流入洞庭湖。},迁客骚人,多会于此。览物之情,得无异乎?
    
    若夫淫雨霏霏,连月不开,阴风怒号,浊浪排空;日星隐曜,山岳潜形;商旅不行,樯倾楫摧;薄暮冥冥,虎啸猿啼。登斯楼也,则有去国怀乡,忧谗畏讥,满目萧然,感极而悲者矣。
    
    至若春和景明,波澜不惊,上下天光,一碧万顷;沙鸥翔集,锦鳞游泳;岸芷汀兰,郁郁青青。而或长烟一空,皓月千里,浮光跃金,静影沉璧,渔歌互答,此乐何极!登斯楼也,则有心旷神怡,宠辱偕忘,把酒临风,其喜洋洋者矣。
    
    嗟夫!予尝求古仁人之心,或异二者之为。何哉?不以物喜,不以己悲;居庙堂之高则忧其民;处江湖之远则忧其君。是进亦忧,退亦忧。然则何时而乐耶?其必曰:“先天下之忧而忧,后天下之乐而乐”乎。噫!微斯人,吾谁与归?
    
    时六年九月十五日。
\end{normalsize}



\chapter{与朱元思书}

\begin{normalsize}
    
    风烟俱净,天山共色。从流飘荡,任意东西。自富阳至桐庐\footnote{〔富阳至桐庐〕富阳与桐庐都在杭州境内,富阳在富春江下游,桐庐在富阳的西南中游。如按上文“从流飘荡”。则应为“从桐庐至富阳”,可能为作者笔误。},一百许里,奇山异水,天下独绝。
    
    水皆缥碧,千丈见底。游鱼细石,直视无碍。急湍甚箭,猛浪若奔。
    
    夹岸高山,皆生寒树。负势竞上,互相轩邈;争高直指,千百成峰。泉水激石,泠泠作响;好鸟相鸣,嘤嘤成韵。蝉则千转不穷,猿则百叫无绝。鸢飞戾天\footnote{〔鸢飞戾天〕出自《诗经·大雅·旱麓》。老鹰高飞入天,这里比喻极力追求名利的人。}者,望峰息心;经纶世务者,窥谷忘反。横柯上蔽,在昼犹昏;疏条交映,有时见日。
\end{normalsize}



\chapter{醉翁亭记}

\begin{normalsize}
    
    环滁<footnote:N1>皆山也。其西南诸峰,林壑尤美,望之蔚然而深秀者,琅琊也<footnote:N2>。山行六七里,渐闻水声潺潺而泻出于两峰之间者,酿泉也。峰回路转,有亭翼然临于泉上者,醉翁亭也。作亭者谁?山之僧智仙也。名之者谁?太守<footnote:N3>自谓也。太守与客来饮于此,饮少辄醉,而年又最高,故自号曰醉翁也。醉翁之意不在酒,在乎山水之间也。山水之乐,得之心而寓之酒也。
    
    若夫日出而林霏开,云归而岩穴暝,晦明变化者,山间之朝暮也。野芳发而幽香,佳木秀而繁阴,风霜高洁,水落而石出者,山间之四时也。朝而往,暮而归,四时之景不同,而乐亦无穷也。
    
    至于负者歌于途,行者休于树,前者呼,后者应,伛偻提携,往来而不绝者,滁人游也。临溪而渔,溪深而鱼肥,酿泉为酒,泉香而酒洌,山肴野蔌,杂然而前陈者,太守宴也。宴酣之乐,非丝非竹,射者中,弈者胜,觥筹交错,起坐而喧哗者,众宾欢也。苍颜白发,颓然乎其间者,太守醉也。
    
    已而夕阳在山,人影散乱。太守归而宾客从也。树林阴翳,鸣声上下,游人去而禽鸟乐也。然而禽鸟知山林之乐,而不知人之乐;人知从太守游而乐,而不知太守之乐其乐也。醉能同其乐,醒能述以文者,太守也。太守谓谁?庐陵<footnote:N4>欧阳修也。
    
    N1:〔滁〕滁州,今安徽省滁州市琅琊区。
    
    N2:〔琅琊〕山名,在滁州市。
    
    N3:〔太守〕秦汉时一郡之长。隋朝存州废郡后不再称太守。一州之长唐代称为刺史,北宋称为知州,但一般仍习惯用太守来称呼。
    
    N4:〔庐陵〕庐陵郡,就是吉洲。现在江西省吉安市。
\end{normalsize}



\chapter{陈涉世家}

\begin{normalsize}
    
    陈胜者,阳城\footnote{〔阳城〕现在河南登封市东南。。}人也,字涉。吴广者,阳夏\footnote{〔阳夏〕现在河南太康县。}人也,字叔。陈涉少时,尝与人佣耕,辍耕之垄上,怅恨久之,曰:“苟富贵,无相忘。”佣者笑而应曰:“若为佣耕,何富贵也?”陈涉太息曰:“嗟乎!燕雀安知鸿鹄之志哉!”
    
    二世元年\footnote{〔二世元年〕公元前209年。秦始皇死后,幼子胡亥继位,称为二世。}七月,发闾左适戍渔阳\footnote{〔渔阳〕渔阳郡,秦朝至唐朝常设的郡,包括今天北京市、天津市、河北省部分地区。郡治是渔阳县,在今北京怀柔区。},九百人屯大泽乡。陈胜、吴广皆次当行,为屯长。会天大雨,道不通,度已失期。失期,法皆斩。陈胜、吴广乃谋曰:“今亡亦死,举大计亦死;等死,死国可乎?”陈胜曰:“天下苦秦久矣。吾闻二世少子也,不当立,当立者乃公子扶苏\footnote{〔公子扶苏〕秦始皇长子,秦始皇驾崩后,赵高和李斯等人矫诏谋杀扶苏,改立公子胡亥为帝,是为沙丘之变。}。扶苏以数谏故,上使外将兵。今或闻无罪,二世杀之。百姓多闻其贤,未知其死也。项燕\footnote{〔项燕〕战国末年楚国著名将领,项梁之父,项羽的祖父,曾大败秦将李信。}为楚将,数有功,爱士卒,楚人怜之。或以为死,或以为亡。今诚以吾众诈自称公子扶苏、项燕,为天下唱,宜多应者。”吴广以为然。乃行卜。卜者知其指意,曰:“足下事皆成,有功。然足下卜之鬼乎!”陈胜、吴广喜,念鬼,曰:“此教我先威众耳。”乃丹书帛曰“陈胜王”,置人所罾鱼腹中。卒买鱼烹食,得鱼腹中书,固以怪之矣。又间令吴广之次所旁丛祠中,夜篝火,狐鸣呼曰:“大楚兴,陈胜王。”卒皆夜惊恐。旦日,卒中往往语,皆指目陈胜。
    
    吴广素爱人,士卒多为用者。将尉醉,广故数言欲亡,忿恚尉,令辱之,以激怒其众。尉果笞广。尉剑挺,广起,夺而杀尉。陈胜佐之,并杀两尉。召令徒属曰:“公等遇雨,皆已失期,失期当斩。藉第令毋斩,而戍死者固十六七。且壮士不死即已,死即举大名耳,王侯将相宁有种乎!”徒属皆曰:“敬受命。”乃诈称公子扶苏、项燕,从民欲也。袒右,称大楚。为坛而盟,祭以尉首。陈胜自立为将军,吴广为都尉。攻大泽乡,收而攻蕲\footnote{〔蕲〕现在安徽宿州市南。}。蕲下,乃令符离\footnote{〔符离〕现在安徽宿州。}人葛婴将兵徇蕲以东。攻铚、酂、苦、柘、谯\footnote{〔铚、酂、苦、柘、谯〕地名,现在安徽、河南一带。}皆下之。行收兵。比至陈\footnote{〔陈〕秦时县名,今河南淮阳。},车六七百乘,骑千余,卒数万人。攻陈,陈守令皆不在,独守丞\footnote{〔守丞〕辅佐郡守的主官,也叫郡丞。}与战谯门中,弗胜,守丞死,乃入据陈。数日,号令召三老\footnote{〔三老〕每乡掌管教化民众、举荐人才的职位。由五十岁以上的人担任。}、豪杰与皆来会计事。三老、豪杰皆曰:“将军身被坚执锐,伐无道,诛暴秦,复立楚国之社稷,功宜为王。”陈涉乃立为王,号为张楚。
    
    当此时,诸郡县苦秦吏者,皆刑其长吏,杀之以应陈涉。乃以吴叔为假王,监诸将以西击荥阳\footnote{〔荥阳〕现在河南荥阳市。}。令陈人武臣、张耳、陈馀徇赵地,令汝阴\footnote{〔汝阴〕汝阴县,现在安徽省阜阳市。}人邓宗徇九江郡\footnote{〔九江郡〕郡名,包括江西全境、安徽的淮南及河南的一小部分,郡治在寿春(今安徽寿县)。}。当此时,楚兵数千人为聚者,不可胜数。
\end{normalsize}



\chapter{出师表}

\begin{normalsize}
    
    先帝\footnote{〔先帝〕指刘备。}创业未半而中道崩殂,今天下三分,益州\footnote{〔益州〕指蜀汉。}疲弊,此诚危急存亡之秋也。然侍卫之臣不懈于内,忠志之士忘身于外者,盖追先帝之殊遇,欲报之于陛下\footnote{〔陛下〕指刘禅。}也。诚宜开张圣听,以光先帝遗德,恢弘志士之气,不宜妄自菲薄,引喻失义,以塞忠谏之路也。
    
    宫中府中,俱为一体;陟罚臧否,不宜异同:若有作奸犯科及为忠善者,宜付有司论其刑赏,以昭陛下平明之理;不宜偏私,使内外异法也。
    
    侍中、侍郎郭攸之、费祎、董允等,此皆良实,志虑忠纯,是以先帝简拔以遗陛下:愚以为宫中之事,事无大小,悉以咨之,然后施行,必能裨补阙漏,有所广益。
    
    将军向宠,性行淑均,晓畅军事,试用于昔日,先帝称之曰“能”,是以众议举宠为督:愚以为营中之事,悉以咨之,必能使行阵和睦,优劣得所。
    
    亲贤臣,远小人,此先汉所以兴隆也;亲小人,远贤臣,此后汉所以倾颓也。先帝在时,每与臣论此事,未尝不叹息痛恨于桓、灵也。侍中、尚书、长史、参军,此悉贞良死节之臣,愿陛下亲之信之,则汉室之隆,可计日而待也。
    
    臣本布衣,躬耕于南阳\footnote{〔南阳〕现在河南省南阳市一带。},苟全性命于乱世,不求闻达于诸侯。先帝不以臣卑鄙,猥自枉屈,三顾臣于草庐之中,咨臣以当世之事,由是感激,遂许先帝以驱驰。后值倾覆,受任于败军之际,奉命于危难之间:尔来二十有一年矣。
    
    先帝知臣谨慎,故临崩寄臣以大事也。受命以来,夙夜忧叹,恐托付不效,以伤先帝之明;故五月渡泸,深入不毛。今南方已定,兵甲已足,当奖率三军,北定中原,庶竭驽钝,攘除奸凶,兴复汉室,还于旧都。此臣所以报先帝而忠陛下之职分也。至于斟酌损益,进尽忠言,则攸之、祎、允之任也。
    
    愿陛下托臣以讨贼兴复之效,不效,则治臣之罪,以告先帝之灵。若无兴德之言,则责攸之、祎、允等之慢,以彰其咎;陛下亦宜自谋,以咨诹善道,察纳雅言,深追先帝遗诏。臣不胜受恩感激。
    
    今当远离,临表涕零,不知所言。
\end{normalsize}



\chapter{隆中对}

\begin{normalsize}
    
    亮\footnote{〔亮〕指诸葛亮。}躬耕陇亩,好为《梁父吟》\footnote{〔《梁父吟》〕又叫《梁甫吟》,古歌曲名。传说诸葛亮曾经写过一首《梁父吟》歌词。}。身长八尺,每自比于管仲\footnote{〔管仲〕名夷吾,春秋时齐桓公的国相,帮助桓公建立霸业。}、乐毅\footnote{〔乐毅〕战国时燕昭王的名将,曾率领燕、赵、韩、魏、楚五国兵攻齐,连陷七十余城。},时人莫之许也。惟博陵崔州平\footnote{〔博陵崔州平〕博陵郡,属冀州,包括现在河北深州、安平、饶阳及安国等市县地。崔州平是博陵安平人,东汉太尉崔烈的儿子,西河郡太守崔钧的弟弟。}、颍川徐庶元直\footnote{〔颍川徐庶元直〕颍川郡,包括现在河南登封市、宝丰以东,尉氏、郾城以西,新密市以南,叶县、舞阳以北地。徐庶,字元直,豫州颍川长社人。在刘备帐下做谋士,向刘备举荐了诸葛亮。后来因母亲被曹军掳获,被迫归曹。}与亮友善,谓为信然。
    
    时先主\footnote{〔先主〕指刘备。}屯新野\footnote{〔新野〕现在河南新野县。}。徐庶见先主,先主器之,谓先主曰:"诸葛孔明者,卧龙也,将军岂愿见之乎?"先主曰:“君与俱来。”庶曰:“此人可就见,不可屈致也。将军宜枉驾顾之。”
    
    由是先主遂诣亮,凡三往,乃见。因屏人曰:“汉室倾颓,奸臣窃命,主上蒙尘。孤不度德量力,欲信大义于天下;而智术浅短,遂用猖蹶,至于今日。然志犹未已,君谓计将安出?"
    
    亮答曰:“自董卓\footnote{〔董卓〕东汉末年权臣,把持朝政、废立皇帝,后被王允设计杀害。}已来,豪杰并起,跨州连郡者不可胜数。曹操\footnote{〔曹操〕东汉末年枭雄,创立曹魏政权,统一北方。}比于袁绍\footnote{〔袁绍〕东汉末年军阀,讨伐董卓的盟主,后被曹操击败。},则名微而众寡。然操遂能克绍,以弱为强者,非惟天时,抑亦人谋也。今操已拥百万之众,挟天子而令诸侯,此诚不可与争锋。孙权\footnote{〔孙权〕汉破虏将军孙坚之子,建立东吴政权,割据江东。}据有江东,已历三世,国险而民附,贤能为之用,此可以为援而不可图也。荆州\footnote{〔荆州〕湖北和湖南的大部分地区,以及河南南部的一部分。}北据汉、沔\footnote{〔汉、沔〕指汉水中下游一带。汉水,古代通称沔水,流经今天的陕西南部、湖北西北部,最终在武汉汇入长江。},利尽南海\footnote{〔南海〕泛指南方两广等近海地区。},东连吴会\footnote{〔吴会〕吴郡和会稽郡的合称,现在江苏长江以南部分和浙江北部。},西通巴蜀\footnote{〔巴蜀〕巴郡、蜀郡,在现在的重庆和四川。},此用武之国,而其主不能守,此殆天所以资将军,将军岂有意乎?益州\footnote{〔益州〕蜀汉的中心区域,大致对应今天的四川省。}险塞,沃野千里,天府之土,高祖\footnote{〔高祖〕指刘邦,汉朝开国皇帝。}因之以成帝业。刘璋\footnote{〔刘璋〕汉室宗亲,益州牧。}暗弱,张鲁\footnote{〔张鲁〕东汉末年割据汉中、益州等地的军阀,五斗米道的第二代天师。}在北,民殷国富而不知存恤,智能之士思得明君。将军既帝室之胄,信义著于四海,总揽英雄,思贤如渴,若跨有荆、益,保其岩阻,西和诸戎,南抚夷越,外结好孙权,内修政理;天下有变,则命一上将将荆州之军以向宛、洛\footnote{〔宛、洛〕河南南阳和洛阳,这里泛指中原。},将军身率益州之众出于秦川\footnote{〔秦川〕指关中平原,位于今天的陕西省中部地区。},百姓孰敢不箪食壶浆,以迎将军者乎?诚如是,则霸业可成,汉室可兴矣。”
    
    先主曰:“善!”于是与亮情好日密。关羽\footnote{〔关羽〕东汉末年名将。与刘备、张飞结义。}、张飞\footnote{〔张飞〕东汉末年名将。与刘备、关羽结义。}等不悦,先主解之曰:“孤之有孔明,犹鱼之有水也。愿诸君勿复言。”羽、飞乃止。
\end{normalsize}



\chapter{唐雎不辱使命}

\begin{normalsize}
    
    秦王\footnote{〔秦王〕指嬴政。当时他还没称始皇帝。}使人谓安陵君\footnote{〔安陵君〕安陵国的国君。安陵是当时的一个小国,在现在河南鄢陵西北,原是魏国的附属国。战国时魏襄王封其弟为安陵君。}曰:“寡人欲以五百里之地易安陵,安陵君其许寡人!”安陵君曰:“大王加惠,以大易小,甚善;虽然,受地于先王,愿终守之,弗敢易!”秦王不说。安陵君因使唐雎\footnote{〔唐雎〕战国时期魏国、安陵国的谋士。}使于秦。
    
    秦王谓唐雎曰:“寡人欲以五百里之地易安陵,安陵君不听寡人,何也?且秦灭韩亡魏,而君以五十里之地存者,以君为长者,故不错意也。今吾以十倍之地,请广于君,而君逆寡人者,轻寡人与?”唐雎对曰:“否,非若是也。安陵君受地于先王而守之,虽千里不敢易也,岂直五百里哉?”
    
    秦王怫然怒,谓唐雎曰:“公亦尝闻天子之怒乎?”唐雎对曰:“臣未尝闻也。”秦王曰:“天子之怒,伏尸百万,流血千里。”唐雎曰:“大王尝闻布衣之怒乎?”秦王曰:“布衣之怒,亦免冠徒跣,以头抢地耳。”唐雎曰:“此庸夫之怒也,非士之怒也。夫专诸之刺王僚也\footnote{〔专诸之刺王僚〕专诸,春秋时刺客,受吴王阖闾之托刺杀吴王僚。},彗星袭月;聂政之刺韩傀也\footnote{〔聂政之刺韩傀〕聂政,战国时侠客,受严遂之托刺杀韩国相国侠累。},白虹贯日;要离之刺庆忌也\footnote{〔要离之刺庆忌〕要离,春秋时刺客,受吴王阖闾之托刺杀吴王僚之子庆忌。},仓鹰击于殿上。此三子者,皆布衣之士也,怀怒未发,休祲降于天,与臣而将四矣。若士必怒,伏尸二人,流血五步,天下缟素,今日是也。”挺剑而起。
    
    秦王色挠,长跪而谢之曰:“先生坐!何至于此!寡人谕矣:夫韩、魏灭亡,而安陵以五十里之地存者,徒以有先生也。”
\end{normalsize}



\chapter{曹刿论战}

\begin{normalsize}
    
    十年\footnote{〔十年〕鲁庄公十年(公元前684年)。}春,齐师伐我。公\footnote{〔公〕鲁庄公,鲁桓公之子,鲁桓公被齐襄公杀死后即位。曾经在齐襄公被杀后扶持齐公子纠争夺君位,但失败。鲁庄公八年,齐公子小白即位为齐桓公,之后讨伐鲁国。}将战,曹刿请见。其乡人曰:“肉食者\footnote{〔肉食者〕吃肉的人,指居高位,得厚禄的人。}谋之,又何间焉?”刿曰:“肉食者鄙,未能远谋。”乃入见。问:“何以战?”公曰:“衣食所安,弗敢专也,必以分人。”对曰:“小惠未遍,民弗从也。”公曰:“牺牲玉帛,弗敢加也,必以信。”对曰:“小信未孚,神弗福也。”公曰:“小大之狱,虽不能察,必以情。”对曰:“忠之属也,可以一战。战则请从。”
    
    公与之乘,战于长勺\footnote{〔长勺〕鲁国地名,现在山东曲阜县北。}。公将鼓之。刿曰:“未可。”齐人三鼓。刿曰:“可矣!”齐师败绩。公将驰之,刿曰:“未可。”下视其辙,登轼而望之,曰:“可矣。”遂逐齐师。
    
    既克,公问其故。对曰:“夫战,勇气也。一鼓作气,再而衰,三而竭。彼竭我盈,故克之。夫大国,难测也,惧有伏焉。吾视其辙乱,望其旗靡,故逐之。”
\end{normalsize}



\chapter{登泰山记}

\begin{normalsize}
    
    泰山\footnote{〔泰山〕在山东泰安北,古称岱宗,又称东岳,为五岳之长。}之阳\footnote{〔阳〕山南水北称为“阳”,山北水南称为“阴”。},汶水\footnote{〔汶〕今称大汶河,源于山东莱芜东北之原山,向西南流,汇入东平湖。}西流;其阴,济水\footnote{〔济〕源于河南济源县西之王屋山,流经山东。清代末年,济水河道为黄河所占。}东流。阳谷皆入汶,阴谷皆入济。当其南北分者,古长城\footnote{〔古长城〕指战国时齐国修筑的长城,西起平阴,经泰山北冈,东至诸城。}也。最高日观峰\footnote{〔日观峰〕泰山顶峰,观日出的胜地。},在长城南十五里。
    
    余以乾隆\footnote{〔乾隆〕清高宗(爱新觉罗·弘历)的年号(公元1736至1796年)。}三十九年十二月,自京师乘风雪,历齐河、长清\footnote{〔齐河、长清〕山东两县名,在泰安西北。},穿泰山西北谷,越长城之限,至于泰安。是月丁未\footnote{〔丁未〕丁未日,农历12月28日。},与知府朱孝纯\footnote{〔朱孝纯〕字子颖,号海愚,山东历城人,当时是泰安府的知府,姚鼐挚友。}子颍由南麓登。四十五里,道皆砌石为磴,其级七千有余。泰山正南面有三谷。中谷绕泰安城下,郦道元\footnote{〔郦道元〕字善长,北魏范阳(今河北涿县)人,著有《水经注》。}所谓环水也。余始循以入,道少半,越中岭,复循西谷,遂至其巅。古时登山,循东谷入,道有天门\footnote{〔天门〕泰山有南天门、东天门、西天门。}。东谷者,古谓之天门溪水,余所不至也。今所经中岭及山巅,崖限当道者,世皆谓之天门云。道中迷雾冰滑,磴几不可登。及既上,苍山负雪,明烛天南。望晚日照城郭,汶水、徂徕\footnote{〔徂徕〕徂徕山,在泰安东南四十里。}如画,而半山居雾若带然。
    
    戊申晦\footnote{〔戊申〕戊申日,农历12月29日。},五鼓,与子颍坐日观亭\footnote{〔日观亭〕日观峰上一个看日出的亭。},待日出。大风扬积雪击面。亭东自足下皆云漫。稍见云中白若樗蒱\footnote{〔樗蒲〕骰子。}数十立者,山也。极天云一线异色,须臾成五采。日上,正赤如丹,下有红光动摇承之。或曰,此东海也。回视日观以西峰,或得日或否,绛皓驳色,而皆若偻。
    
    亭西有岱祠\footnote{〔岱祠〕祭祀东岳大帝的庙宇,也叫岱庙。},又有碧霞元君祠\footnote{〔碧霞元君祠〕祭祀东岳大帝女儿碧霞元君的庙,也叫娘娘庙。}。皇帝行宫在碧霞元君祠东。是日观道中石刻,自唐显庆\footnote{〔显庆〕唐高宗(李治)的年号(公元656至661年)。}以来;其远古刻尽漫失。僻不当道者,皆不及往。
    
    山多石,少土。石苍黑色,多平方,少圜。少杂树,多松,生石罅,皆平顶。冰雪,无瀑水,无鸟兽音迹。至日观数里内无树,而雪与人膝齐。
    
    桐城姚鼐记。
\end{normalsize}



\chapter{公输}

\begin{normalsize}
    
    公输盘\footnote{〔公输盘〕也写作“公输班”、“公输般”,世称鲁班。}为楚造云梯之械,成,将以攻宋。子墨子\footnote{〔子墨子〕墨翟,春秋末战国初期的思想家、政治家。}闻之,起于鲁,行十日十夜而至于郢\footnote{〔郢〕楚国的都城,在今湖北省江陵县附近。},见公输盘。
    
    公输盘曰:“夫子何命焉为?”子墨子曰:“北方有侮臣者,愿借子杀之。”公输盘不说。子墨子曰:“请献十金。”公输盘曰:“吾义固不杀人。”
    
    子墨子起,再拜,曰:“请说之。吾从北方闻子为梯,将以攻宋。宋何罪之有?荆国\footnote{〔荆国〕楚国的别称。}有余于地,而不足于民,杀所不足而争所有余,不可谓智;宋无罪而攻之,不可谓仁;知而不争,不可谓忠。争而不得,不可谓强。义不杀少而杀众,不可谓知类。”公输盘服。子墨子曰:“然胡不已乎?”公输盘曰:“不可,吾既已言之王\footnote{〔王〕楚惠王,春秋末战国初期楚国国君。}矣。”子墨子曰:“胡不见我于王?”公输盘曰:“诺。”
    
    子墨子见王,曰:“今有人于此,舍其文轩,邻有敝舆而欲窃之;舍其锦绣,邻有短褐而欲窃之;舍其粱肉,邻有糠糟而欲窃之——此为何若人?”王曰:“必为有窃疾矣。”子墨子曰:“荆之地方五千里,宋之地方五百里,此犹文轩之与敝舆也。荆有云梦\footnote{〔云梦〕楚国的大湖云梦泽。},犀兕麋鹿满之,江汉之鱼鳖鼋鼍为天下富,宋所谓无雉兔鲋鱼者也,此犹粱肉之与糠糟也。荆有长松文梓楩楠豫章,宋无长木,此犹锦绣之与短褐也。臣以王吏之攻宋也,为与此同类。”王曰:“善哉!虽然,公输盘为我为云梯,必取宋。”
    
    于是见公输盘。子墨子解带为城,以牒为械,公输盘九设攻城之机变,子墨子九距之。公输盘之攻械尽,子墨子之守圉有余。公输盘诎,而曰:“吾知所以距子矣,吾不言。”子墨子亦曰:“吾知子之所以距我者,吾不言。”楚王问其故。子墨子曰:“公输子之意不过欲杀臣。杀臣,宋莫能守,乃可攻也。然臣之弟子禽滑釐等三百人,已持臣守圉之器,在宋城上而待楚寇矣。虽杀臣,不能绝也。”楚王曰:“善哉。吾请无攻宋矣。”
    
    子墨子归,过宋,天雨,庇其闾中,守闾者不内也。故曰:“治于神者,众人不知其功;争于明者,众人知之。”
\end{normalsize}


\newpage

\textbf{注解}:

\vspace{-1em}

\begin{itemize}
    \setlength\itemsep{-0.2em}
    \item〔臣以王吏之攻宋也,为与此同类〕旧本作:“臣以三事之攻宋也,为与此同类,臣见大王之必伤义而不得。”毕云:「《战国策》云『臣以王吏之攻宋。』『王吏』葢『三𠭏』之误,《说文》云『𠭏,古文事。』《尸子》作『王使』,《太平御览》作『王之攻宋』。」顾云:「《国策》『王吏』与此文『三事』,皆有误。疑当云『臣以王之事攻宋也』。」诒让案:「三事」,疑当作「三吏」。《逸周书》大匡篇云「王乃召冢卿三老三吏」,孔晁注云「三吏,三卿也。」《左传》成二年●,「晋侯使巩朔献齐捷于周,王使委于三吏」,杜注云「三吏,三公也。」《神仙传》作「臣闻大王,更议攻宋」,则似是「王吏」之譌。猜测可能是“臣以为王之攻宋也与此(三事)同类”之讹误。
    \item〔治于神者,众人不知其功〕《孟子》:“聖而不可知之謂神”。“天神,引出萬物者也”,“阴阳不测之谓神”,“神者,變化之極,妙萬物而爲言,不可以形詰”,“神而明之,存乎其人”。“神”原本指引起天地万物生灭变化的看不到的玄妙道理,藏于万物背后暗中主导者,是古人对不理解不明白的自然规律的设想和称呼,因此和“明”相对。这里引申为表示不为人知但最重要关键的部分。《尸子·贵言》:“圣人治于神,愚人争于明也。天地之道,莫见其所以长物而物长,莫见其所以亡物而物亡。圣人之道亦然。其兴福也,人莫之见而福兴矣。其除祸也,人莫之知而祸除矣,故曰神人”。治:整理,消灭灾祸、混乱,使安定、太平。暗中治理了危机的人,大众不知道他的功劳。
\end{itemize}

\chapter{孟子见梁惠王}

\begin{normalsize}
    
    孟子见梁惠王\footnote{〔梁惠王〕即魏惠王(前369年至前319年在位),名罃(或作“婴”),魏武侯之子。}。王曰:“叟不远千里而来,亦将有以利吾国乎?” 孟子对曰:“王何必曰利?亦有仁义而已矣。王曰:‘何以利吾国?’大夫曰:‘何以利吾家?’士庶人曰:‘何以利吾身?’上下交征利而国危矣。万乘之国,弑其君者,必千乘之家;千乘之国,弑其君者,必百乘之家。万取千焉,千取百焉,不为不多矣。苟为后义而先利,不夺不餍。未有仁而遗其亲者也,未有义而后其君者也。王亦曰仁义而已矣,何必曰利?”
    
    孟子见梁惠王。王立于沼上,顾鸿雁麋鹿,曰:“贤者亦乐此乎?” 孟子对曰:“贤者而后乐此,不贤者虽有此,不乐也。《诗》云\footnote{〔《诗》云……〕出自《诗经·大雅·灵台》。}:经始灵台,经之营之,庶民攻之,不日成之;经始勿亟,庶民子来;王在灵囿,麀鹿攸伏,麀鹿濯濯,白鸟鹤鹤;王在灵沼,於牣鱼跃。文王\footnote{〔文王〕周文王,名昌,又称西伯昌。商朝末期诸侯国周国君主。其子武王伐纣,建立周朝。}以民力为台为沼,而民欢乐之,谓其台曰灵台,谓其沼曰灵沼,乐其有麋鹿鱼鳖。古之人与民偕乐,故能乐也。《汤誓》曰\footnote{〔《汤誓》〕《尚书》中的一篇,为商汤伐桀誓师词。}:时日害丧,予及女偕亡。民欲与之偕亡,虽有台池鸟兽,岂能独乐哉?”
    
    梁惠王曰:“寡人之于国也,尽心焉耳矣。河内凶,则移其民于河东,移其粟于河内;河东凶亦然。察邻国之政,无如寡人之用心者。邻国之民不加少,寡人之民不加多,何也?”孟子对曰:“王好战,请以战喻。填然鼓之,兵刃既接,弃甲曳兵而走。或百步而后止,或五十步而后止。以五十步笑百步,则何如?”曰:“不可,直不百步耳,是亦走也。”曰:“王如知此,则无望民之多于邻国也。”“不违农时,谷不可胜食也;数罟不入洿池,鱼鳖不可胜食也;斧斤以时入山林,材木不可胜用也。谷与鱼鳖不可胜食,材木不可胜用,是使民养生丧死无憾也。养生丧死无憾,王道之始也。”“五亩之宅,树之以桑,五十者可以衣帛矣;鸡豚狗彘之畜,无失其时,七十者可以食肉矣;百亩之田,勿夺其时,数口之家可以无饥矣;谨庠序之教,申之以孝悌之义,颁白者不负戴于道路矣。七十者衣帛食肉,黎民不饥不寒,然而不王者,未之有也。”“狗彘食人食而不知检,涂有饿殍而不知发,人死,则曰:‘非我也,岁也’。是何异于刺人而杀之,曰:‘非我也,兵也’?王无罪岁,斯天下之民至焉。”
    
    梁惠王曰:“寡人愿安承教。”孟子对曰:“杀人以梃与刃,有以异乎?”曰:“无以异也。”“以刃与政,有以异乎?”曰:“无以异也。曰:“庖有肥肉,厩有肥马,民有饥色,野有饿莩,此率兽而食人也!兽相食,且人恶之;为民父母,行政,不免于率兽而食人,恶在其为民父母也?仲尼\footnote{〔仲尼〕孔子字仲尼。}曰:‘始作俑者,其无后乎!’为其象人而用之也。如之何其使斯民饥而死也?”
    
    梁惠王曰:“晋国,天下莫强焉,叟之所知也。及寡人之身,东败于齐\footnote{〔晋国〕指魏国;魏韩赵三国瓜分晋国,魏国最为强大,所以用“晋国”指代魏国。},长子死焉;西丧地于秦七百\footnote{〔东败于齐〕指公元前343年的马陵之役。魏国伐韩国,韩国求救于齐国。齐军袭魏,魏军败于马陵,主将庞涓自杀,魏太子申被俘。};南辱于楚\footnote{〔南辱于楚〕公元前324年,楚攻魏,破之于襄陵,得八邑。}。寡人耻之,愿比死者壹洒之,如之何则可?” 孟子对曰:“地方百里而可以王。王如施仁政于民,省刑罚,薄税敛,深耕易耨;壮者以暇日修其孝悌忠信,入以事其父兄,出以事其长上,可使制梃以挞秦楚之坚甲利兵矣。彼夺其民时,使不得耕耨以养其父母。父母冻饿,兄弟妻子离散。彼陷溺其民,王往而征之,夫谁与王敌?故曰:仁者无敌。王请勿疑!”
    
    孟子曰:“不仁哉,梁惠王也!仁者以其所爱及其所不爱,不仁者以其所不爱及其所爱。”公孙丑\footnote{〔公孙丑〕孟子弟子,齐国人,是《孟子》的主要作者之一。}问曰:“何谓也?”“梁惠王以土地之故,糜烂其民而战之,大败,将复之,恐不能胜,故驱其所爱子弟以殉之,是之谓以其所不爱及其所爱也。”
\end{normalsize}



\chapter{鱼我所欲也}

\begin{normalsize}
    
    鱼,我所欲也;熊掌,亦我所欲也。二者不可得兼,舍鱼而取熊掌者也。生,亦我所欲也;义,亦我所欲也。二者不可得兼,舍生而取义者也。生亦我所欲,所欲有甚于生者,故不为苟得也;死亦我所恶,所恶有甚于死者,故患有所不辟也。如使人之所欲莫甚于生,则凡可以得生者何不用也?使人之所恶莫甚于死者,则凡可以避患者何不为也?由是则生而有不用也,由是则可以避患而有不为也。是故所欲有甚于生者,所恶有甚于死者。非独贤者有是心也,人皆有之,贤者能勿丧耳。
    
    一箪食,一豆\footnote{〔豆〕古代一种木制的盛食物的器具。}羹,得之则生,弗得则死。呼尔而与之,行道之人弗受;蹴尔而与之,乞人不屑也。万钟\footnote{〔万钟〕指高官厚禄。钟,古代的一种量器,六斛四斗为一钟。}则不辩礼义而受之,万钟于我何加焉!为宫室之美,妻妾之奉,所识穷乏者得我与?乡为身死而不受,今为宫室之美为之;乡为身死而不受,今为妻妾之奉为之;乡为身死而不受,今为所识穷乏者得我而为之。是亦不可以已乎?此之谓失其本心。
\end{normalsize}


\newpage

\textbf{注解}:

\vspace{-1em}

\begin{itemize}
    \setlength\itemsep{-0.2em}
    \item〔呼尔而与之,行道之人弗受〕《礼记·檀弓》记载,有一年齐国出现了严重的饥荒。黔敖在路边施粥,有个饥饿的人用衣袖蒙着脸走来。黔敖吆喝着让他吃粥。他说:“我正因为不吃被轻蔑所给予得来的食物,才落得这个地步!”
\end{itemize}

\chapter{《庄子》故事两则}

\begin{normalsize}
    
    惠子相梁
    
    惠子\footnote{〔惠子〕指惠施,战国时宋国思想家,名家的代表人物。}相梁\footnote{〔梁〕指魏国。魏国曾将都城迁到大梁,因此也叫梁国。},庄子\footnote{〔庄子〕指庄周,战国时宋国思想家,道家的代表人物。}往见之。或谓惠子曰:“庄子来,欲代子相。”于是惠子恐,搜于国中三日三夜。庄子往见之,曰:“南方有鸟,其名为鹓雏,子知之乎?夫鹓雏发于南海,而飞于北海;非梧桐不止,非练实不食,非醴泉不饮。于是鸱得腐鼠,鹓雏过之,仰而视之曰:‘吓~’。今子欲以子之梁国而吓我邪?”
    
    庄子与惠子游于濠梁之上
    
    庄子与惠子游于濠梁\footnote{〔濠梁〕濠水上的桥梁,位于今安徽凤阳临淮镇,是濠河入淮处。}之上。庄子曰:“鯈鱼出游从容,是鱼之乐也。”惠子曰:“子非鱼,安知鱼之乐?”庄子曰:“子非我,安知我不知鱼之乐?”惠子曰:“我非子,固不知子矣;子固非鱼也,子之不知鱼之乐,全矣!”庄子曰:“请循其本。子曰‘汝安知鱼乐’云者,既已知吾知之而问我。我知之濠上也。”
\end{normalsize}


\newpage

\textbf{注解}:

\vspace{-1em}

\begin{itemize}
    \setlength\itemsep{-0.2em}
    \item〔子曰‘汝安知鱼乐’云者,既已知吾知之而问我〕这里是典型的语法诡辩。庄子首先说“鯈鱼出游从容,是鱼之乐也。”惠子说“子安知鱼乐”,是建立在庄子说“鱼之乐”的基础上。但庄子反过来利用语句的歧义,偷换了概念。惠子想说的是“你怎么会有‘自己知道鱼之乐’的想法?”,也就是想说“其实你不知道鱼之乐吧”。但“子安知鱼乐”也可以解释为“你是怎么知道‘鱼之乐’这件事的”,即询问“你知道这件事的方法”,这就等于先承认了“你知道鱼之乐”,只是询问“知道的方法”。这种利用语句的歧义偷换概念的诡辩,恰恰是名家思想关注的地方,即名实的关系。庄子在用惠子的诡辩方法回击惠子。
\end{itemize}

\chapter{邹忌讽齐王纳谏}

\begin{normalsize}
    
    邹忌\footnote{〔邹忌〕战国时齐人,有辩才。齐桓公时就任大臣,威王时为齐相。}修八尺\footnote{〔八尺〕战国时各国尺度不一,从出土文物推算,一尺约相当于今18到23厘米不一。}有余,而形貌昳丽。朝服衣冠,窥镜,谓其妻曰:“我孰与城北徐公美?”其妻曰:“君美甚,徐公何能及君也?”城北徐公,齐国之美丽者也。忌不自信,而复问其妾曰:“吾孰与徐公美?”妾曰:“徐公何能及君也?”旦日,客从外来,与坐谈,问之客曰:“吾与徐公孰美?”客曰:“徐公不若君之美也。”明日徐公来,孰视之,自以为不如;窥镜而自视,又弗如远甚。暮寝而思之,曰:“吾妻之美我者,私我也;妾之美我者,畏我也;客之美我者,欲有求于我也。”
    
    于是入朝见威王\footnote{〔威王〕齐威王,田齐桓公之子,任内贤明能治,使齐国成为战国七雄之一。},曰:“臣诚知不如徐公美。臣之妻私臣,臣之妾畏臣,臣之客欲有求于臣,皆以美于徐公。今齐地方千里,百二十城,宫妇左右莫不私王,朝廷之臣莫不畏王,四境之内莫不有求于王:由此观之,王之蔽甚矣。”
    
    王曰:“善。”乃下令:“群臣吏民,能面刺寡人之过者,受上赏;上书谏寡人者,受中赏;能谤讥于市朝,闻寡人之耳者,受下赏。”令初下,群臣进谏,门庭若市;数月之后,时时而间进;期年之后,虽欲言,无可进者。
    
    燕、赵、韩、魏闻之,皆朝于齐。此所谓战胜于朝廷。
\end{normalsize}



\chapter{前赤壁赋}

\begin{normalsize}
    
    壬戌\footnote{〔壬戌〕宋神宗元丰五年(公元1082年)。}之秋,七月既望,苏子与客泛舟游于赤壁之下。清风徐来,水波不兴。举酒属客,诵明月之诗,歌窈窕之章。少焉,月出于东山之上,徘徊于斗牛\footnote{〔斗牛〕牛宿和斗宿。}之间。白露横江,水光接天。纵一苇之所如,凌万顷之茫然。浩浩乎如冯虚御风,而不知其所止;飘飘乎如遗世独立,羽化而登仙。
    
    于是饮酒乐甚,扣舷而歌之。歌曰:“桂棹兮兰桨,击空明兮溯流光。渺渺兮予怀,望美人兮天一方。”客有吹洞箫者,倚歌而和之。其声呜呜然,如怨如慕,如泣如诉;余音袅袅,不绝如缕。舞幽壑之潜蛟,泣孤舟之嫠妇。
    
    苏子愀然,正襟危坐,而问客曰:“何为其然也?”客曰:“‘月明星稀,乌鹊南飞’,此非曹孟德之诗乎?西望夏口,东望武昌,山川相缪,郁乎苍苍,此非孟德\footnote{〔孟德〕指曹操,三国时枭雄,建立曹魏政权。}之困于周郎\footnote{〔周郎〕指周瑜,三国时东吴名将,在赤壁之战中大破曹魏水军。}者乎?方其破荆州,下江陵,顺流而东也,舳舻千里,旌旗蔽空,酾酒临江,横槊赋诗,固一世之雄也,而今安在哉?况吾与子渔樵于江渚之上,侣鱼虾而友麋鹿,驾一叶之扁舟,举匏尊以相属。寄蜉蝣\footnote{〔蜉蝣〕一种只能活一天的虫子。}于天地,渺沧海之一粟。哀吾生之须臾,羡长江之无穷。挟飞仙以遨游,抱明月而长终。知不可乎骤得,托遗响于悲风。”
    
    苏子曰:“客亦知夫水与月乎?逝者如斯,而未尝往也;盈虚者\footnote{〔盈虚者〕指月亮。}如彼,而卒莫消长也。盖将自其变者而观之,则天地曾不能以一瞬;自其不变者而观之,则物与我皆无尽也,而又何羡乎!且夫天地之间,物各有主,苟非吾之所有,虽一毫而莫取。惟江上之清风,与山间之明月,耳得之而为声,目遇之而成色,取之无禁,用之不竭。是造物者之无尽藏也,而吾与子之所共食。”
    
    客喜而笑,洗盏更酌。肴核既尽,杯盘狼籍。相与枕藉乎舟中,不知东方之既白。
\end{normalsize}



\chapter{归去来兮辞}

\begin{normalsize}
    
    归去来兮,田园将芜胡不归?既自以心为形役,奚惆怅而独悲?悟已往之不谏,知来者之可追。实迷途其未远,觉今是而昨非。舟遥遥以轻飏,风飘飘而吹衣。问征夫以前路,恨晨光之熹微。
    
    乃瞻衡宇,载欣载奔。僮仆欢迎,稚子候门。三径\footnote{〔三径〕汉朝蒋诩隐居之后,在院里竹下开辟三径,只于少数友人来往。后来,三径变成了隐士住处的代称。}就荒,松菊犹存。携幼入室,有酒盈樽。引壶觞以自酌,眄庭柯以怡颜。倚南窗以寄傲,审容膝之易安。园日涉以成趣,门虽设而常关。策扶老以流憩,时矫首而遐观。云无心以出岫,鸟倦飞而知还。景翳翳以将入,抚孤松而盘桓。
    
    归去来兮,请息交以绝游。世与我而相违,复驾言兮焉求?悦亲戚之情话,乐琴书以消忧。农人告余以春及,将有事于西畴。或命巾车\footnote{〔巾车〕有帷帐的马车。},或棹孤舟。既窈窕以寻壑,亦崎岖而经丘。木欣欣以向荣,泉涓涓而始流。善万物之得时,感吾生之行休。
    
    已矣乎!寓形宇内复几时?曷不委心任去留?胡为乎遑遑欲何之?富贵非吾愿,帝乡不可期。怀良辰以孤往,或植杖而耘耔。登东皋以舒啸,临清流而赋诗。聊乘化以归尽,乐夫天命复奚疑!
\end{normalsize}



\chapter{兼爱}

\begin{normalsize}
    
    圣人以治天下为事者也,必知乱之所自起,焉能治之;不知乱之所自起,则不能治。譬之如医之攻人之疾者然,必知疾之所自起,焉能攻之;不知疾之所自起,则弗能攻。治乱者何独不然?必知乱之所自起,焉能治之;不知乱之所自起,则弗能治。
    
    圣人以治天下为事者也,不可不察乱之所自起。当察乱何自起?起不相爱。臣子之不孝君父,所谓乱也。子自爱,不爱父,故亏父而自利;弟自爱,不爱兄,故亏兄而自利;臣自爱,不爱君,故亏君而自利。此所谓乱也。虽父之不慈子,兄之不慈弟,君之不慈臣,此亦天下之所谓乱也。父自爱也,不爱子,故亏子而自利;兄自爱也,不爱弟,故亏弟而自利;君自爱也,不爱臣,故亏臣而自利。是何也?皆起不相爱。虽至天下之为盗贼者,亦然。盗爱其室,不爱异室,故窃异室以利其室;贼爱其身,不爱人,故贼人以利其身。此何也?皆起不相爱。虽至大夫之相乱家、诸侯之相攻国者,亦然。大夫各爱其家,不爱异家,故乱异家以利其家;诸侯各爱其国,不爱异国,故攻异国以利其国。天下之乱物,具此而已矣。
    
    察此何自起?皆起不相爱。若使天下兼相爱,爱人若爱其身,犹有不孝者乎?视父兄与君若其身,恶施不孝?犹有不慈者乎?视弟子与臣若其身,恶施不慈?故不孝不慈亡有。犹有盗贼乎?视人之室若其室,谁窃?视人身若其身,谁贼?故盗贼有亡。犹有大夫之相乱家、诸侯之相攻国者乎?视人家若其家,谁乱?视人国若其国,谁攻?故大夫之相乱家、诸侯之相攻国者亡有。若使天下兼相爱,国与国不相攻,家与家不相乱,盗贼无有,君臣父子皆能孝慈,若此则天下治。
    
    故圣人以治天下为事者,恶得不禁恶而劝爱?故天下兼相爱则治,交相恶则乱。故子墨子曰不可以不劝爱人者,此也。
\end{normalsize}



\chapter{《论语》十二则}

\begin{normalsize}
    
    子曰:“为政以德。譬如北辰\footnote{〔北辰〕北极星。},居其所而众星共之。”
    
    子曰:“道之以政,齐之以刑,民免而无耻。道之以德,齐之以礼,有耻且格。”
    
    哀公\footnote{〔哀公〕春秋时期鲁国君主(公元前494年至前468年在位),鲁定公之子。}问于有若\footnote{〔有若〕字子有,孔子弟子,世称有子,孔子学说继承人之一。}曰:“年饥,用不足,如之何?”有若对曰:“盍彻乎?”曰:“二,吾犹不足,如之何其彻也?”对曰:“百姓足,君孰与不足?百姓不足,君孰与足?”
    
    齐景公\footnote{〔齐景公〕春秋时期齐国君主(公元前547年至前490年在位)。他联合鲁、卫反齐,试图重现齐桓公霸业,但最终失败。}问政于孔子,孔子对曰:“君君,臣臣,父父,子子。”公曰:“善哉!信如君不君、臣不臣、父不父、子不子,虽有粟,吾得而食诸?”
    
    季康子\footnote{〔季康子〕季孙肥,春秋时期鲁国的正卿。鲁国孟孙氏、叔孙氏和季孙氏专权,称为“三桓”,鲁国公权势衰微。}问政于孔子,孔子对曰:“政者,正也。子帅以正,孰敢不正?”
    
    子曰:“其身正,不令而行;其身不正,虽令不从。”
    
    子曰:“道千乘之国,敬事而信,节用而爱人,使民以时。”
    
    哀公问曰:“何为则民服?”孔子对曰:“举直错诸枉,则民服;举枉错诸直,则民不服。”
    
    定公\footnote{〔定公〕鲁定公,春秋诸侯国鲁国君主(公元前509年至前495年在位),请孔子相国,想从“三桓”手中夺回君权,但后来堕三都失败,孔子离开鲁国,鲁定公郁郁而终。}问:“君使臣,臣事君,如之何?”孔子对曰:“君使臣以礼,臣事君以忠。”
    
    子谓子产\footnote{〔子产〕公孙侨,字子美,郑穆公之孙、公子发之子。公元前554年为郑国卿,一度中兴郑国,前522年卒。}:“有君子之道四焉:其行己也恭,其事上也敬,其养民也惠,其使民也义。”
    
    子曰:“大哉尧\footnote{〔尧〕上古帝王,帝喾之子,姓伊祁,名放勋,初封于陶,后封为唐侯,辅佐他的兄长挚,二十岁继任天子,号陶唐,谥曰尧,史称唐尧。}之为君也!巍巍乎,唯天为大,唯尧则之。荡荡乎,民无能名焉。巍巍乎,其有成功也!焕乎,其有文章!”
    
    子曰:“禹\footnote{〔禹〕上古帝王,姒姓,夏后氏,名文命,黄帝的玄孙、颛顼的后代,鲧的儿子。因治水有功,被尊称为“大禹”。舜晚年禅位于禹。},吾无间然矣。菲饮食,而致孝乎鬼神;恶衣服,而致美乎黻冕;卑宫室,而尽力乎沟洫。禹,吾无间然矣。”
\end{normalsize}



\chapter{牧民}

\begin{normalsize}
    
    
\end{normalsize}


\newpage

\textbf{注解}:

\vspace{-1em}

\begin{itemize}
    \setlength\itemsep{-0.2em}
    \item〔礼不逾节,义不自进,廉不蔽恶,耻不从枉〕“逾节”就是不守规矩、不按章办事,或者做了超出自己本分和职责范围的事情。“自进”就是做什么事都为了自己的利益,为了让自己更进一步而不择手段,所谓被“权力欲”“物欲”等欲望支配。“蔽恶”就是包庇自己或他人的过错和犯罪,睁一只眼闭一只眼或找借口开脱。“从枉”就是徇私枉法,为了自己的私利而扭曲事实、违背道理,不正直不真诚,诡辩曲解,搞双重标准,破坏公正。
\end{itemize}

\chapter{劝学}

\begin{normalsize}
    
    君子曰:学不可以已。青,取之于蓝而青于蓝;冰,水为之而寒于水。木直中绳,輮以为轮,其曲中规,虽有槁暴,不复挺者,輮使之然也。故木受绳则直,金就砺则利。君子博学而日参省乎己,则知明而行无过矣。
    
    吾尝终日而思矣,不如须臾之所学也。吾尝跂而望矣,不如登高之博见也。登高而招,臂非加长也,而见者远。顺风而呼,声非加疾也,而闻者彰。假舆马者,非利足也,而致千里。假舟楫者,非能水也,而绝江河。君子生非异也,善假于物也。
    
    积土成山,风雨兴焉。积水成渊,蛟龙生焉。积善成德,而神明自得,圣心备焉。故不积跬步,无以至千里;不积小流,无以成江海。骐骥一跃,不能十步;驽马十驾,功在不舍。锲而舍之,朽木不折;锲而不舍,金石可镂。蚓无爪牙之利,筋骨之强,上食埃土,下饮黄泉,用心一也。蟹六跪而二螯,非蛇蟮之穴无可寄托者,用心躁也。是故无冥冥之志者,无昭昭之明;无惛惛之事者,无赫赫之功。行衢道者不至,事两君者不容。目不能两视而明,耳不能两听而聪。故君子结于一也。
\end{normalsize}



\chapter{始得西山宴游记}

\begin{normalsize}
    
    自余为僇人,居是州,恒惴栗。其隙也,则施施而行,漫漫而游。日与其徒上高山,入深林,穷回溪,幽泉怪石,无远不到。到则披草而坐,倾壶而醉。醉则更相枕以卧,卧而梦。意有所极,梦亦同趣。觉而起,起而归。以为凡是州之山水有异态者,皆我有也,而未始知西山\footnote{〔西山〕湖南零陵县西。}之怪特。
    
    今年九月二十八日,因坐法华西亭\footnote{〔法华西亭〕法华寺,在零陵县城东山上,柳宗元在此建了西亭。},望西山,始指异之。遂命仆人过湘江\footnote{〔湘江〕应为潇水。},缘染溪\footnote{〔染溪〕又作“冉溪”,柳宗元又称为“愚溪”,是潇水的一条小支流。},斫榛莽,焚茅茷,穷山之高而止。攀援而登,箕踞而遨,则凡数州之土壤,皆在衽席之下。其高下之势,岈然洼然,若垤若穴,尺寸千里,攒蹙累积,莫得遁隐。萦青缭白,外与天际,四望如一。然后知是山之特立,不与培塿为类。悠悠乎与颢气俱,而莫得其涯;洋洋乎与造物者游,而不知其所穷。引觞满酌,颓然就醉,不知日之入。苍然暮色,自远而至,至无所见而犹不欲归。心凝形释,与万化冥合。然后知吾向之未始游,游于是乎始。故为之文以志。是岁,元和四年\footnote{〔元和〕唐宪宗李纯年号(公元806年至820年)。}也。
\end{normalsize}



\chapter{师说}

\begin{normalsize}
    
    古之学者必有师。师者,所以传道受业解惑也。人非生而知之者,孰能无惑?惑而不从师,其为惑也终不解矣。生乎吾前,其闻道也固先乎吾,吾从而师之;生乎吾后,其闻道也亦先乎吾,吾从而师之。吾师道也,夫庸知其年之先后生于吾乎?是故无贵无贱,无长无少,道之所存,师之所存也。
    
    嗟乎!师道之不传也久矣!欲人之无惑也难矣!古之圣人,其出人也远矣,犹且从师而问焉;今之众人,其下圣人也亦远矣,而耻学于师。是故圣益圣,愚益愚。圣人之所以为圣,愚人之所以为愚,其皆出于此乎?爱其子,择师而教之;于其身也,则耻师焉,惑矣!彼童子之师,授之书而习其句读者,非吾所谓传其道解其惑者也。句读之不知,惑之不解,或师焉,或不焉,小学而大遗,吾未见其明也。巫医乐师百工之人,不耻相师。士大夫之族,曰师曰弟子云者,则群聚而笑之。问之,则曰:彼与彼年相若也,道相似也。位卑则足羞,官盛则近谀。呜呼!师道之不复可知矣!巫医乐师百工之人,君子不齿,今其智乃反不能及,其可怪也欤!
    
    圣人无常师。孔子师郯子\footnote{〔郯子〕春秋时郯国的国君,相传孔子曾向他请教官职。}、苌弘\footnote{〔苌弘〕东周敬王的大夫,相传孔子曾向他请教古乐。}、师襄\footnote{〔师襄〕春秋时鲁国的乐官,相传孔子曾向他学琴。}、老聃\footnote{〔老聃〕即老子,姓李名耳字聃,春秋时楚国人,思想家,道家学派创始人。相传孔子曾向他学习周礼。}。郯子之徒,其贤不及孔子。孔子曰:“三人行,则必有我师。”是故弟子不必不如师,师不必贤于弟子,闻道有先后,术业有专攻,如是而已。
    
    李氏子蟠\footnote{〔李氏子蟠〕李蟠,韩愈弟子,贞元十九年进士。},年十七,好古文,六艺\footnote{〔六艺〕指六经:《诗》《书》《礼》《乐》《易》《春秋》。}经传皆通习之,不拘于时,学于余。余嘉其能行古道,作《师说》以贻之。
\end{normalsize}



\chapter{阿房宫赋}

\begin{normalsize}
    
    六王\footnote{〔六王〕指战国时齐、楚、燕、韩、赵、魏六国国王。}毕,四海一,蜀山兀,阿房\footnote{〔阿房〕阿房宫,秦始皇晚年修建的奢华宫殿。}出。覆压三百余里,隔离天日。骊山\footnote{〔骊山〕现在陕西省临潼县东南。}北构而西折,直走咸阳\footnote{〔咸阳〕秦朝都城,现在陕西咸阳市。}。二川\footnote{〔二川〕指渭水和樊川。渭水源出甘肃,流经陕西省;樊川即樊水,灞水的支流,在今陕西省。}溶溶,流入宫墙。五步一楼,十步一阁;廊腰缦回,檐牙高啄;各抱地势,钩心斗角。盘盘焉,囷囷焉,蜂房水涡,矗不知其几千万落。长桥卧波,未云何龙?复道行空,不霁何虹?高低冥迷,不知东西。歌台暖响,春光融融。舞殿冷袖,风雨凄凄。一日之内,一宫之间,而气候不齐。
    
    妃嫔媵嫱,王子皇孙,辞楼下殿,辇来于秦;朝歌夜弦,为秦宫人。明星荧荧,开妆镜也;绿云扰扰,梳晓鬟也;渭流涨腻,弃脂水也;烟斜雾横,焚椒兰也。雷霆乍惊,宫车过也;辘辘远听,杳不知其所之也。一肌一容,尽态极妍,缦立远视,而望幸焉。有不见者,三十六年。燕、赵之收藏,韩、魏之经营,齐、楚之精英,几世几年,摽掠其人,倚叠如山。一旦不能有,输来其间。鼎铛玉石,金块珠砾,弃掷逦迤,秦人视之,亦不甚惜。
    
    嗟乎!一人之心,千万人之心也。秦爱纷奢,人亦念其家。奈何取之尽锱铢\footnote{〔锱铢〕古时的重量单位。《说文》:六铢为锱。泛指微少。},用之如泥沙?使负栋之柱,多于南亩之农夫;架梁之椽,多于机上之工女;钉头磷磷,多于在庾之粟粒;瓦缝参差,多于周身之帛缕;直栏横槛,多于九土\footnote{〔九土〕九州,指天下。}之城郭;管弦呕哑,多于市人之言语。使天下之人,不敢言而敢怒。独夫之心,日益骄固。戍卒叫\footnote{〔戍卒叫〕指陈胜、吴广在谪戍渔阳途中起义。},函谷举\footnote{〔函谷举〕指刘邦攻破函谷关。},楚人一炬\footnote{〔楚人一炬〕楚人项羽攻入咸阳后,“烧秦宫室,火三月不灭”。},可怜焦土!
    
    灭六国者,六国也,非秦也。族秦者,秦也,非天下也。嗟乎!使六国各爱其人,则足以拒秦。使秦复爱六国之人,则递三世可至万世而为君,谁得而族灭也?秦人不暇自哀,而后人哀之。后人哀之而不鉴之,亦使后人而复哀后人也。
\end{normalsize}



\chapter{六国论}

\begin{normalsize}
    
    六国破灭,非兵不利,战不善,弊在赂秦。赂秦而力亏,破灭之道也。或曰:六国互丧,率赂秦耶?曰:不赂者以赂者丧,盖失强援,不能独完。故曰:弊在赂秦也。
    
    秦以攻取之外,小则获邑,大则得城。较秦之所得,与战胜而得者,其实百倍。诸侯之所亡,与战败而亡者,其实亦百倍。则秦之所大欲,诸侯之所大患,固不在战矣。思厥先祖父,暴霜露,斩荆棘,以有尺寸之地。子孙视之不甚惜,举以予人,如弃草芥。今日割五城,明日割十城,然后得一夕安寝。起视四境,而秦兵又至矣。然则诸侯之地有限,暴秦之欲无厌,奉之弥繁,侵之愈急。故不战而强弱胜负已判矣。至于颠覆,理固宜然。古人云:“以地事秦,犹抱薪救火,薪不尽,火不灭。”此言得之。
    
    齐人未尝赂秦,终继五国迁灭,何哉?与嬴而不助五国也。五国既丧,齐亦不免矣。燕赵之君,始有远略,能守其土,义不赂秦。是故燕虽小国而后亡,斯用兵之效也。至丹以荆卿为计,始速祸焉。赵尝五战于秦,二败而三胜。后秦击赵者再,李牧连却之。自牧以谗诛,邯郸为郡,惜其用武而不终也。且燕赵处秦革灭殆尽之际,可谓智力孤危,战败而亡,诚不得已。向使三国各爱其地,齐人勿附于秦,刺客不行,良将犹在,则胜负之数,存亡之理,当与秦相较,或未易量。
    
    呜呼!以赂秦之地封天下之谋臣,以事秦之心礼天下之奇才,并力西向,则吾恐秦人食之不得下咽也。悲夫!有如此之势,而为秦人积威之所劫,日削月割,以趋于亡。为国者无使为积威之所劫哉!
    
    夫六国与秦皆诸侯,其势弱于秦,而犹有可以不赂而胜之之势。苟以天下之大,下而从六国破亡之故事,是又在六国下矣。
\end{normalsize}



\chapter{《孟子》六则}

\begin{normalsize}
    
    三代之得天下也以仁,其失天下也以不仁。国之所以废兴存亡者亦然。天子不仁,不保四海;诸侯不仁,不保社稷;卿大夫不仁,不保宗庙;士庶人不仁,不保四体。今恶死亡而乐不仁,是犹恶醉而强酒。
    
    以力假仁者霸,霸必有大国;以德行仁者王,王不待大。汤以七十里,文王以百里。以力服人者,非心服也,力不赡也;以德服人者,中心悦而诚服也,如七十子之服孔子也\footnote{〔七十子之服孔子〕《史记.孔子世家》:“孔子以诗书礼乐教弟子,盖三千焉,身通六艺者七十有二人。”《史记.仲尼弟子列传》:“孔子曰‘受业身通者七十有七人’,皆异能之士也。”}。《诗》云\footnote{〔《诗》云……〕出自《诗经·大雅·文王有声》。}:“自西自东,自南自北,无思不服。”此之谓也。
    
    天时不如地利,地利不如人和。三里之城,七里之郭,环而攻之而不胜。夫环而攻之,必有得天时者矣;然而不胜者,是天时不如地利也。城非不高也,池非不深也,兵革非不坚利也,米粟非不多也,委而去之,是地利不如人和也。故曰:域民不以封疆之界,固国不以山溪之险,威天下不以兵革之利。得道者多助,失道者寡助。寡助之至,亲戚畔之。多助之至,天下顺之。以天下之所顺,攻亲戚之所畔,故君子有不战,战必胜矣。
    
    舜\footnote{〔舜〕古代圣君,三皇五帝之一。他出身贫寒,年轻时在历山耕田。他的父亲顽劣,母亲嚣张,弟弟傲慢,但他仍能以孝道和睦家庭。尧发现他的才能和品德后,将帝位禅让给他。}发于畎亩之中,傅说\footnote{〔傅说〕商朝宰相。他本是筑墙的泥水匠,在傅岩筑墙时被商王武丁发现。}举于版筑之间,胶鬲\footnote{〔胶鬲〕商朝大臣。}举于鱼盐之中,管夷吾\footnote{〔管夷吾〕管仲,字夷吾,出身贫苦,经商为生,后经鲍叔牙举荐为齐相。}举于士,孙叔敖\footnote{〔孙叔敖〕楚国名相。他本是郊野平民,因在家乡主持修建水利,被楚庄王看重,任命为令尹。}举于海,百里奚\footnote{〔百里奚〕秦国名相。他本是虞国大夫。晋献公灭虞后成为奴隶。秦穆公用五张羊皮将他赎身,成为秦国大夫。}举于市。故天将降大任于是人也,必先苦其心志,劳其筋骨,饿其体肤,空乏其身,行拂乱其所为,所以动心忍性,曾益其所不能。人恒过,然后能改;困于心,衡于虑,而后作;征于色,发于声,而后喻。入则无法家拂士,出则无敌国外患者,国恒亡。然后知生于忧患,而死于安乐也。
    
    今之事君者皆曰:“我能为君辟土地,充府库。”今之所谓良臣,古之所谓民贼也。君不乡道,不志于仁,而求富之,是富桀也。“我能为君约与国,战必克。”今之所谓良臣,古之所谓民贼也。君不乡道,不志于仁,而求为之强战,是辅桀也。由今之道,无变今之俗,虽与之天下,不能一朝居也。
    
    不仁者可与言哉?安其危而利其灾,乐其所以亡者。不仁而可与言,则何亡国败家之有?有孺子歌曰:“沧浪之水清兮,可以濯我缨;沧浪之水浊兮,可以濯我足。”孔子曰:“小子听之!清斯濯缨,浊斯濯足矣。自取之也。”夫人必自侮,然后人侮之;家必自毁,而后人毁之;国必自伐,而后人伐之。《太甲》曰\footnote{〔《太甲》曰……〕出自《尚书·商书·太甲》:“天作孽,犹可违;自作孽,不可逭。”}:“天作孽,犹可违;自作孽,不可活。”此之谓也。
\end{normalsize}


\newpage

\textbf{注解}:

\vspace{-1em}

\begin{itemize}
    \setlength\itemsep{-0.2em}
    \item〔三里之城,七里之郭〕內城外郭。《吴越春秋》:“筑城以卫君,造郭以守民,此城郭之始也”。“城”是以政治军事职能为主、作为权力中心的聚落;“郭”是承担城市中商业、手工业、农业及居民区等经济生活职能的外围区域。郭,廓也。廓落在城外也,具体分为“内城外郭”和“西城东郭”两种形制。城、郭都是大致方形的,“三里”和“七里”指的是边长。在战国时期,三里长、七里长的都是较小的城郭。
    \item〔人恒过,然后能改;困于心,衡于虑,而后作;征于色,发于声,而后喻。〕《孟子集注》:“衡,与横同。恒,常也。犹言大率也。横,不顺也。作,奋起也。徵,验也。喻,晓也。此又言中人之性,常必有过,然后能改。盖不能谨于平日,故必事势穷蹙,以至困于心,横于虑,然后能奋发而兴起;不能烛于几微,故必事理暴著,以至验于人之色,发于人之声,然后能警悟而通晓也。”
\end{itemize}

\chapter{召公谏厉王弭谤}

\begin{normalsize}
    
    厉王\footnote{〔厉王〕周厉王,周夷王之子,名胡,西周的第十个国君(前878至前842年)。}虐,国人谤王。召公\footnote{〔召公〕召氏,名虎,谥穆,也称召穆公,召幽伯之子。}告曰:“民不堪命矣!”王怒,得卫巫\footnote{〔卫巫〕卫:卫国。巫:古代以降神事鬼为职业的人。},使监谤者。以告,则杀之。国人莫敢言,道路以目。
    
    王喜,告召公曰:“吾能弭谤矣,乃不敢言。”
    
    召公曰:“是障之也。防民之口,甚于防川。川壅而溃,伤人必多,民亦如之。是故为川者决之使导;为民者宣之使言。故天子听政,使公卿至于列士献诗,瞽献曲,史献书,师箴,瞍赋,矇诵,百工谏,庶人传语,近臣尽规,亲戚补察,瞽、史教诲,耆艾修之,而后王斟酌焉。是以事行而不悖。民之有口也,犹土之有山川也,财用于是乎出;犹其有原隰衍沃也,衣食于是乎生。口之宣言也,善败于是乎兴。行善而备败,其所以阜财用衣食者也。夫民虑之于心而宣之于口,成而行之,胡可壅也?若壅其口,其与能几何?”
    
    王弗听,于是国人莫敢出言。三年,乃流王于彘\footnote{〔彘〕地名,在今山西省霍县境内。}。
\end{normalsize}



\chapter{石钟山记}

\begin{normalsize}
    
    《水经》\footnote{〔《水经》〕汉代记载地理山川的书籍。}云:“彭蠡\footnote{〔彭蠡〕鄱阳湖的又一名称。}之口有石钟山焉。”郦元\footnote{〔郦元〕即郦道元,北魏人,地理学家,著《水经注》,为《水经》作注。}以为下临深潭,微风鼓浪,水石相搏,声如洪钟。是说也,人常疑之。今以钟磬\footnote{〔磬〕古代打击乐器,形状像曲尺,用玉或石制成。}置水中,虽大风浪不能鸣也,而况石乎!至唐李渤\footnote{〔李渤〕唐朝洛阳人,写过一篇《辨石钟山记》。}始访其遗踪,得双石于潭上,扣而聆之,南声函胡,北音清越,桴止响腾,余韵徐歇。自以为得之矣。然是说也,余尤疑之。石之铿然有声者,所在皆是也,而此独以钟名,何哉?
    
    元丰七年\footnote{〔元丰〕宋神宗的年号。}六月丁丑\footnote{〔六月丁丑〕农历六月初九。},余自齐安\footnote{〔齐安〕现在湖北黄州。}舟行适临汝\footnote{〔临汝〕即汝州,现在河南临汝。},而长子迈将赴饶之德兴尉\footnote{〔饶之德兴尉〕饶州德兴县(现在江西德兴)的县尉。},送之至湖口\footnote{〔湖口〕现在江西湖口。},因得观所谓石钟者。寺僧使小童持斧,于乱石间择其一二扣之,硿硿焉。余固笑而不信也。至莫夜月明,独与迈乘小舟,至绝壁下。大石侧立千尺,如猛兽奇鬼,森然欲搏人;而山上栖鹘,闻人声亦惊起,磔磔云霄间;又有若老人咳且笑于山谷中者,或曰此鹳鹤也。余方心动欲还,而大声发于水上,噌吰如钟鼓不绝。
    
    舟人大恐。徐而察之,则山下皆石穴罅,不知其浅深,微波入焉,涵淡澎湃而为此也。舟回至两山间,将入港口,有大石当中流,可坐百人,空中而多窍,与风水相吞吐,有窾坎镗鞳之声,与向之噌吰者相应,如乐作焉。因笑谓迈曰:“汝识之乎?噌吰者,周景王之无射\footnote{〔周景王之无射〕《国语》记载,周景王二十三年(公元前522年)铸成“无射”钟。}也;窾坎镗鞳者,魏庄子之歌钟\footnote{〔魏庄子之歌钟〕《左传》记载,鲁襄公十一年(公元前561年)郑人以歌钟和其他乐器献给晋侯,晋侯分一半赐给晋大夫魏绛。庄子,魏绛的谥号。歌钟,古乐器。}也。古之人不余欺也!”
    
    事不目见耳闻,而臆断其有无,可乎?郦元之所见闻,殆与余同,而言之不详;士大夫终不肯以小舟夜泊绝壁之下,故莫能知;而渔工水师虽知而不能言。此世所以不传也。而陋者乃以斧斤考击而求之,自以为得其实。余是以记之,盖叹郦元之简,而笑李渤之陋也。
\end{normalsize}



\chapter{游褒禅山记}

\begin{normalsize}
    
    褒禅山亦谓之华山\footnote{〔华山〕褒禅山在安徽马鞍山市含山县,属大别山余脉,横亘在巢湖之滨,别名花山(古时“华”通“花”)。褒禅山不是五岳中的西岳华山。}。唐浮图\footnote{〔浮图〕梵语音译词,也写作“浮屠”或“佛图”,本意是佛,这里指佛教徒。}慧褒始舍于其址,而卒葬之;以故其后名之曰“褒禅”。今所谓慧空禅院者,褒之庐冢也。距其院东五里,所谓华山洞者,以其乃华山之阳名之也。距洞百余步,有碑仆道,其文漫灭,独其为文犹可识曰“花山”。今言“华”如“华实”之“华”者,盖音谬也。
    
    其下平旷,有泉侧出,而记游者甚众,所谓前洞也。由山以上五六里,有穴窈然,入之甚寒,问其深,则其好游者不能穷也,谓之后洞。余与四人拥火以入,入之愈深,其进愈难,而其见愈奇。有怠而欲出者,曰:“不出,火且尽。”遂与之俱出。盖余所至,比好游者尚不能十一,然视其左右,来而记之者已少。盖其又深,则其至又加少矣。方是时,余之力尚足以入,火尚足以明也。既其出,则或咎其欲出者,而余亦悔其随之,而不得极夫游之乐也。
    
    于是余有叹焉。古人之观于天地、山川、草木、虫鱼、鸟兽,往往有得,以其求思之深而无不在也。夫夷以近,则游者众;险以远,则至者少。而世之奇伟、瑰怪,非常之观,常在于险远,而人之所罕至焉,故非有志者不能至也。有志矣,不随以止也,然力不足者,亦不能至也。有志与力,而又不随以怠,至于幽暗昏惑而无物以相之,亦不能至也。然力足以至焉,于人为可讥,而在己为有悔;尽吾志也而不能至者,可以无悔矣,其孰能讥之乎?此余之所得也。
    
    余于仆碑,又以悲夫古书之不存,后世之谬其传而莫能名者,何可胜道也哉!此所以学者不可以不深思而慎取之也。
    
    四人者:庐陵\footnote{〔庐陵〕今江西吉安市。}萧君圭君玉,长乐\footnote{〔长乐〕今福建福州市长乐区。}王回深父,余弟安国平父、安上纯父。
    
    至和元年\footnote{〔至和〕宋仁宗的年号。}七月某日,临川\footnote{〔临川〕现在江西抚州市临川区。}王某记。
\end{normalsize}



\chapter{与妻书}

\begin{normalsize}
    
    意映卿卿\footnote{〔卿卿〕旧时夫妻间的爱称,多用于丈夫称呼妻子。}如晤:吾今以此书与汝永别矣!吾作此书时,尚为世中一人;汝看此书时,吾已成为阴间一鬼。吾作此书,泪珠和笔墨齐下,不能书竟,而欲搁笔,又恐汝不察吾衷,谓吾忍舍汝而死,谓吾不知汝之不欲吾死也,故遂忍悲为汝言之。
    
    吾至爱汝!即此爱汝一念,使吾勇于就死也!吾自遇汝以来,常愿天下有情人都成眷属,然遍地腥云,满街狼犬,称心快意,几家能够?司马青衫\footnote{〔司马青衫〕白居易《琵琶行》中有“座中泣下谁最多?江州司马青衫湿”,比喻悲伤落泪。},吾不能学太上之忘情\footnote{〔太上〕圣人。}也。语云,仁者“老吾老以及人之老,幼吾幼以及人之幼”。吾充吾爱汝之心,助天下人爱其所爱,所以敢先汝而死,不顾汝也。汝体吾此心,于悲啼之余,亦以天下人为念,当亦乐牺牲吾身与汝身之福利,为天下人谋永福也。汝其勿悲。
    
    汝忆否四五年前某夕,吾尝语曰:“与使吾先死也,无宁汝先吾而死。”汝初闻言而怒,后经吾婉解,虽不谓吾言为是,而亦无辞相答。吾之意盖谓以汝之弱,必不能禁失吾之悲,吾先死留苦与汝,吾心不忍,故宁请汝先死,吾担悲也。嗟夫,谁知吾卒先汝而死乎!
    
    吾真不能忘汝也!回忆后街之屋,入门穿廊,过前后厅,又三四折有小厅,厅旁一室为吾与汝双棲之所。初婚三四个月,适冬之望日前后,窗外疏梅筛月影,依稀掩映,吾与汝並肩携手,低低切切,何事不语,何情不诉!及今思之,空余泪痕!又回忆六七年前,吾之逃家复归也,汝泣告我:“望今后有远行,必以告妾,妾愿随君行。”吾亦既许汝矣。前十余日回家,即欲乘便以此行之事语汝,及与汝相对,又不能启口;且以汝之有身也,更恐不胜悲,故惟日日呼酒买醉。嗟夫!当时余心之悲,盖不能以寸管形容之。
    
    吾诚愿与汝相守以死。第以今日事势观之,天灾可以死,盗贼可以死,瓜分之日可以死,奸官污吏虐民可以死,吾辈处今日之中国,国中无地无时不可以死!到那时使吾眼睁睁看汝死,或使汝眼睁睁看我死,吾能之乎!抑汝能之乎!即可不死,而离散不相见,徒使两地眼成穿而骨化石\footnote{〔骨化石〕指望夫石传说,多地皆有,大多为妇人伫立望夫归来,日久化而为石。},试问古来几曾见破镜能重圆?则较死为苦也。将奈之何?今日吾与汝幸双健;天下人人不当死而死,与不愿离而离者,不可数计。钟情如我辈者,能忍之乎?此吾所以敢率性就死不顾汝也!吾今死无余憾,国事成不成,自有同志者在。依新已五岁,转眼成人,汝其善抚之,使之肖我。汝腹中之物,吾疑其女也,女必像汝,吾心甚慰;或又是男,则亦教其以父志为志,则我死后,尚有二意洞\footnote{〔意洞〕林觉民的字。}在也,甚幸甚幸!吾家后日当甚贫,贫无所苦,清静过日而已。
    
    吾今与汝无言矣。吾居九泉之下遥闻汝哭声,当哭相和也。吾平日不信有鬼,今则又望其真有。今是人又言心电感应\footnote{〔心电感应〕指人死后心灵还有知觉,能与活人的精神、心情交相感应。}有道,吾亦望其言是实,则吾之死,吾灵尚依依旁汝也,汝不必以无侣悲。
    
    吾平生未尝以吾所志语汝,是吾不是处;然语之,又恐汝日日为吾担忧。吾牺牲百死而不辞,而使汝担忧,的的非吾所忍。吾爱汝至,所以为汝谋者惟恐未尽。汝幸而偶我,又何不幸而生今日中国!吾幸而得汝,又何不幸而生今日之中国!卒不忍独善其身。嗟夫!巾短情长,所未尽者,尚有万千,汝可以模拟得之。吾今不能见汝矣!汝不能舍吾,其时时于梦中得我乎!一恸!
    
    辛未\footnote{〔辛未〕应是“辛亥”,指公元1911年。}三月廿六夜四鼓,意洞手书。
    
    家中诸母皆通文,有不解处,望请其指教,当尽吾意为幸。
\end{normalsize}



\chapter{张衡传}

\begin{normalsize}
    
    张衡字平子,南阳西鄂\footnote{〔南洋西鄂〕南阳郡西鄂县,在今河南南阳。}人也。衡少善属文,游于三辅,因入京师\footnote{〔京师〕东汉首都洛阳(今河南洛阳市)},观太学\footnote{〔太学〕古代设在京城的全国最高学府,西汉武帝开始设立。},遂通五经\footnote{〔五经〕汉武帝时将《诗》《书》《礼》《易》《春秋》定名为“五经”。},贯六艺\footnote{〔六艺〕指礼、乐、射、御、书、数。}。虽才高于世,而无骄尚之情。常从容淡静,不好交接俗人。永元\footnote{〔永元〕东汉和帝(刘肇)的年号(公元79年至106年)。}中,举孝廉不行,连辟公府\footnote{〔公府〕三公的官署。东汉以太尉、司徒、司空为三公。}不就。时天下承平日久,自王侯以下,莫不逾侈。衡乃拟班固\footnote{〔班固〕字孟坚,东汉史学家、文学家。}《两都》\footnote{〔《两都》〕指《两都赋》,分《西都赋》《东都赋》。}作《二京赋》\footnote{〔《二京赋》〕指《西京赋》《东京赋》。},因以讽谏。精思傅会,十年乃成。大将军邓骘\footnote{〔邓骘〕东汉和帝邓皇后的哥哥,立安帝,以大将军的身份辅佐安帝管理政事。}奇其才,累召不应。
    
    衡善机巧,尤致思于天文、阴阳\footnote{〔阴阳〕这里指通过天象推算地理节气。}、历算\footnote{〔历算〕推算年月日和节气,泛指数学。}。安帝雅闻衡善术学,公车\footnote{〔公车〕汉代官署名称,设公车令。}特征拜郎中,再迁为太史令\footnote{〔太史令〕东汉时掌管天文历数的官,与西汉以前掌管天象历法兼有修史之责的太史令职责不完全相同。}。遂乃研核阴阳,妙尽璇玑\footnote{〔璇玑〕上古用于观测天象的仪器“璇玑玉衡”中旋转的部分。}之正,作浑天仪\footnote{〔浑天仪〕表示天象的仪器,类似天球仪,反映浑天说的理论。},著《灵宪》\footnote{〔《灵宪》〕张衡写的历法书,已佚。}《算罔》\footnote{〔《算罔》〕张衡写的数算书,已佚。}论,言甚详明。
    
    顺帝初,再转,复为太史令。衡不慕当世\footnote{〔当世〕指当政的权臣。},所居之官辄积年不徙。自去史职,五载复还。
    
    阳嘉\footnote{〔阳嘉〕东汉顺帝(刘保)的年号(公元132年至135年)。}元年,复造候风地动仪\footnote{〔候风地动仪〕一种感应地震的仪器,已失传。}。以精铜铸成,员径八尺\footnote{〔员径八尺〕圆直径八尺。},合盖隆起,形似酒尊,饰以篆文山龟鸟兽之形。中有都柱\footnote{〔都柱〕中心立柱,一根上大下小的柱子,哪个方向发生地震,柱子便倒向哪边。},傍行八道,施关发机。外有八龙,首衔铜丸,下有蟾蜍,张口承之。其牙机\footnote{〔牙机〕带齿的机械。}巧制,皆隐在尊中,覆盖周密无际。如有地动,尊则振龙,机发吐丸,而蟾蜍衔之。振声激扬,伺者因此觉知。虽一龙发机,而七首不动。寻其方面,乃知震之所在。验之以事,合契若神。自书典所记,未之有也。尝一龙机发而地不觉动,京师学者咸怪其无征。后数日驿\footnote{〔驿〕驿使,古时骑马在驿道上传递文书的人。}至,果地震陇西\footnote{〔陇西〕陇西郡,现在甘肃兰州市临洮县陇西县一带。},于是皆服其妙。自此以后,乃令史官记地动所从方起。
    
    时政事渐损,权移于下,衡因上疏陈事。后迁侍中,帝引在帷幄\footnote{〔帷幄〕室内悬挂的帐幕,天子居处必设帷幄。这里代指天子身边的决策中心。},讽议左右。尝问天下所疾恶者。宦官惧其毁己,皆共目之,衡乃诡对而出。阉竖恐终为其患,遂共谗之。衡常思图身之事,以为吉凶倚伏,幽微难明,乃作《思玄赋》以宣寄情志。
    
    永和\footnote{〔永和〕东汉顺帝的年号(公元136年至141年)。}初,出为河间相。时国王\footnote{〔国王〕指河间惠王刘政,汉章帝孙,按辈分是汉顺帝的叔叔。}骄奢,不遵典宪;又多豪右\footnote{〔豪右〕豪族大户,权势盛大的地方家族。},共为不轨。衡下车,治威严,整法度,阴知奸党名姓,一时收禽,上下肃然,称为政理。视事三年,上书乞骸骨\footnote{〔乞骸骨〕古代官吏因年老请求退职的一种说法。},征拜尚书\footnote{〔尚书〕东汉时是在宫廷中协助皇帝处理政务的官。}。年六十二,永和四年卒。
\end{normalsize}



\chapter{非攻}

\begin{normalsize}
    
    今有一人,入人园圃,窃其桃李,众闻则非之,上为政者得则罚之。此何也?以亏人自利也。至攘人犬豕鸡豚者,其不义又甚入人园圃窃桃李。是何故也?以亏人愈多。苟亏人愈多,其不仁兹甚,罪益厚。至入人栏厩,取人马牛者,其不仁义又甚攘人犬豕鸡豚。此何故也?以其亏人愈多。苟亏人愈多,其不仁兹甚,罪益厚。至杀不辜人也,拖其衣裘、取戈剑者,其不义又甚入人栏厩取人马牛。此何故也?以其亏人愈多。苟亏人愈多,其不仁兹甚,罪益厚。当此,天下\footnote{〔天下〕泛指当时周天子统治的地方,包括各诸侯国。}之君子\footnote{〔君子〕君王之子,指代有德的人。}皆知而非之,谓之不义。今至大为不义攻国,则弗知非,从而誉之,谓之义。此可谓知义与不义之别乎?
    
    杀一人,谓之不义,必有一死罪矣。若以此说往,杀十人,十重不义,必有十死罪矣;杀百人,百重不义,必有百死罪矣。当此,天下之君子皆知而非之,谓之不义。今至大为不义攻国,则弗知非,从而誉之,谓之义。情不知其不义也,故书其言以遗后世。若知其不义也,夫奚说书其不义以遗后世哉?
    
    今有人于此,少见黑曰黑,多见黑曰白,则必以此人为不知白黑之辩矣。少尝苦曰苦,多尝苦曰甘,则必以此人为不知甘苦之辩矣。今小为非,则知而非之;大为非攻国,则不知非,从而誉之,谓之义。此可谓知义与不义之辩乎?是以知天下之君子也,辩义与不义之乱也。
\end{normalsize}



\chapter{归去来兮辞}

\begin{normalsize}
    
    齐人有冯谖者,贫乏不能自存,使人属孟尝君,愿寄食门下。孟尝君曰:“客何好?”曰:“客无好也。”曰:“客何能?”曰:“客无能也。”孟尝君笑而受之曰:“诺。”
    
    左右以君贱之也,食以草具。居有顷,倚柱弹其剑,歌曰:“长铗归来乎!食无鱼。”左右以告。孟尝君曰:“食之,比门下之鱼客。”居有顷,复弹其铗,歌曰:“长铗归来乎!出无车。”左右皆笑之,以告。孟尝君曰:“为之驾,比门下之车客。”于是乘其车,揭其剑,过其友曰:“孟尝君客我。”后有顷,复弹其剑铗,歌曰:“长铗归来乎!无以为家。”左右皆恶之,以为贪而不知足。孟尝君问:“冯公有亲乎?”对曰,“有老母。”孟尝君使人给其食用,无使乏。于是冯谖不复歌。
    
    后孟尝君出记,问门下诸客:“谁习计会,能为文收责于薛者乎?”冯谖署曰:“能。”孟尝君怪之,曰:“此谁也?”左右曰:“乃歌夫长铗归来者也。”孟尝君笑曰:“客果有能也,吾负之,未尝见也。”请而见之,谢曰:“文倦于事,愦于忧,而性懧愚,沉于国家之事,开罪于先生。先生不羞,乃有意欲为收责于薛乎?”冯谖曰:“愿之。”于是约车治装,载券契而行,辞曰:“责毕收,以何市而反?”孟尝君曰:“视吾家所寡有者。”
    
    驱而之薛,使吏召诸民当偿者,悉来合券。券遍合,起,矫命以责赐诸民,因烧其券。民称万岁。
    
    长驱到齐,晨而求见。孟尝君怪其疾也,衣冠而见之,曰:“责毕收乎?来何疾也!”曰:“收毕矣。”“以何市而反?”冯谖曰:“君云‘视吾家所寡有者’。臣窃计,君宫中积珍宝,狗马实外厩,美人充下陈。君家所寡有者,以义耳!窃以为君市义。”孟尝君曰:“市义奈何?”曰:“今君有区区之薛,不拊爱子其民,因而贾利之。臣窃矫君命,以责赐诸民,因烧其券,民称万岁。乃臣所以为君市义也。”孟尝君不说,曰:“诺,先生休矣!”
    
    后期年,齐王谓孟尝君曰:“寡人不敢以先王之臣为臣。”孟尝君就国于薛。未至百里,民扶老携幼,迎君道中。孟尝君顾谓冯谖:“先生所为文市义者,乃今日见之。”
    
    冯谖曰:“狡兔有三窟,仅得免其死耳;今君有一窟,未得高枕而卧也。请为君复凿二窟。”孟尝君予车五十乘,金五百斤,西游于梁,谓惠王曰:“齐放其大臣孟尝君于诸侯,诸侯先迎之者,富而兵强。”于是梁王虚上位,以故相为上将军,遣使者黄金千斤,车百乘,往聘孟尝君。冯谖先驱,诫孟尝君曰:“千金,重币也;百乘,显使也。齐其闻之矣。”梁使三反,孟尝君固辞不往也。
    
    齐王闻之,君臣恐惧,遣太傅赍黄金千斤、文车二驷,服剑一,封书,谢孟尝君曰:“寡人不祥,被于宗庙之祟,沉于谄谀之臣,开罪于君。寡人不足为也;愿君顾先王之宗庙,姑反国统万人乎!”冯谖诫孟尝君曰:“愿请先王之祭器,立宗庙于薛。”庙成,还报孟尝君曰:“三窟已就,君姑高枕为乐矣。”
    
    孟尝君为相数十年,无纤介之祸者,冯谖之计也。
\end{normalsize}


\newpage

\textbf{注解}:

\vspace{-1em}

\begin{itemize}
    \setlength\itemsep{-0.2em}
    \item〔不敢以先王之臣为臣〕《史记·孟尝君列传》:“齐(湣)王惑于秦、楚之毁,以为孟尝君各高其主,而擅齐国之权,遂废孟尝君。”所谓“不敢以先王之臣为臣”是托辞。
    \item〔孟尝君予车五十乘,金五百斤,西游于梁,谓惠王曰〕《史记·孟尝君列传》记载为冯谖游说秦王。
\end{itemize}

\chapter{鸿门宴}

\begin{normalsize}
    
    行略定秦地,函谷关\footnote{〔函谷关〕现在河南灵宝县北崤山中,地势险要,是战国时秦国抵御外敌的重要关口。函谷关以东称为关东。}有兵守关,不得入。又闻沛公\footnote{〔沛公〕即刘邦,秦末随项梁反秦,后与项羽争天下,建立汉朝。}已破咸阳\footnote{〔咸阳〕秦朝都城,现在陕西咸阳市。},项羽\footnote{〔项羽〕项籍,字羽,楚国名将项燕之孙。秦二世初年起兵反秦,在巨鹿之战大破秦军主力,建立西楚,与刘邦争天下,最后在垓下之战兵败身死。}大怒,使当阳君\footnote{〔当阳君〕即英布,秦末汉初名将,项羽封其为九江王,后叛楚归汉。}等击关。项羽遂入,至于戏西\footnote{〔戏西〕戏水之西。戏水,出今陕西临潼县南骊山,北流入渭水。}。沛公军霸上\footnote{〔霸上〕即灞上,现在陕西西安市东,在灞水旁。},未得与项羽相见。沛公左司马曹无伤使人言于项羽曰:“沛公欲王关中,使子婴\footnote{〔子婴〕秦二世胡亥之子,被赵高立为秦王,刘邦入咸阳后投降刘邦,后被项羽杀死。}为相,珍宝尽有之。”项羽大怒,曰:“旦日飨士卒,为击破沛公军!”当是时,项羽兵四十万,在新丰鸿门\footnote{〔新丰鸿门〕现在陕西临潼县西北,秦时称骊邑。鸿门,在临潼县东。};沛公兵十万,在霸上。范增\footnote{〔范增〕项羽的谋士,被项羽尊称为“亚父”。}说项羽曰:“沛公居山东时,贪于财货,好美姬。今入关,财物无所取,妇女无所幸,此其志不在小。吾令人望其气,皆为龙虎,成五采,此天子气也。急击勿失!”
    
    楚左尹项伯者,项羽季父也,素善留侯张良\footnote{〔张良〕字子房,秦末汉初谋臣,韩国相国之孙,曾刺杀秦始皇。后投奔刘邦,被封为留侯。}。张良是时从沛公。项伯乃夜驰之沛公军,私见张良,具告以事,欲呼张良与俱去,曰:“毋从俱死也!”张良曰:“臣为韩王\footnote{〔韩王〕即韩成,韩国宗室,为韩国复国而随项梁反秦,项梁让张良辅佐韩成,韩成让张良辅佐刘邦。后被项羽猜忌杀害。}送沛公,沛公今事有急,亡去不义,不可不语。”
    
    良乃入,具告沛公。沛公大惊,曰:“为之奈何?”张良曰:“谁为大王为此计者?”曰:“鲰生说我曰:‘距关,毋内诸侯,秦地可尽王也’故听之。”良曰:“料大王士卒足以当项王乎?”沛公默然,曰:“固不如也,且为之奈何?”张良曰:“请往谓项伯,言沛公不敢背项王也。”沛公曰:“君安与项伯有故?”张良曰:“秦时与臣游,项伯杀人,臣活之;今事有急,故幸来告良。”沛公曰:“孰与君少长?”良曰:“长于臣。”沛公曰:“君为我呼入,吾得兄事之。”张良出,要项伯。项伯即入见沛公。沛公奉卮酒为寿,约为婚姻,曰:“吾入关,秋毫不敢有所近,籍吏民,封府库,而待将军。所以遣将守关者,备他盗之出入与非常也。日夜望将军至,岂敢反乎!愿伯具言臣之不敢倍德也。”项伯许诺,谓沛公曰:“旦日不可不蚤自来谢项王!”沛公曰:“诺。”于是项伯复夜去,至军中,具以沛公言报项王,因言曰:“沛公不先破关中,公岂敢入乎?今人有大功而击之,不义也。不如因善遇之。”项王许诺。
    
    沛公旦日从百余骑来见项王,至鸿门,谢曰:“臣与将军戮力而攻秦,将军战河北,臣战河南,然不自意能先入关破秦,得复见将军于此。今者有小人之言,令将军与臣有郤。”项王曰:“此沛公左司马曹无伤言之,不然,籍何以至此?”项王即日因留沛公与饮。项王、项伯东向坐;亚父南向坐——亚父者,范增也;沛公北向坐,张良西向侍。范增数目项王,举所佩玉玦以示之者三,项王默然不应。范增起,出,召项庄\footnote{〔项庄〕项羽族人。},谓曰:“君王为人不忍。若入前为寿,寿毕,请以剑舞,因击沛公于坐,杀之。不者,若属皆且为所虏!”庄则入为寿。寿毕,曰:“君王与沛公饮,军中无以为乐,请以剑舞。”项王曰:“诺。”项庄拔剑起舞,项伯亦拔剑起舞,常以身翼蔽沛公,庄不得击。
    
    于是张良至军门见樊哙\footnote{〔樊哙〕刘邦属下猛将。}。樊哙曰:“今日之事何如?”良曰:“甚急!今者项庄拔剑舞,其意常在沛公也。”哙曰:“此迫矣!臣请入,与之同命!”哙即带剑拥盾入军门。交戟之卫士欲止不内,樊哙侧其盾以撞,卫士仆地,哙遂入,披帷西向立,瞋目视项王,头发上指,目眦尽裂。项王按剑而跽曰:“客何为者?”张良曰:“沛公之参乘樊哙者也。”项王曰:“壮士!赐之卮酒。”则与斗卮酒。哙拜谢,起,立而饮之。项王曰:“赐之彘肩!”则与一生彘肩。樊哙覆其盾于地,加彘肩上,拔剑切而啖之。项王曰:“壮士!能复饮乎?”樊哙曰:“臣死且不避,卮酒安足辞!夫秦王有虎狼之心,杀人如不能举,刑人如恐不胜,天下皆叛之。怀王与诸将约曰:‘先破秦入咸阳者王之’,今沛公先破秦入咸阳,毫毛不敢有所近,封闭宫室,还军霸上,以待大王来。故遣将守关者,备他盗出入与非常也。劳苦而功高如此,未有封侯之赏,而听细说,欲诛有功之人。此亡秦之续耳,窃为大王不取也!”项王未有以应,曰:“坐!”樊哙从良坐。坐须臾,沛公起如厕,因招樊哙出。
    
    沛公已出,项王使都尉陈平召沛公。沛公曰:“今者出,未辞也,为之奈何?”樊哙曰:“大行不顾细谨,大礼不辞小让。如今人方为刀俎,我为鱼肉,何辞为?”于是遂去。乃令张良留谢。良问曰:“大王来何操?”曰:“我持白璧一双,欲献项王,玉斗一双,欲与亚父。会其怒,不敢献。公为我献之。”张良曰:“谨诺。”当是时,项王军在鸿门下,沛公军在霸上,相去四十里。沛公则置车骑,脱身独骑,与樊哙、夏侯婴、靳强、纪信\footnote{〔樊哙、夏侯婴、靳强、纪信〕都是刘邦的亲信。}等四人持剑盾步走,从郦山下,道芷阳\footnote{〔芷阳〕秦县名,在今陕西西安市东。}间行。沛公谓张良曰:“从此道至吾军,不过二十里耳。度我至军中,公乃入。”
    
    沛公已去,间至军中。张良入谢,曰:“沛公不胜桮杓,不能辞。谨使臣良奉白璧一双,再拜献大王足下;玉斗一双,再拜奉大将军足下。”项王曰:“沛公安在?”良曰:“闻大王有意督过之,脱身独去,已至军矣。”项王则受璧,置之坐上。亚父受玉斗,置之地,拔剑撞而破之,曰:“唉!竖子不足与谋!夺项王天下者,必沛公也!吾属今为之虏矣!”沛公至军,立诛杀曹无伤。
\end{normalsize}



\chapter{兰亭集序}

\begin{normalsize}
    
    永和九年\footnote{〔永和〕晋穆帝(司马聃)的年号(公元345至356年)。},岁在癸丑\footnote{〔癸丑〕干支纪年的方法。},暮春之初,会于会稽山阴\footnote{〔会稽山阴〕现在浙江绍兴越城区。}之兰亭,修禊事也\footnote{〔禊事〕古代习俗。每年农历三月上旬的巳日,人们群聚于水滨嬉戏洗濯,以祓除不祥和求福。}。群贤毕至,少长咸集。此地有崇山峻岭,茂林修竹;又有清流激湍,映带左右,引以为流觞曲水,列坐其次。虽无丝竹管弦之盛,一觞一咏,亦足以畅叙幽情。
    
    是日也,天朗气清,惠风和畅。仰观宇宙之大,俯察品类之盛,所以游目骋怀,足以极视听之娱,信可乐也。
    
    夫人之相与,俯仰一世,或取诸怀抱,悟言一室之内;或因寄所托,放浪形骸之外。虽趣舍万殊,静躁不同,当其欣于所遇,暂得于己,快然自足,不知老之将至。及其所之既倦,情随事迁,感慨系之矣。向之所欣,俯仰之间,已为陈迹,犹不能不以之兴怀。况修短随化,终期于尽。古人云:“死生亦大矣。”岂不痛哉!
    
    每览昔人兴感之由,若合一契,未尝不临文嗟悼,不能喻之于怀。固知一死生为虚诞,齐彭殇为妄作。后之视今,亦犹今之视昔。悲夫!故列叙时人,录其所述。虽世殊事异,所以兴怀,其致一也。后之览者,亦将有感于斯文。
\end{normalsize}


\newpage

\textbf{注解}:

\vspace{-1em}

\begin{itemize}
    \setlength\itemsep{-0.2em}
    \item〔不知老之将至〕《论语·述而》:“其为人也,发愤忘食,乐以忘忧,不知老之将至云尔。”
    \item〔固知一死生为虚诞,齐彭殇为妄作〕《庄子·德充符》:“老聃曰:‘胡不直使彼以死生为一条,以可不可为一贯者,解其桎梏,其可乎?’”《庄子·齐物论》:“莫寿于殇子,而彭祖为夭。”这里王羲之借老庄之言指代当时喜欢清谈玄学,崇尚黄老之说,以隐逸避世为荣,以出世奋斗为耻的风气,认为这种思想是虚妄的。
\end{itemize}

\chapter{廉颇蔺相如列传}

\begin{normalsize}
    
    廉颇者,赵之良将也。赵惠文王十六年\footnote{〔赵惠文王十六年〕公元前283年。},廉颇为赵将伐齐,大破之,取阳晋\footnote{〔阳晋〕齐国城邑,现在山东菏泽西北。},拜为上卿\footnote{〔上卿〕战国时期诸侯国大臣中最高的官位。},以勇气闻于诸侯。蔺相如者,赵人也,为赵宦者令缪贤舍人。
    
    赵惠文王时,得楚和氏璧\footnote{〔和氏璧〕战国时著名的玉璧,是楚人卞和发现的,故名和氏璧。}。秦昭王\footnote{〔秦昭王〕即秦昭襄王(公元前306年至公元前251年在位)。}闻之,使人遗赵王书,愿以十五城请易璧。赵王与大将军廉颇诸大臣谋:欲予秦,秦城恐不可得,徒见欺;欲勿予,即患秦兵之来。计未定,求人可使报秦者,未得。宦者令缪贤曰:“臣舍人蔺相如可使。”王问:“何以知之?”对曰:“臣尝有罪,窃计欲亡走燕,臣舍人相如止臣,曰:‘君何以知燕王?”臣语曰:‘臣尝从大王与燕王会境上,燕王私握臣手,曰“愿结友”,以此知之,故欲往。”相如谓臣曰:‘夫赵强而燕弱,而君幸于赵王,故燕王欲结于君。今君乃亡赵走燕,燕畏赵,其势必不敢留君,而束君归赵矣。君不如肉袒伏斧质请罪,则幸得脱矣。”臣从其计,大王亦幸赦臣。臣窃以为其人勇士,有智谋,宜可使。”
    
    于是王召见,问蔺相如曰:“秦王以十五城请易寡人之璧,可予不?”相如曰:“秦强而赵弱,不可不许。”王曰:“取吾璧,不予我城,奈何?”相如曰:“秦以城求璧而赵不许,曲在赵。赵予璧而秦不予赵城,曲在秦。均之二策,宁许以负秦曲。”王曰:“谁可使者?”相如曰:“王必无人,臣愿奉璧往使。城入赵而璧留秦;城不入,臣请完璧归赵。”赵王于是遂遣相如奉璧西入秦。
    
    秦王坐章台见相如,相如奉璧奏秦王。秦王大喜,传以示美人及左右,左右皆呼万岁。相如视秦王无意偿赵城,乃前曰:“璧有瑕,请指示王。”王授璧,相如因持璧却立,倚柱,怒发上冲冠,谓秦王曰:“大王欲得璧,使人发书至赵王,赵王悉召群臣议,皆曰‘秦贪,负其强,以空言求璧,偿城恐不可得”,议不欲予秦璧。臣以为布衣之交尚不相欺,况大国乎!且以一璧之故逆强秦之欢,不可。于是赵王乃斋戒五日,使臣奉璧,拜送书于庭。何者?严大国之威以修敬也。今臣至,大王见臣列观,礼节甚倨;得璧,传之美人,以戏弄臣。臣观大王无意偿赵王城邑,故臣复取璧。大王必欲急臣,臣头今与璧俱碎于柱矣!”
    
    相如持其璧睨柱,欲以击柱。秦王恐其破璧,乃辞谢固请,召有司案图,指从此以往十五都予赵。相如度秦王特以诈详为予赵城,实不可得,乃谓秦王曰:“和氏璧,天下所共传宝也,赵王恐,不敢不献。赵王送璧时,斋戒五日,今大王亦宜斋戒五日,设九宾于廷,臣乃敢上璧。”秦王度之,终不可强夺,遂许斋五日,舍相如广成传。相如度秦王虽斋,决负约不偿城,乃使其从者衣褐,怀其璧,从径道亡,归璧于赵。
    
    秦王斋五日后,乃设九宾礼于廷,引赵使者蔺相如。相如至,谓秦王曰:“秦自缪公\footnote{〔缪公〕即秦穆公(前659年至前621年在位)。}以来二十馀君,未尝有坚明约束者也。臣诚恐见欺于王而负赵,故令人持璧归,间至赵矣。且秦强而赵弱,大王遣一介之使至赵,赵立奉璧来。今以秦之强而先割十五都予赵,赵岂敢留璧而得罪于大王乎?臣知欺大王之罪当诛,臣请就汤镬,唯大王与群臣孰计议之。”秦王与群臣相视而嘻。左右或欲引相如去,秦王因曰:“今杀相如,终不能得璧也,而绝秦赵之欢,不如因而厚遇之,使归赵,赵王岂以一璧之故欺秦邪!”卒廷见相如,毕礼而归之。
    
    相如既归,赵王以为贤大夫使不辱于诸侯,拜相如为上大夫。秦亦不以城予赵,赵亦终不予秦璧。
    
    其后秦伐赵,拔石城\footnote{〔石城〕即石邑,现在河北石家庄一带。}。明年,复攻赵,杀二万人。
    
    秦王使使者告赵王,欲与王为好会于西河外渑池\footnote{〔渑池〕现在河南渑池县。}。赵王畏秦,欲毋行。廉颇、蔺相如计曰:“王不行,示赵弱且怯也。”赵王遂行,相如从。廉颇送至境,与王诀曰:“王行,度道里会遇之礼毕,还,不过三十日。三十日不还,则请立太子为王,以绝秦望。”王许之。
    
    遂与秦王会渑池。秦王饮酒酣,曰:“寡人窃闻赵王好音,请奏瑟。”赵王鼓瑟。秦御史前书曰“某年月日,秦王与赵王会饮,令赵王鼓瑟”。蔺相如前曰:“赵王窃闻秦王善为秦声,请奏盆缻秦王,以相娱乐。”秦王怒,不许。于是相如前进缻,因跪请秦王。秦王不肯击缻。相如曰:“五步之内,相如请得以颈血溅大王矣!”左右欲刃相如,相如张目叱之,左右皆靡。于是秦王不怿,为一击缻。相如顾召赵御史书曰“某年月日,秦王为赵王击缻”。秦之群臣曰:“请以赵十五城为秦王寿。”蔺相如亦曰:“请以秦之咸阳为赵王寿。”秦王竟酒,终不能加胜于赵。赵亦盛设兵以待秦,秦不敢动。
    
    既罢归国,以相如功大,拜为上卿,位在廉颇之右。
    
    廉颇曰:“我为赵将,有攻城野战之大功,而蔺相如徒以口舌为劳,而位居我上,且相如素贱人,吾羞,不忍为之下。”宣言曰:“我见相如,必辱之。”相如闻,不肯与会。相如每朝时,常称病,不欲与廉颇争列。已而相如出,望见廉颇,相如引车避匿。于是舍人相与谏曰:“臣所以去亲戚而事君者,徒慕君之高义也。今君与廉颇同列,廉君宣恶言而君畏匿之,恐惧殊甚,且庸人尚羞之,况于将相乎!臣等不肖,请辞去。”
    
    蔺相如固止之,曰:“公之视廉将军孰与秦王?”曰:“不若也。”相如曰:“夫以秦王之威,而相如廷叱之,辱其群臣,相如虽驽,独畏廉将军哉?顾吾念之,强秦之所以不敢加兵于赵者,徒以吾两人在也。今两虎共斗,其势不俱生。吾所以为此者,以先国家之急而后私仇也。”廉颇闻之,肉袒负荆,因宾客至蔺相如门谢罪。曰:“鄙贱之人,不知将军宽之至此也!”卒相与欢,为刎颈之交。
\end{normalsize}



\chapter{伶官传序}

\begin{normalsize}
    
    呜呼!盛衰之理,虽曰天命,岂非人事哉!原庄宗之所以得天下,与其所以失之者,可以知之矣。
    
    世言晋王之将终也,以三矢赐庄宗而告之曰:“梁,吾仇也;燕王,吾所立;契丹与吾约为兄弟;而皆背晋以归梁。此三者,吾遗恨也。与尔三矢,尔其无忘乃父之志!”庄宗受而藏之于庙。其后用兵,则遣从事以一少牢告庙,请其矢,盛以锦囊,负而前驱,及凯旋而纳之。
    
    方其系燕父子以组,函梁君臣之首,入于太庙,还矢先王,而告以成功,其意气之盛,可谓壮哉!及仇雠已灭,天下已定,一夫夜呼,乱者四应,仓皇东出,未及见贼而士卒离散,君臣相顾,不知所归。至于誓天断发,泣下沾襟,何其衰也!岂得之难而失之易欤?抑本其成败之迹,而皆自于人欤?
    
    《书》曰:“满招损,谦得益。”忧劳可以兴国,逸豫可以亡身,自然之理也。故方其盛也,举天下之豪杰,莫能与之争;及其衰也,数十伶人困之,而身死国灭,为天下笑。夫祸患常积于忽微,而智勇多困于所溺,岂独伶人也哉!作《伶官传》。
\end{normalsize}



\chapter{秋水}

\begin{normalsize}
    
    秋水时至,百川灌河。泾流之大,两涘渚崖之间,不辩牛马。于是焉河伯\footnote{〔河伯〕黄河的河神,这里是虚构的对话角色。}欣然自喜,以天下之美为尽在己。顺流而东行,至于北海,东面而视,不见水端。于是焉河伯始旋其面目,望洋向若而叹曰:“野语有之曰:‘闻道百,以为莫己若者’,我之谓也。且夫我尝闻少仲尼\footnote{〔仲尼〕指孔子。}之闻,而轻伯夷\footnote{〔伯夷〕商末孤竹君长子。武王灭商后,耻食周粟,饿死于首阳山。}之义者,始吾弗信,今我睹子之难穷也,吾非至于子之门,则殆矣,吾长见笑于大方之家。”
    
    北海若\footnote{〔北海若〕北海的海神,这里是虚构的对话角色。}曰:“井蛙不可以语于海者,拘于虚也;夏虫不可以语于冰者,笃于时也;曲士不可以语于道者,束于教也。今尔出于崖涘,观于大海,乃知尔丑,尔将可与语大理矣。天下之水,莫大于海。万川归之,不知何时止而不盈;尾闾泄之,不知何时已而不虚。春秋不变,水旱不知。此其过江河之流,不可为量数。而吾未尝以此自多者,自以比形于天地,而受气于阴阳,吾在天地之间,犹小石小木之在大山也。方存乎见少,又奚以自多!计四海之在天地之间也,不似礨空之在大泽乎?计中国之在海内不似稊米之在大仓乎?号物之数谓之万,人处一焉。人卒九州,谷食之所生,舟车之所通,人处一焉。此其比万物也,不似豪末之在于马体乎?五帝之所连,三王之所争,仁人之所忧,任士之所劳,尽此矣!伯夷辞之以为名,仲尼语之以为博。此其自多也,不似尔向之自多于水乎?”
\end{normalsize}



\chapter{谏太宗十思疏}

\begin{normalsize}
    
    臣闻求木之长者,必固其根本;欲流之远者,必浚其泉源;思国之安者,必积其德义。源不深而望流之远,根不固而求木之长,德不厚而思国之安,臣虽下愚,知其不可,而况于明哲乎?人君当神器之重,居域中之大,将崇极天之峻,永保无疆之休。不念居安思危,戒奢以俭,德不处其厚,情不胜其欲,斯亦伐根以求木茂,塞源而欲流长也。
    
    凡百元首,承天景命,莫不殷忧而道著,功成而德衰,有善始者实繁,能克终者盖寡。岂其取之易守之难乎?昔取之而有余,今守之而不足,何也?夫在殷忧必竭诚以待下,既得志则纵情以傲物。竭诚则吴、越为一体\footnote{〔吴越〕春秋时南方的国家,长期相互征伐,最终以越国吞并吴国结束。},傲物则骨肉为行路。虽董之以严刑,震之以威怒,终苟免而不怀仁,貌恭而不心服。怨不在大,可畏惟人;载舟覆舟\footnote{〔吴越〕春秋时南方的国家,长期相互征伐,最终以越国吞并吴国结束。},所宜深慎。奔车朽索,其可忽乎?
    
    君人者,诚能见可欲,则思知足以自戒;将有作,则思知止以安人;念高危,则思谦冲而自牧;惧满溢,则思江海下百川;乐盘游,则思三驱\footnote{〔吴越〕春秋时南方的国家,长期相互征伐,最终以越国吞并吴国结束。}以为度;忧懈怠,则思慎始而敬终;虑壅蔽,则思虚心以纳下;惧谗邪,则思正身以黜恶;恩所加,则思无因喜以谬赏;罚所及,则思无以怒而滥刑。总此十思,宏兹九德,简能而任之,择善而从之,则智者尽其谋,勇者竭其力,仁者播其惠,信者效其忠。文武争驰,君臣无事,可以尽豫游之乐,可以养松乔之寿\footnote{〔吴越〕春秋时南方的国家,长期相互征伐,最终以越国吞并吴国结束。},鸣琴垂拱,不言而化。何必劳神苦思,代下司职,役聪明之耳目,亏无为之大道哉?
\end{normalsize}


\newpage

\textbf{注解}:

\vspace{-1em}

\begin{itemize}
    \setlength\itemsep{-0.2em}
    \item〔载舟覆舟〕这个典故本来出自《荀子·王制》:“君者,舟也;庶人者,水也。水则载舟,水则覆舟。”李世民在《自鉴录》中也说:“舟所以比人君,水所以比黎庶。水能载舟,亦能覆舟。”因此魏徵也引用这个典故。
\end{itemize}

\chapter{五人墓碑记}

\begin{normalsize}
    
    五人者,盖当蓼洲周公\footnote{〔蓼洲周公〕指周顺昌,东林党人,万历年间进士,曾任吏部郎中,因对抗魏忠贤,被诬罪逮捕。}之被逮,激于义而死焉者也。至于今,郡之贤士大夫请于当道,即除魏阉\footnote{〔魏阉〕指太监魏忠贤。}废祠之址以葬之;且立石于其墓之门,以旌其所为。呜呼,亦盛矣哉!
    
    夫五人之死,去今之墓而葬焉,其为时止十有一月耳。夫十有一月之中,凡富贵之子,慷慨得志之徒,其疾病而死,死而湮没不足道者,亦已众矣;况草野之无闻者欤?独五人之皦皦,何也?
    
    予犹记周公之被逮,在丙寅\footnote{〔丙寅〕即天启六年(公元1626年)。}三月之望。吾社\footnote{〔吾社〕指应社,天启四年由张溥、张采等江南士人发起成立,成员多为东林党人,相互呼应。}之行为士先者,为之声义,敛赀财以送其行,哭声震动天地。缇骑\footnote{〔丙寅〕即天启六年(公元1626年)。}按剑而前,问:“谁为哀者?”众不能堪,抶而仆之。是时以大中丞\footnote{〔大中丞〕官职名,指当时苏州巡抚毛一鹭,依附魏忠贤。}抚吴者为魏之私人,公之逮所由使也;吴之民方痛心焉,于是乘其厉声以呵,则噪而相逐。中丞匿于溷藩以免。既而以吴民之乱请于朝,按诛五人,曰颜佩韦、杨念如、马杰、沈扬、周文元\footnote{〔颜佩韦……〕五人的名字。周文元是周顺昌的轿夫,颜佩韦、杨念如、马杰、沈扬是一般市民。},即今之傫然在墓者也。
    
    然五人之当刑也,意气扬扬,呼中丞之名而詈之,谈笑以死。断头置城上,颜色不少变。有贤士大夫发五十金,买五人之头而函之,卒与尸合。故今之墓中全乎为五人也。
    
    嗟乎!大阉之乱,缙绅而能不易其志者,四海之大,有几人欤?而五人生于编伍之间,素不闻诗书之训,激昂大义,蹈死不顾,亦曷故哉?且矫诏纷出,钩党之捕遍于天下,卒以吾郡之发愤一击,不敢复有株治;大阉亦逡巡畏义,非常之谋难于猝发,待圣人之出而投缳道路,不可谓非五人之力也。
    
    由是观之,则今之高爵显位,一旦抵罪,或脱身以逃,不能容于远近,而又有剪发杜门,佯狂不知所之者,其辱人贱行,视五人之死,轻重固何如哉?是以蓼洲周公忠义暴于朝廷,赠谥褒美,显荣于身后;而五人亦得以加其土封,列其姓名于大堤之上,凡四方之士无不有过而拜且泣者,斯固百世之遇也。不然,令五人者保其首领,以老于户牖之下,则尽其天年,人皆得以隶使之,安能屈豪杰之流,扼腕墓道,发其志士之悲哉?故余与同社诸君子,哀斯墓之徒有其石也,而为之记,亦以明死生之大,匹夫之有重于社稷也。
    
    贤士大夫者,冏卿因之吴公\footnote{〔吴公〕吴默,字因之,“冏卿”指官职:太仆卿。},太史文起文公\footnote{〔文公〕文震孟,字文起,“太史”指官职:翰林院修撰。}、孟长姚公\footnote{〔姚公〕姚希孟,字孟长。}也。
\end{normalsize}


\newpage

\textbf{注解}:

\vspace{-1em}

\begin{itemize}
    \setlength\itemsep{-0.2em}
    \item〔是时以大中丞抚吴者为魏之私人,公之逮所由使也;吴之民方痛心焉,于是乘其厉声以呵,则噪而相逐。中丞匿于溷藩以免。〕在张溥笔下,毛一鹭成为市民首要攻击的对象,毛一鹭血腥镇压这次民变。然而,民变组织者文震孟之子文秉提到毛一鹭,却语含感激:“邵辅忠贻书毛抚:‘急以文某、姚某入告,少宰虚席以待。’盖指文肃、姚文毅两公也。毛抚勿为动,止擒颜佩韦、杨念如、周文元、马杰、沈扬五人,具狱斩之。”《熹宗实录》载毛一鹭奏章:“方周顺昌奉旨被逮,县官往即就系,当令府县力促开读,而官旗不应逗遛需索,订期十八,致生远迩之心,此变之所由肇也。苏郡法纪陵夷,已非朝夕。臣等德不足以绥民,威不足以肃众,抑何辞于溺职之罪。除一面将犯官周顺昌交发官旗即日起解至京,伏候圣明处分外,仍次第擒缉倡乱渠魁,另行正法,以重国典。”可见,毛一鹭采取了息事宁人的态度,有意放过应社乡绅士子,只抓捕了少数庶民,但反被张溥写成“违背民意”的证据。温睿临亦载:“巡抚毛一鹭惧祸,根究乱民,杀五人以谢奄。”而各家史书,均未见苏民追打地方官员的记载。张溥攻击毛一鹭,乃是出于党争思维:魏忠贤虽然已经倒台,但魏的地方余党尚在,不赶尽杀绝誓不罢休。这一大清洗思想并非孤立。魏忠贤倒台后,崇祯一直要求阁臣清洗阉党,但阁臣却努力缩小波及面,以免政治动荡,君臣之间数度往复。 “初,逆珰死后,上欲因台谏言定逆案。大学士韩爌、钱龙锡不欲广搜树怨,仅列四五十人以请。上不悦,再令尽列以闻。”崇祯二年正月,崇祯再次催问,阁臣仍旧拖延,于是崇祯“发原奏及前红本未入各官六十九人,各令酌定,于是案列甚广,几无遗矣。”三月,“廷臣上《钦定逆案》,诏刊布中外。共二百五十八人”。《碑记》创作之时,政坛动荡尚未波及毛一鹭,故张溥不惜扭曲事实,引火烧之。《碑记》是一个强烈的政治信号,标志着东林后劲应社(复社前身)正式加入党争营垒。
    \item〔然五人之当刑也,意气扬扬,呼中丞之名而詈之,谈笑以死。〕《五人墓碑记》还发掘出一个新主题:将市民阶层拉入党争队伍,使之成为党争急先锋,以博取舆论支持,扩大政治影响。历史上的党争都局限在朝廷之上,晚明党争即是内廷与外朝之间的纷争。而《碑记》中,下层民众被裹挟进入党争队伍。在张溥笔下,市井民众被士人化,被引入东林队伍中,成为反阉急先锋。然而这也是一个虚构情节。五人临刑,并无痛骂之事。《明季北略》载:“斩五人于阊门吊桥,时颜佩韦等四人俱不畏,独周文元本舆夫,大哭。”《明史》载:“佩韦等皆市人,文元则顺昌舆隶也,论大辟。临刑,五人延颈就刃,语寇慎曰:‘公好官,知我等好义,非乱也。’监司张孝流涕而斩之。”苏州所谓“民变”,只是一些市民被应社煽动,参与党争、誓死反阉之说显属附会。把东林党与民心捆绑到一起,是一个复杂的历史过程。崇祯毁《三朝要典》,清洗阉党,给世人尘埃落定的印象。而阉党扶植的南明小朝廷又非常短命,更加剧了世人的鄙薄之心。加上阉党“所仇怨多在江南”,江南士人与东林、复社有着千丝万缕的联系,他们著文讲学,影响甚大。东林及其后劲复社遂获得广泛的民间支持。张溥把应社成员打扮成组织领导者,将暴动市民解读成听从应社号召、积极投身党争的急先锋,又将毛一鹭扭曲成民变的愤怒对象,从而突出应社巨大的社会号召能量,并将“吾社”继东林而起的政治含义固定在世人头脑中。
\end{itemize}

\chapter{指南录后序}

\begin{normalsize}
    
    德祐二年\footnote{〔德祐〕宋恭帝赵㬎年号(公元1275至1276年)。}正月十九日,予除右丞相,兼枢密使,都督诸路军马。时北兵\footnote{〔北兵〕即元兵。下文以“北”指元政权。}已迫修门\footnote{〔修门〕《楚辞·招魂》:“魂兮归来,入修门些。”本指楚国郢都城门,这里代指南宋都城临安的城门。}外,战、守、迁皆不及施。缙绅、大夫、士萃于左丞相府\footnote{〔左丞相〕指吴坚。德祐二年正月,吴坚升任左丞相兼枢密使,受谢太后命,与贾余庆等先赴元营议降,后为祈请使,赴元大都呈降表,交宋玺。后被羁留大都,当年病故。},莫知计所出。会使辙交驰,北邀当国者相见,众谓予一行,为可以纾祸。国事至此,予不得爱身,意北亦尚可以口舌动也。初,奉使往来,无留北者,予更欲一觇北,归而求救国之策。于是辞相印不拜,翌日,以资政殿学土行。
    
    初至北营,抗词慷慨,上下颇惊动,北亦未敢遽轻吾国。不幸吕师孟\footnote{〔吕师孟〕兵部尚书,叛将吕文焕之侄。}构恶于前,贾余庆\footnote{〔贾余庆〕官同签书枢密院事,知临安府,后代文天祥为右丞相。}献谄于后,予羁縻不得还,国事遂不可收拾。予自度不得脱,则直前诟虏帅失信,数吕师孟叔侄为逆,但欲求死,不复顾利害。北虽貌敬,实则愤怒,二贵酋名曰“馆伴”,夜则以兵围所寓舍,而予不得归矣。未几,贾余庆等以祈请使诣北,北驱予并往,而不在使者之目。予分当引决,然而隐忍以行,昔人云:将以有为也。
    
    至京口\footnote{〔京口〕江苏镇江市,当时为元军占领。},得间奔真州\footnote{〔真州〕今江苏仪征县,当时仍为宋军把守。},即具以北虚实告东西二阃\footnote{〔东西二阃〕指宋淮东制置使李庭芝和淮西制置使夏贵。},约以连兵大举。中兴机会,庶几在此。留二日,维扬帅下逐客之令。不得已,变姓名,诡踪迹,草行露宿,日与北骑相出没于长淮间。穷饿无聊,追购又急;天高地迥,号呼靡及。已而得舟,避渚州\footnote{〔渚州〕指长江中的沙州,当时已被元兵占领。},出北海\footnote{〔北海〕指淮海。},然后渡扬子江,入苏州洋\footnote{〔苏州洋〕今上海市附近的海域。},展转四明、天台\footnote{〔四明、天台〕四明:现在浙江宁波市。天台:现在浙江天台县。},以至于永嘉\footnote{〔永嘉〕现在浙江温州市。}。
    
    呜呼!予之及于死者,不知其几矣。诋大酋,当死;骂逆贼,当死;与贵酋处二十日,争曲直,屡当死;去京口,挟匕首以备不测,几自刭死;经北舰十余里,为巡船所物色,几从鱼腹死;真州逐之城门外,几彷徨死;如扬州,过瓜洲\footnote{〔瓜洲〕扬州南长江中的沙洲。}扬子桥,竟使遇哨,无不死;扬州城下,进退不由,殆例送死;坐桂公塘\footnote{〔桂公塘〕地名,在扬州城外。}土围中,骑数千过其门,几落贼手死;贾家庄\footnote{〔贾家庄〕地名,在扬州城北。}几为巡徼所陵迫死;夜趋高邮,迷失道,几陷死;质明,避哨竹林中,逻者数十骑,几无所逃死;至高邮\footnote{〔高邮〕现在江苏高邮县,也称“高沙”。},制府\footnote{〔制府〕指淮东制置使官府。}檄下,几以捕系死;行城子河\footnote{〔城子河〕在高邮县境内。},出入乱尸中,舟与哨相后先,几邂逅死;至海陵\footnote{〔海陵〕现在江苏泰州市。},如高沙,常恐无辜死;道海安、如皋\footnote{〔海安、如皋〕县名,均属江苏。},凡三百里,北与寇往来其间,无日而非可死;至通州\footnote{〔通州〕现在江苏省南通市。},几以不纳死;以小舟涉鲸波,出无可奈何,而死固付之度外矣!呜呼,死生昼夜事也。死而死矣,而境界危恶,层见错出,非人世所堪。痛定思痛,痛何如哉!
    
    予在患难中,间以诗记所遭。今存其本,不忍废,道中手自抄录。使北营,留北关外\footnote{〔北关外〕指临安城北高亭山。},为一卷;发北关外,历吴门、毘陵\footnote{〔吴门、毘陵〕吴门:今江苏苏州市。毘陵:今江苏常州市。},渡瓜洲,复还京口,为一卷;脱京口,趋真州、扬州、高邮、泰州、通州,为一卷;自海道至永嘉,来三山\footnote{〔三山〕即今福建福州市。因城中有闽山、越王山、九仙山,故名“三山”。},为一卷。将藏之于家,使来者读之,悲予志焉。
    
    呜呼!予之生也幸,而幸生也何为?所求乎为臣,主辱臣死有馀僇;所求乎为子,以父母之遗体行殆而死,有馀责。将请罪于君,君不许;请罪于母,母不许。请罪于先人之墓,生无以救国难,死犹为厉鬼以击贼,义也。赖天之灵,宗庙之福,修我戈矛,从王于师,以为前驱;雪九庙\footnote{〔九庙〕皇帝祭祀祖先共有九庙,这里以九庙指代国家。}之耻,复高祖\footnote{〔高祖〕指宋太祖赵匡胤。}之业;所谓誓不与贼俱生,所谓鞠躬尽力,死而后已,亦义也。嗟夫!若予者,将无往而不得死所矣。向也使予委骨于草莽,予虽浩然无所愧怍,然微以自文于君亲,君亲其谓予何!诚不自意,返吾衣冠,重见日月,使旦夕得正丘首,复何憾哉!复何憾哉!
    
    是年夏五\footnote{〔夏五〕夏天五月。},改元景炎\footnote{〔改元景炎〕由于宋恭帝被元军掳去,德祐二年五月,文天祥等人在福州立赵昰为帝,改元景炎。}。庐陵\footnote{〔庐陵〕现在江西吉安市。}文天祥自序其诗,名曰《指南录》。
\end{normalsize}



\chapter{烛之武退秦师}

\begin{normalsize}
    
    晋侯\footnote{〔晋侯〕即晋文公重耳,在秦穆公支持下夺得君位,春秋第二位霸主。}、秦伯\footnote{〔秦伯〕即秦穆公,扶持在外流亡的晋公子重耳继承君位,春秋第三位霸主。}围郑,以其无礼于晋,且贰于楚也。晋军函陵\footnote{〔函陵〕现在河南新郑市北。},秦军氾南\footnote{〔氾南〕氾水之南。氾水:流经今山东菏泽市定陶区南、曹县北,汇入古菏泽。}。
    
    佚之狐\footnote{〔佚之狐〕郑国大夫。}言于郑伯\footnote{〔郑伯〕即郑文公,晋公子重耳流亡时拒绝接待,于是重耳即位后与秦穆公讨伐郑国。}曰:“国危矣,若使烛之武见秦君,师必退。”公从之。辞曰:“臣之壮也,犹不如人;今老矣,无能为也已。”公曰:“吾不能早用子,今急而求子,是寡人之过也。然郑亡,子亦有不利焉。”许之。
    
    夜缒而出,见秦伯,曰:“秦、晋围郑,郑既知亡矣。若亡郑而有益于君,敢以烦执事。越国以鄙远,君知其难也。焉用亡郑以陪邻?邻之厚,君之薄也。若舍郑以为东道主,行李之往来,共其乏困,君亦无所害。且君尝为晋君赐矣,许君焦、瑕\footnote{〔焦、瑕〕黄河北岸的两个城邑,现在河南焦作一带,也有说在山西陕县一带。},朝济而夕设版焉,君之所知也。夫晋,何厌之有?既东封郑,又欲肆其西封,若不阙秦,将焉取之?阙秦以利晋,唯君图之。”秦伯说,与郑人盟。使杞子、逢孙、杨孙\footnote{〔杞子、逢孙、杨孙〕秦国大夫。两年后杞子密谋反郑,邀秦军伐郑,里应外合。事情败露后,杞子逃往齐国,逢孙、杨孙逃往宋国。}戍之,乃还。
    
    子犯\footnote{〔子犯〕狐偃,字子犯,晋国大夫,晋文公的舅舅。}请击之,公曰:“不可。微夫人之力不及此。因人之力而敝之,不仁;失其所与,不知;以乱易整,不武。吾其还也。”亦去之。
\end{normalsize}



\chapter{子鱼论战}

\begin{normalsize}
    
    
\end{normalsize}



\chapter{孟子四则}

\begin{normalsize}
    
    孟子曰:“不仁哉,梁惠王\footnote{〔梁惠王〕战国时魏国第三任国君。}也!仁者以其所爱及其所不爱,不仁者以其所不爱及其所爱。”公孙丑\footnote{〔公孙丑〕孟子弟子,齐国人,是《孟子》的主要作者之一。}问曰:“何谓也?”“梁惠王以土地之故,糜烂其民而战之,大败,将复之,恐不能胜,故驱其所爱子弟以殉之,是之谓以其所不爱及其所爱也。”
    
    孟子曰:“以力假仁者霸,霸必有大国,以德行仁者王,王不待大。汤以七十里,文王以百里。以力服人者,非心服也,力不赡也;以德服人者,中心悦而诚服也,如七十子之服孔子也。《诗》云:‘自西自东,自南自北,无思不服。’此之谓也。”
    
    孟子曰:“天时不如地利,地利不如人和。三里之城,七里之郭,环而攻之而不胜。夫环而攻之,必有得天时者矣;然而不胜者,是天时不如地利也。城非不高也,池非不深也,兵革非不坚利也,米粟非不多也;委而去之,是地利不如人和也。故曰:域民不以封疆之界,固国不以山溪之险,威天下不以兵革之利。得道者多助,失道者寡助。寡助之至,亲戚畔之;多助之至,天下顺之。以天下之所顺,攻亲戚之所畔;故君子有不战,战必胜矣。”
\end{normalsize}



\chapter{黄花岗烈士事略序}

\begin{normalsize}
    
    满清末造,革命党人历艰难险巇,以坚毅不扰之精神,与民贼相搏,踬踣者屡。死事之惨,以辛亥三月二十九日\footnote{〔辛亥〕即公元1911年。当年4月27日(农历3月29日)国民党人在广州发起黄花岗起义,不幸失败。}围攻两广督署之役为最。吾党\footnote{〔吾党〕指国民党。}菁华付之一炬,其损失可谓大矣!然是役也,碧血横飞,浩气四塞,草木为之含悲,风云因而变色。全国久蛰之人心,乃大兴奋。怨愤所积,如怒涛排壑,不可遏抑,不半载而武昌之革命\footnote{〔武昌之革命〕指1911年10月10日湖北武昌起义。}以成。则斯役之价值,直可惊天地,泣鬼神,与武昌革命之役并寿。
    
    顾自民国肇造,变乱纷乘,黄花岗上一抔土,犹湮没于荒烟蔓草间。延至七年,始有墓碣之建修;十年,始有事略之编纂。而七十二烈士者,又或有记载而语焉不详,或仅存姓名而无事迹,甚者且姓名不可考,如史载田横事\footnote{〔田横〕指汉初田横五百士的故事。},虽以史迁\footnote{〔史迁〕指司马迁,《史记》的作者。}之善传游侠,亦不能为五百人立传,滋可痛矣。
    
    邹君海滨\footnote{〔邹君海滨〕邹鲁(1885年2月20日—1954年2月13日),字海滨,中国国民党和中华民国元老,曾任国立中山大学校长。},以所辑《黄花岗烈士事略》丐序于余。时余方以讨贼\footnote{〔讨贼〕指讨伐北洋军阀。1921年粤军西征讨桂成功。12月,孙中山到桂林组建陆海军大元帅大本营,整军准备北伐。}督师桂林。环顾国内,贼氛方炽,杌靰之象,视清季有加,而余三十年前所主唱之三民主义、五权宪法\footnote{〔三民主义……〕指孙中山在1905年同盟会成立时正式提出的“三民主义”思想,是为旧三民主义。五权宪法指基于三民主义创立的五院制宪制理论。},为诸先烈所不惜牺牲生命以争之者,其不获实行也如故。则余此行所负之责任,尤倍重于三十年前。倘国人皆以先烈之牺牲精神为国奋斗,助余完成此重大之责任,实现吾人理想之真正中华民国,则此一部开国血史,可传而不朽。否则不能继述先烈遗志且光大之,而徒感慨于其遗事,斯诚后死者之羞也!
    
    余为斯序,既痛逝者,并以为国人之读兹编者勖。
\end{normalsize}



\chapter{季氏将伐颛臾}

\begin{normalsize}
    
    季氏\footnote{〔季氏〕季康子,季孙氏,名肥。春秋时鲁国卿大夫,把持朝政。}将伐颛臾\footnote{〔颛臾〕鲁国的属国,故城在今山东费县西北。}。冉有\footnote{〔冉有〕孔子的弟子,名求,字子有。当时是季康子的家臣。}、季路\footnote{〔季路〕孔子的弟子,姓仲,名由,字子路。当时是季康子的家臣。}见于孔子曰:“季氏将有事于颛臾。”孔子曰:“求!无乃尔是过与?夫颛臾,昔者先王\footnote{〔先王〕指周之先王。}以为东蒙主\footnote{〔东蒙主〕指受封于东蒙。东蒙,山名,及蒙山,在今山东蒙阴南。},且在邦域之中矣,是社稷之臣也。何以伐为?”
    
    冉有曰:“夫子\footnote{〔夫子〕即季康子。}欲之,吾二臣者皆不欲也。”孔子曰:“求!周任\footnote{〔周任〕上古的史官。}有言曰:‘陈力就列,不能者止’,危而不持,颠而不扶,则将焉用彼相矣?且尔言过矣。虎兕出于柙,龟玉毁于椟中,是谁之过与?”
    
    冉有曰:“今夫颛臾,固而近于费。今不取,后世必为子孙忧。”孔子曰:“求!君子疾夫舍曰欲之而必为之辞。丘也闻有国有家者,不患寡而患不均,不患贫而患不安。盖均无贫,和无寡,安无倾。夫如是,故远人不服,则修文德以来之。既来之,则安之。今由与求也,相夫子,远人不服、而不能来也;邦分崩离析、而不能守也:而谋动干戈于邦内。吾恐季孙之忧,不在颛臾,而在萧墙\footnote{〔萧墙〕国君宫门内迎门的小墙,又叫做屏。因古时臣子朝见国君,走到此必肃然起敬,故称“萧墙”。这里借指宫廷。}之内也。”
\end{normalsize}



\chapter{秋声赋}

\begin{normalsize}
    
    欧阳子方夜读书,闻有声自西南来者,悚然而听之,曰:“异哉!”初淅沥以萧飒,忽奔腾而砰湃,如波涛夜惊,风雨骤至。其触于物也,鏦鏦铮铮,金铁皆鸣;又如赴敌之兵,衔枚\footnote{〔衔枚〕古时行军或袭击敌军时,让士兵衔枚以防出声。枚,形似竹筷,衔于口中,两端有带,系于脖上。}疾走,不闻号令,但闻人马之行声。予谓童子:“此何声也?汝出视之。”童子曰:“星月皎洁,明河\footnote{〔明河〕天河。}在天,四无人声,声在树间。”
    
    予曰:“噫嘻悲哉!此秋声也,胡为而来哉?盖夫秋之为状也:其色惨淡,烟霏云敛;其容清明,天高日晶;其气栗冽,砭人肌骨;其意萧条,山川寂寥。故其为声也,凄凄切切,呼号愤发。丰草绿缛而争茂,佳木葱茏而可悦;草拂之而色变,木遭之而叶脱。其所以摧败零落者,乃其一气之余烈。夫秋,刑官\footnote{〔刑官〕执掌刑狱的官。《周礼》把官职与天、地、春、夏、秋、冬相配,称为六官。秋天肃杀万物,所以司寇为秋官,执掌刑法,称刑官。}也,于时为阴;又兵象也,于行用金,是谓天地之义气,常以肃杀而为心。天之于物,春生秋实,故其在乐也,商声\footnote{〔商声〕以商音为主音的调式,即现代的D小调。}主西方之音,夷则\footnote{〔夷则〕十二律中第九律。十二律分别是:黄钟、大吕、太簇、夹钟、姑洗、中吕、林钟、蕤宾、夷则、南吕、无射、应钟。古人把十二律和十二月对应。《礼记·月令》:“孟秋之月,律中夷则”,夷则对应初秋七月。}为七月之律。商,伤也,物既老而悲伤;夷,戮也,物过盛而当杀。”
    
    “嗟乎!草木无情,有时飘零。人为动物,惟物之灵;百忧感其心,万事劳其形。有动于中,必摇其精。而况思其力之所不及,忧其智之所不能?宜其渥然丹者为槁木,黟然黑者为星星。奈何以非金石之质,欲与草木而争荣?念谁为之戕贼,亦何恨乎秋声!”
    
    童子莫对,垂头而睡。但闻四壁虫声唧唧,如助予之叹息。
\end{normalsize}



\chapter{滕王阁序}

\begin{normalsize}
    
    豫章故郡\footnote{〔豫章故郡〕滕王阁在江西南昌市,汉时属豫章郡。},洪都新府\footnote{〔洪都新府〕唐改豫章郡为洪州,设都督府。}。星分翼轸\footnote{〔翼轸〕翼和轸都是星宿。},地接衡庐\footnote{〔衡庐〕衡山和庐山。}。襟三江\footnote{〔三江〕太湖的支流松江、娄江、东江,泛指长江中下游的江河。}而带五湖\footnote{〔五湖〕太湖、鄱阳湖、青草湖、丹阳湖、洞庭湖,泛指南方大湖。},控蛮荆\footnote{〔蛮荆〕古楚地,今湖北、湖南一带。}而引瓯越\footnote{〔瓯越〕古越地,即今浙江地区。古东越王建都于东瓯(今浙江省永嘉县),境内有瓯江。}。物华天宝,龙光射牛斗\footnote{〔牛斗〕牛、斗,星宿名。}之墟;人杰地灵,徐孺下陈蕃之榻\footnote{〔徐孺下陈蕃之榻〕徐孺子名稚,东汉豫章南昌人,当时隐士。据《后汉书·徐稚传》,东汉名士陈蕃为豫章太守,不接宾客,惟徐稚来访时,才设一睡榻,徐稚去后又悬置起来。}。雄州雾列,俊采星驰。台隍枕夷夏之交,宾主尽东南之美。都督阎公\footnote{〔都督阎公〕阎伯屿,时任洪州都督。}之雅望,棨戟遥临;宇文新州\footnote{〔宇文新州〕复姓宇文的新州(在今广东境内)刺史,名未详。}之懿范,襜帷暂驻。十旬休假,胜友如云;千里逢迎,高朋满座。腾蛟起凤,孟学士之词宗;紫电青霜,王将军之武库。家君作宰\footnote{〔家君作宰〕勃之父担任交趾县的县令。},路出名区;童子何知,躬逢胜饯。
    
    时维九月,序属三秋。潦水尽而寒潭清,烟光凝而暮山紫。俨骖騑于上路,访风景于崇阿。临帝子\footnote{〔帝子〕和下一句的“天人”都指滕王李元婴。}之长洲,得天人之旧馆。层峦耸翠,上出重霄;飞阁流丹,下临无地。鹤汀凫渚,穷岛屿之萦回;桂殿兰宫,即冈峦之体势。
    
    披绣闼,俯雕甍,山原旷其盈视,川泽纡其骇瞩。闾阎扑地,钟鸣鼎食之家;舸舰弥津,青雀黄龙之舳。云销雨霁,彩彻区明。落霞与孤鹜齐飞,秋水共长天一色。渔舟唱晚,响穷彭蠡\footnote{〔彭蠡〕即今鄱阳湖。}之滨,雁阵惊寒,声断衡阳\footnote{〔衡阳〕今属湖南省,境内有回雁峰,相传秋雁到此就不再南飞,待春而返。}之浦。
    
    遥襟甫畅,逸兴遄飞。爽籁发而清风生,纤歌凝而白云遏。睢园\footnote{〔睢园〕汉梁孝王菟园,梁孝王曾在园中聚集文人饮酒赋诗。}绿竹,气凌彭泽\footnote{〔彭泽〕县名,在今江西湖口县东,此代指陶潜。陶潜,即陶渊明,曾官彭泽县令,世称陶彭泽。}之樽;邺水\footnote{〔邺水〕在邺下(今河北省临漳县)。邺下是曹魏兴起的地方,三曹常在此雅集作诗。曹植在此作《公宴诗》。}朱华,光照临川\footnote{〔临川〕郡名,治所在今江西省抚州市,代指即谢灵运。谢灵运曾任临川内史。}之笔。四美具,二难并。穷睇眄于中天,极娱游于暇日。天高地迥,觉宇宙之无穷;兴尽悲来,识盈虚之有数。望长安于日下,目吴会\footnote{〔吴会〕秦汉会稽郡治所在吴县,郡县连称为吴会。吴郡,治所在今江苏省苏州市。}于云间。地势极而南溟深,天柱高而北辰远。关山难越,谁悲失路之人;萍水相逢,尽是他乡之客。怀帝阍\footnote{〔帝阍〕天帝的守门人,此处借指皇帝的宫门。}而不见,奉宣室\footnote{〔奉宣室〕代指入朝做官。贾谊迁谪长沙四年后,汉文帝复召他回长安,于宣室中问鬼神之事。宣室,汉未央宫正殿,为皇帝召见大臣议事之处。}以何年?
    
    嗟乎!时运不齐,命途多舛。冯唐易老\footnote{〔冯唐易老〕冯唐在汉文帝、汉景帝时不被重用,汉武帝时被举荐,已是九十多岁。},李广难封\footnote{〔李广难封〕李广,汉武帝时名将,多次与匈奴作战,军功卓著,却始终未获封爵。}。屈贾谊于长沙\footnote{〔屈贾谊于长沙〕贾谊在汉文帝时被贬为长沙王太傅。},非无圣主;窜梁鸿于海曲\footnote{〔梁鸿〕东汉人,作《五噫歌》讽刺朝廷,因此得罪汉章帝,避居齐鲁、吴中。},岂乏明时?所赖君子见机,达人知命。老当益壮,宁移白首之心?穷且益坚,不坠青云之志。酌贪泉而觉爽\footnote{〔贪泉〕贪泉,在广州附近的石门,传说饮此水会贪得无厌,晋时吴隐之喝下此水操守反而更加坚定。},处涸辙以犹欢\footnote{〔处涸辙〕《庄子·外物》:“”}。北海虽赊,扶摇可接;东隅已逝,桑榆非晚。孟尝高洁\footnote{〔孟尝〕据《后汉书·孟尝传》,孟尝字伯周,东汉会稽上虞人。曾任合浦太守,以廉洁奉公著称,后因病隐居。桓帝时,虽有人屡次荐举,终不见用。},空余报国之情;阮籍猖狂\footnote{〔阮籍〕字嗣宗,晋代名士,不满世事,佯装狂放,常驾车出游,路不通时就痛哭而返。《晋书·阮籍传》:籍“时率意独驾,不由径路。车迹所穷,辄恸哭而反。”},岂效穷途之哭!
    
    勃,三尺微命,一介书生。无路请缨,等终军\footnote{〔终军〕字子云,汉代济南人。武帝时出使南越,自请“愿受长缨,必羁南越王而致之阙下”,时仅二十馀岁。}之弱冠\footnote{〔弱冠〕古人二十岁行冠礼,表示成年,二十岁称“弱冠”。};有怀投笔\footnote{〔宗悫〕字元干,南朝宋南阳人,年少时向叔父自述志向,云“愿乘长风破万里浪”。后因战功受封。},慕宗悫\footnote{〔投笔〕《后汉书·班超传》有班超投笔从戎的故事。}之长风。舍簪笏于百龄,奉晨昏于万里。非谢家之宝树\footnote{〔谢家之宝树〕指谢玄,比喻好子弟。《世说新语·言语》:“谢太傅(安)问诸子侄‘子弟亦何预人事,而正欲使其佳?’诸人莫有言者。车骑(谢玄)答曰:‘譬如芝兰玉树,欲使其生于庭阶耳。’”},接孟氏之芳邻\footnote{〔孟氏之芳邻〕孟轲的母亲为教育儿子而三迁择邻,最后定居于学宫附近。}。他日趋庭,叨陪鲤对\footnote{〔他日趋庭,叨陪鲤对〕鲤,孔鲤,孔子之子。趋庭,受父亲教诲。《论语·季氏》:“(孔子)尝独立,(孔)鲤趋而过庭。(子)曰:‘学诗乎?’对曰:‘未也。’‘不学诗,无以言。’鲤退而学诗。他日,又独立,鲤趋而过庭。(子)曰:‘学礼乎?’对曰:‘未也。’‘不学礼,无以立。’鲤退而学礼。”};今兹捧袂,喜托龙门\footnote{〔喜托龙门〕《后汉书·李膺传》:“膺以声名自高,士有被其容接者,名为登龙门。”}。杨意不逢\footnote{〔杨意不逢……〕杨意,杨得意的省称。凌云,指司马相如作《大人赋》。据《史记·司马相如列传》,司马相如经蜀人杨得意引荐,方能入朝见汉武帝。又云:“相如既奏《大人》之颂,天子大悦,飘飘有凌云之气。”},抚凌云而自惜。钟期既遇\footnote{〔钟期既遇……〕钟期,钟子期的省称。《列子·汤问》:“伯牙善鼓琴,钟子期善听。伯牙鼓琴……志在流水,钟子期曰:‘善哉!洋洋兮若江河。’”},奏流水以何惭?
    
    呜乎!胜地不常,盛筵难再。兰亭\footnote{〔兰亭〕在今浙江省绍兴市附近。晋穆帝永和九年(353)三月三日上巳节,王羲之与群贤宴集于此,写有名篇《兰亭集序》。}已矣,梓泽\footnote{〔梓泽〕即晋代石崇的金谷园,故址在今河南省洛阳市西北。}丘墟。临别赠言,幸承恩于伟饯;登高作赋,是所望于群公。敢竭鄙怀,恭疏短引;一言均赋,四韵俱成\footnote{〔四韵俱成〕四韵一起写好了。四韵,八句四韵诗,指王勃此时写下的《滕王阁诗》:“滕王高阁临江渚,佩玉鸣鸾罢歌舞。画栋朝飞南浦云,珠帘暮卷西山雨。闲云潭影日悠悠,物换星移几度秋。阁中帝子今何在?槛外长江空自流。”}。请洒潘江,各倾陆海云尔\footnote{〔请洒潘江……〕钟嵘《诗品》:“陆(机)才如海,潘(岳)才如江。”这里形容各宾客的文采。}。
\end{normalsize}



\chapter{五蠹}

\begin{normalsize}
    
    上古之世,人民少而禽兽众,人民不胜禽兽虫蛇。有圣人作,构木为巢以避群害,而民悦之,使王天下,号曰有巢氏。民食果蓏蚌蛤,腥臊恶臭而伤害腹胃,民多疾病。有圣人作,钻燧取火以化腥臊,而民说之,使王天下,号之曰燧人氏。中古之世,天下大水,而鲧、禹决渎。近古之世,桀、纣暴乱,而汤、武征伐。今有构木钻燧于夏后氏之世者,必为鲧、禹笑矣;有决渎于殷、周之世者,必为汤、武笑矣。然则今有美尧、舜、汤、武、禹之道于当今之世者,必为新圣笑矣。是以圣人不期修古,不法常可,论世之事,因为之备。宋有人耕田者,田中有株,兔走触株,折颈而死,因释其耒而守株,冀复得兔。兔不可复得,而身为宋国笑。今欲以先王之政,治当世之民,皆守株之类也。
     
    古者丈夫不耕,草木之实足食也;妇人不织,禽兽之皮足衣也。不事力而养足,人民少而财有余,故民不争。是以厚赏不行,重罚不用,而民自治。今人有五子不为多,子又有五子,大父未死而有二十五孙。是以人民众而货财寡,事力劳而供养薄,故民争,虽倍赏累罚而不免于乱。
     
    尧之王天下也,茅茨不翦,采椽不斫;粝粢之食,䔧藿之羹;冬日麂裘,夏日葛衣;虽监门之服养,不亏于此矣。禹之王天下也,身执耒歃以为民先,股无肢,胫不生毛,虽臣虏之劳,不苦于此矣。以是言之,夫古之让天子者,是去监门之养,而离臣虏之劳也,古传天下而不足多也。今之县令,一日身死,子孙累世絜驾,故人重之。是以人之于让也,轻辞古之天子,难去今之县令者,薄厚之实异也。夫山居而谷汲者,腊而相遗以水;泽居苦水者,买庸而决窦。故饥岁之春,幼弟不饷;穰岁之秋,疏客必食。非疏骨肉爱过客也,多少之实异也。是以古之易财,非仁也,财多也;今之争夺,非鄙也,财寡也。轻辞天子,非高也,势薄也;争士橐,非下也,权重也。故圣人议多少、论薄厚为之政。故罚薄不为慈,诛严不为戾,称俗而行也。故事因于世,而备适于事。
     
    古者大王\footnote{〔大王〕指周文王。丰、镐是丰京和镐京的统称,周朝的都城,在今日陕西西安市。}处丰、镐之间,地方百里,行仁义而怀西戎,遂王天下。徐偃王\footnote{〔徐偃王〕西周时期徐国的国君。传说周穆王巡视各国,听闻徐君威德日远,遣楚国袭其不备,大破之,杀偃王。一说是宋王偃之误。公元前286年,齐联合魏、楚灭宋。宋王偃战败,死于温。}处汉东,地方五百里,行仁义,割地而朝者三十有六国。荆文王\footnote{〔荆文王〕可能指楚文王或楚庄王。}恐其害己也,举兵伐徐,遂灭之。故文王行仁义而王天下,偃王行仁义而丧其国,是仁义用于古不用于今也。故曰:世异则事异。当舜之时,有苗\footnote{〔有苗〕古国名,又称“三苗”。传说尧禅位于舜,三苗不服作乱。禹征三苗后灭亡。有认为现代苗族是有苗文化的孑遗。}不服,禹将伐之。舜曰:“不可。上德不厚而行武,非道也。”乃修教三年,执干戚舞,有苗乃服。共工\footnote{〔共工〕传说中炎帝的后裔,祝融的儿子,尧的臣子。韩非子认为共工之战是中古之事。}之战,铁铦矩者及乎敌,铠甲不坚者伤乎体。是干戚用于古不用于今也。故曰:事异则备变。上古竞于道德,中世逐于智谋,当今争于气力。齐将攻鲁,鲁使子贡\footnote{〔子贡〕端木赐,字子贡,春秋时卫国人,孔子的弟子。贤能善辩,曾任鲁国、卫国的相国,还善于经商。}说之。齐人曰:“子言非不辩也,吾所欲者土地也,非斯言所谓也。”遂举兵伐鲁,去门十里以为界。故偃王仁义而徐亡,子贡辩智而鲁削。以是言之,夫仁义辩智,非所以持国也。去偃王之仁,息子贡之智,循徐、鲁之力使敌万乘,则齐、荆之欲不得行于二国矣。
    
    夫古今异俗,新故异备。如欲以宽缓之政,治急世之民,犹无辔策而御駻马,此不知之患也。且民者固服于势,寡能怀于义。仲尼\footnote{〔仲尼〕孔子字仲尼。},天下圣人也,修行明道以游海内,海内说其仁、美其义而为服役者七十人。盖贵仁者寡,能义者难也。故以天下之大,而为服役者七十人,而仁义者一人。鲁哀公\footnote{〔鲁哀公〕春秋时期鲁国君主(公元前494年至前468年在位),鲁定公之子。},下主也,南面君国,境内之民莫敢不臣。民者固服于势,诚易以服人,故仲尼反为臣而哀公顾为君。仲尼非怀其义,服其势也。故以义则仲尼不服于哀公,乘势则哀公臣仲尼。今学者之说人主也,不乘必胜之势,而务行仁义则可以王,是求人主之必及仲尼,而以世之凡民皆如列徒,此必不得之数也。
    
    儒以文乱法,侠以武犯禁,而人主兼礼之,此所以乱也。夫离法者罪,而诸先王以文学取;犯禁者诛,而群侠以私剑养。故法之所非,君之所取;吏之所诛,上之所养也。法、趣、上、下,四相反也,而无所定,虽有十黄帝不能治也。故行仁义者非所誉,誉之则害功;文学者非所用,用之则乱法。楚之有直躬,其父窃羊,而谒之吏。令尹曰:“杀之!”以为直于君而曲于父,报而罪之。以是观之,夫君之直臣,父之暴子也。鲁人从君战,三战三北。仲尼问其故,对曰:“吾有老父,身死莫之养也。”仲尼以为孝,举而上之。以是观之,夫父之孝子,君之背臣也。故令尹诛而楚奸不上闻,仲尼赏而鲁民易降北。上下之利,若是其异也,而人主兼举匹夫之行,而求致社稷之福,必不几矣。
\end{normalsize}



\chapter{项脊轩志}

\begin{normalsize}
    
    项脊轩\footnote{〔项脊轩〕归有光家的书斋名。},旧南阁子也。室仅方丈,可容一人居。百年老屋,尘泥渗漉,雨泽下注;每移案,顾视无可置者。又北向,不能得日,日过午已昏。余稍为修葺,使不上漏。前辟四窗,垣墙周庭,以当南日,日影反照,室始洞然。又杂植兰桂竹木于庭,旧时栏楯,亦遂增胜。借书满架,偃仰啸歌,冥然兀坐,万籁有声;而庭阶寂寂,小鸟时来啄食,人至不去。三五之夜\footnote{〔三五之夜〕农历每月十五的夜晚。},明月半墙,桂影斑驳,风移影动,珊珊可爱。
    
    然余居于此,多可喜,亦多可悲。先是庭中通南北为一。迨诸父异爨\footnote{〔异爨〕分灶做饭,比喻分家。},内外多置小门,墙往往而是;东犬西吠\footnote{〔东犬西吠〕东边的狗对着西边叫。意思是分家后,狗把原住同一庭院的人当作陌生人。},客逾庖而宴,鸡栖于厅。庭中始为篱,已为墙,凡再变矣。家有老妪,尝居于此。妪,先大母婢也,乳二世,先妣抚之甚厚。室西连于中闺,先妣尝一至。妪每谓余曰:“某所,而母立于兹。”妪又曰:“汝姊在吾怀,呱呱而泣;娘以指叩门扉曰:‘儿寒乎~欲食乎~’吾从板外相为应答。”语未毕,余泣,妪亦泣。余自束发\footnote{〔束发〕古代男孩成童时束发为髻。}读书轩中,一日,大母过余曰:“吾儿,久不见若影,何竟日默默在此,大类女郎也?”比去,以手阖门,自语曰:“吾家读书久不效,儿之成,则可待乎!”顷之,持一象笏至,曰:“此吾祖太常公宣德间执此以朝,他日汝当用之!”瞻顾遗迹,如在昨日,令人长号不自禁。
    
    轩东故尝为厨,人往,从轩前过。余扃牖而居,久之,能以足音辨人。轩凡四遭火,得不焚,殆有神护者。
    
    项脊生曰:“蜀清守丹穴,利甲天下,其后秦皇帝筑女怀清台。刘玄德与曹操争天下,诸葛孔明起陇中。方二人之昧昧于一隅也,世何足以知之,余区区处败屋中,方扬眉、瞬目,谓有奇景。人知之者,其谓与坎井之蛙何异?”
    
    余既为此志,后五年,吾妻来归\footnote{〔来归〕嫁到我家来。归:古代女子出嫁。},时至轩中,从余问古事,或凭几学书。吾妻归宁\footnote{〔归宁〕出嫁的女儿回娘家省亲。},述诸小妹语曰:“闻姊家有阁子,且何谓阁子也?”其后六年,吾妻死,室坏不修。其后二年,余久卧病无聊,乃使人复葺南阁子,其制稍异于前。然自后余多在外,不常居。
    
    庭有枇杷树,吾妻死之年所手植也,今已亭亭如盖矣。
\end{normalsize}



\chapter{过秦论}

\begin{normalsize}
    
    秦孝公\footnote{〔秦孝公〕战国时期秦国君主(公元前362至338年),任用商鞅变法强秦。}据崤函\footnote{〔崤函〕崤山和函谷关的合称,秦国东部重要关隘。}之固,拥雍州\footnote{〔雍州〕古代九州之一,指关中地区秦国核心疆域。}之地,君臣固守以窥周室\footnote{〔周室〕周王室,战国时已衰微,仅名义上是天下共主。},有席卷天下,包举宇内,囊括四海之意,并吞八荒之心。当是时也,商君\footnote{〔商君〕即商鞅,法家代表人物,主持秦国变法改革。}佐之,内立法度,务耕织,修守战之具,外连衡\footnote{〔连衡〕即连横策略,秦国分化瓦解六国联盟的外交手段。}而斗诸侯。于是秦人拱手而取西河\footnote{〔西河〕黄河以西地区,秦魏争夺的战略要地。}之外。
    
    孝公既没,惠文、武、昭襄\footnote{〔惠文、武、昭襄〕秦孝公之后连续三位秦国君主(公元前337至251年)。}蒙故业,因遗策,南取汉中\footnote{〔汉中〕汉水上游盆地,秦楚交界战略要冲。},西举巴、蜀\footnote{〔巴、蜀〕今四川地区,秦并吞的重要粮仓基地。},东割膏腴之地,北收要害之郡。诸侯恐惧,会盟而谋弱秦。不爱珍器重宝肥饶之地,以致天下之士,合从缔交,相与为一。当此之时,齐有孟尝\footnote{〔孟尝〕齐国公子田文,战国四公子之一。},赵有平原\footnote{〔平原〕赵国公子赵胜,战国四公子之一,以善养士闻名。},楚有春申\footnote{〔春申〕楚国令尹黄歇,战国四公子之一,曾主持合纵抗秦。},魏有信陵\footnote{〔信陵〕魏国公子魏无忌,战国四公子之一,窃符救赵典故主角。}。此四君者,皆明智而忠信,宽厚而爱人,尊贤而重士,约从离衡,兼韩、魏、燕、楚、齐、赵、宋、卫、中山之众。于是六国之士,有甯越、徐尚、苏秦、杜赫之属\footnote{〔甯越……之属〕六国著名谋士策士。}为之谋,齐明、周最、陈轸、召滑、楼缓、翟景、苏厉、乐毅之徒\footnote{〔齐明……之徒〕战国时期各国的纵横家与外交人才。}通其意,吴起、孙膑、带佗、倪良、王廖、田忌、廉颇、赵奢之伦\footnote{〔吴起……之伦〕战国时期的著名军事将领。}制其兵。尝以十倍之地,百万之众,叩关而攻秦。秦人开关延敌,九国之师,逡巡而不敢进。秦无亡矢遗镞之费,而天下诸侯已困矣。于是从散约败,争割地而赂秦。秦有余力而制其弊,追亡逐北,伏尸百万,流血漂橹。因利乘便,宰割天下,分裂山河。强国请服,弱国入朝。延及孝文王、庄襄王\footnote{〔孝文王、庄襄王〕秦昭襄王死后的两位短命秦王(公元前250至247年),之后朝政被相国吕不韦把持,直至嬴政22岁即位。},享国之日浅,国家无事。
    
    及至始皇\footnote{〔始皇〕秦始皇嬴政(公元前238至210年),完成统一,建立秦朝。},奋六世之余烈,振长策而御宇内,吞二周\footnote{〔二周〕即西周和东周。}而亡诸侯,履至尊而制六合\footnote{〔六合〕天地四方,代指全天下。},执敲扑而鞭笞天下,威振四海。南取百越\footnote{〔百越〕岭南少数民族部落的统称。}之地,以为桂林、象郡\footnote{〔桂林、象郡〕秦朝征服岭南后置郡,包括今天的广西、广东部分和越南北部。};百越之君,俯首系颈,委命下吏。乃使蒙恬\footnote{〔蒙恬〕秦朝名将,率军北击匈奴并修筑长城。}北筑长城而守藩篱,却匈奴\footnote{〔匈奴〕北方游牧民族,长期威胁中原王朝。}七百余里;胡人不敢南下而牧马,士不敢弯弓而报怨。于是废先王之道,焚百家之言,以愚黔首;隳名城,杀豪杰;收天下之兵,聚之咸阳,销锋镝,铸以为金人十二,以弱天下之民。然后践华为城,因河为池,据亿丈之城,临不测之渊,以为固。良将劲弩守要害之处,信臣精卒陈利兵而谁何。天下已定,始皇之心,自以为关中之固,金城千里,子孙帝王万世之业也。
    
    始皇既没,余威震于殊俗。然陈涉\footnote{〔陈涉〕即陈胜,秦末农民起义领袖。}瓮牖绳枢之子,氓隶之人,而迁徙之徒也;才能不及中人,非有仲尼\footnote{〔仲尼〕即孔子,儒家学说创始人。}、墨翟\footnote{〔墨翟〕即墨子,墨家学说创始人。}之贤,陶朱\footnote{〔陶朱〕即范蠡,春秋时期著名富商。}、猗顿\footnote{〔猗顿〕春秋时期鲁国巨富。}之富。蹑足行伍之间,而倔起阡陌之中,率疲弊之卒,将数百之众,转而攻秦;斩木为兵,揭竿为旗,天下云集响应,赢粮而景从。山东\footnote{〔山东〕崤山以东地区,代指六国故地。}豪俊遂并起而亡秦族矣。
    
    且夫天下非小弱也,雍州之地,崤函之固,自若也。陈涉之位,非尊于齐、楚、燕、赵、韩、魏、宋、卫、中山之君也;锄櫌棘矜,非铦于钩戟长铩也;谪戍之众,非抗于九国之师也;深谋远虑,行军用兵之道,非及乡时之士也。然而成败异变,功业相反,何也?试使山东之国与陈涉度长絜大,比权量力,则不可同年而语矣。然秦以区区之地,致万乘之势,序八州而朝同列,百有余年矣;然后以六合为家,崤函为宫;一夫作难而七庙\footnote{〔七庙〕周代宗庙制度,天子可祭祀七代祖先,故称七庙。}隳,身死人手,为天下笑者,何也?仁义不施而攻守之势异也。
\end{normalsize}



\chapter{后赤壁赋}

\begin{normalsize}
    
    是岁\footnote{〔是岁〕宋神宗元丰五年(公元1082年)。}十月之望,步自雪堂\footnote{〔雪堂〕苏轼在黄州所建的新居,离他在临皋的住处不远,在黄冈东面。堂在大雪时建成,画雪景于四壁,故名“雪堂”。},将归于临皋\footnote{〔临皋〕亭名,在黄冈南长江边上。苏轼初到黄州时住在定惠院,不久就迁至临皋亭。}。二客从予,过黄泥之坂\footnote{〔黄泥之坂〕黄冈东面东坡附近的山坡叫“黄泥坂”。}。霜露既降,木叶尽脱。人影在地,仰见明月。顾而乐之,行歌相答。
    
    已而叹曰:“有客无酒,有酒无肴,月白风清,如此良夜何!”客曰:“今者薄暮,举网得鱼,巨口细鳞,状如松江之鲈。顾安所得酒乎?”归而谋诸妇。妇曰:“我有斗酒,藏之久矣,以待子不时之需。”
    
    于是携酒与鱼,复游于赤壁之下。江流有声,断岸千尺;山高月小,水落石出。曾日月之几何,而江山不可复识矣。予乃摄衣而上,履巉岩,披蒙茸,踞虎豹,登虬龙,攀栖鹘之危巢,俯冯夷\footnote{〔冯夷〕也叫“冰夷”,即河伯,黄河的河神。}之幽宫。盖二客不能从焉。划然长啸,草木震动,山鸣谷应,风起水涌。予亦悄然而悲,肃然而恐,凛乎其不可留也。反而登舟,放乎中流,听其所止而休焉。时夜将半,四顾寂寥。适有孤鹤,横江东来,翅如车轮,玄裳缟衣,戛然长鸣,掠予舟而西也。
    
    须臾客去,予亦就睡。梦一道士,羽衣蹁跹,过临皋之下,揖予而言曰:“赤壁之游乐乎?”问其姓名,俯而不答。“呜呼!噫嘻!我知之矣。畴昔之夜,飞鸣而过我者,非子也邪?”道士顾笑,予亦惊寤。开户视之,不见其处。
\end{normalsize}



\chapter{谏逐客书}

\begin{normalsize}
    
    臣闻吏议逐客,窃以为过矣。昔缪公\footnote{〔缪公〕即秦穆公(前659至前621年在位),春秋五霸之一。}求士,西取由余\footnote{〔由余〕西戎谋士,助秦穆公征服西戎十二国(前623年左右),开拓陇西地区。}于戎,东得百里奚\footnote{〔百里奚〕原为虞国大夫,秦穆公以五张羊皮赎于楚国(前655年),任相国推行改革。}于宛,迎蹇叔\footnote{〔蹇叔〕宋国隐士,经百里奚推荐入秦(前655年),为穆公制定东进战略的核心谋臣。}于宋,来丕豹\footnote{〔丕豹〕晋国大夫丕郑之子,其父被杀后奔秦(前651年),助秦攻晋取得河西部分土地。}、公孙支\footnote{〔公孙支〕字子桑,晋国人,投秦后参与策划殽之战(前627年)等重大军事行动。}于晋。此五子者,不产于秦,而缪公用之,并国二十,遂霸西戎。孝公\footnote{〔孝公〕秦孝公(前361至前338年在位),任用商鞅变法,公元前340年夺取魏国河西之地(今陕西东部)。}用商鞅\footnote{〔商鞅〕卫国人,主持变法,领军攻魏取少梁。}之法,移风易俗,民以殷盛,国以富强,百姓乐用,诸侯亲服,获楚、魏之师,举地千里,至今治强。惠王\footnote{〔惠王〕秦惠文王(前337至前311年在位),前316年派张仪、司马错灭蜀国巴国(今四川),置巴郡、蜀郡。}用张仪\footnote{〔张仪〕魏国人,推行连横策略,前313年诈楚绝齐,前312年丹阳之战夺取楚国汉中(今陕西南部)。}之计,拔三川之地\footnote{〔三川之地〕指黄河、雒水、伊水三川之地,在今河南西北部黄河以南的洛水、伊水流域。韩宣王在此设三川郡。公元前249年秦灭东周,取得三川。},西并巴、蜀,北收上郡\footnote{〔上郡〕郡名,原为楚地,今陕西榆林。魏文侯时置郡,公元前328年被迫献于秦。},南取汉中\footnote{〔汉中〕郡名,今陕西汉中。楚怀王时置郡。公元前312年被秦国攻取。},包九夷\footnote{〔九夷〕指楚国境内西北部的少数部族,在今陕西、湖北、四川三省交界地区。},制鄢、郢\footnote{〔鄢、郢〕楚国曾经的都城。分别在今湖北宜城县东南、江陵市西北。公元前279年秦将白起攻取鄢,翌年又攻取郢。},东据成皋\footnote{〔成皋〕邑名,在今河南荥阳县汜水镇,地势险要,是著名的军事重地。春秋时属郑国称虎牢。公元前375年韩国灭郑后属韩,公元前249年被秦军攻取。}之险,割膏腴之壤,遂散六国之从,使之西面事秦,功施到今。昭王\footnote{〔昭王〕秦昭襄王(前306至前251年在位),前266年用范雎"远交近攻"策略,持续东进。}得范雎\footnote{〔范雎〕魏国人,提出"强干弱枝"政策,前262年发动长平之战,前256年灭西周。},废穰侯\footnote{〔穰侯〕魏冉,秦昭襄王母宣太后之异父弟,拥立秦昭王,任将军,多次为相,受封于穰(今河南邓县),故称穰侯。昭王任用范雎后被免职遣归封地。},逐华阳\footnote{〔华阳〕芈戎,宣太后同父弟,封华阳君,与穰侯共同专权,后被范雎免职遣归封地。},强公室,杜私门,蚕食诸侯,使秦成帝业。此四君者,皆以客之功。由此观之,客何负于秦哉!向使四君却客而不内,疏士而不用,是使国无富利之实,而秦无强大之名也。
    
    今陛下致昆山之玉,有随、和之宝\footnote{〔随和之宝〕即所谓“随侯珠”和“和氏璧”,传说中春秋时随侯所得的夜明珠和楚人卞和来得的美玉。},垂明月之珠,服太阿\footnote{〔太阿〕亦称“泰阿”,宝剑名,相传为春秋著名工匠欧冶子、干将所铸。}之剑,乘纤离之马,建翠凤之旗,树灵鼍之鼓。此数宝者,秦不生一焉,而陛下说之,何也?必秦国之所生然后可,则是夜光之璧不饰朝廷,犀象之器不为玩好,郑、卫之女不充后宫,而骏良駃騠不实外厩,江南金锡不为用,西蜀丹青不为采。所以饰后宫,充下陈,娱心意,说耳目者,必出于秦然后可,则是宛珠之簪、傅玑之珥、阿缟之衣、锦绣之饰不进于前,而随俗雅化佳冶窈窕赵女不立于侧也。夫击瓮叩缶,弹筝搏髀,而歌呼呜呜快耳者,真秦之声也;《郑》《卫》《桑间》《昭》《虞》《武》《象》者,异国之乐也。今弃击瓮叩缶而就《郑》《卫》,退弹筝而取《昭》《虞》,若是者何也?快意当前,适观而已矣。今取人则不然,不问可否,不论曲直,非秦者去,为客者逐。然则是所重者,在乎色、乐、珠玉,而所轻者,在乎人民也。此非所以跨海内、制诸侯之术也。
    
    臣闻地广者粟多,国大者人众,兵强则士勇。是以太山不让土壤,故能成其大;河海不择细流,故能就其深;王者不却众庶,故能明其德。是以地无四方,民无异国,四时充美,鬼神降福,此五帝三王\footnote{〔五帝三王〕指黄帝、颛顼、帝喾、尧、舜。三王,指夏、商、周三代开国君主,即夏禹、商汤和周武王。}之所以无敌也。今乃弃黔首\footnote{〔黔首〕泛指百姓。无爵平民不能服冠,只能以黑巾裹头,故称黔首,秦始皇统一六国后正式称百姓为黔首。《史记·秦始皇本纪》载:二十六年,“更名民曰黔首”。}以资敌国,却宾客以业诸侯,使天下之士退而不敢西向,裹足不入秦,此所谓“藉寇兵而赍盗粮”者也。
    
    夫物不产于秦,可宝者多;士不产于秦,而愿忠者众。今逐客以资敌国,损民以益仇,内自虚而外树怨于诸侯,求国无危,不可得也。
\end{normalsize}



\chapter{利议}

\begin{normalsize}
    
    大夫\footnote{〔大夫〕此处指御史大夫桑弘羊,盐铁会议中主张盐铁官营的朝廷代表。}曰:“作世明主,忧劳万人,思念北边之未这,故使使者举贤良、文学\footnote{〔文学〕指被举荐的贤良文学之士,代表地方豪强与儒家学者里反对国营政策的群体。}高弟,说延有道之士,将欲观殊议异策,虚心倾耳以听,庶几云得。诸生无能出奇计,运图匈奴安边境之策,抱丁竹,守空言,不知道舍之宜,时世之变。议论无所依,如膝痒而搔背。辩讼公门之下,兄兄不可胜听,如品即口以成事。此岂明主所欲闻哉?”
    
    文学曰:“诸生对册,殊路同归。指在于崇礼义,退财利,复往古之道,匡当世之失,莫不云太平。虽未尽可但用,宜略有可行者焉。执事暗于明礼,而喻于利未,沮事隳议。计虑筹策以故至今未决。非儒无成事,公卿欲成利也。”
    
    大夫曰:“色厉而内荏,乱真者也。文表而洗里,乱实者也。文学哀衣博带,窃周公\footnote{〔周公〕姬旦,西周初期政治家,儒家理想中的辅政典范,象征礼制与德治。}之服;鞠躬促急,窃仲尼之容;议论称诵,窃商赐\footnote{〔商赐〕卜商(子夏)和端木赐(子贡),孔子弟子,以辩才著称,喻指文学空有口才无实务能力。}之辞;刺讥言治,过管晏\footnote{〔管晏〕管仲与晏婴,春秋齐国之相,代表务实治国的能臣。}之才。心卑卿相,志小万乘。及授之政,昏乱不治。故以言举人,若以毛相马。此其所以多不称举。诏策曰:‘陈嘉宇内之士,故详延四方豪俊文学博习之士,趋迁官禄。’言者不必有德,何者?言之易而行之难。有舍其车而识其牛,贵其不言而多成事也。吴铎\footnote{〔吴铎〕吴地铜铃,铃锤舌形,象征谏言者之舌,比喻主父偃等多言招祸。}以其舌自破,主父偃\footnote{〔主父偃〕汉武帝时纵横家,提议推恩令削弱诸侯。}以其舌自杀。贺旦\footnote{〔贺旦〕"贺"通"鶡",指黎明报晓的鶡鸟(雉类)。}夜鸣,无益于明;主父鸣痴,无益于死。非有司欲成利,文学桎梏于旧术,牵于问言者也。”
    
    文学曰:“能言之,能行之者,汤武也。能言,不能行者,有司也。文学窃周公之服。有司窃周公之位。文学桎梏于旧术,有司桎梏于财利。主父偃以舌自杀,有司以利自困。夫骥之才千里,非造父\footnote{〔造父〕周穆王御者,驾车技术超凡。}不能使。禹之知万人,非舜为相不能用。故季桓子\footnote{〔季桓子〕鲁国权臣季孙斯,排挤孔子。}听政,柳下惠\footnote{〔柳下惠〕展获,春秋鲁国贤士,道德崇高但不得重用。}忽然不见;孔子之为司寇,然后勃炽。骥,举之在伯乐,共功在造父。造父摄辔,马无驽良,皆可取道。周公之时,士无贤不肖,皆可与言治。故御之良者善调马,相之贤者善使士。今举异才面使减驺御之,是犹恶骥盐车而责之使疾。此贤良文学多不称举也。”
    
    大夫曰:“嘻!诸生榻容无行,多言而不用,情貌不相副。若穿俞之盗,自古而患之。是孔丘斥逐于鲁君,会不用于世也。何者?以其首摄多端,迂时而不要也。故秦王燔去其术而不行,坑之渭中而不用。乃安得鼓口舌,申颜眉,预前议论,是非国家之事也?”
\end{normalsize}



\chapter{论贵粟疏}

\begin{normalsize}
    
    圣王在上,而民不冻饥者,非能耕而食之,织而衣之也,为开其资财之道也。故尧、禹<footnote:N1>有九年之水,汤\footnote{〔汤〕汤,商朝的开国君主。}有七年之旱,而国亡捐瘠者,以畜积多而备先具也。今海内为一,土地人民之众不避汤、禹,加以亡天灾数年之水旱,而畜积未及者,何也?地有遗利,民有余力,生谷之土未尽垦,山泽之利未尽出也,游食之民未尽归农也。
    
    民贫,则奸邪生。贫生于不足,不足生于不农,不农则不地著\footnote{〔地著〕定居一地。《汉书·食货志》:“理民之道,地著为本。”颜师古注:“地著,谓安土也。”},不地著则离乡轻家,民如鸟兽。虽有高城深池,严法重刑,犹不能禁也。夫寒之于衣,不待轻暖;饥之于食,不待甘旨;饥寒至身,不顾廉耻。人情一日不再食则饥,终岁不制衣则寒。夫腹饥不得食,肤寒不得衣,虽慈母不能保其子,君安能以有其民哉?明主知其然也,故务民于农桑,薄赋敛,广畜积,以实仓廪,备水旱,故民可得而有也。
    
    民者,在上所以牧之,趋利如水走下,四方无择也。夫珠玉金银,饥不可食,寒不可衣,然而众贵之者,以上用之故也。其为物轻微易藏,在于把握,可以周海内而无饥寒之患。此令臣轻背其主,而民易去其乡,盗贼有所劝,亡逃者得轻资也。粟米布帛生于地,长于时,聚于力,非可一日成也。数石之重,中人弗胜,不为奸邪所利;一日弗得而饥寒至。是故明君贵五谷而贱金玉。
    
    今农夫五口之家,其服役者不下二人,其能耕者不过百亩,百亩之收不过百石。春耕,夏耘,秋获,冬藏,伐薪樵,治官府,给徭役;春不得避风尘,夏不得避署热,秋不得避阴雨,冬不得避寒冻,四时之间,无日休息。又私自送往迎来,吊死问疾,养孤长幼在其中。勤苦如此,尚复被水旱之灾,急政暴虐,赋敛不时,朝令而暮改。当具有者半贾而卖,无者取倍称之息;于是有卖田宅、鬻子孙以偿债者矣。而商贾大者积贮倍息,小者坐列贩卖,操其奇赢,日游都市,乘上之急,所卖必倍。故其男不耕耘,女不蚕织,衣必文采,食必粱肉;无农夫之苦,有阡陌之得。因其富厚,交通王侯,力过吏势,以利相倾;千里游遨,冠盖相望,乘坚策肥,履丝曳缟。此商人所以兼并农人,农人所以流亡者也。今法律贱商人,商人已富贵矣;尊农夫,农夫已贫贱矣。故俗之所贵,主之所贱也;吏之所卑,法之所尊也。上下相反,好恶乖迕,而欲国富法立,不可得也。
    
    方今之务,莫若使民务农而已矣。欲民务农,在于贵粟;贵粟之道,在于使民以粟为赏罚。今募天下入粟县官,得以拜爵,得以除罪。如此,富人有爵,农民有钱,粟有所渫。夫能入粟以受爵,皆有余者也。取于有余,以供上用,则贫民之赋可损,所谓损有余、补不足,令出而民利者也。顺于民心,所补者三:一曰主用足,二曰民赋少,三曰劝农功。今令民有车骑马一匹者,复卒三人。车骑者,天下武备也,故为复卒。
    
    神农之教曰:“有石城十仞,汤池百步,带甲百万,而无粟,弗能守也。”以是观之,粟者,王者大用,政之本务。令民入粟受爵,至五大夫\footnote{〔五大夫〕先秦至汉代都有的爵位,在二十等爵位里自下数起第九级。}以上,乃复一人耳,此其与骑马之功相去远矣。爵者,上之所擅,出于口而无穷;粟者,民之所种,生于地而不乏。夫得高爵也免罪,人之所甚欲也。使天下人入粟于边,以受爵免罪,不过三岁,塞下之粟必多矣。
\end{normalsize}


\newpage

\textbf{注解}:

\vspace{-1em}

\begin{itemize}
    \setlength\itemsep{-0.2em}
    \item〔五大夫〕自商鞅变法以来,秦国设二十等爵位,以赏军功:一级公士,二级上造,三级簪袅,四级不更,五级大夫,六级官大夫,七级公大夫,八级公乘,九级五大夫,十级左庶长,十一级右庶长,十二级左更,十三级中更,十四级右更,十五级少上造,十六级大上造,十七级驷车庶长,十八级大庶长,十九级关内侯,二十级彻侯。彻侯为最高等,以一县为食邑,并得以自置吏于封地;其次是关内侯,有食邑、封户,但只能食税;大庶长以下皆有岁俸;公士为最低等,临战斩敌甲士(披甲士兵)首一级即赐,得田一顷、宅一处、仆一人。
\end{itemize}

\chapter{论积贮疏}

\begin{normalsize}
    
    管子\footnote{〔管子〕即管仲。后人把他的学说和托名的著作编辑成《管子》一书,共二十四卷。}曰:“仓廪实而知礼节。”民不足而可治者,自古及今,未之尝闻。古之人曰:“一夫不耕,或受之饥;一女不织,或受之寒。” 生之有时,而用之亡度,则物力必屈。古之治天下,至孅至悉也,故其畜积足恃。今背本而趋末,食者甚众,是天下之大残也;淫侈之俗,日日以长,是天下之大贼也。残贼公行,莫之或止;大命将泛,莫之振救。生之者甚少,而靡之者甚多,天下财产何得不蹶!
    
    汉之为汉,几四十年矣,公私之积,犹可哀痛!失时不雨,民且狼顾;岁恶不入,请卖爵子,既闻耳矣。安有为天下阽危者若是而上不惊者?世之有饥穰,天之行也,禹、汤\footnote{〔禹、汤〕禹,传说中古代部落联盟领袖。原为夏后氏部落领袖,奉舜命治水有功,舜死后继其位。汤,商朝的开国君主。}被之矣。即不幸有方二三千里之旱,国胡以相恤?卒然边境有急,数千百万之众,国胡以馈之?兵旱相乘,天下大屈,有勇力者聚徒而衡击;罢夫羸老易子而咬其骨。政治未毕通也,远方之能疑者,并举而争起矣。乃骇而图之,岂将有及乎?
    
    夫积贮者,天下之大命也。苟粟多而财有余,何为而不成?以攻则取,以守则固,以战则胜。怀敌附远,何招而不至!今殴民而归之农,皆著于本;使天下各食其力,末技游食之民,转而缘南亩,则畜积足而人乐其所矣。可以为富安天下,而直为此廪廪也!窃为陛下\footnote{〔陛下〕指汉文帝刘恒(公元前180至157年在位)。}惜之。
\end{normalsize}



\chapter{苏武传}

\begin{normalsize}
    
    武字子卿,少以父任,兄弟并为郎\footnote{〔郎〕官名,汉代专指职位较低皇帝侍从。}。稍迁至栘中厩监。时汉连伐胡,数通使相窥观。匈奴留汉使郭吉\footnote{〔郭吉〕元封元年(公元前110年),汉武帝亲统大军十八万到北地,派郭吉到匈奴,晓谕单于归顺,单于大怒,扣留了郭吉。}、路充国\footnote{〔路充国〕元封四年(公元前107年),匈奴派遣使者至汉,病故。汉派路充国送丧到匈奴,单于以为是被汉杀死,扣留了路充国。}等,前后十余辈。匈奴使来,汉亦留之以相当。天汉元年\footnote{〔天汉〕汉武帝年号(公元前100至前年)。},且鞮侯\footnote{〔且鞮侯〕单于嗣位前的封号。}单于初立,恐汉袭之,乃曰:“汉天子我丈人行也。”尽归汉使路充国等。武帝嘉其义,乃遣武以中郎将\footnote{〔中郎将〕皇帝的侍卫长。}使持节送匈奴使留在汉者,因厚赂单于,答其善意。武与副中郎将张胜及假吏\footnote{〔假吏〕临时委任的使臣属官。}常惠等募士斥候\footnote{〔斥候〕军中警卫侦察的士兵。}百余人俱,既至匈奴,置币遗单于。单于益骄,非汉所望也。
    
    方欲发使送武等,会缑王\footnote{〔缑王〕匈奴的一个亲王。}与长水\footnote{〔长水〕水名,在今陕西省蓝田县西北。}虞常\footnote{〔虞常〕长水人,后投降匈奴。}等谋反匈奴中。缑王者,昆邪王\footnote{〔昆邪王〕匈奴一个部落的王,其领地在河西(今甘肃省西北部)。昆邪王于元狩二年降汉。}姊子也,与昆邪王俱降汉,后随浞野侯\footnote{〔浞野侯〕汉将赵破奴的封号。汉武帝太初二年(公元前103年)率二万骑击匈奴,兵败而降,全军沦没。}没胡中。及卫律\footnote{〔卫律〕本为长水胡人,但在汉地长大,被协律都尉李延年荐为汉使出使匈奴。回汉后,正值李延年因罪全家被捕,卫律怕受牵连,又逃奔匈奴,被封为丁零王。}所将降者,阴相与谋劫单于母阏氏\footnote{〔阏氏〕匈奴王后封号。}归汉。会武等至匈奴。虞常在汉时,素与副张胜相知,私候胜曰:“闻汉天子甚怨卫律,常能为汉伏弩射杀之,吾母与弟在汉,幸蒙其赏赐。”张胜许之,以货物与常。
    
    后月余,单于出猎,独阏氏子弟在。虞常等七十余人欲发,其一人夜亡,告之。单于子弟发兵与战,缑王等皆死,虞常生得。单于使卫律治其事,张胜闻之,恐前语发,以状语武。武曰:“事如此,此必及我。见犯乃死,重负国。”欲自杀,胜、惠共止之。虞常果引张胜。单于怒,召诸贵人议,欲杀汉使者。左伊秩訾\footnote{〔左伊秩訾〕匈奴的王号。}曰:“即谋单于,何以复加?宜皆降之。”
    
    单于使卫律召武受辞。武谓惠等:“屈节辱命,虽生,何面目以归汉!”引佩刀自刺。卫律惊,自抱持武,驰召医。凿地为坎,置煴火,覆武其上,蹈其背以出血。武气绝,半日复息。惠等哭,舆归营。单于壮其节,朝夕遣人候问武,而收系张胜。
    
    武益愈,单于使使晓武,会论虞常,欲因此时降武。剑斩虞常已,律曰:“汉使张胜谋杀单于近臣,当死。单于募降者赦罪。”举剑欲击之,胜请降。律谓武曰:“副有罪,当相坐。”武曰:“本无谋,又非亲属,何谓相坐?”复举剑拟之,武不动。律曰:“苏君,律前负汉归匈奴,幸蒙大恩,赐号称王,拥众数万,马畜弥山,富贵如此!苏君今日降,明日复然。空以身膏草野,谁复知之!”武不应。律曰:“君因我降,与君为兄弟;今不听吾计,后虽复欲见我,尚可得乎?”武骂律曰:“汝为人臣子,不顾恩义,畔主背亲,为降虏于蛮夷,何以汝为见!且单于信汝,使决人死生;不平心持正,反欲斗两主,观祸败!南越\footnote{〔南越〕南越国,现在广东、广西南部一带。}杀汉使者,屠为九郡。宛王\footnote{〔宛王〕指大宛国王毋寡。汉武帝太初元年(公元前104年),宛王毋寡派人杀前来求良马的汉使。武帝即命李广利讨伐大宛,大宛诸贵族乃杀毋寡而降汉。}杀汉使者,头县北阙。朝鲜杀汉使者,即时诛灭。独匈奴未耳。若知我不降明,欲令两国相攻。匈奴之祸,从我始矣。”
    
    律知武终不可胁,白单于。单于愈益欲降,乃幽武置大窖中,绝不饮食。天雨雪,武卧啮雪,与旃毛并咽之,数日不死。匈奴以为神,乃徙武北海\footnote{〔北海〕当时在匈奴北境,即今贝加尔湖。}上无人处,使牧羝,羝乳乃得归。别其官属常惠等各置他所。武既至海上,廪食不至,掘野鼠去草实而食之。杖汉节牧羊,卧起操持,节旄尽落。积五六年,单于弟於靬王\footnote{〔於靬王〕且鞮单于之弟。}弋射海上。武能网纺缴,檠弓弩,於靬王爱之,给其衣食。三岁余,王病,赐武马畜、服匿\footnote{〔服匿〕盛酒酪的容器,类似今天的坛子。}、穹庐\footnote{〔穹庐〕圆顶大篷帐,发展为现今的蒙古包。}。王死后,人众徙去。其冬,丁令\footnote{〔丁令〕即丁灵,匈奴北边的一个部族。}盗武牛羊,武复穷厄。
    
    初,武与李陵\footnote{〔李陵〕字少卿,西汉陇西成纪(今甘肃秦安)人,李广之孙,武帝时曾为侍中。天汉二年(前99年)出征匈奴,兵败投降,后病死匈奴。}俱为侍中\footnote{〔侍中〕官名,皇帝的侍从。}。武使匈奴,明年,陵降,不敢求武。久之,单于使陵至海上,为武置酒设乐。因谓武曰:“单于闻陵与子卿素厚,故使陵来说足下,虚心欲相待。终不得归汉,空自苦亡人之地,信义安所见乎?前长君\footnote{〔长君〕指苏武的长兄苏嘉。}为奉车\footnote{〔奉车〕即“奉车都尉”,皇帝出巡时,负责车马的侍从官。},从至雍\footnote{〔雍〕汉代县名,在今陕西凤翔县南。}棫阳宫\footnote{〔棫阳宫〕秦时所建宫殿,在雍东北。},扶辇下除,触柱折辕,劾大不敬\footnote{〔大不敬〕不敬皇帝,是大不敬罪。},伏剑自刎,赐钱二百万以葬。孺卿\footnote{〔孺卿〕苏武弟苏贤的字。}从祠河东\footnote{〔河东〕汉郡名,在今山西夏县北。}后土\footnote{〔后土〕地神。},宦骑与黄门驸马<footnote:N34>争船,推堕驸马河中溺死,宦骑亡,诏使孺卿逐捕,不得,惶恐饮药而死。来时太夫人\footnote{〔太夫人〕指苏武的母亲。}已不幸,陵送葬至阳陵\footnote{〔阳陵〕汉时有阳陵县,在今陕西咸阳市东。}。子卿妇年少,闻已更嫁矣。独有女弟二人,两女一男,今复十余年,存亡不可知。人生如朝露,何久自苦如此!陵始降时,忽忽如狂,自痛负汉,加以老母系保宫\footnote{〔保宫〕囚禁犯罪大臣及其眷属之处。}。子卿不欲降,何以过陵?且陛下春秋高,法令亡常,大臣亡罪夷灭者数十家,安危不可知,子卿尚复谁为乎?愿听陵计,勿复有云。”
    
    武曰:“武父子亡功德,皆为陛下所成就,位列将,爵通侯\footnote{〔通侯〕汉爵位名,本名彻侯,因避武帝讳改。苏武父苏建曾封为平陵侯。},兄弟亲近,常愿肝脑涂地。今得杀身自效,虽蒙斧钺汤镬,诚甘乐之。臣事君,犹子事父也。子为父死,亡所恨,愿无复再言!”
    
    陵与武饮数日,复曰:“子卿壹听陵言!”武曰:“自分已死久矣!王必欲降武,请毕今日之欢,效死于前!”陵见其至诚,喟然叹曰:“嗟呼,义士!陵与卫律之罪上通于天!”因泣下沾衿,与武决去。
    
    昭帝\footnote{〔昭帝〕汉昭帝刘弗陵,武帝少子(前87年至前74年在位)。即位次年改元始元。于始元六年(公元前81年),与匈奴达成和议。}即位,数年,匈奴与汉和亲。汉求武等,匈奴诡言武死。后汉使复至匈奴,常惠请其守者与俱,得夜见汉使,具自陈道。教使者谓单于,言天子射上林\footnote{〔上林〕即上林苑。故址在今陕西省西安市附近。汉朝皇帝游玩射猎的园林。}中,得雁足有系帛书,言武等在荒泽中。使者大喜,如惠语以让单于。单于视左右而惊,谢汉使曰:“武等实在。”
    
    单于召会武官属,前以降及物故,凡随武还者九人。武以始元六年\footnote{〔京师〕指西汉京城长安。}春至京师。武留匈奴凡十九岁,始以强壮出,及还,须发尽白。
\end{normalsize}



\chapter{原君}

\begin{normalsize}
    
    有生之初,人各自私也,人各自利也。天下有公利而莫或兴之,有公害而莫或除之。有人者出,不以一己之利为利,而使天下受其利;不以一己之害为害,而使天下释其害。此其人之勤劳,必千万于天下之人。夫以千万倍之勤劳,则己又不享其利,必非天下之人情所欲居也。故古人之君,量而不欲入者,许由\footnote{〔许由〕传说中尧欲让天下而不受的隐士,拒君位隐耕箕山。}、务光\footnote{〔务光〕商汤让位时投水拒受的隐士,象征淡泊名利的古贤。}是也;入而又去之者,尧、舜\footnote{〔尧、舜〕古时圣君,主动禅让君位。}是也;初不欲入而不得去者,禹\footnote{〔禹〕开启世袭制的君主,传位于其子启。}是也。岂古之人有所异哉?好逸恶劳,亦犹夫人之情也。
    
    后之为人君者不然。以为天下利害之权皆出于我,我以天下之利尽归于己,以天下之害尽归于人,亦无不可。使天下之人不敢自私,不敢自利,以我之大私为天下之公。始而惭焉,久而安焉,视天下为莫大之产业,传之子孙,受享无穷。汉高帝\footnote{〔汉高帝〕即汉高祖刘邦。}所谓“某业所就,孰与仲多”者,其逐利之情,不觉溢之于辞矣。
    
    此无他,古者以天下为主,君为客,凡君之所毕世而经营者,为天下也。今也以君为主,天下为客,凡天下之无地而得安宁者,为君也。是以其未得之也,屠毒天下之肝脑,离散天下之子女,以博我一人之产业,曾不惨然,曰:“我固为子孙创业也。”其既得之也,敲剥天下之骨髓,离散天下之子女,以奉我一人之淫乐,视为当然,曰:“此我产业之花息也。”然则为天下之大害者,君而已矣!向使无君,人各得自私也,人各得自利也。呜呼!岂设君之道固如是乎?
    
    古者天下之人爱戴其君,比之如父,拟之如天,诚不为过也。今也天下之人,怨恶其君,视之如寇仇,名之为独夫,固其所也。而小儒规规焉以君臣之义无所逃于天地之间,至桀纣\footnote{〔桀纣〕夏桀商纣,都是暴君。}之暴,犹谓汤武\footnote{〔汤武〕商汤周武王,儒家认可的"吊民伐罪"革命圣王典范。}不当诛之,而妄传伯夷、叔齐\footnote{〔伯夷、叔齐〕耻食周粟饿死首阳山的遗民,被用作盲目忠君的反面案例。}无稽之事,乃兆人万姓崩溃之血肉,曾不异夫腐鼠。岂天地之大,于兆人万姓之中,独私其一人一姓乎?是故武王圣人也,孟子之言,圣人之言也。后世之君,欲以如父如天之空名,禁人之窥伺者,皆不便于其言,至废孟子而不立,非导源于小儒乎?
    
    虽然,使后之为君者,果能保此产业,传之无穷,亦无怪乎其私之也。既以产业视之,人之欲得产业,谁不如我?摄缄縢,固扃鐍,一人之智力,不能胜天下欲得之者之众。远者数世,近者及身,其血肉之崩溃,在其子孙矣。昔人愿世世无生帝王家\footnote{〔昔人愿世世无生帝王家〕南朝宋孝武帝在位期间宠爱自己第八个儿子刘子鸾。太子刘子业妒其才,继位后赐死。子鸾临死谓左右曰:“愿身不复生王家。”宋顺帝刘准禅位于齐王,将军王敬则捉住刘准。刘准泣曰:“愿后身世世勿复生天王家!”},而毅宗\footnote{〔毅宗〕即明思宗朱由检(崇祯帝,公元1627-1644在位)。}之语公主,亦曰:“若何为生我家!”痛哉斯言!回思创业时,其欲得天下之心,有不废然摧沮者乎?是故明乎为君之职分,则唐、虞之世,人人能让,许由、务光非绝尘也;不明乎为君之职分,则市井之间,人人可欲,许由、务光所以旷后世而不闻也。然君之职分难明,以俄顷淫乐,不易无穷之悲,虽愚者亦明之矣。
\end{normalsize}


\newpage

\textbf{注解}:

\vspace{-1em}

\begin{itemize}
    \setlength\itemsep{-0.2em}
    \item〔某业所就,孰与仲多〕《史记·高祖本纪》载汉高祖刘邦登帝位后,曾对他父亲说:“您以前常说我是无赖,不能治理产业,不如仲(其兄刘仲)厉害,今天我成就的功业,和仲相比谁多?”这也说明刘邦把治国视为治理产业,把君位视为使用权力营私获利的工具,而不是服务人民的工具。
\end{itemize}

\chapter{乐毅报燕王书}

\begin{normalsize}
    
    昌国君乐毅,为燕昭王合五国之兵\footnote{〔五国之兵〕赵、楚、韩、燕、魏五国联军。}而攻齐,下七十馀城,尽郡县之以属燕。三城\footnote{〔三城〕指齐国的聊城、莒、即墨三城,都在今山东省。}未下,而燕昭王死。惠王即位,用齐人反间,疑乐毅,而使骑劫\footnote{〔骑劫〕燕国将领。}代之将。乐毅奔赵,赵封以为望诸君。齐田单\footnote{〔田单〕战国时齐国大将,屡立战功,封安平君,被齐襄王任为国相。}诈骑劫,卒败燕军,复收七十余城以复齐。
    
    燕王悔,惧赵用乐毅乘燕之弊以伐燕。燕王乃使人让乐毅,且谢之曰:“先王举国而委将军,将军为燕破齐,报先王之仇,天下莫不振动。寡人岂敢一日而忘将军之功哉!会先王弃群臣,寡人新即位,左右误寡人。寡人之使骑劫代将军,为将军久暴露于外,故召将军,且休计事。将军过听,以与寡人有隙,遂捐燕而归赵。将军自为计则可矣,而亦何以报先王之所以遇将军之意乎?”
    
    望诸君\footnote{〔望诸君〕赵国给乐毅的封号。}乃使人献书报燕王曰:臣不佞,不能奉承先王\footnote{〔先王〕即燕惠王之父燕昭王。}之教,以顺左右之心,恐抵斧质之罪,以伤先王之明,而又害于足下之义,故遁逃奔赵。自负以不肖之罪,故不敢为辞说。今王使使者数之罪,臣恐侍御者\footnote{〔侍御者〕侍侯国君的人,实指燕惠王。}之不察先王之所以畜幸臣之理,而又不白于臣之所以事先王之心,故敢以书对。
    
    臣闻贤圣之君不以禄私其亲,功多者授之;不以官随其爱,能当者处之。故察能而授官者,成功之君也;论行而结交者,立名之士也。臣以所学者观之,先王之举错,有高世之心,故假节\footnote{〔节〕外交使臣所持之凭证。}于魏王,而以身得察于燕。先王过举,擢之乎宾客之中,而立之乎群臣之上,不谋于父兄,而使臣为亚卿\footnote{〔亚卿〕官名,地位仅次于上卿。}。臣自以为奉令承教,可以幸无罪矣,故受命而不辞。先王命之曰:“我有积怨深怒于齐,不量轻弱,而欲以齐为事。”臣对曰:“夫齐,霸国\footnote{〔霸国〕齐桓公曾称霸诸侯,故称齐国为霸国。}之余教而骤胜之遗事也,闲于甲兵,习于战攻。王若欲伐之,则必举天下而图之。举天下而图之,莫径于结赵矣。且又淮北\footnote{〔淮北〕淮河以北地区,是齐国属地。}、宋地\footnote{〔宋地〕今江苏铜山、河南商丘、山东曲阜之间的地区,为齐所吞并。},楚、魏之所同愿也。赵若许约,楚、赵、宋尽力,四国攻之,齐可大破也。”先王曰:“善。”臣乃口受令,具符节,南使臣于赵。顾反命,起兵随而攻齐,以天之道,先王之灵,河北\footnote{〔河北〕黄河以北。}之地,随先王举而有之于济\footnote{〔济〕济水,又名兖水,沇水,是黄河下游的一条重要支流。于河南荥阳与黄河干流分离,东流至山东济宁以北入巨野泽,东北于济南经今黄河河道入海。曾与江水(长江)、河水(黄河)、淮水(淮河)并称“四渎”。1855年后由于黄河改道而淤塞消失。}上。济上之军奉令击齐,大胜之。轻卒锐兵,长驱至国。齐王逃遁走莒\footnote{〔莒〕今山东莒县。},仅以身免。珠玉财宝,车甲珍器,尽收入燕。大吕陈于元英\footnote{〔元英〕燕国宫殿名。后面的“历室”、“宁台”也是。},故鼎\footnote{〔故鼎〕指齐国掠夺的燕鼎,复归燕国。}反乎历室,齐器设于宁台。蓟丘\footnote{〔蓟丘〕燕国都城,今北京市西南。}之植,植于汶篁\footnote{〔汶〕汶水,流经齐鲁两国的水名,在今山东中部,又名大汶河。汶水流域是齐、魏反复争夺之地。}。自五伯\footnote{〔五伯〕指五霸,春秋时五位权倾诸侯的君主。}以来,功未有及先王者也。先王以为顺于其志,以臣为不顿命,故裂地而封之,使之得比乎小国诸侯。臣不佞,自以为奉令承教,可以幸无罪矣,故受命而弗辞。
    
    臣闻贤明之君,功立而不废,故著于《春秋》\footnote{〔春秋〕古代编年史都叫春秋。},蚤知之士,名成而不毁,故称于后世。若先王之报怨雪耻,夷万乘之强国,收八百岁\footnote{〔八百岁〕从姜太公建国到这次战争约八百年。}之蓄积,及至弃群臣之日,遗令诏后嗣之馀义,执政任事之臣,所以能循法令,顺庶孽者,施及萌隶,皆可以教于后世。
    
    臣闻善作者不必善成,善始者不必善终。昔者伍子胥说听乎阖闾\footnote{〔阖闾〕春秋时吴国国王,继位者是夫差。},故吴王远迹至于郢\footnote{〔郢〕楚国都城,今湖北江陵西北。};夫差弗是也,赐之鸱夷\footnote{〔鸱夷〕皮革制的口袋。}而浮之江。故吴王夫差不悟先论之可以立功,故沉子胥而弗悔;子胥不蚤见主之不同量,故入江而不改。夫免身全功,以明先王之迹者,臣之上计也。离毁辱之非,堕先王之名者,臣之所大恐也。临不测之罪,以幸为利者,义之所不敢出也。
    
    臣闻古之君子,交绝不出恶声;忠臣之去也,不洁其名。臣虽不佞,数奉教于君子矣。恐侍御者之亲左右之说,而不察疏远之行也。故敢以书报,唯君之留意焉。
\end{normalsize}



\end{document}
