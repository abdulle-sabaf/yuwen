\documentclass[12pt,UTF-8,openany]{ctexbook}
\usepackage{ctex}
\usepackage{titlesec}
\usepackage{xeCJK}
\usepackage{verse}
\usepackage{fontspec,xunicode,xltxtra}
\usepackage{xpinyin}
\usepackage{geometry}
\usepackage{indentfirst}
\usepackage{pifont}
\usepackage{enumitem}
\usepackage[perpage,symbol*]{footmisc}
\usepackage[table,dvipsnames]{xcolor}

\geometry{a5paper,left=1.4cm,right=1.4cm,top=2.3cm,bottom=2.3cm}
\renewcommand{\footnotesize}{\fontsize{8.5pt}{10.5pt}\selectfont}
\setmainfont{Mona Sans Light}
\setCJKmainfont[BoldFont=STZhongsong]{汉字之美仿宋GBK 免费}
\xeCJKDeclareCharClass{CJK}{`0 -> `9}
\xeCJKsetup{AllowBreakBetweenPuncts=true}
\DefineFNsymbols{circled}{{\ding{192}}{\ding{193}}{\ding{194}}{\ding{195}}{\ding{196}}{\ding{197}}{\ding{198}}{\ding{199}}{\ding{200}}{\ding{201}}}
\setfnsymbol{circled}
\xpinyinsetup{ratio=0.5,hsep={.6em plus .6em},vsep={1em}}

\titleformat{\chapter}{\zihao{-1}\bfseries}{ }{16pt}{}
\titleformat{\section}{\zihao{-2}\bfseries}{ }{0pt}{}
\title{\zihao{0} \bfseries 中学古文课文集萃}
\setlength{\lineskip}{24pt}
\setlength{\parskip}{6pt}
\author{}
\date{}
\begin{document}
\maketitle
\tableofcontents
\newpage

\chapter{《论语》十则}

\begin{normalsize}
    
    子曰:“学而时习之,不亦说乎?有朋自远方来,不亦乐乎?人不知而不愠,不亦君子\footnote{〔君子〕君王的儿子,泛指道德高尚的人。}乎?”
    
    子曰:“我非生而知之者,好古,敏以求之者也。”
    
    曾子\footnote{〔曾子〕曾参,字子舆,春秋时鲁国人,孔子的弟子,孔子学说的主要继承人,参与编制了《论语》。}曰:“吾日三省吾身:为人谋而不忠乎?与朋友交而不信乎?传不习乎?”
    
    子曰:“温故而知新,可以为师矣。”
    
    子曰:“学而不思则罔,思而不学则殆。”
    
    子曰:“由\footnote{〔由〕仲由,字子路,春秋时鲁国人,孔子的弟子。忠厚正直,力大勇武。},诲汝知之乎!知之为知之,不知为不知,是知也。”
    
    子曰:“见贤思齐焉,见不贤而内自省也。”
    
    子曰:“三人行,必有我师焉。择其善者而从之,其不善者而改之。”
    
    子贡\footnote{〔子贡〕端木赐,字子贡,春秋时卫国人,孔子的弟子。贤能善辩,曾任鲁国、卫国的相国,还善于经商。}问曰:“有一言而可以终身行之者乎?”子曰:“其恕乎!己所不欲,勿施于人。”
    
    子在川上曰:“逝者如斯夫,不舍昼夜。”
\end{normalsize}


\newpage

\textbf{译文}:

\vspace{1em}

\begin{normalsize}
    
    孔子说:“学习并时常复习,不也很愉快吗?有朋友从远方来,不也很高兴吗?别人不了解我,我却不生气,不也是君子吗?”
    
    孔子说:“我不是生来就知晓(道理),而是喜欢(了解)古时候(的事),努力地设法获得了这些知识。”
    
    曾子说:“我每天多次反省自己:替别人办事是不是尽心尽力了呢?和朋友交往是不是诚实守信了呢?老师传授的知识是不是复习了呢?”
    
    孔子说:“(既能)积累以往(的学说),又知道新的(知识),(那么就)可以做老师了。”
    
    孔子说:“只学习而不思考就会迷惑,只思考而不学习就会怠惰。”
    
    孔子说:“由,(我)教你(的东西)知道了吗?知道就是知道,不知道就是不知道,这才是知道了。”
    
    孔子说:“看见贤明的人要想着(向他)看齐,看见不贤明的人,要在心里自己反省。”
    
    孔子说:“几个人走在一起,(其中)必定有(可以做)我的老师的。找出他们做得好的,就跟着去做,发现做得不好的,就(自己)改正它。”
    
    子贡问孔子:“有没有一句话可以终身奉行的呢?”孔子说:“那就是恕吧!自己不愿意的,不要施加给别人。
    
    孔子在河边感叹道:“流逝的时光就像这河水一样啊,日夜不停。”
    
\end{normalsize}


\newpage

\textbf{注解}:

\vspace{-1em}

\begin{itemize}
    \setlength\itemsep{-0.2em}
    \item〔不亦君子乎〕君王的儿子,表示将来要做君王的人。儒是向统治者解释经书的人。因此儒家讲述道理的对象和理想典范都叫做“君子”,因为儒家认为只有道德高尚,符合儒家学说道德要求和规范的人,才是将来可以做君王的人。
    \item〔吾日三省吾身〕古人以三为约数,表示大概的“好几次”、“不止一两次”,但又明显少于十次的次数。后面“三人行”也是如此。
    \item〔知之为知之,不知为不知,是知也〕什么叫“知道”?不仅要知道自己知道什么,还要知道自己不知道什么。这样才叫知道了。明明知道,却以为自己不知道;或者明明自己不知道,却以为自己知道,都不能叫做真的知道。《论语注疏》:“此章明知也。孔子以子路性刚,好以不知为知,故此抑之。”《荀子·儒效篇》:“知之曰知之,不知曰不知,内不以自诬,外不以自欺,以是尊贤畏法而不敢怠慢,是雅儒者也。”
    \item〔择其善者而从之,其不善者而改之〕找出别人做得好的,就跟着去做,发现别人做得不好的,如果自己也有这个毛病,就(自己)改正它。通过观察他人的长处和短处,就能学到东西,自我改进,这就是“有我师”,而不是真的说在几个人里面找到一个人做我的老师。
    \item〔其恕乎!己所不欲,勿施于人〕这是后世“忠恕”的来源之一。忠,谓尽心为人;恕,谓推己及人。“恕”者“如”也,像对待自己一样对待他人就是“恕”。“己所不欲,勿施于人”就是典型的“恕”的表现。“恕”的本义不是现代汉语中的“宽恕”、“饶恕”的意思。因此“忠恕”也不是说要无条件地宽恕别人,做滥好人。
\end{itemize}

\chapter{使琉球记}

\begin{normalsize}
    
    出使琉球
    
    初八日己丑,晴。午风大。黎明,有二白鸟绕船而飞。午刻,丁风,仍用辰针,计行四更。申刻,过米糠洋。漩皆圆,波浪密而细,如初筛之米,点点零落;“米糠”字,极有形容。日落,计又行三更;船伙长\footnote{〔伙长〕船上掌管罗盘的人。}云:“鸡笼山、花瓶屿去船远,不应见”。是夜,用乙辰针,行船六更。舟中吐者甚多;余日坐将台\footnote{〔将台〕阅兵点将台,指挥者坐的位置。现在称为舰桥。},全不觉险,饮食如常。
    
    初九日庚寅,晴。卯刻,见彭家山\footnote{〔彭家山〕即彭佳屿,在台湾省基隆港北约30海里。}。山列三峰,东高而西下。计自开洋,行船十六更矣;由山北过船。辰刻,转丁未风,用单乙针,行十更船。申正,见钓鱼台\footnote{〔钓鱼台〕即钓鱼岛,在彭佳屿东约80海里,属基隆。},三峰离立如笔架,皆石骨。惟时水天一色,舟平而驶;有白鸟无数绕船而送,不知所自来。入夜,星影横斜,月色破碎,海面尽作火焰,浮沉出没。
    
    初十日辛卯,晴。丁未风,仍用单乙针。东方黑云蔽日,水面白鸟无数。计彭家至此,行船十四更。辰正,见赤尾屿;屿方而赤,东西凸而中凹,凹中又有小峰二。船从山北过。有大鱼二,夹舟行,不见首尾,脊黑而微绿,如十围枯木附于舟侧;舟人举酒相庆。
    
    十一日壬辰,阴。丁未风,仍用单乙针。计赤尾屿至此,行十四更船。午刻,见姑米山。山共八岭,岭各一、二峰,或断或续;舟中人欢声沸海。未刻,大风,暴雨如注,然雨虽暴而风顺。酉刻,舟已近山,计又行五更船。球人以姑米多礁,黑夜不敢进,待明而行。丑刻,有小船来引导;乃放舟由山南行。
    
    十二日癸巳,晴。辰刻,过马齿山。山如犬牙相错,四峰离立,若马行空。计又行七更,船再用甲寅针,取那霸港。考历来针路所见,尚有小琉球、鸡笼山、黄麻屿;此行俱未见。问之琉球伙长,年已六十,往来海面八次,云此次最为简捷,而所见亦仅三山,即至姑米。惟纪更以香,殊难为据。据琉球伙长云:“海上行舟,风小固不能驶,风过大亦不能驶;风大则浪大,浪大力能壅船,进尺仍退二寸。惟风七分、浪五分,最宜驾驶;此次是也。从来渡海,未有平稳而驶如此者。”辰刻,进那霸港。午刻,登岸。倾国人士聚观于路,世孙\footnote{〔世孙〕指琉球中山王国第十五代国王尚温。当时琉球第十四代国王尚穆病逝,尚穆的儿子尚哲已死,李鼎元奉旨到琉球册封尚穆的孙子尚温为下一代国王。由于册封还没完成,所以称为世孙。}率百官迎诏如仪。
    
    琉球见闻
    
    十七日戊戌,晴。阅案头食单,有所谓“龙头虾”者。取视之,长尺馀,绦甲朱髯、血睛火鬣,类世所画龙头,见之悚然!取其壳以为灯,可供两日玩;三日而色变矣。
    
    十八日己亥,雨。栽荔枝于使院庭后,南北分列。移自牧荔园,种曰“陈家紫”。
    
    二十一日壬寅,阴。连日食海味,腹渐作泻。令庖人但供时蔬、淡粥。庖人乃以佳苏鱼进;问之,曰:“此非鱼名也,系削黑鳗鱼脊肉,干而为之”。长五、六寸许,形如梭、质如枯木。食法,先以温水浸洗,裹蕉叶煨之,切片如刨花,连五、七片不断,又如兰花;宜清酱\footnote{〔清酱〕即“酱清”,生抽酱油。},颇有异味。清酱甘美,十倍于闽。惟求“佳苏”之义不得,适有长史至,问亦不解;因呼至前细核之,据云“此品在敝国既多且美,自王官以及贫民皆得食”;意殆谓如家常蔬菜,人人得食也。球人字皆对音,殆实为“家蔬”也。
    
    二十二日癸卯,晴。午后,偕介山\footnote{〔介山〕指这次出使琉球的大使赵文楷,字介山。李鼎元是副使。}策骑游波上山。一名石筍崖,以形似名之也;石垣四周,垣后可望海,沿海多浮石,嵌空玲珑;潮水击之,声作钟磬。东北有山,曰雪崎。又东北,有小石山,曰龟山。稍下为护国寺,国王祷雨之所也。龛内有神,黑而裸,手剑立,状甚狞;名曰“不动”,或曰火神。庭中有景泰七年铸钟一,廊下又有乾隆五十七年新铸钟一。寺后多凤尾蕉,一名铁树。西有石,高五、六尺,黑而润,状如骈佛手。因书“仙人掌”三字于上。
    
    初二日癸丑,大暑,阴。从官往游泊村,归以新稻穗见示,云稻已尽收;乃知球阳地气温暖,稻常早熟,种以十一月,收以五、六月。薯则四时皆种,三熟为丰,四熟则为大丰。稻田少、薯田多。国人以薯为命,米则王官始得食。亦有麦、豆,所产不多。薯一名地瓜,闽人土语。午后,微雨。
    
    初五日丙辰,阴。巳后,大雨。长史\footnote{〔长史〕职官名,相当于幕僚长。}送佛桑\footnote{〔佛桑〕朱槿,也叫赤槿、佛桑、红扶桑、大红花,原产于中国南部,广泛分布于亚洲的观赏花卉。}四株。一种千层如榴,有深红、粉红二色。一种单层,花如灯盘,蕊单出如烛,长二寸许,有红、白二色;朝开暮落,落则瓣卷如烛。花而不实;四季有花,深冬叶始凋谢。此地花开四季者甚多,气暖故也。余感长史意,嘱从客酬以酒;意有花再相致耳。
    
    初七日戊午,晴。辰刻,微雨,旋止。长史复以花二盆见贻,标曰“水翁花”;视之,乃马兰花也。中山草木,多与中朝\footnote{〔中朝〕中央王朝,指中国内地。}异称;盖因国中少书,多不识古来草木之名。如罗汉松,谓之㭴木;冬青,谓之福木;万寿菊,谓之禅菊:其初以意名之,后遂相沿不改。惜未携《群芳谱》来,一一证辨之耳!
    
    十四日乙丑,阴。荔枝栽近一月,新叶茂发,有生机矣。早起,偶步其侧,见新叶有蚀者;薄视之,有虫黄体而苍文,两角、八足,身方而毛,世所谓毛虫类。附叶为巢,蒙如小蛛网;卵生如蚕子而速,大者二寸以来。命仆捉而坑之,尽扫其巢。
    
    二十九日庚辰,晴。是日初见五彩鱼\footnote{〔五彩鱼〕即花斑连鳍䲗,俗名七彩麒麟、五彩青蛙。}。有红绿翠黄诸色,绿鳞红章,五彩相间。土人就形色呼之,无定名。又有一石眉巴鱼\footnote{〔石眉巴鱼〕可能是红鳍笛鲷。},色红如金鱼。余俱不敢食,养盎中以为玩品。又有鳐如白鸟,云飞丈馀始入水,疑即燕鱼\footnote{〔燕鱼〕渤海地区对蓝点马鲛的称呼。}也。
    
    初六日丙戌,大风。是日,食品有蕉实,状如手指,不相属。色黄,味甘,瓤如柚,亦名甘露。闻初熟色青,以糠覆之则黄,与中国制柿无异。其花红,一穗数尺,瓣须五六出。岁实为常,实如其须之数。中国亦有蕉,不闻岁结实,亦无有抽其丝作布者;或其性殊欤?
    
    行成归来
    
    二十日己巳,晴。东北风利,促解缆。卯刻,扬帆出那霸港;岸上、舟中送者如云,举手辞谢之。午刻,雨,入暮不止。伙长恐有暴,收马齿山安护浦下碇。山势横袤二十里,犬牙相错,出没海中,若断若续;分东、西二岛,为中山\footnote{〔中山〕指琉球国。琉球国全名为琉球中山王国。}第一外障。泊处青山围绕,无出路。有鹿见于山间,疑亦海鱼所化。雨景大佳。
    
    二十二日辛未,雨,风仍西北。午刻,晴。偕介山驾小舟登岸;沿沙洲行至山麓,有石高丈馀,玲珑可爱。坐石上观渔,皆赤身入水,无寒色。马齿人善泅,习使然也。
    
    二十四日癸酉,晴。北风少平,促伙长出洋;对以“风信未定”。余曰:“风信定,能无变乎?可行,则行!”介山曰:“姑俟之!”遂止。
    
    二十五日甲戌,晴。北风如故,决令开帆,介山亦以为然;遂于巳刻解缆。子丑风,用辛针。酉刻,过姑米山。终日峭帆,舟转驶,微侧而震,有吐者;余仍日坐将台,饮食如故。
    
    二十九日戊寅,辰卯风微,大雾,针如故。巳刻,稍霁;见温州南杞山\footnote{〔南杞山〕现称南麂岛,在浙江温州市平阳县东南海面。},舟人大喜。少顷,见杞山北有船数十只泊焉;舟人皆喜曰:“此必迎护船也!”雾渐消,山渐近;守备\footnote{〔守备〕清朝武官名,正五品,管理军队总务、军饷、军粮。}登后艄以望,惊报曰:“泊者,贼船也!”余曰:“舟已至此,戒兵无哗!速食,备器械!”余亦饱食。守备又报贼船皆扬帆矣;与介山衣冠出,令吐者、病者悉归舱;登战台,誓众曰:“贼众我寡,尔等未免胆怯。然贼船小、我船大,彼络绎开帆,纵善驾驶,不能并集,犹一与一之势也。且既已遇之,惧亦无益!惟有以死相拼,可望死中求活。此我与汝致命之秋也,生死共之!”众兵勇气顿振,皆曰“惟命!”乃下令曰:“贼船未及三百步,不得放子母炮;未及八十步,不得放枪;未及四十步,不得放箭。如果近,始用长枪相拼。有能毙贼者,重赏;违者,按以军法”。各整暇以俟。
    
    未几,贼船十六只吆喝而来,第一只已入三百步。余举旗麾之,吴得进从舵门放子母炮,立毙四人,击喝者堕海;贼退不及,入百步,枪并发,又毙六人。一只乃退,二只又入三百步,复以炮击之,毙五人;稍进,又击之,复毙四人,乃退去。其时,三只贼船已占上风;暗移子母炮至舵右舷边,连毙贼十二人,焚其头蓬:皆转舵而退。中二船较大,复鼓噪由上风飞至。余曰:“此必贼首也!”密令舵工将船稍横,俟大炮准对贼船,即施放一发,中之。炮响后,烟迷里许;既散,则贼船巳尽退。是役也,王得禄首先士卒,兵丁吴得进、陈成德、林安顺、张大良、王名标、甘耀等枪炮俱无虚发,幸免于危。惟时日将暮,风甚微;恐贼乘夜来袭,默祷于天后\footnote{〔天后〕即妈祖,俗称“海神娘娘”,东南沿海民间崇拜的神灵。}求风。不一时,北风大至,浪飞过船。余倦极,思卧。念前险假遇害,岂复能虑此险!况求风得风,即忧亦无着力处。遂解衣熟睡,付之不见不闻。
    
    十一月朔日己卯,阴。梦中闻舟人哗曰:“到官塘\footnote{〔官塘〕官方的港口。塘:堤岸。}矣!”惊起。介山、从客皆一夜不眠,语余曰:“险至此,服汝能睡。设葬鱼腹,亦为糊涂鬼矣!”余曰:“险奈何”?介山曰:“上则九天\footnote{〔九天〕传说古代天地有九重,“九天”天的最高处,比喻极高处。也作“九重天”、“九霄”。后面“九地”指地的最深处,比喻极低处。},下则九地,声如转水车、锯湿木,时复疟颤;每侧,则篷皆卧水。一浪盖船,则船身入水,惟闻瀑布声垂流不息;其不覆者,幸耳!”余曰:“脱覆,君等能免之乎!余乐拾得一觉,又忘其险,余幸矣!”介山乃大笑。舟人指曰:“前即定海\footnote{〔定海〕指现在浙江省舟山市定海区。清代重要的港口和军事要塞。},可无虑!”申刻,乃得泊。总兵\footnote{〔总兵〕明清武官名。清代总兵统领一地汉军,正二品。}何定江来迎护,余笑谢之。因语以北杞之战,定江惶悚失措;余曰:“馁矣!他事且缓商”。
\end{normalsize}


\newpage

\textbf{译文}:

\vspace{1em}

\begin{normalsize}
    
    出使琉球
    
    初八日己丑,晴。中午风很大。黎明,有两只白鸟绕船飞翔。中午吹西南风,仍朝东南航行,总计航行了四更。下午四点,经过米糠洋。(海里的)漩涡都是圆的,波浪密而细,好像初筛的米,点点零落;“米糠”两字,非常形象。日落之后,总计又航行了三更;船伙长说:“鸡笼山、花瓶屿离船很远,应该见不到了。”这晚上,朝东南方向,航行了六更。船上呕吐的人很多;我天天坐在舰桥上,完全不觉得颠簸,饮食和平常一样。
    
    初九日庚寅,晴。早上五六点时,望见彭家山。彭家山并列三座峰,东峰最高,往西则低下。从出海开始算,船已经航行了十六更了;船从山北过。早上七八点时,转为西南风,船向转东,又航行了十更。下午四点正,望见钓鱼台,三座山峰分立,好像笔架,都是石头。当时水天一色,船平行岸边行驶,有无数白鸟绕船相送,也不知从哪里来的。入夜,星星的倒影(随波)横斜,月色(被浪)破成碎片,海面仿佛到处是点点火焰,浮沉出没。
    
    初十日辛卯,晴。吹西南风,仍朝东航行。东方黑云蔽日,水面上有白鸟无数。从彭家山到这里,合计航行了十四更。早上八点,看见赤尾屿;岛屿是方的,红色,东西向凸出而中部凹进,凹进的地方又有两个小山峰。船从山北经过。有两条大鱼,夹着船游着,看不见首尾,(鱼)背脊黑而微绿,好像十围粗的枯木附在船两侧;船上的人举酒相庆。
    
    十一日壬辰,阴。吹西南风,仍朝东航行。从赤尾屿到这里,合计航行了十四更。中午,看见姑米山。(姑米)山一共八个岭,每个岭各有一、二个山峰,有的断开,有的连续;船上的人欢喜的呼声让海都沸腾了。午后一两点的时候,(刮起了)大风,暴雨如注,但是雨下得虽然大,但风并不大。傍晚五六点,船已经靠近(姑米)山,合计又航行了五更。琉球人因为姑米山(附近)暗礁多,黑夜里不敢进港,等到天亮了才接着航行。下半夜一两点的时候,有小船来引导;于是起锚绕着(姑米)山继续往南航行。
    
    十二日癸巳,晴。早上七八点,经过马齿山。山势如同犬牙交错,有四个山峰相离而矗立,好像马在空中跑。合计又航行了七更,船转向东北航行,向那霸港驶去。考察以往的航海记录里所见到的地方,还有小琉球、鸡笼山、黄麻屿;这次航行都没有见到。向琉球伙长询问,(他)已经六十岁了,往来出海八次,说这次航行航路最简单快捷,而我们(途中)也只看见三座山,(然后)就到了姑米山。只是由于用烧香来记录更数,很难用来作为依据。据琉球伙长说:“海上行船,风太小固然不方便航行,风太大也不方便航行;风大则浪大,浪太大,其力量能把船堵在海上,进一尺就要退二寸。只有风七分、浪五分的情况,最适宜驾驶;这次就是这样。向来渡海,没有试过像这次一样平稳航行的。”早上八点,船进入那霸港。中午,(我们)上岸了。(琉球)全国的人都聚集来,在路上观看,世孙率领百官按照礼仪迎接(皇帝的)诏书。
    
    琉球见闻
    
    十七日戊戌,晴。我翻阅桌子上的菜单,发现有所谓的“龙头虾”。取来一看,长一尺有余,甲壳一节一节的,有红色的长须,眼睛像血滴,鬃毛像火一样,就像世人所画的龙头,看了令人毛骨悚然!取它的壳来做灯,可以拿来玩两天;第三天颜色就变了。
    
    十八日己亥,雨。在使馆后庭栽了荔枝,南北各(栽种了)一列。(荔枝)是从牧荔园移植来的,品种叫“陈家紫”。
    
    二十一日壬寅,阴。连续几天吃海鲜,渐渐腹泻了。让厨子只供应时令蔬菜和稀粥。于是厨子做了佳苏鱼;问他,他说:“这不是鱼的名字,是削了黑鳗鱼的脊肉,腌干做成的。”(佳苏鱼)大概五、六寸长,形状像梭子、质地像枯木。使用的方法是先用温水浸洗,用蕉叶裹着煨,然后像刨花一样切片,(能)连切五、七片不断,好像兰花一样;最好沾酱油来吃,颇有异国风味。(琉球的)酱油,比福建甜美十倍。只是(我)想知道“佳苏”之含义而不得,恰好长史过来,问(他),(他)也不知道;于是把(厨子)叫到跟前仔细查核,据(厨子)说,“这个品种在我国又多又好吃,从国王、官员到穷人都能吃到”;意思大概是说它就好像家常蔬菜,人人都能吃到。琉球人用字都只求对上音就好,大概(佳苏鱼)本来应该叫“家蔬(鱼)”。
    
    二十二日癸卯,晴。午后,和介山一起策马到波上山游玩。(波上山)又叫石筍崖,因为形似而得名;(波上宫)四周是石墙,墙后可以望海。沿海很多露出水面的石头,嵌有空洞,玲珑剔透;潮水冲击它,发出钟磬一样的声音。东北边有一座山,叫雪崎山。再往东北,有一座小石山,叫龟山。往下走是护国寺,是国王祈雨的地方。龛内有黑色裸体的神像,手像剑一样竖立,形状很狰狞,名叫“不动”,有人说是火神。庭院中有一个景泰七年铸的钟,廊屋下又有一个乾隆五十七年新铸的钟。(护国)寺后有很多凤尾蕉,又叫铁树。西边有一块高五六尺的岩石,是黑色的,质地温润,形状像连起来的佛手。
    
    初二日癸丑,大暑,阴。下属官员到泊村游玩,回来拿着新稻穗给我看,说稻谷已经收完了;(我)才知道琉球山南边的土地由于气候温暖,稻谷经常早熟,十一月种稻谷,五、六月就收成了。红薯则四季都种,一年三熟就是丰收,一年四熟就是大丰收。稻田少、薯田多。琉球国人把红薯作为救命粮食,大米则是国王和官员才吃的起的。还有小麦和黄豆,产的不多。红薯在闽人的方言里也叫地瓜。午后,微雨。
    
    初五日丙辰,阴。早上十点多,下大雨。长史送来了四株佛桑。一株有千层花瓣,好像石榴,有深红、粉红二种颜色。一株花瓣是单层的,好像灯盘,花蕊单出,好像蜡烛,约二寸长,有红、白二种颜色;早上开花,傍晚就谢了,落下的花瓣卷起来像蜡烛。(佛桑)只开花不结果;四季都有花,到了深冬才落叶。这个地方很多花四季都开,因为气候温暖。我被长史的好意感动,吩咐下属官员用酒作为谢礼;希望(长史)下次有花再送来。
    
    初七日戊午,晴。早上七八点,微雨,很快就停了。长史又带了两盆花来送我,标的名字是“水翁花”;(我)看了看,其实是马兰花。中山国的草木,大多和中国称呼不一样;因为中山国书籍少,大多不认识自古以来草木的名字。比如把罗汉松叫做㭴木;把冬青叫做福木;把万寿菊叫做禅菊:一开始按自己的想法命名,之后就沿用不改了。可惜(我)没带《群芳谱》来,(没法)一一考证辨识它们!
    
    十四日乙丑,阴。荔枝种下已经接近一个月了,长出了很多新叶子,有生机了。早上起来,偶而走过它旁边,看到新叶子上有虫咬的洞;靠近来看,有一种身体黄色而有黑色纹路的虫子,有两个触角、八只脚,虫身方而有毛,即世人所说的毛虫一类。(它)附在叶子上作巢,像小蜘蛛网一样盖在叶子上;(它)产卵像蚕,但产得更快,大的有两寸多。让仆人抓了埋到坑里,把它的巢全部扫除了。
    
    二十九日庚辰,晴。这一天,第一次见到五彩鱼。(五彩鱼)有红绿翠黄等各种颜色,绿鳞红纹,五彩相间。当地人根据形状和颜色来称呼它们,没有固定的名字。又有一种叫石眉巴鱼的,像金鱼一样,是红色的。我哪个都不敢吃,养在盎里赏玩品味。还有鳐鱼,像白色的鸟一样,可以飞出水面一丈多再入水,(我)怀疑就是燕鱼。
    
    初六日丙戌,大风。这一天,吃的东西里有香蕉果实,形状像手指一样,不相连。(它)是黄色的,吃起来甜,果肉像柚子,也叫甘露。听说刚熟的时候是青色的,用米糠盖在上面就变黄了,和中国处理柿子没有不同。它的花是红的,一穗花有数尺长,花瓣有五六须。一般来说每年都结果,(每穗)果实数量和须一样。中国也有香蕉,但没听过每年都结果的,也没有听过抽丝用来织布的;难道是(水土不同导致的)性状差异吗?
    
    行成归来
    
    二十日己巳,晴。东北风很不错,(我)催促解缆开船。早上五点多,(我们)扬帆驶出那霸港;岸上、船上送的人很多,(我们)挥手辞谢。中午下雨,到了傍晚也没停。伙长怕有风暴,收了帆停靠在马齿山安护浦码头。(马齿山)山势横着有二十里宽,犬牙相错,出没海中,仿佛断开又仿佛连续;(马齿山)分东、西两个岛,是中山国对外的第一屏障。停泊的地方青山围绕,没有出路。有鹿出现在山里,(我)怀疑是海鱼化成的。雨中景色非常好看。
    
    二十二日辛未,雨,仍然吹西北风。中午天气放晴。我和介山驾小船上岸;沿沙洲走到山麓,有一块一丈多高的石,玲珑可爱。坐在石头上看渔民(劳作),都是赤裸着身子入水,不觉得寒冷。马齿人水性好,是从小练出来的。
    
    二十四日癸酉,晴。北风稍微变小了点,(我)催促伙长出海;(伙长)回答说“还没确定风的规律”。我说:“(即便现在)确定了风的规律,就不会再变吗?(现在)可以走,就走!”介山说:“姑且再等等吧!”于是作罢。
    
    二十五日甲戌,晴。仍然吹北风,(我)决定下令开船,介山也同意;于是在九点解缆开船。吹北风,朝西航行。下午五六点,经过姑米山。(船夫)一整天都在操帆,船转来转去地航行,略微倾侧且颠簸,有人呕吐;我仍然每天坐在将台,照常饮食。
    
    二十九日戊寅,早上七点后,吹微弱的东风,大雾弥漫,仍按原定罗盘针位航行。九点以后,天气稍有放晴;看见温州南杞山,船夫们非常高兴。不久后,看见杞山北边有数十只船停泊在那里;船夫们都高兴地说:“这一定是前来迎接和护送我们的船!”雾气逐渐消散,山也渐渐靠近;守备登上船尾眺望,惊慌地报告说:“停泊的船是海盗船!”我说:“船已经到了这里,命令士兵不要喧哗!赶快吃饭,准备武器!”我也饱餐一顿。守备又报告说海盗船都扬起了帆;我和介山整理衣冠后走出船舱,命令呕吐和生病的人都回到舱内;登上战斗平台,向众人发誓说:“敌人众多,我们人少,你们难免会感到胆怯。但敌船小,我们的船大,他们扬帆有先后,即使擅长驾驶,也无法同时齐聚,仍然是以一对一的态势。而且既然已经遇到了敌人,害怕也没有用!只有拼死一搏,才有可能在绝境中求得生存。这是我们拼死搏命的时候,(大家)同生共死!”士兵们的勇气顿时振奋,都说“听从命令!”于是(我们)下令说:“海盗船未靠近到三百步以内,不得发射子母炮;未靠近到八十步以内,不得开枪;未靠近到四十步以内,不得射箭。如果确实靠近,才用长枪与他们拼杀。有能击毙敌人的,重重赏赐;违抗命令的,按照军法惩处。”大家各自整理好武器,等待(敌人靠近)。
    
    没过多久,十六只海盗船吆喝着驶来,第一只已经进入三百步范围。我举起旗子指挥,吴得进从舵门发射子母炮,立即击毙四人,击中吆喝的海盗(使其)坠海;敌人来不及退,驶进百步以内,(我方)枪声齐发,又击毙六人。一只海盗船于是退去,(另外)两只海盗船又进入三百步范围,再次用炮攻击,击毙五人;它稍微靠近后,(我方)又用炮攻击,又击毙四人,(敌人)这才退去。这时,有三只海盗船已经占据了上风口;我们暗中将子母炮移到舵右舷边,连续击毙十二名海盗,烧毁了他们的船帆:海盗船都转舵退去。中间第二条海盗船较大,又鼓噪着从上风方向飞速驶来。我说:“这一定是海盗头目的船!”急忙命令舵工将船稍微横过来,等大炮对准海盗船后,立即发射一炮,击中了目标。炮响后,烟雾弥漫了一里多远;烟雾散去后,海盗船已经全部退去。这场战斗中,王得禄身先士卒,士兵吴得进、陈成德、林安顺、张大良、王名标、甘耀等人枪炮无一虚发,(我们)幸运地挺过了危险。只是当时天色已晚,风力很弱;担心海盗乘夜袭击,我默默向天后祈求来风。不一会儿,北风大作,浪花飞溅过船舷。我极度疲惫,想躺下(休息)。(我)想到之前的险状,要是(当时)遇害了,又哪能再来考虑现在的危险呢?况且之前求风,可不得到了风吗?即使忧虑也无济于事。
    
    十一月朔日己卯,阴。梦中听到船夫大叫:“到官塘了!”我惊醒起身。介山和随从都一夜没睡,对我说:“这么危险的情况,你还能睡着,真服了你。要是葬身鱼腹,也是个糊涂鬼了!”我说:“有多危险?”介山说:“往上好像到了九天之上,往下好像到了九地之下,(隆隆的)声响就像转水车、锯湿木一样,时不时又像得了疟疾一样震颤;每次(船)倾侧,船帆都卧倒在水里。一旦有浪盖在船上,船身都进到水里,只听得像瀑布声垂流不息;没有翻船,靠的是幸运啊!”我说:“倘若翻船了,你们逃得过吗?我乐得睡了一觉,又错过了危险,我太幸运了!”介山于是大笑。船夫指出:“前面就是定海了,不需要再担心了!”下午三四点,(船)才终于靠岸。总兵何定江来迎接护送,我笑着答谢他。于是跟他说起北杞的战斗,定江惊惶失措;我说:“我饿了!这些事以后再说吧。”
    
\end{normalsize}


\newpage

\textbf{注解}:

\vspace{-1em}

\begin{itemize}
    \setlength\itemsep{-0.2em}
    \item〔初九日庚寅〕传统记日的方法。“初九”是阴历记日,表示当月第九天;“庚寅”是干支记日,用天干地支顺序计数,每六十天一循环。两种记日方法并用,交叉比对,准确不出差错。
    \item〔卯刻〕传统记时方法。古人把一天分为十二个时辰,用十二地支记录。每个时辰又分为八刻。初刻、正刻,各一小时。“卯刻”也就是“卯时”,指早上五到七点,“申正”指“申时正刻”,即下午四点正。
    \item〔转丁未风,用单乙针〕传统罗盘方位记法,分八方二十四针,一方三针。又有单针、双针之分。单针指最近一针,双针指介于相邻双针之间。“丁”是南偏西南(南偏西15度),“未”是西南偏南(南偏西30度),“丁未”说明方向介于“丁”、“未”之间。“单乙针”表示以“乙”针(东偏南15度)为准航行。
    \item〔计自开洋,行船十六更矣〕“更”是明清时代发展出来的计量海路里程的方法。最初是把一夜分为五更,每更大约2.4小时。后来把一更航行的里程作为单位,每更大约50里(25公里)。李鼎元此次出海由于出发仓促,没有准备沙漏,靠焚香记时,因此有较大误差。实际路程远比十六更少,下同。
    \item〔鳐如白鸟,云飞丈馀始入水〕文鰩出海南,大者長尺許,有翅,與尾齊,一名飛魚。羣飛水上,海人候之,當有大風。
    \item〔其花红,一穗数尺,瓣须五六出。岁实为常,实如其须之数。〕香蕉的花与一般的花不太一样,并没有典型的花瓣,而是像笋壳一样的苞片。雌花的子房初时细长像须一样,逐渐发育为香蕉。一般来说,香蕉树结果之后第二年会枯萎死掉,然后重新从根部长出新的香蕉树,因此不会每年都结果。
\end{itemize}

\chapter{《世说新语》两则}

\begin{normalsize}
    
    咏雪
    
    谢太傅寒雪日内集,与儿女讲论文义。俄而雪骤,公欣然曰:“白雪纷纷何所似?”兄子胡儿曰:“撒盐空中差可拟。”兄女曰:“未若柳絮因风起。”公大笑乐。即公大兄无奕女,左将军王凝之妻也。
    
    陈太丘与友期
    
    陈太丘与友期行,期日中,过中不至,太丘舍去,去后乃至。元方时年七岁,门外戏。客问元方:“尊君在不?”答曰:“待君久不至,已去。”友人便怒:“非人哉!与人期行,相委而去。”元方曰:“君与家君期日中,日中不至,则是无信;对子骂父,则是无礼。”友人惭,下车引之,元方入门不顾。
\end{normalsize}


\newpage

\textbf{译文}:

\vspace{1em}

\begin{normalsize}
    
    咏雪
    
    谢安在寒冷的雪天举行家庭聚会,给子侄辈的人讲解诗文。不久,雪下得大了,太傅高兴地说:“这纷纷扬扬的白雪像什么呢?”他哥哥的长子谢朗说:“在空中撒盐差不多可以相比。”另一个哥哥的女儿说:“不如比作柳絮凭乘风飞舞。”太傅大笑起来。她就是谢奕的女儿谢道韫,左将军王凝之的妻子。
    
    陈太丘与友期行
    
    陈太丘和朋友相约同行,约定的时间在正午,过了正午朋友还没有到,陈太丘不再等候他而离开了,陈太丘离开后朋友才到。元方当时年龄七岁,在门外玩耍。陈太丘的朋友问元方:“你的父亲在吗?”元方回答道:“我父亲等了您很久您却还没有到,已经离开了。”友人便生气地说道:“真不是君子啊!和别人相约同行,却丢下别人先离开了。”元方说:“您与我父亲约在正午,正午您没到,就是不讲信用;对着孩子骂父亲,就是没有礼貌。”朋友感到惭愧,下了车想去拉元方的手,元方头也不回地走进家门。
    
\end{normalsize}



\chapter{天文地理}

\begin{normalsize}
    
    混沌初开,乾坤始奠。
    
    气之清轻者上浮为天,气之浊重者下凝为地。
    
    天形如卵白,地形如卵黄。
    
    日月五星,谓之七曜;天地与人,谓之三才。
    
    北斗为帝车之象,北辰是太一所居。
    
    叁星为白虎之体,大火是苍龙之心。
    
    二十八宿,是日月之旅舍。黄道白道,载日月之行迹。
    
    叁商二星,其出没不相见。牛女两宿,惟七夕一相逢。
    
    后羿射日,女娲补天。羲和置闰,授民以时。
    
    黄帝画野,始分都邑。夏禹治水,初奠山川。
    
    宇宙之江山不改,古今之称谓各殊。
    
    北京原属幽燕,金台是其异号;南京原为建业,金陵为其别名。
    
    浙江是武林之区,原为越国;江西是豫章之地,昔从九江。
    
    福建省属闽中,湖广地名三楚。
    
    东鲁西鲁,即山东山西之分;东粤西粤,乃广东广西之域。
    
    河南在华夏之中,故曰中州;陕西即长安之地,原为秦境。
    
    四川为西蜀,云南为古滇。
    
    贵州省近蛮方,自古名为黔地。
    
    东岳泰山,西岳华山,南岳衡山,北岳恒山,中岳嵩山,此为天下之五岳。
    
    饶州之鄱阳,岳州之青草,润州之丹阳,鄂州之洞庭,苏州之太湖,此为天下之五湖。
    
    沧海桑田,谓世事之多变。河清海晏,兆天下之升平。
    
    道不拾遗,由在上有善政。途通天堑,知中国有圣人。
\end{normalsize}


\newpage

\textbf{译文}:

\vspace{1em}

\begin{normalsize}
    
    最初混沌开辟成宇宙,然后天和地就确定了下来。
    
    清澈轻盈的气往上漂浮,形成了天;浑浊沉重的气往下凝聚,形成了地。
    
    天的形状像蛋白,地的形状像蛋黄。
    
    日月和五大行星合称“七曜”。
    
    北斗七星是天帝驾的车子,北辰是天帝太一住的地方。
    
    叁宿三星是(西方七宿对应的星象)白虎的躯体,大火是(东方七宿对应的星象)苍龙的心脏。
    
    二十八星宿,是日月周游暂住的处所。黄道和白道,记录了日月运行的轨迹。
    
    叁星与商星此出彼没,永远没有机会相见。牛郎和织女隔着银河相望,每年七月初七才能相会。
    
    后羿射日,女娲补天(,都是古代调整历法的传说)。(帝尧任命羲、和掌管天象观测,制定历法,)羲、和设置闰月,让民众知道务农的时机。
    
    黄帝划分天地的区域,从此有都城和封邑的界限。夏禹治理洪水,首次确定了山川的位置。
    
    天地间的山岭河流(虽然)不曾更改,过去和今天的称呼(却)各有不同。
    
    北京以前属于幽州和燕国,“金台”是它的别号;南京原先叫建业,金陵是它的别名。
    
    浙江从前称为武林,原来是越国的土地;江西是以前的豫章郡,更早的时候是九江郡的一部分。
    
    福建省古时是闽中郡,湖南湖北旧名叫做三楚。
    
    东鲁、西鲁就是山东、山西,东粤、西粤即为广东、广西。
    
    河南位于华夏的中心位置,所以又称为中州;长安为陕西的首府,古代是秦国的国土。
    
    四川就是西蜀,云南古时叫滇国。
    
    贵州靠近南蛮之地,自古以来称为黔。
    
    东岳泰山,西岳华山,南岳衡山,北岳恒山,中岳嵩山,这是天下的五座雄伟的高山。
    
    饶州的鄱阳湖、岳州的青草湖、润州的丹阳湖、鄂州的洞庭湖、苏州的太湖,这是天下的五个大湖。
    
    沧海桑田,用来形容世事多变。海晏河清,说明天下和平繁荣。
    
    路上的失物没人据为己有,是因为当局实行了好的政策。能把天堑变成通途,就知道中国有圣人。
    
\end{normalsize}


\newpage

\textbf{注解}:

\vspace{-1em}

\begin{itemize}
    \setlength\itemsep{-0.2em}
    \item〔叁星为白虎之体,大火是苍龙之心〕《史记·天官书》:“叁为白虎。”《晋书·天文志》:“叁,白兽之体。”《史记·天官书》:“ 叁为白虎。叁星直者,是为衡石。下有叁星,兑,曰罚,为斩艾事。”大火指心宿二,也叫商星。心宿是东方苍龙七宿之一,位于中心。
    \item〔叁商二星,其出没不相见〕《左传·昭公》记载:“昔高辛氏有二子,伯曰阏伯,季曰实沈,居于旷林,不相能也。日寻干戈,以相征讨。后帝不臧,迁阏伯于商丘,主辰。商人是因,故辰为商星。迁实沈于大夏,主参。唐人是因,以服事夏商。”
    \item〔二十八宿,是日月之旅舍〕《论衡·谈天》:“二十八宿为日月舍,犹地有邮亭,为长吏廨矣。邮亭著地,亦如星舍著天也。”
    \item〔日月五星,谓之七曜〕日月和水星、金星、火星、木星、土星合称“七曜”。《春秋谷梁传注疏》:“七曜者,日月五星皆照天下,故谓之‘七曜’。五星者,即东方岁星,南方荧惑,西方太白,北方辰星,中央镇星是也。”
    \item〔帝车之象〕《史记·天官书》:“斗为帝车,运于中央,临制四乡。”。
    \item〔北辰是太一所居〕北辰就是北极星。《尔雅·释天》:“北极谓之北辰。” 太一是天帝的名称。《易纬·乾凿度》所说:“太一者,北辰之神名也,居其所曰太帝。”
    \item〔羲和置闰,授民以时〕《尚书·尧典》:“乃命羲和,钦若昊天,历象日月星辰,敬授人时。……帝曰:‘咨!汝羲暨和。期三百有六旬有六日,以闰月定四时成岁。’”
    \item〔黄帝画野,始分都邑。夏禹治水,初奠山川〕《汉书·地理志》:“昔在黄帝,作舟车以济不通,旁行天下,方制万里,画野分州,得百里之国万区。”《尚书·吕刑》:“禹平水土,主名山川。”
    \item〔福建省属闽中〕秦代设闽中郡,包括福建,浙江宁海县及灵江、瓯江、飞云江流域。后来福建也称闽中。
    \item〔湖广地名三楚〕元代设置湖广等处行中书省,包括湖南湖北,明代湖广行省,按春秋时楚国名号称为“楚”。《淮南子》:“楚人地南卷沅湘,北绕颍泗,西包巴蜀,东裹郯邳,颍汝以为洫,江汉以为池,垣之以邓林。”秦汉时有西楚、东楚、南楚的分野,《史记·货殖列传》:“夫自淮北沛、陈、汝南、南郡,此西楚也。其俗剽轻,易发怒,地薄,寡于积聚。彭城以东,东海、吴、广陵,此东楚也。衡山、九江、江南、豫章、长沙,是南楚也,其俗大类西楚。”即今天湖北省长江以北、河南中东部以及江苏安徽北部一带总称为西楚,江苏中南部与浙江北部等长三角一带为东楚,长江以南的江西与湖南等地为南楚。总称三楚。“三楚”的分野在汉代初年可谓社会共识,一直成为区别各地民风、物产等社会环境的代称,足以证明楚之影响巨大。东汉末年,黄巾之乱带来的人口迁徙让长江中下游的吴越地区逐渐富庶起来。孙氏家族在江东地区,即今天的江苏南部、浙江和江西一带,建立了东吴政权,开启了六朝时期,逐渐以三吴都会自居,最早脱离楚地的范畴。至隋唐以后,楚地逐渐收缩为湖南湖北为核心的荆湖地区。
    \item〔黄帝画野,始分都邑。夏禹治水,初奠山川〕《汉书·地理志》:“昔在黄帝,作舟车以济不通,旁行天下,方制万里,画野分州,得百里之国万区。”《尚书·吕刑》:“禹平水土,主名山川。”
    \item〔黄帝画野,始分都邑。夏禹治水,初奠山川〕《汉书·地理志》:“昔在黄帝,作舟车以济不通,旁行天下,方制万里,画野分州,得百里之国万区。”《尚书·吕刑》:“禹平水土,主名山川。”
    \item〔黄帝画野,始分都邑。夏禹治水,初奠山川〕《汉书·地理志》:“昔在黄帝,作舟车以济不通,旁行天下,方制万里,画野分州,得百里之国万区。”《尚书·吕刑》:“禹平水土,主名山川。”
    \item〔贵州省近蛮方〕蛮:指南方未开化的部落。方:区域,上古称部落邦国,如土方、鬼方等。
    \item〔饶州之鄱阳〕隋朝设饶州,后为鄱阳郡,今有江西上饶市。鄱阳湖古称彭蠡泽、彭泽,是江西北部的大湖。
    \item〔岳州之青草〕隋朝设岳州,历史上也叫巴陵、巴州,今有湖南岳阳市。青草湖是湖南古代的大湖,位于岳阳市西南,洞庭湖的南部,并与之相连。南北朝时期已连为一体。清末之后逐渐淤积枯萎,现已不存。
    \item〔润州之丹阳〕隋唐设润州,在今江苏西南部和安徽交界。丹阳湖是江南古代的大泽,因秦置丹阳县得名,也叫丹湖、南湖,位于润州西南,今南京南面,高淳、溧水、当涂一带。因泥沙淤积和围垦,逐渐消亡,如今剩余部分有石臼湖。
    \item〔鄂州之洞庭〕隋朝设鄂州,包括今湖北地区,历史上也叫江夏、武昌,治所在今湖北武汉市。这里的洞庭指唐宋时湖南北部、长江以南的大湖。晋唐以来,由于地势升降原因形成的青草湖(洞庭湖的前身)水域面积不断扩大,成为除长江之外楚地南北分界的自然地理标识。至北宋时期,随着水域扩展,使原来在汉晋时期彼此支离的洞庭、青草、赤沙3个湖泊在高水位时得以连成汪洋一片。也正是在北宋时期,以洞庭湖为界线的荆湖北路与荆湖南路的行政区划分离也随之出现。不过,由于宋代的路一级区划并非实际具有独立管辖权的高级别行政单位,所以洞庭南北的荆湖楚地在本质上联结依旧紧密。曾经作为荆襄首府的江陵城(湖北荆州)在隋唐以后逐渐没落,宋以后鄂州(明代武昌,今湖北武汉市)逐渐成为长江中游的首府。元代设湖广行省,将鄂州行省并入湖广行省,最后迁治所于鄂州。当时湖广行省的管辖范围为今湖北、湖南、广西和广东、贵州的一部分,是当时最大的行省。明朝设湖广布政使司,简称“楚”。《大明清类天文分野之书》:“自房、陵、白帝而东,尽汉之南郡、江夏,东达庐江南部,滨彭蠡之西,得长沙、武陵,又逾南纪,尽郁林、合浦之地,自沅湘上流,西达黔安之左,皆全楚之分。”
    \item〔苏州之太湖〕隋朝设苏州,包括今江苏、浙江地区,历史上也叫吴州,治所在今江苏苏州市。太湖是江苏南部大湖,古称震泽。
\end{itemize}

\chapter{童趣}

\begin{normalsize}
    
    余忆童稚时,能张目对日,明察秋毫,见藐小之物必细察其纹理,故时有物外之趣。
    
    夏蚊成雷,私拟作群鹤舞于空中,心之所向,则或千或百,果然鹤也;昂首观之,项为之强。又留蚊于素帐中,徐喷以烟,使之冲烟而飞鸣,作青云白鹤观,果如鹤唳云端,为之怡然称快。
    
    余常于土墙凹凸处,花台小草丛杂处,蹲其身,使与台齐;定神细视,以丛草为林,以虫蚁为兽,以土砾凸者为丘,凹者为壑,神游其中,怡然自得。
    
    一日,见二虫斗草间,观之,兴正浓,忽有庞然大物,拔山倒树而来,盖一癞虾蟆,舌一吐而二虫尽为所吞。余年幼,方出神,不觉呀然一惊。神定,捉虾蟆,鞭数十,驱之别院。
\end{normalsize}


\newpage

\textbf{译文}:

\vspace{1em}

\begin{normalsize}
    
    我回想小的时候,能够张大眼睛对着太阳,能看清最细微的东西。
    
    夏天蚊子的叫声像打雷一样,我把它们比作群鹤在空中飞舞,心里这么想,那成千成百的蚊子果然都变成仙鹤了;我抬着头看它们,脖颈都为此而变得僵硬了。我又将几只蚊子留在素帐中,用烟慢慢地喷过去,让它们在烟雾中飞着叫着,我把它当做一幅青云白鹤的景观,果然就像鹤唳云端一样,我为这景象高兴地拍手叫好。
    
    我常在土墙高低不平的地方,在花台杂草丛生的地方,蹲下身子,使自己和花台相平,聚精会神地观察,把草丛当做树林,把虫子、蚂蚁当做野兽,把土块凸出部分当做山丘,凹陷的部分当做山谷,全神贯注地沉浸在想象之中,心情愉悦自觉是一种满足。
    
    有一天,我看见两只小虫在草间相斗,蹲下来观察它们,兴趣正浓厚,忽然有个极大的家伙,掀翻山压倒树而来了,原来是一只癞蛤蟆,舌头一吐,两只虫子全被它吃掉了。我那时年纪很小,正看得出神,不禁‘呀’的一声惊叫起来。待到神情安定下来,捉住癞蛤蟆,鞭打了几十下,把它驱赶到别的院子里去了。
    
\end{normalsize}



\chapter{智子疑邻}

\begin{normalsize}
    
    宋有富人。天雨墙坏,其子曰:“不筑,必将有盗。”其邻人之父亦云。暮而果大亡其财。其家甚智其子,而疑邻人之父。
\end{normalsize}


\newpage

\textbf{译文}:

\vspace{1em}

\begin{normalsize}
    
    宋国有个富人。天下雨,他家的墙坏了,他儿子说:“不把墙修好,一定会有盗贼的。”他邻居家的老人也这样说。到了晚上,他果然丢失了很多财物。这家人很赞赏自己的儿子聪明,却怀疑邻居家的老人(是盗贼)。
    
\end{normalsize}



\chapter{口技}

\begin{normalsize}
    
    京中有善口技者。会宾客大宴,于厅事之东北角,施八尺屏障,口技人坐屏障中,一桌、一椅、一扇、一抚尺而已。众宾团坐。少顷,但闻屏障中抚尺\footnote{〔抚尺〕曲艺演员表演时用以拍桌子以引起听众注意的木块。}一下,满坐寂然,无敢哗者。
    
    遥遥闻深巷中犬吠,便有妇人惊觉欠伸,其夫呓语。既而儿醒,大啼。夫亦醒。妇抚儿乳,儿含乳啼,妇拍而呜之。又一大儿醒,絮絮不止。当是时,妇手拍儿声,口中呜声,儿含乳啼声,大儿初醒声,夫叱大儿声,一齐凑发,众妙毕备。满坐宾客无不伸颈侧目,微笑默叹,以为妙绝。
    
    未几,夫齁声起,妇拍儿亦渐拍渐止。微闻有鼠作作索索,盆器倾侧,妇梦中咳嗽。宾客意少舒,稍稍正坐。
    
    忽一人大呼:“火起”,夫起大呼,妇亦起大呼。两儿齐哭。俄而百千人大呼,百千儿哭,百千犬吠。中间力拉崩倒之声,火爆声,呼呼风声,百千齐作;又夹百千求救声,曳屋许许声,抢夺声,泼水声。凡所应有,无所不有。虽人有百手,手有百指,不能指其一端;人有百口,口有百舌,不能名其一处也。于是宾客无不变色离席,奋袖出臂,两股战战,几欲先走。
    
    而忽然抚尺一下,群响毕绝。撤屏视之,一人、一桌、一椅、一扇、一抚尺而已。
\end{normalsize}


\newpage

\textbf{译文}:

\vspace{1em}

\begin{normalsize}
    
    京城里有个擅长表演口技的人。正赶上有户人家宴请宾客,在大厅的东北角,安放了一座八尺高的屏风,表演口技的艺人坐在屏风里面,里面只放了一张桌子、一把椅子、一把扇子、一块醒木罢了。众多宾客围绕着屏风而坐。一会儿,只听见屏风里面醒木一拍,全场马上静悄悄的,没有人敢大声说话。
    
    远远地听到幽深的巷子中有狗叫声,就有妇女惊醒后打呵欠和伸懒腰的声音,她的丈夫在说梦话。过了一会儿孩子醒了,大声哭着。丈夫也醒了。妇人抚慰孩子喂奶,孩子含着乳头哭,妇女又轻声哼唱着哄小孩入睡。又有一个大儿子醒了,絮絮叨叨地说个不停。在这时候,妇女用手拍孩子的声音,口里哼着哄孩子的声音,孩子边含乳头边哭的声音,大孩子刚醒过来的声音,丈夫责骂大孩子的声音,同时响起,各种声音都模仿得极像。满座的宾客没有一个不伸长脖子,偏着头看,微笑,心中默默赞叹,认为妙极了。
    
    过了一会儿,丈夫打起了呼噜声,妇女拍孩子的声音也渐渐消失。隐隐听到有老鼠作作索索活动的声音,盆子翻倒倾斜,妇女在梦中发出了咳嗽声。宾客们的心情稍微放松了些,渐渐端正了坐姿。
    
    忽然有一个人大声呼叫:“起火啦”,丈夫起来大声呼叫,妇人也起来大声呼叫。两个小孩子一齐哭了起来。一会儿,有成百上千人大声呼叫,成百上千的小孩哭叫,成百上千条狗汪汪地叫。其中夹杂着劈里啪啦,房屋倒塌的声音,烈火燃烧物品爆裂的声音,呼呼的风声,千百种声音一齐发出;又夹杂着成百上千人的求救的声音,众人拉塌燃烧着的房屋时一齐用力的呼喊声,抢救东西的声音,救火的声音。凡是在这种情况下应该有的声音,没有一样没有的。即使一个人有上百只手,每只手有上百个指头,也不能指明其中的任何一种声音来;即使一个人有上百张嘴,每张嘴里有上百条舌头,也不能说清其中一个地方。在这种情况下,宾客们没有一个不变了脸色,离开席位,捋起衣袖,伸出手臂,两腿打着哆嗦,差一点争先恐后地跑了。
    
    忽然醒木一拍,各种声响全部消失了。撤去屏风一看里面,只有一个人、一张桌子、一把椅子、一把扇子、一块醒木罢了。
    
\end{normalsize}



\chapter{伤仲永}

\begin{normalsize}
    
    金溪\footnote{〔金溪〕现在江西抚州市金溪县。}民方仲永,世隶耕。仲永生五年,未尝识书具,忽啼求之。父异焉,借旁近与之,即书诗四句,并自为其名。其诗以养父母、收族为意,传一乡秀才观之。自是指物作诗立就,其文理皆有可观者。邑人奇之,稍稍宾客其父。或以钱币乞之。父利其然也,日扳仲永环谒于邑人,不使学。
    
    余闻之也久。明道\footnote{〔明道〕宋仁宗(赵祯)年号(公元1032至1033年)。}中,从先人还家,于舅家见之,十二三矣。令作诗,不能称前时之闻。又七年,还自扬州,复到舅家问焉。曰:“泯然众人矣。”
    
    王子曰:仲永之通悟,受之天也。其受之天也,贤于材人远矣。卒之为众人,则其受于人者不至也。彼其受之天也,如此其贤也,不受之人,且为众人;今夫不受之天,固众人,又不受之人,得为众人而已耶?
\end{normalsize}


\newpage

\textbf{译文}:

\vspace{1em}

\begin{normalsize}
    
    金溪县的平民方仲永,世代以耕田为业。仲永五岁时,不曾认识书写工具,(有一天)忽然哭着要这些东西。父亲觉得诧异,从邻家借来书写工具给他,仲永立即写了四句诗,并且自己题上自己的名字。这首诗的意思是赡养父母、与同一宗族的人搞好关系,传给全乡的秀才观赏。从此,指定物品(让他)作诗,(他能)立即完成,(诗的)文采和道理都有值得观赏的地方。同县的人对此感到奇怪,渐渐地,都以宾客之礼对待他父亲。有人花钱请方仲永作诗。他的父亲以此认为有利可图,每天拉着仲永四处拜访同县的人,不让(他)学习。
    
    我听说这件事也已经很久了。明道年间,我随先父回到家乡,在舅舅家里见到了方仲永,(他)已经十二三岁了。让(他)作诗,(他写的诗)不能与从前的名声相称。又(过了)七年,(我)从扬州回来,再次到舅舅家,问起方仲永的情况。(舅舅)回答说:“他的才能已完全消失,和常人一样了。”
    
    王安石说:仲永的通达聪慧是先天得到的。他先天的才能,远胜于一般有才能的人。他最终成为一个平常人,是因为他受到后天的教育没有达到要求。他那样天生聪明,如此有才智的人,没有受到后天的教育,尚且要成为普通人,现在那些天生就不聪明,本来就是普通的人,又不接受后天的教育,能够成为普通人就为止了吗?
    
\end{normalsize}



\chapter{孙权劝学}

\begin{normalsize}
    
    初,权谓吕蒙曰:“卿今当涂掌事,不可不学!”蒙辞以军中多务。权曰:“孤岂欲卿治经为博士邪?但当涉猎,见往事耳。卿言多务,孰若孤?孤常读书,自以为大有所益。”蒙乃始就学。及鲁肃过寻阳,与蒙论议,大惊曰:“卿今者才略,非复吴下阿蒙!”蒙曰:“士别三日,即更刮目相待,大兄何见事之晚乎!”肃遂拜蒙母,结友而别。
\end{normalsize}


\newpage

\textbf{译文}:

\vspace{1em}

\begin{normalsize}
    
    当初,孙权对吕蒙说:“你现在掌管政事,不可以不学习!”吕蒙用军中事务繁多来推托。孙权说:“我难道想要你研究儒家经典,成为专掌经学传授的学官吗?我只是觉得你应当粗略地阅读,了解历史。你说军务繁多,难道比得上我吗?我经常读书,自己觉得获益很多。”于是吕蒙开始从事学习。后来鲁肃经过寻阳,和吕蒙一起讨论议事,很惊讶地说:“你现在军事方面的才干和谋略,不再是在吴下时没有才学的阿蒙了!”吕蒙说:“与读书的士子分别几天,就应当用新的眼光看待,大兄你知晓事情为什么这么晚呢?”于是鲁肃拜见吕蒙的母亲,和吕蒙结为好友后离开了。
    
\end{normalsize}



\chapter{爱莲说}

\begin{normalsize}
    
    水陆草木之花,可爱者甚蕃。晋陶渊明\footnote{〔陶渊明〕名潜,字元亮,东晋末杰出的诗人、辞赋家、散文家。}独爱菊。自李唐\footnote{〔李唐〕指唐代。唐朝皇帝姓李,故称李唐。}来,世人盛爱牡丹。予独爱莲之出淤泥而不然,濯清涟而不妖,中通外直,不蔓不枝,香远益清,亭亭净植,可远观而不可亵玩焉。
    
    予谓菊,花之隐逸者也;牡丹,花之富贵者也;莲,花之君子者也。噫!菊之爱,陶后鲜有闻。莲之爱,同予者何人?牡丹之爱,宜乎众矣。
\end{normalsize}


\newpage

\textbf{译文}:

\vspace{1em}

\begin{normalsize}
    
    水上和陆地上草木的花中,可以喜爱的有很多。晋代陶渊明唯独喜爱菊花。自从唐朝以来,世上的人们十分喜爱牡丹。我唯独喜欢莲花,它从淤泥中生长出来,却不沾染污秽;它经过清水的洗涤后,却不显得妖媚;它的茎中间是贯通的,外形是笔直的,不生枝蔓,不长枝节;香气传播得越远越清幽,它洁净地直立着,可以远远地观赏,不可靠近去玩弄。
    
    我认为,菊花,是花中的隐士;牡丹,是花中的富贵人家;莲花,是花中的君子。唉!对于菊花的喜爱,(自从)陶渊明之后就很少听说了。对于莲花的喜爱,像我一样的还有什么人?对于牡丹的喜爱,人该是很多了。
    
\end{normalsize}



\chapter{记承天寺夜游}

\begin{normalsize}
    
    元丰六年\footnote{〔元丰〕宋神宗(赵顼)年号(公元1078年至1085年)。元丰六年是公元1083年,是苏轼因乌台诗案被贬黄州第四年。}十月十二日夜,解衣欲睡,月色入户,欣然起行。念无与为乐者,遂至承天寺\footnote{〔承天寺〕在今湖北黄冈县城南。}寻张怀民\footnote{〔张怀民〕苏轼的朋友。}。怀民亦未寝,相与步于中庭。庭下如积水空明,水中藻荇交横,盖竹柏影也。何夜无月?何处无竹柏?但少闲人如吾两人者耳。
\end{normalsize}


\newpage

\textbf{译文}:

\vspace{1em}

\begin{normalsize}
    
    元丰六年十月十二日夜晚,我正准备解衣入睡,恰好看到月光从门户照进来,于是高兴地起身出门。想到没有和我一起游乐的人,就到承天寺找张怀民。张怀民也还没有入睡,(我们)就一同在庭院里散步。庭院中满是月光,像积水一样澄澈透明。水中水藻、水草纵横交错,原来是院中竹柏的影子。哪一个夜晚没有月亮?哪个地方没有竹子和柏树呢?
    
\end{normalsize}



\chapter{大道之行也}

\begin{normalsize}
    
    大道之行也,天下为公。选贤与能,讲信修睦。故人不独亲其亲,不独子其子,使老有所终,壮有所用,幼有所长,鳏、寡、孤、独、废疾者皆有所养。男有分,女有归。货恶其弃于地也,不必藏于己;力恶其不出于身也,不必为己。是故谋闭而不兴,盗窃乱贼而不作,故外户而不闭。是谓大同。
\end{normalsize}


\newpage

\textbf{译文}:

\vspace{1em}

\begin{normalsize}
    
    大道得以实行的话,天下是大家所共有的。(大家)推选有才德、能干事的人,讲求诚信和睦的气氛。因此人们不只是把自己的父母当做父母,不只是把自己的孩子当做孩子,让老人能够终其天年,成年人能够为社会效力,年幼的人能够顺利成长,老而无妻的人、老而无夫的人、幼而无父的人、老而无子的人、残疾人都能得到供养。男子有职务,女子有归宿。反对把财货丢弃在地上,不一定是要自己私藏;反对惜身不出力,不一定是为自己打算。因此,谋私就不会盛行,盗窃、造反的事情就不会发生,对外的门就不用关了。这就是大同社会。
    
\end{normalsize}


\newpage

\textbf{注解}:

\vspace{-1em}

\begin{itemize}
    \setlength\itemsep{-0.2em}
    \item〔货恶其弃于地也,不必藏于己;力恶其不出于身也,不必为己〕《礼记正义》:“货,谓财货也。既天下共之,不独藏府库,但若人不收录,弃掷山林,则物坏世穷,无所资用,故各收宝而藏之。是恶弃地耳,非是藏之为巳,有乏者便与也。”行大道的人认为把财货丢弃在地上是不好的,所以收在自己家里,但不会去怀疑这是为了藏起来给自己用的借口。财货即便放在自己家里,也是共有的。放在自己家里是为了保存财货,当人有需要时再施与,而不是为了独吞。《礼记正义》:“力,谓为事用力。言凡所事,不惮劬劳,而各竭筋力者,正是恶于相欺,惜力不出于身耳。非是欲自营赡。”行大道的人认为,做事情的时候要尽力而为,不要惜身不出力,但不会去怀疑这是让别人出力而自己不出力、让别人供养自己的借口。所有人都要出力,不假装已经尽力而藏私欺瞒,事情才能办好。郑玄注:“劳事不惮,施无吝心,仁厚之教也。”出力的时候不会忌惮,施与的时候不会不舍得,这是仁厚带来的好处。
    \item〔男有分,女有归〕《礼记正义》:“分,职也。无才者耕,有能者仕,各当其职,无失分也。”有才能的人就做官,没才能的人就耕地,表示按才华能力分配合适的工作,称为“有分”。《礼记正义》:“女谓嫁为归。君上有道,不为失时,故有归也。”古代女性没有独立的能力,因此将合适的妇女嫁到合适的家庭,称为“有归”。
\end{itemize}

\chapter{观潮}

\begin{normalsize}
    
    浙江\footnote{〔浙江〕就是钱塘江。}之潮,天下之伟观也。自既望\footnote{〔既望〕农历十六日。}以至十八日为最盛。方其远出海门,仅如银线;既而渐近,则玉城雪岭际天而来,大声如雷霆,震撼激射,吞天沃日,势极雄豪。杨诚斋诗云“海涌银为郭,江横玉系腰”\footnote{〔杨诚斋〕杨万里,字廷秀,号诚斋,南宋文学家。“海涌银为郭,江横玉系腰”出自《浙江观潮》。}者是也。
    
    每岁京尹\footnote{〔京尹〕京都临安府,现在浙江杭州市。}出浙江亭教阅水军,艨艟数百,分列两岸;既而尽奔腾分合五阵\footnote{〔五阵〕指两、伍、专、参、偏五种阵法。}之势,并有乘骑弄旗标枪舞刀于水面者,如履平地。倏尔黄烟四起,人物略不相睹,水爆轰震,声如崩山。烟消波静,则一舸无迹,仅有敌船为火所焚,随波而逝。
    
    吴儿善泅者数百,皆披发文身,手持十幅大彩旗,争先鼓勇,溯迎而上,出没于鲸波万仞中,腾身百变,而旗尾略不沾湿,以此夸能。而豪民贵宦,争赏银彩。
    
    江干上下十余里间,珠翠罗绮溢目。车马塞途,饮食百物皆倍穹常时。而僦赁看幕,虽席地不容间也。
\end{normalsize}


\newpage

\textbf{译文}:

\vspace{1em}

\begin{normalsize}
    
    钱塘江的海潮是天下间最壮观的。每年的八月十六至八月十八期间,海潮最盛大。当海潮从远方海口出现的时候,只像一条白色的银线一般,过了一会儿慢慢逼近,白浪高耸就像白玉砌成的城堡、白雪堆成的山岭一般,波涛好像从天上堆压下来,发出很大的声音,就像震耳的雷声一般,汹涌澎湃,犹如吞没了蓝天、冲洗了太阳,非常雄壮豪迈。这就是杨诚斋诗中所说的:“海涌银为郭,江横玉系腰”。
    
    每年临安府的长官到浙江亭外检阅水军,巨大的战舰数百艘分别排列于江的两岸,一会儿全部的战舰都往前疾驶,一会儿分开;一会儿聚合,形成五种阵势,并有人骑着马匹耍弄旗帜标枪,舞弄大刀于水面之上,就好像步行在平地一般。忽然间黄色的烟雾四处窜起,人物一点点都看不见,水中的爆破声轰然震动,就像高山崩塌一般。过一会儿烟雾消散,水波平静,看不见任何一条大船,只有演习中充当敌军战船的军舰被火焚烧,随着水波而沉于海底。
    
    浙江一带善于游泳的健儿数百人,每个人都披散着头发,身上满是刺青,手里拿着十幅长的大彩旗,奋勇争先逆着水流踏浪而上在极高的波涛之中,忽隐忽现腾越着身子,姿势变化万千,然而旗尾一点点也没有被水沾湿,以此来夸耀自己的才能。民间富豪、贵族官员,争着赏赐(他们)金银绸缎。
    
    江岸南北上下十余里之间,满眼都是衣饰华丽的观众。车马太多,把道路都塞住了,街上贩卖的饮食物品,比平时价格高出一倍。游客租借观赏的帐篷太多,(拥挤得)连容纳一席之地的空间也没有。
    
\end{normalsize}



\chapter{核舟记}

\begin{normalsize}
    
    明有奇巧人曰王叔远,能以径寸之木为宫室、器皿、人物,以至鸟兽、木石,罔不因势象形,各具情态。尝贻余核舟一,盖大苏泛赤壁云。
    
    舟首尾长约八分有奇,高可二黍许。中轩敞者为舱,箬篷覆之。旁开小窗,左右各四,共八扇。启窗而观,雕栏相望焉。闭之,则右刻“山高月小,水落石出”,左刻“清风徐来,水波不兴”,石青糁之。
    
    船头坐三人,中峨冠而多髯者为东坡,佛印居右,鲁直居左。苏、黄共阅一手卷。东坡右手执卷端,左手抚鲁直背。鲁直左手执卷末,右手指卷,如有所语。东坡现右足,鲁直现左足,各微侧,其两膝相比者,各隐卷底衣褶中。佛印绝类弥勒,袒胸露乳,矫首昂视,神情与苏黄不属。卧右膝,诎右臂支船,而竖其左膝,左臂挂念珠倚之,珠可历历数也。
    
    舟尾横卧一楫。楫左右舟子各一人。居右者椎髻仰面,左手倚一衡木,右手攀右趾,若啸呼状。居左者右手执蒲葵扇,左手抚炉,炉上有壶,其人视端容寂,若听茶声然。
    
    其船背稍夷,则题名其上,文曰“天启壬戌秋日,虞山王毅叔远甫刻”,细若蚊足,钩画了了,其色墨。又用篆章一,文曰“初平山人”,其色丹。
    
    通计一舟,为人五,为窗八,为箬篷,为楫,为炉,为壶,为手卷,为念珠各一;对联、题名并篆文,为字共三十有四。而计其长,曾不盈寸。盖简桃核修狭者为之。
    
    魏子详瞩既毕,诧曰:嘻,技亦灵怪矣哉!《庄》《列》所载,称惊犹鬼神者良多,然谁有游削于不寸之质,而须麋瞭然者?假有人焉,举我言以复于我,亦必疑其诳。乃今亲睹之。由斯以观,棘刺之端,未必不可为母猴也。嘻,技亦灵怪矣哉!
\end{normalsize}


\newpage

\textbf{译文}:

\vspace{1em}

\begin{normalsize}
    
    明朝有个手艺奇妙精巧的人叫王叔远,他能用直径一寸左右的木头雕刻成宫室、器皿、人物,以及飞鸟走兽、树木石头,而且无不按着木头的原形来雕饰模拟物态,因而雕刻得各有各的情趣神态。他曾经赠送我一只用桃核雕刻成的小船,刻的是苏东坡泛舟游览赤壁的情景。
    
    核舟从头到尾大约有八分多长,高二分左右。中部高起而宽敞的地方是船舱,上面覆盖着箬竹船篷。船舱两旁开有小窗,左边和右边各四扇,总共八扇。打开窗子看,可见雕花的船栏杆,左右相对。关上窗子,可欣赏到右边窗上刻着“山高月小,水落石出”八字,左边窗上刻着“清风徐来,水波不兴”八字,都涂了石青颜色。
    
    船头上坐着三个人,当中戴高帽满腮胡须的是苏东坡,右边是佛印和尚,左边是黄鲁直。苏、黄两人正在共看一幅手卷。东坡右手拿着手卷的右端,左手搭在鲁直的背上。鲁直左手拿着手卷的末端,右手指着手卷,好象在讲什么话。东坡露出右脚,鲁直露出左脚,各微侧着身体,他们紧靠着的两膝,各隐现在手卷底下的衣服皱褶中。佛印极象弥勒佛,敞开胸怀,裸露双乳,抬头仰望着天空,神态表情与苏、黄二人不一样。他平方右膝,曲着右臂支撑在船板上,左腿曲膝竖起,左臂挂着念珠靠在左膝上,念珠可以一粒一粒清楚地数出来。
    
    船尾横放着一支桨。桨两旁各有一个船夫。右边那个梳着椎形发髻,仰面朝天,左手靠在一根横木上,右手扳住右脚趾头,象嘬着嘴唇在吹口哨的样子。左边那个右手拿着一柄蒲葵扇,左手摸着炉子,炉子上放一把水壶,那个人目光注视茶炉,脸色平静,好象在凝神倾听茶水烧煮的声音。
    
    这只船的底部比较平坦,就在上面题上名字,题的字是“天启壬戌秋日,虞山王毅叔远甫刻”,笔划细得象蚊子脚,一钩一画都清清楚楚,字色黑。又用上一颗篆字印章,文字是“初平山人”,红颜色。
    
    总计这只船上,刻有五个人,八扇窗,箬竹船篷、船桨、茶炉、水壶、手卷、念珠各一件;对联、题名以及篆字印章,刻的字共有三十四个。可是量量核舟的长度,甚至还不满一寸。这原是挑选狭长的桃核雕刻成的。
    
    魏子仔细地看了这只核舟后,惊叹道:哇,这技艺也真是神奇啊!《庄子》、《列子》书中所记载的能工巧匠,令人惊奇有如鬼斧神工的很多,可是有谁在不到一寸的材料上运刀自如地进行雕刻,而又能刻得胡须眉毛都清清楚楚的?如果有那么一个人,拿我的话来告诉我,我也一定会怀疑他在说谎。可现在这却是我亲眼目睹的事实。从这件作品来看,在棘木刺的尖端,未必不能雕刻出母猴来。啊,这技艺也真是神奇啊!
    
\end{normalsize}



\chapter{湖心亭看雪}

\begin{normalsize}
    
    崇祯五年\footnote{〔崇祯〕明思宗(朱由检)的年号(公元1628至1644年)。}十二月,余住西湖\footnote{〔西湖〕杭州市西的浅水湖,中国著名文化景观。}。大雪三日,湖中人鸟声俱绝。是日更定矣,余拏一小舟,拥毳衣炉火,独往湖心亭看雪。雾凇沆砀,天与云、与山、与水,上下一白。湖上影子,惟长堤一痕、湖心亭一点、与余舟一芥、舟中人两三粒而已。
    
    到亭上,有两人铺毡对坐,一童子烧酒,炉正沸。见余大喜,曰:“湖中焉得更有此人?”拉余同饮。余强饮三大白而别。问其姓氏,是金陵人\footnote{〔金陵〕现江苏南京市。},客此。及下船,舟子喃喃曰:“莫说相公痴,更有痴似相公者。”
\end{normalsize}


\newpage

\textbf{译文}:

\vspace{1em}

\begin{normalsize}
    
    崇祯五年十二月,我住在西湖边。大雪接连下了几日,湖中游人全无,连飞鸟的声音都消失了。这天晚上八点左右,我撑着一叶小舟,裹着裘皮衣服,围着火炉,独自前往湖心亭看雪。湖面上冰花一片弥漫,天、云、山、水混为一体,白茫茫一片。湖上的影子,只有西湖长堤在雪中隐隐露出的一道痕迹,一点湖心亭的轮廓,和我的微小如草的小舟,舟中的两三粒人影罢了。
    
    到了湖心亭上,我看见有两人铺好毛毯,相对而坐,一个童子正把酒炉里的酒烧得滚沸。他们看见我,非常高兴地说:“(想不到)在湖中还会有您这样有闲情逸致的人!”于是拉着我一同饮酒。我尽力喝了三大杯,然后和他们道别。问他们的姓氏,得知他们是金陵人,在此地客居。等到了下船的时候,船夫喃喃地说:“不要说相公您痴,还有像相公您一样痴的人啊!”
    
\end{normalsize}



\chapter{陋室铭}

\begin{normalsize}
    
    山不在高,有仙则名。水不在深,有龙则灵。斯是陋室,惟吾德馨。台痕上阶绿,草色入帘青。谈笑有鸿儒,往来无白丁。可以调素琴,阅金经。无丝竹之乱耳,无案牍之劳形。南阳诸葛庐\footnote{〔南阳诸葛庐〕南阳:今河南省南阳市一带。诸葛亮在出山之前,曾在南阳卧龙岗中隐居躬耕。诸葛亮,字孔明,三国时蜀汉丞相,著名的政治家和军事家。},西蜀子云亭\footnote{〔西蜀子云亭〕扬雄:字子云,西汉时文学家,蜀郡成都人。}。孔子云:何陋之有\footnote{〔孔子云……〕出自《论语·子罕第九》:“君子居之,何陋之有?”}?
\end{normalsize}


\newpage

\textbf{译文}:

\vspace{1em}

\begin{normalsize}
    
    山不在于高,有了仙人就会有名气。水不在于深,有了龙就会有灵气。这是间简陋的房子,只有我品德的馨香。苔痕碧绿,长到台上,草色青葱,映入帘里。到这里谈笑的都是博学之人,来往的没有知识浅薄之人。可以弹奏不加装饰的琴,阅读佛经。没有弦管奏乐的声音扰乱耳朵,没有官府的公文使身体劳累。南阳有诸葛亮的草庐,西蜀有扬子云的亭子。孔子说:有什么简陋的呢?
    
\end{normalsize}



\chapter{三峡}

\begin{normalsize}
    
    自三峡\footnote{〔三峡〕指长江上游重庆、湖北两省间的瞿塘峡、巫峡和西陵峡。}七百里中,两岸连山,略无阙处。重岩叠嶂,隐天蔽日。自非亭午夜分,不见曦月。
    
    至于夏水襄陵,沿溯阻绝。或王命急宣,有时朝发白帝\footnote{〔白帝〕白帝城,在重庆奉节市东。},暮到江陵\footnote{〔江陵〕今湖北省江陵县。},其间千二百里,虽乘奔御风,不似疾也。
    
    春冬之时,则素湍绿潭,回清倒影。绝巘多生怪柏,悬泉瀑布,飞漱其间,清荣峻茂,良多趣味。
    
    每至晴初霜旦,林寒涧肃,常有高猿长啸,属引凄异,空谷传响,哀转久绝。故渔者歌曰:“巴东三峡巫峡长,猿鸣三声泪沾裳。”
\end{normalsize}


\newpage

\textbf{译文}:

\vspace{1em}

\begin{normalsize}
    
    在七百里长的三峡中,两岸都是相连的高山,中间没有空缺的地方。重重叠叠的山峰像屏障一样,遮住了天空和太阳。如果不是正午或半夜,就看不到太阳和月亮。
    
    到了夏天,江水漫上两岸的丘陵的时候,顺流而下和逆流而上的船只都被阻隔了。 如果有时皇上的命令要紧急传达,早晨从白帝城出发,傍晚就到了江陵,这中间有一千二百多里,即使骑着奔驰的快马,驾着风,也不如船行的快啊。
    
    每到春季和冬季,白色的急流,回旋着清波,碧绿的潭水,映出了(山石林木)的倒影。高峻的顶峰上生长着许多奇形怪状的柏树,悬挂着的瀑布冲荡在岩石山涧中,水清、树荣、山高、草盛,实在是有许多趣味。
    
    每到秋雨初晴、降霜的时候,树林山涧一片清凉寂静,经常有猿猴在高处长啸,叫声不断,声音凄凉怪异,空荡的山谷里传来了回声,悲哀婉转,很长时间才消失。所以打鱼的人唱道:“巴东三峡巫峡长,猿鸣三声泪沾裳。”
    
\end{normalsize}



\chapter{桃花源记}

\begin{normalsize}
    
    晋太元\footnote{〔太元〕东晋孝武帝的年号(公元376至396年)。}中,武陵\footnote{〔武陵〕武陵郡,现在湖南常德市一带。}人捕鱼为业。缘溪行,忘路之远近。忽逢桃花林,夹岸数百步,中无杂树,芳草鲜美,落英缤纷。渔人甚异之。复前行,欲穷其林。
    
    林尽水源,便得一山,山有小口,仿佛若有光。便舍船,从口入。初极狭,才通人。复行数十步,豁然开朗。土地平旷,屋舍俨然,有良田美池桑竹之属。阡陌交通,鸡犬相闻。其中往来种作,男女衣着,悉如外人。黄发垂髫,并怡然自乐。
    
    见渔人,乃大惊,问所从来。具答之。便要还家,设酒杀鸡作食。村中闻有此人,咸来问讯。自云先世避秦时乱,率妻子邑人来此绝境,不复出焉,遂与外人间隔。问今是何世,乃不知有汉,无论魏晋。此人一一为具言所闻,皆叹惋。余人各复延至其家,皆出酒食。停数日,辞去。此中人语云:“不足为外人道也。”
    
    既出,得其船,便扶向路,处处志之。及郡下,诣太守\footnote{〔太守〕一郡之长。},说如此。太守即遣人随其往,寻向所志,遂迷,不复得路。
    
    南阳\footnote{〔南阳〕今河南省南阳市一带。}刘子骥,高尚士也,闻之,欣然规往,未果,寻病终。后遂无问津者。
\end{normalsize}


\newpage

\textbf{译文}:

\vspace{1em}

\begin{normalsize}
    
    东晋太元年间,武陵有个以捕鱼为生的人。有一天他沿着溪水划船而行,忘记了路有多远。忽然遇到一片桃花林,在溪流两岸的几百步之内,中间没有其他的树木, 芬芳的青草新鲜美好,地上的落花繁多交杂。渔人对此感到非常惊异。他继续向前行船,想要走到桃花林的尽头。
    
    桃林在溪水发源的地方就到头了,渔人就发现了一座小山,山上有个小洞口,洞里面隐隐约约透着点光亮。渔人便下了船,从洞口走了进去。刚开始非常狭窄,仅容一人通过。又向前行走了几十步,突然变得开阔明亮。渔人眼前这片土地平坦开阔,房屋排列得非常整齐,还有肥沃的田地、美丽的池塘,以及桑树、竹子等等。田间小路交错相通,(村落间)可以互相听到鸡鸣狗叫的声音。人们在田间来来往往耕种劳作,男女的穿戴装束,完全如同世俗之外的人一样。老年人和小孩儿都高兴而满足。
    
    这里的人看见了渔人,就非常惊讶,问他是从什么地方而来。渔人都详细地作了回答。这里的人便邀请渔人到家中做客,摆了酒、杀了鸡用来款待他。村里面的其他人听说来了这么一个人,全都来打听消息。他们自己说他们的先祖是为了躲避秦朝时期的战乱,带领妻子儿女和乡邻来到这个与世隔绝的地方,再没有人出去过,所以和外面的人隔绝了一切往来。村里的人问渔人现在是什么世道,他们居然不知道有汉朝,更不用说魏、晋两朝了。渔人把自己知道的所有事都一一说了出来,村民们听了都感叹惋惜。其余的人各自又把渔人邀请到自己的家中,都拿出自己的美酒佳肴来款待他。渔人停留了几日后,就向村里的人告辞。村里的人告诉他:“这里的情况不值得对外面的人说啊。”
    
    渔人出来之后,找到了自己的船,就顺着来时的路回去,处处都做了标记。他到了郡城武陵,就去拜见太守,说了桃花源的见闻。太守立即派遣人员跟随他前往,寻找渔人先前作的标记,竟然迷路了,后来再也找不到通往桃花源的路了。
    
    南阳有个叫刘子骥的人,是一个品德高尚的隐士,他听到了这个消息,愉快地计划前往桃花源,(但)未能实现,不久后就病死了。后来就再也没有探寻桃花源的人了。
    
\end{normalsize}



\chapter{满井游记}

\begin{normalsize}
    
    燕地寒,花朝节\footnote{〔花朝节〕旧时以农历二月十二日为花朝节。说这一天是百花生日。}后,余寒犹厉。冻风时作,作则飞沙走砾。局促一室之内,欲出不得。每冒风驰行,未百步辄返。
    
    廿二日\footnote{〔廿二日〕农历二月二十二日。}天稍和,偕数友出东直\footnote{〔东直〕北京东直门,在旧城东北角。满井在东直门北三四里。},至满井。高柳夹堤,土膏微润,一望空阔,若脱笼之鹄。于时冰皮始解,波色乍明,鳞浪层层,清澈见底,晶晶然如镜之新开而冷光之乍出于匣也。山峦为晴雪所洗,娟然如拭,鲜妍明媚,如倩女之靧面而髻鬟之始掠也。柳条将舒未舒,柔梢披风,麦田浅鬣寸许。游人虽未盛,泉而茗者,罍而歌者,红装而蹇者,亦时时有。风力虽尚劲,然徒步则汗出浃背。凡曝沙之鸟,呷浪之鳞,悠然自得,毛羽鳞鬣之间皆有喜气。始知郊田之外未始无春,而城居者未之知也。
    
    夫不能以游堕事而潇然于山石草木之间者,惟此官也。而此地适与余近,余之游将自此始,恶能无纪?己亥\footnote{〔己亥〕明朝万历二十七年(公元1599年)。}之二月也。
\end{normalsize}


\newpage

\textbf{译文}:

\vspace{1em}

\begin{normalsize}
    
    燕地一带(气候)寒冷,花朝节过后,严寒的余威还很厉害。时常刮起冷风,起风时沙土飞扬,碎石子乱滚。(我)(被)拘束在一间屋子里,想出去(却)不行。每次顶着风急速行走,没(走)一百步就得回头了。
    
    二十二日天气略微暖和,(我)和几个朋友一起从东直门出去,到了满井。高高的柳树长在河堤的两旁,肥沃的土地微微湿润,一眼看过去空阔无际,(我)好像是从笼中飞出去的天鹅。在这时水面的一层冰开始融化了,水波开始发出亮光,像鱼鳞似的浪纹层层推动,清澈透明,可以看到底,(水面)亮晶晶的好像镜子新打开,清冷的光突然从镜匣里射出一样。山峦被融化的雪水洗干净,美好的样子像擦拭过一样,鲜艳悦目,像美丽的少女洗了脸刚梳好髻鬟一样。柳条将要舒展却还未舒展,柔软的柳梢在风中散开,麦苗高约一寸。游人虽然不是很多,汲泉水煮茶喝的,端着酒杯唱歌的,穿着艳装骑驴的,也时常有。风力虽然依然猛烈,但是步行却会汗水湿透背。在沙滩上晒太阳的鸟,浮到水面戏水的鱼,安适愉快,自得其乐,一切动物之中都有喜悦的气氛。(我)才知道郊田的外面未尝没有春天,只是居住在城里的人不知道这一点。
    
    不会因为游玩而耽误公事,能无拘无束潇洒在山石草木之间游玩的,只有(我)这个闲官了吧。而这个地方正好和我(住的地方)很近,我将从这里开始游玩,怎能不记载一下呢?(写于)己亥年的二月。
    
\end{normalsize}



\chapter{马说}

\begin{normalsize}
    
    世有伯乐\footnote{〔伯乐〕孙阳。春秋时人,擅长相马,得到秦穆公信赖,被封为“伯乐将军”。},然后有千里马。千里马常有,而伯乐不常有。故虽有名马,祇辱于奴隶人之手,骈死于槽枥之间,不以千里称也。
    
    马之千里者,一食或尽粟一石\footnote{〔石〕容量单位,十斗为一石。}。食马者,不知其能千里而食也。是马也,虽有千里之能,食不饱,力不足,才美不外见,且欲与常马等不可得,安求其能千里也?
    
    策之不以其道,食之不能尽其材,鸣之而不能通其意,执策而临之,曰:“天下无马!”呜呼,其真无马邪?其真不知马也!
\end{normalsize}


\newpage

\textbf{译文}:

\vspace{1em}

\begin{normalsize}
    
    世上先有伯乐,然后才有千里马。千里马常有,但是伯乐不常有。因此即使有名贵的马,只能辱没在马夫的手里,跟普通的马一同死在槽枥之间,不以千里马著称。
    
    日行千里的马,吃一顿有时能吃尽一石粮食。饲养马的人不懂得它有能日行千里的能力而像普通的马来喂养它。这样的马,虽然有日行千里的才能,但吃不饱,力气不足,才能和品德就显现不出来,想要和普通的马等同尚且不可能,怎么能要求它日行千里呢?
    
    驱使千里马不能按照正确的方法;喂养它,不能够充分发挥它的才能;听千里马嘶鸣,却不能懂得它的意思,只是握着马鞭站到它的跟前,说:“天下没有千里马!”唉,难道(这世上)是真的没有千里马吗?恐怕是真的不认识千里马吧!
    
\end{normalsize}



\chapter{五柳先生传}

\begin{normalsize}
    
    先生不知何许人也,亦不详其姓字。宅边有五柳树,因以为号焉。闲静少言,不慕荣利。好读书,不求甚解;每有会意,便欣然忘食。性嗜酒,家贫不能常得。亲旧知其如此,或置酒而招之;造饮辄尽,期在必醉。既醉而退,曾不吝情去留。环堵萧然,不蔽风日;短褐穿结,箪瓢屡空,晏如也。常著文章自娱,颇示己志。忘怀得失,以此自终。
    
    赞曰:黔娄\footnote{〔黔娄〕战国时齐国的隐士。}之妻有言:“不戚戚于贫贱,不汲汲于富贵。”其言兹若人之俦乎?衔觞赋诗,以乐其志。无怀氏\footnote{〔无怀氏〕传说中的上古氏族。}之民欤,葛天氏\footnote{〔葛天氏〕传说中的上古氏族。}之民欤?
\end{normalsize}


\newpage

\textbf{译文}:

\vspace{1em}

\begin{normalsize}
    
    五柳先生不知道是哪里的人,也不知道他的姓名。房子旁边种着五棵柳树,就以此为号。(他)安安静静的,很少说话,不羡慕荣华利禄。爱好读书,只求领会要旨,不在一字一句的解释上过分深究;每当对书中的内容有所领会的时候,就会高兴得忘了吃饭。天性喜欢喝酒,但家境贫寒而不能常喝。亲戚朋友知道他有此嗜好,有时摆了酒席来招待他;(他)去喝酒就喝个尽兴,希望一定喝醉。(他)(只要)喝醉了就回家去,也没有舍不得离开。 简陋的居室里空空荡荡,不能遮蔽住风和阳光;粗布短衣上打了补丁,盛饭的篮子和喝水用的瓢里经常是空的,但他依然安然自若。经常以写文章来自娱自乐,很是能表达自己的志趣。不把自己得失放在心上,就这样过完自己的一生。
    
    赞语说:黔娄的妻子曾经说过:“不为贫贱而忧愁,不热衷于发财做官。”这话大概说的就是五柳先生一类的人吧?一边喝酒一边作诗,为自己的志向感到快乐。他大概是无怀氏或葛天氏的时候的百姓吧?
    
\end{normalsize}



\chapter{小石潭记}

\begin{normalsize}
    
    从小丘\footnote{〔小丘〕即钴鉧潭西小丘,见前一篇《钴鉧潭西小丘记》。}西行百二十步,隔篁竹,闻水声,如鸣佩环,心乐之。伐竹取道,下见小潭,水尤清冽。全石以为底,近岸,卷石底以出,为坻,为屿,为嵁,为岩。青树翠蔓,蒙络摇缀,参差披拂。
    
    潭中鱼可百许头,皆若空游无所依,日光下澈,影布石上。佁然不动,俶尔远逝,往来翕忽。似与游者相乐。
    
    潭西南而望,斗折蛇行,明灭可见。其岸势犬牙差互,不可知其源。
    
    坐潭上,四面竹树环合,寂寥无人,凄神寒骨,悄怆幽邃。以其境过清,不可久居,乃记之而去。
    
    同游者:吴武陵\footnote{〔吴武陵〕信州(今重庆奉节一带)人,唐宪宗元和初进士,因罪贬官永州,与作者友善。},龚古\footnote{〔龚古〕作者朋友。},余弟宗玄\footnote{〔宗玄〕作者的堂弟。}。隶而从者,崔氏二小生:曰恕己,曰奉壹。
\end{normalsize}


\newpage

\textbf{译文}:

\vspace{1em}

\begin{normalsize}
    
    从小土丘往西走约一百二十步,隔着竹林,听到水声,好象挂在身上的玉佩、玉环相互碰撞的声音,心里很是高兴。(于是)砍伐竹子,开出一条道路,下面显现出一个小小的水潭,潭水特别清凉。潭以整块石头为底,靠近岸边,石底向上弯曲,露出水面,像各种各样的石头和小岛。青葱的树木,翠绿的藤蔓,遮掩缠绕,摇动下垂,参差不齐,随风飘动。
    
    潭中大约有一百来条鱼,都好像在空中游动,没有什么依靠似的。阳光往下一直照到潭底,鱼儿的影子映在水底的石上。(鱼儿)呆呆地静止不动,忽然间(又)向远处游去,来来往往,轻快敏捷,好像跟游人逗乐似的。
    
    向石潭的西南方向望去,(溪流)像北斗七星那样的曲折,(又)像蛇爬行一样的蜿蜒,(有时)看得见,(有时)看不见。两岸的形状像犬牙似的参差不齐,看不出溪水的源头在哪里。
    
    坐在石潭边上,四面被竹林树木包围着,静悄悄的,空无一人,(这气氛)使人感到心神凄凉,寒气透骨,幽静深远,弥漫着忧伤的气息。因为环境过于凄清,不能长时间地待下去,就记下这番景致离开了。
    
    一同去游览的有吴武陵、龚古,我的弟弟宗玄。跟着一同去的还有姓崔的两个年轻人,一个叫恕己,一个叫奉壹。
    
\end{normalsize}



\chapter{岳阳楼记}

\begin{normalsize}
    
    庆历四年\footnote{〔庆历四年〕庆历,宋仁宗赵祯的年号(公元1041至1048年)。庆历四年是公元1044年。}春,滕子京\footnote{〔滕子京〕滕宗谅,字子京,范仲淹的朋友。任泾州知州抵御西夏有功,经范仲淹举荐,擢天章阁待制。范仲淹发起庆历新政后遭攻击,因“泾州公款案”被贬,庆历四年任岳州太守。}谪守巴陵郡\footnote{〔巴陵郡〕岳州,治所在今湖南省岳阳市。}。越明年,政通人和,百废具兴。乃重修岳阳楼,增其旧制,刻唐贤今人诗赋于其上。属予作文以记之。
    
    予观夫巴陵胜状,在洞庭一湖。衔远山,吞长江,浩浩汤汤,横无际涯;朝晖夕阴,气象万千。此则岳阳楼之大观也,前人之述备矣。然则北通巫峡\footnote{〔巫峡〕长江三峡之一。},南极潇湘\footnote{〔潇湘〕潇水是湘水的支流。湘水流入洞庭湖。},迁客骚人,多会于此。览物之情,得无异乎?
    
    若夫淫雨霏霏,连月不开,阴风怒号,浊浪排空;日星隐曜,山岳潜形;商旅不行,樯倾楫摧;薄暮冥冥,虎啸猿啼。登斯楼也,则有去国怀乡,忧谗畏讥,满目萧然,感极而悲者矣。
    
    至若春和景明,波澜不惊,上下天光,一碧万顷;沙鸥翔集,锦鳞游泳;岸芷汀兰,郁郁青青。而或长烟一空,皓月千里,浮光跃金,静影沉璧,渔歌互答,此乐何极!登斯楼也,则有心旷神怡,宠辱偕忘,把酒临风,其喜洋洋者矣。
    
    嗟夫!予尝求古仁人之心,或异二者之为。何哉?不以物喜,不以己悲;居庙堂之高则忧其民;处江湖之远则忧其君。是进亦忧,退亦忧。然则何时而乐耶?其必曰:“先天下之忧而忧,后天下之乐而乐”乎。噫!微斯人,吾谁与归?
    
    时六年九月十五日。
\end{normalsize}


\newpage

\textbf{译文}:

\vspace{1em}

\begin{normalsize}
    
    庆历四年的春天,滕子京降职到岳州做太守。到了第二年,政事顺利,百姓和乐,很多长年荒废的事业又重新兴办起来了。还重新修建了岳阳楼,扩大它旧有的规模,还在上面刻上唐代贤人和当代人的诗赋。(滕子京)并嘱咐(我)写一篇文章用来记述这件事。
    
    我看那巴陵郡的美丽的景色,集中在洞庭湖上。洞庭湖连接着远处的群山,吞吐长江的江水,水波浩荡,宽阔无边;或早或晚(一天里)时阴时晴,景象千变万化。这就是岳阳楼的雄伟景象。前人对它的描述已经很详尽了。然而,因为这里往北面通向巫峡,南面直到潇水、湘水,被降职远调的官吏和南来北往的诗人,大多在这里聚会。
    
    如果遇上阴雨连绵繁密,有时连着整个月没有晴天,寒风怒吼,浊浪冲天,太阳和星星隐藏了光辉,山岳隐没了形体;商人和旅客无法通行,桅杆倒下,船桨折断;傍晚天色昏暗,虎在长啸,猿在哀啼。(此时)登上岳阳楼,就会产生离开国都,怀念家乡,担心(人家)说坏话,惧怕批评指责的感觉,满眼是萧条的景象,感慨悲伤到极点啊。
    
    至于春风和煦,阳光明媚的日子,湖面风平浪静,天色湖光相接,一片碧绿,广阔无际;沙鸥时而飞翔,时而停歇,美丽的鱼儿在湖中游来游去;湖岸上的小草和沙洲上的兰花,香气浓郁,草木茂盛。而有时大片烟雾完全消散,皎洁的月光一泻千里,(月光照耀下的)水波闪耀着金光;无风时静静的月影好似沉入水中的玉璧,渔夫的歌声一唱一和,这样的乐趣哪有穷尽!(此时)登上岳阳楼,就会有心胸开阔,精神愉悦,忘却荣辱得失,举起酒杯面对和风,喜气洋洋的感觉。
    
    唉!我曾经探求过古时品德高尚的人的思想,或许不同于(以上)两种心情。这是为什么呢?他们不因为外物的好坏和个人的得失而或喜或悲;在朝廷作官的人为百姓担忧;不在朝廷作官的人为君王担忧。这样在朝为官也担忧,在野为民也担忧。既然这样,那么,什么时候才快乐呢?那一定要说“在天下人忧虑之前先忧虑,在天下人快乐之后再快乐”吧?唉!(如果)没有这种人,我同谁一道呢?
    
    写于庆历六年九月十五日。
    
\end{normalsize}



\chapter{与朱元思书}

\begin{normalsize}
    
    风烟俱净,天山共色。从流飘荡,任意东西。自富阳至桐庐\footnote{〔富阳至桐庐〕富阳与桐庐都在杭州境内,富阳在富春江下游,桐庐在富阳的西南中游。如按上文“从流飘荡”。则应为“从桐庐至富阳”,可能为作者笔误。},一百许里,奇山异水,天下独绝。
    
    水皆缥碧,千丈见底。游鱼细石,直视无碍。急湍甚箭,猛浪若奔。
    
    夹岸高山,皆生寒树。负势竞上,互相轩邈;争高直指,千百成峰。泉水激石,泠泠作响;好鸟相鸣,嘤嘤成韵。蝉则千转不穷,猿则百叫无绝。鸢飞戾天\footnote{〔鸢飞戾天〕出自《诗经·大雅·旱麓》。老鹰高飞入天,这里比喻极力追求名利的人。}者,望峰息心;经纶世务者,窥谷忘反。横柯上蔽,在昼犹昏;疏条交映,有时见日。
\end{normalsize}


\newpage

\textbf{译文}:

\vspace{1em}

\begin{normalsize}
    
    风和烟都散尽了,天和山是一样的颜色。(我的小船)随着江流飘荡,时而偏东,时而偏西。从富阳到桐庐,一百来里奇异的山水,是天下独一无二的景色。
    
    江水都是青白色,千丈深的地方都能看得清楚。游动的鱼儿和细碎的沙石,也可以看得清清楚楚,毫无障碍。湍急的水流比箭还快,迅猛的浪涛像飞奔的骏马。
    
    江两岸的高山上,全都生长着使人看了有寒意的树。山峦凭借着(高峻的)地势,争着向上,仿佛都在相互争着往高处和远处伸展,笔直地向上,直插云天,形成了无数的山峰。(山间的)泉水冲击着岩石,发出泠泠的响声;美丽的百鸟互相和鸣,鸣声嘤嘤,和谐动听。蝉儿和猿猴也长时间地叫个不断。极力追求名利的人,看到(这些雄奇的)高峰,(就会)平息热衷于功名利禄的心;管理世俗事务的人,看到(这些幽美的)山谷,(就会)忘记返回。横斜的树枝在上面遮蔽着,即使是在白天也像黄昏时那样昏暗;稀疏的枝条交相掩映,有时还可以(从枝叶的空隙中)见到阳光。
    
\end{normalsize}



\chapter{醉翁亭记}

\begin{normalsize}
    
    环滁\footnote{〔滁〕滁州,今安徽省滁州市琅琊区。}皆山也。其西南诸峰,林壑尤美,望之蔚然而深秀者,琅琊也\footnote{〔琅琊〕山名,在滁州市。}。山行六七里,渐闻水声潺潺而泻出于两峰之间者,酿泉也。峰回路转,有亭翼然临于泉上者,醉翁亭也。作亭者谁?山之僧智仙也。名之者谁?太守\footnote{〔太守〕秦汉时一郡之长。隋朝存州废郡后不再称太守。一州之长唐代称为刺史,北宋称为知州,但一般仍习惯用太守来称呼。}自谓也。太守与客来饮于此,饮少辄醉,而年又最高,故自号曰醉翁也。醉翁之意不在酒,在乎山水之间也。山水之乐,得之心而寓之酒也。
    
    若夫日出而林霏开,云归而岩穴暝,晦明变化者,山间之朝暮也。野芳发而幽香,佳木秀而繁阴,风霜高洁,水落而石出者,山间之四时也。朝而往,暮而归,四时之景不同,而乐亦无穷也。
    
    至于负者歌于途,行者休于树,前者呼,后者应,伛偻提携,往来而不绝者,滁人游也。临溪而渔,溪深而鱼肥,酿泉为酒,泉香而酒洌,山肴野蔌,杂然而前陈者,太守宴也。宴酣之乐,非丝非竹,射者中,弈者胜,觥筹交错,起坐而喧哗者,众宾欢也。苍颜白发,颓然乎其间者,太守醉也。
    
    已而夕阳在山,人影散乱。太守归而宾客从也。树林阴翳,鸣声上下,游人去而禽鸟乐也。然而禽鸟知山林之乐,而不知人之乐;人知从太守游而乐,而不知太守之乐其乐也。醉能同其乐,醒能述以文者,太守也。太守谓谁?庐陵\footnote{〔庐陵〕庐陵郡,就是吉洲。现在江西省吉安市。}欧阳修也。
\end{normalsize}


\newpage

\textbf{译文}:

\vspace{1em}

\begin{normalsize}
    
    环绕着滁州城的都是山。城西南方向的各个山峰,树林和山谷尤其美丽,远望那树木茂盛,又幽深又秀丽的,是琅琊山。沿着山路行走了六七里,渐渐地听到潺潺的流水声,从两座山峰中间倾泻而下的,是酿泉。山势回环,路也跟着拐弯,有一座亭子四角翘起,像鸟张开翅膀一样,坐落在泉水边上,这就是醉翁亭。修建亭子的人是谁?是山里的老僧智仙。给它起名字的人是谁?是太守用自己的别号(醉翁)来命名的。太守和宾客来这里喝酒,喝一点就醉了,而年纪又最大,所以给自己起了个名号叫醉翁。醉翁的情趣不在于(喝)酒,而在于欣赏山水美景。欣赏山水美景的乐趣,是领会在心里,而寄托在喝酒上的。
    
    像那太阳出来,树林中的雾气散去,云聚拢过来,山里就昏暗了,或暗或明,变化不一,这就是山间早晚的景象。野花开了,散发出一股清幽的香味,好看的树木枝叶繁茂,形成一片浓郁的绿荫,天气高爽,霜色洁白,水面低落下去,石头裸露出来,是山中四季的景色。早晨上山,傍晚返回,四季的景色不同,那乐趣也是没有穷尽的。
    
    至于背着东西的人在路上歌唱,走路的人在树下休息,前面的人呼喊,后面的人应答,老老小小,来来往往络绎不绝的,是滁州人在游山啊。到溪边捕鱼,溪水深鱼儿肥,用泉水酿酒,泉水香甜,酒水清澈,山中的野味野菜,杂乱地摆放在前面,这是太守在举行酒宴。宴会喝酒的乐趣,不在于音乐,投壶的人射中了目标,下棋的人得胜了,酒杯和酒筹交互错杂,时起时坐,大声喧哗的,是众位宾客欢乐的样子。脸色苍老,头发花白,醉醺醺地坐在众人中间的,是太守喝醉了。
    
    不久太阳落到山顶,人的影子散乱一地。太守下山回家,宾客跟随着。树林茂密阴蔽,鸟儿到处鸣叫,那是因为是游人离开后鸟儿们在快乐啊。然而鸟儿只知道山林的乐趣,却不知道游人的乐趣,游人只知道跟随太守游玩的乐趣,却不知道太守以宾客的快乐为快乐。醉了能够同大家一起快乐,醒来能够用文章记述这乐事的人,是太守。太守是谁?是庐陵人欧阳修。
    
\end{normalsize}



\chapter{陈涉世家}

\begin{normalsize}
    
    陈胜者,阳城\footnote{〔阳城〕现在河南登封市东南。。}人也,字涉。吴广者,阳夏\footnote{〔阳夏〕现在河南太康县。}人也,字叔。陈涉少时,尝与人佣耕,辍耕之垄上,怅恨久之,曰:“苟富贵,无相忘。”佣者笑而应曰:“若为佣耕,何富贵也?”陈涉太息曰:“嗟乎!燕雀安知鸿鹄之志哉!”
    
    二世元年\footnote{〔二世元年〕公元前209年。秦始皇死后,幼子胡亥继位,称为二世。}七月,发闾左适戍渔阳\footnote{〔渔阳〕渔阳郡,秦朝至唐朝常设的郡,包括今天北京市、天津市、河北省部分地区。郡治是渔阳县,在今北京怀柔区。},九百人屯大泽乡。陈胜、吴广皆次当行,为屯长。会天大雨,道不通,度已失期。失期,法皆斩。陈胜、吴广乃谋曰:“今亡亦死,举大计亦死;等死,死国可乎?”陈胜曰:“天下苦秦久矣。吾闻二世少子也,不当立,当立者乃公子扶苏\footnote{〔公子扶苏〕秦始皇长子,秦始皇驾崩后,赵高和李斯等人矫诏谋杀扶苏,改立公子胡亥为帝,是为沙丘之变。}。扶苏以数谏故,上使外将兵。今或闻无罪,二世杀之。百姓多闻其贤,未知其死也。项燕\footnote{〔项燕〕战国末年楚国著名将领,项梁之父,项羽的祖父,曾大败秦将李信。}为楚将,数有功,爱士卒,楚人怜之。或以为死,或以为亡。今诚以吾众诈自称公子扶苏、项燕,为天下唱,宜多应者。”吴广以为然。乃行卜。卜者知其指意,曰:“足下事皆成,有功。然足下卜之鬼乎!”陈胜、吴广喜,念鬼,曰:“此教我先威众耳。”乃丹书帛曰“陈胜王”,置人所罾鱼腹中。卒买鱼烹食,得鱼腹中书,固以怪之矣。又间令吴广之次所旁丛祠中,夜篝火,狐鸣呼曰:“大楚兴,陈胜王。”卒皆夜惊恐。旦日,卒中往往语,皆指目陈胜。
    
    吴广素爱人,士卒多为用者。将尉醉,广故数言欲亡,忿恚尉,令辱之,以激怒其众。尉果笞广。尉剑挺,广起,夺而杀尉。陈胜佐之,并杀两尉。召令徒属曰:“公等遇雨,皆已失期,失期当斩。藉第令毋斩,而戍死者固十六七。且壮士不死即已,死即举大名耳,王侯将相宁有种乎!”徒属皆曰:“敬受命。”乃诈称公子扶苏、项燕,从民欲也。袒右,称大楚。为坛而盟,祭以尉首。陈胜自立为将军,吴广为都尉。攻大泽乡,收而攻蕲\footnote{〔蕲〕现在安徽宿州市南。}。蕲下,乃令符离\footnote{〔符离〕现在安徽宿州。}人葛婴将兵徇蕲以东。攻铚、酂、苦、柘、谯\footnote{〔铚、酂、苦、柘、谯〕地名,现在安徽、河南一带。}皆下之。行收兵。比至陈\footnote{〔陈〕秦时县名,今河南淮阳。},车六七百乘,骑千余,卒数万人。攻陈,陈守令皆不在,独守丞\footnote{〔守丞〕辅佐郡守的主官,也叫郡丞。}与战谯门中,弗胜,守丞死,乃入据陈。数日,号令召三老\footnote{〔三老〕每乡掌管教化民众、举荐人才的职位。由五十岁以上的人担任。}、豪杰与皆来会计事。三老、豪杰皆曰:“将军身被坚执锐,伐无道,诛暴秦,复立楚国之社稷,功宜为王。”陈涉乃立为王,号为张楚。
    
    当此时,诸郡县苦秦吏者,皆刑其长吏,杀之以应陈涉。乃以吴叔为假王,监诸将以西击荥阳\footnote{〔荥阳〕现在河南荥阳市。}。令陈人武臣、张耳、陈馀徇赵地,令汝阴\footnote{〔汝阴〕汝阴县,现在安徽省阜阳市。}人邓宗徇九江郡\footnote{〔九江郡〕郡名,包括江西全境、安徽的淮南及河南的一小部分,郡治在寿春(今安徽寿县)。}。当此时,楚兵数千人为聚者,不可胜数。
\end{normalsize}


\newpage

\textbf{译文}:

\vspace{1em}

\begin{normalsize}
    
    陈胜,是阳城人,字涉。吴广,是阳夏人,字叔。陈涉年少的时候,曾经和人一起被雇佣耕田。一次陈涉耕田时停了下来,来到田埂上休息,愤恨不平了很长时间,说:“如果有一天富贵了,可不要相互忘记。”一同受雇的耕者笑着回答他说:“你不过是个受雇耕田的,能有什么富贵呢?”陈涉叹息着说:“唉!
    
    秦二世元年七月,征发闾巷左侧的贫民九百人去守卫渔阳,停驻在大泽乡。陈胜、吴广都被编进队伍里,担任屯长。正遇天降大雨,道路不通,估计着已经误了到达的期限。耽误期限,根据秦朝法律就都会被斩首。陈胜、吴广于是商议说:“如今逃走也是死,举行起义干一番大事业也是死,一样是死,为国家举大事而死怎么样?”陈胜说:“天下百姓苦于秦朝的暴政已经很长时间了。我听说二世是始皇的小儿子,不应该即位,应该即位的是公子扶苏。扶苏由于数次劝谏始皇的缘故,始皇就派他到外地统领军队。如今有人听说他并没犯罪,二世就杀掉了他。百姓大多听说过他的贤能,并不知道他已经死了。项燕作为楚国的将军,多次立有战功,爱护手下士卒,楚国人都很爱戴他。有的人认为他死了,有的人认为他逃走了。如今我们这些人谎称是公子扶苏、项燕,成为天下反秦的倡导者,应该会有许多人响应。”吴广认为他说得对。于是进行占卜。为他们占卜的人知道他们的想法,说:“先生要做的事情都能成功,创建大功业。但先生何不把这件事向鬼神卜问呢!”陈胜、吴广非常高兴,思量着向鬼神卜问是什么意思,说:“这是让我先在群众中树立威信。”就用丹砂在绸上写上“陈胜王”,放入别人用网捕捉到的鱼肚子里。士兵们买回那条鱼煮着吃,得到鱼肚子里的帛书,自然就觉得奇怪了。陈胜又暗中让吴广前往戍卒驻地附近丛林中的神庙里,在夜间点起篝火,学着狐狸的叫声说“大楚兴,陈胜王”。士兵们都在夜里惊惧恐慌。第二天早上,士兵们到处议论纷纷,暗中指点、目视陈胜。
    
    吴广一向关心别人,士兵中有许多人都听他的。一次带队的县尉喝酒醉了,吴广故意再三说自己想要逃走,使县尉恼怒,侮辱自己,以此来激怒众人。县尉果然用竹板打吴广。县尉拔出佩剑,吴广跳起来,夺剑杀了县尉。陈胜协助吴广,一起杀掉了两名县尉。陈胜召集起士兵们宣告说:“各位遇上了大雨,都已经错过了前往渔阳的期限,错过期限就应当被斩首。即使不斩首,而守卫边防本来在十个人中就会死去六七个。况且壮士不死就罢了,死就要闯出一番名声,难道王侯将相都是天生的贵种吗!”士兵们都说:“恭敬地接受您的命令。”陈胜等人于是对外假称是公子扶苏、项燕,顺从百姓的意愿。大家都裸露出右臂,打着“大楚”的旗号。修筑高坛盟誓,以县尉的首级作为祭天的祭品。陈胜自封为将军,吴广担任都尉。攻打大泽乡,征集士兵后攻打蕲县。攻克蕲县后,就命令符离人葛婴率领军队攻打蕲县以东的地区。攻打铚、酂、苦、柘、谯这些地区,全部攻占。在行军时招收士卒。等攻到陈县时,起义军已经有了六七百辆战车,一千多名骑兵,几万名步兵。攻打陈县,陈县的郡守和县令都不在城中,只有守丞带着部队和陈胜的军队在谯门中作战,没能取胜,陈县的守丞死了,起义军趁机入城占领了陈县。几天后,陈胜下令召集当地的三老、豪杰都来集会谋划事情。三老和豪杰都说:“将军亲自披着铠甲,手拿着锐利武器,讨伐诛戮残暴无道的秦朝,恢复楚国的社稷,根据功劳应该称王。”陈涉于是自立为王,定国号为张楚。
    
    这时候,各郡县因秦朝官吏的治理而受苦的百姓,都劫持他们当地的长官,杀掉他们以响应陈涉。(起义军)就以吴广为假王,督促各路将领向西攻打荥阳。陈涉命令陈县人武臣、张耳、陈余攻打赵地,命汝阴人邓宗攻打九江郡。在此时,楚地的义军几千人聚在一起的,数都数不过来。
    
\end{normalsize}



\chapter{出师表}

\begin{normalsize}
    
    先帝\footnote{〔先帝〕指刘备。}创业未半而中道崩殂,今天下三分,益州\footnote{〔益州〕指蜀汉。}疲弊,此诚危急存亡之秋也。然侍卫之臣不懈于内,忠志之士忘身于外者,盖追先帝之殊遇,欲报之于陛下\footnote{〔陛下〕指刘禅。}也。诚宜开张圣听,以光先帝遗德,恢弘志士之气,不宜妄自菲薄,引喻失义,以塞忠谏之路也。
    
    宫中府中,俱为一体;陟罚臧否,不宜异同:若有作奸犯科及为忠善者,宜付有司论其刑赏,以昭陛下平明之理;不宜偏私,使内外异法也。
    
    侍中、侍郎郭攸之、费祎、董允等,此皆良实,志虑忠纯,是以先帝简拔以遗陛下:愚以为宫中之事,事无大小,悉以咨之,然后施行,必能裨补阙漏,有所广益。
    
    将军向宠,性行淑均,晓畅军事,试用于昔日,先帝称之曰“能”,是以众议举宠为督:愚以为营中之事,悉以咨之,必能使行阵和睦,优劣得所。
    
    亲贤臣,远小人,此先汉所以兴隆也;亲小人,远贤臣,此后汉所以倾颓也。先帝在时,每与臣论此事,未尝不叹息痛恨于桓、灵也。侍中、尚书、长史、参军,此悉贞良死节之臣,愿陛下亲之信之,则汉室之隆,可计日而待也。
    
    臣本布衣,躬耕于南阳\footnote{〔南阳〕现在河南省南阳市一带。},苟全性命于乱世,不求闻达于诸侯。先帝不以臣卑鄙,猥自枉屈,三顾臣于草庐之中,咨臣以当世之事,由是感激,遂许先帝以驱驰。后值倾覆,受任于败军之际,奉命于危难之间:尔来二十有一年矣。
    
    先帝知臣谨慎,故临崩寄臣以大事也。受命以来,夙夜忧叹,恐托付不效,以伤先帝之明;故五月渡泸,深入不毛。今南方已定,兵甲已足,当奖率三军,北定中原,庶竭驽钝,攘除奸凶,兴复汉室,还于旧都。此臣所以报先帝而忠陛下之职分也。至于斟酌损益,进尽忠言,则攸之、祎、允之任也。
    
    愿陛下托臣以讨贼兴复之效,不效,则治臣之罪,以告先帝之灵。若无兴德之言,则责攸之、祎、允等之慢,以彰其咎;陛下亦宜自谋,以咨诹善道,察纳雅言,深追先帝遗诏。臣不胜受恩感激。
    
    今当远离,临表涕零,不知所言。
\end{normalsize}


\newpage

\textbf{译文}:

\vspace{1em}

\begin{normalsize}
    
    先帝开创统一中原的大业还未完成一半却中途去世了。现在天下分为三个国家,我们蜀汉国力薄弱,处境艰难,这实在是国家危急存亡的时刻啊。然而侍奉保卫(陛下)的官员在宫廷内不懈怠,战场上忠诚有志的将士们奋不顾身,这是他们追念先帝对他们的特别的知遇之恩,想要报答陛下。
    
    皇宫中和朝廷中本都是一个整体,赏罚褒贬,不应该有所不同。
    
    侍中郭攸之、费祎和侍郎董允等人,都是善良诚实的人,他们的志向和心思忠诚无二,所以先帝把他们选拔出来辅佐陛下。
    
    将军向宠,性情品行善良平正,通晓军事,从前任用的时候,先帝称赞说他有才干,因此大家评议举荐他做中部督。
    
    亲近贤臣,疏远小人,这是西汉兴盛的原因;亲近小人,疏远贤臣,这是东汉衰败的原因。先帝在世的时候,常常跟我谈论这些事情,对于桓帝、灵帝没有一次不(发出叹息)感到痛心遗憾的。侍中、尚书、长史、参军,这些人都是忠贞诚实、能够以死报国的忠臣,希望陛下亲近他们,信任他们,那么汉朝的复兴就指日可待了。
    
    我本来是一介平民,在南阳务农亲耕,只想在乱世中苟且保全性命,不谋求在诸侯之中闻名显达。先帝不因为我身份低微、见识短浅,而委屈自己,三次去我的茅庐拜访我。征询我对时局大事的意见,因此使我感动奋激,答应为先帝奔走效劳。
    
    先帝知道我做事小心谨慎,所以临终时把国家大事托付给我。接受遗命以来,我整天担忧叹息,唯恐先帝托付给我的事不能完成,以致损伤先帝的知人之明,所以我五月渡过泸水,深入到人烟稀少的地方。现在南方已经平定,武器装备已经充足,应当激励、率领全军将士向北方进军,平定中原,希望竭尽我平庸的才能,铲除奸邪凶恶的敌人,恢复汉朝的基业,回到旧日的国都。这是我用来报答先帝,效忠陛下的职责本分。至于处理事务,斟酌情理,有所兴革,毫无保留地进献忠诚的建议,那就是郭攸之、费祎、董允等人的责任了。
    
    希望陛下能够把讨伐曹魏,兴复汉室的任务交给我,没有成效就治我的罪,从而用来告慰先帝的在天之灵。如果没有发扬圣德的言论,就责罚郭攸之、费祎、董允等人的怠慢,来揭示他们的过失;陛下也应自行谋划,询问治国的好方法,采纳正确的言论,深切追念先帝临终留下的教诲。我感激不尽。
    
    
    
\end{normalsize}



\chapter{隆中对}

\begin{normalsize}
    
    亮\footnote{〔亮〕指诸葛亮。}躬耕陇亩,好为《梁父吟》\footnote{〔《梁父吟》〕又叫《梁甫吟》,古歌曲名。传说诸葛亮曾经写过一首《梁父吟》歌词。}。身长八尺,每自比于管仲\footnote{〔管仲〕名夷吾,春秋时齐桓公的国相,帮助桓公建立霸业。}、乐毅\footnote{〔乐毅〕战国时燕昭王的名将,曾率领燕、赵、韩、魏、楚五国兵攻齐,连陷七十余城。},时人莫之许也。惟博陵崔州平\footnote{〔博陵崔州平〕博陵郡,属冀州,包括现在河北深州、安平、饶阳及安国等市县地。崔州平是博陵安平人,东汉太尉崔烈的儿子,西河郡太守崔钧的弟弟。}、颍川徐庶元直\footnote{〔颍川徐庶元直〕颍川郡,包括现在河南登封市、宝丰以东,尉氏、郾城以西,新密市以南,叶县、舞阳以北地。徐庶,字元直,豫州颍川长社人。在刘备帐下做谋士,向刘备举荐了诸葛亮。后来因母亲被曹军掳获,被迫归曹。}与亮友善,谓为信然。
    
    时先主\footnote{〔先主〕指刘备。}屯新野\footnote{〔新野〕现在河南新野县。}。徐庶见先主,先主器之,谓先主曰:"诸葛孔明者,卧龙也,将军岂愿见之乎?"先主曰:“君与俱来。”庶曰:“此人可就见,不可屈致也。将军宜枉驾顾之。”
    
    由是先主遂诣亮,凡三往,乃见。因屏人曰:“汉室倾颓,奸臣窃命,主上蒙尘。孤不度德量力,欲信大义于天下;而智术浅短,遂用猖蹶,至于今日。然志犹未已,君谓计将安出?"
    
    亮答曰:“自董卓\footnote{〔董卓〕东汉末年权臣,把持朝政、废立皇帝,后被王允设计杀害。}已来,豪杰并起,跨州连郡者不可胜数。曹操\footnote{〔曹操〕东汉末年枭雄,创立曹魏政权,统一北方。}比于袁绍\footnote{〔袁绍〕东汉末年军阀,讨伐董卓的盟主,后被曹操击败。},则名微而众寡。然操遂能克绍,以弱为强者,非惟天时,抑亦人谋也。今操已拥百万之众,挟天子而令诸侯,此诚不可与争锋。孙权\footnote{〔孙权〕汉破虏将军孙坚之子,建立东吴政权,割据江东。}据有江东,已历三世,国险而民附,贤能为之用,此可以为援而不可图也。荆州\footnote{〔荆州〕湖北和湖南的大部分地区,以及河南南部的一部分。}北据汉、沔\footnote{〔汉、沔〕指汉水中下游一带。汉水,古代通称沔水,流经今天的陕西南部、湖北西北部,最终在武汉汇入长江。},利尽南海\footnote{〔南海〕泛指南方两广等近海地区。},东连吴会\footnote{〔吴会〕吴郡和会稽郡的合称,现在江苏长江以南部分和浙江北部。},西通巴蜀\footnote{〔巴蜀〕巴郡、蜀郡,在现在的重庆和四川。},此用武之国,而其主不能守,此殆天所以资将军,将军岂有意乎?益州\footnote{〔益州〕蜀汉的中心区域,大致对应今天的四川省。}险塞,沃野千里,天府之土,高祖\footnote{〔高祖〕指刘邦,汉朝开国皇帝。}因之以成帝业。刘璋\footnote{〔刘璋〕汉室宗亲,益州牧。}暗弱,张鲁\footnote{〔张鲁〕东汉末年割据汉中、益州等地的军阀,五斗米道的第二代天师。}在北,民殷国富而不知存恤,智能之士思得明君。将军既帝室之胄,信义著于四海,总揽英雄,思贤如渴,若跨有荆、益,保其岩阻,西和诸戎,南抚夷越,外结好孙权,内修政理;天下有变,则命一上将将荆州之军以向宛、洛\footnote{〔宛、洛〕河南南阳和洛阳,这里泛指中原。},将军身率益州之众出于秦川\footnote{〔秦川〕指关中平原,位于今天的陕西省中部地区。},百姓孰敢不箪食壶浆,以迎将军者乎?诚如是,则霸业可成,汉室可兴矣。”
    
    先主曰:“善!”于是与亮情好日密。关羽\footnote{〔关羽〕东汉末年名将。与刘备、张飞结义。}、张飞\footnote{〔张飞〕东汉末年名将。与刘备、关羽结义。}等不悦,先主解之曰:“孤之有孔明,犹鱼之有水也。愿诸君勿复言。”羽、飞乃止。
\end{normalsize}


\newpage

\textbf{译文}:

\vspace{1em}

\begin{normalsize}
    
    诸葛亮亲自在田地中耕种,喜爱吟唱《梁父吟》。他身高八尺,常常把自己和管仲、乐毅相比,当时人们都不承认这件事。只有博陵的崔州平,颍川的徐庶与诸葛亮关系甚好,说确实是这样。
    
    适逢刘备驻扎在新野。徐庶拜见刘备,刘备很器重他,徐庶对刘备说:“诸葛孔明这个人,是人间卧伏着的龙啊,将军可愿意见他?”刘备说:“您和他一起来吧。”徐庶说:“这个人只能你去他那里拜访,不可以委屈他,召他上门来,将军你应该屈尊亲自去拜访他”。
    
    因此先帝就去隆中拜访诸葛亮,总共去了三次,才见到诸葛亮。于是刘备叫旁边的人退下,说:“汉室的统治崩溃,奸邪的臣子盗用政令,皇上蒙受风尘遭难出奔。我不能衡量自己的德行能否服人,估计自己的力量能否胜任,想要为天下人伸张大义,然而我才智与谋略短浅,就因此失败,弄到今天这个局面。但是我的志向到现在还没有罢休,您认为该采取怎样的办法呢?”
    
    诸葛亮回答道:“自董卓独掌大权以来,各地豪杰同时起兵,占据州、郡的人数不胜数。曹操与袁绍相比,声望少之又少,然而曹操最终之所以能打败袁绍,凭借弱小的力量战胜强大的原因,不仅依靠的是天时好,而且也是人的谋划得当。现在曹操已拥有百万大军,挟持皇帝来号令诸侯,这确实不能与他争强。孙权占据江东,已经历三世了,地势险要,民众归附,又任用了有才能的人,孙权这方面只可以把他作为外援,但是不可谋取他。荆州北靠汉水、沔水,一直到南海的物资都能得到,东面和吴郡、会稽郡相连,西边和巴郡、蜀郡相通,这是大家都要争夺的地方,但是它的主人却没有能力守住它,这大概是天拿它用来资助将军的,将军你可有占领它的意思呢?益州地势险要,有广阔肥沃的土地,自然条件优越,高祖凭借它建立了帝业。刘璋昏庸懦弱,张鲁在北面占据汉中,那里人民殷实富裕,物产丰富,刘璋却不知道爱惜,有才能的人都渴望得到贤明的君主。将军既是皇室的后代,而且声望很高,闻名天下,广泛地罗致英雄,思慕贤才,如饥似渴,如果能占据荆、益两州,固守险要之地,和西边的各个民族和好,又安抚南边的少数民族,对外联合孙权,对内革新政治;一旦天下形势发生了变化,就派一员上将率领荆州的军队直指中原一带,将军您亲自率领益州的军队从秦川出击,老百姓谁敢不用竹篮盛着饭食,用壶装着酒来欢迎将军您呢?如果真能这样做,那么称霸的事业就可以成功,汉室天下就可以复兴了。”
    
    刘备说:“好!”从此与诸葛亮的关系一天天亲密起来。关羽、张飞等人不高兴了,刘备劝解他们说:“我有了孔明,就像鱼得到水一样。希望你们不要再说什么了。”关羽、张飞于是不再说什么了。
    
\end{normalsize}



\chapter{唐雎不辱使命}

\begin{normalsize}
    
    秦王\footnote{〔秦王〕指嬴政。当时他还没称始皇帝。}使人谓安陵君\footnote{〔安陵君〕安陵国的国君。安陵是当时的一个小国,在现在河南鄢陵西北,原是魏国的附属国。战国时魏襄王封其弟为安陵君。}曰:“寡人欲以五百里之地易安陵,安陵君其许寡人!”安陵君曰:“大王加惠,以大易小,甚善;虽然,受地于先王,愿终守之,弗敢易!”秦王不说。安陵君因使唐雎\footnote{〔唐雎〕战国时期魏国、安陵国的谋士。}使于秦。
    
    秦王谓唐雎曰:“寡人欲以五百里之地易安陵,安陵君不听寡人,何也?且秦灭韩亡魏,而君以五十里之地存者,以君为长者,故不错意也。今吾以十倍之地,请广于君,而君逆寡人者,轻寡人与?”唐雎对曰:“否,非若是也。安陵君受地于先王而守之,虽千里不敢易也,岂直五百里哉?”
    
    秦王怫然怒,谓唐雎曰:“公亦尝闻天子之怒乎?”唐雎对曰:“臣未尝闻也。”秦王曰:“天子之怒,伏尸百万,流血千里。”唐雎曰:“大王尝闻布衣之怒乎?”秦王曰:“布衣之怒,亦免冠徒跣,以头抢地耳。”唐雎曰:“此庸夫之怒也,非士之怒也。夫专诸之刺王僚也\footnote{〔专诸之刺王僚〕专诸,春秋时刺客,受吴王阖闾之托刺杀吴王僚。},彗星袭月;聂政之刺韩傀也\footnote{〔聂政之刺韩傀〕聂政,战国时侠客,受严遂之托刺杀韩国相国侠累。},白虹贯日;要离之刺庆忌也\footnote{〔要离之刺庆忌〕要离,春秋时刺客,受吴王阖闾之托刺杀吴王僚之子庆忌。},仓鹰击于殿上。此三子者,皆布衣之士也,怀怒未发,休祲降于天,与臣而将四矣。若士必怒,伏尸二人,流血五步,天下缟素,今日是也。”挺剑而起。
    
    秦王色挠,长跪而谢之曰:“先生坐!何至于此!寡人谕矣:夫韩、魏灭亡,而安陵以五十里之地存者,徒以有先生也。”
\end{normalsize}


\newpage

\textbf{译文}:

\vspace{1em}

\begin{normalsize}
    
    秦王派人对安陵君说:“我想要用方圆五百里的土地交换安陵,安陵君一定要答应我啊!”安陵君说:“大王加以恩惠,用大的地盘交换我们小的地盘,这再好不过了;虽然是这样,但这是我从先王那继承的封地,我愿意一生守护它,不敢交换!”秦王知道后不高兴。于是安陵君就派遣唐雎出使到秦国。
    
    秦王对唐雎说:“我用方圆五百里的土地交换安陵,安陵君却不听从我,这是为什么?况且秦国使韩国魏国灭亡,但安陵却凭借方圆五十里的土地幸存下来的原因,是因为我把安陵君看作忠厚的长者,所以不打他的主意。现在我用安陵十倍的土地,让安陵君扩大自己的领土,但是他违背我的意愿,是他看不起我吗?”唐雎回答说:“不,并不是这样的。安陵君从先王那里继承了封地,只想守护它,即使是方圆千里的土地也不敢交换,更何况只是五百里的土地(就能交换)呢?”
    
    秦王勃然大怒,对唐雎说:“先生曾听说过天子发怒吗?”唐雎回答说:“我未曾听说过。”秦王说:“天子发怒(的时候),会倒下百万人的尸体,鲜血流淌千里。”唐雎说:“大王曾经听说过平民发怒吗?”秦王说:“平民发怒,也不过就是摘掉帽子,光着脚,把头往地上撞罢了。”唐雎说:“这是平庸无能的人发怒,不是有才能有胆识的人发怒。专诸刺杀吴王僚的时候,就如彗星的尾巴扫过月亮;聂政刺杀韩傀的时候,就如一道白光直贯穿太阳;要离刺杀庆忌的时候,就如苍鹰扑到宫殿上。他们三个人都是平民中有才能有胆识的人,心里的愤怒还没发作出来,上天就降示了吉凶的征兆。现在算上我,将成为四个人了。假若有胆识有能力的人(被逼得)一定要发怒,那么就让两个人的尸体倒下,五步之内淌满鲜血,天下百姓因此穿丧服,今天的情形就是这样了。”
    
    秦王变了脸色,直身而跪坐,向唐雎道歉说:“先生请坐!何必到这个地步!我明白了:韩国、魏国灭亡,但安陵却凭借方圆五十里的土地保全下来的原因,只是因为有先生您啊!”
    
\end{normalsize}



\chapter{曹刿论战}

\begin{normalsize}
    
    十年\footnote{〔十年〕鲁庄公十年(公元前684年)。}春,齐师伐我。公\footnote{〔公〕鲁庄公,鲁桓公之子,鲁桓公被齐襄公杀死后即位。曾经在齐襄公被杀后扶持齐公子纠争夺君位,但失败。鲁庄公八年,齐公子小白即位为齐桓公,之后讨伐鲁国。}将战,曹刿请见。其乡人曰:“肉食者\footnote{〔肉食者〕吃肉的人,指居高位,得厚禄的人。}谋之,又何间焉?”刿曰:“肉食者鄙,未能远谋。”乃入见。问:“何以战?”公曰:“衣食所安,弗敢专也,必以分人。”对曰:“小惠未遍,民弗从也。”公曰:“牺牲玉帛,弗敢加也,必以信。”对曰:“小信未孚,神弗福也。”公曰:“小大之狱,虽不能察,必以情。”对曰:“忠之属也,可以一战。战则请从。”
    
    公与之乘,战于长勺\footnote{〔长勺〕鲁国地名,现在山东曲阜县北。}。公将鼓之。刿曰:“未可。”齐人三鼓。刿曰:“可矣!”齐师败绩。公将驰之,刿曰:“未可。”下视其辙,登轼而望之,曰:“可矣。”遂逐齐师。
    
    既克,公问其故。对曰:“夫战,勇气也。一鼓作气,再而衰,三而竭。彼竭我盈,故克之。夫大国,难测也,惧有伏焉。吾视其辙乱,望其旗靡,故逐之。”
\end{normalsize}


\newpage

\textbf{译文}:

\vspace{1em}

\begin{normalsize}
    
    鲁庄公十年的春天,齐国的军队攻打鲁国。鲁庄公准备迎战,曹刿请求进见。他的同乡对他说:“大官们自会谋划这件事的,你又何必参与其间呢?”曹刿说:“大官们目光短浅,不能深谋远虑。”于是入宫进见鲁庄公。曹刿问鲁庄公:“您凭什么条件同齐国打仗?”庄公说:“衣食这类用来安生的东西,我不敢独自亨用,一定把它分给别人。”曹刿回答说:“这是小恩小惠,不能遍及百姓,百姓是不会跟从您的。”庄公说:“祭祀用的牛羊、玉帛之类,我不敢虚报,一定对神诚实。”曹刿回答说:“这是小信用,还不能使神信任您,神是不会保佑您的。”庄公说:“对于大大小小的诉讼案件,我虽不能一一明察,一定诚心诚意来处理。”曹刿回答说:“这是忠于职守的一种表现,可以凭这个条件打一仗。作战时请(让我)跟从(您)去。”
    
    鲁庄公和曹刿同乘一辆战车,在长勺和齐军作战。一开始,鲁庄公就要击鼓进军。曹刿说:“还不行。”齐军击鼓三次后,曹刿说:“可以击鼓进军了。”齐军被打得大败。鲁庄公就要下令驱车追击齐军,曹刿说:“还不行。”曹刿下车看了看地上齐军战车辗过的痕迹,又登上车前的横木远望齐军撤退的情况,说:“可以(追击)了。”于是追击齐军。
    
    战胜(齐军)以后,鲁庄公问取胜的原因。曹刿回答说:“打仗是靠勇气的。第一次击崐鼓,能够振作士兵的勇气,第二次击鼓,士兵的勇气就减弱了,第三次击鼓后士兵的勇气就消耗完了。他们的勇气已经完了,我们的勇气正旺盛,所以战胜了他们。但大国难以捉摸,恐怕有埋伏。我看到他们战车的车轮痕迹很乱,望见他们的军旗也已经倒下了,所以下令追击他们。”
    
\end{normalsize}



\chapter{登泰山记}

\begin{normalsize}
    
    泰山\footnote{〔泰山〕在山东泰安北,古称岱宗,又称东岳,为五岳之长。}之阳\footnote{〔阳〕山南水北称为“阳”,山北水南称为“阴”。},汶水\footnote{〔汶〕今称大汶河,源于山东莱芜东北之原山,向西南流,汇入东平湖。}西流;其阴,济水\footnote{〔济〕源于河南济源县西之王屋山,流经山东。清代末年,济水河道为黄河所占。}东流。阳谷皆入汶,阴谷皆入济。当其南北分者,古长城\footnote{〔古长城〕指战国时齐国修筑的长城,西起平阴,经泰山北冈,东至诸城。}也。最高日观峰\footnote{〔日观峰〕泰山顶峰,观日出的胜地。},在长城南十五里。
    
    余以乾隆\footnote{〔乾隆〕清高宗(爱新觉罗·弘历)的年号(公元1736至1796年)。}三十九年十二月,自京师乘风雪,历齐河、长清\footnote{〔齐河、长清〕山东两县名,在泰安西北。},穿泰山西北谷,越长城之限,至于泰安。是月丁未\footnote{〔丁未〕丁未日,农历12月28日。},与知府朱孝纯\footnote{〔朱孝纯〕字子颖,号海愚,山东历城人,当时是泰安府的知府,姚鼐挚友。}子颍由南麓登。四十五里,道皆砌石为磴,其级七千有余。泰山正南面有三谷。中谷绕泰安城下,郦道元\footnote{〔郦道元〕字善长,北魏范阳(今河北涿县)人,著有《水经注》。}所谓环水也。余始循以入,道少半,越中岭,复循西谷,遂至其巅。古时登山,循东谷入,道有天门\footnote{〔天门〕泰山有南天门、东天门、西天门。}。东谷者,古谓之天门溪水,余所不至也。今所经中岭及山巅,崖限当道者,世皆谓之天门云。道中迷雾冰滑,磴几不可登。及既上,苍山负雪,明烛天南。望晚日照城郭,汶水、徂徕\footnote{〔徂徕〕徂徕山,在泰安东南四十里。}如画,而半山居雾若带然。
    
    戊申晦\footnote{〔戊申〕戊申日,农历12月29日。},五鼓,与子颍坐日观亭\footnote{〔日观亭〕日观峰上一个看日出的亭。},待日出。大风扬积雪击面。亭东自足下皆云漫。稍见云中白若樗蒱\footnote{〔樗蒲〕骰子。}数十立者,山也。极天云一线异色,须臾成五采。日上,正赤如丹,下有红光动摇承之。或曰,此东海也。回视日观以西峰,或得日或否,绛皓驳色,而皆若偻。
    
    亭西有岱祠\footnote{〔岱祠〕祭祀东岳大帝的庙宇,也叫岱庙。},又有碧霞元君祠\footnote{〔碧霞元君祠〕祭祀东岳大帝女儿碧霞元君的庙,也叫娘娘庙。}。皇帝行宫在碧霞元君祠东。是日观道中石刻,自唐显庆\footnote{〔显庆〕唐高宗(李治)的年号(公元656至661年)。}以来;其远古刻尽漫失。僻不当道者,皆不及往。
    
    山多石,少土。石苍黑色,多平方,少圜。少杂树,多松,生石罅,皆平顶。冰雪,无瀑水,无鸟兽音迹。至日观数里内无树,而雪与人膝齐。
    
    桐城姚鼐记。
\end{normalsize}


\newpage

\textbf{译文}:

\vspace{1em}

\begin{normalsize}
    
    泰山的南面,汶河向西流去;泰山的北面,济水向东流去。南面山谷的水都流入汶水,北面山谷的 水都流入济水。在那南北山谷分界的地方,是古长城。最高的日观峰,在古长城以南十五里。
    
    我在乾隆三十九年十二月从京城里出发,冒着风雪启程,经过齐河县、长清县,穿过泰山西北面的山谷,跨过长城的城墙,到达泰安。这月丁未日(十二月二十八日),我和泰安知府朱孝纯从南边的山脚登山。攀行四十五里远,道路都是石板砌成的石级,共有七千多级。泰山正南面有三条山谷,中谷的水环绕泰安城,(这就是)郦道元书中所说的环水。我开始顺着(中谷)进去。道路不到一半,翻过中岭,再沿着西边的那条山谷走,就到了泰山的顶巅。古时候登泰山,沿着东边的山谷进入,道路中有座天门。这东边的山谷,古时候称它为“天门溪水”,是我没有到过的。现在经过的中岭和山顶,挡在路上的像门槛一样的山崖,世上人都称它为“天门”。一路上大雾弥漫、冰冻溜滑,石板石阶几乎无法攀登。等到已经登上山顶,只见青山上覆盖着白雪,(雪)光照亮了南面的天空。远望夕阳映照着泰安城,汶水、徂徕山就像是一幅美丽的山水画,停留在半山腰处的云雾,又像是一条舞动的飘带似的。
    
    戊申这一天是月底,五更的时候,(我)和子颖坐(在)日观亭里,等待日出。这时大风扬起的积雪扑面打来。日观亭东面从脚底往下一片云雾弥漫。依稀可见云中几十个白色的像骰子似的东西,(那是)山。天边云彩形成一条线(呈现出)奇异的颜色,一会儿又变成五颜六色。太阳升起来了,纯正的红色像朱砂一样,下面有红光晃动摇荡着托着它。有人说,这(就是)东海。回首观望日观峰以西的山峰,有的照到日光,有的照不到,或红或白,颜色错杂,都像弯腰曲背鞠躬致敬的样子。
    
    日观亭西面有一座东岳大帝庙,又有一座碧霞元君(东岳大帝的女儿)庙。皇帝的行宫(出外巡行时居住的处所)就在碧霞元君庙的东面。这一天,(还)观看了路上的石刻,都是从唐朝显庆年间以来的,那些更古老的石碑都已经模糊或缺失了。那些偏僻不对着道路的石刻,都赶不上去看了。
    
    山上石头多,泥土少。山石都呈青黑色,大多是平的、方形的,很少圆形的。杂树很少,多是松树,松树都生长在石头的缝隙里,树顶是平的。冰天雪地,没有瀑布,没有飞鸟走兽的声音和踪迹。日观峰附近几里以内没有(什么)树木,积雪厚得同人的膝盖一样平齐。
    
    桐城姚鼐记。
    
\end{normalsize}



\chapter{公输}

\begin{normalsize}
    
    公输盘\footnote{〔公输盘〕也写作“公输班”、“公输般”,世称鲁班。}为楚造云梯之械,成,将以攻宋。子墨子\footnote{〔子墨子〕墨翟,春秋末战国初期的思想家、政治家。}闻之,起于鲁,行十日十夜而至于郢\footnote{〔郢〕楚国的都城,在今湖北省江陵县附近。},见公输盘。
    
    公输盘曰:“夫子何命焉为?”子墨子曰:“北方有侮臣者,愿借子杀之。”公输盘不说。子墨子曰:“请献十金。”公输盘曰:“吾义固不杀人。”
    
    子墨子起,再拜,曰:“请说之。吾从北方闻子为梯,将以攻宋。宋何罪之有?荆国\footnote{〔荆国〕楚国的别称。}有余于地,而不足于民,杀所不足而争所有余,不可谓智。宋无罪而攻之,不可谓仁。知而不争,不可谓忠。争而不得,不可谓强。义不杀少而杀众,不可谓知类。”公输盘服。子墨子曰:“然胡不已乎?”公输盘曰:“不可,吾既已言之王\footnote{〔王〕楚惠王,春秋末战国初期楚国国君。}矣。”子墨子曰:“胡不见我于王?”公输盘曰:“诺。”
    
    子墨子见王,曰:“今有人于此,舍其文轩,邻有敝舆而欲窃之;舍其锦绣,邻有短褐而欲窃之;舍其粱肉,邻有糠糟而欲窃之。此为何若人?”王曰:“必为有窃疾矣。”子墨子曰:“荆之地方五千里,宋之地方五百里,此犹文轩之与敝舆也。荆有云梦\footnote{〔云梦〕楚国的大湖云梦泽。},犀兕麋鹿满之,江汉之鱼鳖鼋鼍为天下富,宋所谓无雉兔鲋鱼者也,此犹粱肉之与糠糟也。荆有长松文梓楩楠豫章,宋无长木,此犹锦绣之与短褐也。臣以王吏之攻宋也,为与此同类。”王曰:“善哉!虽然,公输盘为我为云梯,必取宋。”
    
    于是见公输盘。子墨子解带为城,以牒为械。公输盘九设攻城之机变,子墨子九距之。公输盘之攻械尽,子墨子之守圉有余。公输盘诎,而曰:“吾知所以距子矣,吾不言。”子墨子亦曰:“吾知子之所以距我者,吾不言。”楚王问其故。子墨子曰:“公输子之意不过欲杀臣。杀臣,宋莫能守,乃可攻也。然臣之弟子禽滑釐等三百人,已持臣守圉之器,在宋城上而待楚寇矣。虽杀臣,不能绝也。”楚王曰:“善哉。吾请无攻宋矣。”
    
    子墨子归,过宋,天雨,庇其闾中,守闾者不内也。故曰:“治于神者,众人不知其功;争于明者,众人知之。”
\end{normalsize}


\newpage

\textbf{译文}:

\vspace{1em}

\begin{normalsize}
    
    公输盘为楚国造了云梯这种攻城器械,已经制成了,准备用它攻打宋国。墨子听说了,就从齐国起身,走了十天十夜才到楚国的国都郢,见到公输盘。
    
    公输盘说:“先生有什么见教?”墨子说:“北方有人欺侮了我,我想请你帮我杀了他。”公输盘很不高兴。墨子说:“我愿意献给你十两黄金。”公输盘说:“我奉行义,决不杀人。”
    
    墨子站了起来,对公输盘拜了又拜,说:“请听我说。我在北方听说你造云梯,准备用它攻打宋国。宋国有什么罪呢?楚国有多余的土地,人口却不足,现在牺牲(本就)不足的人口,掠夺(本就)有余的土地,不能算是聪明。宋国无罪却去攻打它,不能算是仁爱。知道这些道理却不去争辩,不能算是忠。争辩却没有成功,不能算是强。你奉行义,不愿杀一个人,却去杀害宋国众多的百姓,不能说懂得类推事理。”公输盘被说服了。墨子又问他:“那么,为什么不取消进攻宋国这件事呢?”公输盘说:“不能,我已经对楚王说了。”墨子说:“为什么不把我介绍给楚王呢?”公输盘说:“好吧。”
    
    墨子见了楚王,说:“现在这里有一个人,舍弃自己华贵的彩车,邻居有辆破车,却想去偷;舍弃自己的锦绣衣裳,邻居有粗布衣服,却想去偷;舍弃自己的美食佳肴,邻居只有糟糠,却想去偷。这是个怎样的人呢?”楚王回答说:“这人一定患了偷窃病。”墨子说:“楚国的地方,方圆五千里,宋国的地方,方圆五百里,这就像彩车与破车相比。楚国有云梦大泽,犀、兕、麋鹿满地都有,长江、汉水中有鱼鳖、鼋鼍,富甲天下,宋国却连野鸡、兔子、鲫鱼都没有,这就像美食佳肴与糟糠相比。楚国有巨松、梓树、楠、樟等名贵木材,宋国连棵大树都没有,这就像华丽的丝织品与粗布短衣相比。我认为大王派人进攻宋国,与这三件事是同类的。”楚王说:“说得好啊!即便如此,公输盘已经给我造了云梯,(我)一定要攻取宋国。”
    
    于是(楚王)召见公输盘。墨子解下腰带,围作城墙的样子,用小木片作为守备的器械。公输盘多次运用机巧多变的方法攻城,墨子一次次抵拒了他的进攻。公输盘攻城用的器械用尽了,墨子守城的方法还有余。公输盘败了,但是却说:“我知道用什么办法对付你了,但我不说。”墨子说:“我知道你怎么对付我,我也不说。”楚王问是怎么回事?墨子回答说:“公输盘的意思,不过是杀了我。杀了我,宋国就没有人能防守了,楚国就可以进攻宋国了。但是,我的弟子禽滑釐等三百人,已经拿着我守御用的器械,正在宋国的都城上等待楚国侵犯呢。即使杀了我,(但守御的人)却是杀不尽的。”楚王说:“说得好啊!我想还是不要攻打宋国了。”
    
    墨子从楚国归来,经过宋国,遇到大雨,想到里巷门下避雨,看守巷门的人却不让他进去。所以说:“暗中化解了大危机的人,大家不知道他的功劳;而在明处争辩的,大家都知道。”
    
\end{normalsize}


\newpage

\textbf{注解}:

\vspace{-1em}

\begin{itemize}
    \setlength\itemsep{-0.2em}
    \item〔臣以王吏之攻宋也,为与此同类〕旧本作:“臣以三事之攻宋也,为与此同类,臣见大王之必伤义而不得。”毕云:“《战国策》云‘臣以王吏之攻宋。’‘王吏’葢‘三𠭏’之误,《说文》云‘𠭏,古文事。’《尸子》作‘王使’,《太平御览》作‘王之攻宋’。”顾云:“《国策》‘王吏’与此文‘三事’,皆有误。疑当云‘臣以王之事攻宋也’。”诒让案:“三事”,疑当作“三吏”。《逸周书》大匡篇云“王乃召冢卿三老三吏”,孔晁注云“三吏,三卿也。”《左传》成公二年,“晋侯使巩朔献齐捷于周,王使委于三吏”,杜注云“三吏,三公也。”《神仙传》作“臣闻大王,更议攻宋”,则似是“王吏”之譌。猜测可能是“臣以为王之攻宋也与此(三事)同类”之讹误。
    \item〔治于神者,众人不知其功〕《孟子》:“聖而不可知之謂神”。“天神,引出萬物者也”,“阴阳不测之谓神”,“神者,變化之極,妙萬物而爲言,不可以形詰”,“神而明之,存乎其人”。“神”原本指引起天地万物生灭变化的看不到的玄妙道理,藏于万物背后暗中主导者,是古人对不理解不明白的自然规律的设想和称呼,因此和“明”相对。这里引申为表示不为人知但最重要关键的部分。《尸子·贵言》:“圣人治于神,愚人争于明也。天地之道,莫见其所以长物而物长,莫见其所以亡物而物亡。圣人之道亦然。其兴福也,人莫之见而福兴矣。其除祸也,人莫之知而祸除矣,故曰神人”。治:整理,消灭灾祸、混乱,使安定、太平。暗中治理了危机的人,大众不知道他的功劳。
\end{itemize}

\chapter{孟子见梁惠王}

\begin{normalsize}
    
    孟子见梁惠王\footnote{〔梁惠王〕即魏惠王(前369年至前319年在位),名罃(或作“婴”),魏武侯之子。}。王曰:“叟不远千里而来,亦将有以利吾国乎?” 孟子对曰:“王何必曰利?亦有仁义而已矣。王曰:‘何以利吾国?’大夫曰:‘何以利吾家?’士庶人曰:‘何以利吾身?’上下交征利而国危矣。万乘之国,弑其君者,必千乘之家;千乘之国,弑其君者,必百乘之家。万取千焉,千取百焉,不为不多矣。苟为后义而先利,不夺不餍。未有仁而遗其亲者也,未有义而后其君者也。王亦曰仁义而已矣,何必曰利?”
    
    孟子见梁惠王。王立于沼上,顾鸿雁麋鹿,曰:“贤者亦乐此乎?” 孟子对曰:“贤者而后乐此,不贤者虽有此,不乐也。《诗》云\footnote{〔《诗》云……〕出自《诗经·大雅·灵台》。}:经始灵台,经之营之,庶民攻之,不日成之;经始勿亟,庶民子来;王在灵囿,麀鹿攸伏,麀鹿濯濯,白鸟鹤鹤;王在灵沼,於牣鱼跃。文王\footnote{〔文王〕周文王,名昌,又称西伯昌。商朝末期诸侯国周国君主。其子武王伐纣,建立周朝。}以民力为台为沼,而民欢乐之,谓其台曰灵台,谓其沼曰灵沼,乐其有麋鹿鱼鳖。古之人与民偕乐,故能乐也。《汤誓》曰\footnote{〔《汤誓》〕《尚书》中的一篇,为商汤伐桀誓师词。}:时日害丧,予及女偕亡。民欲与之偕亡,虽有台池鸟兽,岂能独乐哉?”
    
    梁惠王曰:“寡人之于国也,尽心焉耳矣。河内凶,则移其民于河东,移其粟于河内;河东凶亦然。察邻国之政,无如寡人之用心者。邻国之民不加少,寡人之民不加多,何也?”孟子对曰:“王好战,请以战喻。填然鼓之,兵刃既接,弃甲曳兵而走。或百步而后止,或五十步而后止。以五十步笑百步,则何如?”曰:“不可,直不百步耳,是亦走也。”曰:“王如知此,则无望民之多于邻国也。”“不违农时,谷不可胜食也;数罟不入洿池,鱼鳖不可胜食也;斧斤以时入山林,材木不可胜用也。谷与鱼鳖不可胜食,材木不可胜用,是使民养生丧死无憾也。养生丧死无憾,王道之始也。”“五亩之宅,树之以桑,五十者可以衣帛矣;鸡豚狗彘之畜,无失其时,七十者可以食肉矣;百亩之田,勿夺其时,数口之家可以无饥矣;谨庠序之教,申之以孝悌之义,颁白者不负戴于道路矣。七十者衣帛食肉,黎民不饥不寒,然而不王者,未之有也。”“狗彘食人食而不知检,涂有饿殍而不知发,人死,则曰:‘非我也,岁也’。是何异于刺人而杀之,曰:‘非我也,兵也’?王无罪岁,斯天下之民至焉。”
    
    梁惠王曰:“寡人愿安承教。”孟子对曰:“杀人以梃与刃,有以异乎?”曰:“无以异也。”“以刃与政,有以异乎?”曰:“无以异也。曰:“庖有肥肉,厩有肥马,民有饥色,野有饿莩,此率兽而食人也!兽相食,且人恶之;为民父母,行政,不免于率兽而食人,恶在其为民父母也?仲尼\footnote{〔仲尼〕孔子字仲尼。}曰:‘始作俑者,其无后乎!’为其象人而用之也。如之何其使斯民饥而死也?”
    
    梁惠王曰:“晋国,天下莫强焉,叟之所知也。及寡人之身,东败于齐\footnote{〔晋国〕指魏国;魏韩赵三国瓜分晋国,魏国最为强大,所以用“晋国”指代魏国。},长子死焉;西丧地于秦七百\footnote{〔东败于齐〕指公元前343年的马陵之役。魏国伐韩国,韩国求救于齐国。齐军袭魏,魏军败于马陵,主将庞涓自杀,魏太子申被俘。};南辱于楚\footnote{〔南辱于楚〕公元前324年,楚攻魏,破之于襄陵,得八邑。}。寡人耻之,愿比死者壹洒之,如之何则可?” 孟子对曰:“地方百里而可以王。王如施仁政于民,省刑罚,薄税敛,深耕易耨;壮者以暇日修其孝悌忠信,入以事其父兄,出以事其长上,可使制梃以挞秦楚之坚甲利兵矣。彼夺其民时,使不得耕耨以养其父母。父母冻饿,兄弟妻子离散。彼陷溺其民,王往而征之,夫谁与王敌?故曰:仁者无敌。王请勿疑!”
    
    孟子曰:“不仁哉,梁惠王也!仁者以其所爱及其所不爱,不仁者以其所不爱及其所爱。”公孙丑\footnote{〔公孙丑〕孟子弟子,齐国人,是《孟子》的主要作者之一。}问曰:“何谓也?”“梁惠王以土地之故,糜烂其民而战之,大败,将复之,恐不能胜,故驱其所爱子弟以殉之,是之谓以其所不爱及其所爱也。”
\end{normalsize}


\newpage

\textbf{译文}:

\vspace{1em}

\begin{normalsize}
    
    孟子拜见梁惠王。惠王说:“老先生不辞千里长途的辛劳而来,是不是将给我国带来利益呢?”孟子答道:“王何必非要说利呢?(我带来的)只有仁义而已。如果王只是说‘怎样才有利于我的国家呢?’,大夫也说‘怎样才有利于我的封地呢?’,那一般士子和庶民也会说‘怎样才有利于我自己呢?’。这样,上上下下都互相追讨利益,国家便危险了。在拥有一万辆兵车的国家里,杀掉它的国君的,一定是拥有一千辆兵车的贵族;在拥有一千辆兵车的国家里,杀掉它的国君的,一定是拥有一百辆兵车的贵族。在一万辆兵车的国家里,拥有一千辆兵车,在一千辆的国家里,拥有一百辆兵车,(这些贵族拥有的)不能不说是很多了。假若他先考虑‘利’,后考虑‘义’,那他不夺取(整个国家),是不会满足的。仁人没有遗弃父母的,义人没有怠慢君主的。
    
    孟子拜见梁惠王。王站在池塘边,一边欣赏着鸟兽,一边说:“有德行的人也享受这种快乐吗?”孟子答道:“只有有贤明的人才能体会到这种快乐,不贤明的人即使有这一切,也没法享受。《诗经·大雅·灵台》中说:开始筑灵台,勘测又标明;百姓都来做,慢慢就完成;王说才开始,不要太着急;百姓如儿子,都来出把力;王到鹿苑中,母鹿正栖息;母鹿肥又亮,白鸟毛如雪;王到灵沼上,满池鱼跳跃。周文王虽然用了百姓的力量筑高台挖深池,但百姓乐意这样做,他们管这台叫‘灵台’,管这池叫‘灵沼’,还乐意那里有许多麋鹿和鱼鳖。古时候的圣君贤王因为能与老百姓同乐,才能得到真正的快乐。《汤誓》中记载:太阳啊,你什么时候灭亡呢?我宁肯和你一道去死。老百姓恨不得与他同归于尽,即使有高台深池,珍禽异兽,他又如何能独自享受呢?”
    
    梁惠王说:“我对于国家,总算尽了心啦。河内遇到饥荒,就把那里的老百姓迁移到河东去,把河东的粮食转移到河内;河东遇到饥荒也是这样做。了解一下邻国的政治,没有像我这样用心的。邻国的百姓不见减少,我的百姓不见增多,这是为什么呢?”孟子答道:“大王喜欢打仗,让我用战争做比喻吧。咚咚地敲响战鼓,两军开始交战,战败的扔掉盔甲拖着武器逃跑。有人逃了一百步然后停下来,有的人逃了五十步然后停下来。凭自己只跑了五十步而耻笑别人跑了一百步,那怎么样呢?”梁惠王说:“不行。只不过没有跑上一百步罢了,那也是逃跑啊。”孟子说:“大王如果懂得这个道理,就不要指望自己的百姓比邻国多了。“不耽误农业生产的季节,粮食就会吃不完。密网不下到池塘里,鱼鳖之类的水产就会吃不完。按一定的季节入山伐木,木材就会用不完。粮食和水产吃不完,木材用不完,这就使百姓对生养死葬没有什么不满了。百姓对生养死葬没有什么不满,这是王道的开端。“五亩大的住宅场地,种上桑树,五十岁的人就可以穿丝织品了。鸡、猪、狗的畜养,不要耽误它们的繁殖时机,七十岁的人就可以吃肉食了。
    
    梁惠王说:“我愿意耐心地接受您的教诲。”孟子答道:“杀人用棍子和用刀子,有什么不同吗?”王说:“没有什么不同。”“用刀子和用政治(杀人),有什么不同吗?”王说:“没有什么不同。”孟子又说:“厨房里有厚实的肉,马厩里有健壮的马,老百姓却面有菜色,郊野外也饿殍横陈,这就是率领着禽兽来吃人!野兽间弱肉强食,人尚且厌恶;作为老百姓的父母官来从政,还不能做到不率领着禽兽来吃人,这又怎么算是老百姓的父母官呢?孔子曾说:‘最开始用俑来陪葬的人,该会断子绝孙吧’!这是因为俑如同活人,却用来陪葬。又怎能让老百姓活活饿死呢?”
    
    梁惠王说:“魏国的强大,天下没有比得上的,老先生是知道的。但到了我这一代,先是东边败给齐国,长子都死在那儿;西边割让了七百里土地给秦国;南边又被楚国所羞辱。我为此深感屈辱,希望为死难者报仇雪恨,要怎样办才好呢?”孟子答道:“百里见方的小国,也可以称王。您如果向百姓施行仁政,减免刑罚,减轻赋税,让他们能够深翻土,勤除草;青壮年在闲暇时能讲求孝顺父母、敬爱兄长、为人忠心、诚实守信的德行,并用来在家里侍奉父兄,在朝廷服事上级,这样,就是举着木棒也足以抗击披坚执锐的秦楚大军了。那秦国楚国,侵夺了老百姓的生产时间,使他们不能耕种来养活父母。父母因此受冻挨饿,兄弟妻儿东逃西散。那秦王楚王让他们的百姓陷入痛苦,您去讨伐他们,那还有谁来与您为敌呢?所以说:仁人没有敌人。请您不要怀疑!”
    
    孟子说:“梁惠王太不仁了!仁人把他给他爱的人(的恩德)推及到他不爱的人,不仁的人却把他给他不爱的人(的祸害)推及到他爱的人。”公孙丑问道:“这是什么意思呢?”(孟子答道):“梁惠王为了争夺土地的缘故,驱使他的百姓去作战,暴尸荒野,骨肉糜烂。
    
\end{normalsize}



\chapter{鱼我所欲也}

\begin{normalsize}
    
    鱼,我所欲也;熊掌,亦我所欲也。二者不可得兼,舍鱼而取熊掌者也。生,亦我所欲也;义,亦我所欲也。二者不可得兼,舍生而取义者也。生亦我所欲,所欲有甚于生者,故不为苟得也;死亦我所恶,所恶有甚于死者,故患有所不辟也。如使人之所欲莫甚于生,则凡可以得生者何不用也?使人之所恶莫甚于死者,则凡可以避患者何不为也?由是则生而有不用也,由是则可以避患而有不为也。是故所欲有甚于生者,所恶有甚于死者。非独贤者有是心也,人皆有之,贤者能勿丧耳。
    
    一箪食,一豆\footnote{〔豆〕古代一种木制的盛食物的器具。}羹,得之则生,弗得则死。呼尔而与之,行道之人弗受;蹴尔而与之,乞人不屑也。万钟\footnote{〔万钟〕指高官厚禄。钟,古代的一种量器,六斛四斗为一钟。}则不辩礼义而受之,万钟于我何加焉!为宫室之美,妻妾之奉,所识穷乏者得我与?乡为身死而不受,今为宫室之美为之;乡为身死而不受,今为妻妾之奉为之;乡为身死而不受,今为所识穷乏者得我而为之。是亦不可以已乎?此之谓失其本心。
\end{normalsize}


\newpage

\textbf{译文}:

\vspace{1em}

\begin{normalsize}
    
    鱼,是我所想要的;熊掌,也是我所想要的。如果这两种东西不能同时都得到的话,我就只好放弃鱼而选取熊掌了。生命,也是我所想要的;道义,也是我所想要的。如果这两种东西不能同时都得到的话,我就只好舍弃生命而选取道义了。生命是我所喜爱的,但我所喜爱的还有胜过生命的东西,所以我不做苟且偷生的事;死亡是我所厌恶的,厌恶的有比死亡更厉害的东西,所以有的灾祸我不躲避。如果人们所喜爱的东西没有超过生命的,那么凡是能够用来求得生存的手段,有什么不可以使用呢?如果人们所厌恶的事情没有超过死亡的,那么凡是能够用来逃避灾祸的坏事,有什么不可以做的呢?采用某种手段就能够活命,可是有的人却不肯采用;采用某种办法就能够躲避灾祸,可是有的人也不肯采用。由此可见,他们所喜爱的有比生命更宝贵的东西;厌恶的有比死亡更厉害的东西。不仅贤人有这种本心,人人都有,(只不过)贤明的人可以不丧失(这种本心)罢了。
    
    一碗饭,一碗汤,吃了就能活下去,不得到它就会饿死。呼喝着给别人,行人也不愿接受;用脚踢给别人,乞丐也不屑于接受。高官厚禄却不辨是否合乎礼义就接受了它,这样,高官厚禄对我有什么好处呢?是为了住宅的华丽,妻妾的侍奉和熟识的穷人感激我吗?从前(有人)为了(道义)(宁愿)死也不愿接受(别人的施舍),却为了住宅的华丽却接受了它;从前(有人)为了(道义)(宁愿)死也不愿接受(别人的施舍),现在(有人)却为了妻妾的侍奉却接受了它;从前(有人)为了(道义)(宁愿)死也不愿接受(别人的施舍),如今(有人)却为了让所认识穷困贫乏的人感激他们的恩德而接受了它。这种(行为)难道不可以停止吗?这就叫做丧失了自己的本心。
    
\end{normalsize}


\newpage

\textbf{注解}:

\vspace{-1em}

\begin{itemize}
    \setlength\itemsep{-0.2em}
    \item〔呼尔而与之,行道之人弗受〕《礼记·檀弓》记载,有一年齐国出现了严重的饥荒。黔敖在路边施粥,有个饥饿的人用衣袖蒙着脸走来。黔敖吆喝着让他吃粥。他说:“我正因为不吃被轻蔑所给予得来的食物,才落得这个地步!”
\end{itemize}

\chapter{《庄子》故事两则}

\begin{normalsize}
    
    惠子相梁
    
    惠子\footnote{〔惠子〕指惠施,战国时宋国思想家,名家的代表人物。}相梁\footnote{〔梁〕指魏国。魏国曾将都城迁到大梁,因此也叫梁国。},庄子\footnote{〔庄子〕指庄周,战国时宋国思想家,道家的代表人物。}往见之。或谓惠子曰:“庄子来,欲代子相。”于是惠子恐,搜于国中三日三夜。庄子往见之,曰:“南方有鸟,其名为鹓雏,子知之乎?夫鹓雏发于南海,而飞于北海;非梧桐不止,非练实不食,非醴泉不饮。于是鸱得腐鼠,鹓雏过之,仰而视之曰:‘吓~’。今子欲以子之梁国而吓我邪?”
    
    庄子与惠子游于濠梁之上
    
    庄子与惠子游于濠梁\footnote{〔濠梁〕濠水上的桥梁,位于今安徽凤阳临淮镇,是濠河入淮处。}之上。庄子曰:“鯈鱼出游从容,是鱼之乐也。”惠子曰:“子非鱼,安知鱼之乐?”庄子曰:“子非我,安知我不知鱼之乐?”惠子曰:“我非子,固不知子矣;子固非鱼也,子之不知鱼之乐,全矣!”庄子曰:“请循其本。子曰‘汝安知鱼乐’云者,既已知吾知之而问我。我知之濠上也。”
\end{normalsize}


\newpage

\textbf{译文}:

\vspace{1em}

\begin{normalsize}
    
    惠子相梁
    
    惠施在梁国做国相,庄子去看望他。有人告诉惠施说:“庄子到梁国来,是想取代你做宰相。”于是惠施非常害怕,在国都搜捕了三天三夜。庄子前去见他,说:“南方有一种鸟,它的名字叫鹓鶵,你知道它吗?那鹓鶵是从南海起飞,要飞到北海去;不是梧桐树就不栖息,不是竹子所结的子就不吃,不是甘甜的泉水就不喝。在此时鹞鹰拾到一只腐臭的老鼠,鹓鶵从它面前飞过,鹞鹰看到仰头发出‘喝~’的怒斥声。难道现在你也想用你的梁国来威吓我吗?”
    
    庄子与惠子游于濠梁之上
    
    庄子与惠施在濠水的桥上游玩。庄子说:“鲦鱼在河水中游得多么悠闲自得,这是鱼的快乐啊。”惠施说:“你不是鱼,怎么知道鱼的快乐?”庄子说:“你不是我,怎么知道我不知道鱼的快乐?”惠施说:“我不是你,固然不知道你;你本来就不是鱼,你不知道鱼的快乐,是完全可以肯定的!”庄子说:“请追溯话题本源。你说‘你哪儿知道鱼快乐’的话,说明你已经知道我知道鱼快乐而在问我。我是在濠水的桥上知道的啊。”
    
\end{normalsize}


\newpage

\textbf{注解}:

\vspace{-1em}

\begin{itemize}
    \setlength\itemsep{-0.2em}
    \item〔子曰‘汝安知鱼乐’云者,既已知吾知之而问我〕这里是典型的语法诡辩。庄子首先说“鯈鱼出游从容,是鱼之乐也。”惠子说“子安知鱼乐”,是建立在庄子说“鱼之乐”的基础上。但庄子反过来利用语句的歧义,偷换了概念。惠子想说的是“你怎么会有‘自己知道鱼之乐’的想法?”,也就是想说“其实你不知道鱼之乐吧”。但“子安知鱼乐”也可以解释为“你是怎么知道‘鱼之乐’这件事的”,即询问“你知道这件事的方法”,这就等于先承认了“你知道鱼之乐”,只是询问“知道的方法”。这种利用语句的歧义偷换概念的诡辩,恰恰是名家思想关注的地方,即名实的关系。庄子在用惠子的诡辩方法回击惠子。
\end{itemize}

\chapter{邹忌讽齐王纳谏}

\begin{normalsize}
    
    邹忌\footnote{〔邹忌〕战国时齐人,有辩才。齐桓公时就任大臣,威王时为齐相。}修八尺\footnote{〔八尺〕战国时各国尺度不一,从出土文物推算,一尺约相当于今18到23厘米不一。}有余,而形貌昳丽。朝服衣冠,窥镜,谓其妻曰:“我孰与城北徐公美?”其妻曰:“君美甚,徐公何能及君也?”城北徐公,齐国之美丽者也。忌不自信,而复问其妾曰:“吾孰与徐公美?”妾曰:“徐公何能及君也?”旦日,客从外来,与坐谈,问之客曰:“吾与徐公孰美?”客曰:“徐公不若君之美也。”明日徐公来,孰视之,自以为不如;窥镜而自视,又弗如远甚。暮寝而思之,曰:“吾妻之美我者,私我也;妾之美我者,畏我也;客之美我者,欲有求于我也。”
    
    于是入朝见威王\footnote{〔威王〕齐威王,田齐桓公之子,任内贤明能治,使齐国成为战国七雄之一。},曰:“臣诚知不如徐公美。臣之妻私臣,臣之妾畏臣,臣之客欲有求于臣,皆以美于徐公。今齐地方千里,百二十城,宫妇左右莫不私王,朝廷之臣莫不畏王,四境之内莫不有求于王:由此观之,王之蔽甚矣。”
    
    王曰:“善。”乃下令:“群臣吏民,能面刺寡人之过者,受上赏;上书谏寡人者,受中赏;能谤讥于市朝,闻寡人之耳者,受下赏。”令初下,群臣进谏,门庭若市;数月之后,时时而间进;期年之后,虽欲言,无可进者。
    
    燕、赵、韩、魏闻之,皆朝于齐。此所谓战胜于朝廷。
\end{normalsize}


\newpage

\textbf{译文}:

\vspace{1em}

\begin{normalsize}
    
    邹忌身高八尺多,而且身材魁梧,容貌美丽。(有一天)早晨(他)穿好衣服戴好 帽子,照镜子,对他的妻子说:“我与城北徐公相比,哪一个美?”他的妻子说:“您美极了,徐公哪里能比得上您呢?” 城北的徐公,是齐国的美男子。邹忌不相信自己(会比徐公美),于是又问他的妾说:“我与徐公相比谁更美?”妾说:“徐公哪里能比得上您呢?” 第二天,一位客人从外面来(拜访),邹忌与他坐着闲谈,问他:“我和徐公谁更美?”客人说:“徐公不如您美啊。” 第二天,徐公来了,邹忌仔细地端详他,自己认为不如(徐公美);再照镜子看看自己,又觉得远不如人家。晚上,(他)躺在床上想这件事,说:“我的妻子认为我美的原因,是偏爱我;妾认为我美的原因,是惧怕我;客人认为我美的原因,是有事情想要求于我。”
    
    因此邹忌上朝拜见齐威王,说:“我确实知道(自己)不如徐公美。(可是)我的妻子偏爱我,我的妾惧怕我,我的客人对我有所求,所以(他们)都认为我比徐公美。如今的齐国,土地方圆千里,有一百二十座城池,宫中的妃子及身边的侍从,没有不偏爱大王的,朝中的大臣,没有人不惧怕您的,国内的百姓,没有不对大王有所求的。
    
    齐威王说:“ 好!”于是就下了一道命令:“所有大臣、官吏、百姓,能够当面批评我过错的人,得上等奖赏;能够上书劝谏我的人,得中等奖赏;能够在公共场所指责议论(我的过失),(并能)传到我的耳朵里的人,得下等奖赏。” 政令刚一下达,许多官员都来进言规劝,宫庭就像集市一样(喧闹);几个月以后,有时偶尔还有人来进谏;满一年以后,即使想说,也没有什么可进谏的了。
    
    燕、赵、韩、魏等国听说了这件事,都到齐国来朝见(齐王)。这就是内政修明,不需用兵就能战胜敌国。
    
\end{normalsize}



\chapter{前赤壁赋}

\begin{normalsize}
    
    壬戌\footnote{〔壬戌〕宋神宗元丰五年(公元1082年)。}之秋,七月既望,苏子与客泛舟游于赤壁之下。清风徐来,水波不兴。举酒属客,诵明月之诗,歌窈窕之章。少焉,月出于东山之上,徘徊于斗牛\footnote{〔斗牛〕牛宿和斗宿。}之间。白露横江,水光接天。纵一苇之所如,凌万顷之茫然。浩浩乎如冯虚御风,而不知其所止;飘飘乎如遗世独立,羽化而登仙。
    
    于是饮酒乐甚,扣舷而歌之。歌曰:“桂棹兮兰桨,击空明兮溯流光。渺渺兮予怀,望美人兮天一方。”客有吹洞箫者,倚歌而和之。其声呜呜然,如怨如慕,如泣如诉;余音袅袅,不绝如缕。舞幽壑之潜蛟,泣孤舟之嫠妇。
    
    苏子愀然,正襟危坐,而问客曰:“何为其然也?”客曰:“‘月明星稀,乌鹊南飞’,此非曹孟德\footnote{〔曹孟德〕指曹操,三国时枭雄,建立曹魏政权。}之困于周郎\footnote{〔周郎〕指周瑜,三国时东吴名将,在赤壁之战中大破曹魏水军。}之诗乎?西望夏口,东望武昌,山川相缪,郁乎苍苍,此非孟德者乎?方其破荆州,下江陵,顺流而东也,舳舻千里,旌旗蔽空,酾酒临江,横槊赋诗,固一世之雄也,而今安在哉?况吾与子渔樵于江渚之上,侣鱼虾而友麋鹿,驾一叶之扁舟,举匏尊以相属。寄蜉蝣\footnote{〔蜉蝣〕一种只能活一天的虫子。}于天地,渺沧海之一粟。哀吾生之须臾,羡长江之无穷。挟飞仙以遨游,抱明月而长终。知不可乎骤得,托遗响于悲风。”
    
    苏子曰:“客亦知夫水与月乎?逝者如斯,而未尝往也;盈虚者\footnote{〔盈虚者〕指月亮。}如彼,而卒莫消长也。盖将自其变者而观之,则天地曾不能以一瞬;自其不变者而观之,则物与我皆无尽也,而又何羡乎!且夫天地之间,物各有主,苟非吾之所有,虽一毫而莫取。惟江上之清风,与山间之明月,耳得之而为声,目遇之而成色,取之无禁,用之不竭。是造物者之无尽藏也,而吾与子之所共食。”
    
    客喜而笑,洗盏更酌。肴核既尽,杯盘狼籍。相与枕藉乎舟中,不知东方之既白。
\end{normalsize}


\newpage

\textbf{译文}:

\vspace{1em}

\begin{normalsize}
    
    壬戌年的秋天,七月十六日,我和客人泛舟于赤壁之下。清风徐徐地吹来,水面上没有波浪。举起酒杯,邀客人同饮,吟诵起《明月》诗篇的“窈窕”一章。一会儿,月亮从东山上升起,徘徊在斗宿、牛宿之间。白濛濛的雾气笼罩着江面,波光闪动的水面遥接着天边。我们任凭小舟自由漂流,游走在浩淼无垠的江面上。江水浩瀚啊,船儿像凌空驾风而行,而不知道将停留在什么地方;人儿飘飘啊,像独自站在了尘世之外,要生出翅膀飞升成仙。
    
    这时候,喝着酒,心中更加的快乐,便敲着船舷唱起歌来。歌词说:“桂木做的棹啊兰木做的桨,拍击着清澈明亮的江水啊,在月光浮动的江面上逆水行走。我的情思悠远深沉啊,心中思念的美人,却在遥远的地方。”客人中有会吹洞箫的,随着歌声吹奏起来,那箫声呜咽,像在埋怨,像在思慕,像在抽泣,像在倾诉。一曲奏完,余音悠长,像轻丝一样不能断绝。深渊里潜藏的蛟龙为之起舞,孤舟中的悲凉的寡妇为之哭泣。
    
    我不禁黯然神伤,于是整理好衣襟,端坐起来,问客人说:“为什么奏出这样悲凉的音乐呢?”客人回答说:“‘月光明亮,星儿稀少,乌鹊向南飞去’,这不是曹孟德的诗句吗?从这里向西望去是夏口,向东望去是武昌,山水相缠绵,景色郁郁苍苍,这不就是曹操被周瑜打败的地方吗?当他夺取荆州,攻下江陵,顺江东下的时候,战船连接千里,旌旗遮蔽天空;他把酒临江,横握长矛赋诗,那真是一世的豪杰啊,可如今却在哪里呢?何况我和你在江中的小洲上捕鱼砍柴,以鱼虾为伴,以麋鹿为友,驾着一叶小舟,举着酒杯互相劝酒。将如同蜉蝣一样短暂的生命寄托于天地之间,渺小得像大海里的一粒米。悲叹我们生命的短暂,羡慕长江的不尽东流。愿与神仙相伴而遨游,也想同明月相守而长存。知道这样的愿望是不能突然实现的,于是只能借着箫声将这无穷的遗恨寄托在悲凉的风中。”
    
    我对客人说:“你也知道那水和月的道理吗?江水是这样不停地流走,可它依然存在啊;月亮时而圆时而缺,但它始终是那个月亮,并没有消损和增长。如果从变化的角度去看,那么天地间的万事万物,没有一刻能够保持不变;如果从不变的角度去看,那么事物和我们本身都不会有穷尽的时候,又有什么可羡慕的呢?再说那天地之间的万事万物都有自己的主宰,如果不是我们的东西,即使是一丝一毫也不能得到。只有江上的清风与山间的明月,耳朵听到了,就成为了声音,眼睛看到了,就成为了色彩,得到它们没有人禁止,享用它们没有竭尽的时候。这是大自然无穷无尽的宝藏啊,是我和你可以共同享受的东西。”
    
    客人们听了这番话都高兴地笑了起来,于是洗净了酒杯,重斟再饮。菜肴和水果都已经吃完,酒杯和盘子杂乱地放着。我与客人们相互枕着靠着在船里睡着了,不知不觉中东方已然发白。
    
\end{normalsize}



\chapter{归去来兮辞}

\begin{normalsize}
    
    归去来兮,田园将芜胡不归?既自以心为形役,奚惆怅而独悲?悟已往之不谏,知来者之可追。实迷途其未远,觉今是而昨非。舟遥遥以轻飏,风飘飘而吹衣。问征夫以前路,恨晨光之熹微。
    
    乃瞻衡宇,载欣载奔。僮仆欢迎,稚子候门。三径\footnote{〔三径〕汉朝蒋诩隐居之后,在院里竹下开辟三径,只于少数友人来往。后来,三径变成了隐士住处的代称。}就荒,松菊犹存。携幼入室,有酒盈樽。引壶觞以自酌,眄庭柯以怡颜。倚南窗以寄傲,审容膝之易安。园日涉以成趣,门虽设而常关。策扶老以流憩,时矫首而遐观。云无心以出岫,鸟倦飞而知还。景翳翳以将入,抚孤松而盘桓。
    
    归去来兮,请息交以绝游。世与我而相违,复驾言兮焉求?悦亲戚之情话,乐琴书以消忧。农人告余以春及,将有事于西畴。或命巾车\footnote{〔巾车〕有帷帐的马车。},或棹孤舟。既窈窕以寻壑,亦崎岖而经丘。木欣欣以向荣,泉涓涓而始流。善万物之得时,感吾生之行休。
    
    已矣乎!寓形宇内复几时?曷不委心任去留?胡为乎遑遑欲何之?富贵非吾愿,帝乡不可期。怀良辰以孤往,或植杖而耘耔。登东皋以舒啸,临清流而赋诗。聊乘化以归尽,乐夫天命复奚疑!
\end{normalsize}


\newpage

\textbf{译文}:

\vspace{1em}

\begin{normalsize}
    
    回家去吧!田园快要荒芜了,为什么不回去呢?既然自己使心为身所驱役,又何必怅惘而独自悲戚呢?我已明悟过去的错误已经不可挽回,未来的事还来得及补救。我确实走入了迷途,但还不算太远,已觉悟如今的做法是对的,而曾经的行为是错的。船在水面上轻轻地飘荡着前进,微风吹拂着衣裳,衣袂翩翩。
    
    刚刚看到自己简陋的家门,心中欣喜,奔跑过去。家中童仆欢喜地前来迎接,孩子们守候在门前。院子里的小路快要荒芜了,松菊还长在那里。带着孩子们进了屋,美酒已经盛满了酒樽。我端起酒壶酒杯自斟自饮,看看院子里的树木,觉得很愉快;倚着南窗寄托我的傲世之情,深知这狭小之地容易使我心安。天天到园里行走,自成一种乐趣,小园的门经常地关闭着;拄着拐杖出去走走,随时随地休息,时时抬头望着远方(的天空)。云气自然而然地从山峰飘浮而出,倦飞的鸟儿也知道飞回巢中;日光暗淡,太阳快要落下去了,我手抚着孤松徘徊着不忍离去。
    
    回家去吧!请让我同外界断绝交游。世事与我所想的相违背,还要驾车出去追求什么?以亲人间的知心话为愉悦,以弹琴读书为乐来消除忧愁;农夫告诉我春天到了,西边田野里要开始耕种了。有时驾着有布篷的小车,有时划着一条小船,有时经过幽深曲折的山谷,有时走过高低不平的山路。草木茂盛,细水缓流,(我)羡慕自然界的万物一到春天便及时生长茂盛,感叹自己的一生行将结束。
    
    算了吧!活在世上还能有多久?为什么不随心所欲,听凭自然的生死?为什么心神不定,想要到哪里去?富贵不是我所求,修成神仙是没有希望的。爱惜那良辰美景我独自去欣赏,有时放下手杖,拿起农具除草培土;登上东边的山坡我放声呼啸,傍着清清的溪流把诗歌吟唱。姑且顺随自然的变化,走到生命的尽头。乐天安命,还有什么可疑虑的呢?
    
\end{normalsize}



\chapter{兼爱}

\begin{normalsize}
    
    圣人以治天下为事者也,必知乱之所自起,焉能治之;不知乱之所自起,则不能治。譬之如医之攻人之疾者然,必知疾之所自起,焉能攻之;不知疾之所自起,则弗能攻。治乱者何独不然?必知乱之所自起,焉能治之;不知乱之所自起,则弗能治。
    
    圣人以治天下为事者也,不可不察乱之所自起。当察乱何自起?起不相爱。臣子之不孝君父,所谓乱也。子自爱,不爱父,故亏父而自利;弟自爱,不爱兄,故亏兄而自利;臣自爱,不爱君,故亏君而自利。此所谓乱也。虽父之不慈子,兄之不慈弟,君之不慈臣,此亦天下之所谓乱也。父自爱也,不爱子,故亏子而自利;兄自爱也,不爱弟,故亏弟而自利;君自爱也,不爱臣,故亏臣而自利。是何也?皆起不相爱。虽至天下之为盗贼者,亦然。盗爱其室,不爱异室,故窃异室以利其室;贼爱其身,不爱人,故贼人以利其身。此何也?皆起不相爱。虽至大夫之相乱家、诸侯之相攻国者,亦然。大夫各爱其家,不爱异家,故乱异家以利其家;诸侯各爱其国,不爱异国,故攻异国以利其国。天下之乱物,具此而已矣。
    
    察此何自起?皆起不相爱。若使天下兼相爱,爱人若爱其身,犹有不孝者乎?视父兄与君若其身,恶施不孝?犹有不慈者乎?视弟子与臣若其身,恶施不慈?故不孝不慈亡有。犹有盗贼乎?视人之室若其室,谁窃?视人身若其身,谁贼?故盗贼有亡。犹有大夫之相乱家、诸侯之相攻国者乎?视人家若其家,谁乱?视人国若其国,谁攻?故大夫之相乱家、诸侯之相攻国者亡有。若使天下兼相爱,国与国不相攻,家与家不相乱,盗贼无有,君臣父子皆能孝慈,若此则天下治。
    
    故圣人以治天下为事者,恶得不禁恶而劝爱?故天下兼相爱则治,交相恶则乱。故子墨子曰不可以不劝爱人者,此也。
\end{normalsize}


\newpage

\textbf{译文}:

\vspace{1em}

\begin{normalsize}
    
    圣人是把治理天下作为(自己的)事务的人,必须知道混乱从哪里产生,才能对它进行治理。如果不知道混乱从哪里产生,就不能进行治理。这就好像大夫给人治病一样,必须知道疾病产生的根源,才能进行医治。如果不知道疾病产生的根源,就不能医治。
    
    圣人是把治理天下作为(自己的)事务的人,不可不考察混乱产生的根源。混乱从哪里产生呢?起于人与人不相爱。臣与子不孝敬君和父,就是所谓乱。儿子爱自己而不爱父亲,因而损害父亲以自利;弟弟爱自己而不爱兄长,因而损害兄长以自利;臣下爱自己而不爱君上,因而损害君上以自利,这就是所谓混乱。反过来,即使父亲不慈爱儿子,兄长不慈爱弟弟,君上不慈爱臣下,这也是天下的所谓混乱。父亲爱自己而不爱儿子,所以损害儿子以自利;兄长爱自己而不爱弟弟,所以损害弟弟以自利;君上爱自己而不爱臣下,所以损害臣下以自利。这是什么缘故呢?都是起于不相爱。即使是天下做盗贼的也是这样。盗只爱自己的家,不爱别人的家,所以偷盗别人家而自得利益。贼只爱自身而不爱他人。所以抢夺他人身上东西而自得利益。这是什么原故呢?都是起于不相爱的缘故啊!即使是大夫互相侵扰其家,诸侯互相攻打其国也是这样的。大夫各自爱自己的家,不爱别人的家,所以扰乱他人的家而使自家得利。
    
    细察它们的起因,都起于不相爱。假若天下都能相亲相爱,爱别人就像爱自己,还能有不孝的吗?看待父亲、兄弟和君上像自己一样,怎么会做出不孝的事呢?还会有不慈爱的吗?看待弟弟、儿子与臣下像自己一样,怎么会做出不慈的事呢?所以不孝不慈都没有了。还有盗贼吗?看待别人的家像自己的家一样,谁会盗窃?看待别人就像自己一样,谁会害人?所以盗贼没有了。还有卿大夫相互侵扰封地,诸侯相互攻伐封国吗?看待别人的家族就像自己的家族,谁会侵犯?看待别人的封国就像自己的封国,谁会攻伐?所以卿大夫相互侵扰封地,诸侯相互攻伐封国,都没有了。假若天下的人都相亲相爱,国家与国家不相互攻伐,家族与家族不相互侵扰,盗贼没有了,君臣父子间都能孝敬慈爱,像这样,天下也就治理了。
    
    所以圣人既然是以治理天下为职业的人,怎么能不禁止相互仇恨而鼓励相爱呢?因此天下的人相亲相爱就会治理好,相互憎恶则会混乱。所以墨子说:“不能不鼓励爱别人”,道理就在此。
    
\end{normalsize}



\chapter{《论语》十二则}

\begin{normalsize}
    
    子曰:“为政以德。譬如北辰\footnote{〔北辰〕北极星。},居其所而众星共之。”
    
    子曰:“道之以政,齐之以刑,民免而无耻。道之以德,齐之以礼,有耻且格。”
    
    哀公\footnote{〔哀公〕春秋时期鲁国君主(公元前494年至前468年在位),鲁定公之子。}问于有若\footnote{〔有若〕字子有,孔子弟子,世称有子,孔子学说继承人之一。}曰:“年饥,用不足,如之何?”有若对曰:“盍彻乎?”曰:“二,吾犹不足,如之何其彻也?”对曰:“百姓足,君孰与不足?百姓不足,君孰与足?”
    
    齐景公\footnote{〔齐景公〕春秋时期齐国君主(公元前547年至前490年在位)。他联合鲁、卫反齐,试图重现齐桓公霸业,但最终失败。}问政于孔子,孔子对曰:“君君,臣臣,父父,子子。”公曰:“善哉!信如君不君、臣不臣、父不父、子不子,虽有粟,吾得而食诸?”
    
    季康子\footnote{〔季康子〕季孙肥,春秋时期鲁国的正卿。鲁国孟孙氏、叔孙氏和季孙氏专权,称为“三桓”,鲁国公权势衰微。}问政于孔子,孔子对曰:“政者,正也。子帅以正,孰敢不正?”
    
    子曰:“其身正,不令而行;其身不正,虽令不从。”
    
    子曰:“道千乘之国,敬事而信,节用而爱人,使民以时。”
    
    哀公问曰:“何为则民服?”孔子对曰:“举直错诸枉,则民服;举枉错诸直,则民不服。”
    
    定公\footnote{〔定公〕鲁定公,春秋诸侯国鲁国君主(公元前509年至前495年在位),请孔子相国,想从“三桓”手中夺回君权,但后来堕三都失败,孔子离开鲁国,鲁定公郁郁而终。}问:“君使臣,臣事君,如之何?”孔子对曰:“君使臣以礼,臣事君以忠。”
    
    子谓子产\footnote{〔子产〕公孙侨,字子美,郑穆公之孙、公子发之子。公元前554年为郑国卿,一度中兴郑国,前522年卒。}:“有君子之道四焉:其行己也恭,其事上也敬,其养民也惠,其使民也义。”
    
    子曰:“大哉尧\footnote{〔尧〕上古帝王,帝喾之子,姓伊祁,名放勋,初封于陶,后封为唐侯,辅佐他的兄长挚,二十岁继任天子,号陶唐,谥曰尧,史称唐尧。}之为君也!巍巍乎,唯天为大,唯尧则之。荡荡乎,民无能名焉。巍巍乎,其有成功也!焕乎,其有文章!”
    
    子曰:“禹\footnote{〔禹〕上古帝王,姒姓,夏后氏,名文命,黄帝的玄孙、颛顼的后代,鲧的儿子。因治水有功,被尊称为“大禹”。舜晚年禅位于禹。},吾无间然矣。菲饮食,而致孝乎鬼神;恶衣服,而致美乎黻冕;卑宫室,而尽力乎沟洫。禹,吾无间然矣。”
\end{normalsize}


\newpage

\textbf{译文}:

\vspace{1em}

\begin{normalsize}
    
    孔子说:“要用道德去治理国家。好比北极星,它安居在自己的位置上,所有的星辰都拱卫着它。”
    
    孔子说:“用政令来治理百姓,用刑罚来制约百姓,百姓(会知道如何)免于犯罪,但没有廉耻之心。如果用道德来统治百姓,用礼教来约束百姓,百姓不但有廉耻之心,而且会纠正自己的错误。”
    
    鲁哀公问有若说:“年成歉收,(我)用度不足,应该怎么办呢?”有若回答说:“何不实行十抽一的税率呢?”哀公说:“十抽二,尚且不够用,怎么能去实行十抽一呢?”有若回答说:“如果百姓用度足,国君怎么会用度不足呢?如果百姓用度不足,国君用度怎么会足呢?”
    
    齐景公问孔子如何为政,孔子回答说:“君主要像君主,臣子要像臣子,父亲要像父亲,儿子要像儿子。”景公说:“说得好啊!如果真的国君不像国君,臣子不像臣子,父亲不像父亲,儿子不像儿子,即使有粮食,我能够吃得着吗?”
    
    季康子问孔子如何为政,孔子回答说:“‘政’的意思就是端正。您端正自己作为表率,谁还敢不端正?”
    
    孔子说:“自身端正,不用命令(人们就会)遵行;自身不端正,虽发命令(人们也)不听从。”
    
    孔子说:“治理拥有一千辆兵车的国家,应该慎重地处理国事,守信用;节约用度而爱惜人民;役使民众要适时适度。”
    
    鲁哀公问道:“(我)怎么做才能使人民服(我)呢?”孔子答道:“把正直的人提拔上来,使他们位居不正直的人之上,人民就服你;如果把不正直的人提拔上来,使他们位居正直的人之上,人民就不服你。”
    
    鲁定公问:“君主役使臣子,臣子侍奉君主,应该怎么做?”孔子答道:“君主应该遵照礼节役使臣子,臣子应该忠诚尽心服务君主。”
    
    孔子评论子产说:“他有四个方面符合君子的标准:他待人处世谦逊有礼,侍奉国君认真慎重,养育民众时肯施恩惠,役使民众时也讲道理。”
    
    孔子说:“尧真是伟大的君主呀!多么崇高呀,唯有天最伟大,只有尧能效法于天(来治理天下)。(他的道理)多么广博呀,百姓不知道怎么描述(他治理天下的方法)。他实现的功绩,多么崇高啊!他制定的法度,多么灿烂美好呀!”
    
    孔子说:“禹,我对他没有意见了。他自己的饮食吃得很差,却用丰盛的祭品孝敬鬼神;他自己平时穿得很坏,却把祭祀的服饰和冠冕做得华美;他自己居住的房屋很差,却把力量完全用于沟渠水利上。禹,我对他没有意见了。”
    
\end{normalsize}



\chapter{牧民}

\begin{normalsize}
    
    凡有地牧民者,务在四时,守在仓廪。国多财,则远者来;地辟举,则民留处。仓廪实,则知礼节;衣食足,则知荣辱;上服度,则六亲固;四维张,则君令行。故省刑之要,在禁文巧;守国之度,在饰四维;顺民之经,在明鬼神、只山川、敬宗庙、恭祖旧。不务天时,则财不生;不务地利,则仓廪不盈。野芜旷,则民乃菅;上无量,则民乃妄。文巧不禁,则民乃淫;不璋两原,则刑乃繁。不明鬼神,则陋民不悟;不只山川,则威令不闻;不敬宗庙,则民乃上校;不恭祖旧,则孝悌不备。四维不张,国乃灭亡。
    
    国有四维,一维绝则倾,二维绝则危,三维绝则覆,四维绝则灭。倾可正也,危可安也,覆可起也,灭不可复错也。何谓四维?一曰礼,二曰义,三曰廉,四曰耻。礼不逾节,义不自进,廉不蔽恶,耻不从枉。故不逾节,则上位安;不自进,则民无巧诈;不蔽恶 ,则行自全;不从枉,则邪事不生。
    
    政之所兴,在顺民心;政之所废,在逆民心。民恶忧劳,我佚乐之;民恶贫贱,我富贵之;民恶危坠,我存安之;民恶灭绝,我生育之。能佚乐之,则民为之忧劳;能富贵之,则民为之贫贱;能存安之,则民为之危坠;能生育之,则民为之灭绝。故刑罚不足以畏其意,杀戮不足以服其心。故刑罚繁而意不恐,则令不行矣;杀戮众而心不服,则上位危矣。故从其四欲,则远者自亲;行其四恶,则近者叛之。故知予之为取者,政之宝也。
    
    错国於不倾之地。积粮於不涸之仓。藏财於不竭之府。下令於流水之原。使民於不争之官。明必死之路。开必得之门。不为不可成,不求不可得,不处不可久,不行不可复。错国於不倾之地者,授有德也。积粮於不涸之仓者,务五谷也。藏财於不竭之府者,养桑麻育六畜\footnote{〔六畜〕指马、牛、羊、鸡、狗、猪六种家畜。}也。下令於流水之原者,令顺民心也。使民於不争之官者,使各为其所长也。明必死之路者,严刑罚也。开必得之门者,信庆赏也。不为不可成者,量民力也。不求不可得者,不强民以其所恶也。不处不可久者,不偷取一时也。不行不可复者,不欺其民也。故授有德,则国安。务五谷,则食足。养桑麻、育六畜,则民富。令顺民心,则威令行。使民各为其所长,则用备。严刑罚,则民远邪。信庆赏,则民轻难。量民力,则事无不成。不强民以其所恶,则诈伪不生。不偷取一时,则民无怨心。不欺其民,则下亲其上。
    
    以家为乡,乡不可为也;以乡为国,国不可为也;以国为天下,天下不可为也。以家为家,以乡为乡,以国为国,以天下为天下。毋曰不同生,远者不听;毋曰不同乡,远者不行;毋曰不同国,远者不从。如地如天,何私何亲?如月如日,唯君之节。
    
    御民之辔,在上之所贵;道民之门,在上之所先;召民之路,在上之所好恶。故君求之,则臣得之;君嗜之,则臣食之;君好之,则臣服之;君恶之,则臣匿之。毋蔽汝恶,毋异汝度,贤者将不汝助。言室满室,言堂满堂,是谓圣王。城郭沟渠,不足以固守;兵甲强力,不足以应敌;博地多财,不足以有众。惟有道者,能备患於未形也,故祸不萌。
    
    天下不患无臣,患无君以使之;天下不患无财,患无人以分之。故知时者,可立以为长;无私者,可置以为政;审於时而察於用,而能备官者,可奉以为君也。缓者,後於事;吝於财者失所亲;信小人者失士。
\end{normalsize}


\newpage

\textbf{译文}:

\vspace{1em}

\begin{normalsize}
    
    凡拥有土地管治民众的人,必须致力于四时农事,确保粮食贮备。国家财力充足,远方的人们就能自动迁来,荒地开发得好,本国的人民就能安心留住。粮食富裕,人们就知道礼节;衣食丰足,人们就懂得荣辱。君主的服用合乎法度,六亲就可以相安无事;四维发扬,君令就可以贯彻推行。因此,减少刑罚的关键,在于禁止花言巧语文过饰非;巩固国家的准则,在于表彰四维;教训人民的根本办法,则在于:尊敬鬼神、祭祀山川、敬重祖宗和宗亲故旧。不注意天时,财富就不能增长;不注意地利,粮食就不会充足。田野荒芜废弃,人民也将由此而惰怠;君主挥霍无度,则人民胡作妄为;不注意禁止奢侈,则人民放纵淫荡;不堵塞这两个根源;犯罪者就会大量增多。不尊鬼神,小民就不能感悟;不祭山川,威令就不能远播;不敬祖宗,老百姓就会犯上;不尊重宗亲故旧,孝悌就不完备。四维不发扬,国家就会灭亡。
    
    国有四维,缺了一维,国家就倾斜;缺了两维,国家就危险;缺了三维,国家就倒覆;缺了四维,国家就会灭亡。倾斜可以扶正,危险可以挽救,倒覆可以再起,灭亡了就不可再建立了。什么是四维呢?一是礼,二是义,三是廉,四是耻。有礼,人们就不会超越应守的规范;有义,就不会不择手段追求名利;有廉,就不会掩饰罪恶;有耻,就不会趋从私心扭曲事实。人们不越出应守的规范,为君者的地位就安定;不贪图名利,人们就不巧谋欺诈;不掩饰罪恶,行为就自然端正;不趋从私心扭曲事实,不公正的事情也就不会发生了。
    
    政令所以能推行,在于顺应民心;政令所以废弛,在于违背民心。人民怕忧劳,我便使他安乐;人民怕贫贱,我便使他富贵;人民怕危难,我便使他安定;人民怕灭绝,我便使他生育繁息。因为我能使人民安乐,他们就可以为我承受忧劳;我能使人民富贵,他们就可以为我忍受贫贱;我能使人民安定,他们就可以为我承担危难;我能使人民生育繁息,他们也就不惜为我而牺牲了。单靠刑罚不足以使人民真正害伯,仅凭杀戮不足以使人民心悦诚服。刑罚繁重而人心不惧,法令就无法推行了;杀戮多行而人心不服,为君者的地位就危险了。因此,满足上述四种人民的愿望,疏远的自会亲近;强行上述四种人民厌恶的事情,亲近的也会叛离。由此可知,“予之于民就是取之于民” 这个原则,是治国的法宝。
    
    把国家建立在稳固的基础上。把粮食积存在取之不尽的粮仓里。把财货贮藏在用之不竭的府库里。把政令下达在流水源头上。把人民使用在无所争议的岗位上。向人们指出犯罪必死的道路。向人们敞开立功必赏的大门。不强干办不到的事,不追求得不到的利,不立足于难以持久的位置,不推行不可重复的事情。所谓把国家建立在稳固的基础上,就是把政权交给有道德的人。所谓把粮食积存在取之不尽的粮仓里,就是要努力从事粮食生产。所谓把财富贮藏在用之不竭的府库里,就是要种植桑麻、饲养六畜。所谓把政令下达在流水源头上,就是要令顺民心。所谓把人民使用在无所争议的岗位上,就是要尽其所长。所谓向人民指出犯罪必死的道路,就是刑罚严厉。所谓向人民敞开立功必赏的大门,就是奖赏信实。所谓不强干办不到的事,就是要度量民力。所谓不追求得不到的利,就是不强迫人民去做他们厌恶的事情。所谓不立足于难以持久的位置,就是不贪图一时侥幸。所谓不推行不可重复的事情,就是不欺骗人民。这样,把政权交给有道德的人,国家就能安定。努力从事粮食生产,民食就会充足。种植桑麻、饲养六畜,人民就可以富裕。能作到令顺民心,威令就可以贯彻。使人民各尽所长,用品就能齐备。刑罚严厉,人民就不去干坏事。奖赏信实,人民就不怕死难。量民力而行事,就可以事无不成。不强使人民干他们厌恶的事情,欺诈作假的行为就不会发生。不贪图一时侥幸,人民就不会抱怨。不欺骗人民,人民就拥戴君上。
    
    按照治家的要求治理乡,乡不能治好;按照治乡的要求治理国,国不能治好;按照治国的要求治理天下,天下不可能治好。应该按照治家的要求治家,按照治乡的要求治乡,按照治国的要求治国,按照治天下的要求治理天下。不要因为不同姓,不听取外姓人的意见;不要因为不同乡,不采纳外乡入的办法;诸候国不要因为不同国,而不听从别国人的主张。像天地一样对待万物,哪有什么偏私偏爱;君主遵守规范本分行事,应该像日月运行一样有常。
    
    驾驭人民的辔头(往哪个方向),看君主重视什么;引导人民走什么门路,看君主提倡什么;号召人民走什么途径,看君主的好恶是什么。君主想要的东西,臣下就会为他取得;君主迷恋的东西,臣下就会以其营生;君主喜欢的事情,臣下就会实行;君主厌恶的事情,臣下就会规避。因此,不要掩蔽你的过错,不要擅改你的法度;(否则,)贤者将无法对你帮助。在室内讲话,要使全室的人知道;在堂上讲话,要使满堂的人知道。这样开诚布公,才称得上圣明的君主。单靠城郭沟渠,不一定能固守;仅有强大的武力和装备,不一定能御敌;地大物博,群众不一定就拥护。
    
    天下不怕没有能臣,怕的是没有君主去使用他们;天下不怕没有财货,怕的是无人去管理它们。所以,通晓天时的,可以任用为首长;没有私心的,可以安排作官吏;通晓天时,善于用材,而又能任用官吏的,就可以奉为君主了。处事迟钝的人,总是落后于形势;吝啬财物的人,总是无人亲近;偏信小人的人,总是失掉贤能的人材。
    
\end{normalsize}


\newpage

\textbf{注解}:

\vspace{-1em}

\begin{itemize}
    \setlength\itemsep{-0.2em}
    \item〔礼不逾节,义不自进,廉不蔽恶,耻不从枉〕“逾节”就是不守规矩、不按章办事,或者做了超出自己本分和职责范围的事情。“自进”就是做什么事都为了自己的利益,为了让自己更进一步而不择手段,所谓被“权力欲”“物欲”等欲望支配。“蔽恶”就是包庇自己或他人的过错和犯罪,睁一只眼闭一只眼或找借口开脱。“从枉”就是徇私枉法,为了自己的私利而扭曲事实、违背道理,不正直不真诚,诡辩曲解,搞双重标准,破坏公正。
\end{itemize}

\chapter{劝学}

\begin{normalsize}
    
    君子曰:学不可以已。青,取之于蓝而青于蓝;冰,水为之而寒于水。木直中绳,輮以为轮,其曲中规,虽有槁暴,不复挺者,輮使之然也。故木受绳则直,金就砺则利。君子博学而日参省乎己,则知明而行无过矣。
    
    吾尝终日而思矣,不如须臾之所学也。吾尝跂而望矣,不如登高之博见也。登高而招,臂非加长也,而见者远。顺风而呼,声非加疾也,而闻者彰。假舆马者,非利足也,而致千里。假舟楫者,非能水也,而绝江河。君子生非异也,善假于物也。
    
    积土成山,风雨兴焉。积水成渊,蛟龙生焉。积善成德,而神明自得,圣心备焉。故不积跬步,无以至千里;不积小流,无以成江海。骐骥一跃,不能十步;驽马十驾,功在不舍。锲而舍之,朽木不折;锲而不舍,金石可镂。蚓无爪牙之利,筋骨之强,上食埃土,下饮黄泉,用心一也。蟹六跪而二螯,非蛇蟮之穴无可寄托者,用心躁也。是故无冥冥之志者,无昭昭之明;无惛惛之事者,无赫赫之功。行衢道者不至,事两君者不容。目不能两视而明,耳不能两听而聪。故君子结于一也。
\end{normalsize}


\newpage

\textbf{译文}:

\vspace{1em}

\begin{normalsize}
    
    君子说:学习不可以半途而止。靛青是从蓼蓝草里提炼出来的,却比蓼蓝草还要青;冰是由水凝结而成的,却比水还要寒冷。木材直得符合拉直的墨线,用煣的工艺把它制成车轮,那么木材的弯度就合乎圆的标准了。即使又被风吹日晒而干枯了,木材也不会再挺直,是因为经过加工使它成为这样的。所以木材用墨线量过再经辅具加工就能取直,刀剑在磨刀石上磨过就能变得锋利,君子广博地学习并且每天检验反省自己,那么他就会智慧明达而且行为没有过失了。
    
    我曾经整天思索,却不如片刻学到的知识多;我曾经踮起脚远望,却不如登到高处看得广阔。登到高处招手,胳膊没有加长,可是别人在远处也能看见;顺着风呼叫,声音没有变得洪亮,可是听的人在远处也能听得很清楚。借助车马的人,并不是脚走得快,却可以达到千里之外;借助舟船的人,并不善于游泳,却可以横渡江河。君子的资质秉性跟一般人没有不同,只是君子善于借助外物罢了。
    
    堆积土石成了高山,风雨从这里兴起;汇积水流成为深渊,蛟龙从这儿产生;积累善行养成高尚的道德,精神得到提升,圣人的心境由此具备。所以不积累一步半步的行程,就没有办法达到千里之远;不积累细小的流水,就没有办法汇成江河大海。骏马一跨跃,也不足十步远;劣马连走十天,它的成功在于不停止。如果刻几下就停下来了,那么腐朽的木头也刻不断。如果不停地刻下去,那么金石也能雕刻成功。蚯蚓没有锐利的爪子和牙齿,强健的筋骨,却能向上吃到泥土,向下喝到地下的泉水,这是由于它用心专一。螃蟹有六条腿,两个蟹钳,但是没有蛇、鳝的洞穴它就无处藏身,这是因为它用心浮躁。因此没有刻苦钻研的心志,学习上就不会有显著成绩;没有埋头苦干的实践,事业上就不会有巨大成就。在十字路口徘徊不定的人,任何一条路的尽头都不能到达;同时事奉两个君主的人,任何一方都不会容纳他。眼睛不可能同时看清楚两样东西,耳朵不可能同时听清楚两个声音。所以君子的意志坚定专一。
    
\end{normalsize}



\chapter{始得西山宴游记}

\begin{normalsize}
    
    自余为僇人,居是州,恒惴栗。其隙也,则施施而行,漫漫而游。日与其徒上高山,入深林,穷回溪,幽泉怪石,无远不到。到则披草而坐,倾壶而醉。醉则更相枕以卧,卧而梦。意有所极,梦亦同趣。觉而起,起而归。以为凡是州之山水有异态者,皆我有也,而未始知西山\footnote{〔西山〕湖南零陵县西。}之怪特。
    
    今年九月二十八日,因坐法华西亭\footnote{〔法华西亭〕法华寺,在零陵县城东山上,柳宗元在此建了西亭。},望西山,始指异之。遂命仆人过湘江\footnote{〔湘江〕应为潇水。},缘染溪\footnote{〔染溪〕又作“冉溪”,柳宗元又称为“愚溪”,是潇水的一条小支流。},斫榛莽,焚茅茷,穷山之高而止。攀援而登,箕踞而遨,则凡数州之土壤,皆在衽席之下。其高下之势,岈然洼然,若垤若穴,尺寸千里,攒蹙累积,莫得遁隐。萦青缭白,外与天际,四望如一。然后知是山之特立,不与培塿为类。悠悠乎与颢气俱,而莫得其涯;洋洋乎与造物者游,而不知其所穷。引觞满酌,颓然就醉,不知日之入。苍然暮色,自远而至,至无所见而犹不欲归。心凝形释,与万化冥合。然后知吾向之未始游,游于是乎始。故为之文以志。是岁,元和四年\footnote{〔元和〕唐宪宗李纯年号(公元806年至820年)。}也。
\end{normalsize}


\newpage

\textbf{译文}:

\vspace{1em}

\begin{normalsize}
    
    自从我成为被贬受辱的人,住在这个州,常常忧惧不安。公务之余,就缓步行走,没有目的地出游。每天和自己的随从爬高山、钻深林,走到迂回曲折的山间小溪,只要有清幽的泉水,奇形怪状的石头,没有(因为)远而不到的。到了就拨开杂草坐下,倾尽壶中的酒,喝的大醉。喝醉后便进一步相互枕靠着睡在地上,躺下就做梦。心中想到了哪里,梦也就做到那里。醒来后就起身回家。我原以为永州山水中稍有特异地方,都已被我游览了,而未曾知道西山的怪异和奇特。
    
    今年九月二十八日,由于坐在法华寺西亭,遥望西山,才发现它很奇特。我于是命令仆人越过湘江,沿染溪而行,砍去杂乱丛生的草木,烧掉茂盛的茅草,到山巅才止步。攀援着爬上山顶,像簸箕一样坐下观赏风景,周围几个州的土地都聚集在我的衽席下。它高处之下的地势,(高的地方)像深山一样深邃,(低的地方)像深池一样低陷,有的像是蚁穴外隆起的小土堆,有的像是蚂蚁洞,千里之遥如在尺寸之间,聚集收拢,层层堆叠,没有一个景象能逃脱(我的眼光)而隐藏起来的。青山白水相互萦绕,远处与天边交会,从四面望去,浑然一体。这样以后才知道这座山确实特立不群,与一般的小土丘大不一样。(西山的)高大渺远与天地间的浩瀚大气一样,永无边际;(西山的)广阔无边可与天地自然结友交游,永无尽期。我于是拿起酒壶,斟满酒杯,畅怀痛饮,醉倒在地,不觉间日薄西山。苍茫暮色,自远而近,慢慢地天黑得什么也看不见了,而我却了无归意。精神凝聚安定,形体得到解脱,和万物的变化暗暗相合。我这才认识到以往都不算游览,真正的游览从此才开始。所以我写文章把这件事记下来。这一年是元和四年。
    
\end{normalsize}



\chapter{师说}

\begin{normalsize}
    
    古之学者必有师。师者,所以传道受业解惑也。人非生而知之者,孰能无惑?惑而不从师,其为惑也终不解矣。生乎吾前,其闻道也固先乎吾,吾从而师之;生乎吾后,其闻道也亦先乎吾,吾从而师之。吾师道也,夫庸知其年之先后生于吾乎?是故无贵无贱,无长无少,道之所存,师之所存也。
    
    嗟乎!师道之不传也久矣!欲人之无惑也难矣!古之圣人,其出人也远矣,犹且从师而问焉;今之众人,其下圣人也亦远矣,而耻学于师。是故圣益圣,愚益愚。圣人之所以为圣,愚人之所以为愚,其皆出于此乎?爱其子,择师而教之;于其身也,则耻师焉,惑矣!彼童子之师,授之书而习其句读者,非吾所谓传其道解其惑者也。句读之不知,惑之不解,或师焉,或不焉,小学而大遗,吾未见其明也。巫医乐师百工之人,不耻相师。士大夫之族,曰师曰弟子云者,则群聚而笑之。问之,则曰:彼与彼年相若也,道相似也。位卑则足羞,官盛则近谀。呜呼!师道之不复可知矣!巫医乐师百工之人,君子不齿,今其智乃反不能及,其可怪也欤!
    
    圣人无常师。孔子师郯子\footnote{〔郯子〕春秋时郯国的国君,相传孔子曾向他请教官职。}、苌弘\footnote{〔苌弘〕东周敬王的大夫,相传孔子曾向他请教古乐。}、师襄\footnote{〔师襄〕春秋时鲁国的乐官,相传孔子曾向他学琴。}、老聃\footnote{〔老聃〕即老子,姓李名耳字聃,春秋时楚国人,思想家,道家学派创始人。相传孔子曾向他学习周礼。}。郯子之徒,其贤不及孔子。孔子曰:“三人行,则必有我师。”是故弟子不必不如师,师不必贤于弟子,闻道有先后,术业有专攻,如是而已。
    
    李氏子蟠\footnote{〔李氏子蟠〕李蟠,韩愈弟子,贞元十九年进士。},年十七,好古文,六艺\footnote{〔六艺〕指六经:《诗》《书》《礼》《乐》《易》《春秋》。}经传皆通习之,不拘于时,学于余。余嘉其能行古道,作《师说》以贻之。
\end{normalsize}


\newpage

\textbf{译文}:

\vspace{1em}

\begin{normalsize}
    
    古代求学的人必定有老师。老师,是用来传授道理、教授儒家经典著作、解释疑难问题的。人不是一生下来就懂得道理,谁能没有疑惑?有了疑惑,如果不跟老师学习,他所存在的疑惑,就始终不能解开。出生在我之前的人,他懂得的道理本来就比我早,我跟从他学习,以他为老师;出生在我之后的人,如果他懂得道理也比我早,我也跟从他,以他为老师。我是向他学习道理,哪里去考虑他的年龄比我大还是比我小呢?因此,没有地位高低贵贱,没有年纪大小,道理存在的地方,就是老师所在的地方。
    
    唉!古代从师的传统不流传已经很久了,想要人没有疑惑太难了!古代的圣人,他们超出一般人很远,尚且要跟从老师请教;现在的普通人,他们才智不及圣人也很远,却以向老师学习为耻。因此,圣人更加圣明,愚人更加愚昧。圣人成为圣人的原因,愚人成为愚人的原因,大概就是出于这个缘故吧?爱自己的孩子,选择老师来教他;但是对于他自己,却以跟从老师学习为可耻,真是糊涂啊!那些教小孩子的(启蒙)老师,教他读书,学习书中的文句的停顿,并不是我所说的传授道理,解答疑难问题的老师。不知句子停顿,有疑惑解决不了,有的向老师学习,有的不向老师学习,小的方面要学习,大的方面却放弃了,我没有看到他的明达。巫医、乐师、各种工匠这些人,不以互相学习为耻。士大夫这一类人,听到称“老师”称“弟子”的人,就聚在一起嘲笑他们。问他们,就说:“他和他年龄差不多,懂得的道理也差不多。(以)地位低(的人为师),就觉得羞耻,(以)官职高(的人为师),就近乎谄媚了。”哎!求师的风尚难以恢复由此可以知道了!巫医、乐师、各种工匠这些人,君子不屑一提,现在他们的智慧竟然反而比不上这些人了,这真是奇怪啊!
    
    圣人没有固定的老师。孔子曾以郯子、苌弘、师襄、老聃为师。郯子这些人,他们的贤能都比不上孔子。孔子说:“几个人一起走,其中一定有可以当我的老师的人。”因此学生不一定不如老师,老师不一定比学生贤能,听到的道理有早有晚,学问技艺各有专长,如此罢了。
    
    李家的孩子蟠,年龄十七,喜欢古文,六经的经文和传文都学完了,不受世俗的拘束,向我求学。我赞许他能够遵行古人从师的途径,写这篇《师说》来赠送他。
    
\end{normalsize}



\chapter{阿房宫赋}

\begin{normalsize}
    
    六王\footnote{〔六王〕指战国时齐、楚、燕、韩、赵、魏六国国王。}毕,四海一,蜀山兀,阿房\footnote{〔阿房〕阿房宫,秦始皇晚年修建的奢华宫殿。}出。覆压三百余里,隔离天日。骊山\footnote{〔骊山〕现在陕西省临潼县东南。}北构而西折,直走咸阳\footnote{〔咸阳〕秦朝都城,现在陕西咸阳市。}。二川\footnote{〔二川〕指渭水和樊川。渭水源出甘肃,流经陕西省;樊川即樊水,灞水的支流,在今陕西省。}溶溶,流入宫墙。五步一楼,十步一阁;廊腰缦回,檐牙高啄;各抱地势,钩心斗角。盘盘焉,囷囷焉,蜂房水涡,矗不知其几千万落。长桥卧波,未云何龙?复道行空,不霁何虹?高低冥迷,不知东西。歌台暖响,春光融融。舞殿冷袖,风雨凄凄。一日之内,一宫之间,而气候不齐。
    
    妃嫔媵嫱,王子皇孙,辞楼下殿,辇来于秦;朝歌夜弦,为秦宫人。明星荧荧,开妆镜也;绿云扰扰,梳晓鬟也;渭流涨腻,弃脂水也;烟斜雾横,焚椒兰也。雷霆乍惊,宫车过也;辘辘远听,杳不知其所之也。一肌一容,尽态极妍,缦立远视,而望幸焉。有不见者,三十六年。燕、赵之收藏,韩、魏之经营,齐、楚之精英,几世几年,摽掠其人,倚叠如山。一旦不能有,输来其间。鼎铛玉石,金块珠砾,弃掷逦迤,秦人视之,亦不甚惜。
    
    嗟乎!一人之心,千万人之心也。秦爱纷奢,人亦念其家。奈何取之尽锱铢\footnote{〔锱铢〕古时的重量单位。《说文》:六铢为锱。泛指微少。},用之如泥沙?使负栋之柱,多于南亩之农夫;架梁之椽,多于机上之工女;钉头磷磷,多于在庾之粟粒;瓦缝参差,多于周身之帛缕;直栏横槛,多于九土\footnote{〔九土〕九州,指天下。}之城郭;管弦呕哑,多于市人之言语。使天下之人,不敢言而敢怒。独夫之心,日益骄固。戍卒叫\footnote{〔戍卒叫〕指陈胜、吴广在谪戍渔阳途中起义。},函谷举\footnote{〔函谷举〕指刘邦攻破函谷关。},楚人一炬\footnote{〔楚人一炬〕楚人项羽攻入咸阳后,“烧秦宫室,火三月不灭”。},可怜焦土!
    
    灭六国者,六国也,非秦也。族秦者,秦也,非天下也。嗟乎!使六国各爱其人,则足以拒秦。使秦复爱六国之人,则递三世可至万世而为君,谁得而族灭也?秦人不暇自哀,而后人哀之。后人哀之而不鉴之,亦使后人而复哀后人也。
\end{normalsize}


\newpage

\textbf{译文}:

\vspace{1em}

\begin{normalsize}
    
    六国灭亡,秦始皇统一中国后,伐光了蜀山的树木,阿房宫才盖起来。阿房宫占地三百多里,楼阁高耸,遮天蔽日。从骊山之北构筑宫殿,曲折地向西延伸,一直修到秦京咸阳。渭水和樊川两条河,水波荡漾地流入宫墙。五步一栋楼,十步一座阁。走廊曲折像缦带一般回环,飞檐像禽鸟在高处啄食。楼阁各依地势的高下而建,像是互相环抱,宫室高低屋角,像钩一样联结,飞檐彼此相向,又像在争斗。盘旋地、曲折地,密接如蜂房,回旋如水涡,不知矗立着几千万座。长桥横卧在渭水上,人们看了要惊讶:天上没有云,怎么出现了龙?复道横空而过,彩色斑烂,人们看了要诧异:不是雨过天晴,哪里来的彩虹?楼阁随着地势高高低低,使人迷糊,辨不清东西方向。台上歌声悠扬,充满暖意,使人感到有如春光那样和煦。
    
    那些亡了国的妃嫔和公主们,辞别了自己国家的楼阁、宫殿,被一车车送来秦国,日夜献歌奏乐,成了秦国的宫人。星光闪烁,原来是她们打开了梳妆镜子;绿云缭绕,原来是她们正在早晨梳理发髻;渭水河面上浮起一层垢腻,是她们倒掉的残脂剩粉;空中烟雾弥漫,是她们在焚烧椒兰香料。声如雷霆,让她们惊恐难安,是皇宫的马车驰过;听着车声渐远,直到寂静,不知到哪儿去了。宫人们用尽心思修饰容貌,打扮得极其娇媚妍丽,耐心地久立远视,盼望皇帝能亲自驾临。可是有许多宫女整整等了三十六年,还未见到皇帝。燕、赵收藏的财宝,韩、魏聚敛的金玉,齐、楚搜求的珍奇,这都是多少世代、多少年月以来,从人民那里掠夺来的,堆积得如山一样。一旦国家灭亡,不能占有了,统统运进了阿房宫。在这里把宝鼎看作铁锅,美玉当石头,又视黄金为土块,珍珠为沙石,随意丢弃,秦人看见了也不觉得可惜。
    
    唉!一个人的心思,和千万人的心思一样啊。秦始皇喜爱奢侈,老百姓也顾念自己的家业。为什么搜刮人民的财物一分一厘都不放过,挥霍时却像泥沙一样毫不珍惜呢?阿房宫中的柱子,比田里的农夫还多;架在梁上的椽子,比织布机上的女2工还多;建筑物上的钉头,比粮仓里的粟粒还多;横直密布的瓦缝,比身上衣服的线缝还多;栏杆纵横,比天下的城郭还多;嘈杂的器乐声,比闹市的人说话声还多。(这)使天下的老百姓敢怒而不敢言。秦始皇这个独夫,却越来越骄横顽固。于是,陈胜、吴广揭竿而起,四方响应,刘邦攻破函谷关,项羽放了一把火,可惜富丽堂皇的阿房宫变成了一片焦土。
    
    唉!灭亡六国的是六国自己,而不是秦国。灭亡秦国的是秦国自己,而不是天下百姓。唉!如果六国统治者都是爱护本国人民,那么就有足够的力量抗拒秦国。如果秦国统治者同样能爱护六国的人民,那么秦就能从三世传下去,甚至可以传到万世,都为君王,谁还能灭掉秦国呢?秦统治者来不及为自己的灭亡哀叹,只好让后世的人为他们哀叹。
    
\end{normalsize}



\chapter{六国论}

\begin{normalsize}
    
    六国破灭,非兵不利,战不善,弊在赂秦。赂秦而力亏,破灭之道也。或曰:六国互丧,率赂秦耶?曰:不赂者以赂者丧,盖失强援,不能独完。故曰:弊在赂秦也。
    
    秦以攻取之外,小则获邑,大则得城。较秦之所得,与战胜而得者,其实百倍。诸侯之所亡,与战败而亡者,其实亦百倍。则秦之所大欲,诸侯之所大患,固不在战矣。思厥先祖父,暴霜露,斩荆棘,以有尺寸之地。子孙视之不甚惜,举以予人,如弃草芥。今日割五城,明日割十城,然后得一夕安寝。起视四境,而秦兵又至矣。然则诸侯之地有限,暴秦之欲无厌,奉之弥繁,侵之愈急。故不战而强弱胜负已判矣。至于颠覆,理固宜然。古人云:“以地事秦,犹抱薪救火,薪不尽,火不灭。”此言得之。
    
    齐人未尝赂秦,终继五国迁灭,何哉?与嬴而不助五国也。五国既丧,齐亦不免矣。燕赵之君,始有远略,能守其土,义不赂秦。是故燕虽小国而后亡,斯用兵之效也。至丹以荆卿为计,始速祸焉。赵尝五战于秦,二败而三胜。后秦击赵者再,李牧连却之。自牧以谗诛,邯郸为郡,惜其用武而不终也。且燕赵处秦革灭殆尽之际,可谓智力孤危,战败而亡,诚不得已。向使三国各爱其地,齐人勿附于秦,刺客不行,良将犹在,则胜负之数,存亡之理,当与秦相较,或未易量。
    
    呜呼!以赂秦之地封天下之谋臣,以事秦之心礼天下之奇才,并力西向,则吾恐秦人食之不得下咽也。悲夫!有如此之势,而为秦人积威之所劫,日削月割,以趋于亡。为国者无使为积威之所劫哉!
    
    夫六国与秦皆诸侯,其势弱于秦,而犹有可以不赂而胜之之势。苟以天下之大,下而从六国破亡之故事,是又在六国下矣。
\end{normalsize}


\newpage

\textbf{译文}:

\vspace{1em}

\begin{normalsize}
    
    六国的灭亡,不是(因为他们的)武器不锋利,仗打得不好,弊端在于用土地来贿赂秦国。拿土地贿赂秦国亏损了自己的力量,(这就)是灭亡的原因。有人问:“六国一个接一个的灭亡,难道全部是因为贿赂秦国吗?”(回答)说:“不贿赂秦国的国家因为有贿赂秦国的国家而灭亡。原因是不贿赂秦国的国家失掉了强有力的外援,不能独自保全。
    
    秦国除了用战争夺取土地以外,(还受到诸侯的贿赂),小的就获得邑镇,大的就获得城池。比较秦国受贿赂所得到的土地与战胜别国所得到的土地,(前者)实际多百倍。六国诸侯(贿赂秦国)所丧失的土地与战败所丧失的土地相比,实际也要多百倍。那么秦国最想要的,与六国诸侯最担心的,本来就不在于战争。想到他们的祖辈和父辈,冒着寒霜雨露,披荆斩棘,才有了很少的一点土地。子孙对那些土地却不很爱惜,全都拿来送给别人,就像扔掉小草一样不珍惜。今天割掉五座城,明天割掉十座城,这才能睡一夜安稳觉。明天起床一看四周边境,秦国的军队又来了。既然这样,那么诸侯的土地有限,强暴的秦国的欲望永远不会满足,(诸侯)送给他的越多,他侵犯得就越急迫。所以用不着战争,谁强谁弱,谁胜谁负就已经决定了。到了覆灭的地步,道理本来就是这样子的。古人说:“用土地侍奉秦国,就好像抱柴救火,柴不烧完,火就不会灭。”这话说的很正确。
    
    齐国不曾贿赂秦国,(可是)最终也随着五国灭亡了,为什么呢?(是因为齐国)跟秦国交好而不帮助其他五国。五国已经灭亡了,齐国也就没法幸免了。燕国和赵国的国君,起初有长远的谋略,能够守住他们的国土,坚持正义,不贿赂秦国。因此燕虽然是个小国,却后来才灭亡,这就是用兵抗秦的效果。等到后来燕太子丹用派遣荆轲刺杀秦王作对付秦国的计策,这才招致了(灭亡的)祸患。赵国曾经与秦国交战五次,打了两次败仗,三次胜仗。后来秦国两次攻打赵国,(赵国大将)李牧接连打退秦国的进攻。等到李牧因受诬陷而被杀死,(赵国都城)邯郸变成(秦国的一个)郡,可惜赵国用武力抗秦而没能坚持到底。而且燕赵两国正处在秦国把其他国家快要消灭干净的时候,可以说是智谋穷竭,国势孤立危急,战败了而亡国,确实是不得已的事。假使韩、魏、楚三国都爱惜他们的国土,齐国不依附秦国,(燕国的)刺客不去(刺秦王),(赵国的)良将李牧还活着,那么胜败的命运,存亡的理数,倘若与秦国相比较,也许还不容易衡量(出高低来)呢。
    
    唉!(如果六国诸侯)用贿赂秦国的土地来封给天下的谋臣,用侍奉秦国的心来礼遇天下的奇才,齐心合力地向西(对付秦国),那么,我恐怕秦国人饭也不能咽下去。真可悲啊!有这样的有利形势,却被秦国积久的威势所胁迫,天天割地,月月割地,以至于走向灭亡。治理国家的人不要被积久的威势所胁迫啊!
    
    六国和秦国都是诸侯之国,他们的势力比秦国弱,却还有可以不贿赂秦国而战胜它的优势。如果凭借偌大国家,却追随六国灭亡的前例,这就比不上六国了。
    
\end{normalsize}



\chapter{《孟子》六则}

\begin{normalsize}
    
    三代之得天下也以仁,其失天下也以不仁。国之所以废兴存亡者亦然。天子不仁,不保四海;诸侯不仁,不保社稷;卿大夫不仁,不保宗庙;士庶人不仁,不保四体。今恶死亡而乐不仁,是犹恶醉而强酒。
    
    以力假仁者霸,霸必有大国;以德行仁者王,王不待大。汤以七十里,文王以百里。以力服人者,非心服也,力不赡也;以德服人者,中心悦而诚服也,如七十子之服孔子也\footnote{〔七十子之服孔子〕《史记.孔子世家》:“孔子以诗书礼乐教弟子,盖三千焉,身通六艺者七十有二人。”《史记.仲尼弟子列传》:“孔子曰‘受业身通者七十有七人’,皆异能之士也。”}。《诗》云\footnote{〔《诗》云……〕出自《诗经·大雅·文王有声》。}:“自西自东,自南自北,无思不服。”此之谓也。
    
    天时不如地利,地利不如人和。三里之城,七里之郭,环而攻之而不胜。夫环而攻之,必有得天时者矣;然而不胜者,是天时不如地利也。城非不高也,池非不深也,兵革非不坚利也,米粟非不多也,委而去之,是地利不如人和也。故曰:域民不以封疆之界,固国不以山溪之险,威天下不以兵革之利。得道者多助,失道者寡助。寡助之至,亲戚畔之。多助之至,天下顺之。以天下之所顺,攻亲戚之所畔,故君子有不战,战必胜矣。
    
    舜\footnote{〔舜〕古代圣君,三皇五帝之一。他出身贫寒,年轻时在历山耕田。他的父亲顽劣,母亲嚣张,弟弟傲慢,但他仍能以孝道和睦家庭。尧发现他的才能和品德后,将帝位禅让给他。}发于畎亩之中,傅说\footnote{〔傅说〕商朝宰相。他本是筑墙的泥水匠,在傅岩筑墙时被商王武丁发现。}举于版筑之间,胶鬲\footnote{〔胶鬲〕商朝大臣。}举于鱼盐之中,管夷吾\footnote{〔管夷吾〕管仲,字夷吾,出身贫苦,经商为生,后经鲍叔牙举荐为齐相。}举于士,孙叔敖\footnote{〔孙叔敖〕楚国名相。他本是郊野平民,因在家乡主持修建水利,被楚庄王看重,任命为令尹。}举于海,百里奚\footnote{〔百里奚〕秦国名相。他本是虞国大夫。晋献公灭虞后成为奴隶。秦穆公用五张羊皮将他赎身,成为秦国大夫。}举于市。故天将降大任于是人也,必先苦其心志,劳其筋骨,饿其体肤,空乏其身,行拂乱其所为,所以动心忍性,曾益其所不能。人恒过,然后能改;困于心,衡于虑,而后作;徵于色,发于声,而后喻。入则无法家拂士,出则无敌国外患者,国恒亡。然后知生于忧患,而死于安乐也。
    
    今之事君者皆曰:“我能为君辟土地,充府库。”今之所谓良臣,古之所谓民贼也。君不乡道,不志于仁,而求富之,是富桀也。“我能为君约与国,战必克。”今之所谓良臣,古之所谓民贼也。君不乡道,不志于仁,而求为之强战,是辅桀也。由今之道,无变今之俗,虽与之天下,不能一朝居也。
    
    不仁者可与言哉?安其危而利其灾,乐其所以亡者。不仁而可与言,则何亡国败家之有?有孺子歌曰:“沧浪之水清兮,可以濯我缨;沧浪之水浊兮,可以濯我足。”孔子曰:“小子听之!清斯濯缨,浊斯濯足矣,自取之也。”夫人必自侮,然后人侮之;家必自毁,而后人毁之;国必自伐,而后人伐之。《太甲》曰\footnote{〔《太甲》曰……〕出自《尚书·商书·太甲》:“天作孽,犹可违;自作孽,不可逭。”}:“天作孽,犹可违;自作孽,不可活。”此之谓也。
\end{normalsize}


\newpage

\textbf{译文}:

\vspace{1em}

\begin{normalsize}
    
    夏、商、周三代获得天下是由于仁,他们失去天下是由于不仁。国家的兴起和衰败,生存和灭亡也是如此。天子如果不仁,便不能保有天下;诸侯如果不仁,便不能保有国家;卿大夫如果不仁,便不能保有他的祖庙;士和百姓如果不仁,便不能保全他们的身体。现在有的人怕死怕失去(财富地位),却乐于不仁,就好比怕醉却强行喝酒一样。
    
    仗着实力假借仁义可以称霸,称霸一定要凭借国力的强大;依靠道德来实行仁义的可以为王,为王不必凭借强大的国力。商汤的封地也就方圆七十里,周文王的封地也就方圆百里(,但最终得了天下)。仗着实力来使人服从的,人家心里不会服气,只因为实力不够(才服从);依靠道德来使人服从的,才会让人打心底里喜欢而真的服气,就好像(孔子的)七十多位弟子归服孔子一样。《诗经》里说:“从西从东、从南从北(来朝的),没有不服的。”说的就是这个。
    
    有利于作战的天气、时令,比不上有利于作战的地理形势;有利于作战的地理形势,比不上作战中的人心所向、内部团结。一座方圆三里的小城,有方圆七里的外城,四面包围起来攻打它,却不能取胜。采用四面包围的方式攻城,一定是得到有利于作战的天气、时令了,可是不能取胜,这(是)有利于作战的天气时令不如有利于作战的地理形势。城墙并不是不高啊,护城河并不是不深呀,武器装备也并不是不精良,粮食供给也并不是不充足啊,但是,守城一方还是弃城而逃,这是因为作战的地理形势再好,也比不上人心所向、内部团结啊。所以说,使人民定居下来而不迁到别的地方去,不能靠疆域的边界,巩固国防不能凭借险要的山河,威慑天下不能凭借锐利的武器。能行“仁政”的君王,帮助支持他的人就多,不施行“仁政”的君主,支持帮助他的人就少。支持帮助他的人少到了极点,连内外亲属也会背叛他;支持帮助他的人多到了极点,天下所有人都会归顺他。凭着天下人都归顺他的条件,去攻打那连亲属都反对背叛的君王。所以,能行仁政的君主不战则已,战就一定能胜利。
    
    舜从田野耕作之中被起用,傅说从筑墙的劳作之中被起用,胶鬲从贩鱼卖盐中被起用,管夷吾是从狱官手里被释放并加以任用的,孙叔敖从海滨隐居的地方被起用,百里奚被从奴隶集市里赎买回来并被起用。所以上天要把重任降临在某人的身上,必定要先使他的内心痛苦,使他的筋骨劳累,使他经受饥饿之苦,以致肌肤消瘦,使他身处贫困之中,使他做事不顺,通过那样的途径来使他的心灵受到震撼,使他的性格坚韧起来,增加他所不具备的能力。人总是犯了错后才能悔改;内心忧困,思想阻塞,然后才能奋起;事情表现在脸色上,流露在言谈中,才能了解明白。一个国家,在内如果没有坚守法度的大臣和足以辅佐君王的贤士,在外没有与之匹敌的邻国和来自外国的祸患,就常常会有覆灭的危险。这样,就知道常处忧愁祸患之中可以使人生存,常处安逸快乐之中可以使人死亡的道理了。
    
    今天侍奉君主的人都说:“我能够替君主开拓土地,充实府库。”今天的所谓良臣,正是古代说的民贼。君主不向往道德,无意于仁,却想让他富足,这等于让夏桀富足。“我能够替君主联合诸侯,每战必胜。”今天的所谓良臣,正是古代说的民贼。君主不向往道德,无意于仁,却想助他强行打仗,这等于辅助夏桀。顺着当今这条路走下去,也不改变当今的风俗习气,即便给他整个天下,他也是一天都坐不安稳的。
    
    不仁的人难道是可以用言语教化的吗?危险了还觉得安全,遭灾了还觉得有利,喜欢导致他败亡的东西。假如不仁的人是可以用言语教化的,那世上又如何会有亡国败家的惨祸呢?从前有个小孩唱道:“沧浪的水清啊,可以洗我的帽缨;沧浪的水浊呀,可以洗我的双足。”孔子说:“小子们听好了!(看到)清水就(想到用来)洗帽缨,(看到)浊水(就想到)洗双足,是自己招致的。”人必定先自己做了让自己受侮辱的事,别人才侮辱他;家必定先出了让自己毁坏的事,别人才毁坏它;邦国必定先发生让自己遭到讨伐的事,别人才讨伐它。《尚书·太甲》说:“天造的罪孽,还可以违抗;自己造的罪孽,不会有活路。”说的就是这个。
    
\end{normalsize}


\newpage

\textbf{注解}:

\vspace{-1em}

\begin{itemize}
    \setlength\itemsep{-0.2em}
    \item〔三里之城,七里之郭〕內城外郭。《吴越春秋》:“筑城以卫君,造郭以守民,此城郭之始也”。“城”是以政治军事职能为主、作为权力中心的聚落;“郭”是承担城市中商业、手工业、农业及居民区等经济生活职能的外围区域。郭,廓也。廓落在城外也,具体分为“内城外郭”和“西城东郭”两种形制。城、郭都是大致方形的,“三里”和“七里”指的是边长。在战国时期,三里长、七里长的都是较小的城郭。
    \item〔人恒过,然后能改;困于心,衡于虑,而后作;徵于色,发于声,而后喻。〕《孟子集注》:“衡,与横同。恒,常也。犹言大率也。横,不顺也。作,奋起也。徵,验也。喻,晓也。此又言中人之性,常必有过,然后能改。盖不能谨于平日,故必事势穷蹙,以至困于心,横于虑,然后能奋发而兴起;不能烛于几微,故必事理暴著,以至验于人之色,发于人之声,然后能警悟而通晓也。”这里说的是矛盾的对立统一之间的转化关系。人经常犯错,错得多了,体会到了错的后果,才会想着去改。心中有困扰、忧虑,才会去振作。人与人之间的看法意见,到了表现在脸色上,发声说出来的地步,才会让人知晓警醒。矛盾的一方面暴露出来了,才会向着另一方面发展。从这些现象出发,孟子归纳出“生于忧患,死于安乐”的结论。”
    \item〔清斯濯缨,浊斯濯足矣,自取之也。〕《孟子集注》:“言水之清濁有以自取之也。”看到清水,就想用来清洗帽缨;看到浊水,就想用来清洗双脚。这是因为人的观念里,帽缨就是整洁的,应该用最清澈的水来洗;双脚就是脏的,所以用浑浊的水来洗也没关系。所以清水清洗帽缨之后还是清澈,而浊水洗脚之后更加浑浊,是自己的性质招致的。后面把这个道理引导到人、家、国上:只有一个人先有导致自己被欺侮的性质,才会招致欺侮。家和国也一样,如果内部强盛和谐,就不会招致祸端,只有内部出了问题,才会导致毁坏、讨伐。
\end{itemize}

\chapter{召公谏厉王弭谤}

\begin{normalsize}
    
    厉王\footnote{〔厉王〕周厉王,周夷王之子,名胡,西周的第十个国君(前878至前842年)。}虐,国人谤王。召公\footnote{〔召公〕召氏,名虎,谥穆,也称召穆公,召幽伯之子。}告曰:“民不堪命矣!”王怒,得卫巫\footnote{〔卫巫〕卫:卫国。巫:古代以降神事鬼为职业的人。},使监谤者。以告,则杀之。国人莫敢言,道路以目。
    
    王喜,告召公曰:“吾能弭谤矣,乃不敢言。”
    
    召公曰:“是障之也。防民之口,甚于防川。川壅而溃,伤人必多,民亦如之。是故为川者决之使导;为民者宣之使言。故天子听政,使公卿至于列士献诗,瞽献曲,史献书,师箴,瞍赋,矇诵,百工谏,庶人传语,近臣尽规,亲戚补察,瞽、史教诲,耆艾修之,而后王斟酌焉。是以事行而不悖。民之有口也,犹土之有山川也,财用于是乎出;犹其有原隰衍沃也,衣食于是乎生。口之宣言也,善败于是乎兴。行善而备败,其所以阜财用衣食者也。夫民虑之于心而宣之于口,成而行之,胡可壅也?若壅其口,其与能几何?”
    
    王弗听,于是国人莫敢出言。三年,乃流王于彘\footnote{〔彘〕地名,在今山西省霍县境内。}。
\end{normalsize}


\newpage

\textbf{译文}:

\vspace{1em}

\begin{normalsize}
    
    周厉王暴虐,国都的人都批评他,说他不好。召穆公对厉王说:“老百姓忍受不了暴政了!”厉王听了勃然大怒,找卫国的巫人去监视批评国君的人。接到巫人的告发,就杀掉说坏话的人。国都的人不敢说话,路上相见,用眼神示意。
    
    周厉王很高兴,对召穆公说:“我终于消灭了坏话,他们不敢说话!”
    
    召公回答说:“你这样做是堵住民众的嘴。阻止民众说话,比堵塞河川还要严重。河流堵塞后一旦决堤,伤人一定很多,人民也是这样。因此治水的人疏通河道使它畅通,治民的人只能开导民众,让人说话。所以君王处理政事,让三公九卿以至各级官吏进献讽喻诗,乐师进献民间乐曲,史官进献有借鉴意义的史籍,少师诵读箴言,盲人吟咏诗篇,有眸子而看不见的盲人诵读讽谏之言,掌管营建事务的百工纷纷进谏,平民则将自己的意见转达给君王,近侍之臣尽规劝之责,君王的同宗都能补其过失,察其是非,乐师和史官以歌曲、史籍加以谆谆教导,元老们再进一步修饰整理,然后由君王斟酌取舍,付之实施。这样,国家的政事得以实行而不违背道理。民众有口,就像大地有高山河流一样,社会的物资财富全靠它出产;又像高原和低地都有平坦肥沃的良田一样,衣物食物全靠它产生。人们用嘴巴发表议论,政事的好坏得失才能表露出来。人民觉得好的就尽力实行,人民觉得失败的就设法预防,这是增加衣食财富的途径啊。人们心中思虑的事情从口里说出来,形成意见然后传开来,怎么可以堵呢?如果堵住人民的嘴,能有多少人支持(你)呢?”
    
    周厉王不听,于是国都的人再也不敢出来说话。过了三年,就把周厉王放逐到彘地去了。
    
\end{normalsize}



\chapter{石钟山记}

\begin{normalsize}
    
    《水经》\footnote{〔《水经》〕汉代记载地理山川的书籍。}云:“彭蠡\footnote{〔彭蠡〕鄱阳湖的又一名称。}之口有石钟山焉。”郦元\footnote{〔郦元〕即郦道元,北魏人,地理学家,著《水经注》,为《水经》作注。}以为下临深潭,微风鼓浪,水石相搏,声如洪钟。是说也,人常疑之。今以钟磬\footnote{〔磬〕古代打击乐器,形状像曲尺,用玉或石制成。}置水中,虽大风浪不能鸣也,而况石乎!至唐李渤\footnote{〔李渤〕唐朝洛阳人,写过一篇《辨石钟山记》。}始访其遗踪,得双石于潭上,扣而聆之,南声函胡,北音清越,桴止响腾,余韵徐歇。自以为得之矣。然是说也,余尤疑之。石之铿然有声者,所在皆是也,而此独以钟名,何哉?
    
    元丰七年\footnote{〔元丰〕宋神宗的年号。}六月丁丑\footnote{〔六月丁丑〕农历六月初九。},余自齐安\footnote{〔齐安〕现在湖北黄州。}舟行适临汝\footnote{〔临汝〕即汝州,现在河南临汝。},而长子迈将赴饶之德兴尉\footnote{〔饶之德兴尉〕饶州德兴县(现在江西德兴)的县尉。},送之至湖口\footnote{〔湖口〕现在江西湖口。},因得观所谓石钟者。寺僧使小童持斧,于乱石间择其一二扣之,硿硿焉。余固笑而不信也。至莫夜月明,独与迈乘小舟,至绝壁下。大石侧立千尺,如猛兽奇鬼,森然欲搏人;而山上栖鹘,闻人声亦惊起,磔磔云霄间;又有若老人咳且笑于山谷中者,或曰此鹳鹤也。余方心动欲还,而大声发于水上,噌吰如钟鼓不绝。
    
    舟人大恐。徐而察之,则山下皆石穴罅,不知其浅深,微波入焉,涵淡澎湃而为此也。舟回至两山间,将入港口,有大石当中流,可坐百人,空中而多窍,与风水相吞吐,有窾坎镗鞳之声,与向之噌吰者相应,如乐作焉。因笑谓迈曰:“汝识之乎?噌吰者,周景王之无射\footnote{〔周景王之无射〕《国语》记载,周景王二十三年(公元前522年)铸成“无射”钟。}也;窾坎镗鞳者,魏庄子之歌钟\footnote{〔魏庄子之歌钟〕《左传》记载,鲁襄公十一年(公元前561年)郑人以歌钟和其他乐器献给晋侯,晋侯分一半赐给晋大夫魏绛。庄子,魏绛的谥号。歌钟,古乐器。}也。古之人不余欺也!”
    
    事不目见耳闻,而臆断其有无,可乎?郦元之所见闻,殆与余同,而言之不详;士大夫终不肯以小舟夜泊绝壁之下,故莫能知;而渔工水师虽知而不能言。此世所以不传也。而陋者乃以斧斤考击而求之,自以为得其实。余是以记之,盖叹郦元之简,而笑李渤之陋也。
\end{normalsize}


\newpage

\textbf{译文}:

\vspace{1em}

\begin{normalsize}
    
    《水经》说:“鄱阳湖的湖口有石钟山。”郦道元认为在下面靠近深潭,微风振动波浪,水和石互相碰撞,发出的声音好像大钟一般。这个说法,人们常常怀疑它。如果现在把钟磬放在水中,即使大风大浪也不能使它发出声响,何况是石头呢!到了唐代的李渤才去探寻它的所在地,在深潭边找到两块山石,敲打它们,听它们的声音,南边的声音重浊而模糊,北边的声音清脆而响亮,鼓槌停止了敲击,声音还在传播,余音慢慢地消失。他自己认为找到了这个石钟山命名的原因。但是这个说法,我更加怀疑。能发出铿锵声音的山石,到处都是,可是唯独这座山用钟来命名,为什么呢?
    
    元丰七年六月初九,我从齐安坐船到临汝去,大儿子苏迈将要去就任饶州的德兴县的县尉,我送他到湖口,趁此能够观察所说的“石钟”。庙里的和尚叫小童拿着斧头,在乱石中间选一两处敲打它,硿硿地发出声响。我只是笑,并不相信。到了晚上,月光明亮,我和苏迈坐着小船来到绝壁下面。巨大的山石竖立着,有千尺,好像凶猛的野兽和奇异的鬼怪一样,阴森森地想要向人扑去;山上宿巢的老鹰,听到人声也受惊飞起来,在高空中发出磔磔地鸟鸣声;又有像老人在山谷中边咳边笑的声音,有人说这是鹳鹤。我正心惊想要回去,忽然巨大的声音从水上发出,声音洪亮像钟鼓声连续不断。
    
    凡事不亲眼看到亲耳听到,却根据主观猜测去推断它的有或没有,可以吗?郦道元所看到的、所听到的,大概和我一样,但是描述它不详细;士大夫终究不愿用小船在夜里在悬崖绝壁的下面停泊,所以不能知道;但渔人和船工,虽然知道却又不能用文字表达、记载。这就是世上没有流传下来石钟山得名由来的原因。而浅陋的人竟然用斧头敲打石头来寻求石钟山得名的原因,自以为得到了这个事情的真相。我因此记下以上的经过,叹惜郦道元简略,嘲笑李渤的浅陋。
    
    
    
\end{normalsize}



\chapter{游褒禅山记}

\begin{normalsize}
    
    褒禅山亦谓之华山\footnote{〔华山〕褒禅山在安徽马鞍山市含山县,属大别山余脉,横亘在巢湖之滨,别名花山(古时“华”通“花”)。褒禅山不是五岳中的西岳华山。}。唐浮图\footnote{〔浮图〕梵语音译词,也写作“浮屠”或“佛图”,本意是佛,这里指佛教徒。}慧褒始舍于其址,而卒葬之;以故其后名之曰“褒禅”。今所谓慧空禅院者,褒之庐冢也。距其院东五里,所谓华山洞者,以其乃华山之阳名之也。距洞百余步,有碑仆道,其文漫灭,独其为文犹可识曰“花山”。今言“华”如“华实”之“华”者,盖音谬也。
    
    其下平旷,有泉侧出,而记游者甚众,所谓前洞也。由山以上五六里,有穴窈然,入之甚寒,问其深,则其好游者不能穷也,谓之后洞。余与四人拥火以入,入之愈深,其进愈难,而其见愈奇。有怠而欲出者,曰:“不出,火且尽。”遂与之俱出。盖余所至,比好游者尚不能十一,然视其左右,来而记之者已少。盖其又深,则其至又加少矣。方是时,余之力尚足以入,火尚足以明也。既其出,则或咎其欲出者,而余亦悔其随之,而不得极夫游之乐也。
    
    于是余有叹焉。古人之观于天地、山川、草木、虫鱼、鸟兽,往往有得,以其求思之深而无不在也。夫夷以近,则游者众;险以远,则至者少。而世之奇伟、瑰怪,非常之观,常在于险远,而人之所罕至焉,故非有志者不能至也。有志矣,不随以止也,然力不足者,亦不能至也。有志与力,而又不随以怠,至于幽暗昏惑而无物以相之,亦不能至也。然力足以至焉,于人为可讥,而在己为有悔;尽吾志也而不能至者,可以无悔矣,其孰能讥之乎?此余之所得也。
    
    余于仆碑,又以悲夫古书之不存,后世之谬其传而莫能名者,何可胜道也哉!此所以学者不可以不深思而慎取之也。
    
    四人者:庐陵\footnote{〔庐陵〕今江西吉安市。}萧君圭君玉,长乐\footnote{〔长乐〕今福建福州市长乐区。}王回深父,余弟安国平父、安上纯父。
    
    至和元年\footnote{〔至和〕宋仁宗的年号。}七月某日,临川\footnote{〔临川〕现在江西抚州市临川区。}王某记。
\end{normalsize}


\newpage

\textbf{译文}:

\vspace{1em}

\begin{normalsize}
    
    褒禅山也称为华山。唐代和尚慧褒当初在这里筑室居住,死后又葬在那里;因为这个缘故,后人就称此山为褒禅山。现在人们所说的慧空禅院,就是慧褒弟子在其墓旁盖的屋舍。距离那禅院东边五里,是人们所说的华阳洞,因为它在华山南面而这样命名。距离山洞一百多步,有一座石碑倒在路旁,上面的文字已被剥蚀、损坏近乎磨灭,只有从勉强能认得出的地方还可以辨识出“花山”的字样。如今将“华”读为“华实”的“华”,大概是读音上的错误。
    
    由此向下的那个山洞平坦而空阔,有一股山泉从旁边涌出,在这里游览、题记的人很多,这就叫做“前洞”。经由山路向上五六里,有个洞穴,一派幽深的样子,进去便感到寒气逼人,探究它的深度,就是那些喜欢游险的人也未能走到尽头——这是人们所说的“后洞”。我与四个人打着火把走进去,进去越深,前进越困难,而所见到的景象也就更加奇妙。有个懈怠而想退出的伙伴说:“再不出去,火把就要熄灭了。”于是,只好都跟他退出来。我们走进去的深度,比起那些喜欢游险的人来,大概还不足十分之一,然而看看左右的石壁,来此而题记的人已经很少了。洞内更深的地方,大概来到的游人就更少了。当决定从洞内退出时,我的体力还足够前进,火把还能够继续照明。我们出洞以后,就有人埋怨那主张退出的人,我也后悔跟他出来,而未能极尽游洞的乐趣。
    
    对于这件事我有所感慨。古人观察天地、山川、草木、虫鱼、鸟兽,往往有所得益,是因为他们探究、思考深邃而且广泛。平坦而又近的地方,前来游览的人便多;危险而又远的地方,前来游览的人便少。但是世上奇妙雄伟、珍异奇特、非同寻常的景观,常常在那险阻、僻远少有人至的地方,所以,不是有意志的人是不能到达的。虽然有了志气,也不盲从别人而停止,但是体力不足的,也不能到达。有了志气与体力,也不盲从别人、有所懈怠,但到了那幽深昏暗而使人感到模糊迷惑的地方却没有必要的物件来支持,也不能到达。可是,力量足以达到目的而未能达到,在别人看来是可以讥笑的,在自己来说也是有所悔恨的;尽了自己的主观努力而未能达到,便可以无所悔恨,这难道谁还能讥笑吗?这就是我这次游山的收获。
    
    我对于那座倒地的石碑,又感叹古代刻写的文献未能存留,后世讹传而无人弄清其真相的事,怎么能说得完呢?这就是学者不可不深入思考并谨慎地援用资料的缘故。
    
    同游的四个人是:庐陵人萧君圭,字君玉;长乐人王回,字深甫;我的弟弟王安国,字平甫;王安上,字纯甫。
    
    至和元年七月,临川人王安石记。
    
\end{normalsize}



\chapter{与妻书}

\begin{normalsize}
    
    意映卿卿\footnote{〔卿卿〕旧时夫妻间的爱称,多用于丈夫称呼妻子。}如晤:吾今以此书与汝永别矣!吾作此书时,尚为世中一人;汝看此书时,吾已成为阴间一鬼。吾作此书,泪珠和笔墨齐下,不能书竟,而欲搁笔,又恐汝不察吾衷,谓吾忍舍汝而死,谓吾不知汝之不欲吾死也,故遂忍悲为汝言之。
    
    吾至爱汝!即此爱汝一念,使吾勇于就死也!吾自遇汝以来,常愿天下有情人都成眷属,然遍地腥云,满街狼犬,称心快意,几家能够?司马青衫\footnote{〔司马青衫〕白居易《琵琶行》中有“座中泣下谁最多?江州司马青衫湿”,比喻悲伤落泪。},吾不能学太上之忘情\footnote{〔太上〕圣人。}也。语云,仁者“老吾老以及人之老,幼吾幼以及人之幼”。吾充吾爱汝之心,助天下人爱其所爱,所以敢先汝而死,不顾汝也。汝体吾此心,于悲啼之余,亦以天下人为念,当亦乐牺牲吾身与汝身之福利,为天下人谋永福也。汝其勿悲。
    
    汝忆否四五年前某夕,吾尝语曰:“与使吾先死也,无宁汝先吾而死。”汝初闻言而怒,后经吾婉解,虽不谓吾言为是,而亦无辞相答。吾之意盖谓以汝之弱,必不能禁失吾之悲,吾先死留苦与汝,吾心不忍,故宁请汝先死,吾担悲也。嗟夫,谁知吾卒先汝而死乎!
    
    吾真不能忘汝也!回忆后街之屋,入门穿廊,过前后厅,又三四折有小厅,厅旁一室为吾与汝双棲之所。初婚三四个月,适冬之望日前后,窗外疏梅筛月影,依稀掩映,吾与汝並肩携手,低低切切,何事不语,何情不诉!及今思之,空余泪痕!又回忆六七年前,吾之逃家复归也,汝泣告我:“望今后有远行,必以告妾,妾愿随君行。”吾亦既许汝矣。前十余日回家,即欲乘便以此行之事语汝,及与汝相对,又不能启口;且以汝之有身也,更恐不胜悲,故惟日日呼酒买醉。嗟夫!当时余心之悲,盖不能以寸管形容之。
    
    吾诚愿与汝相守以死。第以今日事势观之,天灾可以死,盗贼可以死,瓜分之日可以死,奸官污吏虐民可以死,吾辈处今日之中国,国中无地无时不可以死!到那时使吾眼睁睁看汝死,或使汝眼睁睁看我死,吾能之乎!抑汝能之乎!即可不死,而离散不相见,徒使两地眼成穿而骨化石\footnote{〔骨化石〕指望夫石传说,多地皆有,大多为妇人伫立望夫归来,日久化而为石。},试问古来几曾见破镜能重圆?则较死为苦也。将奈之何?今日吾与汝幸双健;天下人人不当死而死,与不愿离而离者,不可数计。钟情如我辈者,能忍之乎?此吾所以敢率性就死不顾汝也!吾今死无余憾,国事成不成,自有同志者在。依新已五岁,转眼成人,汝其善抚之,使之肖我。汝腹中之物,吾疑其女也,女必像汝,吾心甚慰;或又是男,则亦教其以父志为志,则我死后,尚有二意洞\footnote{〔意洞〕林觉民的字。}在也,甚幸甚幸!吾家后日当甚贫,贫无所苦,清静过日而已。
    
    吾今与汝无言矣。吾居九泉之下遥闻汝哭声,当哭相和也。吾平日不信有鬼,今则又望其真有。今是人又言心电感应\footnote{〔心电感应〕指人死后心灵还有知觉,能与活人的精神、心情交相感应。}有道,吾亦望其言是实,则吾之死,吾灵尚依依旁汝也,汝不必以无侣悲。
    
    吾平生未尝以吾所志语汝,是吾不是处;然语之,又恐汝日日为吾担忧。吾牺牲百死而不辞,而使汝担忧,的的非吾所忍。吾爱汝至,所以为汝谋者惟恐未尽。汝幸而偶我,又何不幸而生今日中国!吾幸而得汝,又何不幸而生今日之中国!卒不忍独善其身。嗟夫!巾短情长,所未尽者,尚有万千,汝可以模拟得之。吾今不能见汝矣!汝不能舍吾,其时时于梦中得我乎!一恸!
    
    辛未\footnote{〔辛未〕应是“辛亥”,指公元1911年。}三月廿六夜四鼓,意洞手书。
    
    家中诸母皆通文,有不解处,望请其指教,当尽吾意为幸。
\end{normalsize}


\newpage

\textbf{译文}:

\vspace{1em}

\begin{normalsize}
    
    意映爱妻如见:我现在用这封信跟你永别了!我写这封信的时候,还是世上的一个人,你看到这封信的时候,我已经成为阴间的一个鬼。我写这封信时,泪珠和笔墨一起洒落下来,不忍写完而想搁笔,又担心你不能体察我的衷情,以为我忍心抛弃你而去死,以为我不了解你是多么希望我活下去,所以就强忍着悲痛给你写下去。
    
    我极其爱你!就是这爱你的念头,使我勇敢地走向死亡啊!我自从遇到你以来,常常希望普天下的“有情人”都能够结成恩爱夫妻;然而遍地是腥血、满街是狼犬,有几家能够称心快意地过日子呢?人民的灾难使我和白居易那样泪湿青衫,我不能学古代圣人那样忘情。古语说:有仁爱心肠的人“尊敬我家里的长辈,从而推广到尊敬别人家里的长辈;爱护我家里的儿女,从而推广到爱护别人家里的儿女”。我扩充一片爱你的心,去帮助天下人也能爱自己所爱的人,所以我果敢决定在你死以前先死,只好忍心丢下你而不顾了。你要体谅我的一片苦心,在哭泣之余,也从全国人民的幸福着想,一定会乐于牺牲我和你个人的幸福,去为全国同胞谋求永久的幸福。你千万不要悲伤。
    
    你记得吗?四五年前某个晚上,我曾经告诉你说:“与其使我先死,不如你比我先死。”你开始听了发怒,后来经过我委婉的解释,你虽然不认为我的话是对的,但也无言回答我。我的意思原是说凭你的纤弱,一定经受不住失掉我的悲痛,我先死把痛苦留给你,我是不忍心的,所以宁愿让你先死,我来担当一切苦难与悲痛。
    
    我确确实实不能忘记你啊!回忆后街上的家宅,进门,穿过长廊,经过前厅、后厅,再拐三四个弯,有个小厅,厅旁有个房间,就是我们夫妻住的地方。新婚后的三、四个月,恰巧是冬天,一个望日前后,窗外月光透过稀疏的梅枝,照射下来,就好像从筛于的孔眼里漏出一样,月色和梅影迷朦相映;我跟你肩并肩,手拉手,轻声细语,何事不谈,何情不诉?现在想起来,只留下满面泪痕。又回想起六、七年前,我离家归来,你哭着对我说:“希望你今后如有远行,一定事先告诉我,我愿意跟随你一起去。”我也答应了你。前十几天我回到家中,就想乘便把这次行动的事告诉你,等到跟你相对时,又忍张口,而且因为你已经怀孕,更加担心你经受不住悲痛,所以只有天天喝酒以求醉。唉!时我内心的悲痛,是不能用笔墨来形容的。
    
    我确实是希望跟你共同生活到死。但拿今天的形势看来,天灾能够造成死亡,盗贼能够造成死亡,国家被列强瓜分那天能够造成死亡,贪官污吏虐待平民百姓能够造成死亡,我们这代人身处今天的中国,国内每个地方,每时每刻,都可能造成死亡!到那个时候使我眼睁睁看你死,或者让你眼睁睁看我死,我能这样做么?还是你能这样做么?即使能够不死,而我们夫妻离散不能相会,白白地使两人望眼欲穿,化骨为石,试问,自古以来有几对夫妻离散而又重新团聚?生离是比死别更为痛苦的。该怎么办呢?今天我跟你有幸健在;全国人民中不当死而死、不愿分离而被迫分离的,多得不能用数字来计算。像我们这样感情浓挚的人,能忍看这种惨状吗?这就是我断然干脆地为革命而死、舍你不顾的原因。我现在为革命死毫无遗恨,国家大事成与不成自有同志们在。依新现已五岁,转眼就要成人,你可要好好抚育他,使他像我一样。你腹中怀着的孩子,我猜是个女孩,女孩一定像你,(如果那样)我的内心感到非常宽慰;或许又是个男孩,那么也要教育他,以父亲的志向为志向,那么,我死了以后还有两个林觉民呢,幸运极了,幸运极了!我家以后的生活肯定非常贫困;贫困不要紧,清静些过日子罢了。
    
    我要跟你说的话就这些。我在阴间远远地听到你的哭声,一定以哭相应和。我平时不相信有鬼,而今又希望它真有。现在有人提出死人与活人之间有心电感应,我也希望他们说的是事实,那么我死后,我的灵魂还依偎在你身旁,你不必因为失去伴侣而悲痛。
    
    我平日从没有把我的志向告诉你,是我不对的地方;然而告诉了你,又恐怕你天天为我担忧。我对于牺牲,即使是死一百次我都不会推辞,可是让你为此担忧,确确实实不是我能忍心的。我爱你到了极点,所以为你考虑的只怕不周到。你有幸嫁了我,又怎么不幸而生在今天的中国!我很幸运得到你,又怎么不幸而生在今天的中国!我们总不能忍心只图自己幸福。唉!方巾短小而情意深长,没有表达完的,还有万万千千,你能够想象到的。我现在不能见到你了,你舍不得我,大约会常常在梦里见到我吧!悲痛极了!
    
    辛亥年三月二十六日夜间四更时候,意洞亲手写。
    
    家里的伯母叔母们都通晓文字,有不明白的地方,希望去请她们指教,应当把我的心意完全领会了就好。
    
\end{normalsize}



\chapter{张衡传}

\begin{normalsize}
    
    张衡字平子,南阳西鄂\footnote{〔南洋西鄂〕南阳郡西鄂县,在今河南南阳。}人也。衡少善属文,游于三辅,因入京师\footnote{〔京师〕东汉首都洛阳(今河南洛阳市)},观太学\footnote{〔太学〕古代设在京城的全国最高学府,西汉武帝开始设立。},遂通五经\footnote{〔五经〕汉武帝时将《诗》《书》《礼》《易》《春秋》定名为“五经”。},贯六艺\footnote{〔六艺〕指礼、乐、射、御、书、数。}。虽才高于世,而无骄尚之情。常从容淡静,不好交接俗人。永元\footnote{〔永元〕东汉和帝(刘肇)的年号(公元79年至106年)。}中,举孝廉不行,连辟公府\footnote{〔公府〕三公的官署。东汉以太尉、司徒、司空为三公。}不就。时天下承平日久,自王侯以下,莫不逾侈。衡乃拟班固\footnote{〔班固〕字孟坚,东汉史学家、文学家。}《两都》\footnote{〔《两都》〕指《两都赋》,分《西都赋》《东都赋》。}作《二京赋》\footnote{〔《二京赋》〕指《西京赋》《东京赋》。},因以讽谏。精思傅会,十年乃成。大将军邓骘\footnote{〔邓骘〕东汉和帝邓皇后的哥哥,立安帝,以大将军的身份辅佐安帝管理政事。}奇其才,累召不应。
    
    衡善机巧,尤致思于天文、阴阳\footnote{〔阴阳〕这里指通过天象推算地理节气。}、历算\footnote{〔历算〕推算年月日和节气,泛指数学。}。安帝雅闻衡善术学,公车\footnote{〔公车〕汉代官署名称,设公车令。}特征拜郎中,再迁为太史令\footnote{〔太史令〕东汉时掌管天文历数的官,与西汉以前掌管天象历法兼有修史之责的太史令职责不完全相同。}。遂乃研核阴阳,妙尽璇玑\footnote{〔璇玑〕上古用于观测天象的仪器“璇玑玉衡”中旋转的部分。}之正,作浑天仪\footnote{〔浑天仪〕表示天象的仪器,类似天球仪,反映浑天说的理论。},著《灵宪》\footnote{〔《灵宪》〕张衡写的历法书,已佚。}《算罔》\footnote{〔《算罔》〕张衡写的数算书,已佚。}论,言甚详明。
    
    顺帝初,再转,复为太史令。衡不慕当世\footnote{〔当世〕指当政的权臣。},所居之官辄积年不徙。自去史职,五载复还。
    
    阳嘉\footnote{〔阳嘉〕东汉顺帝(刘保)的年号(公元132年至135年)。}元年,复造候风地动仪\footnote{〔候风地动仪〕一种感应地震的仪器,已失传。}。以精铜铸成,员径八尺\footnote{〔员径八尺〕圆直径八尺。},合盖隆起,形似酒尊,饰以篆文山龟鸟兽之形。中有都柱\footnote{〔都柱〕中心立柱,一根上大下小的柱子,哪个方向发生地震,柱子便倒向哪边。},傍行八道,施关发机。外有八龙,首衔铜丸,下有蟾蜍,张口承之。其牙机\footnote{〔牙机〕带齿的机械。}巧制,皆隐在尊中,覆盖周密无际。如有地动,尊则振龙,机发吐丸,而蟾蜍衔之。振声激扬,伺者因此觉知。虽一龙发机,而七首不动。寻其方面,乃知震之所在。验之以事,合契若神。自书典所记,未之有也。尝一龙机发而地不觉动,京师学者咸怪其无征。后数日驿\footnote{〔驿〕驿使,古时骑马在驿道上传递文书的人。}至,果地震陇西\footnote{〔陇西〕陇西郡,现在甘肃兰州市临洮县陇西县一带。},于是皆服其妙。自此以后,乃令史官记地动所从方起。
    
    时政事渐损,权移于下,衡因上疏陈事。后迁侍中,帝引在帷幄\footnote{〔帷幄〕室内悬挂的帐幕,天子居处必设帷幄。这里代指天子身边的决策中心。},讽议左右。尝问天下所疾恶者。宦官惧其毁己,皆共目之,衡乃诡对而出。阉竖恐终为其患,遂共谗之。衡常思图身之事,以为吉凶倚伏,幽微难明,乃作《思玄赋》以宣寄情志。
    
    永和\footnote{〔永和〕东汉顺帝的年号(公元136年至141年)。}初,出为河间相。时国王\footnote{〔国王〕指河间惠王刘政,汉章帝孙,按辈分是汉顺帝的叔叔。}骄奢,不遵典宪;又多豪右\footnote{〔豪右〕豪族大户,权势盛大的地方家族。},共为不轨。衡下车,治威严,整法度,阴知奸党名姓,一时收禽,上下肃然,称为政理。视事三年,上书乞骸骨\footnote{〔乞骸骨〕古代官吏因年老请求退职的一种说法。},征拜尚书\footnote{〔尚书〕东汉时是在宫廷中协助皇帝处理政务的官。}。年六十二,永和四年卒。
\end{normalsize}


\newpage

\textbf{译文}:

\vspace{1em}

\begin{normalsize}
    
    张衡,字平子,是南阳郡西鄂县人。张衡年轻时就擅长写文章,曾到“三辅”一带游学,趁机进了洛阳,在太学学习,于是通晓五经,贯通六艺。虽然才华比一般的人高,但并不因此而骄傲自大。(他)平时举止从容,态度平静,不喜欢与世俗之人交往。永元年间,他被推举为孝廉,却不应荐,屡次被公府征召,都没有就任。此时社会长期太平无事,从王公贵族到一般官吏,没有不过度奢侈的。张衡于是摹仿班固的《两都赋》写了《二京赋》,用它来(向朝廷)讽喻规劝。(这篇赋,他)精心构思润色,用了十年才完成。大将军邓骘认为他的才能出众,屡次征召他,他也不去应召。
    
    张衡善于器械制造方面的巧思,尤其在天文气象和历法的推算等方面很用心。汉安帝常听说他擅长术数方面的学问,命公车特地征召他,任命他为郎中,然后又迁升为太史令。于是,张衡就精心研究考核阴阳之学(包括天文气象历法诸种学问),精辟地研究出测天文仪器的正确道理,制作浑天仪,著成《灵宪》《算罔论》等书籍,论述极其详尽。
    
    (汉)顺帝初年,(张衡)又两次转任,又做了太史令之职。张衡不趋附当时的那些达官显贵,他所担任的官职,总是多年得不到提升。自他从太史令上离任后,过了五年,又回到这里。
    
    顺帝阳嘉元年,张衡又制造了候风地动仪。这个地动仪是用纯铜铸造的,直径有八尺,上下两部分相合盖住,中央凸起,样子像个大酒樽,外表用篆体文字和山龟鸟兽的图案装饰。内部中央有根粗大的铜柱,铜柱的周围伸出八条滑道,还装置着枢纽,用来拨动机件。外面有八条龙,龙口各含一枚铜丸,龙头下面各有一个蛤蟆,张着嘴巴,准备接住龙口吐出的铜丸。仪器的枢纽和机件制造得很精巧,都隐藏在酒尊形的仪器中,覆盖严密得没有一点缝隙。如果发生地震,仪器外面的龙就震动起来,机关发动,龙口吐出铜丸,下面的蛤蟆就把它接住。铜丸震击的声音清脆响亮,守候机器的人因此得知发生地震的消息。地震发生时只有一条龙的机关发动,另外七个龙头丝毫不动。按照震动的龙头所指的方向去寻找,就能知道地震的方位。用实际发生的地震来检验仪器,彼此完全相符,真是灵验如神。从古籍的记载中,还看不到曾有这样的仪器。有一次,一条龙的机关发动了,可是洛阳并没有感到地震,京城的学者都奇怪它这次没有应验。几天后,驿站上传送文书的人来了,证明果然在陇西地区发生地震,大家这才都叹服地动仪的绝妙。从此以后,朝廷就责成史官根据地动仪记载每次地震发生的方位。
    
    当时政治昏暗,中央权力向下转移,张衡于是给皇帝上书陈述这些事。后来被升为侍中,皇帝让他进皇宫,在皇帝左右,对国家的政事提意见。皇帝曾经向张衡问起天下人所痛恨的是谁。宦官害怕张衡说出他们,都给他使眼色,张衡于是没对皇帝说实话。但那些宦党终究害怕张衡成为祸患,于是一起诋毁他。张衡常常思谋自身安全的事,认为福祸相因,幽深微妙,难以看清,于是写了《思玄赋》表达和寄托自己的情思。
    
    (汉顺帝)永和初年,张衡调离京城,担任河间王的相。当时河间王骄横奢侈,不遵守制度法令;又有很多豪族大户,豪门大户他们一起胡作非为。张衡上任之后治理严厉,整饬法令制度,暗中探得奸党的姓名,一下子同时逮捕,拘押起来,于是上下敬畏恭顺,称赞政事处理得好。(张衡)在河间相位上任职三年,给朝廷上书,请求辞职回家,朝廷任命他为尚书。张衡活了六十二岁,于永和四年去世。
    
\end{normalsize}



\chapter{非攻}

\begin{normalsize}
    
    今有一人,入人园圃,窃其桃李,众闻则非之,上为政者得则罚之。此何也?以亏人自利也。至攘人犬豕鸡豚者,其不义又甚入人园圃窃桃李。是何故也?以亏人愈多。苟亏人愈多,其不仁兹甚,罪益厚。至入人栏厩,取人马牛者,其不仁义又甚攘人犬豕鸡豚。此何故也?以其亏人愈多。苟亏人愈多,其不仁兹甚,罪益厚。至杀不辜人也,拖其衣裘、取戈剑者,其不义又甚入人栏厩取人马牛。此何故也?以其亏人愈多。苟亏人愈多,其不仁兹甚,罪益厚。当此,天下\footnote{〔天下〕泛指当时周天子统治的地方,包括各诸侯国。}之君子\footnote{〔君子〕君王之子,指代有德的人。}皆知而非之,谓之不义。今至大为不义攻国,则弗知非,从而誉之,谓之义。此可谓知义与不义之别乎?
    
    杀一人,谓之不义,必有一死罪矣。若以此说往,杀十人,十重不义,必有十死罪矣;杀百人,百重不义,必有百死罪矣。当此,天下之君子皆知而非之,谓之不义。今至大为不义攻国,则弗知非,从而誉之,谓之义。情不知其不义也,故书其言以遗后世。若知其不义也,夫奚说书其不义以遗后世哉?
    
    今有人于此,少见黑曰黑,多见黑曰白,则必以此人为不知白黑之辩矣。少尝苦曰苦,多尝苦曰甘,则必以此人为不知甘苦之辩矣。今小为非,则知而非之;大为非攻国,则不知非,从而誉之,谓之义。此可谓知义与不义之辩乎?是以知天下之君子也,辩义与不义之乱也。
\end{normalsize}


\newpage

\textbf{译文}:

\vspace{1em}

\begin{normalsize}
    
    如果有一个人,进人家果园,偷人家桃李,大家听到就谴责他,上面执政的人捉获就惩罚他。这为什么呢?因为他损人利己。至于偷人家鸡犬大猪小猪的,比进人家果园偷桃李更不义。这是什么原故呢?因为他损人更多。如果损人越多,他越是不仁,罪越重。至于进人家牲口棚,牵走人家马牛的,这比偷人家鸡犬大猪小猪更不义。这是什么原故呢?因为他损人更多。如果损人越多,他越是不仁,罪越重。至于杀无辜的人,剥下人家的衣服皮袄,拿走戈剑,这比进人家牲口棚牵走马牛又更不义。这是什么原故呢?因为他损人更严重。如果损人越严重,他就越不仁,罪越大。对此,天下的君子都明白其中的道理并会认为它们不对,说这些是不义的。今天最不义的事,是进攻别国,却不知道谴责,反而称赞它,说它是义。这能说知道义与不义的分别吗?
    
    说杀一个人,是不义,一定有一人负死罪。如果照这个说法类推下去,杀十个人,十倍不义,必定有十倍死罪的罪过了;杀一百人,百倍不义,必定有百倍死罪的罪过了。对此,天下的君子都明白其中的道理,(都)认为这样做不对,(都)说这是不义的。今天最不义的事,是进攻别国,却不知道反对,反而称赞它,说它义。这大概是不知道进攻别国是不义的,所以把(进攻别国的)话记载下来传给后世。如果知道它是不义的,那还有什么理由记载不义的事传给后世呢?
    
    今天(要是)有人在这里,见一点黑说是黑,见一片黑却说是白,那么一定说这人是不知分辨黑白的了。尝一点苦说苦,尝多了苦却说是甜,那么一定说这个人是不知分辨苦甜的了。今天干小的坏事,知道后就谴责它;干大的坏事,攻打别国,就不知道谴责,跟着别人称赞它,说它正义。这能说知道分辨义与不义吗?由此可知世上的君子,分辨义与不义是多么混乱啊。
    
\end{normalsize}



\chapter{冯谖客孟尝君}

\begin{normalsize}
    
    齐人有冯谖者,贫乏不能自存,使人属孟尝君,愿寄食门下。孟尝君曰:“客何好?”曰:“客无好也。”曰:“客何能?”曰:“客无能也。”孟尝君笑而受之曰:“诺。”
    
    左右以君贱之也,食以草具。居有顷,倚柱弹其剑,歌曰:“长铗归来乎!食无鱼。”左右以告。孟尝君曰:“食之,比门下之鱼客。”居有顷,复弹其铗,歌曰:“长铗归来乎!出无车。”左右皆笑之,以告。孟尝君曰:“为之驾,比门下之车客。”于是乘其车,揭其剑,过其友曰:“孟尝君客我。”后有顷,复弹其剑铗,歌曰:“长铗归来乎!无以为家。”左右皆恶之,以为贪而不知足。孟尝君问:“冯公有亲乎?”对曰,“有老母。”孟尝君使人给其食用,无使乏。于是冯谖不复歌。
    
    后孟尝君出记,问门下诸客:“谁习计会,能为文收责于薛者乎?”冯谖署曰:“能。”孟尝君怪之,曰:“此谁也?”左右曰:“乃歌夫长铗归来者也。”孟尝君笑曰:“客果有能也,吾负之,未尝见也。”请而见之,谢曰:“文倦于事,愦于忧,而性懧愚,沉于国家之事,开罪于先生。先生不羞,乃有意欲为收责于薛乎?”冯谖曰:“愿之。”于是约车治装,载券契而行,辞曰:“责毕收,以何市而反?”孟尝君曰:“视吾家所寡有者。”
    
    驱而之薛,使吏召诸民当偿者,悉来合券。券遍合,起,矫命以责赐诸民,因烧其券。民称万岁。
    
    长驱到齐,晨而求见。孟尝君怪其疾也,衣冠而见之,曰:“责毕收乎?来何疾也!”曰:“收毕矣。”“以何市而反?”冯谖曰:“君云‘视吾家所寡有者’。臣窃计,君宫中积珍宝,狗马实外厩,美人充下陈。君家所寡有者,以义耳!窃以为君市义。”孟尝君曰:“市义奈何?”曰:“今君有区区之薛,不拊爱子其民,因而贾利之。臣窃矫君命,以责赐诸民,因烧其券,民称万岁。乃臣所以为君市义也。”孟尝君不说,曰:“诺,先生休矣!”
    
    后期年,齐王谓孟尝君曰:“寡人不敢以先王之臣为臣。”孟尝君就国于薛。未至百里,民扶老携幼,迎君道中。孟尝君顾谓冯谖:“先生所为文市义者,乃今日见之。”
    
    冯谖曰:“狡兔有三窟,仅得免其死耳;今君有一窟,未得高枕而卧也。请为君复凿二窟。”孟尝君予车五十乘,金五百斤,西游于梁,谓惠王曰:“齐放其大臣孟尝君于诸侯,诸侯先迎之者,富而兵强。”于是梁王虚上位,以故相为上将军,遣使者黄金千斤,车百乘,往聘孟尝君。冯谖先驱,诫孟尝君曰:“千金,重币也;百乘,显使也。齐其闻之矣。”梁使三反,孟尝君固辞不往也。
    
    齐王闻之,君臣恐惧,遣太傅赍黄金千斤、文车二驷,服剑一,封书,谢孟尝君曰:“寡人不祥,被于宗庙之祟,沉于谄谀之臣,开罪于君。寡人不足为也;愿君顾先王之宗庙,姑反国统万人乎!”冯谖诫孟尝君曰:“愿请先王之祭器,立宗庙于薛。”庙成,还报孟尝君曰:“三窟已就,君姑高枕为乐矣。”
    
    孟尝君为相数十年,无纤介之祸者,冯谖之计也。
\end{normalsize}


\newpage

\textbf{译文}:

\vspace{1em}

\begin{normalsize}
    
    齐国有一人叫冯谖,因为太穷而不能养活自己,便托人告诉孟尝君,表示意愿在他的门下寄居为食客。孟尝君问:“他有何爱好?”回答说:“没有什么爱好。”又问:“他有何才干?”回答说:“没什么才能。”孟尝君听了后笑了笑,但还是接受了他。
    
    旁边的人认为孟尝君看不起冯谖,就让他吃粗劣的饭菜。过了一段时间,冯谖倚着柱子弹着自己的剑,唱道:“长剑我们回去吧!没有鱼吃。”左右的人把这事告诉了孟尝君。孟尝君说:“让他吃鱼,按照中等门客的生活待遇。”又过了一段时间,冯谖弹着他的剑,唱道:“长剑我们回去吧!外出没有车子。”左右的人都取笑他,并把这件事告诉给孟尝君。孟尝君说:“给他车子,按照上等门客的生活待遇。”冯谖于是乘坐他的车,高举着他的剑,去拜访他的朋友,十分高兴地说:“孟尝君待我为上等门客。”此后不久,冯谖又弹着他的剑,唱道:“长剑我们回去吧!没有能力养家。”此时,左右的手下都开始厌恶冯谖,认为他贪得无厌。而孟尝君听说此事后问他:“冯公有双亲吗?”冯谖回答说:“家中有老母亲。”于是孟尝君派人供给他母亲吃用,不使她感到缺乏。于是从那之后,冯谖不再唱歌。
    
    后来,孟尝君出文告征询他的门客:“谁熟习会计的事?可以为我到薛地收取债务?”冯谖在本上署了自己的名,并签上一个“能”字。孟尝君见了名字感到很惊奇,问:“这是谁呀?”左右的人说:“就是唱那‘长铗归来’的人。”孟尝君笑道:“这位客人果真有才能,我亏待了他,还没见过面呢!”他立即派人请冯谖来相见,当面赔礼道:“我被琐事搞得精疲力竭,被忧虑搅得心烦意乱;加之我懦弱无能,整天埋在国家大事之中,以致怠慢了您,而您却并不见怪,倒愿意往薛地去为我收债,是吗?”冯谖回答道:“愿意去。”于是套好车马,整治行装,载上契约票据动身了。辞行的时候冯谖问:“债收完了,买什么回来?”
    
    冯谖赶着车到薛,派官吏把该还债务的百姓找来核验契据。核验完毕后,他假托孟尝君的命令,把所有的债款赏赐给欠债人,然后当场把债券烧掉。百姓都高呼“万岁”。
    
    冯谖赶着车,马不停蹄,直奔齐都,清晨就求见孟尝君。冯谖回得如此迅速,孟尝君感到很奇怪,立即穿好衣、戴好帽,去见他,问道:“债都收完了吗?怎么回得这么快?”冯谖说:“都收了。”“买什么回来了?”孟尝君问。冯谖回答道:“您曾说‘看我家缺什么’,我私下考虑您宫中积满珍珠宝贝,外面马房多的是猎狗、骏马,后庭多的是美女,您家里所缺的只不过是‘仁义’罢了,所以我用债款为您买了‘义’。”孟尝君道:“买仁义是怎么回事?”冯谖道:“现在您不过有块小小的薛邑,如果不抚爱百姓,视民如子,而用商贾之道向人民图利。因此我擅自假造您的命令,把债款赏赐给百姓,顺便烧掉了契据,以至百姓欢呼‘万岁’。这就是我用来为您买义的方式啊。”孟尝君听后很不快地说:“嗯,先生,退下吧。”
    
    过了一年,齐闵王对孟尝君说:“我可不敢把先王的臣子当作我的臣子。”孟尝君只好到他的领地薛去。还差百里未到,薛地的人民扶老携幼,都在路旁迎接孟尝君到来。孟尝君见此情景,回头看着冯谖道:“您为我买义,今天(我)见到了。”
    
    冯谖说:“狡猾机灵的兔子有三个洞才能免遭死患,现在您只有一个洞,还不能高枕无忧,请让我再去为您挖两个洞吧。”孟尝君应允了,就给了五十辆车子,五百斤金。冯谖往西到了魏国,他对惠王说:“现在齐国把他的大臣孟尝君放逐到国外去,哪位诸侯先迎住他,就可使自己的国家富庶强盛。”于是惠王把相位空出来,把原来的相国调为上将军,并派使者带着千斤金,百辆车子去聘请孟尝君。冯谖先赶车回去,告诫孟尝君说:“金千斤,这是很重的聘礼了;百辆车子,这算显贵的使臣了。齐国君臣大概听说这事了吧。”魏国的使臣往返了多次,孟尝君坚决推辞而不去魏国。
    
    齐王听到这些情况,君臣都惊慌害怕起来,就派遣太傅送一千斤金、两辆彩车、一把佩剑,写一封书信向孟尝君道歉说:“我太不慎重了,遭受祖宗降下的灾祸,又被那些逢迎讨好的臣子所迷惑,得罪了您。我是不值得您帮助的;希望您能顾念先王的宗庙,姑且回来统率全国人民吧!”冯谖提醒孟尝君说:“希望您向齐王请来先王传下的祭器,在薛地建立宗庙。”宗庙建成了,冯谖回来报告孟尝君说:“三个洞穴都已凿成了,您可以暂且高枕而卧,安心享乐了!”
    
    孟尝君做了几十年相,没有一点祸患,都是(由于)冯谖的计谋啊。
    
\end{normalsize}


\newpage

\textbf{注解}:

\vspace{-1em}

\begin{itemize}
    \setlength\itemsep{-0.2em}
    \item〔不敢以先王之臣为臣〕《史记·孟尝君列传》:“齐(湣)王惑于秦、楚之毁,以为孟尝君各高其主,而擅齐国之权,遂废孟尝君。”所谓“不敢以先王之臣为臣”是托辞。
    \item〔孟尝君予车五十乘,金五百斤,西游于梁,谓惠王曰〕《史记·孟尝君列传》记载为冯谖游说秦王。
\end{itemize}

\chapter{鸿门宴}

\begin{normalsize}
    
    行略定秦地,函谷关\footnote{〔函谷关〕现在河南灵宝县北崤山中,地势险要,是战国时秦国抵御外敌的重要关口。函谷关以东称为关东。}有兵守关,不得入。又闻沛公\footnote{〔沛公〕即刘邦,秦末随项梁反秦,后与项羽争天下,建立汉朝。}已破咸阳\footnote{〔咸阳〕秦朝都城,现在陕西咸阳市。},项羽\footnote{〔项羽〕项籍,字羽,楚国名将项燕之孙。秦二世初年起兵反秦,在巨鹿之战大破秦军主力,建立西楚,与刘邦争天下,最后在垓下之战兵败身死。}大怒,使当阳君\footnote{〔当阳君〕即英布,秦末汉初名将,项羽封其为九江王,后叛楚归汉。}等击关。项羽遂入,至于戏西\footnote{〔戏西〕戏水之西。戏水,出今陕西临潼县南骊山,北流入渭水。}。沛公军霸上\footnote{〔霸上〕即灞上,现在陕西西安市东,在灞水旁。},未得与项羽相见。沛公左司马曹无伤使人言于项羽曰:“沛公欲王关中,使子婴\footnote{〔子婴〕秦二世胡亥之子,被赵高立为秦王,刘邦入咸阳后投降刘邦,后被项羽杀死。}为相,珍宝尽有之。”项羽大怒,曰:“旦日飨士卒,为击破沛公军!”当是时,项羽兵四十万,在新丰鸿门\footnote{〔新丰鸿门〕现在陕西临潼县西北,秦时称骊邑。鸿门,在临潼县东。};沛公兵十万,在霸上。范增\footnote{〔范增〕项羽的谋士,被项羽尊称为“亚父”。}说项羽曰:“沛公居山东时,贪于财货,好美姬。今入关,财物无所取,妇女无所幸,此其志不在小。吾令人望其气,皆为龙虎,成五采,此天子气也。急击勿失!”
    
    楚左尹项伯者,项羽季父也,素善留侯张良\footnote{〔张良〕字子房,秦末汉初谋臣,韩国相国之孙,曾刺杀秦始皇。后投奔刘邦,被封为留侯。}。张良是时从沛公。项伯乃夜驰之沛公军,私见张良,具告以事,欲呼张良与俱去,曰:“毋从俱死也!”张良曰:“臣为韩王\footnote{〔韩王〕即韩成,韩国宗室,为韩国复国而随项梁反秦,项梁让张良辅佐韩成,韩成让张良辅佐刘邦。后被项羽猜忌杀害。}送沛公,沛公今事有急,亡去不义,不可不语。”
    
    良乃入,具告沛公。沛公大惊,曰:“为之奈何?”张良曰:“谁为大王为此计者?”曰:“鲰生说我曰:‘距关,毋内诸侯,秦地可尽王也’故听之。”良曰:“料大王士卒足以当项王乎?”沛公默然,曰:“固不如也,且为之奈何?”张良曰:“请往谓项伯,言沛公不敢背项王也。”沛公曰:“君安与项伯有故?”张良曰:“秦时与臣游,项伯杀人,臣活之;今事有急,故幸来告良。”沛公曰:“孰与君少长?”良曰:“长于臣。”沛公曰:“君为我呼入,吾得兄事之。”张良出,要项伯。项伯即入见沛公。沛公奉卮酒为寿,约为婚姻,曰:“吾入关,秋毫不敢有所近,籍吏民,封府库,而待将军。所以遣将守关者,备他盗之出入与非常也。日夜望将军至,岂敢反乎!愿伯具言臣之不敢倍德也。”项伯许诺,谓沛公曰:“旦日不可不蚤自来谢项王!”沛公曰:“诺。”于是项伯复夜去,至军中,具以沛公言报项王,因言曰:“沛公不先破关中,公岂敢入乎?今人有大功而击之,不义也。不如因善遇之。”项王许诺。
    
    沛公旦日从百余骑来见项王,至鸿门,谢曰:“臣与将军戮力而攻秦,将军战河北,臣战河南,然不自意能先入关破秦,得复见将军于此。今者有小人之言,令将军与臣有郤。”项王曰:“此沛公左司马曹无伤言之,不然,籍何以至此?”项王即日因留沛公与饮。项王、项伯东向坐;亚父南向坐——亚父者,范增也;沛公北向坐,张良西向侍。范增数目项王,举所佩玉玦以示之者三,项王默然不应。范增起,出,召项庄\footnote{〔项庄〕项羽族人。},谓曰:“君王为人不忍。若入前为寿,寿毕,请以剑舞,因击沛公于坐,杀之。不者,若属皆且为所虏!”庄则入为寿。寿毕,曰:“君王与沛公饮,军中无以为乐,请以剑舞。”项王曰:“诺。”项庄拔剑起舞,项伯亦拔剑起舞,常以身翼蔽沛公,庄不得击。
    
    于是张良至军门见樊哙\footnote{〔樊哙〕刘邦属下猛将。}。樊哙曰:“今日之事何如?”良曰:“甚急!今者项庄拔剑舞,其意常在沛公也。”哙曰:“此迫矣!臣请入,与之同命!”哙即带剑拥盾入军门。交戟之卫士欲止不内,樊哙侧其盾以撞,卫士仆地,哙遂入,披帷西向立,瞋目视项王,头发上指,目眦尽裂。项王按剑而跽曰:“客何为者?”张良曰:“沛公之参乘樊哙者也。”项王曰:“壮士!赐之卮酒。”则与斗卮酒。哙拜谢,起,立而饮之。项王曰:“赐之彘肩!”则与一生彘肩。樊哙覆其盾于地,加彘肩上,拔剑切而啖之。项王曰:“壮士!能复饮乎?”樊哙曰:“臣死且不避,卮酒安足辞!夫秦王有虎狼之心,杀人如不能举,刑人如恐不胜,天下皆叛之。怀王与诸将约曰:‘先破秦入咸阳者王之’,今沛公先破秦入咸阳,毫毛不敢有所近,封闭宫室,还军霸上,以待大王来。故遣将守关者,备他盗出入与非常也。劳苦而功高如此,未有封侯之赏,而听细说,欲诛有功之人。此亡秦之续耳,窃为大王不取也!”项王未有以应,曰:“坐!”樊哙从良坐。坐须臾,沛公起如厕,因招樊哙出。
    
    沛公已出,项王使都尉陈平召沛公。沛公曰:“今者出,未辞也,为之奈何?”樊哙曰:“大行不顾细谨,大礼不辞小让。如今人方为刀俎,我为鱼肉,何辞为?”于是遂去。乃令张良留谢。良问曰:“大王来何操?”曰:“我持白璧一双,欲献项王,玉斗一双,欲与亚父。会其怒,不敢献。公为我献之。”张良曰:“谨诺。”当是时,项王军在鸿门下,沛公军在霸上,相去四十里。沛公则置车骑,脱身独骑,与樊哙、夏侯婴、靳强、纪信\footnote{〔樊哙、夏侯婴、靳强、纪信〕都是刘邦的亲信。}等四人持剑盾步走,从郦山下,道芷阳\footnote{〔芷阳〕秦县名,在今陕西西安市东。}间行。沛公谓张良曰:“从此道至吾军,不过二十里耳。度我至军中,公乃入。”
    
    沛公已去,间至军中。张良入谢,曰:“沛公不胜桮杓,不能辞。谨使臣良奉白璧一双,再拜献大王足下;玉斗一双,再拜奉大将军足下。”项王曰:“沛公安在?”良曰:“闻大王有意督过之,脱身独去,已至军矣。”项王则受璧,置之坐上。亚父受玉斗,置之地,拔剑撞而破之,曰:“唉!竖子不足与谋!夺项王天下者,必沛公也!吾属今为之虏矣!”沛公至军,立诛杀曹无伤。
\end{normalsize}


\newpage

\textbf{译文}:

\vspace{1em}

\begin{normalsize}
    
    楚军将要攻取关中,到达函谷关,有(刘邦的)军队把守,不能进入。又听说刘邦已经攻破咸阳,项羽非常恼火,就派遣当阳君等攻打函谷关。于是项羽进入关中,到达戏水之西。沛公驻军霸上,还没有跟项羽见面。沛公的左司马曹无伤派人对项羽说:“沛公打算在关中称王,任命子婴为国相,珍宝全部占有它。”项羽大怒道:“明天犒劳士兵,给我去打垮沛公的部队!”在这时,项羽的军队有四十万,驻扎在新丰鸿门;沛公的军队有十万,驻扎在霸上。范增劝说项羽道:“沛公住在崤山以东时,贪图财货,喜欢漂亮的女人。如今入了关,财物什么都不拿,也不迷恋女色,这样看来,他的野心不小。我叫人观望他那里的气运,都是龙虎的形状,呈现五彩的颜色,这是天子的气运呀!赶快攻打,不要失去机会。”
    
    楚国的左尹项伯,是项羽的叔父,一向同留侯张良交好。张良这时正跟随着刘邦。项伯就连夜骑马跑到刘邦的军营,私下会见张良,把事情详细地告诉了他,想叫张良和他一起离开,说:“不要和(刘邦)他们一起死了。”张良说:“我是韩王派给沛公的人,现在沛公遇到危急的事,逃走是不守信义的,不能不告诉他。”
    
    张良就进去,(把情况)详细告诉刘邦。刘邦大吃一惊,说:“对这件事怎么办?”张良说:“谁替大王作出这个计策的?”(刘邦)回答说:“浅陋无知的小人劝我说:‘把守住函谷关,不要让诸侯进来,秦国所有的地盘都可以由你称王了’所以(我)听信了他的话。”张良说:“估计大王的军队能够抵挡住项王的军队吗?”刘邦沉默(一会儿)说:“本来不如人家,将怎么办呢?”张良说:“请让我亲自去告诉项伯,说刘邦不敢背叛项王。”刘邦说:“你怎么和项伯有交情的?”张良说:“在秦朝的时候,项伯和我有交往,项伯杀了人,我救活了他;现在有了紧急的情况,所以幸亏他来告诉我。”刘邦说:“他你年龄,谁大谁小?”张良说:“他比我大。”刘邦说:“你替我(把他)请进来,我得用对待兄长的礼节待他。”张良出去,邀请项伯。项伯随即进来见刘邦。刘邦就奉上一杯酒为项伯祝福,(并)约定为亲家,说:“我进入关中,极小的财物都不敢沾染,登记官吏,人民,封闭了(收藏财物的)府库,以等待将军(的到来)。所以派遣官兵去把守函谷关的原因是,为了防备其他盗贼进来和意外的变故。日日夜夜盼望着将军的到来,怎么敢反叛呢!希望你(对项王)详细地说明,我是不敢忘恩负义的。”项伯答应了,跟刘邦说:“明天你不能不早些来亲自向项王谢罪!”刘邦说:“好。”于是项伯又连夜离开,回到(项羽)军营里,详细地把刘邦的话报告项王,趁机说:“刘邦不先攻破关中,您怎么敢进来呢?现在人家有大功(你)却要打人家,这是不仁义的。不如就趁机友好地款待他。”项王答应了。
    
    刘邦第二天带领一百多人马来见项羽,到达鸿门,谢罪说:“我和将军合力攻打秦国,将军在黄河以北作战。我在黄河以南作战,然而自己没有料想到能够先入关攻破秦国,能够在这里再看到将军您。现在有小人的流言,使将军和我有了隔阂……”项羽说:“这是你左司马曹无伤说的,不是这样的话,我怎么会这样呢?”项羽当天就留刘邦同他一起饮酒。项羽、项伯面向东坐;亚父面向南坐——亚父这个人,就是范增;刘邦面向北坐;张良面向西陪坐。范增多次使眼色给项羽,举起(他)所佩带的玉玦向项羽示意多次,项羽默默地没有反应。范增站起来,出去召来项庄,对项庄说:“君王为人不狠心。你进去上前祝酒,祝酒完了,请求舞剑助兴,趁机把刘邦击倒在座位上,杀掉他。不然的话,你们都将被他所俘虏!”项庄就进去祝酒。祝酒完了,说:“君王和沛公饮酒,军营里没有什么可以娱乐,请让我用舞剑助兴吧。”项羽说:“好。” 项庄就拔出剑舞起来,项伯也拔出剑舞起来,常常像鸟张开翅膀一样,用身体掩护刘邦,项庄得不到(机会)刺杀(刘邦)。
    
    于是张良到军营门口找樊哙。樊哙问:“今天的事情怎么样?”张良说:“很危急!现在项庄拔剑起舞,他的意图常在沛公身上啊!”樊哙说:“形势紧迫啊!请让我进去,跟他同生死!”于是樊哙拿着剑,持着盾牌,冲入军门。持戟交叉守卫军门的卫士想阻止他进去,樊哙侧着盾牌撞去,卫士跌倒在地上,樊哙就进去了,掀开帷帐朝西站着,瞪着眼睛看着项王,头发直竖起来,眼角都裂开了。项王握着剑挺起身问:“客人是干什么的?”张良说:“是沛公的参乘樊哙。”项王说:“壮士!赏他一杯酒。”左右就递给他一大杯酒。樊哙拜谢后,起身,站着把酒喝了。项王又说:“赏他一条猪的前腿。”于是给了他一个生的猪前腿。樊哙把他的盾牌扣在地上,把猪的前腿放在盾上,拔出剑来切着吃。项王说:“壮士!还能喝酒吗?”樊哙说:“我死都不怕,一杯酒有什么可推辞的?秦王有虎狼一样的心肠,杀人惟恐不能杀尽,惩罚人惟恐不能用尽酷刑,所以天下人都背叛他。怀王曾和诸将约定:‘先打败秦军进入咸阳的人封作王’,现在沛公先打败秦军进了咸阳,一点儿东西都不敢动用,封闭了宫室,军队退回到霸上,等待大王到来。特意派遣将领把守函谷关,是为了防备其他的盗贼和意外的变故。这样劳苦功高,没有得到封侯的赏赐,反而听信小人的谗言,想杀有功的人。这只是灭亡了的秦朝的继续罢了,我以为大王不应该采取这种做法!”项王没有话回答,说:“坐!”樊哙挨着张良坐下。坐了一会儿,刘邦起身上厕所,趁机把樊哙叫了出来。
    
    刘邦出去后,项王派都尉陈平去叫刘邦。刘邦说:“现在出来,还没有告辞,这该怎么办?”樊哙说:“做大事不必顾及小节,有大礼节不回避小的责备。现在人家好比是菜刀和砧板,我们好比是鱼和肉,为什么要告辞呢?”于是就决定离去。刘邦就让张良留下来道歉。张良问:“大王来时带了什么东西?”刘邦说:“我带了一对玉璧,想献给项王;一双玉斗,想送给亚父。正碰上他们发怒,不敢奉献。你替我把它们献上吧。”张良说:“遵命。”这时候,项王的军队驻在鸿门,刘邦的军队驻在霸上,相距四十里。刘邦就留下车辆和随从人马,独自骑马脱身,和樊哙、夏侯婴、靳强、纪信四人拿着剑和盾牌徒步逃跑,从郦山脚下,取道芷阳,秘密地走。刘邦对张良说:“从这条路到我们军营,不过二十里罢了。你估计我回到军营里了,再进去。”
    
    刘邦离去后,从小路回到军营里。张良进去道歉,说:“沛公不胜酒力,不能当面告辞。让我奉上白璧一双,拜两拜敬献给大王;玉斗一双,拜两拜献给大将军。”项王说:“沛公在哪里?”张良说:“听说大王有意要责备他,脱身独自离开,已经回到军营了。”项王就接受了玉璧,把它放在座位上。亚父接过玉斗,放在地上,拔出剑来敲碎了它,说:“唉!这小子不值得和他共谋大业!夺走项王天下的一定是沛公!我们这些人就要被他俘虏了!”刘邦回到军营,立即杀了曹无伤。
    
\end{normalsize}



\chapter{兰亭集序}

\begin{normalsize}
    
    永和九年\footnote{〔永和〕晋穆帝(司马聃)的年号(公元345至356年)。},岁在癸丑\footnote{〔癸丑〕干支纪年的方法。},暮春之初,会于会稽山阴\footnote{〔会稽山阴〕现在浙江绍兴越城区。}之兰亭,修禊事也\footnote{〔禊事〕古代习俗。每年农历三月上旬的巳日,人们群聚于水滨嬉戏洗濯,以祓除不祥和求福。}。群贤毕至,少长咸集。此地有崇山峻岭,茂林修竹;又有清流激湍,映带左右,引以为流觞曲水,列坐其次。虽无丝竹管弦之盛,一觞一咏,亦足以畅叙幽情。
    
    是日也,天朗气清,惠风和畅。仰观宇宙之大,俯察品类之盛,所以游目骋怀,足以极视听之娱,信可乐也。
    
    夫人之相与,俯仰一世,或取诸怀抱,悟言一室之内;或因寄所托,放浪形骸之外。虽趣舍万殊,静躁不同,当其欣于所遇,暂得于己,快然自足,不知老之将至。及其所之既倦,情随事迁,感慨系之矣。向之所欣,俯仰之间,已为陈迹,犹不能不以之兴怀。况修短随化,终期于尽。古人云:“死生亦大矣。”岂不痛哉!
    
    每览昔人兴感之由,若合一契,未尝不临文嗟悼,不能喻之于怀。固知一死生为虚诞,齐彭殇为妄作。后之视今,亦犹今之视昔。悲夫!故列叙时人,录其所述。虽世殊事异,所以兴怀,其致一也。后之览者,亦将有感于斯文。
\end{normalsize}


\newpage

\textbf{译文}:

\vspace{1em}

\begin{normalsize}
    
    永和九年是癸丑年,三月上旬,我们聚在会稽郡山阴城的兰亭行春禊。诸多贤士能人都汇聚到这里,年长、年少者都聚集在这里。兰亭这个地方有高峻的山峰,茂盛高密的树林和竹丛;又有清澈激荡的水流,在亭子的左右辉映环绕,我们把水引来作为漂传酒杯的环形渠水,排列坐在曲水旁边。虽然没有管弦齐奏的盛况,但喝着酒作着诗,也足够来畅快表达幽深内藏的感情了。
    
    这一天,天气晴朗,和风习习。抬头纵观广阔的天空,俯看观察大地上繁多的万物,用来舒展眼力,开阔胸怀,足够来极尽视听的欢娱,实在很快乐。
    
    人与人相互交往,很快便度过一生。有的人在室内畅谈自己的胸怀抱负;有的人就着自己所爱好的事物,寄托自己的情怀,不受约束,放纵无羁的生活。虽然各有各的爱好,安静与躁动各不相同,但当他们对所接触的事物感到高兴时,一时感到自得,感到高兴和满足,竟然不知道衰老将要到来。等到对于自己所喜爱的事物感到厌倦,心情随着当前的境况而变化,感慨随之产生了。过去所喜欢的东西,转瞬间,已经成为旧迹,尚且不能不因为它引发心中的感触,况且寿命长短,听凭造化,最后归结于消灭。古人说:“死生毕竟是件大事啊。”怎么能不让人悲痛呢?
    
    每当我看到前人兴怀感慨的原因,与我所感叹的好像符契一样相合,没有不面对着他们的文章而嗟叹感伤的,在心里又不能清楚地说明。本来知道把生死等同的说法是不真实的,把长寿和短命等同起来的说法是妄造的。后人看待今人,也就像今人看待前人。可悲呀!所以一个一个记下当时与会的人,录下他们所作的诗篇。纵使时代变了,事情不同了,但触发人们情怀的原因,他们的思想情趣是一样的。后世的读者,也将对这次集会的诗文有所感慨。
    
\end{normalsize}


\newpage

\textbf{注解}:

\vspace{-1em}

\begin{itemize}
    \setlength\itemsep{-0.2em}
    \item〔不知老之将至〕《论语·述而》:“其为人也,发愤忘食,乐以忘忧,不知老之将至云尔。”
    \item〔固知一死生为虚诞,齐彭殇为妄作〕《庄子·德充符》:“老聃曰:‘胡不直使彼以死生为一条,以可不可为一贯者,解其桎梏,其可乎?’”《庄子·齐物论》:“莫寿于殇子,而彭祖为夭。”这里王羲之借老庄之言指代当时喜欢清谈玄学,崇尚黄老之说,以隐逸避世为荣,以出世奋斗为耻的风气,认为这种思想是虚妄的。
\end{itemize}

\chapter{廉颇蔺相如列传}

\begin{normalsize}
    
    廉颇者,赵之良将也。赵惠文王十六年\footnote{〔赵惠文王十六年〕公元前283年。},廉颇为赵将伐齐,大破之,取阳晋\footnote{〔阳晋〕齐国城邑,现在山东菏泽西北。},拜为上卿\footnote{〔上卿〕战国时期诸侯国大臣中最高的官位。},以勇气闻于诸侯。蔺相如者,赵人也,为赵宦者令缪贤舍人。
    
    赵惠文王时,得楚和氏璧\footnote{〔和氏璧〕战国时著名的玉璧,是楚人卞和发现的,故名和氏璧。}。秦昭王\footnote{〔秦昭王〕即秦昭襄王(公元前306年至公元前251年在位)。}闻之,使人遗赵王书,愿以十五城请易璧。赵王与大将军廉颇诸大臣谋:欲予秦,秦城恐不可得,徒见欺;欲勿予,即患秦兵之来。计未定,求人可使报秦者,未得。宦者令缪贤曰:“臣舍人蔺相如可使。”王问:“何以知之?”对曰:“臣尝有罪,窃计欲亡走燕,臣舍人相如止臣,曰:‘君何以知燕王?”臣语曰:‘臣尝从大王与燕王会境上,燕王私握臣手,曰“愿结友”,以此知之,故欲往。”相如谓臣曰:‘夫赵强而燕弱,而君幸于赵王,故燕王欲结于君。今君乃亡赵走燕,燕畏赵,其势必不敢留君,而束君归赵矣。君不如肉袒伏斧质请罪,则幸得脱矣。”臣从其计,大王亦幸赦臣。臣窃以为其人勇士,有智谋,宜可使。”
    
    于是王召见,问蔺相如曰:“秦王以十五城请易寡人之璧,可予不?”相如曰:“秦强而赵弱,不可不许。”王曰:“取吾璧,不予我城,奈何?”相如曰:“秦以城求璧而赵不许,曲在赵。赵予璧而秦不予赵城,曲在秦。均之二策,宁许以负秦曲。”王曰:“谁可使者?”相如曰:“王必无人,臣愿奉璧往使。城入赵而璧留秦;城不入,臣请完璧归赵。”赵王于是遂遣相如奉璧西入秦。
    
    秦王坐章台见相如,相如奉璧奏秦王。秦王大喜,传以示美人及左右,左右皆呼万岁。相如视秦王无意偿赵城,乃前曰:“璧有瑕,请指示王。”王授璧,相如因持璧却立,倚柱,怒发上冲冠,谓秦王曰:“大王欲得璧,使人发书至赵王,赵王悉召群臣议,皆曰‘秦贪,负其强,以空言求璧,偿城恐不可得”,议不欲予秦璧。臣以为布衣之交尚不相欺,况大国乎!且以一璧之故逆强秦之欢,不可。于是赵王乃斋戒五日,使臣奉璧,拜送书于庭。何者?严大国之威以修敬也。今臣至,大王见臣列观,礼节甚倨;得璧,传之美人,以戏弄臣。臣观大王无意偿赵王城邑,故臣复取璧。大王必欲急臣,臣头今与璧俱碎于柱矣!”
    
    相如持其璧睨柱,欲以击柱。秦王恐其破璧,乃辞谢固请,召有司案图,指从此以往十五都予赵。相如度秦王特以诈详为予赵城,实不可得,乃谓秦王曰:“和氏璧,天下所共传宝也,赵王恐,不敢不献。赵王送璧时,斋戒五日,今大王亦宜斋戒五日,设九宾于廷,臣乃敢上璧。”秦王度之,终不可强夺,遂许斋五日,舍相如广成传。相如度秦王虽斋,决负约不偿城,乃使其从者衣褐,怀其璧,从径道亡,归璧于赵。
    
    秦王斋五日后,乃设九宾礼于廷,引赵使者蔺相如。相如至,谓秦王曰:“秦自缪公\footnote{〔缪公〕即秦穆公(前659年至前621年在位)。}以来二十馀君,未尝有坚明约束者也。臣诚恐见欺于王而负赵,故令人持璧归,间至赵矣。且秦强而赵弱,大王遣一介之使至赵,赵立奉璧来。今以秦之强而先割十五都予赵,赵岂敢留璧而得罪于大王乎?臣知欺大王之罪当诛,臣请就汤镬,唯大王与群臣孰计议之。”秦王与群臣相视而嘻。左右或欲引相如去,秦王因曰:“今杀相如,终不能得璧也,而绝秦赵之欢,不如因而厚遇之,使归赵,赵王岂以一璧之故欺秦邪!”卒廷见相如,毕礼而归之。
    
    相如既归,赵王以为贤大夫使不辱于诸侯,拜相如为上大夫。秦亦不以城予赵,赵亦终不予秦璧。
    
    其后秦伐赵,拔石城\footnote{〔石城〕即石邑,现在河北石家庄一带。}。明年,复攻赵,杀二万人。
    
    秦王使使者告赵王,欲与王为好会于西河外渑池\footnote{〔渑池〕现在河南渑池县。}。赵王畏秦,欲毋行。廉颇、蔺相如计曰:“王不行,示赵弱且怯也。”赵王遂行,相如从。廉颇送至境,与王诀曰:“王行,度道里会遇之礼毕,还,不过三十日。三十日不还,则请立太子为王,以绝秦望。”王许之。
    
    遂与秦王会渑池。秦王饮酒酣,曰:“寡人窃闻赵王好音,请奏瑟。”赵王鼓瑟。秦御史前书曰“某年月日,秦王与赵王会饮,令赵王鼓瑟”。蔺相如前曰:“赵王窃闻秦王善为秦声,请奏盆缻秦王,以相娱乐。”秦王怒,不许。于是相如前进缻,因跪请秦王。秦王不肯击缻。相如曰:“五步之内,相如请得以颈血溅大王矣!”左右欲刃相如,相如张目叱之,左右皆靡。于是秦王不怿,为一击缻。相如顾召赵御史书曰“某年月日,秦王为赵王击缻”。秦之群臣曰:“请以赵十五城为秦王寿。”蔺相如亦曰:“请以秦之咸阳为赵王寿。”秦王竟酒,终不能加胜于赵。赵亦盛设兵以待秦,秦不敢动。
    
    既罢归国,以相如功大,拜为上卿,位在廉颇之右。
    
    廉颇曰:“我为赵将,有攻城野战之大功,而蔺相如徒以口舌为劳,而位居我上,且相如素贱人,吾羞,不忍为之下。”宣言曰:“我见相如,必辱之。”相如闻,不肯与会。相如每朝时,常称病,不欲与廉颇争列。已而相如出,望见廉颇,相如引车避匿。于是舍人相与谏曰:“臣所以去亲戚而事君者,徒慕君之高义也。今君与廉颇同列,廉君宣恶言而君畏匿之,恐惧殊甚,且庸人尚羞之,况于将相乎!臣等不肖,请辞去。”
    
    蔺相如固止之,曰:“公之视廉将军孰与秦王?”曰:“不若也。”相如曰:“夫以秦王之威,而相如廷叱之,辱其群臣,相如虽驽,独畏廉将军哉?顾吾念之,强秦之所以不敢加兵于赵者,徒以吾两人在也。今两虎共斗,其势不俱生。吾所以为此者,以先国家之急而后私仇也。”廉颇闻之,肉袒负荆,因宾客至蔺相如门谢罪。曰:“鄙贱之人,不知将军宽之至此也!”卒相与欢,为刎颈之交。
\end{normalsize}


\newpage

\textbf{译文}:

\vspace{1em}

\begin{normalsize}
    
    廉颇是赵国优秀的将领。赵惠文王十六年,时为赵国将军的廉颇率领赵军征讨齐国,大败齐军,夺取了阳晋,晋升为上卿,从此他以英勇善战闻名于诸侯各国。蔺相如,赵国人,他是赵国的宦官首领缪贤家的门客。
    
    赵惠文王在位的时候,得到了楚人的和氏璧。秦昭王听说了这件事,就派人给赵王送来一封书信,表示愿意用十五座城池交换和氏璧。赵王同大将军廉颇以及诸大臣们商量:如果把宝玉给了秦国,秦国的城邑恐怕不可能得到,白白地被欺骗;如果不给他,又恐怕秦国来攻打。尚未找到合适的解决办法,寻找一个能到秦国去回复的使者,也未能找到。宦官令缪贤说:“我的门客蔺相如可以出使。”赵王问:“你是怎么知道他可以出使的?”缪贤回答说:“微臣曾犯过罪,私下打算逃亡到燕国去,我的门客相如劝阻我不要去,问我说:‘您怎么会了解燕王呢?”我对他说:‘我曾随从大王在国境上与燕王会见,燕王私下握住我的手,说“情愿跟您交个朋友”,因此了解他,所以打算投奔燕王。”相如对我说:‘赵国强,燕国弱,而您受宠于赵王,所以燕王想要和您结交。现在您是从赵国逃亡到燕国去,燕国惧怕赵国,这种形势下燕王必定不敢收留您,而且还会把您捆绑起来送回赵国。您不如脱掉上衣,露出肩背,伏在斧刃之下请求治罪,这样也许侥幸被赦免。”臣听从了他的意见,大王也开恩赦免了为臣。为臣私下认为这人是个勇士,有智谋,应该可以出使。”
    
    于是赵王立即召见,问蔺相如:“秦王用十五座城池请求交换我的和氏璧,能不能给他?”相如说:“秦国强,赵国弱,不能不答应它。”赵王说:“得了我的宝璧,不给我城邑,怎么办?”相如说:“秦国请求用城换璧,赵国如不答应,赵国理亏。赵国给了璧而秦国不给赵国城邑,秦国理亏。衡量一下两种对策,宁可答应它,使秦国来承担理亏的责任。”赵王说:“谁可以前往?”相如说:“大王如果无人可派,臣愿捧护宝璧前往出使。城邑归属赵国了,就把宝璧留给秦国;城邑不能归赵国,我一定把和氏璧完好地带回赵国。”赵王于是就派遣蔺相如带好和氏璧,西行入秦。
    
    秦王坐在章台上接见蔺相如,相如捧璧呈献给秦王。秦王非常高兴,把宝璧传着给妻妾和左右侍从看,左右都高呼万岁。相如看出秦王没有用城邑抵偿赵国的意思,便走上前去说:“璧上有个小斑点,让我指给大王看。”秦王把璧交给他,相如于是手持璧玉退后几步靠在柱子上,怒发冲冠,对秦王说:“大王想得到宝璧,派人送信给赵王,赵王召集全体大臣商议,大家都说:‘秦国贪得无厌,倚仗它的强大,想用空话得到宝璧,说给我们城邑恐怕不可能”,商议的结果是不想把宝璧给秦国。但是我认为平民百姓之间的交往尚且互相不欺骗,更何况是大国之间呢!况且为了一块璧玉的缘故就使强大的秦国不高兴,也是不应该的。于是赵王斋戒了五天,派我捧着宝璧,在朝廷上将国书交给我。为什么要这样呢?是尊重大国的威望以表示敬意呀。如今我来到贵国,大王却在一般的台观上接见我,礼节十分傲慢;得到宝璧后,传给姬妾们观看,这样来戏弄我。我观察大王没有给赵王十五城的诚意,所以我又取回宝璧。大王如果一定要逼我,我的头今天就同宝璧一起在柱子上撞碎!”
    
    相如手持宝璧,斜视庭柱,就要向庭柱上撞去。秦王怕他把宝璧撞碎,便向他道歉,坚决请求他不要如此,并召来有司查看地图,指明从某地到某地的十五座城邑都给赵国。相如估计秦王只不过用欺诈手段假装给赵国城邑,实际上赵国根本不可能得到,于是就对秦王说:“和氏璧是天下公认的宝物,赵王惧怕贵国,不敢不奉献出来。赵王送璧之前,斋戒了五天,如今大王也应斋戒五天,在殿堂上安排九宾大典,我才敢献上宝璧。”秦王估量,毕竟不可能强力夺取,于是就答应斋戒五天,把相如安置在广成宾馆。相如估计秦王虽然答应斋戒,也必定背约不给城邑,便派他的随从穿上粗麻布衣服,怀中藏好宝璧,从小路逃出,把宝璧送回赵国。
    
    秦王斋戒五天后,就在殿堂上安排了九宾的大典礼,宴请赵国使者蔺相如。相如来到后,对秦王说:“秦国从穆公以来的二十余位君主,从没有一个是能切实遵守信约的。我实在是害怕被大王欺骗而对不起赵王,所以派人带着宝璧回去,已从小路回到赵国了。况且秦国强大赵国弱小,大王派遣一位使臣到赵国,赵国立即就会把璧送来。如今凭着秦国的强大,先把十五座城邑割让给赵国,赵国哪里敢留下宝璧而得罪大王呢?我知道欺骗大王是应该被诛杀的,我愿意接受汤镬之刑,只希望大王和各位大臣从长计议此事。”秦王和群臣面面相觑,发出苦笑之声。侍从有人要拉相如去受刑,秦王趁机说:“如今杀了相如,终归还是得不到宝璧,反而破坏了秦赵两国的交情,不如趁此好好款待他,放他回到赵国,赵王难道会为了一块璧玉的缘故而欺骗秦国吗!”最终还是在殿堂上隆重地接见了相如,大礼完后让他回了国。
    
    相如回国后,赵王认为他是一位有德行、有才能的贤大夫,出使诸侯国,能做到不辱使命,于是封相如为上大夫。此后秦国并没有把城邑给赵国,赵国也始终不给秦国宝璧。
    
    此后秦国攻打赵国,夺取了石城。第二年,秦国再次攻打赵国,杀死两万人。
    
    秦王派使者告诉赵王,想在西河外的渑池与赵王进行一次友好会见。赵王害怕秦国,打算不去。廉颇、蔺相如商量道:“大王如果不去,就显得赵国既软弱又胆小。”赵王于是前去赴会,蔺相如随行。廉颇送到边境,和赵王诀别说:“大王此行,估计路程和会谈结束,再加上返回的时间,不会超过三十天。如果三十天还没回来,就请您允许我们立太子为王,以断绝秦国要挟的妄想。”赵王答应了。
    
    于是(赵王)去渑池与秦王会见。秦王饮到酒兴正浓时,说:“我私下里听说赵王爱好音乐,请您奏瑟一曲。”赵王就弹起瑟来。秦国的史官上前来写道:“某年某月某日,秦王与赵王一起饮酒,令赵王弹瑟”。蔺相如上前说:“赵王私下里听说秦王擅长秦地土乐,请让我给秦王捧上盆,来相互为乐。”秦王发怒,不答应。这时蔺相如向前进献瓦缻,并跪下请秦王演奏。秦王不肯击缻。蔺相如说:“在这五步之内,如果我自杀,脖颈里的血可以溅在大王身上了!”秦王的侍从们想要杀蔺相如,蔺相如睁圆双眼大声斥骂他们,侍从们都吓得倒退。因此秦王很不高兴,也只好敲了一下缻。相如回头来招呼赵国史官写道:“某年某月某日,秦王为赵王击缻”。秦国的大臣们说:“请你们用赵国的十五座城池向秦王献礼。”蔺相如也说:“请你们用秦国的咸阳向赵王献礼。”直到酒宴结束,秦王始终也未能压倒赵王。赵国也部署了大批军队来防备秦国,因而秦国也不敢轻举妄动。
    
    渑池会结束以后回到赵国,由于蔺相如功劳大,被封为上卿,官位在廉颇之上。
    
    廉颇说:“作为赵国的将军,我有攻战城池作战旷野的大功劳,而蔺相如只不过靠能说会道立了点功,可是他的地位却在我之上,况且蔺相如本来就出身卑贱,我感到羞耻,无法容忍在他的下面。”并且扬言说:“我遇见蔺相如,一定要羞辱他一番。”蔺相如听到这话后,不愿意和廉颇相会。每到上朝时,蔺相如常常声称有病,不愿和廉颇去争位次的先后。没过多久,蔺相如外出,远远看到廉颇,蔺相如就掉转车子回避。于是蔺相如的门客就一起来向蔺相如抗议说:“我们之所以离开亲人来侍奉您,是仰慕您高尚的节义呀。如今您与廉颇官位相同,廉颇传出坏话,而您却害怕躲避着他,胆怯得也太过分了,一般人尚且感到羞耻,更何况是身为将相的人呢!我们这些人没有出息,请让我们辞去吧。”
    
    蔺相如坚决地挽留他们,说:“诸位认为廉将军和秦王相比谁更厉害?”众人都说:“廉将军比不上秦王。”蔺相如说:“以秦王的威势,而我尚敢在朝廷上呵斥他,羞辱他的群臣,我蔺相如虽然无能,难道会害怕廉将军吗!但是我想到,强大的秦国之所以不敢对赵国用兵,就是因为有我们两人在呀。如今我们俩相斗,就如同两猛虎争斗一般,势必不能同时生存。我之所以这样忍让,就是将国家的危难放在前面,而将个人的私怨搁在后面罢了。”廉颇听说了这些话,就脱去上衣,露出上身,背着荆鞭,由宾客引领,来到蔺相如的门前请罪。他说:“我这个粗野卑贱的人,想不到将军的胸怀如此宽大啊!”二人终于相互交欢和好,成了生死与共的好友。
    
\end{normalsize}



\chapter{伶官传序}

\begin{normalsize}
    
    呜呼!盛衰之理,虽曰天命,岂非人事哉!原庄宗\footnote{〔庄宗〕后唐庄宗李存勖,沙陀人。前期骁勇善战,公元923年灭后梁称帝,但宠信伶人,致三年后兴教门之变身亡。}之所以得天下,与其所以失之者,可以知之矣。
    
    世言晋王\footnote{〔晋王〕李克用,李存勖之父,唐末河东节度使,镇压黄巢起义有功封晋王,奠定后唐基业。}之将终也,以三矢赐庄宗而告之曰:“梁\footnote{〔梁〕朱温建立的后梁,五代首个政权,与李克用集团结仇,终被李存勖所灭。},吾仇也;燕王\footnote{〔燕王〕刘守光,原卢龙节度使,公元909年自称大燕皇帝,背弃与李克用盟约投梁,后被李存勖擒杀。},吾所立;契丹与吾约为兄弟;而皆背晋以归梁。此三者,吾遗恨也。与尔三矢,尔其无忘乃父之志!”庄宗受而藏之于庙。其后用兵,则遣从事以一少牢\footnote{〔少牢〕诸侯祭祀规格,用羊、猪二牲。}告庙,请其矢,盛以锦囊,负而前驱,及凯旋而纳之。
    
    方其系燕父子以组,函梁君臣之首,入于太庙\footnote{〔太庙〕帝王祖庙,庄宗将俘虏刘仁恭父子、梁末帝首级献祭太庙,标志复仇功成。},还矢先王,而告以成功,其意气之盛,可谓壮哉!及仇雠已灭,天下已定,一夫夜呼,乱者四应,仓皇东出,未及见贼而士卒离散,君臣相顾,不知所归。至于誓天断发,泣下沾襟,何其衰也!岂得之难而失之易欤?抑本其成败之迹,而皆自于人欤?
    
    《书》\footnote{〔《书》〕指《尚书》。}曰:“满招损,谦得益。”忧劳可以兴国,逸豫可以亡身,自然之理也。故方其盛也,举天下之豪杰,莫能与之争;及其衰也,数十伶人\footnote{〔伶人〕宫廷艺人,特指郭从谦等得势伶官,发动叛乱射杀庄宗。}困之,而身死国灭,为天下笑。夫祸患常积于忽微,而智勇多困于所溺,岂独伶人也哉!作《伶官传》。
\end{normalsize}


\newpage

\textbf{译文}:

\vspace{1em}

\begin{normalsize}
    
    唉!盛衰的道理,虽说是天命决定的,难道说不是人事造成的吗?推究庄宗所以取得天下,与他所以失去天下的原因,就可以明白了。
    
    世人传说晋王临死时,把三枝箭赐给庄宗,并告诉他说:“梁国是我的仇敌,燕王是我推立的,契丹与我约为兄弟,可是后来都背叛我去投靠了梁。这三件事是我的遗恨。交给你三枝箭,你不要忘记你父亲报仇的志向。”庄宗受箭收藏在祖庙。以后庄宗出兵打仗,便派手下的随从官员,用猪羊去祭告祖先,从宗庙里恭敬地取出箭来,装在漂亮的丝织口袋里,使人背着在军前开路,等打了胜仗回来,仍旧把箭收进宗庙。
    
    当他用绳子绑住燕王父子,用小木匣装着梁国君臣的头,走进祖庙,把箭交还到晋王的灵座前,告诉他生前报仇的志向已经完成,他那神情气概,是多么威风!等到仇敌已经消灭,天下已经安定,一人在夜里发难,作乱的人四面响应,他慌慌张张出兵东进,还没见到乱贼,部下的兵士就纷纷逃散,君臣们你看着我,我看着你,不知道哪里去好。到了割下头发来对天发誓,抱头痛哭,眼泪沾湿衣襟的可怜地步,怎么那样的衰败呢!难道说是因为取得天下难,而失去天下容易才像这样的吗?还是认真推究他成功失败的原因,都是由于人事呢?
    
    《尚书》上说:“自满会招来损害,谦虚能得到益处。”忧劳可以使国家兴盛,安乐可以使自身灭亡,这是自然的道理。因此,当他兴盛时,普天下的豪杰,没有谁能和他相争;到他衰败时,数十个乐官就把他困住,最后身死国灭,被天下人耻笑。祸患常常是由一点一滴极小的错误积累而酿成的,纵使是聪明有才能和英勇果敢的人,也多半沉溺于某种爱好之中,受其迷惑而结果陷于困穷,难道只有乐工吗?于是作《伶官传》。
    
\end{normalsize}



\chapter{秋水}

\begin{normalsize}
    
    秋水时至,百川灌河。泾流之大,两涘渚崖之间,不辩牛马。于是焉河伯\footnote{〔河伯〕黄河的河神,这里是虚构的对话角色。}欣然自喜,以天下之美为尽在己。顺流而东行,至于北海,东面而视,不见水端。于是焉河伯始旋其面目,望洋向若而叹曰:“野语有之曰:‘闻道百,以为莫己若者’,我之谓也。且夫我尝闻少仲尼\footnote{〔仲尼〕指孔子。}之闻,而轻伯夷\footnote{〔伯夷〕商末孤竹君长子。武王灭商后,耻食周粟,饿死于首阳山。}之义者,始吾弗信,今我睹子之难穷也,吾非至于子之门,则殆矣,吾长见笑于大方之家。”
    
    北海若\footnote{〔北海若〕北海的海神,这里是虚构的对话角色。}曰:“井蛙不可以语于海者,拘于虚也;夏虫不可以语于冰者,笃于时也;曲士不可以语于道者,束于教也。今尔出于崖涘,观于大海,乃知尔丑,尔将可与语大理矣。天下之水,莫大于海。万川归之,不知何时止而不盈;尾闾泄之,不知何时已而不虚。春秋不变,水旱不知。此其过江河之流,不可为量数。而吾未尝以此自多者,自以比形于天地,而受气于阴阳,吾在天地之间,犹小石小木之在大山也。方存乎见少,又奚以自多!计四海之在天地之间也,不似礨空之在大泽乎?计中国之在海内不似稊米之在大仓乎?号物之数谓之万,人处一焉。人卒九州,谷食之所生,舟车之所通,人处一焉。此其比万物也,不似豪末之在于马体乎?五帝之所连,三王之所争,仁人之所忧,任士之所劳,尽此矣!伯夷辞之以为名,仲尼语之以为博。此其自多也,不似尔向之自多于水乎?”
\end{normalsize}


\newpage

\textbf{译文}:

\vspace{1em}

\begin{normalsize}
    
    秋雨如期而至,千百条小河注入黄河。水流宽阔,两岸和水中洲岛之间,连牛马都分辨不清。于是乎,河伯洋洋自得,认为天下的美景都集中在他自己这里。顺着流水向东方行走,一直到达北海,面向东看去,看不到水的尽头。这时,河伯转变了自己的脸色,抬头仰视着海神若叹息说:“俗话所说的‘知道的道理很多了,便认为没有谁能比得上自己’,这正是说我呀。再说,我曾经听说(有人)认为仲尼的学识少,伯夷的义行不值得看重,开始我还不敢相信,现在我亲眼目睹了大海您大到难以穷尽,如果我没有来到您的身边,那就很危险了,我将要永远被明白大道理的人嘲笑。”
    
    北海若说:“不可与井底之蛙谈论大海,因为它的眼界受狭小居处的局限;不可与夏天的虫子谈论冰,因为它受到时令的局限;不可与见识浅陋的乡曲书生谈论大道理,因为他受到了礼教的束缚。现在你河伯从黄河两岸间走出,看到了大海,才知道你自己的鄙陋,可以跟你谈论一些大道理了。天下的水,没有比海更大的。千万条河流流归大海,没有停止的时候,而大海却并不因此而盈满;尾闾不停地排泄海水,不知到什么时候停止,但大海并没有减少。无论春天还是秋天大海水位不变,无论水灾还是旱灾大海没有感觉。大海的容量超过了长江、黄河的水流,简直不能用数字来计算。但是我并没有因此而自夸,我自认为自己列身于天地之间,接受了阴阳之气,我在天地之间,好比是小石块、小树木在高山一样。正感到自己的渺小,又怎么会自傲自夸?考虑到四海在天地之间,不正像小小的蚁穴存在于大湖之中吗?考虑到中原地区在四海之内,不正像米粒存放在大粮仓之中吗?称呼物类的数目可以说“万”,而人类只不过居于万物中的一种。人类虽遍布九州,但其所居之地也只占谷食所生、舟车所通之地中的万分之一。拿人和万物相比,不正像一根毫毛在马身上一样吗?五帝所延续的(业绩),三王所争夺的(天下),仁人志士所忧虑的(事情),以天下为己任的贤能之士为之劳苦的(目标),都不过如此而已。伯夷以辞让周王授予的职位而取得名声,孔子以谈说‘仁’、‘礼’而显示渊博。他们这样自我夸耀,不正像你当初因河水上涨而自夸一样吗?”
    
\end{normalsize}



\chapter{谏太宗十思疏}

\begin{normalsize}
    
    臣闻求木之长者,必固其根本;欲流之远者,必浚其泉源;思国之安者,必积其德义。源不深而望流之远,根不固而求木之长,德不厚而思国之安,臣虽下愚,知其不可,而况于明哲乎?人君当神器之重,居域中之大,将崇极天之峻,永保无疆之休。不念居安思危,戒奢以俭,德不处其厚,情不胜其欲,斯亦伐根以求木茂,塞源而欲流长也。
    
    凡百元首,承天景命,莫不殷忧而道著,功成而德衰,有善始者实繁,能克终者盖寡。岂其取之易守之难乎?昔取之而有余,今守之而不足,何也?夫在殷忧必竭诚以待下,既得志则纵情以傲物。竭诚则吴、越为一体\footnote{〔吴越〕春秋时南方的国家,长期相互征伐,最终以越国吞并吴国结束。},傲物则骨肉为行路。虽董之以严刑,震之以威怒,终苟免而不怀仁,貌恭而不心服。怨不在大,可畏惟人;载舟覆舟\footnote{〔载舟覆舟〕出自《荀子·王制》:“君者,舟也;庶人者,水也。水则载舟,水则覆舟。”},所宜深慎。奔车朽索,其可忽乎?
    
    君人者,诚能见可欲,则思知足以自戒;将有作,则思知止以安人;念高危,则思谦冲而自牧;惧满溢,则思江海下百川;乐盘游,则思三驱\footnote{〔三驱〕出自《易经》:“王以三驱”。}以为度;忧懈怠,则思慎始而敬终;虑壅蔽,则思虚心以纳下;惧谗邪,则思正身以黜恶;恩所加,则思无因喜以谬赏;罚所及,则思无以怒而滥刑。总此十思,宏兹九德,简能而任之,择善而从之,则智者尽其谋,勇者竭其力,仁者播其惠,信者效其忠。文武争驰,君臣无事,可以尽豫游之乐,可以养松乔之寿\footnote{〔松乔〕赤松子和王子乔,古代传说中的仙人。},鸣琴垂拱,不言而化。何必劳神苦思,代下司职,役聪明之耳目,亏无为之大道哉?
\end{normalsize}


\newpage

\textbf{译文}:

\vspace{1em}

\begin{normalsize}
    
    我听说想要树木生长,一定要稳固它的根;想要泉水流得远,一定要疏通它的源泉;想要国家安定,一定要厚积道德仁义。源泉不深却希望泉水流得远,根系不牢固却想要树木生长,道德不深厚却想要国家安定,我虽然地位低见识浅,(也)知道这是不可能的,更何况(您这)聪明睿智(的人)呢!国君处于皇帝的重要位置,在天地间尊大,就要推崇皇权的高峻,永远保持政权的和平美好。如果不在安逸的环境中想着危难,戒奢侈,行节俭,道德不能保持宽厚,性情不能克服欲望,这也(如同)挖断树根来求得树木茂盛,堵塞源泉而想要泉水流得远啊。
    
    (古代)所有的帝王,承受了上天赋予的重大使命,他们没有一个不为国家深切地忧虑而且治理成效显著的,但大功告成之后国君的品德就开始衰微了。国君开头做得好的确实很多,能够坚持到底的大概不多,难道是取得天下容易守住天下困难吗?过去夺取天下时力量有余,现在守卫天下却力量不足,这是为什么呢?通常处在深重的忧虑之中一定能竭尽诚心来对待臣民,已经成功,就放纵自己的性情来傲视别人。竭尽诚心,就连吴国和越国也能结成一家;傲慢他人,就连至亲也会变成陌路人。即使用严酷的刑罚来督责(人们),用威风怒气来吓唬(人们),(臣民)只求苟且免于刑罚而不怀念感激国君的仁德,表面上恭敬而在内心里却不服气。(臣民)对国君的怨恨不在大小,可怕的只是百姓;(他们像水一样)能够负载船只,也能颠覆船只,这是应当深切谨慎的。疾驰的马车却用腐烂的绳索驾驭,怎么可以疏忽大意呢?
    
    做国君的人,如果真的能够做到一见到能引起(自己)喜好的东西就要想到用知足来自我克制,将要兴建什么就要想到适可而止来使百姓安定,想到帝位高高在上就想到要谦虚并加强自我约束,害怕骄傲自满就想到要像江海那样能够(处于)众多河流的下游,喜爱狩猎就想到网三面留一面,担心意志松懈就想到(做事)要慎始慎终,担心(言路)不通受蒙蔽就想到虚心采纳臣下的意见,考虑到(朝中可能会出现)谗佞奸邪就想到使自身端正(才能)罢黜奸邪,施加恩泽就要考虑到不要因为一时高兴而奖赏不当,动用刑罚就要想到不要因为一时发怒而滥用刑罚。全面做到这十件应该深思的事,弘扬这九种美德,选拔有才能的人而任用他,挑选好的意见而听从它,那么有智慧的人就能充分献出他的谋略,勇敢的人就能完全使出他的力量,仁爱的人就能散播他的恩惠,诚信的人就能献出他的忠诚。文臣武将争先恐后前来效力,国君和大臣没有大事烦扰,可以尽情享受出游的快乐,可以颐养得像赤松子与王子乔那样长寿,皇上弹着琴垂衣拱手就能治理好天下,不用再说什么,天下人就已经都有教化了。为什么一定要(自己)劳神费思,代替臣下管理职事,役使自己灵敏、明亮的耳、眼,减损顺其自然就能治理好天下的大道理呢!
    
\end{normalsize}


\newpage

\textbf{注解}:

\vspace{-1em}

\begin{itemize}
    \setlength\itemsep{-0.2em}
    \item〔载舟覆舟〕这个典故本来出自《荀子·王制》:“君者,舟也;庶人者,水也。水则载舟,水则覆舟。”李世民在《自鉴录》中也说:“舟所以比人君,水所以比黎庶。水能载舟,亦能覆舟。”因此魏徵也引用这个典故。
\end{itemize}

\chapter{五人墓碑记}

\begin{normalsize}
    
    五人者,盖当蓼洲周公\footnote{〔蓼洲周公〕指周顺昌,东林党人,万历年间进士,曾任吏部郎中,因对抗魏忠贤,被诬罪逮捕。}之被逮,激于义而死焉者也。至于今,郡之贤士大夫请于当道,即除魏阉\footnote{〔魏阉〕指太监魏忠贤。}废祠之址以葬之;且立石于其墓之门,以旌其所为。呜呼,亦盛矣哉!
    
    夫五人之死,去今之墓而葬焉,其为时止十有一月耳。夫十有一月之中,凡富贵之子,慷慨得志之徒,其疾病而死,死而湮没不足道者,亦已众矣;况草野之无闻者欤?独五人之皦皦,何也?
    
    予犹记周公之被逮,在丙寅\footnote{〔丙寅〕即天启六年(公元1626年)。}三月之望。吾社\footnote{〔吾社〕指应社,天启四年由张溥、张采等江南士人发起成立,成员多为东林党人,相互呼应。}之行为士先者,为之声义,敛赀财以送其行,哭声震动天地。缇骑\footnote{〔丙寅〕即天启六年(公元1626年)。}按剑而前,问:“谁为哀者?”众不能堪,抶而仆之。是时以大中丞\footnote{〔大中丞〕官职名,指当时苏州巡抚毛一鹭,依附魏忠贤。}抚吴者为魏之私人,公之逮所由使也;吴之民方痛心焉,于是乘其厉声以呵,则噪而相逐。中丞匿于溷藩以免。既而以吴民之乱请于朝,按诛五人,曰颜佩韦、杨念如、马杰、沈扬、周文元\footnote{〔颜佩韦……〕五人的名字。周文元是周顺昌的轿夫,颜佩韦、杨念如、马杰、沈扬是一般市民。},即今之傫然在墓者也。
    
    然五人之当刑也,意气扬扬,呼中丞之名而詈之,谈笑以死。断头置城上,颜色不少变。有贤士大夫发五十金,买五人之头而函之,卒与尸合。故今之墓中全乎为五人也。
    
    嗟乎!大阉之乱,缙绅而能不易其志者,四海之大,有几人欤?而五人生于编伍之间,素不闻诗书之训,激昂大义,蹈死不顾,亦曷故哉?且矫诏纷出,钩党之捕遍于天下,卒以吾郡之发愤一击,不敢复有株治;大阉亦逡巡畏义,非常之谋难于猝发,待圣人之出而投缳道路,不可谓非五人之力也。
    
    由是观之,则今之高爵显位,一旦抵罪,或脱身以逃,不能容于远近,而又有剪发杜门,佯狂不知所之者,其辱人贱行,视五人之死,轻重固何如哉?是以蓼洲周公忠义暴于朝廷,赠谥褒美,显荣于身后;而五人亦得以加其土封,列其姓名于大堤之上,凡四方之士无不有过而拜且泣者,斯固百世之遇也。不然,令五人者保其首领,以老于户牖之下,则尽其天年,人皆得以隶使之,安能屈豪杰之流,扼腕墓道,发其志士之悲哉?故余与同社诸君子,哀斯墓之徒有其石也,而为之记,亦以明死生之大,匹夫之有重于社稷也。
    
    贤士大夫者,冏卿因之吴公\footnote{〔吴公〕吴默,字因之,“冏卿”指官职:太仆卿。},太史文起文公\footnote{〔文公〕文震孟,字文起,“太史”指官职:翰林院修撰。}、孟长姚公\footnote{〔姚公〕姚希孟,字孟长。}也。
\end{normalsize}


\newpage

\textbf{译文}:

\vspace{1em}

\begin{normalsize}
    
    (墓中的)五个人,就是当周蓼洲先生被捕的时候,被正义所激励而死于这件事的。到了现在,本郡有声望的士大夫们向有关当局请求,就清理已被废除的魏忠贤生祠旧址来安葬他们;并且在他们的墓门之前竖立碑石,来表彰他们的事迹。啊,也算是盛大隆重的事情呀!
    
    这五人的死,距离现在建墓安葬,时间只不过十一个月罢了。在这十一个月当中,所有富贵人家的子弟,意气激昂,得志的人,他们因患病而死,死后埋没不值得称道的人,也太多了;何况乡间没有声名的人呢?唯独这五个人声名显赫,为什么呢?
    
    我还记得周公被捕,是在丙寅年农历三月十五。我们社里那些道德品行可以作为读书人的榜样的人,替他伸张正义,募集钱财送他起程,哭声震天动地。缇骑按着剑柄上前,问:“(在)为谁悲痛?”大家不能再忍受了,把他们打倒在地。当时以大中丞职衔担任苏州一带巡抚的是魏忠贤的党羽,周公被捕就是由他主使的;苏州的老百姓正在痛恨他,这时趁着他厉声呵骂的时候,就一齐喊叫着追赶他。这位大中丞藏在厕所里才得以逃脱。不久,他以苏州人民发动暴乱的罪名向朝廷请示,追究这件事,杀了五个人,他们是颜佩韦、杨念如、马杰、沈扬、周文元,就是现在一起埋葬在墓中的这五个人。
    
    然而,当五个人临刑的时候,神情慷慨自若,呼喊着中丞的名字斥骂他,谈笑着死去了。砍下的头放在城头上,脸上的神情一点也没改变。有位有名望的人拿出五十两银子,买下五个人的头并用棺材收起来,最终与尸体合到了一起。所以现在墓中是完完整整的五个人。
    
    唉!当魏忠贤作乱的时候,能够不改变自己志节的做官的人,那么大的中国,能有几个呢?但这五个人生于民间,从来没受过《诗》《书》的教诲,却能被大义所激励,踏上死地,义无反顾,又是什么缘故呢?况且当时假托的皇帝的诏书纷纷传出,追捕同党的人遍于天下,终于因为我们苏州人民的发愤抗击,使阉党不敢再将我们株连治罪;魏忠贤也迟疑不决,畏惧正义,篡夺帝位的阴谋难于立刻发动,直到当今的皇上继位,(魏忠贤畏罪)吊死在路上,不能不说是这五个人的功劳呀。
    
    由此看来,那么如今这些高官显贵们,一旦犯罪受罚,有的人脱身逃走,不能被远近的百姓所容纳;也有的削发为僧、闭门不出,或假装疯狂不知逃到何处的,他们那可耻的人格,卑贱的行为,比起这五个人的死来,轻重的差别到底怎么样呢?因此周蓼洲先生的忠义显露在朝廷,赠给他的谥号美好而光荣,在死后享受到荣耀;而这五个人也能够修建一座大坟墓,在大堤之上立碑刻名,四方的士子经过这里没有不跪拜流泪的,这实在是百代难得的机遇啊。不这样的话,假使让这五个人保全性命在家中一直生活到老,尽享天年,人人都能够像对待奴仆一样使唤他们,又怎么能让豪杰们屈身下拜,在墓道上扼腕叹息,抒发他们(作为)有志之士的悲叹呢?所以我和我们同社的诸位先生,惋惜这墓前空有一块石碑,就为它作了这篇碑记,也用来说明生死意义的重大,(即使)一个普通百姓对于国家也有重要的作用啊。
    
    几位有声望的士大夫是:太仆卿吴因之公,太史文起文公,姚孟长公。
    
\end{normalsize}


\newpage

\textbf{注解}:

\vspace{-1em}

\begin{itemize}
    \setlength\itemsep{-0.2em}
    \item〔是时以大中丞抚吴者为魏之私人,公之逮所由使也;吴之民方痛心焉,于是乘其厉声以呵,则噪而相逐。中丞匿于溷藩以免。〕在张溥笔下,毛一鹭成为市民首要攻击的对象,毛一鹭血腥镇压这次民变。然而,民变组织者文震孟之子文秉提到毛一鹭,却语含感激:“邵辅忠贻书毛抚:‘急以文某、姚某入告,少宰虚席以待。’盖指文肃、姚文毅两公也。毛抚勿为动,止擒颜佩韦、杨念如、周文元、马杰、沈扬五人,具狱斩之。”《熹宗实录》载毛一鹭奏章:“方周顺昌奉旨被逮,县官往即就系,当令府县力促开读,而官旗不应逗遛需索,订期十八,致生远迩之心,此变之所由肇也。苏郡法纪陵夷,已非朝夕。臣等德不足以绥民,威不足以肃众,抑何辞于溺职之罪。除一面将犯官周顺昌交发官旗即日起解至京,伏候圣明处分外,仍次第擒缉倡乱渠魁,另行正法,以重国典。”可见,毛一鹭采取了息事宁人的态度,有意放过应社乡绅士子,只抓捕了少数庶民,但反被张溥写成“违背民意”的证据。温睿临亦载:“巡抚毛一鹭惧祸,根究乱民,杀五人以谢奄。”而各家史书,均未见苏民追打地方官员的记载。张溥攻击毛一鹭,乃是出于党争思维:魏忠贤虽然已经倒台,但魏的地方余党尚在,不赶尽杀绝誓不罢休。这一大清洗思想并非孤立。魏忠贤倒台后,崇祯一直要求阁臣清洗阉党,但阁臣却努力缩小波及面,以免政治动荡,君臣之间数度往复。 “初,逆珰死后,上欲因台谏言定逆案。大学士韩爌、钱龙锡不欲广搜树怨,仅列四五十人以请。上不悦,再令尽列以闻。”崇祯二年正月,崇祯再次催问,阁臣仍旧拖延,于是崇祯“发原奏及前红本未入各官六十九人,各令酌定,于是案列甚广,几无遗矣。”三月,“廷臣上《钦定逆案》,诏刊布中外。共二百五十八人”。《碑记》创作之时,政坛动荡尚未波及毛一鹭,故张溥不惜扭曲事实,引火烧之。《碑记》是一个强烈的政治信号,标志着东林后劲应社(复社前身)正式加入党争营垒。
    \item〔然五人之当刑也,意气扬扬,呼中丞之名而詈之,谈笑以死。〕《五人墓碑记》还发掘出一个新主题:将市民阶层拉入党争队伍,使之成为党争急先锋,以博取舆论支持,扩大政治影响。历史上的党争都局限在朝廷之上,晚明党争即是内廷与外朝之间的纷争。而《碑记》中,下层民众被裹挟进入党争队伍。在张溥笔下,市井民众被士人化,被引入东林队伍中,成为反阉急先锋。然而这也是一个虚构情节。五人临刑,并无痛骂之事。《明季北略》载:“斩五人于阊门吊桥,时颜佩韦等四人俱不畏,独周文元本舆夫,大哭。”《明史》载:“佩韦等皆市人,文元则顺昌舆隶也,论大辟。临刑,五人延颈就刃,语寇慎曰:‘公好官,知我等好义,非乱也。’监司张孝流涕而斩之。”苏州所谓“民变”,只是一些市民被应社煽动,参与党争、誓死反阉之说显属附会。把东林党与民心捆绑到一起,是一个复杂的历史过程。崇祯毁《三朝要典》,清洗阉党,给世人尘埃落定的印象。而阉党扶植的南明小朝廷又非常短命,更加剧了世人的鄙薄之心。加上阉党“所仇怨多在江南”,江南士人与东林、复社有着千丝万缕的联系,他们著文讲学,影响甚大。东林及其后劲复社遂获得广泛的民间支持。张溥把应社成员打扮成组织领导者,将暴动市民解读成听从应社号召、积极投身党争的急先锋,又将毛一鹭扭曲成民变的愤怒对象,从而突出应社巨大的社会号召能量,并将“吾社”继东林而起的政治含义固定在世人头脑中。
\end{itemize}

\chapter{指南录后序}

\begin{normalsize}
    
    德祐二年\footnote{〔德祐〕宋恭帝赵㬎年号(公元1275至1276年)。}正月十九日,予除右丞相,兼枢密使,都督诸路军马。时北兵\footnote{〔北兵〕即元兵。下文以“北”指元政权。}已迫修门\footnote{〔修门〕《楚辞·招魂》:“魂兮归来,入修门些。”本指楚国郢都城门,这里代指南宋都城临安的城门。}外,战、守、迁皆不及施。缙绅、大夫、士萃于左丞相府\footnote{〔左丞相〕指吴坚。德祐二年正月,吴坚升任左丞相兼枢密使,受谢太后命,与贾余庆等先赴元营议降,后为祈请使,赴元大都呈降表,交宋玺。后被羁留大都,当年病故。},莫知计所出。会使辙交驰,北邀当国者相见,众谓予一行,为可以纾祸。国事至此,予不得爱身,意北亦尚可以口舌动也。初,奉使往来,无留北者,予更欲一觇北,归而求救国之策。于是辞相印不拜,翌日,以资政殿学土行。
    
    初至北营,抗词慷慨,上下颇惊动,北亦未敢遽轻吾国。不幸吕师孟\footnote{〔吕师孟〕兵部尚书,叛将吕文焕之侄。}构恶于前,贾余庆\footnote{〔贾余庆〕官同签书枢密院事,知临安府,后代文天祥为右丞相。}献谄于后,予羁縻不得还,国事遂不可收拾。予自度不得脱,则直前诟虏帅失信,数吕师孟叔侄为逆,但欲求死,不复顾利害。北虽貌敬,实则愤怒,二贵酋名曰“馆伴”,夜则以兵围所寓舍,而予不得归矣。未几,贾余庆等以祈请使诣北,北驱予并往,而不在使者之目。予分当引决,然而隐忍以行,昔人云:将以有为也。
    
    至京口\footnote{〔京口〕江苏镇江市,当时为元军占领。},得间奔真州\footnote{〔真州〕今江苏仪征县,当时仍为宋军把守。},即具以北虚实告东西二阃\footnote{〔东西二阃〕指宋淮东制置使李庭芝和淮西制置使夏贵。},约以连兵大举。中兴机会,庶几在此。留二日,维扬帅下逐客之令。不得已,变姓名,诡踪迹,草行露宿,日与北骑相出没于长淮间。穷饿无聊,追购又急;天高地迥,号呼靡及。已而得舟,避渚州\footnote{〔渚州〕指长江中的沙州,当时已被元兵占领。},出北海\footnote{〔北海〕指淮海。},然后渡扬子江,入苏州洋\footnote{〔苏州洋〕今上海市附近的海域。},展转四明、天台\footnote{〔四明、天台〕四明:现在浙江宁波市。天台:现在浙江天台县。},以至于永嘉\footnote{〔永嘉〕现在浙江温州市。}。
    
    呜呼!予之及于死者,不知其几矣。诋大酋,当死;骂逆贼,当死;与贵酋处二十日,争曲直,屡当死;去京口,挟匕首以备不测,几自刭死;经北舰十余里,为巡船所物色,几从鱼腹死;真州逐之城门外,几彷徨死;如扬州,过瓜洲\footnote{〔瓜洲〕扬州南长江中的沙洲。}扬子桥,竟使遇哨,无不死;扬州城下,进退不由,殆例送死;坐桂公塘\footnote{〔桂公塘〕地名,在扬州城外。}土围中,骑数千过其门,几落贼手死;贾家庄\footnote{〔贾家庄〕地名,在扬州城北。}几为巡徼所陵迫死;夜趋高邮,迷失道,几陷死;质明,避哨竹林中,逻者数十骑,几无所逃死;至高邮\footnote{〔高邮〕现在江苏高邮县,也称“高沙”。},制府\footnote{〔制府〕指淮东制置使官府。}檄下,几以捕系死;行城子河\footnote{〔城子河〕在高邮县境内。},出入乱尸中,舟与哨相后先,几邂逅死;至海陵\footnote{〔海陵〕现在江苏泰州市。},如高沙,常恐无辜死;道海安、如皋\footnote{〔海安、如皋〕县名,均属江苏。},凡三百里,北与寇往来其间,无日而非可死;至通州\footnote{〔通州〕现在江苏省南通市。},几以不纳死;以小舟涉鲸波,出无可奈何,而死固付之度外矣!呜呼,死生昼夜事也。死而死矣,而境界危恶,层见错出,非人世所堪。痛定思痛,痛何如哉!
    
    予在患难中,间以诗记所遭。今存其本,不忍废,道中手自抄录。使北营,留北关外\footnote{〔北关外〕指临安城北高亭山。},为一卷;发北关外,历吴门、毘陵\footnote{〔吴门、毘陵〕吴门:今江苏苏州市。毘陵:今江苏常州市。},渡瓜洲,复还京口,为一卷;脱京口,趋真州、扬州、高邮、泰州、通州,为一卷;自海道至永嘉,来三山\footnote{〔三山〕即今福建福州市。因城中有闽山、越王山、九仙山,故名“三山”。},为一卷。将藏之于家,使来者读之,悲予志焉。
    
    呜呼!予之生也幸,而幸生也何为?所求乎为臣,主辱臣死有馀僇;所求乎为子,以父母之遗体行殆而死,有馀责。将请罪于君,君不许;请罪于母,母不许。请罪于先人之墓,生无以救国难,死犹为厉鬼以击贼,义也。赖天之灵,宗庙之福,修我戈矛,从王于师,以为前驱;雪九庙\footnote{〔九庙〕皇帝祭祀祖先共有九庙,这里以九庙指代国家。}之耻,复高祖\footnote{〔高祖〕指宋太祖赵匡胤。}之业;所谓誓不与贼俱生,所谓鞠躬尽力,死而后已,亦义也。嗟夫!若予者,将无往而不得死所矣。向也使予委骨于草莽,予虽浩然无所愧怍,然微以自文于君亲,君亲其谓予何!诚不自意,返吾衣冠,重见日月,使旦夕得正丘首,复何憾哉!复何憾哉!
    
    是年夏五\footnote{〔夏五〕夏天五月。},改元景炎\footnote{〔改元景炎〕由于宋恭帝被元军掳去,德祐二年五月,文天祥等人在福州立赵昰为帝,改元景炎。}。庐陵\footnote{〔庐陵〕现在江西吉安市。}文天祥自序其诗,名曰《指南录》。
\end{normalsize}


\newpage

\textbf{译文}:

\vspace{1em}

\begin{normalsize}
    
    德祐二年二月十九日,我受任右丞相兼枢密使,统率全国各路兵马。当时元兵已经逼近都城北门外,交战、防守、转移都来不及做了。满朝大小官员会集在左丞相吴坚家里,都不知道该怎么办。适逢双方使者的车辆往来频繁,元军邀约宋朝主持国事的人前去相见,大家认为我去一趟就可以解除祸患。国事到了这种地步,我不能顾惜自己了;估计元方也许可以用言词打动。当初,使者奉命往来,并没有被扣留在北方的,我就更想察看一下元方的虚实,回来谋求救国的计策。于是,辞去右丞相职位,第二天,以资政殿学士的身份前往。
    
    刚到元营时,据理抗争,言词激昂慷慨,元军上下都很惊慌震动,他们也未敢立即轻视我国。可不幸的是,吕师孟早就同我结怨,贾余庆又紧跟着媚敌献计,于是我被拘留不能回国,国事就不可收拾了。我揣度不能脱身,就径直上前痛骂元军统帅不守信用,列举吕师孟叔侄的叛国行径,只要求死,不再考虑个人的利害。元军虽然表面尊敬,其实却很愤怒,两个重要头目名义上是到宾馆来陪伴,夜晚就派兵包围我的住所,我就不能回国了。不久,贾余庆等以祈请使的身份到元京大都去,元军驱使我一同前往,但不列入使者的名单。我按理应当自杀,然而仍然含恨忍辱地前去。
    
    到了京口,得到机会逃奔到真州,我立即把元方的虚实情况告诉淮东、淮西两位制置使,相约他们联兵讨元。复兴宋朝的机会,大概就在此一举了。留住了两天,驻守维扬的统帅竟下了逐客令。不得已,我只能改变姓名,隐蔽踪迹,在草地上和郊野外奔走歇宿,日日为躲避元军的骑兵出没在淮河一带。困窘饥饿,无依无靠,元军悬赏追捕得又很紧急,天高地远,号呼不应。后来得到一条船,避开元军占据的沙洲,逃出江口以北的海面,然后渡过扬子江口,进入苏州洋,辗转在四明、天台等地,最后到达永嘉。
    
    唉!我到达死亡的境地不知有多少次了!痛骂元军统帅该当死;辱骂叛国贼该当死;与元军头目相处二十天,争论是非曲直,多次该当死;离开京口,带着匕首以防意外,几次想要自杀死;经过元军兵舰停泊的地方十多里,被巡逻船只搜寻,几乎投江喂鱼而死;真州守将把我逐出城门外,几乎彷徨而死;到扬州,路过瓜洲扬子桥,假使遇上元军哨兵,也不会不死;扬州城下,进退两难,几乎等于送死;坐在桂公塘的土围中,元军数千骑兵从门前经过,几乎落到敌人手中而死;在贾家庄几乎被巡察兵凌辱逼迫死;夜晚奔向高邮,迷失道路,几乎陷入沼泽而死;天亮时,到竹林中躲避哨兵,巡逻的骑兵有好几十,几乎无处逃避而死;到了高邮,制置使官署的通缉令下达,几乎被捕而死;经过城子河,在乱尸中出入,我乘的船和敌方哨船一前一后行进,几乎不期而遇被杀死;到海陵,往高沙,常担心无罪而死;经过海安、如皋,总计三百里,元兵与盗贼往来其间,没有一天不可能死;到通州,几乎由于不被收留而死;靠了一条小船渡过惊涛骇浪,实在是没有办法,对于死本已置之度外了!唉!死和生,不过是昼夜之间的事罢了,死就死了,可是像我这样境界险恶,危难层迭交错涌现,实在不是世间的人所能忍受的。痛苦过去以后,再去追思当时的痛苦,那是何等的悲痛啊!
    
    我在患难中,有时用诗记述个人的遭遇,现在还保存着那些底稿,不忍心废弃,在逃亡路上亲手抄录。现在将出使元营,被扣留在北门外的,作为一卷;从北门外出发,经过吴门、毗陵,渡过瓜洲,又回到京口的,作为一卷;逃出京口,奔往真州、扬州、高邮、泰州、通州的,作为一卷;从海路到永嘉、来三山的,作为一卷。我将把这诗稿收藏在家中,使后来的人读了它,为我的志向而悲叹。
    
    唉!我能死里逃生算是幸运了,可幸运地活下来要干什么呢?要求做一个忠臣,国君受到侮辱,做臣子的即使死了也还是有罪的;要求做一个孝子,用父母留给自己的身体去冒险,即使死了也有罪责。将向国君请罪,国君不答应;向母亲请罪,母亲不答应;我只好向祖先的坟墓请罪。人活着不能拯救国难,死后还要变成恶鬼去杀贼,这就是义;依靠上天的神灵、祖宗的福泽,修整武备,跟随国君出征,做为先锋,洗雪朝廷的耻辱,恢复开国皇帝的事业,也就是古人所说的:“誓不与贼共存”,“恭敬谨慎地竭尽全力,直到死了方休”,这也是义。唉!像我这样的人,将是无处不是可以死的地方了。以前,假使我把尸骨抛在荒野里,我虽然正大光明问心无愧,但也不能掩饰自己对国君、对父母的过错,国君和父母又将会怎么讲我呢?实在料不到我终于返回宋朝,重整衣冠,又见到皇帝、皇后,即使立刻死在故国的土地上,我还有什么遗憾呢!还有什么遗憾呢!
    
    这一年夏天五月,改年号为景炎。庐陵文天祥为自己的诗集作序,诗集名《指南录》。
    
\end{normalsize}



\chapter{烛之武退秦师}

\begin{normalsize}
    
    晋侯\footnote{〔晋侯〕即晋文公重耳,在秦穆公支持下夺得君位,春秋第二位霸主。}、秦伯\footnote{〔秦伯〕即秦穆公,扶持在外流亡的晋公子重耳继承君位,春秋第三位霸主。}围郑,以其无礼于晋,且贰于楚也。晋军函陵\footnote{〔函陵〕现在河南新郑市北。},秦军氾南\footnote{〔氾南〕氾水之南。氾水:流经今山东菏泽市定陶区南、曹县北,汇入古菏泽。}。
    
    佚之狐\footnote{〔佚之狐〕郑国大夫。}言于郑伯\footnote{〔郑伯〕即郑文公,晋公子重耳流亡时拒绝接待,于是重耳即位后与秦穆公讨伐郑国。}曰:“国危矣,若使烛之武见秦君,师必退。”公从之。辞曰:“臣之壮也,犹不如人;今老矣,无能为也已。”公曰:“吾不能早用子,今急而求子,是寡人之过也。然郑亡,子亦有不利焉。”许之。
    
    夜缒而出,见秦伯,曰:“秦、晋围郑,郑既知亡矣。若亡郑而有益于君,敢以烦执事。越国以鄙远,君知其难也。焉用亡郑以陪邻?邻之厚,君之薄也。若舍郑以为东道主,行李之往来,共其乏困,君亦无所害。且君尝为晋君赐矣,许君焦、瑕\footnote{〔焦、瑕〕黄河北岸的两个城邑,现在河南焦作一带,也有说在山西陕县一带。},朝济而夕设版焉,君之所知也。夫晋,何厌之有?既东封郑,又欲肆其西封,若不阙秦,将焉取之?阙秦以利晋,唯君图之。”秦伯说,与郑人盟。使杞子、逢孙、杨孙\footnote{〔杞子、逢孙、杨孙〕秦国大夫。两年后杞子密谋反郑,邀秦军伐郑,里应外合。事情败露后,杞子逃往齐国,逢孙、杨孙逃往宋国。}戍之,乃还。
    
    子犯\footnote{〔子犯〕狐偃,字子犯,晋国大夫,晋文公的舅舅。}请击之,公曰:“不可。微夫人之力不及此。因人之力而敝之,不仁;失其所与,不知;以乱易整,不武。吾其还也。”亦去之。
\end{normalsize}


\newpage

\textbf{译文}:

\vspace{1em}

\begin{normalsize}
    
    (鲁僖公三十年)晋文公和秦穆公联合围攻郑国,理由是郑国没有礼遇晋国,并且从属于晋的同时又从属于楚。晋军驻扎在函陵,秦军驻扎在氾水的南面。
    
    佚之狐对郑文公说:“国家危险了,假如派烛之武去拜见秦穆公,秦国的军队一定会撤退。”郑文公听从了建议。烛之武推辞说:“我壮年的时候,尚且不如别人;现在老了,没什么能做的了。”郑文公说:“我没有及早任用您,现在由于情况危急因而求您,这是我的过错。然而郑国灭亡了,对您也不利啊!”烛之武就答应了这件事。
    
    (烛之武)在夜晚用绳子(把自己)从城楼放下去,见秦穆公,说:“秦、晋两国围攻郑国,郑国已经知道要灭亡了。假如灭掉郑国对您有好处,怎敢冒昧地拿“亡郑”这件事麻烦您。然而越过别国把远方的郑国作为秦国的东部边邑,您知道这是困难的,为什么要灭掉郑国而给晋国增加土地呢?晋国的势力雄厚了,您秦国的势力也就相对削弱了。如果您放弃围攻郑国而把它当作东方道路上招待过客的主人,出使的人来来往往,郑国可以随时供给他们缺乏的东西,对您也没有什么害处。而且您曾经给予晋惠公恩惠,惠公曾经答应给您焦、瑕二座城池。然而惠公早上渡过黄河回国,晚上就在那里筑城防御,这是您所知道的。晋国,怎么会有满足的时候呢?现在它已经在东边使郑国成为它的边境,又想要向西扩大边界。如果不使秦国土地亏损,它到哪里去夺取土地?削弱秦国对晋国有利,希望您考虑一下!”秦伯非常高兴,就与郑国签订了盟约。
    
    晋国大夫子犯请求出兵攻击秦军。晋文公说:“不行!如不是秦国国君的力量,就没有我的今天。依靠别人的力量而又反过来损害他,这是不仁义的;失掉自己的同盟者,这是不明智的;用混乱相攻取代联合一致,是不符合武德的。我们还是回去吧!”
    
\end{normalsize}



\chapter{子鱼论战}

\begin{normalsize}
    
    楚人伐宋以救郑。宋公\footnote{〔宋公〕即宋襄公,名兹甫,宋桓公次子,前650年至前637年在位。}将战。大司马\footnote{〔大司马〕掌管军政、军赋的官职,这里指公孙固。}固谏曰:“天之弃商久矣,君将兴之,弗可赦也已。”弗听。冬十一月己巳朔,宋公及楚人战于泓\footnote{〔泓〕泓水,在今河南柘城县西。}。宋人既成列,楚人未既济。司马\footnote{〔司马〕统帅军队的高级长官,这里指公子目夷,字子鱼,宋襄公的庶兄。}曰:“彼众我寡,及其未既济也,请击之。”公曰:“不可。”既济而未成列,又以告。公曰:“未可。”既陈而后击之,宋师败绩。公伤股,门官歼焉。
    
    国人皆咎公。公曰:“君子不重伤,不禽二毛。古之为军也,不以阻隘也。寡人虽亡国之余,不鼓不成列。”子鱼曰:“君未知战。勍敌之人,隘而不列,天赞我也。阻而鼓之,不亦可乎?犹有惧焉!且今之勍者,皆我敌也。虽及胡耇,获则取之,何有于二毛?明耻教战,求杀敌也。伤未及死,如何勿重?若爱重伤,则如勿伤;爱其二毛,则如服焉。三军以利用也,金鼓以声气也。利而用之,阻隘可也;声盛致志,鼓儳可也。”
\end{normalsize}


\newpage

\textbf{译文}:

\vspace{1em}

\begin{normalsize}
    
    楚军攻打宋国以援救郑国。宋襄公将要迎战,大司马公孙固劝阻说,“上天遗弃商朝已经很久了,君王要振兴它,不可,这样做是得不到宽恕的。”襄公不听。冬季,十一月初一日,宋襄公与楚国人在泓水边上作战。宋军已经排成战斗的行列,楚国人没有全部渡过泓水。司马子鱼说:“对方人多,我方人少,趁着他们没有全部渡过泓水,请攻击他们。”宋襄公说:“不行。”楚军全部渡河,但尚未排好阵势,(子鱼)再次报告(宋襄公)。宋襄公说:“还不行。”(楚军)摆好阵势(宋军)才攻击楚军。宋军大败,宋襄公大腿受伤,国君的卫士被杀绝了。
    
    国人都责备宋襄公。襄公说:“君子不再伤害已经受伤的人,不俘虏头发斑白的老人。古代用兵的道理,不凭借险隘的地形阻击敌人。我虽然是亡国者的后代,(也)不攻击没有排成阵势的敌人。”子鱼说:“主公不懂得作战。面对强大的敌人,(敌人)因地势险阻而未成阵势,这是上天帮助我们;阻碍并攻击他们,不也可以吗?还有什么害怕的呢?而且现在强大的,都是我们的敌人。即使是年纪很大的人,能俘虏就抓回来,还管什么头发斑白的敌人?教导士兵作战,使他们知道退缩就是耻辱来鼓舞战斗的勇气,教战士掌握战斗的方法,就是为了杀死敌人。(敌人)受伤却还没有死,为什么不能再杀伤他们?如果怜惜(他们,不愿)再去伤害受伤的敌人,不如一开始就不伤害他们;怜惜头发斑白的敌人,不如(对敌人)屈服。军队凭借有利的时机而行动,锣鼓用来鼓舞士兵的勇气。利用有利的时机,当(敌人)遇到险阻,(我们)可以进攻。声气充沛盛大,增强士兵的战斗意志,攻击未成列的敌人是可以的。”
    
\end{normalsize}



\chapter{孟子四则}

\begin{normalsize}
    
    孟子曰:“不仁哉,梁惠王\footnote{〔梁惠王〕战国时魏国第三任国君。}也!仁者以其所爱及其所不爱,不仁者以其所不爱及其所爱。”公孙丑\footnote{〔公孙丑〕孟子弟子,齐国人,是《孟子》的主要作者之一。}问曰:“何谓也?”“梁惠王以土地之故,糜烂其民而战之,大败,将复之,恐不能胜,故驱其所爱子弟以殉之,是之谓以其所不爱及其所爱也。”
    
    孟子曰:“以力假仁者霸,霸必有大国,以德行仁者王,王不待大。汤以七十里,文王以百里。以力服人者,非心服也,力不赡也;以德服人者,中心悦而诚服也,如七十子之服孔子也。《诗》云:‘自西自东,自南自北,无思不服。’此之谓也。”
    
    孟子曰:“天时不如地利,地利不如人和。三里之城,七里之郭,环而攻之而不胜。夫环而攻之,必有得天时者矣;然而不胜者,是天时不如地利也。城非不高也,池非不深也,兵革非不坚利也,米粟非不多也;委而去之,是地利不如人和也。故曰:域民不以封疆之界,固国不以山溪之险,威天下不以兵革之利。得道者多助,失道者寡助。寡助之至,亲戚畔之;多助之至,天下顺之。以天下之所顺,攻亲戚之所畔;故君子有不战,战必胜矣。”
\end{normalsize}


\newpage

\textbf{译文}:

\vspace{1em}

\begin{normalsize}
    
    梁惠王说:“我对于国家,总算尽了心啦。河内遇到饥荒,就把那里的老百姓迁移到河东去,把河东的粮食转移到河内;河东遇到饥荒也是这样做。了解一下邻国的政治,没有像我这样用心的。邻国的百姓不见减少,我的百姓不见增多,这是为什么呢?”
    
    孟子回答说:“大王喜欢打仗,让我用战争做比喻吧。咚咚地敲响战鼓,两军开始交战,战败的扔掉盔甲拖着武器逃跑。有人逃了一百步然后停下来,有的人逃了五十步然后停下来。凭自己只跑了五十步而耻笑别人跑了一百步,那怎么样呢?”梁惠王说:“不行。
    
    “不耽误农业生产的季节,粮食就会吃不完。密网不下到池塘里,鱼鳖之类的水产就会吃不完。按一定的季节入山伐木,木材就会用不完。粮食和水产吃不完,木材用不完,这就使百姓对生养死葬没有什么不满了。百姓对生养死葬没有什么不满,这是王道的开端。“五亩大的住宅场地,种上桑树,五十岁的人就可以穿丝织品了。鸡、猪、狗的畜养,不要耽误它们的繁殖时机,七十岁的人就可以吃肉食了。百亩大的田地,不要耽误它的耕作时节,数口之家就可以不受饥饿了。
    
\end{normalsize}



\chapter{黄花岗烈士事略序}

\begin{normalsize}
    
    满清末造,革命党人历艰难险巇,以坚毅不扰之精神,与民贼相搏,踬踣者屡。死事之惨,以辛亥三月二十九日\footnote{〔辛亥〕即公元1911年。当年4月27日(农历3月29日)国民党人在广州发起黄花岗起义,不幸失败。}围攻两广督署之役为最。吾党\footnote{〔吾党〕指国民党。}菁华付之一炬,其损失可谓大矣!然是役也,碧血横飞,浩气四塞,草木为之含悲,风云因而变色。全国久蛰之人心,乃大兴奋。怨愤所积,如怒涛排壑,不可遏抑,不半载而武昌之革命\footnote{〔武昌之革命〕指1911年10月10日湖北武昌起义。}以成。则斯役之价值,直可惊天地,泣鬼神,与武昌革命之役并寿。
    
    顾自民国肇造,变乱纷乘,黄花岗上一抔土,犹湮没于荒烟蔓草间。延至七年,始有墓碣之建修;十年,始有事略之编纂。而七十二烈士者,又或有记载而语焉不详,或仅存姓名而无事迹,甚者且姓名不可考,如史载田横事\footnote{〔田横〕指汉初田横五百士的故事。},虽以史迁\footnote{〔史迁〕指司马迁,《史记》的作者。}之善传游侠,亦不能为五百人立传,滋可痛矣。
    
    邹君海滨\footnote{〔邹君海滨〕邹鲁(1885年2月20日—1954年2月13日),字海滨,中国国民党和中华民国元老,曾任国立中山大学校长。},以所辑《黄花岗烈士事略》丐序于余。时余方以讨贼\footnote{〔讨贼〕指讨伐北洋军阀。1921年粤军西征讨桂成功。12月,孙中山到桂林组建陆海军大元帅大本营,整军准备北伐。}督师桂林。环顾国内,贼氛方炽,杌靰之象,视清季有加,而余三十年前所主唱之三民主义、五权宪法\footnote{〔三民主义……〕指孙中山在1905年同盟会成立时正式提出的“三民主义”思想,是为旧三民主义。五权宪法指基于三民主义创立的五院制宪制理论。},为诸先烈所不惜牺牲生命以争之者,其不获实行也如故。则余此行所负之责任,尤倍重于三十年前。倘国人皆以先烈之牺牲精神为国奋斗,助余完成此重大之责任,实现吾人理想之真正中华民国,则此一部开国血史,可传而不朽。否则不能继述先烈遗志且光大之,而徒感慨于其遗事,斯诚后死者之羞也!
    
    余为斯序,既痛逝者,并以为国人之读兹编者勖。
\end{normalsize}


\newpage

\textbf{译文}:

\vspace{1em}

\begin{normalsize}
    
    满清末年,革命者们经历了各种艰难险阻,凭着坚毅不屈的精神,和反动派搏斗,失败了很多次。其中死得最惨烈的,要数辛亥革命那年三月二十九日围攻两广总督衙门的那次战斗。我党的精华在这一仗中损失殆尽,这损失可说是太大了!但是这场战斗,烈士们的鲜血四处飞溅,浩然正气充满天地,草木都为他们感到悲伤,风云都因此而变了颜色。全国长期被压抑的人心,也因此而大大振奋。长期积压的怨恨,就像愤怒的波涛冲击山谷一样,无法阻挡,不到半年,武昌起义的革命就成功了。那么这场战斗的价值,简直可以惊动天地,让鬼神哭泣,和武昌起义的革命战斗一样永垂不朽。
    
    但是自从民国建立以来,变乱一个接着一个,黄花岗上的那一抔土,还是被荒烟蔓草埋没着。直到民国七年,才开始修建墓碑;民国十年,才开始编纂烈士的事迹。而七十二位烈士中,有的虽然有记载但说得不够详细,有的只留下姓名而没有事迹,更严重的甚至连姓名都难以考证,就像史书上记载的田横的事,虽然司马迁善于为游侠立传,也不能为五百人都立传,这实在是让人痛心啊。
    
    邹海滨君把他编的《黄花岗烈士事略》拿来请我写序。当时我正为了讨伐逆贼在桂林督师。我看看国内,逆贼的气焰正嚣张,动荡不安的迹象,比起清末还要厉害,而我三十年前就倡导的三民主义、五权宪法,是各位先烈不惜牺牲生命去争取的,却还是没有实行。那么我现在肩负的责任,比三十年前更加重大了。如果国人都能学习先烈们的牺牲精神,为国家奋斗,帮助我完成这重大的责任,实现我们理想的真正的中华民国,那么这一部开国的血史,就可以流传下去而不朽。不然的话,不能继承先烈们的遗志并且发扬光大,只是对他们的事迹感慨一番,这实在是我们这些后来者的羞耻啊!
    
    我为此作序,既为逝去的人悲痛,也用来勉励读这本书的国人。
    
\end{normalsize}



\chapter{季氏将伐颛臾}

\begin{normalsize}
    
    季氏\footnote{〔季氏〕季康子,季孙氏,名肥。春秋时鲁国卿大夫,把持朝政。}将伐颛臾\footnote{〔颛臾〕鲁国的属国,故城在今山东费县西北。}。冉有\footnote{〔冉有〕孔子的弟子,名求,字子有。当时是季康子的家臣。}、季路\footnote{〔季路〕孔子的弟子,姓仲,名由,字子路。当时是季康子的家臣。}见于孔子曰:“季氏将有事于颛臾。”孔子曰:“求!无乃尔是过与?夫颛臾,昔者先王\footnote{〔先王〕指周之先王。}以为东蒙主\footnote{〔东蒙主〕指受封于东蒙。东蒙,山名,及蒙山,在今山东蒙阴南。},且在邦域之中矣,是社稷之臣也。何以伐为?”
    
    冉有曰:“夫子\footnote{〔夫子〕即季康子。}欲之,吾二臣者皆不欲也。”孔子曰:“求!周任\footnote{〔周任〕上古的史官。}有言曰:‘陈力就列,不能者止’,危而不持,颠而不扶,则将焉用彼相矣?且尔言过矣。虎兕出于柙,龟玉毁于椟中,是谁之过与?”
    
    冉有曰:“今夫颛臾,固而近于费。今不取,后世必为子孙忧。”孔子曰:“求!君子疾夫舍曰欲之而必为之辞。丘也闻有国有家者,不患寡而患不均,不患贫而患不安。盖均无贫,和无寡,安无倾。夫如是,故远人不服,则修文德以来之。既来之,则安之。今由与求也,相夫子,远人不服、而不能来也;邦分崩离析、而不能守也:而谋动干戈于邦内。吾恐季孙之忧,不在颛臾,而在萧墙\footnote{〔萧墙〕国君宫门内迎门的小墙,又叫做屏。因古时臣子朝见国君,走到此必肃然起敬,故称“萧墙”。这里借指宫廷。}之内也。”
\end{normalsize}


\newpage

\textbf{译文}:

\vspace{1em}

\begin{normalsize}
    
    季孙氏将要讨伐颛臾。冉有、季路拜见孔子说:“季孙氏要对颛臾用兵。”孔子说:“冉求!恐怕应该责备你们吧。那颛臾,先王曾把他当作主管东蒙山祭祀的人,而且它地处鲁国境内,是鲁国的藩属国,为什么要讨伐它呢?”
    
    冉有说:“季孙要这么干,我们两个做臣下的都不愿意。”孔子说:“冉求!周任有句话说:‘能施展其才能则就其职位,不能这样做就不就其职位’,如果盲人摇晃着要倒下却不去扶持,颤颤巍巍将要跌倒却不去搀扶,那么何必要用那个搀扶的人呢?况且你的话错了。老虎和犀牛从笼子里跑出,(占卜用的)龟甲和(祭祀用的)玉器在匣子里被毁坏,这是谁的过错呢?”
    
    冉有说:“如今的颛臾,城墙坚固而且靠近季孙氏的封地。现在不夺取,后世一定会成为子孙们的忧虑。”孔子说:“冉求!君子厌恶那些不肯说(自己)想要那样而偏要找借口的人。我听说拥有邦国封邑的诸侯和拥有家族封邑的大夫,他们不担忧东西少而担忧分配不均,不担忧贫困而担忧社会不安定。若是财富分配公平合理,便无所谓贫穷;境内和平团结,便不会觉得人少;境内平安,国家便不会倾危。做到这样,远方的人还不归服,就再修仁义礼乐的政教来招待他们。他们来了,就得使他们安心。如今由与求两人辅佐季孙氏,远方的人不归服,却不能使他们来归顺;国家四分五裂却不能保持它的稳定统一;反而策划在境内兴起干戈。我恐怕季孙氏的忧虑,不在颛臾,而是在鲁国内部。”
    
\end{normalsize}



\chapter{秋声赋}

\begin{normalsize}
    
    欧阳子方夜读书,闻有声自西南来者,悚然而听之,曰:“异哉!”初淅沥以萧飒,忽奔腾而砰湃,如波涛夜惊,风雨骤至。其触于物也,鏦鏦铮铮,金铁皆鸣;又如赴敌之兵,衔枚\footnote{〔衔枚〕古时行军或袭击敌军时,让士兵衔枚以防出声。枚,形似竹筷,衔于口中,两端有带,系于脖上。}疾走,不闻号令,但闻人马之行声。予谓童子:“此何声也?汝出视之。”童子曰:“星月皎洁,明河\footnote{〔明河〕天河。}在天,四无人声,声在树间。”
    
    予曰:“噫嘻悲哉!此秋声也,胡为而来哉?盖夫秋之为状也:其色惨淡,烟霏云敛;其容清明,天高日晶;其气栗冽,砭人肌骨;其意萧条,山川寂寥。故其为声也,凄凄切切,呼号愤发。丰草绿缛而争茂,佳木葱茏而可悦;草拂之而色变,木遭之而叶脱。其所以摧败零落者,乃其一气之余烈。夫秋,刑官\footnote{〔刑官〕执掌刑狱的官。《周礼》把官职与天、地、春、夏、秋、冬相配,称为六官。秋天肃杀万物,所以司寇为秋官,执掌刑法,称刑官。}也,于时为阴;又兵象也,于行用金,是谓天地之义气,常以肃杀而为心。天之于物,春生秋实,故其在乐也,商声\footnote{〔商声〕以商音为主音的调式,即现代的D小调。}主西方之音,夷则\footnote{〔夷则〕十二律中第九律。十二律分别是:黄钟、大吕、太簇、夹钟、姑洗、中吕、林钟、蕤宾、夷则、南吕、无射、应钟。古人把十二律和十二月对应。《礼记·月令》:“孟秋之月,律中夷则”,夷则对应初秋七月。}为七月之律。商,伤也,物既老而悲伤;夷,戮也,物过盛而当杀。”
    
    “嗟乎!草木无情,有时飘零。人为动物,惟物之灵;百忧感其心,万事劳其形。有动于中,必摇其精。而况思其力之所不及,忧其智之所不能?宜其渥然丹者为槁木,黟然黑者为星星。奈何以非金石之质,欲与草木而争荣?念谁为之戕贼,亦何恨乎秋声!”
    
    童子莫对,垂头而睡。但闻四壁虫声唧唧,如助予之叹息。
\end{normalsize}


\newpage

\textbf{译文}:

\vspace{1em}

\begin{normalsize}
    
    欧阳先生夜里正在读书,(忽然)听到有声音从西南方向传来,心里不禁悚然,仔细一听,说道:“奇怪啊!”一开始听起来像淅淅沥沥的雨声,夹杂着萧萧飒飒的风吹树木声,忽然就变得汹涌澎湃起来,像是江河夜间波涛突起、风雨骤然而至。碰到物体上发出铿锵之声,又好像金属撞击的声音,再(仔细)听,又像衔枚奔走去袭击敌人的军队,听不到任何号令声,只听见有人马行进的声音。(于是)我对童子说:“这是什么声音?你出去看看。”童子回答说:“月色皎皎、星光灿烂、浩瀚银河、高悬中天,四下里没有人的声音,那声音是从树林间传来的。”
    
    我叹道:“唉,可悲啊!这就是秋声呀,它为何而来呢(它怎么突然就来了呢)?大概是那秋天的样子,它的色调暗淡、烟飞云收;它的形貌清新明净、天空高远、日色明亮;它的气候寒冷、刺人肌骨;它的意境寂寞冷落,没有生气、川流寂静、山林空旷。所以它发出的声音时而凄凄切切,呼号发生迅猛,不可遏止。绿草浓密丰美,争相繁茂,树木青翠茂盛而使人快乐。然而,一旦秋风吹起,拂过草地,草就要变色;掠过森林,树就要落叶。它之所以能折断枝叶、凋落花草,使树木凋零,是因为一种构成天地万物的混然之气(秋气)的余威。秋天是刑官执法的季节,它在季节上说属于阴;秋天又是兵器和用兵的象征,在五行上属于金。这就是常说的天地之严凝之气,它常常以肃杀为意志。
    
    “唉!草木是无情之物,尚有衰败零落之时。人为动物,在万物中又最有灵性,无穷无尽的忧虑煎熬他的心绪,无数琐碎烦恼的事来劳累他的身体。只要内心被外物触动,就一定会动摇他的精神。更何况常常思考自己的力量所做不到的事情,忧虑自己的智慧所不能解决的问题?自然会使他红润的面色变得苍老枯槁,乌黑的头发变得鬓发花白。(既然这样,)为什么却要以并非金石的肌体,去像草木那样争一时的荣盛呢?(人)应当仔细考虑究竟是谁给自己带来了这么多残害,又何必去怨恨这秋声呢?”
    
    书童没有应答,低头沉沉睡去。只听得四壁虫鸣唧唧,像在附和我的叹息。
    
\end{normalsize}



\chapter{屈原列传}

\begin{normalsize}
    
    屈原者,名平,楚之同姓\footnote{〔楚之同姓〕楚王族本姓芈,楚武王熊通的儿子瑕封于屈,他的后代遂以屈为姓,瑕是屈原的祖先。楚国王族的同姓。屈、景、昭氏都是楚国的王族同姓。}也。为楚怀王\footnote{〔楚怀王〕楚威王的儿子,名熊槐(公元前328年至前299年在位)。}左徒\footnote{〔左徒〕楚国官名,职位仅次于令尹。}。博闻强志,明于治乱,娴于辞令。入则与王图议国事,以出号令;出则接遇宾客,应对诸侯。王甚任之。
    
    上官大夫与之同列,争宠而心害其能。怀王使屈原造为宪令,屈平属草稿未定。上官大夫\footnote{〔上官大夫〕楚大夫。}见而欲夺之,屈平不与,因谗之曰:“王使屈平为令,众莫不知。每一令出,平伐其功,曰以为‘非我莫能为也’。”王怒而疏屈平。
    
    屈平疾王听之不聪也,谗谄之蔽明也,邪曲之害公也,方正之不容也,故忧愁幽思而作《离骚》\footnote{〔《离骚》〕屈原的代表作,自叙生平的长篇抒情诗。关于诗题,后人有二说。一释“离”为“罹”的通假字,离骚就是遭受忧患。二是释“离”为离别,离骚就是离别的忧愁。}。“离骚”者,犹离忧也。夫天者,人之始也;父母者,人之本也。人穷则反本,故劳苦倦极,未尝不呼天也;疾痛惨怛,未尝不呼父母也。屈平正道直行,竭忠尽智以事其君,谗人间之,可谓穷矣。信而见疑,忠而被谤,能无怨乎?屈平之作《离骚》,盖自怨生也。《国风》好色而不淫,《小雅》怨诽而不乱。若《离骚》者,可谓兼之矣。上称帝喾\footnote{〔帝喾〕喾,古代传说中的帝王。相传是黄帝的曾孙,号高辛氏。},下道齐桓\footnote{〔齐桓〕即齐桓公(公元前685年至前643年在位),名小白,春秋五霸之一。},中述汤\footnote{〔汤〕商朝的开国君主。}、武\footnote{〔武〕指周武王,灭商建立西周王朝。},以刺世事。明道德之广崇,治乱之条贯,靡不毕见。其文约,其辞微,其志洁,其行廉。其称文小而其指极大,举类迩而见义远。其志洁,故其称物芳;其行廉,故死而不容。自疏濯淖污泥之中,蝉蜕于浊秽,以浮游尘埃之外,不获世之滋垢,皭然泥而不滓者也。推此志也,虽与日月争光可也。
    
    屈平既绌,其后秦欲伐齐。齐与楚从亲\footnote{〔从亲〕合纵相亲。从:同“纵”。当时楚、齐等六国联合抗秦,称为合纵,楚怀王曾为纵长。},惠王\footnote{〔惠王〕秦惠王(公元前337年至311年在位)。}患之。乃令张仪\footnote{〔张仪〕魏人,主张“连横”,游说六国事奉秦国,为秦惠王所重。}详去秦,厚币委质事楚,曰:“秦甚憎齐,齐与楚从亲,楚诚能绝齐,秦愿献商、於\footnote{〔商、於〕秦地名。商,在今陕西商州市东南。於,在今河南内乡东。}之地六百里。”楚怀王贪而信张仪,遂绝齐,使使如秦受地。张仪诈之曰:“仪与王约六里,不闻六百里。”楚使怒去,归告怀王。怀王怒,大兴师伐秦。秦发兵击之,大破楚师于丹、淅\footnote{〔丹、淅〕二水名。丹水发源于陕西商州市西北,东南流入河南。淅水,发源于南卢氏县,南流而入丹水。屈匄:(gài):楚大将军。汉中:今湖北西北部、陕西东南部一带。},斩首八万,虏楚将屈匄,遂取楚之汉中地。怀王乃悉发国中兵,以深入击秦,战于蓝田\footnote{〔蓝田〕秦县名,在今陕西蓝田西。}。魏闻之,袭楚至邓\footnote{〔邓〕春秋时蔡地,后属楚,在今河南邓州市一带。}。楚兵惧,自秦归。而齐竟怒,不救楚,楚大困。
    
    明年\footnote{〔明年〕指楚怀王十八年(公元前311年)。},秦割汉中地与楚以和。楚王曰:“不愿得地,愿得张仪而甘心焉。”张仪闻,乃曰:“以一仪而当汉中地,臣请往如楚。”如楚,又因厚币用事者臣靳尚\footnote{〔靳尚〕楚大夫。一说即上文的上官大夫。},而设诡辩于怀王之宠姬郑袖。怀王竟听郑袖,复释去张仪。是时屈原既疏,不复在位,使于齐,顾反,谏怀王曰:“何不杀张仪?”怀王悔,追张仪,不及。
    
    其后,诸侯共击楚,大破之,杀其将唐昧\footnote{〔唐昧〕楚将。楚怀王二十八年(公元前301年),秦、齐、韩、魏攻楚,杀唐昧。}。
    
    时秦昭王\footnote{〔秦昭王〕秦惠王之子,公元前306年至前251年在位。}与楚婚,欲与怀王会。怀王欲行,屈平曰:“秦,虎狼之国,不可信,不如毋行。”怀王稚子子兰劝王行:“奈何绝秦欢!”怀王卒行。入武关\footnote{〔武关〕秦国的南关,在今陕西省商州市东。},秦伏兵绝其后,因留怀王,以求割地。怀王怒,不听。亡走赵,赵不内。复之秦,竟死于秦而归葬。长子顷襄王\footnote{〔顷襄王〕名熊横,公元前298年至前262年在位。}立,以其弟子兰为令尹\footnote{〔令尹〕楚国的最高行政长官。}。楚人既咎子兰以劝怀王入秦而不反也,令尹子兰闻之,大怒,卒使上官大夫短屈原于顷襄王。顷襄王怒而迁之。
    
    屈平既嫉之,虽放流,眷顾楚国,系心怀王,不忘欲反。冀幸君之一悟,俗之一改也。其存君兴国,而欲反覆之,一篇之中,三致志焉。然终无可奈何,故不可以反。卒以此见怀王之终不悟也。人君无愚、智、贤、不肖,莫不欲求忠以自为,举贤以自佐。然亡国破家相随属,而圣君治国累世\footnote{〔世〕三十年为一世。}而不见者,其所谓忠者不忠,而所谓贤者不贤也。怀王以不知忠臣之分,故内惑于郑袖,外欺于张仪,疏屈平而信上官大夫、令尹子兰,兵挫地削,亡其六郡,身客死于秦,为天下笑。此不知人之祸也。《易》\footnote{〔《易》〕即《周易》,又称《易经》。这里引用的是《易经·井卦》的爻辞。}曰:“井泄不食,为我心恻,可以汲。王明,并受其福。”王之不明,岂足福哉!
    
    屈原至于江滨,被发行吟泽畔,颜色憔悴,形容枯槁。渔父见而问之曰:“子非三闾大夫\footnote{〔三闾大夫〕楚国掌管王族屈、景、昭三氏事务的官。}欤?何故而至此?”屈原曰:“举世混浊而我独清,众人皆醉而我独醒,是以见放。”渔父曰:“夫圣人者,不凝滞于物,而能与世推移。举世混浊,何不随其流而扬其波?众人皆醉,何不哺其糟而啜其醨?何故怀瑾握瑜,而自令见放为?”屈原曰:“吾闻之,新沐者必弹冠,新浴者必振衣。人又谁能以身之察察,受物之汶汶者乎?宁赴常流而葬乎江鱼腹中耳。又安能以皓皓之白,而蒙世之温蠖乎?”乃作《怀沙》\footnote{〔《怀沙》〕在今本《楚辞》中,是《九章》的一篇。}之赋。于是怀石,遂自投汨罗\footnote{〔汨罗〕江名,在湖南东北部,流经汨罗县入洞庭湖。}以死。
    
    屈原既死之后,楚有宋玉、唐勒、景差之徒者\footnote{〔宋玉、唐勒、景差〕相传为楚顷襄王时人,屈原的弟子,有《九辩》等作品传世。唐勒、景差:约与宋玉同时,都是当时的词赋家。},皆好辞而以赋见称。然皆祖屈原之从容辞令,终莫敢直谏。其后楚日以削,数十年竟为秦所灭\footnote{〔“数十年”句〕公元前223年秦灭楚。}。
    
    太史公\footnote{〔太史公〕司马迁自称。}曰:“余读《离骚》《天问》《招魂》《哀郢》\footnote{〔《天问》、《招魂》、《哀郢》〕都是屈原的作品。一说《招魂》为宋玉所作。《哀郢》是《九章》中的一篇。},悲其志。适长沙,观屈原所自沉渊,未尝不垂涕,想见其为人。及见贾生\footnote{〔贾生〕即贾谊(公元前200年前168年),洛阳(今河南洛阳东)人。西汉政论家、文学家。}吊之,又怪屈原以彼其材游诸侯,何国不容,而自令若是!读《鵩鸟赋》\footnote{〔《鵩鸟赋》〕贾谊所作。},同死生,轻去就,又爽然自失矣。”
\end{normalsize}


\newpage

\textbf{译文}:

\vspace{1em}

\begin{normalsize}
    
    屈原,名字叫平,与楚国的王族同姓。做楚怀王的左徒。(他)见识广博,记忆力很强,通晓国家治乱的道理,擅长外交辞令。对内,同楚王谋划商讨国家大事,颁发号令;对外,接待宾客,应酬答对各国诸侯。楚王很信任他。
    
    上官大夫和他职位相等,想争得楚王对他的宠爱,便心里嫉妒屈原的贤能。楚怀王派屈原制定国家的法令,屈原编写的草稿尚未定稿。上官大夫看见了,就想把草稿强取为己有,屈原不赞同。上官大夫就谗毁他说:“君王让屈原制定法令,大家没人不知道的,每出一道法令,屈原就炫耀自己的功劳,说:‘除了我,没有人能制定法令了’。”楚王听了很生气,因而疏远了屈原。
    
    屈原痛心楚怀王听信谗言,不能分辨是非,谄媚国君的人遮蔽了楚怀王的明见,品行不正的小人损害国家,端方正直的人不被昏君谗臣所容,所以忧愁深思,就创作了《离骚》。“离骚”,就是遭遇忧愁的意思。上天,是人的原始;父母,是人的根本。人处境困难时,总是要追念上天和父母(希望给以援助),所以劳累疲倦时,没有不呼叫上天的;病痛和内心悲伤时,没有不呼叫父母的。屈原使(自己)道德端正,使(自己)品行正直,竭尽忠心用尽智慧来侍奉他的国君,却被小人离间,可以说处境很困难。诚信而被怀疑,尽忠却被诽谤,能没有怨愤吗?屈原作《离骚》,大概是自己的怨愤所引起的。《诗经》中的《国风》,写男女恋情而不过度,《小雅》有怨刺之言,但不直接愤怒。屈原的《离骚》诗,则两者之美兼而有之。(他)远古提到帝喾,近古提到齐桓公,中古提到商汤、周武王,利用古代帝王这些事用来讽刺当世社会。阐明道德的广大崇高,治乱的条理,没有不全表现出来的。他的文章简约,语言含蓄,他的志趣高洁,行为正直。就其文字来看,不过是寻常事情,但是它的旨趣是极大的,列举的是近事,而表达的意思却十分深远。他的志趣高洁,所以作品中多用美人芳草作比喻;他的行为正直,所以至死不容于世。他自动地远离污泥浊水,像蝉脱壳那样摆脱污秽环境,以便超脱世俗之外,不沾染尘世的污垢,出于污泥而不染,依旧保持高洁的品德。推赞这种志行,即使同日月争光都可以。
    
    屈原被罢免了,后来秦国准备攻打齐国。齐国和楚国结成合纵联盟互相亲善,秦惠王对此担忧。就派张仪假装脱离秦国,用厚礼和信物呈献给楚王,对怀王说:“秦国非常憎恨齐国,齐国与楚国却合纵相亲,如果楚国确实能和齐国绝交,秦国愿意献上商、於之间的六百里土地。”楚怀王起了贪心,信任了张仪,就和齐国绝交,然后派使者到秦国接受土地。张仪抵赖说:“我和楚王约定的只是六里,没有听说过六百里。”楚国使者愤怒地离开秦国,回去报告怀王。怀王发怒,大规模出动军队去讨伐秦国。秦国发兵反击,在丹水和淅水一带大破楚军,杀了八万人,俘虏了楚国的大将屈匄,于是夺取了楚国的汉中一带。怀王又发动全国的兵力,深入秦地攻打秦国,交战于蓝田。魏国听到这一情况,袭击楚国一直打到邓地。楚军恐惧,从秦国撤退。齐国终于因为怀恨楚国,不来援救,楚国处境极端困窘。
    
    第二年,秦国割汉中之地与楚国讲和。楚王说:“我不愿得到土地,只希望得到张仪就甘心了。”张仪听说后,就说:“用一个张仪来抵当汉中地方,我请求到楚国去。”到了楚国,他又用丰厚的礼品贿赂当权的大臣靳尚,通过他在怀王宠姬郑袖面前编造了一套谎话。怀王竟然听信郑袖,又放走了张仪。这时屈原已被疏远,不在朝中任职,出使在齐国,回来后,劝谏怀王说:“为什么不杀张仪?”怀王很后悔,派人追张仪,已经来不及了。
    
    后来,各国诸侯联合攻打楚国,大败楚军,杀了楚国将领唐昧。
    
    当时秦昭王与楚国通婚,要求和怀王会面。怀王想去,屈原说:“秦国是虎狼一样的国家,不可信任,不如不去。”怀王的小儿子子兰劝怀王去,说:“怎么可以断绝和秦国的友好关系!”怀王终于前往。一进入武关,秦国的伏兵就截断了他的后路,于是扣留怀王,强求割让土地。怀王很愤怒,不听秦国的要挟。他逃往赵国,赵国不肯接纳。只好又到秦国,最后死在秦国,尸体运回楚国安葬。怀王的长子顷襄王即位,任用他的弟弟子兰为令尹。楚国人都抱怨子兰,因为他劝怀王入秦而最终未能回来。令尹子兰得知屈原怨恨他,非常愤怒,终于让上官大夫在顷襄王面前说屈原的坏话。
    
    屈原也为此怨恨子兰,虽然流放在外,仍然眷恋着楚国,心里挂念着怀王,念念不忘返回朝廷。他希望国君总有一天醒悟,世俗总有一天改变。屈原关怀君王,想振兴国家改变楚国的形势,一篇作品中,都再三表现出来这种想法。然而终于无可奈何,所以不能够返回朝廷。由此可以看出怀王始终没有觉悟啊。国君无论愚笨或明智、贤明或昏庸,没有不想求得忠臣来为自己服务,选拔贤才来辅助自己的。然而国破家亡的事接连发生,而圣明君主治理好国家的多少世代也没有出现,这是因为所谓忠臣并不忠,所谓贤臣并不贤。怀王因为不明白忠臣的职分,所以在内被郑袖所迷惑,在外被张仪所欺骗,疏远屈原而信任上官大夫和令尹子兰,军队被挫败,土地被削减,失去了六个郡,自己也被扣留死在秦国,为天下人所耻笑。这是不了解人(而导致)的祸害。《易经》说:“水井淘干净了,但是没人饮用,(这)让我心里感到难过,(因为这是)可以汲取饮用的。君主如果贤明,大家都能得到幸福。”现在君主是这样的不贤明,哪里还谈得上幸福呢!
    
    屈原到了江滨,披散头发,在水泽边一面走,一面吟咏着,脸色憔悴,形体面貌像枯死的树木一样毫无生气。渔父看见他,便问道:“您不是三闾大夫吗?为什么来到这儿?”屈原说:“整个世界都是混浊的,只有我一人清白;众人都沉醉,只有我一人清醒,因此被放逐。”渔父说:“聪明贤哲的人,不受外界事物的束缚,而能够随着世俗变化。整个世界都混浊,为什么不随大流而且推波助澜呢?众人都沉醉,为什么不吃点酒糟,喝点薄酒?为什么要怀抱美玉一般的品质,却使自己被放逐呢?”屈原说:“我听说,刚洗过头的一定要弹去帽上的灰沙,刚洗过澡的一定要抖掉衣上的尘土。谁能让自己清白的身躯,蒙受外物的污染呢?宁可投入长流的大江而葬身于江鱼的腹中。又哪能使自己高洁的品质,去蒙受世俗的尘垢呢?”于是他写了《怀沙》赋。因此抱着石头,就自投汨罗江而死。
    
    屈原死了以后,楚国(还)有宋玉、唐勒、景差一些人,都爱好文学,由于擅长写赋受到人们称赞。然而(他们)都效法屈原的委婉文辞,始终没有人敢于直谏。从这以后,楚国一天比一天缩小,几十年后,终于被秦国所灭亡。
    
    太史公说:我读《离骚》《天问》《招魂》《哀郢》,为他的志向不能实现而悲伤。到长沙,经过屈原自沉的地方,未尝不流下眼泪,追怀他的为人。看到贾谊凭吊他的文章,文中又责怪屈原如果凭他的才能去游说诸侯,哪个国家不会容纳,却自己选择了这样的道路!读了《鵩鸟赋》,把生和死等同看待,认为被贬和任用是不重要的,这又使我感到茫茫然失落什么了。
    
\end{normalsize}



\chapter{滕王阁序}

\begin{normalsize}
    
    豫章故郡\footnote{〔豫章故郡〕滕王阁在江西南昌市,汉时属豫章郡。},洪都新府\footnote{〔洪都新府〕唐改豫章郡为洪州,设都督府。}。星分翼轸\footnote{〔翼轸〕翼和轸都是星宿。},地接衡庐\footnote{〔衡庐〕衡山和庐山。}。襟三江\footnote{〔三江〕太湖的支流松江、娄江、东江,泛指长江中下游的江河。}而带五湖\footnote{〔五湖〕太湖、鄱阳湖、青草湖、丹阳湖、洞庭湖,泛指南方大湖。},控蛮荆\footnote{〔蛮荆〕古楚地,今湖北、湖南一带。}而引瓯越\footnote{〔瓯越〕古越地,即今浙江地区。古东越王建都于东瓯(今浙江省永嘉县),境内有瓯江。}。物华天宝,龙光射牛斗\footnote{〔牛斗〕牛、斗,星宿名。}之墟;人杰地灵,徐孺下陈蕃之榻\footnote{〔徐孺下陈蕃之榻〕徐孺子名稚,东汉豫章南昌人,当时隐士。据《后汉书·徐稚传》,东汉名士陈蕃为豫章太守,不接宾客,惟徐稚来访时,才设一睡榻,徐稚去后又悬置起来。}。雄州雾列,俊采星驰。台隍枕夷夏之交,宾主尽东南之美。都督阎公\footnote{〔都督阎公〕阎伯屿,时任洪州都督。}之雅望,棨戟遥临;宇文新州\footnote{〔宇文新州〕复姓宇文的新州(在今广东境内)刺史,名未详。}之懿范,襜帷暂驻。十旬休假,胜友如云;千里逢迎,高朋满座。腾蛟起凤,孟学士之词宗;紫电青霜,王将军之武库。家君作宰\footnote{〔家君作宰〕勃之父担任交趾县的县令。},路出名区;童子何知,躬逢胜饯。
    
    时维九月,序属三秋。潦水尽而寒潭清,烟光凝而暮山紫。俨骖騑于上路,访风景于崇阿。临帝子\footnote{〔帝子〕和下一句的“天人”都指滕王李元婴。}之长洲,得天人之旧馆。层峦耸翠,上出重霄;飞阁流丹,下临无地。鹤汀凫渚,穷岛屿之萦回;桂殿兰宫,即冈峦之体势。
    
    披绣闼,俯雕甍,山原旷其盈视,川泽纡其骇瞩。闾阎扑地,钟鸣鼎食之家;舸舰弥津,青雀黄龙之舳。云销雨霁,彩彻区明。落霞与孤鹜齐飞,秋水共长天一色。渔舟唱晚,响穷彭蠡\footnote{〔彭蠡〕即今鄱阳湖。}之滨,雁阵惊寒,声断衡阳\footnote{〔衡阳〕今属湖南省,境内有回雁峰,相传秋雁到此就不再南飞,待春而返。}之浦。
    
    遥襟甫畅,逸兴遄飞。爽籁发而清风生,纤歌凝而白云遏。睢园\footnote{〔睢园〕汉梁孝王菟园,梁孝王曾在园中聚集文人饮酒赋诗。}绿竹,气凌彭泽\footnote{〔彭泽〕县名,在今江西湖口县东,此代指陶潜。陶潜,即陶渊明,曾官彭泽县令,世称陶彭泽。}之樽;邺水\footnote{〔邺水〕在邺下(今河北省临漳县)。邺下是曹魏兴起的地方,三曹常在此雅集作诗。曹植在此作《公宴诗》。}朱华,光照临川\footnote{〔临川〕郡名,治所在今江西省抚州市,代指即谢灵运。谢灵运曾任临川内史。}之笔。四美具,二难并。穷睇眄于中天,极娱游于暇日。天高地迥,觉宇宙之无穷;兴尽悲来,识盈虚之有数。望长安于日下,目吴会\footnote{〔吴会〕秦汉会稽郡治所在吴县,郡县连称为吴会。吴郡,治所在今江苏省苏州市。}于云间。地势极而南溟深,天柱高而北辰远。关山难越,谁悲失路之人;萍水相逢,尽是他乡之客。怀帝阍\footnote{〔帝阍〕天帝的守门人,此处借指皇帝的宫门。}而不见,奉宣室\footnote{〔奉宣室〕代指入朝做官。贾谊迁谪长沙四年后,汉文帝复召他回长安,于宣室中问鬼神之事。宣室,汉未央宫正殿,为皇帝召见大臣议事之处。}以何年?
    
    嗟乎!时运不齐,命途多舛。冯唐易老\footnote{〔冯唐易老〕冯唐在汉文帝、汉景帝时不被重用,汉武帝时被举荐,已是九十多岁。},李广难封\footnote{〔李广难封〕李广,汉武帝时名将,多次与匈奴作战,军功卓著,却始终未获封爵。}。屈贾谊于长沙\footnote{〔屈贾谊于长沙〕贾谊在汉文帝时被贬为长沙王太傅。},非无圣主;窜梁鸿于海曲\footnote{〔梁鸿〕东汉人,作《五噫歌》讽刺朝廷,因此得罪汉章帝,避居齐鲁、吴中。},岂乏明时?所赖君子见机,达人知命。老当益壮,宁移白首之心?穷且益坚,不坠青云之志。酌贪泉而觉爽\footnote{〔贪泉〕贪泉,在广州附近的石门,传说饮此水会贪得无厌,晋时吴隐之喝下此水操守反而更加坚定。},处涸辙以犹欢\footnote{〔处涸辙〕《庄子·外物》:“”}。北海虽赊,扶摇可接;东隅已逝,桑榆非晚。孟尝高洁\footnote{〔孟尝〕据《后汉书·孟尝传》,孟尝字伯周,东汉会稽上虞人。曾任合浦太守,以廉洁奉公著称,后因病隐居。桓帝时,虽有人屡次荐举,终不见用。},空余报国之情;阮籍猖狂\footnote{〔阮籍〕字嗣宗,晋代名士,不满世事,佯装狂放,常驾车出游,路不通时就痛哭而返。《晋书·阮籍传》:籍“时率意独驾,不由径路。车迹所穷,辄恸哭而反。”},岂效穷途之哭!
    
    勃,三尺微命,一介书生。无路请缨,等终军\footnote{〔终军〕字子云,汉代济南人。武帝时出使南越,自请“愿受长缨,必羁南越王而致之阙下”,时仅二十馀岁。}之弱冠\footnote{〔弱冠〕古人二十岁行冠礼,表示成年,二十岁称“弱冠”。};有怀投笔\footnote{〔宗悫〕字元干,南朝宋南阳人,年少时向叔父自述志向,云“愿乘长风破万里浪”。后因战功受封。},慕宗悫\footnote{〔投笔〕《后汉书·班超传》有班超投笔从戎的故事。}之长风。舍簪笏于百龄,奉晨昏于万里。非谢家之宝树\footnote{〔谢家之宝树〕指谢玄,比喻好子弟。《世说新语·言语》:“谢太傅(安)问诸子侄‘子弟亦何预人事,而正欲使其佳?’诸人莫有言者。车骑(谢玄)答曰:‘譬如芝兰玉树,欲使其生于庭阶耳。’”},接孟氏之芳邻\footnote{〔孟氏之芳邻〕孟轲的母亲为教育儿子而三迁择邻,最后定居于学宫附近。}。他日趋庭,叨陪鲤对\footnote{〔他日趋庭,叨陪鲤对〕鲤,孔鲤,孔子之子。趋庭,受父亲教诲。《论语·季氏》:“(孔子)尝独立,(孔)鲤趋而过庭。(子)曰:‘学诗乎?’对曰:‘未也。’‘不学诗,无以言。’鲤退而学诗。他日,又独立,鲤趋而过庭。(子)曰:‘学礼乎?’对曰:‘未也。’‘不学礼,无以立。’鲤退而学礼。”};今兹捧袂,喜托龙门\footnote{〔喜托龙门〕《后汉书·李膺传》:“膺以声名自高,士有被其容接者,名为登龙门。”}。杨意不逢\footnote{〔杨意不逢……〕杨意,杨得意的省称。凌云,指司马相如作《大人赋》。据《史记·司马相如列传》,司马相如经蜀人杨得意引荐,方能入朝见汉武帝。又云:“相如既奏《大人》之颂,天子大悦,飘飘有凌云之气。”},抚凌云而自惜。钟期既遇\footnote{〔钟期既遇……〕钟期,钟子期的省称。《列子·汤问》:“伯牙善鼓琴,钟子期善听。伯牙鼓琴……志在流水,钟子期曰:‘善哉!洋洋兮若江河。’”},奏流水以何惭?
    
    呜乎!胜地不常,盛筵难再。兰亭\footnote{〔兰亭〕在今浙江省绍兴市附近。晋穆帝永和九年(353)三月三日上巳节,王羲之与群贤宴集于此,写有名篇《兰亭集序》。}已矣,梓泽\footnote{〔梓泽〕即晋代石崇的金谷园,故址在今河南省洛阳市西北。}丘墟。临别赠言,幸承恩于伟饯;登高作赋,是所望于群公。敢竭鄙怀,恭疏短引;一言均赋,四韵俱成\footnote{〔四韵俱成〕四韵一起写好了。四韵,八句四韵诗,指王勃此时写下的《滕王阁诗》:“滕王高阁临江渚,佩玉鸣鸾罢歌舞。画栋朝飞南浦云,珠帘暮卷西山雨。闲云潭影日悠悠,物换星移几度秋。阁中帝子今何在?槛外长江空自流。”}。请洒潘江,各倾陆海云尔\footnote{〔请洒潘江……〕钟嵘《诗品》:“陆(机)才如海,潘(岳)才如江。”这里形容各宾客的文采。}。
\end{normalsize}


\newpage

\textbf{译文}:

\vspace{1em}

\begin{normalsize}
    
    汉代的豫章旧郡,现在称洪都府。它处在翼、轸二星的分管区域,与庐山和衡山接壤。以三江为衣襟,以五湖为腰带,控制楚地,连接瓯越。这里地上物产的精华,乃是天的宝物,宝剑的光气直射牛、斗二星之间;人有俊杰是因为地有灵秀之气,徐孺子竟然在太守陈蕃家下榻(世说新语记载,太守陈蕃赏识徐孺子,专门为其在家中设置榻,当徐孺子来的时候,就将榻放下来,徐孺子走了就将榻吊起来,此处应该是称赞滕王阁的东道主欣赏才俊,也有夸赞宾客的成分)。雄伟的州城像雾一样涌起,杰出的人才像星星一样多。城池倚据在荆楚和华夏交接的地方,宴会上客人和主人都是东南一带的俊杰。声望崇高的阎都督公,(使)打着仪仗(的高人)远道而来;德行美好的宇文新州刺史,(让)驾着车马(的雅士)也在此暂时驻扎。正好赶上十日一休的假日,才华出众的朋友多得如云;迎接千里而来的客人,尊贵的朋友坐满宴席。文章的辞彩如蛟龙腾空、凤凰飞起,那是文词宗主孟学士;紫电和清霜这样的宝剑,出自王将军的武库里。家父做交趾县令,我探望父亲路过这个有名的地方(指洪州);我年幼无知,(却有幸)参加这场盛大的宴会。
    
    时间是九月,季节为深秋。蓄积的雨水已经消尽,潭水寒冷而清澈,烟光雾气凝结,傍晚的山峦呈现出紫色。驾着豪华的马车行驶在高高的道路上,到崇山峻岭中观望风景。来到滕王营建的长洲上,看见他当年修建的楼阁。重叠的峰峦耸起一片苍翠,上达九霄;凌空架起的阁道上,朱红的油彩鲜艳欲滴,从高处往下看,地好像没有了似的。仙鹤野鸭栖止的水边平地和水中小洲,极尽岛屿曲折回环的景致;桂树与木兰建成的宫殿,随着冈峦高低起伏的态势。
    
    打开精美的阁门,俯瞰雕饰的屋脊,放眼远望辽阔的山原充满视野,迂回的河流湖泊使人看了惊叹。房屋排满地面,有不少官宦人家;船只布满渡口,都装饰着青雀黄龙的头形。云消雨散,阳光普照,天空明朗。落霞与孤独的野鸭一齐飞翔,秋天的江水和辽阔的天空浑然一色。渔船唱着歌傍晚回来,歌声响遍鄱阳湖畔;排成行列的大雁被寒气惊扰,叫声消失在衡山南面的水边。
    
    远望的胸怀顿时舒畅,飘逸的兴致油然而生。排箫发出清脆的声音,引来阵阵清风;纤细的歌声仿佛凝住不散,阻止了白云的飘动。今日的宴会很像是当年睢园竹林的聚会,在座的诗人文士狂饮的气概压过了陶渊明;又有邺水的曹植咏荷花那样的才气,文采可以直射南朝诗人谢灵运。良辰、美景、赏心、乐事,四美都有,贤主、嘉宾,难得却得。放眼远望半空中,在闲暇的日子里尽情欢乐。天高地远,感到宇宙的无边无际;兴致已尽,悲随之来,认识到事物的兴衰成败有定数。远望长安在夕阳下,遥看吴越在云海间。地势偏远,南海深不可测;天柱高耸,北极星远远悬挂。雄关高山难以越过,有谁同情不得志的人?在座的各位如浮萍在水上相聚,都是客居异乡的人。
    
    唉!命运不顺畅,路途多艰险。冯唐容易老,李广封侯难。把贾谊贬到长沙,并非没有圣明的君主;让梁鸿到海边隐居,难道不是在政治昌明的时代?能够依赖的是君子察觉事物细微的先兆,通达事理的人知道社会人事的规律。老了应当更有壮志,哪能在白发苍苍时改变自己的心志?处境艰难反而更加坚强,不放弃远大崇高的志向。喝了贪泉的水,仍然觉得心清气爽;处在干涸的车辙中,还能乐观开朗。北海虽然遥远,乘着旋风仍可以到达;少年的时光虽然已经消逝,珍惜将来的岁月还不算晚。孟尝品行高洁,却空有一腔报国的热情;怎能效法阮籍狂放不羁,在无路可走时便恸哭而返?
    
    我,地位低下,一个书生。没有请缨报国的机会,虽然和终军的年龄相同;像班超那样有投笔从戎的胸怀,也仰慕宗悫“乘风破浪”的志愿。宁愿舍弃一生的功名富贵,到万里之外去早晚侍奉父亲。不敢说是谢玄那样的人才,却结识了诸位名家。过些天到父亲那里聆听教诲,一定要像孔鲤那样趋庭有礼,对答如流;今天举袖作揖谒见阎公,好像登上龙门一样。司马相如倘若没有遇到杨得意那样引荐的人,虽有文才也只能独自叹惋。既然遇到钟子期那样的知音,演奏高山流水的乐曲又有什么羞惭呢?
    
    唉!名胜的地方不能长存,盛大的宴会难以再遇。当年兰亭宴饮集会的盛况已成为陈迹了,繁华的金谷园也成为荒丘废墟。临别赠言,作为有幸参加这次盛宴的纪念;登高作赋,那就指望在座的诸公了。冒昧给大家献丑,恭敬地写下这篇小序,我的一首四韵小诗也已写成。请各位像潘岳、陆机那样,展现如江似海的文才吧。
    
\end{normalsize}



\chapter{五蠹}

\begin{normalsize}
    
    上古之世,人民少而禽兽众,人民不胜禽兽虫蛇。有圣人作,构木为巢以避群害,而民悦之,使王天下,号曰有巢氏。民食果蓏蚌蛤,腥臊恶臭而伤害腹胃,民多疾病。有圣人作,钻燧取火以化腥臊,而民说之,使王天下,号之曰燧人氏。中古之世,天下大水,而鲧、禹决渎。近古之世,桀、纣暴乱,而汤、武征伐。今有构木钻燧于夏后氏之世者,必为鲧、禹笑矣;有决渎于殷、周之世者,必为汤、武笑矣。然则今有美尧、舜、汤、武、禹之道于当今之世者,必为新圣笑矣。是以圣人不期修古,不法常可,论世之事,因为之备。宋有人耕田者,田中有株,兔走触株,折颈而死,因释其耒而守株,冀复得兔。兔不可复得,而身为宋国笑。今欲以先王之政,治当世之民,皆守株之类也。
     
    古者丈夫不耕,草木之实足食也;妇人不织,禽兽之皮足衣也。不事力而养足,人民少而财有余,故民不争。是以厚赏不行,重罚不用,而民自治。今人有五子不为多,子又有五子,大父未死而有二十五孙。是以人民众而货财寡,事力劳而供养薄,故民争,虽倍赏累罚而不免于乱。
     
    尧之王天下也,茅茨不翦,采椽不斫;粝粢之食,䔧藿之羹;冬日麂裘,夏日葛衣;虽监门之服养,不亏于此矣。禹之王天下也,身执耒歃以为民先,股无肢,胫不生毛,虽臣虏之劳,不苦于此矣。以是言之,夫古之让天子者,是去监门之养,而离臣虏之劳也,古传天下而不足多也。今之县令,一日身死,子孙累世絜驾,故人重之。是以人之于让也,轻辞古之天子,难去今之县令者,薄厚之实异也。夫山居而谷汲者,腊而相遗以水;泽居苦水者,买庸而决窦。故饥岁之春,幼弟不饷;穰岁之秋,疏客必食。非疏骨肉爱过客也,多少之实异也。是以古之易财,非仁也,财多也;今之争夺,非鄙也,财寡也。轻辞天子,非高也,势薄也;争士橐,非下也,权重也。故圣人议多少、论薄厚为之政。故罚薄不为慈,诛严不为戾,称俗而行也。故事因于世,而备适于事。
     
    古者大王\footnote{〔大王〕指周文王。丰、镐是丰京和镐京的统称,周朝的都城,在今日陕西西安市。}处丰、镐之间,地方百里,行仁义而怀西戎,遂王天下。徐偃王\footnote{〔徐偃王〕西周时期徐国的国君。传说周穆王巡视各国,听闻徐君威德日远,遣楚国袭其不备,大破之,杀偃王。一说是宋王偃之误。公元前286年,齐联合魏、楚灭宋。宋王偃战败,死于温。}处汉东,地方五百里,行仁义,割地而朝者三十有六国。荆文王\footnote{〔荆文王〕可能指楚文王或楚庄王。}恐其害己也,举兵伐徐,遂灭之。故文王行仁义而王天下,偃王行仁义而丧其国,是仁义用于古不用于今也。故曰:世异则事异。当舜之时,有苗\footnote{〔有苗〕古国名,又称“三苗”。传说尧禅位于舜,三苗不服作乱。禹征三苗后灭亡。有认为现代苗族是有苗文化的孑遗。}不服,禹将伐之。舜曰:“不可。上德不厚而行武,非道也。”乃修教三年,执干戚舞,有苗乃服。共工\footnote{〔共工〕传说中炎帝的后裔,祝融的儿子,尧的臣子。韩非子认为共工之战是中古之事。}之战,铁铦矩者及乎敌,铠甲不坚者伤乎体。是干戚用于古不用于今也。故曰:事异则备变。上古竞于道德,中世逐于智谋,当今争于气力。齐将攻鲁,鲁使子贡\footnote{〔子贡〕端木赐,字子贡,春秋时卫国人,孔子的弟子。贤能善辩,曾任鲁国、卫国的相国,还善于经商。}说之。齐人曰:“子言非不辩也,吾所欲者土地也,非斯言所谓也。”遂举兵伐鲁,去门十里以为界。故偃王仁义而徐亡,子贡辩智而鲁削。以是言之,夫仁义辩智,非所以持国也。去偃王之仁,息子贡之智,循徐、鲁之力使敌万乘,则齐、荆之欲不得行于二国矣。
    
    夫古今异俗,新故异备。如欲以宽缓之政,治急世之民,犹无辔策而御駻马,此不知之患也。且民者固服于势,寡能怀于义。仲尼\footnote{〔仲尼〕孔子字仲尼。},天下圣人也,修行明道以游海内,海内说其仁、美其义而为服役者七十人。盖贵仁者寡,能义者难也。故以天下之大,而为服役者七十人,而仁义者一人。鲁哀公\footnote{〔鲁哀公〕春秋时期鲁国君主(公元前494年至前468年在位),鲁定公之子。},下主也,南面君国,境内之民莫敢不臣。民者固服于势,诚易以服人,故仲尼反为臣而哀公顾为君。仲尼非怀其义,服其势也。故以义则仲尼不服于哀公,乘势则哀公臣仲尼。今学者之说人主也,不乘必胜之势,而务行仁义则可以王,是求人主之必及仲尼,而以世之凡民皆如列徒,此必不得之数也。
    
    儒以文乱法,侠以武犯禁,而人主兼礼之,此所以乱也。夫离法者罪,而诸先王以文学取;犯禁者诛,而群侠以私剑养。故法之所非,君之所取;吏之所诛,上之所养也。法、趣、上、下,四相反也,而无所定,虽有十黄帝不能治也。故行仁义者非所誉,誉之则害功;文学者非所用,用之则乱法。楚之有直躬,其父窃羊,而谒之吏。令尹曰:“杀之!”以为直于君而曲于父,报而罪之。以是观之,夫君之直臣,父之暴子也。鲁人从君战,三战三北。仲尼问其故,对曰:“吾有老父,身死莫之养也。”仲尼以为孝,举而上之。以是观之,夫父之孝子,君之背臣也。故令尹诛而楚奸不上闻,仲尼赏而鲁民易降北。上下之利,若是其异也,而人主兼举匹夫之行,而求致社稷之福,必不几矣。
\end{normalsize}


\newpage

\textbf{译文}:

\vspace{1em}

\begin{normalsize}
    
    上古时代,人民少而禽兽多,人民经受不住禽兽虫蛇的侵害。有位圣人起来,用树枝搭成像鸟巢一样的住处来避免各种禽兽的侵害,人民就爱戴他,让他统治天下,号称有巢氏。人民吃野生的瓜果和河里的蛤蜊,有腥臊难闻的气味而伤害肠胃,人民有很多疾病。有位圣人起来,钻木取火,烧熟食物以去掉腥臊气味,人民就爱戴他,让他统治天下,号称燧人氏。中古时代,天下出现洪水,鲧和禹疏通河道。近古时代,夏桀和商纣残暴昏乱,商汤和周武王起兵征讨。假如在夏朝时还有构木为巢、钻燧取火的人,一定会被鲧和禹所嘲笑;假如在殷商和周代还有像鲧和禹那样疏通河道的人,一定会被商汤和周武王所嘲笑。然而,假如当今之世有人赞美尧、舜、汤、武、禹那一套办法,也一定会被新时代的圣人所嘲笑。因此,新时代的圣人不羡慕远古时代,不效法永恒不变的常规,而是研究当代的实际情况,从而采取相应的措施。宋国有个农民,他的田地里有一个树桩,有一天一只兔子奔跑时撞到树桩上,碰断脖子死了,这个农民因此就放下农具而守候在树桩旁,希望再次得到死兔。兔子当然不可能再得到了,而他自己却受到宋国人嘲笑。现在还想用先王的政治来治理当代的民众,也就像守株待兔一样可笑。
    
    古时侯男人不耕地,是因为草木的果实充足够吃;妇女不纺织,是因为禽兽的毛皮充足够穿。不用费力劳动而生活给养就很充足,人民数量少而财物有多余的,所以人民不争夺。因此不必施行厚赏,也不用进行重罚,人民自然安定。现在的人一个人有五个孩子不算多,每个孩子分别又有五个孩子,祖父还没死就有二十五个孙子。因此,人民数量增多而财物缺少,费力劳动而供养微薄,所以人民相互争夺,即使加倍地奖赏和多次地惩罚,也难免祸乱。
    
    尧统治天下时,茅草屋顶不用修剪,栎木椽子不用砍削;吃的是粗粮,喝的是野菜汤;冬天披的是质量很差的兽皮衣,夏天穿的是用葛纤维做的粗布衣;现在即使是看门的人吃穿也不会比这更差了。禹统治天下时,自己拿着农具走在民众的前面,累得大腿肌肉消瘦,小腿上汗毛也磨掉了,现在即使是奴隶的劳动,也不比这更苦。就此而言,古人让出天子王位,不过是去掉看门人那样的供养,离开奴隶般的劳苦而已,所以古代人把天下传给别人也不值得赞扬。当今的县令,即便有一天过世了,他的子孙接连几代都会有马车坐,所以人们看重县令的位置。因此,人们对于让位这件事,很容易辞掉古代的天子,却很难辞去现在的县令,这是因为利益待遇的大小实在很不相同啊。住在山上要到深谷去打水的人,遇到节日,用水作礼物互相赠送;住在洼地苦于水涝的人,却要雇人去挖渠排水。所以荒年的春天,对自己幼小的弟弟也没有食物供给;丰年的秋天,对来往很少的远方客人也一定招待吃喝。这不是疏远自己的亲人偏爱过路的客人,而是因为收成的多少实在很不相同啊。因此,古人轻视财物,并不是讲仁慈,而是财物很多;今人争夺财物,并不是太贪吝,而是财物太少。轻易地辞掉天子职位,不是什么品德高尚,而是因为古代的权势很小;争夺官职和依附权贵,不是什么品德卑下,而是因为今天的权势很大。所以圣人研究社会财富的多少,考虑权势的轻重,然后制定相应的政治措施。所以惩罚轻不是仁慈,诛杀严不是凶暴,是适应社会情况而行事。因此,政事随着时代的变化而变化,措施必须适应已经变化了的政事。
    
\end{normalsize}



\chapter{项脊轩志}

\begin{normalsize}
    
    项脊轩\footnote{〔项脊轩〕归有光家的书斋名。},旧南阁子也。室仅方丈,可容一人居。百年老屋,尘泥渗漉,雨泽下注;每移案,顾视无可置者。又北向,不能得日,日过午已昏。余稍为修葺,使不上漏。前辟四窗,垣墙周庭,以当南日,日影反照,室始洞然。又杂植兰桂竹木于庭,旧时栏楯,亦遂增胜。借书满架,偃仰啸歌,冥然兀坐,万籁有声;而庭阶寂寂,小鸟时来啄食,人至不去。三五之夜\footnote{〔三五之夜〕农历每月十五的夜晚。},明月半墙,桂影斑驳,风移影动,珊珊可爱。
    
    然余居于此,多可喜,亦多可悲。先是庭中通南北为一。迨诸父异爨\footnote{〔异爨〕分灶做饭,比喻分家。},内外多置小门,墙往往而是;东犬西吠\footnote{〔东犬西吠〕东边的狗对着西边叫。意思是分家后,狗把原住同一庭院的人当作陌生人。},客逾庖而宴,鸡栖于厅。庭中始为篱,已为墙,凡再变矣。家有老妪,尝居于此。妪,先大母婢也,乳二世,先妣抚之甚厚。室西连于中闺,先妣尝一至。妪每谓余曰:“某所,而母立于兹。”妪又曰:“汝姊在吾怀,呱呱而泣;娘以指叩门扉曰:‘儿寒乎~欲食乎~’吾从板外相为应答。”语未毕,余泣,妪亦泣。余自束发\footnote{〔束发〕古代男孩成童时束发为髻。}读书轩中,一日,大母过余曰:“吾儿,久不见若影,何竟日默默在此,大类女郎也?”比去,以手阖门,自语曰:“吾家读书久不效,儿之成,则可待乎!”顷之,持一象笏至,曰:“此吾祖太常公宣德间执此以朝,他日汝当用之!”瞻顾遗迹,如在昨日,令人长号不自禁。
    
    轩东故尝为厨,人往,从轩前过。余扃牖而居,久之,能以足音辨人。轩凡四遭火,得不焚,殆有神护者。
    
    项脊生曰:“蜀清守丹穴,利甲天下,其后秦皇帝筑女怀清台。刘玄德与曹操争天下,诸葛孔明起陇中。方二人之昧昧于一隅也,世何足以知之,余区区处败屋中,方扬眉、瞬目,谓有奇景。人知之者,其谓与坎井之蛙何异?”
    
    余既为此志,后五年,吾妻来归\footnote{〔来归〕嫁到我家来。归:古代女子出嫁。},时至轩中,从余问古事,或凭几学书。吾妻归宁\footnote{〔归宁〕出嫁的女儿回娘家省亲。},述诸小妹语曰:“闻姊家有阁子,且何谓阁子也?”其后六年,吾妻死,室坏不修。其后二年,余久卧病无聊,乃使人复葺南阁子,其制稍异于前。然自后余多在外,不常居。
    
    庭有枇杷树,吾妻死之年所手植也,今已亭亭如盖矣。
\end{normalsize}


\newpage

\textbf{译文}:

\vspace{1em}

\begin{normalsize}
    
    项脊轩,是原来的南阁子。室内面积只有一丈见方,可以容纳一个人居住。这座百年老屋,(屋顶墙上的)泥土从上边漏下来,雨水也一直往下流;我每次动书桌,环视四周都没有可以安置桌案的地方。又屋子方位朝北,不能被阳光照到,一过了中午(屋内)就已昏暗。我稍稍修理了一下,使它不从上面漏土漏雨。向前开了四扇窗子,用矮墙在庭院周围环绕,用来挡住南面射来的日光,日光反射照耀,室内才明亮起来。又在庭院里随意地种上兰花、桂树、竹子等草木,往日的栏杆,也增加了新的光彩。书摆满了书架,我仰头高声吟诵诗歌,有时又静静地独自端坐,听自然界各种各样的声音;庭院、台阶前静悄悄的,小鸟不时飞下来啄食,人走到它跟前也不离开。农历十五的夜晚,明月高悬,照亮半截墙壁,桂树的影子交杂错落,微风吹过影子摇动,可爱极了。
    
    然而我住在这里,有许多值得高兴的事,也有许多悲伤的事。在这以前,庭院南北相通成为一体。等到伯父、叔父们分了家,室内外设置了许多小门,墙壁到处都是;狗把原住同一庭院的人当作陌生人,客人得越过厨房去吃饭,鸡在厅堂里栖息。庭院中开始是篱笆隔开,然后又砌成了墙,一共变了两次。家中有个老婆婆,曾经在这里居住过。这个老婆婆,是我已经去世的祖母的婢女,给两代人喂过奶,先母对她很好。房子的西边和内室相连,去世的母亲曾经来过这里。老婆婆常常对我说:“这个地方,你母亲曾经站在这里。”老婆婆又说:“你姐姐在我怀中,呱呱地哭泣;你母亲用手指敲着房门说:‘孩子是冷呢,还是想吃东西呢?’我隔着门一一回答。”话还没有说完,我就哭起来,老婆婆也流下了眼泪。我从十五岁起就在轩内读书,有一天,祖母来看我,说:“我的孩子,好久没有见到你的身影了,为什么整天默默地呆在这里,像个女孩子一样呀?”等到离开时,用手关上门,自言自语地说:“我家祖祖辈辈都读书,长期以来却没有成就功名,这孩子取得成就,指日可待了!”不一会,拿着一个象笏过来,说:“这是我的祖父太常公宣德年间拿着去朝见皇帝用的,以后你一定会用到它!”
    
    项脊轩的东边曾经是厨房,人们到那里去,必须从轩前经过。我关着窗子住在里面,时间长了,能够根据脚步声辨别是谁。项脊轩一共遭过四次火灾,能够不被焚毁,大概是有神灵在保护着吧。
    
    我认为:巴蜀寡妇名叫清的,守着丹砂矿井,得到的好处为天下第一,后来秦始皇为表彰她而筑了女怀清台。刘备和曹操争夺天下,诸葛孔明从隆中出山建功立业。当这两个人无声无息地住在偏僻的地方时,世人哪里能知道他们?我住在这小小的破屋中,当我扬眉眨眼时,认为这破屋中自有不平凡的事物。
    
    我作了这篇文章之后,过了五年,我的妻子嫁到我家来,她时常来到轩中,向我问一些旧时的事情,有时伏在桌旁学写字。我妻子回娘家探亲,回来转述她的小妹妹们的话说:“听说姐姐家有个小阁楼,那么,什么叫小阁楼呢?”这以后六年,我的妻子去世,项脊轩破败没有整修。又过了两年,我很长时间生病卧床没有什么(精神上的)寄托,就派人再次修缮南阁子,格局跟过去稍有不同。然而这之后我多在外边,不常住在这里。
    
    庭院中有一株枇杷树,是我妻子去世那年她亲手种的,如今已经高高挺立着,枝叶繁茂像伞一样了。
    
\end{normalsize}



\chapter{促织}

\begin{normalsize}
    
    宣德\footnote{〔宣德〕明宣宗年号。}间,宫中尚促织之戏,岁征民间。此物故非西\footnote{〔西〕这里指陕西。}产;有华阴令\footnote{〔华阴令〕华阴县县令。县令:一县之长。}欲媚上官,以一头进,试使斗而才,因责常供。令以责之里正\footnote{〔里正〕一里之长。}。市中游侠儿\footnote{〔游侠儿〕指游手好闲、不务正业的年轻人。}得佳者笼养之,昂其直,居为奇货。里胥\footnote{〔里胥〕管理乡里事务的差吏。}猾黠,假此科敛丁口\footnote{〔科敛丁口〕按人头征税,摊派费用。丁,成年男子。口:家里人。},每责一头,辄倾数家之产。
    
    邑有成名者,操童子业\footnote{〔童子业〕指读书考秀才。童子,童生。},久不售。为人迂讷,遂为猾胥报充里正役,百计营谋不能脱。不终岁,薄产累尽。会征促织,成不敢敛户口,而又无所赔偿,忧闷欲死。妻曰:“死何裨益?不如自行搜觅,冀有万一之得。”成然之。早出暮归,提竹筒铜丝笼,于败堵丛草处,探石发穴,靡计不施,迄无济。即捕得三两头,又劣弱不中于款。宰\footnote{〔宰〕县令。}严限追比\footnote{〔追比〕旧时地方官吏严逼人民,限期交税、交差、逾期受杖责。},旬余,杖至百,两股间脓血流离,并虫亦不能行捉矣。转侧床头,惟思自尽。
    
    时村中来一驼背巫,能以神卜。成妻具资诣问。见红女白婆,填塞门户。入其舍,则密室垂帘,帘外设香几。问者爇香于鼎,再拜。巫从旁望空代祝,唇吻翕辟,不知何词。各各竦立以听。少间,帘内掷一纸出,即道人意中事,无毫发爽。成妻纳钱案上,焚拜如前人。食顷,帘动,片纸抛落。拾视之,非字而画:中绘殿阁,类兰若\footnote{〔兰若〕寺庙,即梵语“阿兰若”。};后小山下,怪石乱卧,针针丛棘,青麻头\footnote{〔青麻头〕和下文的“蝴蝶”、“螳螂”、“油利垯”、“青丝额”,都是上品蟋蟀的名字。}伏焉;旁一蟆,若将跳舞。展玩不可晓。然睹促织,隐中胸怀,折藏之,归以示成。
    
    成反复自念,得无教我猎虫所耶?细瞻景状,与村东大佛阁真逼似。乃强起扶杖,执图诣寺后,有古陵蔚起。循陵而走,见蹲石鳞鳞,俨然类画。遂于蒿莱中侧听徐行,似寻针芥。而心目耳力俱穷,绝无踪响。冥搜未已,一癞头蟆猝然跃去。成益愕,急逐趁之,蟆入草间。蹑迹披求,见有虫伏棘根。遽扑之,入石穴中。掭以尖草,不出;以筒水灌之,始出,状极俊健。逐而得之。审视,巨身修尾,青项金翅。大喜,笼归,举家庆贺,虽连城拱璧不啻也,上于盆而养之,蟹白栗黄,备极护爱,留待限期,以塞官责。
    
    成有子九岁,窥父不在,窃发盆。虫跃掷径出,迅不可捉。及扑入手,已股落腹裂,斯须就毙。儿惧,啼告母。母闻之,面色灰死,大骂曰:“业根,死期至矣!而翁归,自与汝复算耳!”儿涕而出。
    
    未几,成归,闻妻言,如被冰雪。怒索儿,儿渺然不知所往。既得其尸于井,因而化怒为悲,抢呼欲绝。夫妻向隅,茅舍无烟,相对默然,不复聊赖。日将暮,取儿藁葬。近抚之,气息惙然。喜置榻上,半夜复苏。夫妻心稍慰,但儿神气痴木,奄奄思睡。成蟋蟀笼虚,顾之则气断声吞,亦不敢复究儿。自昏达曙,目不交睫。东曦既驾,僵卧长愁。忽闻门外虫鸣,惊起觇视,虫宛然尚在。喜而捕之,一鸣辄跃去,行且速。覆之以掌,虚若无物;手裁举,则又超忽而跃。急趋之,折过墙隅,迷其所在。徘徊四顾,见虫伏壁上。审谛之,短小,黑赤色,顿非前物。成以其小,劣之。惟彷徨瞻顾,寻所逐者。壁上小虫忽跃落襟袖间,视之,形若土狗\footnote{〔土狗〕蝼蛄的别名。},梅花翅,方首,长胫,意似良。喜而收之。将献公堂,惴惴恐不当意,思试之斗以觇之。
    
    村中少年好事者驯养一虫,自名“蟹壳青”,日与子弟角,无不胜。欲居之以为利,而高其直,亦无售者。径造庐访成,视成所蓄,掩口胡卢而笑,因出己虫,纳比笼中。成视之,庞然修伟,自增惭怍,不敢与较。少年固强之。顾念蓄劣物终无所用,不如拼博一笑,因合纳斗盆。小虫伏不动,蠢若木鸡。少年又大笑。试以猪鬣毛撩拨虫须,仍不动。少年又笑。屡撩之,虫暴怒,直奔,遂相腾击,振奋作声。俄见小虫跃起,张尾伸须,直龁敌领。少年大骇,急解令休止。虫翘然矜鸣,似报主知。成大喜。方共瞻玩,一鸡瞥来,径进以啄。成骇立愕呼,幸啄不中,虫跃去尺有咫。鸡健进,逐逼之,虫已在爪下矣。成仓猝莫知所救,顿足失色。旋见鸡伸颈摆扑,临视,则虫集冠上,力叮不释。成益惊喜,掇置笼中。
    
    翼日进宰,宰见其小,怒诃成。成述其异,宰不信。试与他虫斗,虫尽靡。又试之鸡,果如成言。乃赏成,献诸抚军\footnote{〔抚军〕官名,巡抚的别称,总管一省的民政和军政。}。抚军大悦,以金笼进上,细疏\footnote{〔疏〕臣下向君主陈述事情的一种公文。}其能。既入宫中,举天下所贡蝴蝶、螳螂、油利挞、青丝额一切异状遍试之,无出其右者。每闻琴瑟之声,则应节而舞。益奇之。上大嘉悦,诏赐抚臣名马衣缎。抚军不忘所自,无何,宰以卓异闻。宰悦,免成役,又嘱学使\footnote{〔学使〕提督学政(学台),是专管教育和考试的官。}俾入邑庠\footnote{〔邑庠〕县学,即成为秀才。}。后岁余,成子精神复旧。自言身化促织,轻捷善斗,今始苏耳。抚军亦厚赉成。不数岁,田百顷,楼阁万椽,牛羊蹄躈各千计;一出门,裘马过世家焉。
    
    异史氏\footnote{〔异史氏〕作者自称。作者把志怪当作记史,所以称“异史”。}曰:“天子偶用一物,未必不过此已忘;而奉行者即为定例。加以官贪吏虐,民日贴妇卖儿,更无休止。故天子一跬步,皆关民命,不可忽也。独是成氏子以蠹贫,以促织富,裘马扬扬。当其为里正、受扑责时,岂意其至此哉?天将以酬长厚者,遂使抚臣、令尹\footnote{〔令尹〕县令,府尹。这里是用古称泛指官。},并受促织恩荫。闻之:一人飞升,仙及鸡犬。信夫!”
\end{normalsize}


\newpage

\textbf{译文}:

\vspace{1em}

\begin{normalsize}
    
    宣德年间,皇室里盛行斗蟋蟀的赌博,每年都要向民间征收。这东西本来不是陕西出产的;有个华阴县的县官,想巴结上司,把一只蟋蟀献上去,上司试着让它斗了一下,显出了勇敢善斗的才能,上级于是责令他一直供应。县官又把供应的差事派给各乡的公差。于是市上的那些游手好闲的年轻人,捉到好的蟋蟀就用竹笼装着喂养它,抬高它的价格;储存起来,当作珍奇的货物一样等待高价出售。乡里的差役们狡猾刁诈,借这个机会向老百姓摊派费用,每摊派一只蟋蟀,就常常使好几户人家破产。
    
    县里有个叫成名的人,是个念书人,长期未考中秀才。为人拘谨,不善说话,就被刁诈的小吏报到县里,叫他担任里正的差事;他想尽方法还是摆脱不掉(任里正这差事)。不到一年,微薄的家产逐渐耗尽。正好又碰上征收蟋蟀,成名不敢勒索老百姓,但又没有抵偿的钱,忧愁苦闷,想要寻死。他妻子说:“死有什么益处呢?不如自己去寻找,希望有万分之一的的收获。”成名认为这些话对。就从早上出去晚上回家,提着竹筒丝笼,在毁坏的墙脚、荒草丛生的地方,挖石头,掏大洞,各种办法都用尽了, 终究没有找到。即使捕捉到两三只,也是又弱又小不符合规格。县官严定期限,严厉追逼,成名在十几天中被打了上百板子,两条腿脓血淋漓,连蟋蟀也不能去捉了。在床上翻来覆去只想自杀。
    
    这时,村里来了个驼背巫婆,(她)能借鬼神预卜凶吉。成名妻子准备了礼钱去求神。只见红妆的少女和白发的老婆婆,挤在门口。成名的妻子走进巫婆的屋里,只看见暗室挂着帘子,帘外摆着香案。求神的人在香炉上上香,拜了又拜。巫婆在旁边望着空中替他们祷告,嘴唇一张一合,不知在说些什么。大家都肃敬地站着听。 一会儿,室内丢一张纸条出来,上面写着求神的人心中想问的事,没有丝毫差错。成名的妻子把钱放在案上,像前边的人一样烧香跪拜。约一顿饭的工夫,帘子掀动,一片纸抛落下来。拾起看,不是字而是一幅画:当中绘着殿阁,像寺院;(殿阁)后面的山脚下,奇形怪状的石头到处横卧,在丛丛荆棘中,一只蟋蟀藏在那里;旁边一只蛤蟆,像要跳起来的样子。她展开琢磨,不能理解。但是看到上面画着蟋蟀,暗合自己的心事,就把纸片折叠好装起来,回家后交给成名看。
    
    成名反复思索,恐怕是指给我捉蟋蟀的地方吧?细看图上面的景物,和村东的大佛阁很相像。于是他就忍痛爬起来,扶着杖,拿着图来到寺庙的后面,(看到)有一座古坟高高隆起。成名沿着古坟向前跑,只见一块块石头,好像鱼鳞似的排列着,真像画中的一样。他于是在野草中一面侧耳细听一面慢走,好像在找一根针和一株小草似的。然而心力、视力、听力都用尽了,结果还是一点蟋蟀的踪迹响声都没有。他正用心探索着,突然一只癞蛤蟆跳过去了。成名更加惊奇了,急忙去追它,癞蛤蟆(已经)跳入草中。他便跟着癞蛤蟆的踪迹,分开丛草去寻找,只见一只蟋蟀趴在棘根下面。他急忙扑过去捉它,蟋蟀跳进了石洞。他用细草撩拨,蟋蟀不出来;又用竹筒取水灌进石洞里,蟋蟀才出来,形状极其俊美健壮。他便追赶着抓住了它。仔细一看,只见蟋蟀个儿大,尾巴长,青色的颈项,金黄色的翅膀。成名特别高兴,用笼子装上提回家,全家庆贺,把它看得比价值连城的宝玉还珍贵,装在盆子里并且用蟹肉栗子粉喂它,爱护得周到极了,只等到了期限,拿它送到县里去缴差。
    
    成名有个儿子,年九岁,看到爸爸不在(家),偷偷打开盆子来看。蟋蟀一下子跳出来了,快得来不及捕捉。等抓到手后,(蟋蟀)的腿已掉了,肚子也破了,一会儿就死了。孩子害怕了,就哭着告诉母亲。母亲听了,(吓得)面色灰白,大惊说:“祸根,你的死期到了!你父亲回来,自然会跟你算帐!”孩子哭着跑了。
    
    不多时,成名回来了,听了妻子的话,全身好像盖上冰雪一样。怒气冲冲地去找儿子,儿子无影无踪不知到哪里去了。后来在井里找到他的尸体,于是怒气立刻化为悲痛,呼天喊地,悲痛欲绝。夫妻二人对着墙角流泪哭泣,茅屋里没有炊烟,面对面坐着不说一句话,再也没有了依靠。直到傍晚时,才拿上草席准备把孩子埋葬。夫妻走近一摸,还有一丝微弱的气息。他们高兴地把他放在床上,半夜里孩子又苏醒过来。夫妻二人心里稍稍宽慰一些,但是孩子神气呆呆的,气息微弱,只想睡觉。但是看到蟋蟀笼子空着,成名就急得气也吐不出,也不敢再追究儿子的责任。从晚上到天明,连眼睛也没合一下。东方的太阳已经升起来了,他还直挺挺地躺在床上发愁。他忽然听到门外有蟋蟀的叫声,吃惊地起来细看时,那只蟋蟀完好还在。他高兴得动手捉它,那蟋蟀叫了一声就跳走了,跳得非常快。他用手掌去罩住它,手心空荡荡地好像没有什么东西;手刚举起,却又远远地跳开了。成名急忙追它,转过墙角,又不知它的去向了。他东张西望,四下寻找,才看见蟋蟀趴在墙壁上。成名仔细看它,个儿短小,黑红色,立刻觉得它不像先前那只。成名因它个儿小,看不上它。(成名)仍不住地来回寻找,找他所追捕的那只。(这时)墙壁上的那只小蟋蟀,忽然跳到他的衣袖里去了。再仔细看它,形状像蝼蛄,梅花翅膀,方头长腿,觉得好像还不错。他高兴地收养了它,准备献给官府,但是心里还很不踏实,怕不合县官的心意,他想先试着让它斗一下,看它怎么样。
    
    村里一个喜欢多事的年轻人,养着一只蟋蟀,自己给它取名叫“蟹壳青”,(他)每日跟其他少年斗(蟋蟀)没有一次不胜的。他想留着它居为奇货来牟取暴利,便抬高价格,但是也没有人买。(有一天)少年直接上门来找成名,看到成名所养的蟋蟀,只是掩着口笑,接着取出自己的蟋蟀,放进比试的笼子里。成名一看对方那只蟋蟀又长又大,自己越发羞愧,不敢拿自己的小蟋蟀跟少年的“蟹壳青”较量。少年坚持要斗。但成名心想养着这样低劣的东西,终究没有什么用处,不如让它斗一斗,换得一笑了事,于是把两个蟋蟀放在一个斗盆里。小蟋蟀趴着不动,呆呆地像个木鸡。少年又大笑。(接着)试着用猪鬣撩拨小蟋蟀的触须,小蟋蟀仍然不动。少年又大笑了。撩拨了它好几次,小蟋蟀突然大怒,直往前冲,于是互相斗起来,腾身举足,彼此相扑,振翅叫唤。一会儿,只见小蟋蟀跳起来,张开尾,竖起须,一口直咬着对方的脖颈。少年大惊,急忙分开,使它们停止扑斗。小蟋蟀抬着头振起翅膀得意地鸣叫着,好像给主人报捷一样。成名大喜。(两人正在观赏)突然来了一只鸡,直向小蟋蟀啄去。成名吓得(站在那里)惊叫起来,幸喜没有啄中,小蟋蟀一跳有一尺多远。鸡又大步地追逼过去,小蟋蟀已被压在鸡爪下了。成名吓得惊慌失措,不知怎么救它,急得直跺脚,脸色都变了。忽然又见鸡伸长脖子扭摆着头,到跟前仔细一看,原来小蟋蟀已蹲在鸡冠上用力叮着不放。成名越发惊喜,捉下放在笼中。
    
    第二天,成名把蟋蟀献给县官,县官见它小,怒斥成名。成名讲述了这只蟋蟀的奇特本领,县官不信。试着和别的蟋蟀搏斗,所有的都被斗败了。又试着和鸡斗,果然和成名所说的一样。于是就奖赏了成名,把蟋蟀献给了巡抚。巡抚特别喜欢,用金笼装着献给皇帝,并且上了奏本,仔细地叙述了它的本领。到了宫里后,凡是全国贡献的蝴蝶、螳螂、油利挞、青丝额及各种稀有的蟋蟀,都与(小蟋蟀)斗过了,没有一只能占它的上风。它每逢听到琴瑟的声音,都能按照节拍跳舞。(大家)越发觉得出奇。皇帝更加喜欢,便下诏赏给巡抚好马和锦缎。巡抚不忘记好处是从哪来的,不久,县官也以才能卓越而闻名了。县官一高兴,就免了成名的差役,又嘱咐主考官,让成名中了秀才。过了一年多,成名的儿子精神复原了。他说他变成一只蟋蟀,轻快而善于搏斗,现在才苏醒过来。巡抚也重赏了成名。不到几年,成名就有一百多顷田地,很多高楼殿阁,还有成百上千的牛羊;每次出门,身穿轻裘,骑上高头骏马,比世代做官的人家还阔气。
    
    异史氏说:“皇帝偶尔使用一件东西,未必不是用过它就忘记了;然而下面执行的人却把它作为一成不变的惯例。加上官吏贪婪暴虐,老百姓一年到头抵押妻子卖掉孩子,还是没完没了。所以皇帝的一举一动,都关系着老百姓的性命,不可忽视啊!这个叫成名的人因为官吏的侵害而贫穷,又因为进贡蟋蟀而致富,穿上名贵的皮衣,坐上豪华的车马,得意扬扬。当他充当里正,受到责打的时候,哪里想到他会有这种境遇呢?老天要用这酬报那些老实忠厚的人,就连巡抚、县官都受到蟋蟀的恩惠了。听说一人得道成仙,连鸡狗都可以上天。这话真是一点不假啊!”
    
\end{normalsize}



\chapter{过秦论}

\begin{normalsize}
    
    秦孝公\footnote{〔秦孝公〕战国时期秦国君主(公元前362至338年),任用商鞅变法强秦。}据崤函\footnote{〔崤函〕崤山和函谷关的合称,秦国东部重要关隘。}之固,拥雍州\footnote{〔雍州〕古代九州之一,指关中地区秦国核心疆域。}之地,君臣固守以窥周室\footnote{〔周室〕周王室,战国时已衰微,仅名义上是天下共主。},有席卷天下,包举宇内,囊括四海之意,并吞八荒之心。当是时也,商君\footnote{〔商君〕即商鞅,法家代表人物,主持秦国变法改革。}佐之,内立法度,务耕织,修守战之具,外连衡\footnote{〔连衡〕即连横策略,秦国分化瓦解六国联盟的外交手段。}而斗诸侯。于是秦人拱手而取西河\footnote{〔西河〕黄河以西地区,秦魏争夺的战略要地。}之外。
    
    孝公既没,惠文、武、昭襄\footnote{〔惠文、武、昭襄〕秦孝公之后连续三位秦国君主(公元前337至251年)。}蒙故业,因遗策,南取汉中\footnote{〔汉中〕汉水上游盆地,秦楚交界战略要冲。},西举巴、蜀\footnote{〔巴、蜀〕今四川地区,秦并吞的重要粮仓基地。},东割膏腴之地,北收要害之郡。诸侯恐惧,会盟而谋弱秦。不爱珍器重宝肥饶之地,以致天下之士,合从缔交,相与为一。当此之时,齐有孟尝\footnote{〔孟尝〕齐国公子田文,战国四公子之一。},赵有平原\footnote{〔平原〕赵国公子赵胜,战国四公子之一,以善养士闻名。},楚有春申\footnote{〔春申〕楚国令尹黄歇,战国四公子之一,曾主持合纵抗秦。},魏有信陵\footnote{〔信陵〕魏国公子魏无忌,战国四公子之一,窃符救赵典故主角。}。此四君者,皆明智而忠信,宽厚而爱人,尊贤而重士,约从离衡,兼韩、魏、燕、楚、齐、赵、宋、卫、中山之众。于是六国之士,有甯越、徐尚、苏秦、杜赫之属\footnote{〔甯越……之属〕六国著名谋士策士。}为之谋,齐明、周最、陈轸、召滑、楼缓、翟景、苏厉、乐毅之徒\footnote{〔齐明……之徒〕战国时期各国的纵横家与外交人才。}通其意,吴起、孙膑、带佗、倪良、王廖、田忌、廉颇、赵奢之伦\footnote{〔吴起……之伦〕战国时期的著名军事将领。}制其兵。尝以十倍之地,百万之众,叩关而攻秦。秦人开关延敌,九国之师,逡巡而不敢进。秦无亡矢遗镞之费,而天下诸侯已困矣。于是从散约败,争割地而赂秦。秦有余力而制其弊,追亡逐北,伏尸百万,流血漂橹。因利乘便,宰割天下,分裂山河。强国请服,弱国入朝。延及孝文王、庄襄王\footnote{〔孝文王、庄襄王〕秦昭襄王死后的两位短命秦王(公元前250至247年),之后朝政被相国吕不韦把持,直至嬴政22岁即位。},享国之日浅,国家无事。
    
    及至始皇\footnote{〔始皇〕秦始皇嬴政(公元前238至210年),完成统一,建立秦朝。},奋六世之余烈,振长策而御宇内,吞二周\footnote{〔二周〕即西周和东周。}而亡诸侯,履至尊而制六合\footnote{〔六合〕天地四方,代指全天下。},执敲扑而鞭笞天下,威振四海。南取百越\footnote{〔百越〕岭南少数民族部落的统称。}之地,以为桂林、象郡\footnote{〔桂林、象郡〕秦朝征服岭南后置郡,包括今天的广西、广东部分和越南北部。};百越之君,俯首系颈,委命下吏。乃使蒙恬\footnote{〔蒙恬〕秦朝名将,率军北击匈奴并修筑长城。}北筑长城而守藩篱,却匈奴\footnote{〔匈奴〕北方游牧民族,长期威胁中原王朝。}七百余里;胡人不敢南下而牧马,士不敢弯弓而报怨。于是废先王之道,焚百家之言,以愚黔首;隳名城,杀豪杰;收天下之兵,聚之咸阳,销锋镝,铸以为金人十二,以弱天下之民。然后践华为城,因河为池,据亿丈之城,临不测之渊,以为固。良将劲弩守要害之处,信臣精卒陈利兵而谁何。天下已定,始皇之心,自以为关中之固,金城千里,子孙帝王万世之业也。
    
    始皇既没,余威震于殊俗。然陈涉\footnote{〔陈涉〕即陈胜,秦末农民起义领袖。}瓮牖绳枢之子,氓隶之人,而迁徙之徒也;才能不及中人,非有仲尼\footnote{〔仲尼〕即孔子,儒家学说创始人。}、墨翟\footnote{〔墨翟〕即墨子,墨家学说创始人。}之贤,陶朱\footnote{〔陶朱〕即范蠡,春秋时期著名富商。}、猗顿\footnote{〔猗顿〕春秋时期鲁国巨富。}之富。蹑足行伍之间,而倔起阡陌之中,率疲弊之卒,将数百之众,转而攻秦;斩木为兵,揭竿为旗,天下云集响应,赢粮而景从。山东\footnote{〔山东〕崤山以东地区,代指六国故地。}豪俊遂并起而亡秦族矣。
    
    且夫天下非小弱也,雍州之地,崤函之固,自若也。陈涉之位,非尊于齐、楚、燕、赵、韩、魏、宋、卫、中山之君也;锄櫌棘矜,非铦于钩戟长铩也;谪戍之众,非抗于九国之师也;深谋远虑,行军用兵之道,非及乡时之士也。然而成败异变,功业相反,何也?试使山东之国与陈涉度长絜大,比权量力,则不可同年而语矣。然秦以区区之地,致万乘之势,序八州而朝同列,百有余年矣;然后以六合为家,崤函为宫;一夫作难而七庙\footnote{〔七庙〕周代宗庙制度,天子可祭祀七代祖先,故称七庙。}隳,身死人手,为天下笑者,何也?仁义不施而攻守之势异也。
\end{normalsize}


\newpage

\textbf{译文}:

\vspace{1em}

\begin{normalsize}
    
    秦孝公占据着崤山和函谷关的险固地势,拥有雍州的土地,君臣牢固地守卫着,借以窥视周王室(的权力)(秦孝公)有席卷天下,包办天宇之间,囊括四海的意图,并统天下的雄心。正当这时,商鞅辅佐他,对内建立法规制度,从事耕作纺织,修造防守和进攻的器械;对外实行连衡策略,使诸侯自相争斗。因此,秦人轻而易举地夺取了黄河以西的土地。
    
    秦孝公死了以后,惠文王、武王、昭襄王承继先前的基业,沿袭前代的策略,向南夺取汉中,向西攻取巴、蜀,向东割取肥沃的地区,向北占领非常重要的地区。诸侯恐慌害怕,集会结盟,商议削弱秦国。不吝惜奇珍贵重的器物和肥沃富饶的土地,用来招纳天下的优秀人才,采用合纵的策略缔结盟约,互相援助,成为一体。在这个时候,齐国有孟尝君,赵国有平原君,楚国有春申君,魏国有信陵君。这四位封君,都见识英明有智谋,心地诚而讲信义,待人宽宏厚道而爱惜人民,尊重贤才而重用士人,以合纵之约击破秦的连横之策,联合韩、魏、燕、楚、齐、赵、宋、卫、中山的部队。在这时,六国的士人,有宁越、徐尚、苏秦、杜赫等人为他们出谋划策,齐明、周最、陈轸、召滑、楼缓、翟景、苏厉、乐毅等人沟通他们的意见,吴起、孙膑、带佗、倪良、王廖、田忌、廉颇、赵奢等人统率他们的军队。他们曾经用十倍于秦的土地,上百万的军队,攻打函谷关来攻打秦国。秦人打开函谷关口迎战敌人,九国的军队有所顾虑徘徊不敢入关。秦人没有一兵一卒的耗费,然而天下的诸侯就已窘迫不堪了。因此,纵约失败了,各诸侯国争着割地来贿赂秦国。秦有剩余的力量趁他们困乏而制服他们,追赶逃走的败兵,百万败兵横尸道路,流淌的血液可以漂浮盾牌。秦国凭借这便利的形势,割取天下的土地,重新划分山河的区域。强国主动表示臣服,弱国入秦朝拜。延续到孝文王、庄襄王,统治的时间不长,秦国并没有什么大事发生。
    
    到始皇的时候,发展六世遗留下来的功业,以武力来统治各国,将东周,西周和各诸侯国统统消灭,登上皇帝的宝座来统治天下,用严酷的刑罚来奴役天下的百姓,威风震慑四海。秦始皇向南攻取百越的土地,把它划为桂林郡和象郡,百越的君主低着头,颈上捆着绳子(愿意服从投降),把性命交给司法官吏。秦始皇于是又命令蒙恬在北方修筑长城,守卫边境,使匈奴退却七百多里;胡人不敢向下到南边来放牧,勇士不敢拉弓射箭来报仇。秦始皇接着就废除古代帝王的治世之道,焚烧诸子百家的著作,来使百姓愚蠢;毁坏高大的城墙,杀掉英雄豪杰;收缴天下的兵器,集中在咸阳,销毁兵刃和箭头,冶炼它们铸造十二个铜人,以便削弱百姓的反抗力量。然后凭借华山为城墙,依据黄河为城池,凭借着高耸的华山,往下看着深不可测的黄河,认为这是险固的地方。好的将领手执强弩,守卫着要害的地方,可靠的官员和精锐的士卒,拿着锋利的兵器,盘问过往行人。天下已经安定,始皇心里自己认为这关中的险固地势、方圆千里的坚固的城防,是子子孙孙称帝称王直至万代的基业。
    
    始皇去世之后,他的余威(依然)震慑着边远地区。可是,陈涉不过是个破瓮做窗户、草绳做户枢的贫家子弟,是氓、隶一类的人,(后来)做了被迁谪戍边的卒子;才能不如普通人,并没有孔丘、墨翟那样的贤德,也不像陶朱、猗顿那样富有。(他)跻身于戍卒的队伍中,从田野间突然奋起发难,率领着疲惫无力的士兵,指挥着几百人的队伍,掉转头来进攻秦国,砍下树木作武器,举起竹竿当旗帜,天下豪杰像云一样聚集,回声似的应和他,许多人都背着粮食,如影随形。崤山以东的英雄豪杰于是一齐起事,消灭了秦的家族。
    
    况且那天下并没有缩小削弱,雍州的地势,崤山和函谷关的险固,是保持原来的样子。陈涉的地位,没有比齐、楚、燕、赵、韩、魏、宋、卫、中山的国君更加尊贵;锄头木棍也不比钩戟长矛更锋利;那迁谪戍边的士兵也不能和九国部队抗衡;深谋远虑,行军用兵的方法,也比不上先前九国的武将谋臣。可是条件好者失败而条件差者成功,功业完全相反,为什么呢?假使拿东方诸侯国跟陈涉比一比长短大小,量一量权势力量,就更不能相提并论了。然而秦凭借着它的小小的地方,发展到兵车万乘的国势,管辖全国,使六国诸侯都来朝见,已经一百多年了;这之后把天下作为家业,用崤山、函谷关作为自己的内宫;陈涉一人起义国家就灭亡了,秦王子婴死在别人(项羽)手里,被天下人耻笑,这是为什么呢?就因为不施行仁政而使攻守的形势发生了变化啊。
    
\end{normalsize}



\chapter{后赤壁赋}

\begin{normalsize}
    
    是岁\footnote{〔是岁〕宋神宗元丰五年(公元1082年)。}十月之望,步自雪堂\footnote{〔雪堂〕苏轼在黄州所建的新居,离他在临皋的住处不远,在黄冈东面。堂在大雪时建成,画雪景于四壁,故名“雪堂”。},将归于临皋\footnote{〔临皋〕亭名,在黄冈南长江边上。苏轼初到黄州时住在定惠院,不久就迁至临皋亭。}。二客从予,过黄泥之坂\footnote{〔黄泥之坂〕黄冈东面东坡附近的山坡叫“黄泥坂”。}。霜露既降,木叶尽脱。人影在地,仰见明月。顾而乐之,行歌相答。
    
    已而叹曰:“有客无酒,有酒无肴,月白风清,如此良夜何!”客曰:“今者薄暮,举网得鱼,巨口细鳞,状如松江之鲈。顾安所得酒乎?”归而谋诸妇。妇曰:“我有斗酒,藏之久矣,以待子不时之需。”
    
    于是携酒与鱼,复游于赤壁之下。江流有声,断岸千尺;山高月小,水落石出。曾日月之几何,而江山不可复识矣。予乃摄衣而上,履巉岩,披蒙茸,踞虎豹,登虬龙,攀栖鹘之危巢,俯冯夷\footnote{〔冯夷〕也叫“冰夷”,即河伯,黄河的河神。}之幽宫。盖二客不能从焉。划然长啸,草木震动,山鸣谷应,风起水涌。予亦悄然而悲,肃然而恐,凛乎其不可留也。反而登舟,放乎中流,听其所止而休焉。时夜将半,四顾寂寥。适有孤鹤,横江东来,翅如车轮,玄裳缟衣,戛然长鸣,掠予舟而西也。
    
    须臾客去,予亦就睡。梦一道士,羽衣蹁跹,过临皋之下,揖予而言曰:“赤壁之游乐乎?”问其姓名,俯而不答。“呜呼!噫嘻!我知之矣。畴昔之夜,飞鸣而过我者,非子也邪?”道士顾笑,予亦惊寤。开户视之,不见其处。
\end{normalsize}


\newpage

\textbf{译文}:

\vspace{1em}

\begin{normalsize}
    
    这一年十月十五日,我从雪堂出发,准备回临皋亭。有两位客人跟随着我,一起走过黄泥坂。这时霜露已经降下,叶全都脱落。我们的身影倒映在地上,抬头望见明月高悬。四下里瞧瞧,心里十分快乐;于是一面走一面吟诗,相互酬答。
    
    过了一会儿,我叹惜地说:“有客人却没有酒,即使有酒也没有菜肴。月色皎洁,清风吹拂,这样美好的夜晚,我们怎么度过呢?”一位客人说:“今天傍晚,我撒网捕到了鱼,大嘴巴,细鳞片,形状就像吴淞江的鲈鱼。不过,到哪里去弄到酒呢?”我回家和妻子商量,妻子说:“我有一斗酒,保藏了很久,为了应付您突然的需要。”
    
    就这样,我们携带着酒和鱼,再次到赤壁的下面游览。长江的流水发出声响,陡峭的江岸高峻直耸;山峦很高,月亮显得小了,水位降低,礁石露了出来。才相隔多少日子,上次游览所见的江景山色再也认不出来了!我就撩起衣襟上岸,踏着险峻的山岩,拨开纷乱的野草;蹲在虎豹形状的怪石上,又不时拉住形如虬龙的树枝,攀上猛禽做窝的悬崖,下望水神冯夷的深宫。两位客人都不能跟着我到这个极高处。我大声地长啸,草木被震动,高山与我共鸣,深谷响起了回声,大风刮起,波浪汹涌。我也觉得忧愁悲哀,感到恐惧而静默屏息,觉得这里令人畏惧,不可久留。回到船上,把船划到江心,任凭它漂流到哪里就在那里停泊。这时快到半夜,望望四周,觉得冷清寂寞得很。正好有一只鹤,横穿江面从东边飞来,翅膀像车轮一样大小,尾部的黑羽如同黑裙子,身上的白羽如同洁白的衣衫,它戛戛地拉长声音叫着,擦过我们的船向西飞去。
    
    过了会儿,客人离开了,我也回家睡觉。梦见一位道士,穿着羽毛编织成的衣裳,轻快地走来,走过临皋亭的下面,向我拱手作揖说:“赤壁的游览快乐吗?”我问他的姓名,他低头不回答。“噢!哎呀!我知道你的底细了。昨天夜晚,边飞边叫着从我这里经过的人,不是你吗?”道士回头笑了起来,我也忽然惊醒。开门一看,却看不到他在什么地方。
    
\end{normalsize}



\chapter{谏逐客书}

\begin{normalsize}
    
    臣闻吏议逐客,窃以为过矣。昔缪公\footnote{〔缪公〕即秦穆公(前659至前621年在位),春秋五霸之一。}求士,西取由余\footnote{〔由余〕西戎谋士,助秦穆公征服西戎十二国(前623年左右),开拓陇西地区。}于戎,东得百里奚\footnote{〔百里奚〕原为虞国大夫,秦穆公以五张羊皮赎于楚国(前655年),任相国推行改革。}于宛,迎蹇叔\footnote{〔蹇叔〕宋国隐士,经百里奚推荐入秦(前655年),为穆公制定东进战略的核心谋臣。}于宋,来丕豹\footnote{〔丕豹〕晋国大夫丕郑之子,其父被杀后奔秦(前651年),助秦攻晋取得河西部分土地。}、公孙支\footnote{〔公孙支〕字子桑,晋国人,投秦后参与策划殽之战(前627年)等重大军事行动。}于晋。此五子者,不产于秦,而缪公用之,并国二十,遂霸西戎。孝公\footnote{〔孝公〕秦孝公(前361至前338年在位),任用商鞅变法,公元前340年夺取魏国河西之地(今陕西东部)。}用商鞅\footnote{〔商鞅〕卫国人,主持变法,领军攻魏取少梁。}之法,移风易俗,民以殷盛,国以富强,百姓乐用,诸侯亲服,获楚、魏之师,举地千里,至今治强。惠王\footnote{〔惠王〕秦惠文王(前337至前311年在位),前316年派张仪、司马错灭蜀国巴国(今四川),置巴郡、蜀郡。}用张仪\footnote{〔张仪〕魏国人,推行连横策略,前313年诈楚绝齐,前312年丹阳之战夺取楚国汉中(今陕西南部)。}之计,拔三川之地\footnote{〔三川之地〕指黄河、雒水、伊水三川之地,在今河南西北部黄河以南的洛水、伊水流域。韩宣王在此设三川郡。公元前249年秦灭东周,取得三川。},西并巴、蜀,北收上郡\footnote{〔上郡〕郡名,原为楚地,今陕西榆林。魏文侯时置郡,公元前328年被迫献于秦。},南取汉中\footnote{〔汉中〕郡名,今陕西汉中。楚怀王时置郡。公元前312年被秦国攻取。},包九夷\footnote{〔九夷〕指楚国境内西北部的少数部族,在今陕西、湖北、四川三省交界地区。},制鄢、郢\footnote{〔鄢、郢〕楚国曾经的都城。分别在今湖北宜城县东南、江陵市西北。公元前279年秦将白起攻取鄢,翌年又攻取郢。},东据成皋\footnote{〔成皋〕邑名,在今河南荥阳县汜水镇,地势险要,是著名的军事重地。春秋时属郑国称虎牢。公元前375年韩国灭郑后属韩,公元前249年被秦军攻取。}之险,割膏腴之壤,遂散六国之从,使之西面事秦,功施到今。昭王\footnote{〔昭王〕秦昭襄王(前306至前251年在位),前266年用范雎"远交近攻"策略,持续东进。}得范雎\footnote{〔范雎〕魏国人,提出"强干弱枝"政策,前262年发动长平之战,前256年灭西周。},废穰侯\footnote{〔穰侯〕魏冉,秦昭襄王母宣太后之异父弟,拥立秦昭王,任将军,多次为相,受封于穰(今河南邓县),故称穰侯。昭王任用范雎后被免职遣归封地。},逐华阳\footnote{〔华阳〕芈戎,宣太后同父弟,封华阳君,与穰侯共同专权,后被范雎免职遣归封地。},强公室,杜私门,蚕食诸侯,使秦成帝业。此四君者,皆以客之功。由此观之,客何负于秦哉!向使四君却客而不内,疏士而不用,是使国无富利之实,而秦无强大之名也。
    
    今陛下致昆山之玉,有随、和之宝\footnote{〔随和之宝〕即所谓“随侯珠”和“和氏璧”,传说中春秋时随侯所得的夜明珠和楚人卞和来得的美玉。},垂明月之珠,服太阿\footnote{〔太阿〕亦称“泰阿”,宝剑名,相传为春秋著名工匠欧冶子、干将所铸。}之剑,乘纤离之马,建翠凤之旗,树灵鼍之鼓。此数宝者,秦不生一焉,而陛下说之,何也?必秦国之所生然后可,则是夜光之璧不饰朝廷,犀象之器不为玩好,郑、卫之女不充后宫,而骏良駃騠不实外厩,江南金锡不为用,西蜀丹青不为采。所以饰后宫,充下陈,娱心意,说耳目者,必出于秦然后可,则是宛珠之簪、傅玑之珥、阿缟之衣、锦绣之饰不进于前,而随俗雅化佳冶窈窕赵女不立于侧也。夫击瓮叩缶,弹筝搏髀,而歌呼呜呜快耳者,真秦之声也;《郑》《卫》《桑间》《昭》《虞》《武》《象》者,异国之乐也。今弃击瓮叩缶而就《郑》《卫》,退弹筝而取《昭》《虞》,若是者何也?快意当前,适观而已矣。今取人则不然,不问可否,不论曲直,非秦者去,为客者逐。然则是所重者,在乎色、乐、珠玉,而所轻者,在乎人民也。此非所以跨海内、制诸侯之术也。
    
    臣闻地广者粟多,国大者人众,兵强则士勇。是以太山不让土壤,故能成其大;河海不择细流,故能就其深;王者不却众庶,故能明其德。是以地无四方,民无异国,四时充美,鬼神降福,此五帝三王\footnote{〔五帝三王〕指黄帝、颛顼、帝喾、尧、舜。三王,指夏、商、周三代开国君主,即夏禹、商汤和周武王。}之所以无敌也。今乃弃黔首\footnote{〔黔首〕泛指百姓。无爵平民不能服冠,只能以黑巾裹头,故称黔首,秦始皇统一六国后正式称百姓为黔首。《史记·秦始皇本纪》载:二十六年,“更名民曰黔首”。}以资敌国,却宾客以业诸侯,使天下之士退而不敢西向,裹足不入秦,此所谓“藉寇兵而赍盗粮”者也。
    
    夫物不产于秦,可宝者多;士不产于秦,而愿忠者众。今逐客以资敌国,损民以益仇,内自虚而外树怨于诸侯,求国无危,不可得也。
\end{normalsize}


\newpage

\textbf{译文}:

\vspace{1em}

\begin{normalsize}
    
    我听说官吏在商议驱逐客卿这件事,私下里认为是错误的。从前秦穆公寻求贤士,西边从西戎取得由余,东边从宛地得到百里奚,又从宋国迎来蹇叔,还从晋国招来丕豹、公孙支。这五位贤人,不生在秦国,而秦穆公重用他们,吞并国家二十多个,于是称霸西戎。秦孝公采用商鞅的新法,移风易俗,人民因此殷实,国家因此富强,百姓乐意为国效力,诸侯亲附归服,战胜楚国、魏国的军队,攻取土地上千里,至今政治安定,国力强盛。秦惠王采纳张仪的计策,攻下三川地区,西进兼并巴、蜀两国,北上收得上郡,南下攻取汉中,席卷九夷各部,控制鄢、郢之地,东面占据成皋天险,割取肥田沃土,于是拆散六国的合纵同盟,使他们朝西侍奉秦国,当初建立的功业一直延续到今天。昭王得到范雎,废黜穰侯,驱逐华阳君,加强巩固了王室的权力,堵塞了权贵垄断政治的局面,蚕食诸侯领土,使秦国成就帝王大业。这四位君主,都依靠了客卿的功劳。由此看来,客卿哪有什么对不住秦国的地方呢!倘若四位君主拒绝远客而不予接纳,疏远贤士而不加任用,这就会使国家没有丰厚的实力,而让秦国没有强大的名声了。
    
    如今陛下罗致昆山的美玉,宫中有随侯之珠,和氏之璧,衣饰上缀着光如明月的宝珠,身上佩带着太阿宝剑,乘坐的是名贵的千里马,树立的是以翠凤羽毛为饰的旗子,陈设的是蒙着灵鼍之皮的好鼓。这些宝贵之物,没有一种是秦国产的,而陛下却很喜欢它们,这是为什么呢?如果一定要是秦国出产的才许可采用,那么这种夜光宝玉,决不会成为宫廷的装饰;犀角、象牙雕成的器物,也不会成为陛下的玩好之物;郑、卫二地能歌善舞的女子,也不会填满陛下的后宫;北方的名骥良马,决不会充实到陛下的马房;江南的铜锡不会为陛下所用,西蜀的丹青也不会作为彩饰的颜料。用以装饰后宫、广充侍妾、爽心快意、悦人耳目的所有这些都要是秦国生长、生产的然后才可用的话,那么点缀有珠宝的簪子,耳上的玉坠,丝织的衣服,锦绣的装饰,就都不会进献到陛下面前;那些闲雅变化而能随俗推移的妖冶美好的佳丽,也不会立于陛下的身旁。敲击瓮、缶来奏乐,弹着琴筝、 拍打大腿以应和节拍、呜呜呀呀地高唱着来使耳朵痛快,这才是真正的秦国音乐;那郑、卫桑间的歌声,《韶虞》《武象》等乐曲,可算是外国的音乐了。如今陛下却抛弃了秦国地道的敲击瓦器的音乐,而取用郑、卫淫靡悦耳之音,不要秦筝而要《韶虞》,这是为什么呢?难道不是因为外国音乐可以快意,可以满足耳目功能的需要么?可陛下对用人却不是这样,不问是否可用,不管是非曲直,凡不是秦国的就要离开,凡是客卿都要驱逐。这样做就说明,陛下所看重的,只在珠玉声色方面;而所轻视的,却是人民大众。这不是能用来驾驭天下,制服诸侯的方法啊!
    
    我听说田地广就粮食多,国家大就人口众,武器精良将士就骁勇。因此,泰山不拒绝泥土,所以能成就它的高大;江河湖海不舍弃细流,所以能成就它的深邃;有志建立王业的人不嫌弃民众,所以能彰明他的德行。因此,土地不分东西南北,百姓不论异国它邦,那样便会一年四季富裕美好,天地鬼神降赐福运,这就是五帝、三王无可匹敌的缘故。抛弃百姓使之去帮助敌国,拒绝宾客使之去侍奉诸侯,使天下的贤士退却而不敢西进,停止脚步不进入秦国,这就叫做“借武器给敌寇,送粮食给盗贼”啊。
    
    许多东西并不产于秦,然而可当做宝物(利用)的却很多;贤士许多都不是出生在秦国,然而愿意效忠秦国的却很多。如今驱逐宾客来资助敌国,减损百姓来充实对手,内部自己造成空虚而外部在诸侯中构筑怨恨,那要谋求国家没有危难,是不可能的啊。
    
\end{normalsize}



\chapter{利议}

\begin{normalsize}
    
    大夫\footnote{〔大夫〕此处指御史大夫桑弘羊,盐铁会议中主张盐铁官营的朝廷代表。}曰:“作世明主,忧劳万人,思念北边之未这,故使使者举贤良、文学\footnote{〔文学〕指被举荐的贤良文学之士,代表地方豪强与儒家学者里反对国营政策的群体。}高弟,说延有道之士,将欲观殊议异策,虚心倾耳以听,庶几云得。诸生无能出奇计,运图匈奴安边境之策,抱丁竹,守空言,不知道舍之宜,时世之变。议论无所依,如膝痒而搔背。辩讼公门之下,兄兄不可胜听,如品即口以成事。此岂明主所欲闻哉?”
    
    文学曰:“诸生对册,殊路同归。指在于崇礼义,退财利,复往古之道,匡当世之失,莫不云太平。虽未尽可但用,宜略有可行者焉。执事暗于明礼,而喻于利未,沮事隳议。计虑筹策以故至今未决。非儒无成事,公卿欲成利也。”
    
    大夫曰:“色厉而内荏,乱真者也。文表而洗里,乱实者也。文学哀衣博带,窃周公\footnote{〔周公〕姬旦,西周初期政治家,儒家理想中的辅政典范,象征礼制与德治。}之服;鞠躬促急,窃仲尼之容;议论称诵,窃商赐\footnote{〔商赐〕卜商(子夏)和端木赐(子贡),孔子弟子,以辩才著称,喻指文学空有口才无实务能力。}之辞;刺讥言治,过管晏\footnote{〔管晏〕管仲与晏婴,春秋齐国之相,代表务实治国的能臣。}之才。心卑卿相,志小万乘。及授之政,昏乱不治。故以言举人,若以毛相马。此其所以多不称举。诏策曰:‘陈嘉宇内之士,故详延四方豪俊文学博习之士,趋迁官禄。’言者不必有德,何者?言之易而行之难。有舍其车而识其牛,贵其不言而多成事也。吴铎\footnote{〔吴铎〕吴地铜铃,铃锤舌形,象征谏言者之舌,比喻主父偃等多言招祸。}以其舌自破,主父偃\footnote{〔主父偃〕汉武帝时纵横家,提议推恩令削弱诸侯。}以其舌自杀。贺旦\footnote{〔贺旦〕"贺"通"鶡",指黎明报晓的鶡鸟(雉类)。}夜鸣,无益于明;主父鸣痴,无益于死。非有司欲成利,文学桎梏于旧术,牵于问言者也。”
    
    文学曰:“能言之,能行之者,汤武也。能言,不能行者,有司也。文学窃周公之服。有司窃周公之位。文学桎梏于旧术,有司桎梏于财利。主父偃以舌自杀,有司以利自困。夫骥之才千里,非造父\footnote{〔造父〕周穆王御者,驾车技术超凡。}不能使。禹之知万人,非舜为相不能用。故季桓子\footnote{〔季桓子〕鲁国权臣季孙斯,排挤孔子。}听政,柳下惠\footnote{〔柳下惠〕展获,春秋鲁国贤士,道德崇高但不得重用。}忽然不见;孔子之为司寇,然后勃炽。骥,举之在伯乐,共功在造父。造父摄辔,马无驽良,皆可取道。周公之时,士无贤不肖,皆可与言治。故御之良者善调马,相之贤者善使士。今举异才面使减驺御之,是犹恶骥盐车而责之使疾。此贤良文学多不称举也。”
    
    大夫曰:“嘻!诸生榻容无行,多言而不用,情貌不相副。若穿俞之盗,自古而患之。是孔丘斥逐于鲁君,会不用于世也。何者?以其首摄多端,迂时而不要也。故秦王燔去其术而不行,坑之渭中而不用。乃安得鼓口舌,申颜眉,预前议论,是非国家之事也?”
\end{normalsize}


\newpage

\textbf{译文}:

\vspace{1em}

\begin{normalsize}
    
    大夫说:“当今的贤明皇帝为万民忧劳,考虑到北方边境还不安定,所以派遣使者四处选拔贤良方正和文学高策,广泛征请有见地的人才,想考察他们非凡的议论和谋略,虚心倾听他们的意见,希望总会有所收获。可是他们这些儒生没有能力提出超人的计谋和抗击匈奴、安定边境的策略,而只会死抱着陈旧的书本,死守着无用的教条,既不知道什么是该努力做的,什么是该舍弃不做的,也不懂得去适应形势的变化。议论问题无的放矢,就象膝盖痒了却去搔背一样。在朝廷上争论,吵吵嚷嚷,使人讨厌,就象‘品’字是由‘口’字垒成的一样,这哪里是贤明的君主所想听到的呢?”
    
    文学说:“我们儒生们回答皇帝的策问,路子不同,目的却只有一个。都是为了崇尚礼义,排斥财利,恢复古代推行的道,用以纠正当代的过失。没有一个不是为了求得天下太平。虽然我们的意见不一定全部可用,但大体上是可行的。您不懂得怎样阐明礼义,而只知道追求财利,败坏了事情,损害了建议,因此谋划策略至今还不能决定下来。这不是因为我们儒生成不了大事,而是因为你们这些公卿大臣一心要牟取财利用职权!”
    
    大夫说:“一个人表面严厉但内心怯儒,这就容易以假乱真。一件衣服外表是刺乡,里子却是粗麻,这也会使虚实难辨。你们这些‘贤良文学’,穿着宽大的衣服,系着宽大的带子,打扮得活象周公;一副毕恭毕敬、诚惶诚恐的样子,又活象孔丘。谈论‘先王之道’,就象子夏、子贡一样有口才;批评时政,谈论治国之道,好象比管仲、晏婴还要有才能。你们打心眼儿里看不起国家的执政大臣甚至连皇帝也不放在你们眼里。但等到真的给你们权力,让你们管理国家,就会把国家搞得一塌糊涂,不可收拾。所以说,如果根据会不会说漂亮话来选拔人材,那就象只看看毛色就来鉴别马的好坏一样。这才是你们的实际才能和被荐举时的名声多不相称的原因所在呵!皇帝的文告说:‘我看重天下有才能的人,所以广泛征请四方有才有德、博古通今的人士,赶快提拔他们做官,增加他们的俸禄。’但是,能说会道的人,却不一定有才干。为什么这样说呢?因为光说很容易,做起来就难了。有的人不在意他的车子,却给他的牛做上标记,(唯恐丢失,)就是因为看重牛不说话但能干好多活。吴地出产的大铃因为它的铜舌长年敲击而破裂,主父偃也因为他的嘴巴厉害而害了自己。
    
    文学说:“能说又能做的,是商汤和周武王。只会说但做不到的,是你们这些当官的。如果说我们文学窃取的是周公的衣服,你们官吏却窃据了周公的职位。我们文学被儒家的老一套所束缚,你们官吏却又被财利束缚住了。主父偃因为他的舌头害了自己,你们却是为了财利而使自己陷入困境。千里马能日行千里,但没有造父就驾驭不了;禹的智慧能顶得上一万人,但没有舜做相就得不到重用。所以,季桓子一执政,柳下惠就突然弃官躲藏起来;孔子做了司寇,贤能之士才大量出现。骥这种千里马,把它从马群里挑选出来的是伯乐,使它发挥出功效的却是造父。造父赶车,马不论好坏,都能听使唤。周公执政的时候,士不论贤和不贤,都可以参与治理天下。所以说,好的赶车人善于调练马,好的相善于使用人。现在选拔出奇才却派无能之辈去驾驭,这就象把千里马套在笨重的盐车上而又要它飞快地奔跑一样。这就是我们贤良文学和受荐举时的名声不相称的原因。”
    
    大夫说:“哈哈!你们这些儒生卑贱而又品质恶劣,光会说不会做,表里不一。就象穿墙越壁的小偷一样,自古就是社会的祸害。这就是孔丘被鲁君驱逐出国,根本不被当世所用的缘故。为什么呢?因为他那套周礼烦琐得要命,和时代离得太远,使人不得要领。所以秦始皇烧去了那些宣扬孔孟的书,不照那一套去做,把那些反动的儒生活埋在咸阳,根本不用他们。这样一来,怎么能让你们这些家伙摇唇鼓舌,眉飞色舞,参加到大臣的行列里高谈阔论,批评国家的政事呢!”
    
\end{normalsize}



\chapter{论贵粟疏}

\begin{normalsize}
    
    圣王在上,而民不冻饥者,非能耕而食之,织而衣之也,为开其资财之道也。故尧、禹\footnote{〔尧、禹〕尧,帝喾之子,传说中古代君主,晚年禅位于舜。禹,传说中古代君主,奉舜命治水有功,舜死后继其位。}有九年之水,汤\footnote{〔汤〕汤,商朝的开国君主。}有七年之旱,而国亡捐瘠者,以畜积多而备先具也。今海内为一,土地人民之众不避汤、禹,加以亡天灾数年之水旱,而畜积未及者,何也?地有遗利,民有余力,生谷之土未尽垦,山泽之利未尽出也,游食之民未尽归农也。
    
    民贫,则奸邪生。贫生于不足,不足生于不农,不农则不地著\footnote{〔地著〕定居一地。《汉书·食货志》:“理民之道,地著为本。”颜师古注:“地著,谓安土也。”},不地著则离乡轻家,民如鸟兽。虽有高城深池,严法重刑,犹不能禁也。夫寒之于衣,不待轻暖;饥之于食,不待甘旨;饥寒至身,不顾廉耻。人情一日不再食则饥,终岁不制衣则寒。夫腹饥不得食,肤寒不得衣,虽慈母不能保其子,君安能以有其民哉?明主知其然也,故务民于农桑,薄赋敛,广畜积,以实仓廪,备水旱,故民可得而有也。
    
    民者,在上所以牧之,趋利如水走下,四方无择也。夫珠玉金银,饥不可食,寒不可衣,然而众贵之者,以上用之故也。其为物轻微易藏,在于把握,可以周海内而无饥寒之患。此令臣轻背其主,而民易去其乡,盗贼有所劝,亡逃者得轻资也。粟米布帛生于地,长于时,聚于力,非可一日成也。数石之重,中人弗胜,不为奸邪所利;一日弗得而饥寒至。是故明君贵五谷而贱金玉。
    
    今农夫五口之家,其服役者不下二人,其能耕者不过百亩,百亩之收不过百石。春耕,夏耘,秋获,冬藏,伐薪樵,治官府,给徭役;春不得避风尘,夏不得避署热,秋不得避阴雨,冬不得避寒冻,四时之间,无日休息。又私自送往迎来,吊死问疾,养孤长幼在其中。勤苦如此,尚复被水旱之灾,急政暴虐,赋敛不时,朝令而暮改。当具有者半贾而卖,无者取倍称之息;于是有卖田宅、鬻子孙以偿债者矣。而商贾大者积贮倍息,小者坐列贩卖,操其奇赢,日游都市,乘上之急,所卖必倍。故其男不耕耘,女不蚕织,衣必文采,食必粱肉;无农夫之苦,有阡陌之得。因其富厚,交通王侯,力过吏势,以利相倾;千里游遨,冠盖相望,乘坚策肥,履丝曳缟。此商人所以兼并农人,农人所以流亡者也。今法律贱商人,商人已富贵矣;尊农夫,农夫已贫贱矣。故俗之所贵,主之所贱也;吏之所卑,法之所尊也。上下相反,好恶乖迕,而欲国富法立,不可得也。
    
    方今之务,莫若使民务农而已矣。欲民务农,在于贵粟;贵粟之道,在于使民以粟为赏罚。今募天下入粟县官,得以拜爵,得以除罪。如此,富人有爵,农民有钱,粟有所渫。夫能入粟以受爵,皆有余者也。取于有余,以供上用,则贫民之赋可损,所谓损有余、补不足,令出而民利者也。顺于民心,所补者三:一曰主用足,二曰民赋少,三曰劝农功。今令民有车骑马一匹者,复卒三人。车骑者,天下武备也,故为复卒。
    
    神农之教曰:“有石城十仞,汤池百步,带甲百万,而无粟,弗能守也。”以是观之,粟者,王者大用,政之本务。令民入粟受爵,至五大夫\footnote{〔五大夫〕先秦至汉代都有的爵位,在二十等爵位里自下数起第九级。}以上,乃复一人耳,此其与骑马之功相去远矣。爵者,上之所擅,出于口而无穷;粟者,民之所种,生于地而不乏。夫得高爵也免罪,人之所甚欲也。使天下人入粟于边,以受爵免罪,不过三岁,塞下之粟必多矣。
\end{normalsize}


\newpage

\textbf{译文}:

\vspace{1em}

\begin{normalsize}
    
    在圣明的君王统治下,百姓不挨饿受冻,这并非是因为君王能亲自种粮食给他们吃,织布匹给他们穿,而是由于他能给人民开辟财源。所以尽管唐尧、夏禹之时有过九年的水灾,商汤之时有过七年的旱灾,但那时没有因饿死而被抛弃和饿瘦的人,这是因为贮藏积蓄的东西多,事先早已作好了准备。现在全国统一,土地之大,人口之多,不亚于汤、禹之时,又没有连年的水旱灾害,但积蓄却不如汤、禹之时,这是什么道理呢?原因在于土地还有潜力,百姓还有余力,能长谷物的土地还没全部开垦,山林湖沼的资源尚未完全开发,游手好闲之徒还没全都回乡务农。
    
    百姓生活贫困了,就会去做邪恶的事。贫困是由于不富足,不富足是由于不务农,不从事农业就不能在一个地方定居下来,不能定居就会离开乡土,轻视家园,象鸟兽一样四处奔散。这样的话,国家即使有高大的城墙,深险的护城河,严厉的法令,残酷的刑罚,还是不能禁止他们。人在寒冷的时候,不会等有了轻暖的皮衣才穿;饥饿的时候,也不会等有了美味才吃;饥寒交迫,就顾不上廉耻了。人之常情是:一天不吃两顿饭就要挨饿,整年不做衣服穿就会受冻。那么,肚子饿了没饭吃,身上冷了无衣穿,即使是慈母也不能留住她的儿子,国君又怎能保有他的百姓呢?贤明的君主懂得这个道理,所以让人民从事农业生产,减轻他们的赋税,大量贮备粮食,以便充实仓库,防备水旱灾荒,因此也就能够拥有人民。
    
    百姓呢,在于君主用什么办法来管理他们,他们追逐利益就象水往低处流一样,不管东南西北。珠玉金银这些东西,饿了不能当饭吃,冷了不能当衣穿;然而人们还是看重它,这是因为君主需要它的缘故。珠玉金银这些物品,轻便小巧,容易收藏,拿在手里,可以周游全国而无饥寒的威胁。这就会使臣子轻易地背弃他的君主,而百姓也随便地离开家乡,盗贼受到了鼓励,犯法逃亡的人有了便于携带的财物。粟米和布帛的原料生在地里,在一定的季节里成长,收获也需要人力,并非短时间内可以成事。几石重的粮食,一般人拿不动它,也不为奸邪的人所贪图;可是这些东西一天得不到就要挨饿受冻。因此,贤明的君主重视五谷而轻视金玉。
    
    现在农夫中的五口之家,家里可以参加劳作的不少于二人,能够耕种的土地不超过百亩,百亩的收成,不超过百石。他们春天耕地,夏天耘田,秋天收获,冬天储藏,还得砍木柴,修理官府的房舍,服劳役;春天不能避风尘,夏天不能避署热,秋天不能避阴雨,冬天不能避寒冻,一年四季,没有一天休息;在私人方面,又要交际往来,吊唁死者,看望病人,抚养孤老,养育幼儿,一切费用都要从农业收入中开支。农民如此辛苦,还要遭受水旱灾害,官府又要急征暴敛,随时摊派,早晨发命令,晚上就要交纳。交赋税的时候,有粮食的人,半价贱卖后完税;没有粮食的人,只好以加倍的利息借债纳税;于是就出现了卖田地房屋、卖子孙来还债的事情。而那些商人们,大的囤积货物,获取加倍的利息;小的开设店铺,贩卖货物,牟取利润。他们每日都去集市游逛,趁政府急需货物的机会,所卖物品的价格就成倍抬高。所以商人家中男的不必耕地耘田,女的不用养蚕织布,穿的必定是华美的衣服,吃的必定是上等米和肉;没有农夫的劳苦,却占有农桑的收获。依仗自己富厚的钱财,与王侯接交,势力超过官吏,凭借资产相互倾轧;他们遨游各地,车乘络绎不绝,乘着坚固的车,赶着壮实的马,脚穿丝鞋,身披绸衣。这就是商人兼并农民土地,农民流亡在外的原因。当今虽然法律轻视商人,而商人实际上已经富贵了;法律尊重农民,而农民事实上却已贫贱了。所以一般俗人所看重的,正是君主所轻贱的;一般官吏所鄙视的,正是法律所尊重的。上下相反,好恶颠倒,在这种情况下,要想使国家富裕,法令实施,那是不可能的。
    
    当今的迫切任务,没有比使人民务农更为重要的了。而要想使百姓从事农业,关键在于抬高粮价;抬高粮价的办法,在于让百姓拿粮食来求赏或免罚。现在应该号召天下百姓交粮给政府,纳粮的可以封爵,或赎罪。这样,富人就可以得到爵位,农民就可以得到钱财,粮食就不会囤积而得到流通。那些能交纳粮食得到爵位的,都是富有产业的人。从富有的人那里得到货物来供政府用,那么贫苦百姓所担负的赋税就可以减轻,这就叫做拿富有的去补不足的,法令一颁布百姓就能够得益。依顺百姓心愿,有三个好处:一是君主需要的东西充足,二是百姓的赋税减少,三是鼓励从事农业生产。按现行法令,民间能输送一匹战马的,就可以免去三个人的兵役。战马是国家战备所用,所以可以使人免除兵役。
    
    神农氏曾教导说:“有七八丈高的石砌城墙,有百步之宽贮满沸水的护城河,上百万全副武装的兵士,然而没有粮食,那是守不住的。”这样看来,粮食是君王最需要的资财,是国家最根本的政务。现在让百姓交粮买爵,封到五大夫以上,才免除一个人的兵役,这与一匹战马的功用相比差得太远了。赐封爵位,是皇上专有的权力,只要一开口,就可以无穷无尽地封给别人;粮食,是百姓种出来的,生长在土地中而不会缺乏。能够封爵与赎罪,是人们十分向往的。假如叫天下百姓都献纳粮食,用于边塞,以此换取爵位或赎罪,那么不用三年,边地粮食必定会多起来。
    
\end{normalsize}


\newpage

\textbf{注解}:

\vspace{-1em}

\begin{itemize}
    \setlength\itemsep{-0.2em}
    \item〔五大夫〕自商鞅变法以来,秦国设二十等爵位,以赏军功:一级公士,二级上造,三级簪袅,四级不更,五级大夫,六级官大夫,七级公大夫,八级公乘,九级五大夫,十级左庶长,十一级右庶长,十二级左更,十三级中更,十四级右更,十五级少上造,十六级大上造,十七级驷车庶长,十八级大庶长,十九级关内侯,二十级彻侯。彻侯为最高等,以一县为食邑,并得以自置吏于封地;其次是关内侯,有食邑、封户,但只能食税;大庶长以下皆有岁俸;公士为最低等,临战斩敌甲士(披甲士兵)首一级即赐,得田一顷、宅一处、仆一人。
\end{itemize}

\chapter{论积贮疏}

\begin{normalsize}
    
    管子\footnote{〔管子〕即管仲。后人把他的学说和托名的著作编辑成《管子》一书,共二十四卷。}曰:“仓廪实而知礼节。”民不足而可治者,自古及今,未之尝闻。古之人曰:“一夫不耕,或受之饥;一女不织,或受之寒。” 生之有时,而用之亡度,则物力必屈。古之治天下,至孅至悉也,故其畜积足恃。今背本而趋末,食者甚众,是天下之大残也;淫侈之俗,日日以长,是天下之大贼也。残贼公行,莫之或止;大命将泛,莫之振救。生之者甚少,而靡之者甚多,天下财产何得不蹶!
    
    汉之为汉,几四十年矣,公私之积,犹可哀痛!失时不雨,民且狼顾;岁恶不入,请卖爵子,既闻耳矣。安有为天下阽危者若是而上不惊者?世之有饥穰,天之行也,禹、汤\footnote{〔禹、汤〕禹,传说中古代部落联盟领袖。原为夏后氏部落领袖,奉舜命治水有功,舜死后继其位。汤,商朝的开国君主。}被之矣。即不幸有方二三千里之旱,国胡以相恤?卒然边境有急,数千百万之众,国胡以馈之?兵旱相乘,天下大屈,有勇力者聚徒而衡击;罢夫羸老易子而咬其骨。政治未毕通也,远方之能疑者,并举而争起矣。乃骇而图之,岂将有及乎?
    
    夫积贮者,天下之大命也。苟粟多而财有余,何为而不成?以攻则取,以守则固,以战则胜。怀敌附远,何招而不至!今殴民而归之农,皆著于本;使天下各食其力,末技游食之民,转而缘南亩,则畜积足而人乐其所矣。可以为富安天下,而直为此廪廪也!窃为陛下\footnote{〔陛下〕指汉文帝刘恒(公元前180至157年在位)。}惜之。
\end{normalsize}


\newpage

\textbf{译文}:

\vspace{1em}

\begin{normalsize}
    
    管子说:“粮仓充足,百姓就懂得礼节。”百姓缺吃少穿而可以治理得好的,从古到今,没有听说过这事。古代的人说:“一个男子不耕地,有人就要因此挨饿;一个女子不织布,有人就要因此受冻。”生产东西有时节的限制,而消费它却没有限度,那么社会财富一定会缺乏。古代的人治理国家,考虑得极为细致和周密,所以他们的积贮足以依靠。现在人们弃农经商(不生产而)吃粮的人很多,这是国家的大祸患。过度奢侈的风气一天天地滋长,这也是国家的大祸害。这两种大祸害公然盛行,没有人去稍加制止;国家的命运将要覆灭,没有人去挽救;生产的人极少,而消费的人很多,国家的财富怎能不枯竭呢?
    
    汉朝从建国以来,快四十年了,公家和个人的积贮还少得令人痛心。错过季节不下雨,百姓就将忧虑不安,年景不好,百姓纳不了税,就要请求卖掉自己的爵级和孩子。这样的事情皇上已经耳有所闻了,哪有治理国家已经危险到这种地步而皇上不震惊的呢?世上有灾荒,这是自然界常有的现象,夏禹、商汤都曾遭受过。假如不幸有纵横二三千里地方的大旱灾,国家用什么去救济灾区?如果突然边境上有紧急情况,成千上万的军队,国家拿什么去发放粮饷?假若兵灾旱灾交互侵袭,国家财富极其缺乏,胆大力壮的人就聚集歹徒横行抢劫,年老体弱的人就互换子女来吃;政治的力量还没有完全达到各地,边远地方敢于同皇上对抗的人,就一同举兵起来造反了。于是皇上才惊慌不安地谋划对付他们,难道还来得及吗?
    
    积贮,是国家的命脉。如果粮食多财力充裕,干什么事情会做不成?凭借它去进攻就能攻取,凭借它去防守就能巩固,凭借它去作战就能战胜。使敌对的人归降,使远方的人顺附,招谁而不来呢?现在如果驱使百姓,让他们归向农业,都附着于本业,使天下的人靠自己的劳动而生活,工商业者和不劳而食的游民,都转向田间从事农活,那么积贮就会充足,百姓就能安居乐业了。本来可以做到使国家富足安定,却竟造成了这种令人危惧的局面!我真替陛下痛惜啊!
    
\end{normalsize}



\chapter{苏武传}

\begin{normalsize}
    
    武字子卿,少以父任,兄弟并为郎\footnote{〔郎〕官名,汉代专指职位较低皇帝侍从。}。稍迁至栘中厩监。时汉连伐胡,数通使相窥观。匈奴留汉使郭吉\footnote{〔郭吉〕元封元年(公元前110年),汉武帝亲统大军十八万到北地,派郭吉到匈奴,晓谕单于归顺,单于大怒,扣留了郭吉。}、路充国\footnote{〔路充国〕元封四年(公元前107年),匈奴派遣使者至汉,病故。汉派路充国送丧到匈奴,单于以为是被汉杀死,扣留了路充国。}等,前后十余辈。匈奴使来,汉亦留之以相当。天汉元年\footnote{〔天汉〕汉武帝年号(公元前100至前年)。},且鞮侯\footnote{〔且鞮侯〕单于嗣位前的封号。}单于初立,恐汉袭之,乃曰:“汉天子我丈人行也。”尽归汉使路充国等。武帝嘉其义,乃遣武以中郎将\footnote{〔中郎将〕皇帝的侍卫长。}使持节送匈奴使留在汉者,因厚赂单于,答其善意。武与副中郎将张胜及假吏\footnote{〔假吏〕临时委任的使臣属官。}常惠等募士斥候\footnote{〔斥候〕军中警卫侦察的士兵。}百余人俱,既至匈奴,置币遗单于。单于益骄,非汉所望也。
    
    方欲发使送武等,会缑王\footnote{〔缑王〕匈奴的一个亲王。}与长水\footnote{〔长水〕水名,在今陕西省蓝田县西北。}虞常\footnote{〔虞常〕长水人,后投降匈奴。}等谋反匈奴中。缑王者,昆邪王\footnote{〔昆邪王〕匈奴一个部落的王,其领地在河西(今甘肃省西北部)。昆邪王于元狩二年降汉。}姊子也,与昆邪王俱降汉,后随浞野侯\footnote{〔浞野侯〕汉将赵破奴的封号。汉武帝太初二年(公元前103年)率二万骑击匈奴,兵败而降,全军沦没。}没胡中。及卫律\footnote{〔卫律〕本为长水胡人,但在汉地长大,被协律都尉李延年荐为汉使出使匈奴。回汉后,正值李延年因罪全家被捕,卫律怕受牵连,又逃奔匈奴,被封为丁零王。}所将降者,阴相与谋劫单于母阏氏\footnote{〔阏氏〕匈奴王后封号。}归汉。会武等至匈奴。虞常在汉时,素与副张胜相知,私候胜曰:“闻汉天子甚怨卫律,常能为汉伏弩射杀之,吾母与弟在汉,幸蒙其赏赐。”张胜许之,以货物与常。
    
    后月余,单于出猎,独阏氏子弟在。虞常等七十余人欲发,其一人夜亡,告之。单于子弟发兵与战,缑王等皆死,虞常生得。单于使卫律治其事,张胜闻之,恐前语发,以状语武。武曰:“事如此,此必及我。见犯乃死,重负国。”欲自杀,胜、惠共止之。虞常果引张胜。单于怒,召诸贵人议,欲杀汉使者。左伊秩訾\footnote{〔左伊秩訾〕匈奴的王号。}曰:“即谋单于,何以复加?宜皆降之。”
    
    单于使卫律召武受辞。武谓惠等:“屈节辱命,虽生,何面目以归汉!”引佩刀自刺。卫律惊,自抱持武,驰召医。凿地为坎,置煴火,覆武其上,蹈其背以出血。武气绝,半日复息。惠等哭,舆归营。单于壮其节,朝夕遣人候问武,而收系张胜。
    
    武益愈,单于使使晓武,会论虞常,欲因此时降武。剑斩虞常已,律曰:“汉使张胜谋杀单于近臣,当死。单于募降者赦罪。”举剑欲击之,胜请降。律谓武曰:“副有罪,当相坐。”武曰:“本无谋,又非亲属,何谓相坐?”复举剑拟之,武不动。律曰:“苏君,律前负汉归匈奴,幸蒙大恩,赐号称王,拥众数万,马畜弥山,富贵如此!苏君今日降,明日复然。空以身膏草野,谁复知之!”武不应。律曰:“君因我降,与君为兄弟;今不听吾计,后虽复欲见我,尚可得乎?”武骂律曰:“汝为人臣子,不顾恩义,畔主背亲,为降虏于蛮夷,何以汝为见!且单于信汝,使决人死生;不平心持正,反欲斗两主,观祸败!南越\footnote{〔南越〕南越国,现在广东、广西南部一带。}杀汉使者,屠为九郡。宛王\footnote{〔宛王〕指大宛国王毋寡。汉武帝太初元年(公元前104年),宛王毋寡派人杀前来求良马的汉使。武帝即命李广利讨伐大宛,大宛诸贵族乃杀毋寡而降汉。}杀汉使者,头县北阙。朝鲜杀汉使者,即时诛灭。独匈奴未耳。若知我不降明,欲令两国相攻。匈奴之祸,从我始矣。”
    
    律知武终不可胁,白单于。单于愈益欲降,乃幽武置大窖中,绝不饮食。天雨雪,武卧啮雪,与旃毛并咽之,数日不死。匈奴以为神,乃徙武北海\footnote{〔北海〕当时在匈奴北境,即今贝加尔湖。}上无人处,使牧羝,羝乳乃得归。别其官属常惠等各置他所。武既至海上,廪食不至,掘野鼠去草实而食之。杖汉节牧羊,卧起操持,节旄尽落。积五六年,单于弟於靬王\footnote{〔於靬王〕且鞮单于之弟。}弋射海上。武能网纺缴,檠弓弩,於靬王爱之,给其衣食。三岁余,王病,赐武马畜、服匿\footnote{〔服匿〕盛酒酪的容器,类似今天的坛子。}、穹庐\footnote{〔穹庐〕圆顶大篷帐,发展为现今的蒙古包。}。王死后,人众徙去。其冬,丁令\footnote{〔丁令〕即丁灵,匈奴北边的一个部族。}盗武牛羊,武复穷厄。
    
    初,武与李陵\footnote{〔李陵〕字少卿,西汉陇西成纪(今甘肃秦安)人,李广之孙,武帝时曾为侍中。天汉二年(前99年)出征匈奴,兵败投降,后病死匈奴。}俱为侍中\footnote{〔侍中〕官名,皇帝的侍从。}。武使匈奴,明年,陵降,不敢求武。久之,单于使陵至海上,为武置酒设乐。因谓武曰:“单于闻陵与子卿素厚,故使陵来说足下,虚心欲相待。终不得归汉,空自苦亡人之地,信义安所见乎?前长君\footnote{〔长君〕指苏武的长兄苏嘉。}为奉车\footnote{〔奉车〕即“奉车都尉”,皇帝出巡时,负责车马的侍从官。},从至雍\footnote{〔雍〕汉代县名,在今陕西凤翔县南。}棫阳宫\footnote{〔棫阳宫〕秦时所建宫殿,在雍东北。},扶辇下除,触柱折辕,劾大不敬\footnote{〔大不敬〕不敬皇帝,是大不敬罪。},伏剑自刎,赐钱二百万以葬。孺卿\footnote{〔孺卿〕苏武弟苏贤的字。}从祠河东\footnote{〔河东〕汉郡名,在今山西夏县北。}后土\footnote{〔后土〕地神。},宦骑与黄门驸马<footnote:N34>争船,推堕驸马河中溺死,宦骑亡,诏使孺卿逐捕,不得,惶恐饮药而死。来时太夫人\footnote{〔太夫人〕指苏武的母亲。}已不幸,陵送葬至阳陵\footnote{〔阳陵〕汉时有阳陵县,在今陕西咸阳市东。}。子卿妇年少,闻已更嫁矣。独有女弟二人,两女一男,今复十余年,存亡不可知。人生如朝露,何久自苦如此!陵始降时,忽忽如狂,自痛负汉,加以老母系保宫\footnote{〔保宫〕囚禁犯罪大臣及其眷属之处。}。子卿不欲降,何以过陵?且陛下春秋高,法令亡常,大臣亡罪夷灭者数十家,安危不可知,子卿尚复谁为乎?愿听陵计,勿复有云。”
    
    武曰:“武父子亡功德,皆为陛下所成就,位列将,爵通侯\footnote{〔通侯〕汉爵位名,本名彻侯,因避武帝讳改。苏武父苏建曾封为平陵侯。},兄弟亲近,常愿肝脑涂地。今得杀身自效,虽蒙斧钺汤镬,诚甘乐之。臣事君,犹子事父也。子为父死,亡所恨,愿无复再言!”
    
    陵与武饮数日,复曰:“子卿壹听陵言!”武曰:“自分已死久矣!王必欲降武,请毕今日之欢,效死于前!”陵见其至诚,喟然叹曰:“嗟呼,义士!陵与卫律之罪上通于天!”因泣下沾衿,与武决去。
    
    昭帝\footnote{〔昭帝〕汉昭帝刘弗陵,武帝少子(前87年至前74年在位)。即位次年改元始元。于始元六年(公元前81年),与匈奴达成和议。}即位,数年,匈奴与汉和亲。汉求武等,匈奴诡言武死。后汉使复至匈奴,常惠请其守者与俱,得夜见汉使,具自陈道。教使者谓单于,言天子射上林\footnote{〔上林〕即上林苑。故址在今陕西省西安市附近。汉朝皇帝游玩射猎的园林。}中,得雁足有系帛书,言武等在荒泽中。使者大喜,如惠语以让单于。单于视左右而惊,谢汉使曰:“武等实在。”
    
    单于召会武官属,前以降及物故,凡随武还者九人。武以始元六年\footnote{〔京师〕指西汉京城长安。}春至京师。武留匈奴凡十九岁,始以强壮出,及还,须发尽白。
\end{normalsize}


\newpage

\textbf{译文}:

\vspace{1em}

\begin{normalsize}
    
    苏武字子卿,年轻时凭着父亲的职位,兄弟三人都做了皇帝的侍从,并逐渐被提升为掌管皇帝鞍马鹰犬射猎工具的官。当时汉朝廷不断讨伐匈奴,多次互派使节彼此暗中侦察。匈奴扣留了汉使节郭吉、路充国等前后十余批人。匈奴使节前来,汉朝庭也扣留他们以相抵。天汉元年,且鞮刚刚立为单于,唯恐受到汉的袭击,于是说:“汉皇帝,是我的长辈。”全部送还了汉廷使节路充国等人。汉武帝赞许他这种通晓情理的做法,于是派遣苏武以中郎将的身份出使,持旄节护送扣留在汉的匈奴使者回国,顺便送给单于很丰厚的礼物,以答谢他的好意。苏武同副中郎将张胜以及临时委派的使臣属官常惠等,加上招募来的士卒、侦察人员百多人一同前往。到了匈奴那里,摆列财物赠给单于。单于越发傲慢,不是汉所期望的那样。
    
    单于正要派使者护送苏武等人归汉,适逢缑王与长水人虞常等人在匈奴内部谋反。缑 王是昆邪王姐姐的儿子,与昆邪王一起降汉,后来又跟随浞野侯赵破奴重新陷胡地,在卫律统率的那些投降者中,暗同策划绑架单于的母亲阏氏归汉。正好碰上苏武等人到匈奴。虞常在汉的时候,一向与副使张胜有交往,私下拜访张胜,说:“听说汉天子很怨恨卫律,我虞常能为汉廷埋伏弩弓将他射死。我的母亲与弟弟都在汉,希望受到汉廷的照顾。”张胜许诺了他,把财物送给了虞常。
    
    一个多月后,单于外出打猎,只有阏氏和单于的子弟在家。虞常等七十余人将要起事,其中一人夜晚逃走,把他们的计划报告了阏氏及其子弟。单于子弟发兵与他们交战,缑王等都战死;虞常被活捉。单于派卫律审处这一案件,张胜听到这个消息,担心他和虞常私下所说的那些话被揭发,便把事情经过告诉了苏武。苏武说:“事情到了如此地步,这样一定会牵连到我们。受到侮辱才去死,更对不起国家!”因此想自杀,张胜、常惠一起制止了他。虞常果然供出了张胜。单于大怒,召集许多贵族前来商议,想杀掉汉使者。左伊秩訾说:“假如是谋杀单于,又用什么更严的刑法呢?应当都叫他们投降。”
    
    单于派卫律召唤苏武来受审讯。苏武对常惠说:“丧失气节、玷辱使命,即使活着,还有什么脸面回到汉廷去呢!”说着拔出佩带的刀自刎,卫律大吃一惊,自己抱住、扶好苏武,派人骑快马去找医生。医生在地上挖一个坑,在坑中点燃微火,然后把苏武脸朝下放在坑上,轻轻地敲打他的背部,让淤血流出来。苏武本来已经断了气,这样过了好半天才重新呼吸。常惠等人哭泣着,用车子把苏武拉回营帐。单于钦佩苏武的节操,早晚派人探望、询问苏武,而把张胜逮捕监禁起来。
    
    苏武的伤势逐渐好转,单于派使者通知苏武,一起来审处虞常,想借这个机会使苏武投降。剑斩虞常后,卫律说:“汉使张胜,谋杀单于亲近的大臣,应当处死。单于招降的人,赦免他们的罪。”举剑要击杀张胜,张胜请求投降。卫律对苏武说:“副使有罪,应该连坐到你。”苏武说:“我本来就没有参予谋划,又不是他的亲属,怎么谈得上连坐?”卫律又举剑对准苏武,苏武岿然不动。卫律说:“苏君,我卫律以前背弃汉廷,归顺匈奴,幸运地受到单于的大恩,赐我爵号,让我称王;拥有奴隶数万、马和其他牲畜满山,如此富贵!苏君你今日投降,明日也是这样。白白地用身体给草地做肥料,又有谁知道你呢!”苏武毫无反应。卫律说:“你顺着我而投降,我与你结为兄弟;今天不听我的安排,以后再想见我,还能得到机会吗?”苏武痛骂卫律说:“你做人家的臣下和儿子,不顾及恩德义理,背叛皇上、抛弃亲人,在异族那里做投降的奴隶,我为什么要见你!况且单于信任你,让你决定别人的死活,而你却居心不平,不主持公道,反而想要使汉皇帝和匈奴单于二主相斗,旁观两国的灾祸和损失!南越王杀汉使者,结果九郡被平定。宛王杀汉使者,自己头颅被悬挂在宫殿的北门。朝鲜王杀汉使者,随即被讨平。唯独匈奴未受惩罚。你明知道我决不会投降,想要使汉和匈奴互相攻打。匈奴灭亡的灾祸,将从我开始了!”
    
    卫律知道苏武终究不可胁迫投降,报告了单于。单于越发想要使他投降,就把苏武囚禁起来,放在大地窖里面,不给他喝的吃的。天下雪,苏武卧着嚼雪,同毡毛一起吞下充饥,几日不死。匈奴以为神奇,就把苏武迁移到北海边没有人的地方,让他放牧公羊,说等到公羊生了小羊才得归汉。同时把他的部下及其随从人员常惠等分别安置到别的地方。苏武迁移到北海后,粮食运不到,只能掘取野鼠所储藏的野生果实来吃。他拄着汉廷的符节牧羊,睡觉、起来都拿着,以致系在节上的牦牛尾毛全部脱尽。一共过了五、六年,单于的弟弟於靬王到北海上打猎。苏武会编结打猎的网,矫正弓弩,於靬王颇器重他,供给他衣服、食品。三年多过后,於靬王得病,赐给苏武马匹和牲畜、盛酒酪的瓦器、圆顶的毡帐篷。王死后,他的部下也都迁离。这年冬天,丁令人盗去了苏武的牛羊,苏武又陷入穷困。
    
    当初,苏武与李陵都为侍中。苏武出使匈奴的第二年,李陵投降匈奴,不敢访求苏武。时间一久,单于派遣李陵去北海,为苏武安排了酒宴和歌舞。李陵趁机对苏武说:“单于听说我与你交情一向深厚,所以派我来劝说足下,愿谦诚地相待你。你终究不能回归本朝了,白白地在荒无人烟的地方受苦,你对汉廷的信义又怎能有所表现呢?以前你的大哥苏嘉做奉车都尉,跟随皇上到雍的棫宫,扶着皇帝的车驾下殿阶,碰到柱子,折断了车辕,被定为大不敬的罪,用剑自杀了,只不过赐钱二百万用以下葬。你弟弟孺卿跟随皇上去祭祀河东土神,骑着马的宦官与驸马争船,把驸马推下去掉到河中淹死了。骑着马的宦官逃走了。皇上命令孺卿去追捕,他抓不到,因害怕而服毒自杀。我离开长安的时候,你的母亲已去世,我送葬到阳陵。你的夫人年纪还轻,听说已改嫁了,家中只有两个妹妹,两个女儿和一个男孩,如今又过了十多年,生死不知。人生像早晨的露水,何必长久地像这样折磨自己!我刚投降时,终日若有所失,几乎要发狂,自己痛心对不起汉廷,加上老母拘禁在保宫,你不想投降的心情,怎能超过当时我李陵呢!并且皇上年纪大了,法令随时变更,大臣无罪而全家被杀的有十几家,安危不可预料。你还打算为谁守节呢?
    
    苏武说:“我苏武父子无功劳和恩德,都是皇帝栽培提拔起来的,官职升到列将,爵位封为通侯,兄弟三人都是皇帝的亲近之臣,常常愿意为朝庭牺牲一切。现在得到牺牲自己以效忠国家的机会,即使受到斧钺和汤镬这样的极刑,我也心甘情愿。大臣效忠君王,就像儿子效忠父亲。儿子为父亲而死,没有什么可恨,希望你不要再说了!”
    
    李陵与苏武共饮了几天,又说:“你一定要听从我的话。”苏武说:“我料定自己已经是死去的人了!单于一定要逼迫我投降,那么就请结束今天的欢乐,让我死在你的面前!”李陵见苏武对朝廷如此真诚,慨然长叹道:“啊,义士!我李陵与卫律的罪恶,上能达天!”说着眼泪直流,浸湿了衣襟,告别苏武而去。
    
    汉昭帝登位,几年后,匈奴和汉达成和议。汉廷寻求苏武等人,匈奴撒谎说苏武已死。后来汉使者又到匈奴,常惠请求看守他的人同他一起去,在夜晚见到了汉使,原原本本地述说了几年来在匈奴的情况。告诉汉使者要他对单于说:“天子在上林苑中射猎,射得一只大雁,脚上系着帛书,上面说苏武等人在北海。”汉使者万分高兴,按照常惠所教的话去责问单于。单于看着身边的人十分惊讶,向汉使道歉说:“苏武等人的确还活着。”
    
    单于召集苏武的部下,除了以前已经投降和死亡的,总共跟随苏武回来的有九人。苏武于汉昭帝始元六年(前81年)春回到长安。苏武被扣在匈奴共十九年,当初壮年出使,等到回来,胡须头发全都白了。
    
\end{normalsize}



\chapter{原君}

\begin{normalsize}
    
    有生之初,人各自私也,人各自利也。天下有公利而莫或兴之,有公害而莫或除之。有人者出,不以一己之利为利,而使天下受其利;不以一己之害为害,而使天下释其害。此其人之勤劳,必千万于天下之人。夫以千万倍之勤劳,则己又不享其利,必非天下之人情所欲居也。故古人之君,量而不欲入者,许由\footnote{〔许由〕传说中尧欲让天下而不受的隐士,拒君位隐耕箕山。}、务光\footnote{〔务光〕商汤让位时投水拒受的隐士,象征淡泊名利的古贤。}是也;入而又去之者,尧、舜\footnote{〔尧、舜〕古时圣君,主动禅让君位。}是也;初不欲入而不得去者,禹\footnote{〔禹〕开启世袭制的君主,传位于其子启。}是也。岂古之人有所异哉?好逸恶劳,亦犹夫人之情也。
    
    后之为人君者不然。以为天下利害之权皆出于我,我以天下之利尽归于己,以天下之害尽归于人,亦无不可。使天下之人不敢自私,不敢自利,以我之大私为天下之公。始而惭焉,久而安焉,视天下为莫大之产业,传之子孙,受享无穷。汉高帝\footnote{〔汉高帝〕即汉高祖刘邦。}所谓“某业所就,孰与仲多”者,其逐利之情,不觉溢之于辞矣。
    
    此无他,古者以天下为主,君为客,凡君之所毕世而经营者,为天下也。今也以君为主,天下为客,凡天下之无地而得安宁者,为君也。是以其未得之也,屠毒天下之肝脑,离散天下之子女,以博我一人之产业,曾不惨然,曰:“我固为子孙创业也。”其既得之也,敲剥天下之骨髓,离散天下之子女,以奉我一人之淫乐,视为当然,曰:“此我产业之花息也。”然则为天下之大害者,君而已矣!向使无君,人各得自私也,人各得自利也。呜呼!岂设君之道固如是乎?
    
    古者天下之人爱戴其君,比之如父,拟之如天,诚不为过也。今也天下之人,怨恶其君,视之如寇仇,名之为独夫,固其所也。而小儒规规焉以君臣之义无所逃于天地之间,至桀纣\footnote{〔桀纣〕夏桀商纣,都是暴君。}之暴,犹谓汤武\footnote{〔汤武〕商汤周武王,儒家认可的"吊民伐罪"革命圣王典范。}不当诛之,而妄传伯夷、叔齐\footnote{〔伯夷、叔齐〕耻食周粟饿死首阳山的遗民,被用作盲目忠君的反面案例。}无稽之事,乃兆人万姓崩溃之血肉,曾不异夫腐鼠。岂天地之大,于兆人万姓之中,独私其一人一姓乎?是故武王圣人也,孟子之言,圣人之言也。后世之君,欲以如父如天之空名,禁人之窥伺者,皆不便于其言,至废孟子而不立,非导源于小儒乎?
    
    虽然,使后之为君者,果能保此产业,传之无穷,亦无怪乎其私之也。既以产业视之,人之欲得产业,谁不如我?摄缄縢,固扃鐍,一人之智力,不能胜天下欲得之者之众。远者数世,近者及身,其血肉之崩溃,在其子孙矣。昔人愿世世无生帝王家\footnote{〔昔人愿世世无生帝王家〕南朝宋孝武帝在位期间宠爱自己第八个儿子刘子鸾。太子刘子业妒其才,继位后赐死。子鸾临死谓左右曰:“愿身不复生王家。”宋顺帝刘准禅位于齐王,将军王敬则捉住刘准。刘准泣曰:“愿后身世世勿复生天王家!”},而毅宗\footnote{〔毅宗〕即明思宗朱由检(崇祯帝,公元1627-1644在位)。}之语公主,亦曰:“若何为生我家!”痛哉斯言!回思创业时,其欲得天下之心,有不废然摧沮者乎?是故明乎为君之职分,则唐、虞之世,人人能让,许由、务光非绝尘也;不明乎为君之职分,则市井之间,人人可欲,许由、务光所以旷后世而不闻也。然君之职分难明,以俄顷淫乐,不易无穷之悲,虽愚者亦明之矣。
\end{normalsize}


\newpage

\textbf{译文}:

\vspace{1em}

\begin{normalsize}
    
    人类社会开始之后,人都是自私的,也是自利的。社会上对公众有利的事却无人兴办它,对公众有害的事也无人去除掉它。有这样一个人出来,他不以自己一人的利益作为利益,却让天下人得到他的利益;不以自己一人的祸患作为祸患,却让天下人免受他的祸患。那个人的勤苦辛劳,必定是天下人的千万倍。拿出千万倍的勤苦辛劳,而自己却又不享受利益,这必然不是天下常人之情所愿意的。所以古时的君主,考虑后而不愿就位的,是许由、务光等人;就位而又离位的,是尧、舜等人;起先不愿就位而最终却未能离位的,是大禹了。难道说古代人有什么不同吗?喜好安逸,厌恶劳动,也像常人情况一样啊。
    
    后代做人君的却不是这样了。他们认为天下的利害大权都出于自己,我将天下的利益都归于自己,将天下的祸患都归于别人,也没有什么不可以的。让天下的人不敢自私,不敢自利,将自己的大私作为天下的公利。开始时对此还觉得惭愧,时间久了也就心安理得了,将天下看作是广大的产业,把它传给子孙,享受无穷。正如汉高祖所说的“我的产业所达到的成就,与二哥相比,究竟谁多呢?”
    
    这没有其他原因,古时将天下看成是主,将君主看作是客,凡是君主一世所经营的,都是为了天下人。现在将君主看作主,将天下看作是客,凡是天下没有一地能够得到安宁的,正是在于为君主啊。因而当他未得到天下时,使天下的人民肝脑涂地,使天下的子女离散,以增多自己一个人的产业,对此并不感到悲惨,还说:“我本来就是为子孙创业呀。”当他已得到天下后,就敲诈剥夺天下人的骨髓,离散天下人的子女,以供奉自己一人的荒淫享乐,把这视作理所当然,说:“这些都是我的产业的利息呀。”既然这样,作为天下最大的祸害,只是君主而已!当初假使没有君主,人们都能得到自己的东西,人们都能得到自己的利益。唉!难道设立君主的道理本来就是这样的吗?
    
    古时候天下的人都爱戴他们的君主,把他比作父亲,拟作青天,实在是不算过分。如今天下的人都怨恨他们的君主,将他看成仇敌一样,称他为“独夫”,本来就是他应该得到的结果。但小儒死守旧义,认为君臣间的关系存在于天地之间,难以逃脱,甚至像夏桀、殷纣那样残暴,竟还说商汤、周武王不应杀他们,而编造流传伯夷、叔齐的无从查考之事,把千千万万老百姓的死,看成与老鼠的死没有什么两样。难道天地这样大,却在千千万万的百姓之中,只偏爱君主的一人一姓吗?因此周武王是圣人,孟子的言论是圣人的言论。后世的君主们,企图用"如父如天"的虚名,禁止他人对君位的觊觎,都因这些言论不利于统治,甚至贬黜孟子地位不立其祀,难道不是源于那些浅陋儒生的曲解吗?
    
    虽是这样,如果后代做君主的,果真能保住这产业,把它永远传下去,也不怪他将天下当作私有了。既然将它看作产业,旁人想得到产业的念头,有谁不像自己呢?于是用绳捆紧,用锁加固,但一个人的智慧和力量,并不能战胜天下要得到它的众多的人。远的不过几代,近的就在自身,他们血肉的崩溃,就应在子孙的身上了。旧时就有人希望下辈子不要投生到帝王家中,而明毅宗对公主也说:“你为什么要生在我家!”这话真痛啊!回想他们祖上创业之时,志在占据天下的雄心,哪有不垂头沮丧的呢?因此明白作君主的职责,那么唐尧、虞舜的时代,人人都能推让君位,许由、务光也并非超尘绝俗的人;不明了作君的职责,那么就连市井之间,人人都想得到君位,许由、务光因而绝迹于后世而听不到了。虽然君主的职分难以明了,但用片刻的荒淫享乐,不值得换取无穷的悲哀,即使是愚蠢的人也能明白这一道理的。
    
\end{normalsize}


\newpage

\textbf{注解}:

\vspace{-1em}

\begin{itemize}
    \setlength\itemsep{-0.2em}
    \item〔某业所就,孰与仲多〕《史记·高祖本纪》载汉高祖刘邦登帝位后,曾对他父亲说:“您以前常说我是无赖,不能治理产业,不如仲(其兄刘仲)厉害,今天我成就的功业,和仲相比谁多?”这也说明刘邦把治国视为治理产业,把君位视为使用权力营私获利的工具,而不是服务人民的工具。
\end{itemize}

\chapter{乐毅报燕王书}

\begin{normalsize}
    
    昌国君乐毅,为燕昭王合五国之兵\footnote{〔五国之兵〕赵、楚、韩、燕、魏五国联军。}而攻齐,下七十馀城,尽郡县之以属燕。三城\footnote{〔三城〕指齐国的聊城、莒、即墨三城,都在今山东省。}未下,而燕昭王死。惠王即位,用齐人反间,疑乐毅,而使骑劫\footnote{〔骑劫〕燕国将领。}代之将。乐毅奔赵,赵封以为望诸君。齐田单\footnote{〔田单〕战国时齐国大将,屡立战功,封安平君,被齐襄王任为国相。}诈骑劫,卒败燕军,复收七十余城以复齐。
    
    燕王悔,惧赵用乐毅乘燕之弊以伐燕。燕王乃使人让乐毅,且谢之曰:“先王举国而委将军,将军为燕破齐,报先王之仇,天下莫不振动。寡人岂敢一日而忘将军之功哉!会先王弃群臣,寡人新即位,左右误寡人。寡人之使骑劫代将军,为将军久暴露于外,故召将军,且休计事。将军过听,以与寡人有隙,遂捐燕而归赵。将军自为计则可矣,而亦何以报先王之所以遇将军之意乎?”
    
    望诸君\footnote{〔望诸君〕赵国给乐毅的封号。}乃使人献书报燕王曰:臣不佞,不能奉承先王\footnote{〔先王〕即燕惠王之父燕昭王。}之教,以顺左右之心,恐抵斧质之罪,以伤先王之明,而又害于足下之义,故遁逃奔赵。自负以不肖之罪,故不敢为辞说。今王使使者数之罪,臣恐侍御者\footnote{〔侍御者〕侍侯国君的人,实指燕惠王。}之不察先王之所以畜幸臣之理,而又不白于臣之所以事先王之心,故敢以书对。
    
    臣闻贤圣之君不以禄私其亲,功多者授之;不以官随其爱,能当者处之。故察能而授官者,成功之君也;论行而结交者,立名之士也。臣以所学者观之,先王之举错,有高世之心,故假节\footnote{〔节〕外交使臣所持之凭证。}于魏王,而以身得察于燕。先王过举,擢之乎宾客之中,而立之乎群臣之上,不谋于父兄,而使臣为亚卿\footnote{〔亚卿〕官名,地位仅次于上卿。}。臣自以为奉令承教,可以幸无罪矣,故受命而不辞。先王命之曰:“我有积怨深怒于齐,不量轻弱,而欲以齐为事。”臣对曰:“夫齐,霸国\footnote{〔霸国〕齐桓公曾称霸诸侯,故称齐国为霸国。}之余教而骤胜之遗事也,闲于甲兵,习于战攻。王若欲伐之,则必举天下而图之。举天下而图之,莫径于结赵矣。且又淮北\footnote{〔淮北〕淮河以北地区,是齐国属地。}、宋地\footnote{〔宋地〕今江苏铜山、河南商丘、山东曲阜之间的地区,为齐所吞并。},楚、魏之所同愿也。赵若许约,楚、赵、宋尽力,四国攻之,齐可大破也。”先王曰:“善。”臣乃口受令,具符节,南使臣于赵。顾反命,起兵随而攻齐,以天之道,先王之灵,河北\footnote{〔河北〕黄河以北。}之地,随先王举而有之于济\footnote{〔济〕济水,又名兖水,沇水,是黄河下游的一条重要支流。于河南荥阳与黄河干流分离,东流至山东济宁以北入巨野泽,东北于济南经今黄河河道入海。曾与江水(长江)、河水(黄河)、淮水(淮河)并称“四渎”。1855年后由于黄河改道而淤塞消失。}上。济上之军奉令击齐,大胜之。轻卒锐兵,长驱至国。齐王逃遁走莒\footnote{〔莒〕今山东莒县。},仅以身免。珠玉财宝,车甲珍器,尽收入燕。大吕陈于元英\footnote{〔元英〕燕国宫殿名。后面的“历室”、“宁台”也是。},故鼎\footnote{〔故鼎〕指齐国掠夺的燕鼎,复归燕国。}反乎历室,齐器设于宁台。蓟丘\footnote{〔蓟丘〕燕国都城,今北京市西南。}之植,植于汶篁\footnote{〔汶〕汶水,流经齐鲁两国的水名,在今山东中部,又名大汶河。汶水流域是齐、魏反复争夺之地。}。自五伯\footnote{〔五伯〕指五霸,春秋时五位权倾诸侯的君主。}以来,功未有及先王者也。先王以为顺于其志,以臣为不顿命,故裂地而封之,使之得比乎小国诸侯。臣不佞,自以为奉令承教,可以幸无罪矣,故受命而弗辞。
    
    臣闻贤明之君,功立而不废,故著于《春秋》\footnote{〔春秋〕古代编年史都叫春秋。},蚤知之士,名成而不毁,故称于后世。若先王之报怨雪耻,夷万乘之强国,收八百岁\footnote{〔八百岁〕从姜太公建国到这次战争约八百年。}之蓄积,及至弃群臣之日,遗令诏后嗣之馀义,执政任事之臣,所以能循法令,顺庶孽者,施及萌隶,皆可以教于后世。
    
    臣闻善作者不必善成,善始者不必善终。昔者伍子胥说听乎阖闾\footnote{〔阖闾〕春秋时吴国国王,继位者是夫差。},故吴王远迹至于郢\footnote{〔郢〕楚国都城,今湖北江陵西北。};夫差弗是也,赐之鸱夷\footnote{〔鸱夷〕皮革制的口袋。}而浮之江。故吴王夫差不悟先论之可以立功,故沉子胥而弗悔;子胥不蚤见主之不同量,故入江而不改。夫免身全功,以明先王之迹者,臣之上计也。离毁辱之非,堕先王之名者,臣之所大恐也。临不测之罪,以幸为利者,义之所不敢出也。
    
    臣闻古之君子,交绝不出恶声;忠臣之去也,不洁其名。臣虽不佞,数奉教于君子矣。恐侍御者之亲左右之说,而不察疏远之行也。故敢以书报,唯君之留意焉。
\end{normalsize}


\newpage

\textbf{译文}:

\vspace{1em}

\begin{normalsize}
    
    昌国君乐毅,替燕昭王联合五国的军队,攻入齐国,连下七十多座城池,都划归燕国。还有三座城邑未攻下,燕昭王就去世了。燕惠王继位,中了齐人的反间计,怀疑乐毅,派骑劫代替他。乐毅逃到赵国,赵王封他为望诸君。齐国大将田单用计骗了骑劫,打败燕军,收复七十多座城邑,恢复了齐国的领土。
    
    燕王后悔了,又怕赵国任用乐毅,乘燕国战败之机来攻燕。燕王便派人去责备乐毅,又向乐毅表歉意,说:“先王把整个燕国托付将军,将军为燕攻破了齐国,为先王报了仇,天下人莫不震动。寡人怎敢一刻忘记将军的功勋啊!不幸先王抛弃群臣而去,寡人刚刚继位,左右蒙骗了寡人。不过,寡人派骑劫代替将军,只是因为将军长久在野外作战,所以调将军回国,休养休养,共商国是。将军却误信流言,和寡人有了隔阂,抛弃燕国而投奔赵国。为将军自己打算,固然可以;但是又怎样报答先王对将军的恩情呢?”
    
    望诸君乐毅便派人进献书信,回答惠王说:臣不才,不能奉承先王的遗命,顺从大王左右的心意,恐怕回来受到刀斧之刑,以致损害先王知人之明,又使您亏于君臣之义,只得投奔赵国。承担了不贤的罪名,因此不敢有什么说辞。现在大王派人来数说臣的罪过,恐怕大王左右不能体会先王重用臣的理由,也不明白臣所以事奉先王的心意,才敢写信答复大王。
    
    臣听说,贤圣的君主,不把爵禄私赏给自己的亲人,只有立功多的才授予;不把官职随便授予自己宠幸的人,只有才能相当的才任命。所以,考察才能而授官,是成就功业的君主;根据德行而结交,是树立名声的贤士。臣以所学的知识来观察,觉得先王处理国事,高于世俗的理想,因此借用魏王的使节,得以到燕国亲身考察。先王对臣过看重,从宾客中选拔出来,安置在群臣之上,不与王室的长辈商量,便任命臣为亚卿。臣自以为能够奉行命令、秉承教导,可以侥幸免于罪过,也就毫不辞让,接受了任命。先王命令臣,说:“我跟齐国积累了深仇大恨,那怕国小力微,也想报齐国之仇。”臣回答说:“齐国本来有霸主的传统,打过多次胜仗,熟悉军事,长于攻战。大王如果要伐齐,必须发动天下的兵力来对付它。要发动天下的兵力,最好是先同赵国结盟。还有淮北,本是宋国的土地,被齐国独吞了,楚魏两国都想得一份。赵如果赞同,约同楚魏尽力帮助,以四国的力量进攻,就可大破齐国了。”先王说:“好。”臣便接受命令,准备符节,南下出使赵国。很快回国复命,发兵攻齐。顺应上天之道,倚仗先王的声威,黄河以北的齐国土地,都随着先王进兵济上而为燕国所有了,济水上的燕军,奉令出击,大获胜利。士卒轻装,武器锐利,长驱直入,攻占齐都。齐王逃奔至莒,幸免一死。所有的珠玉财宝,车甲珍器,归燕国所有。大吕钟陈列在元英殿上,燕国的宝鼎又运回历室殿,齐国的宝器都摆设在燕国的宁台。原来树立在蓟丘的燕国旗帜,插到齐国汶水两岸的竹田。自从五霸以来,没有谁的功勋能赶上先王。先王很惬意,认为臣没有贻误他的命令,所以裂土封,使臣得比于小国诸侯。臣不才,自信能够奉行命令,秉承教导,可以侥幸免于罪过,因此毫不推辞而接受了封爵。
    
    臣听说,贤明的君主,建立了功业就不让它废弃,所以才能记载于史册;有预见的贤士,成名之后决不让它败坏,所以为后世称赞。像先王这样报仇雪恨,征服了万辆兵车的强国,没收它八百年的积蓄,直到逝世那天,还留下叮嘱嗣君的遗训,使执政任事的官员能遵循法令,安抚亲疏上下,推及百姓奴隶,这都是能够教育后世的啊。
    
    臣听说,善于创造不一定善于完成,善始不一定善终。从前,伍子胥说动了阖闾,因此吴王能够远征到楚国的郢都;夫差不这样做,将伍子胥赐死后装入皮囊,投于江中。夫差却不信伍子胥的预见能够立功,因此把伍子胥溺死江中而不悔;伍子胥不能预见新旧两主的气量不同,因此直到被投入江还不改变他的怨愤。所以,脱身免祸,保伐齐的大功,用以表明先王的业绩,这是臣的上策。遭受诋毁和侮辱的错误处置,毁害先王的美名,这是臣最大的恐惧。面临着不测之罪,却又助赵攻燕,妄图私利,我决不干这不义之事。
    
    臣听说,古代的君子,和朋友断绝交往,也决不说对方的坏话;忠臣含冤离开本国,也不为自己表白。臣虽然不才,也曾多次受过君子的教诲。只是恐怕大王轻信左右的说辞,而没有明察疏远的人的言行。因此冒昧回信说明,希望您多加考虑。
    
\end{normalsize}



\end{document}
