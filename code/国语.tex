
第33课  小房子
妹妹搭了八九所小房子,指着说:“那边是田。这边是房子。白天,我们到田里去种田。晚上,我们回到屋子里睡觉。”
  
第34课  妹妹哭了
小狗跑过来,把八九所小房子冲塌。妹妹哭了。妈妈说:“哭什么呢?小狗冲塌你的小房子,你不会再搭起来吗?”
 
第35课  再搭起来
妹妹再搭起小房子来,又把十多条柳条,插在小房子前面。他拍着手说:“比以前更好了。我们种田回来,可以在柳树下坐坐。”
 
第44课  小猫姓什么
“小猫姓什么,你知道吗?”“小猫姓小。”怎么知道他姓小?“大家叫他小白小白,他不是姓小吗?”“不对,不对。小白两个字是他的名字。”“那么他姓什么?”“我也不知道。”
 
第47课  泉水和小草
泉水从山上下来。小草问:“你到那里去?”泉水说:“我到地上去。”小草说:“请你停一停。我和你一同玩。”泉水说:“这里我停不住脚。再会吧!”
 
第48课  泉水和老树
泉水从山上下来。老树说:“我看你很忙。可以停一停吗?”泉水问:“你有什么事?”老树说:“请你帮助我。我渴得很。”泉水跑过老树的根下,对老树说:“现在你不渴了。再会吧!”
 
第49课  泉水到了河里
泉水到了河里,许多朋友欢迎他。太阳光拍拍他的背。白鹅到河里看他。小鱼和他一同玩。又有不少的花草,都对他点头。泉水说:“这里好朋友很多。我在这里住一下吧。”
 
第50课  小河
小河小河向东流,流呀流呀不回头。小河小河流得快,快呀快呀到大海。海里风景好,说不完,看不了。
 
 
《十只猪过桥》
十只猪过桥,母猪在前面,小猪跟在后面。过了桥,母猪回过身来,指着小猪说:“一二三四五六七八九我们共有十只,怎么少了一只呢?”
 
《绿衣邮差上门来》
绿衣天使上门来,送来小小一个袋。什么东西在袋里?薄薄几张纸,纸上许多黑蚂蚁。蚂蚁不做声,事事说得清。你想是什么?说来给我听。
 
《请问尊姓》
永儿的爸爸对永儿说:“如果有客人来,先要问他尊姓。”明天,对门的徐先生来看永儿的爸爸,永儿说:“徐先生,请问尊姓?”
 
《种痘》
爸爸种豆,种在地上。
医生种痘,种在臂上。
弟弟对医生说: “这是我的臂,不是园地。你种错了没有?”
医生说“大家要种痘,种痘防天花。”
 
《现在都洗干净了》
弟弟洗过了脸,说:“泥人脸上不很干净,我来洗一洗。”
他用水洗泥人的脸。洗过了再看,他说:“泥人的脸怎么洗不干净的?”
 
《我要老了》
哥哥落了一颗牙齿,他说“我要老了”
妈妈说: “没有的事! ”
哥哥说: “祖父老了,他的牙齿落完了。现在我落了牙齿,不是也要老了吗?”
妈妈笑着说: “你会生出新牙齿来的。”
哥哥问: “真的吗? 祖父的新牙齿为什么还没生出来呢?”
 
《雁》
秋天,有一群群的雁在天空飞过,发出清亮的叫声。雁的家乡在西伯利亚地方。那里秋天就飞雪,到了冬天,什么东西都被冰雪盖没了。太阳只露一下子脸,立刻又落了下去。如果再往北去,便是北极,那里足有半个年头见不到太阳的面。在这样又寒冷又黑暗的地方,雁怎么能够生活呢? 所以一到秋天,它们就结队迁移,向南方飞来。
雁的飞行队很有秩序,常常排成“人”字形、“之”字形、“—”字形的行列; 因此,我国的诗人把它叫做“雁字”。它们飞行的时候,由一只富有经验的统率着全队。那一只疲乏了,便由第二只有经验的代替它的职务。它们向我国南方沿海一带飞去,一路停落下来休息。当停落下来之前,常常在空中盘旋地飞着,侦察地面有没有危险。它们饥饿的时候,连麦苗和青草都吃。可是它们到底是水鸟,最喜欢在湖边、江滩搜寻它们的食物。
到了春深,它们的家乡渐渐暖和起来,冰雪融化了。太阳照得很长久,每天只有两三小时的黑夜。如果再往北去,就有整整的六个月,太阳老是在远远的天边。那时草木很快地生长起来,各种虫豸也繁殖得很多。于是雁从南方飞回去; 用芦杆等东西做底盘,再放上枯叶和羽毛,做成了窝,就把卵生在里边。母雁孵它的卵是非常专心的,除非十分饥饿,它决不肯离开一步。
一个月之后,小雁出壳了,一出壳就能活泼地走动。母雁带领着它们,到有水的地方去觅食。那里虫豸既多,得食自极容易; 侵害它们的动物很少,行动又极自由。雁在这样安适的地方生活,真是其乐无比。可是,这样安适的地方不是常年不变的。过了夏天,便是秋天,冰雪又要来管领这地方了。因此,雁必须每年一回离开故乡,向南方来旅行。
 
《不用文字的书和信》
从前有一个民族送给相邻的民族一封信。这封信一共四样东西: 一只死鸟,一只死老鼠,一只死青蛙,还有五支箭。这些东西包含着什么意思呢? 就是说: “你们能像鸟儿一样在天空中飞,像老鼠一样在地底下藏,像青蛙一样在湖面上跳跃吗? 如果不能,休想跟我们打仗。什么时候你们的脚踏上我们的土地,我们就用乱箭来对你们! ”
 
《三棵银杏树》
我家屋后有一片空地,十丈见方的开阔,前边、右边沿着河,左边是人家的墙。三棵银杏树站在那里。一棵靠着右边,把影子投到河里。两棵在中央,并着肩,手牵着手似的,像两个亲密的朋友。
这三棵银杏树多大年纪了,没有人能够知道。我父亲说,他小的时候,树就有这么高大了,经过了三十年的岁月,似乎还是这么高大。
三棵树的正干都很直: 枝干也是直的多,偶然有几枝屈曲得很古怪,像画幅上画的。每年冬天,赤裸的枝干上生出无数的小粒来。这些小粒渐渐长大,长得像牛、羊的奶头。到了春天,绿叶从奶头似的部分伸展出来。我们欢喜地说道: “银杏树又穿上新衣裳了! ”
空地上有了这广大的绿荫,正是游戏的好场所; 我们便在那里赛跑,唱歌,扮演狩猎的戏剧。经过的船只往往在右边那一棵的树荫下停泊,摇船的乘此吸一管烟或者煮一锅饭,这时候,一缕缕的烟便袅起来了。
银杏树的花太小了,很容易使人忽略。去年秋天,我一边拾银杏果,一边问父亲道:“银杏树为什么不开花呢?”父亲笑道: “不开花哪里来的果? 待来春留心看吧。”今年春天,我看见了银杏的花了,那是很可爱的、白里带点淡黄的小花。
说起银杏果,不由得想起街头“烫手炉,热白果”的叫卖声来。白果是银杏果的核,炒过一下,剥了壳,去了衣,便是绿玉一般的一颗仁,虽然并不甜,却有一种特别的清味。这东西我们都喜欢吃。
秋风阵阵地吹,折扇形的黄叶落得满地。风又把地上的黄叶吹起来; 我们拍手叫道: “一群黄蝴蝶飞起来了! ”待黄叶落尽,三棵老树又赤裸了。屈曲得很古怪的枝干上,偶然有一两只鹰停在那里,好久好久不动一动,衬着天空的背景,正像一幅古画。
 
《小萤虫》
小萤虫,点灯笼,
飞到西,飞到东。
飞到河边上,小鱼正做梦。
飞到树林里,小鸟睡得浓。
飞过张家墙,张家姊姊忙裁缝。
飞过李家墙,李家哥哥做夜工。
小萤虫,小萤虫,
何不飞上天,做颗星儿住天空?
 
《蜗牛看花》
墙顶开朵小红花,墙下蜗牛去看花。
这条路程并不短,背着壳儿向上爬。
壳儿虽小好藏身,不愁风吹和雨打。
爬得累了歇一会,抬头不动好像傻。
爬爬歇歇三天半,才到墙顶看到花。
无数花开朵朵红,一齐笑脸欢迎他。
 
《孙中山先生》 对话体
“这个人是谁?”
“孙中山先生。”
“孙中山先生是哪里人?”
“广东人。”
“孙中山先生做过什么事?”
“他要中国好,他同许多人立起中华民国来。”
“从前我们中国人叫中华民国吗?”
“当然不叫中华民国。我们说起中华民国,就该想到孙中山先生。”
 
《比老虎更可怕》 对话体
孔子同他的学生经过泰山旁边,看见一个妇人靠着一座新坟啼哭,声音非常凄惨,就教学生子路去慰问她。
子路上前问道: “大嫂子,你哭得这样凄惨,为的什么?”
那妇人说: “从前,我的公公给老虎衔去了; 去年,我的丈夫给老虎吃了; 如今,我的儿子又给老虎咬死了! ”
孔子问道: “既然这儿有老虎,为什么不搬到没有老虎的别处去住呢?”
那妇人说: “这儿没有苛刻的政治,我还舍不得离开这儿。”
孔子回过头来对学生们说: “你们记住,苛刻的政治比凶猛的老虎更可怕呢! ”
 
下面是课文的标题,可以从中得到些启发
 
第一册
一 先生早 / 二 坐下来 / 三 我讲话 / 四 小黄狗 / 五 你玩皮球 / 六 我爱皮球 / 七 走出去 / 八 晚上 / 九 这个是月亮 / 一〇 好月亮 / 一一 妈妈走来看 / 一二 窗子外 / 一三 月亮像眉毛 / 一四 母鸡小鸡 / 一五 那里有青草 / 一六 这是什么东西 / 一七 小小房子 / 一八 国庆日(一) / 一九 国庆日(二) / 二〇 国庆歌 / 二一 小牛画图 / 二二 这个像皮球 / 二三 这个像馒头 / 二四 现在画得像了 / 二五 那一张画得好 / 二六 到小羊家里去 / 二七 帮小羊烧茶 / 二八 帮小羊煮饭 / 二九 让我也来帮 / 三〇 请请请 / 三一 爸爸种菜 / 三二 满园菜 / 三三 妈妈裁衣 / 三四 两件新衣 / 三五 北风吹 / 三六 雪花 /三七 雪人 /三八 你们都不对 / 三九 十个好朋友 / 四〇 谁敲门 / 四一 我们一同玩 / 四二 你看像什么
 
第二册
一 可爱的泥人 / 二 “我也不知道” / 三 “小猫姓什么” / 四 种痘 / 五 今天早上 / 六 “谁洗得干净” / 七 “现在都洗干净了” / 八 “你是小鸟” / 九 “你是小鱼” / 一〇 “你是青虫” / 一一 茶话会 / 一二 种下几棵树 / 一三 都靠十个朋友 / 一四 “这里我站不住脚” / 一五 “现在你不渴了” / 一六 泉水到了河里 / 一七 春风来 / 一八 柳树条 / 一九 不用翅膀上天飞 / 二〇 张家姐姐回来了 / 二一 黄家哥哥出了门 / 二二 纸船 / 二三 “我要做蜜” / 二四 “我要开路” / 二五 “我要做丝” / 二六 “叫我做什么事呢” / 二七 “我们开店吧” / 二八 大家开店 / 二九 店柜和招牌 / 三〇 “你做买客” / 三一 走到店前 / 三二 纸盒改做的房子 / 三三 “鸡的家” / 三四 大家来住 / 三五 南风吹 / 三六 一箩麦 / 三七 “脏东西” / 三八 “大家当心防苍蝇呀” / 三九 跌到草地上 / 四〇 “那里来的信” / 四一 你猜是什么 / 四二 第一次写的信
第二卷(第三、四册)
 
第三册
一 欢迎新朋友 / 二 新书 / 三 “我坐飞机了” / 四 月亮船 / 五 原来做了一个梦 / 六 小萤虫 / 七 跌到水盆里 / 八 沉到水底去了 / 九 “这是我们的窠” / 一〇 “看你出去不出去” / 一一 狗和骨头 / 一二 桥上两只羊 / 一三 听错了 / 一四 看错了 / 一五 糖说的话 / 一六 盐说的话 / 一七 谁好看 / 一八 白小羊最好看 / 一九 两排白石头 / 二〇 “我要老了” / 二一 “跑上山去呀” / 二二 牛肉 / 二三 采棉 / 二四 剪羊毛 / 二五 孙中山先生 / 二六 略 / 二七 “何不去旅行” / 二八 他们一齐离开树枝 / 二九 好大的风呀 / 三〇 “点个火” / 三一 “烧熟了再吃吧” / 三二 他们家里有了火了 / 三三 凿石头 / 三四 自己打成的东西 / 三五 冬天 / 三六 麻雀看朋友 / 三七 新年庆贺会 / 三八 骑马 / 三九 北边冷地方(一) / 四〇 北边冷地方(二) / 四一 借书 / 四二 《图画故事》
 
第四册
一 “来得太早了“ / 二 “原来没有用的” / 三 小时表 / 四 玻璃瓶 / 五 “这里是池塘” / 六 海 / 七 三头小羊 / 八 好朋友 / 九 朝放羊 / 一〇 赛猪会 / 一一 猪说的话 / 一二 “我的身体被缚住了” / 一三 “我饿了” / 一四 人山 / 一五 小人国 / 一六 一粒种子 / 一七 初进光明的世界 / 一八 燕子来了 / 一九 燕子飞 / 二〇 儿童节 / 二一 拔萝卜 / 二二 哈哈 / 二三 一个大人 / 二四 “把我拾起来” / 二五 “我望下面就是家乡” / 二六 蜗牛看花 / 二七 龟和兔子赛跑 / 二八 “我能劳动” / 二九 懒惰的人 / 三〇 寻找小猫 / 三一 “送给我的爸爸” / 三二 一会儿 / 三三 猜谜 / 三四 “你说了出来吧” / 三五 “可怕的病来了” / 三六 标语 / 三七 商量 / 三八 墙角蜘蛛 / 三九 “偷鸡不着蚀把糈” / 四〇 树林里的声音 / 四一 黑人 / 四二 他们快乐极了
第三卷(第五、六册)
 
第五册
一 自己的画 / 二 挑选和计划 / 三 比从前更有趣味了 / 四 我们的学校 / 五 牵牛花蕾 / 六 向日葵花 / 七 一个农人 / 八 农人和野兔 / 九 查字典 / 一〇 不开口的先生 / 一一 移菊 / 一二 木工 / 一三 “尽量吃呀” / 一四 “现在我们约定了” / 一五 菊花开了 / 一六 略 / 一七 略 / 一八 运动会 / 一九 三脚赛跑 / 二〇 轮流赛跑 / 二一 鲫鱼和蟹 / 二二 渔人的网 / 二三 这个话不错(一) / 二四 这个话不错(二) / 二五 市集 / 二六 “等一会吧” / 二七 “这里的情形是这样的” / 二八 火车站 / 二九 航船埠头 / 三〇 种麦合唱 / 三一 修理农具 / 三二 奇怪的石头 / 三三 “那里什么都变了” / 三四 虎类的小动物 / 三五 怪东西 / 三六 白胡须老人 / 三七 没有想到 / 三八 日记 / 三九 池塘 / 四〇 大年夜(一) / 四一 大年夜(二) / 四二 大年夜(三)
 
第六册
一 中华 / 二 黄河的话 / 三 破碎的瓦罐头 / 四 商代人的书 / 五 沙漠 / 六 “忽然起了大风” / 七 初春的风 / 八 春天来了 / 九 孙中山先生和农人 / 一〇 游中山陵记 / 一一 龟和狐 / 一二 听狮子叫 / 一三 胆量和力量 / 一四 黄花冈 / 一五 荆轲 / 一六 大家动手 / 一七 菜秧 / 一八 上海来的信(一) / 一九 上海来的信(二) / 二〇 笨人 / 二一 百灵搬家 / 二二 不死药 / 二三 “可爱的同学” / 二四 “生了几天病” / 二五 一封电报 / 二六 两句话 / 二七 处处留心(一) / 二八 处处留心(二) / 二九 插秧 / 三〇 戽水 / 三一 完全不一样(一) / 三二 完全不一样(二) / 三三 小鸟的回家 / 三四 公园里(日记) / 三五 比虎更凶猛的东西 / 三六 秦始皇 / 三七 救火 / 三八 水灾 / 三九 蜻蜓 / 四〇 月夜 / 四一 种瓜人的夜 / 四二 天上的桥
第四卷(第七、八册)
 
第七册
一 长江 / 二 游泰山记 / 三 母亲的生日(一) / 四 母亲的生日(二) / 五 秋天的早上 / 六 一群鸽子 / 七 民国二十年的旧报纸 / 八 拿出我们的力量来(演说) / 九 月食的一夜(一) / 一〇 月食的一夜(二) / 一一 设立图书馆意见书 / 一二 新收到的书——伊索寓言 / 一三 蒲公英 / 一四 梧桐子 / 一五 荒岛上的鲁滨逊(一) / 一六 荒岛上的鲁滨逊(二) / 一七 合群生活 / 一八 大扫除 / 一九 做皮鞋的作场 / 二〇 晏子 / 二一 蔺相如 / 二二 景阳冈(一) / 二三 景阳冈(二) / 二四 孙中山先生伦敦遇难(一) / 二五 孙中山先生伦敦遇难(二) / 二六 河神的新娘(一) / 二七 河神的新娘(二) / 二八 霜的工作 / 二九 冬眠的蛙 / 三〇 愚公 / 三一 缩地的法术 / 三二 我们有一双眼睛 / 三三 望远镜和显微镜 / 三四 中学生的信 / 三五 小学生的信 / 三六 黑先生——最能干的工人 / 三七 请黑先生出来 / 三八 “让我拿一根火柴” / 三九 他们的唾沫 / 四〇 新年 / 四一 杂耍 / 四二 口技
 
第八册
一 踢毽子 / 二 轻气球 / 三 乘飞机记 / 四 风和水 / 五 穷人和富翁 / 六 纪念室 / 七 毕业生的信 / 八 一个邮差 / 九 云 / 一〇 春天 / 一一 孙中山先生逝世 / 一二 一个恶魔 / 一三 林则徐(一) / 一四 林则徐(二) / 一五 我们的力量 / 一六 筑路队 / 一七 孔庙和孔林 / 一八 瀑布 / 一九 书的生产(一) / 二〇 书的生产(二) / 二一 简笔字 / 二二 我国的文字 / 二三 火烧赤壁(一) / 二四 火烧赤壁(二) / 二五 养蚕 / 二六 木兰(一) / 二七 木兰(二) / 二八 五月 / 二九 南京路上 / 三〇 爱迪生的故事 / 三一 从农家出来的画家 / 三二 月光曲 / 三三 碧桐会 / 三四 留声机 / 三五 最古的祖先那里来的呢 / 三六 达尔文 / 三七 毛虫和白菜 / 三八 动物园 / 三九 演讲的材料 / 四〇 演讲的声调 / 四一 演讲的姿势 / 四二 对于乞丐要给钱吗(辩论)